\chapter{Derivational Morphology}

\section{Verbal Derivation}
\subsection{Vowel changes}
\subsubsection{from adjectives}
\begin{enumerate}
	\item Diphthongization of the short vowels \bt{E}, \bt{i} and \bt{y} to \bt{e\tsa{I}}, \bt{o} and \bt{u} to \bt{o\dpu}, and the fronting of \bt{a} to \bt{E}, or the lowering of the long high vowels \bt{i:} to \bt{E} and \bt{u:} and \bt{y:} to \bt{o}. Diphthongs, on the other hand are simplified, with fronting diphthongs becoming \bt{E:} and backing diphthongs becoming \bt{o:}. This also involves the affricatization of the coda consonant or the sporadic addition of a \bt{k} or \bt{\ttb{ts}}. The resultant verb has the general meaning of `to become \mk{adj},' although most have taken more idiomatic meanings. This process is no longer productive and thus cannot be used to derived new verbs.
	
	\begin{table}[h!]
		\small\centering
		\begin{tabu}to 0.9\textwidth{>{\bfseries}YYM[0.2]>{\bfseries}YY[1.6]}
			losz&`new'&$\rightarrow$&loucá&`to feel unfamiliar'\\
			sztune&`small'&$\rightarrow$&sztouncá &`to feel inferior'\\
			hlýda & `difficult'&$\rightarrow$&hlocká &`to suffer'\\
			szlau & `sour'&$\rightarrow$&szlóká & `to smell bad'\\
			obesz & `different&$\rightarrow$&obejcá&`to change'\\
		\end{tabu}
	\end{table}
	
\end{enumerate}