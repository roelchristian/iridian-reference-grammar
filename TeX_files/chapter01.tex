\chapter{Phonology}
\section{Vowels}\index{vowel}
\subsection{Oral Vowels}\index{vowel!oral}
Iridian has six pairs of corresponding long and short vowels. With the exception of \bt{a~a:} and \bt{u~u:}, long vowels are tenser than their short counterparts. In addition standard Iridian also features the high central vowel \nt{1} as an allophone of \nt{E} and \nt{I} and the low central \nt{5} as an allophone of \nt{a}, in unstressed positions.

\begin{table}[h!]\index{vowel!inventory}
	\small
	\caption{Vowel inventory of standard Iridian.}
	\centering
	\begin{tabularx}{0.7\textwidth}{YMMMM}
		\toprule
		&\multicolumn{2}{c}{\sc front}&\multirow{2}{*}{\sc central}&\multirow{2}{*}{\sc back}\\
		\cmidrule{2-3} &{Unrounded} &{Rounded}&&\\\midrule
		Close & \textipa{I i:} & \textipa{Y y:}&\textipa{(1)}& \textipa{u u:}\\
		Mid & \textipa{E e:} & && \textipa{O o:}\\
		Open&&&\textipa{(5) a a:}&\\
		\bottomrule
		\label{table:vowels}
	\end{tabularx}
\end{table}

Phonetic realization is generally consistent with orthography as seen in Table \ref{table:vowels-orth} below.

\begin{table}[h!]
	\small
	\caption{Orthographic representation of vowels.}
	\centering
	\begin{tabularx}{0.7\textwidth}{YMMYMM}
		\toprule
		& Short & Long & & Short & Long\\ \addlinespace
		\midrule \addlinespace
		\bt{a} & a &\'a & \bt{o} & o &\'o \\ \addlinespace
		\bt{e} & e &\'e & \bt{u} & u &\'u\\ \addlinespace
		\bt{i} & i &\'i & \bt{y} & y, \"y &\'y\\ \addlinespace
		\bottomrule
		\label{table:vowels-orth}
	\end{tabularx}
\end{table}	

Note that since \ird{y} is used to represent palatalization of a consonant in coda position, word-final short \bt{y} sound is written as \ird{\"y}. Both \bt{Y} and \bt{y} are diphthongized to \nt{Y\tsa{5}} and \nt{y:\tsa{5}} respectively at the end of a word.

\ex
\ird{m\"y}, \trsl{grandmother}\\
\nt{mY\tsa{5}} 
\xe

\ex
\ird{n\'em\"y}, \trsl{grandfather}\\
\nt{"ne:mY\tsa{5}} 
\xe

\ex
\ird{ahl\'y}, \trsl{juice}\\
\nt{"Paxly:\tsa{5}} 
\xe

\subsection{Diphthongs}\index{diphthong}
Iridian has three phonemic oral diphthongs: \ird{au} \nt{a\tsa{u}}, \ird{ei} \nt{\dte}, \ird{ou} \nt{\dto}. In addition, the diphthongs \ird{oi} \nt{O\tsa{I}} and \ird{ui} \nt{u\tsa{I}} also occur phonemically, but their occurence is marginal, normally appearing only in fixed expressions such as interjections and expletives.

\ex
\ird{Avui!} \trsl{Damn it!}\\
\nt{"P5vu\tsa{I}P} 
\xe

\ex
\ird{p\v{s}ehui}, \trsl{annoying}\\
\nt{"pCexu\tsa{I}P} 
\xe

\ex
\ird{Oi!} \trsl{Hey!}\\
\nt{PO\tsa{I}P} 
\xe

In most dialects however the diphthong \nt{\dte} has almost completely merged with \ird{\'e} \nt{e:}, although some divergent dialects in the south may realize the diphthong as \nt{i:}.


\ex
\ird{neite}, \trsl{word}\\
\nt{"n\dte tE} or \nt{"ne:tE} 
\xe

\subsection{Nasal and Nasalized Vowels}\index{vowel!nasal}\index{vowel!nasalized}
Iridian has three nasal vowels: \ird{\k{a}} \nt{\~5\~w}, \ird{\k{e}} \nt{\~E\~w} and \ird{\k{o}} \nt{\~O} (rarely \nt{\~O\~w}). Nasal vowels are not disinguished for length. In addition, nasal consonants in coda position are normally deleted, and the preceding vowel becomes phonemically nasal. In cases of nasal coda deletion, \ird{a} \nt{5} and \ird{e} \nt{E} are also dipthongized to \nt{\~5\~w} and \nt{\~E\~w}. Unstressed \nt{\~E\~w} is further reduced to \nt{\~@\~w}. This brings the inventory of nasal and nasalized consonants in Iridian to the following: \bt{\~5\~w \~E\~w \~@\~w \~I \~O \~u}

\ex
\ird{bi\k{e}c}, \trsl{cat}\\
\nt{b\sx{j}\~E\~w\jn{ts}} 
\xe

\ex
\ird{n\k{a}\v{s}}, \trsl{castle}\\
\nt{n\~5\~wC} 
\xe

\subsection{Vowel Length}\index{vowel length}\index{long vowel|see{vowel length}}

Vowel length is phonemic in Iridian. Length is represented by an acute accent\index{acute accent} over the long vowel. The short-long vowel pairs differ in quality as well as length, with the short vowels being more and the long vowels being tenser in addition to being longer. Diphthongs are phonetically considered as long vowels.

\begin{table}[h!]
	\small\centering
	\caption{Vowel length and quality.}
	\begin{tabu} to 0.7 \textwidth{MMM}
		\toprule \addlinespace
		{\sc archiphoneme} & {\sc lax/short} &{\sc tense/long}\\ \addlinespace\midrule
		\addlinespace
		\bt{a}	& \nt{a}	& \nt{a:}		\\\addlinespace
		\bt{e}	& \nt{E}	& \nt{e:}		\\\addlinespace
		\bt{i}	& \nt{I}	& \nt{i:}		\\\addlinespace
		\bt{o}	& \nt{O}	& \nt{o:}		\\\addlinespace
		\bt{u}	& \nt{u}	& \nt{u:}		\\\addlinespace
		\bt{y}	& \nt{Y}	& \nt{y:}		\\\addlinespace
		\bottomrule		
	\end{tabu} 
\end{table}

\subsection{Vowel Reduction}\index{vowel reduction}
The short vowels \bt{E} and \bt{I} are reduced to \nt{1} in unstressed positions. In less careful speech, this could even the elision of the vowel and the formation of consonant clusters or the realization of the preceding consonant as syllabic (especially if it is a liquid). Final \bt{E} is not reduced in a word-final position if preceding a pause.  
\pex
	\a
	\ird{p\'alet}, \trsl{paint}\\
	\nt{"pa:l1t} or \nt{pa:\llt t}
	\a
	but \ird{p\'aletka}, \trsl{painter}\\
	\nt{pa:"lEtk5}
\xe
\ex \ird{hope}, \trsl{shoes}\\
\nt{"xOpE}
\xe
\ex \ird{etapa}, \trsl{stage}\\
\nt{1"tap5}
\xe
Short \bt{a} is realized as \nt{5} in unstressed positions.
\ex \ird{k\'oma}, \trsl{cup}\\
\nt{"\jn{kx}o:m5}
\xe
\ex \ird{ma\v{s}inist}, \trsl{engineer}\\
\nt{m5"CIn1st}
\xe


\section{Consonants}\index{consonants}
Table \ref{table:fullconsonant} shows a complete list of consonant phonemes in Standard Iridian, with the allophones appearing in parentheses. In total, Iridian has 19 consonant phonemes but with 21 additional allophonic variants.
\begin{table}[h!]
	\centering \footnotesize
	\caption{Full consonant inventory of standard Iridian.}\label{table:fullconsonant}
	\begin{tabu} to \textwidth{Y[2]MMMM}
		\toprule 
		& Labial	& Alveolar		& Palatal	& Velar	\\
		\midrule \addlinespace
		Plosive & p~b		& t~d	 		& (c~\textbardotlessj)	& \textipa{k~g} \\ \addlinespace
		Nasal	& \textipa{m~(M)}			& n				& (\nn)	& 	\textipa{(N)}	\\ \addlinespace
		Liquid	&				& r~(\textipa{\rrr})~l				&	(\textipa{L})	&	\\ \addlinespace
		Sib. Fric.& 			&s~z	  & \textipa{C~\textctz} &\\ \addlinespace
		Non-Sib. Fricative& 		v	&	&   (ç) &\textipa{x~(G)}\\ \addlinespace
		Sib. Affricate &				& \textipa{\ttb{ts}~(\ttb{dz})}	& \textipa{\ttb{tC}~(\ttb{d\textctz})}  &  \\ \addlinespace
		Non-Sib. Aff. &				& \textipa{(\ttb{\|[tT}~\ttb{\|[dD})}	& \textipa{(\ttb{cç}~\ttb{\jjg J})}  & \textipa{(\ttb{kx}~\ttb{gG})} \\ \addlinespace
		Approximant & (\textipa{\|`B}) & (\textipa{\|`D}) &j&(\textipa{\*w~w})\\
		\bottomrule
	\end{tabu}
\end{table}

\subsection{Plosives}

\par Initial stops are affricated when following a pause, so that the velar pair \bt{k~g} are realized as \nt{\ttb{kx}~\ttb{gG}}, the palatal pair \nt{c~J} as \nt{\ttb{cç}~\ttb{\jjg J}}\footnote{Though phonotactically permitted, \nt{\jn{\jjg J}} has not been attested.} and the dental pair \bt{t~d} as \nt{\ttb{\|[tT}~\ttb{\|[dD}} (the last pair is written without the diacritics in the examples). This sound change can be traced to the initial aspirated stops \rec{\asp{k}}, \rec{\asp{g}}, \rec{\asp{t}} and \rec{\asp{d}} in Old Iridian weakening to affricates. The labial stops \bt{p~b} are unaffected by this process as most instances of \rec{\asp{p}} and \rec{\asp{b}} have merged to \nt{b} or \nt{v} in modern Iridian.

\ex {\small
	\begin{tabularx}{0.8\textwidth}{>{\bfseries}YYY}
		gulag		&\nt{"\ttb{gG}Ul5x}&`gulag'\\
		ka\v{s}t	& \nt{\ttb{kx}aCt} &`blood'\\
		tom			& \nt{\ttb{tT}\~O}&`powder'\\
		d\'um		& \nt{\ttb{dD}\~u}&`house'\\
		tieho		&\nt{"\ttb{cç}ExO}&`god'\\
	\end{tabularx}
	}
\xe

\par The velar stops \bt{k~g} are lenited to the velar fricatives \nt{x~G} intervocalically, before a voiceless stop, after a vocalized l if followed by another vowel or a voiceless stop, or before the nasal consonants \bt{n} or \bt{m} if following a vowel immediately. This lenition also occurs word-finally unless followed by a voiced obstruent, in which case, subject to word-final devoicing, they merge to \nt{x}.
\ex {\small
	\begin{tabularx}{0.8\textwidth}{>{\bfseries}YYY}
		seg			&\nt{sEx}&`flower'\\
		jeko		&\nt{"jExO}&`bed'\\
		naga		&\nt{"naG5}&`friend'\\
		agno\v{s}ce	& \nt{P5G"nOC\ttb{tC}E}&`agnostic'\\
		akta		&\nt{"Paxt5}&`show'\\
		\v{s}elk	&\nt{CEwx}&`beginning'\\
	\end{tabularx}
	}
\xe

\par This lenition can also be observed with the voiced stops \bt{b} and \bt{d} which become the approximants \bt{\|`{B}} and \bt{\|`D} (written without the diacritic hereafter) intervocalically or between a vocalized \bt{l} and another vowel.
\ex {\small
	\begin{tabularx}{0.8\textwidth}{>{\bfseries}YYY}
		nada & \nt{"n5D5} & `box'\\
		loba\v{c}&\nt{"lOB5\ttb{tC}}&`earthquake'\\
		álba &\nt{"a:wB5}& `hands'\\
	\end{tabularx}
	}
\xe

The first stop in a stop cluster is normally unreleased or fricated.

\ex {\small
	\begin{tabularx}{0.8\textwidth}{>{\bfseries}YYY}
		doktor		& \nt{"\jn{dD}OxtO\zz}			&\trsl{doctor}\\
		apt\'a		& \nt{"P5p\textcorner{}ta:} or \nt{"P5\textphi ta:}&`to polish'\\
	\end{tabularx}
	}
\xe

\par The glottal stop \textipa{[P]} is often not regarded as a separate phoneme.
It can occur in three cases: (1) before an onset vowel when following a pause, e.g., \ird{avt} \nt{Pa\dpu{}t} \trsl{car}; (2) between two vowels that do not form a diphthong, e.g., \ird{naome\v{s}á} \nt{n5PO"mESa:}, `to whisper'; or (3) emphatically, especially in interjections, e.g., \ird{Oi!} \nt{PO\tsa{I}P}, \trsl{Hey!}, \ird{K\'ap!} \nt{\jn{kx}a:pP}, \trsl{Look out! ({\emph{lit.}, danger)}}.


\subsection{Nasals}
Iridian has two nasal consonants \textipa{/m~n/}. \bt{n} cannot appear before bilabials and similarly \bt{m} cannot appear before velars. The palatal \bt{\nn} is an allophone of \bt{n} in environments affected by \textbf{niehockvo}. Both m and n cause a preceding short vowel to be nasalized, with the consonant itself subsequently dropped.

\ex{\small
	\begin{tabularx}{0.8\textwidth}{>{\bfseries}YYY}
		niho		& \nt{nIxO} & `nothing'\\
		tramvek		&\nt{"tr\~5\~wvEx}&`tramway'\\
		sinv\'oní		&\nt{s\~I"vo:ni:}&`symphony'\\
		banki\v{s}t	& \nt{"b\~5\~wkICt} & `banker'\\
	\end{tabularx}
}
\xe

The velar \nt{N} is not native in Iridian but can sometimes be observed, especially in loanwords, where it can be realized as nasalization of the preceding vowel when in the syllable coda or as \nt{N} intervocalically, although \nt{Ng} or \nt{Nk} is also common.

\ex
\ird{aktung}, \trsl{signage} (from Ger. Achtung, caution)\\
\nt{"Paxt\~u} or \nt{"PaxtuN} or \nt{"PaxtuNk}
\xe

\ex
\ird{anglevn\'i}, \trsl{English language}\\
\nt{P5N"lEwni:} or \nt{P5Ng"lEwni:}
\xe

\subsection{Liquids}

Iridian has two liquids: the rhotic \bt{r} and the lateral \bt{l}.

The rhotic \bt{r} is realized in one of three ways. Word-initially it is pronounced as the uvular fricative \nt{\textinvscr} (or as the uvular trill fricative \nt{\textraising{\textscr}}, depending on the speaker, but both transcribed here simply as \nt{\rrr}). The realization as \nt{\rrr} is also often used when pronouncing words emphatically. When in the coda position and before a pause \nt{r} is realized as \bt{\textctz}. This pronunciation was originally that of a voiceless alveolar trill \nt{\textsubring{r}} but this has simplified to \nt{\textraising{r}} and finally to \nt{\textctz} in Standard Iridian. The \nt{\textsubring{r}} or \nt{\textraising{r}} pronunciation may nevertheless persist in some southern dialects, primarily due to Czech influence. Note that \nt{\textctz} is not affected by word-final devoicing. Elsewhere \bt{r} is realized as the flap \nt{R}. Palatal \bt{r\sx{j}} is in general more stable, realized simply as \nt{r\sx{j}}, although when in the coda position and if not followed by a vowel, it is realized as \nt{\textctz}.

The lateral \bt{l} is actually the velarized alveolar lateral approximant \nt{\textltilde}. Nonetheless the sound has been transcribed throughout as \bt{l}. In the coda position \bt{l} is completely vocalized and is transcribed here as \nt{w}. The palatalized \bt{l\sx{j}} is the palatal lateral approximant \nt{L} and is transcribed as such.

\ex
	{\small
	\begin{tabularx}{0.8\textwidth}{>{\bfseries}YYY}
		lievout		& \nt{"LEvo\dpu{}t} 	& \trsl{pregnant}\\
		mielko	 	& \nt{"m\sx{j}EwxO}		& \trsl{sugar}\\
		\'or		& \nt{o:\zz}				& \trsl{hour}\\
		git\'ary	& \nt{"\jn{gG}Ita:\zz}		& \trsl{guitar}\\
		r\'ad		& \nt{\rrr a:t}			& \trsl{building}\\
	\end{tabularx}}
\xe

\subsection{Fricatives and Affricates}
The palatal sibilants \bt{C~\zz} can be realized as either the palatal \nt{C~\zz} or the post-alveolar \nt{S~Z} with the former being more common. The same is true with the palatal affricates \bt{\jn{tC}~\jn{d\zz}}, realized as other \nt{\jn{tC}~\jn{d\zz}} or \nt{\jn{tZ}~\jn{dZ}}, with the former also being more prevalent. The voiced affricated \bt{\jn{d\zz}}, normally written \ird{d\v{z}}, is marginal, and most loanwords originally containing \nt{\jn{d\zz}} or \nt{\jn{dZ}} are assimilated as \nt{\zz}.
\ex
\ird{\v{z}urn\'al}, \trsl{magazine}\\
\nt{"\zz urna:w}
\xe

The voiceless labial fricative \textipa{/f/} is not a native phoneme in Iridian and most loanwords containing \textipa{/f/} are assimilated to \textipa{/v/}.

\ex{\small
	\begin{tabularx}{0.8\textwidth}{>{\bfseries}YYY}
		vóto		& \nt{"vo:tO}		&`photograph'\\
 		kávéma\v{s}t	& \nt{\jn{kx}a:"ve:m5Ct}	&`coffeeshop'\\
 		Vranca		& \nt{"vr\~5\~w\ttb{ts}5} & `France'\\
	\end{tabularx}}
\xe

\par \bt{v} is realized as the approximant \bt{w} in syllable coda or before a stop. The sequence \ird{kv} and \ird{gv} is further lenited to the labialized velar fricatives \bt{x\sx{w}~G\sx{w}}. The voiceless \bt{x\sx{w}} (from both \textbf{kv} and \textbf{hv}) is in free variation with \bt{\*w}, with the latter being the more common pronunciation, especially among younger speakers.For simplicity both \bt{x\sx{w}} and \bt{\*w} will be transcribed as \bt{\*w}.
\ex
{\small
	\begin{tabularx}{0.8\textwidth}{>{\bfseries}YYY}
		kvártir		& \nt{"\*wa:rcI\zz} 			& `apartment'\\
		szvi\k{e}ce		& \nt{Svi\~E\~w\ttb{ts}E}	& `candle, light'\\
		gvor\v{z}	&\nt{G\sx{w}OrC}&\trsl{speech}\\
	\end{tabularx}}
\xe

\par Modern Iridian has lost the distinction between \bt{h} and \bt{x}, with both $\langle$ch$\rangle$ and $\langle$h$\rangle$,\footnote{Most instances of $\langle$ch$\rangle$ have been replaced with $\langle$h$\rangle$ following various spelling reforms.} historically representing \bt{x} and \bt{h}, respectively, merging to the velar fricative \bt{x}. This becomes \bt{ç} before voiceless stops word-initially or when following a front vowel, or before the front vowels \bt{i} and \bt{I}. The palatal \bt{ç} is the prepalatal [\textsubplus{ç}], intermediate between \bt{ç} and \bt{C}. The sequence $\langle$hl$\rangle$ and $\langle$kl$\rangle$ are realized as \bt{\jn{t\llb}}.

\ex{\small
	\begin{tabularx}{0.8\textwidth}{>{\bfseries}YYY}
		hluvek		& \nt{\jn{t"\llb} uvEx}		& `apartment'\\
		hrona		& \nt{"xrOn5}	& `three'\\
		hteny		& \nt{çtE\nn}	&`person'\\
		neiht		& \nt{ne:çt} & `color'\\
		klube		& \nt{"\jn{t\llb}uBE} &\trsl{club}\\
	\end{tabularx}
}
\xe

\subsubsection{Voicing}
\par Iridian consonants are generally affected by two systems of phonological opposition: a primary distinction between voice and unvoiced consonants, and a secondary distinction between hard and soft consonants (i.e., normal and palatalized consonants).
\par Consonant voicing is phonemic. Voiced consonants are called muddy or dark (\textbf{mrknie}) while unvoiced consonants are called clear (\textbf{hocke}). Iridian has a strong tendency to devoice consonants, a process called \textbf{niehockvo} (clearing, lightening).

\par Voiced consonants are devoiced when followed by a voiceless obstruent, or in word-final position, unless followed by a vowel or a voiced obstruent. Conversely, voiceless obstruents become voiced when followed by another voiced obstruent.

\begin{table}[h!]
	\centering \small
	\begin{tabularx}{0.8\textwidth}{>{\bfseries}YYY}
		avt &\textipa{[P5ft]}& `car'\\
		szkad& \textipa{[Sk5t]} & `serious'\\
		kdavidy & \textipa{["gd5v\sx{j}Ic]} & `clean'\\
		ryz &\textipa{[rIs]}& `rice'\\
	\end{tabularx}
\end{table}

\section{Phonotactics}

\subsection{Syllable structure}

Ignoring the possible complexity of the onset, nucleus or coda, the basic structure of an Iridian syllable is CV(C), with C representing a consonant and V a vowel. Iridian has relatively few phonotactic constraints, allowing, at a maximum, syllables of the form (C)\tsup{2}CV(C)\tsup{3}. Nevertheless, most syllables fall in either of the four groups CV, CVC, CCV and CVCC

\begin{table}[h!]
	\centering
	\caption{Blevin's criteria as they apply to Iridian.}
	\begin{tabularx}{0.6\textwidth}{YM}
		\toprule
		& {\sc parameter}\\
		\midrule
		Obligatory onset & Yes\\
		Coda & No\\
		Complex onset & Yes\\
		Complex nucleus & Yes*\\
		Complex coda & Yes\\
		Edge effect & \\
		\bottomrule
	\end{tabularx}
\end{table}


\subsection{Onset}

\par Iridian does not allow a null onset (vowel in the syllable onset), i.e., the most basic Iridian syllable should be of the form CV. Words that superficially appear as having a null onset syllable in the initial position are actually preceded by a glottal stop. An epenthetic glottal stop is also added between vowels in a sequence that do not otherwise form dipthongs, or before a vowel in a word-initial position in loanwords.

\begin{center} \small
	\begin{tabularx}{0.8\textwidth}{>{\bfseries}YYY}
		Americe & \bt{P5mE"R\sx{j}I\ttb{ts}E} &`America'\\
		uide&\bt{PYDE}& `gong'\\
		ekt&\bt{PExt}&`forehead'\\
	\end{tabularx}
\end{center}

\begin{table}[h!]
	\small \centering
	\caption{Allowed word-initial CC clusters}
	\begin{tabularx}{\textwidth}{YMMMMMMMMMMMMMMMMMMMM}
		\toprule
		&p&b&t&d&k&g&m&n&r&l&s&z&\v{s}&\v{z}&v&\v{c}&dc&c&dz&h\\
		\midrule
		p&&&+&&&&&+&+&+&+&&+&&&&&&&\\
		b&&&&&&&&&+&+&&&&&&&&&&\\
		t&&&&&&&+&&+&+&&&&&+&&&&&\\
		d&&&&&&&+&+&+&+&&&&&+&&&&&\\
		k&&&+&+&&&&+&+&+&+&&+&&+&&&&&\\
		g&&&&&&&&+&+&+&&&&&+&&&&&\\
		m&&&&&&&&+&&&&&&&&&&&&\\
		n&&&&&&&&&&+&&&&&&&&&&\\
		r&&&&&&&&&&&&&&&&&&&&\\
		l&&&&&&&&&&&&&&&&&&&&\\
		s&&&&&&&&&&&&&&&&&&&&+\\
		z&&+&&+&&&+&+&+&+&&&&&+&&&&&\\
		\v{s}&+&&+&&+&&+&+&+&+&&&&&+&+&&+&&+\\
		\v{z}&&&&&&&&&&&&&&&&&&&&\\
		v&&&+&+&+&&&+&+&+&+&&+&&&&&+&&\\
		\v{c}&&&+&&+&&&&&+&&&&&&&&&&\\
		c&&&+&&+&&&+&+&+&&&&&&&&&&+\\
		h&&&+&&&&&&+&+&&&&&+&&&&&\\
		\bottomrule
		\addlinespace
		\multicolumn{21}{l}{\footnotesize + allowed cluster}
	\end{tabularx}
\end{table}

\par The following CC clusters are allowed to be in onset position:

\begin{enumerate}
	\item Stop followed by a liquid:
		\begin{enumerate}
			\item \bt{pr}: \ird{pragy} \bt{pr5c}, `sand'; \ird{pramou} \bt{pr5"mo\dpu}, `petal'
			\item \bt{tr}: \ird{trâ} \bt{tr\~5\~w}, `bread'; \ird{truig} \bt{trYx}, `ball
			\item \bt{kr}: \ird{krova} \bt{"krOv5}, `egg'; \ird{kramy} \bt{kr5m\sx{j}}, `toe'
			\item \bt{pl}: \ird{plan} \bt{pl5n}, `plan'; \ird{ploika}, \bt{"pl\o x5} `knot'
			\item \bt{tl}:\footnote{This is realized as \bt{t\llb} or even \bt{\llb}.} \ird{tlyk} \bt{t\llb Ix}, `pig'; \ird{tlum} \bt{t\llb Um}
			\item \bt{kl}:\footnote{Realized as \bt{\ttb{t\llb}} in Standard Iridian or as \bt{k\r*{l}} in some dialects.} \ird{klug} \bt{\ttb{t\llb}Ux}, foot; \ird{klúbe} \bt{"\ttb{t\llb}u:bE}, `club'
			\item \bt{br}: \ird{brok} \bt{brOx}, `female teenager'; \ird{bremy} \bt{brEm\sx{j}}, `ugly'
			\item \bt{dr}: \ird{drono} \bt{drOnO}, `brother'; \ird{drúi} \bt{dry:} `enemy'
			\item \bt{gr}: \ird{grec} \bt{grE\ttb{ts}}, `flag'; \ird{gryny} \bt{grI\nn} `peace'
			\item \bt{bl}: \ird{bloht} \bt{blOxt}, `mud'; \ird{bleu} \bt{bl\dty} `neck'
			\item \bt{dl}\footnote{This has merged to \ird{tl} in Standard Iridian.}: \ird{dleva} \bt{"\ttb{t\llb} Ev5}, `low'; \ird{dlouhe} \bt{\ttb{t\llb}\dto xE} `duck'
			\item \bt{gl}: \ird{gloibek} \bt{"gl\o bEx}
		\end{enumerate}
	\item Dental or velar stops followed by \bt{v}: \footnote{\bt{v} is realized as \bt{V} in this context. See section of stops for details on \ird{kv} and \ird{gv}.}
		\begin{enumerate}
			\item \bt{tv}:
			\item \bt{dv}:
			\item \bt{kv}: \ird{kvártir} \bt{"\*wOrcIr}, apartment; \ird{kveno} \bt{"\*wEnO}, `kitten'
			\item \bt{gv}: \ird{gvarusz} \bt{G\sx{w}5"rUS}, `speech'; \ird{gvecs} \bt{G\sx{w}E\ttb{tS}}, `dinner'
		\end{enumerate}
	\item \bt{k} or \bt{p} followed by \bt{t} or its soft counterpart; \bt{k} followed by \bt{d} or its soft counterpart:
		\begin{enumerate}
			\item \bt{kt}: \ird{kto} \bt{ktO}, `smile'; \ird{ktiesz} \bt{kcES}, `ache'
			\item \bt{pt}: \ird{pteva} \bt{ptEv5}, `leaf'; \ird{ptiará} \bt{pc5R5}, `count'
			\item \bt{kd}:\footnote{This is always realized as \bt{gd}.} 
		\end{enumerate}
	\item \bt{k} or \bt{p} before \bt{s} or \bt{S} or their soft counterparts:
	\begin{enumerate}
		\item \bt{ps}:\footnote{This is a marginal cluster, occuring only in mostly Greek loanwords.} \ird{psyhologa} \bt{psIxOlO"G5}, `psychologist';
		\item \bt{pS}: \ird{pszehuj} \bt{"pSExu\tsa{I}}, \eng{annoyance}; \ird{pszêcem} \bt{"pS\~E\~w\ttb{ts}Em}, \eng{grain}
		\item \bt{ks}:\footnote{This is another marginal cluster, occuring only in mostly Greek loanwords.} 
		\item \bt{kS}: \ird{kszêtva} \bt{"kS\~E\~wtv5}, \eng{chain}; \ird{kszévet} \bt{"kSe:vEt}, \eng{basket}
	\end{enumerate}
	\item Dental stops followed by \bt{m}:
	\begin{enumerate}
		\item \bt{tm}: \ird{tmeny} \bt{tmE\nn}, \eng{belt}; \ird{tmou} \bt{tm\dto}, \eng{waist}
		\item \bt{dm}:
	\end{enumerate}
		\item \bt{p}, \bt{d}, \bt{k} or \bt{g}  followed by \bt{n}:
	\begin{enumerate}
		\item \bt{pn}:
		\item \bt{dn}:
		\item \bt{kn}:
		\item \bt{gn}:\footnote{Realized as \bt{Gn} after a vowel-final word and \bt{kn} elsewhere.} \ird{gnasz} \bt{kn5S}, \eng{school}; \ird{gnuma} \bt{knUm5}, \eng{mattress}
	\end{enumerate}

		\item \bt{m} followed by \bt{n} or \bt{n} followed by \bt{l}:
	\begin{enumerate}
		\item \bt{mn}: \ird{mnucs} \bt{mnU\ttb{tS}}, \eng{husband}; \ird{mnouvaty} \bt{"mn\dto v5c}, \eng{hunchback}
		\item \bt{nl}:\footnote{This is realized as palatal \bt{\nn L}.} \ird{nlâsz} \bt{\nn L\~5\~wS}, \eng{castle}; \ird{nlúi} \bt{\nn Ly:}, \eng{horse}
	\end{enumerate}

		\item \bt{S} followed by a voiceless stop:
	\begin{enumerate}
		\item \bt{Sp}:
		\item \bt{St}
		\item \bt{Sk}:
	\end{enumerate}

		\item \bt{\textctz} before \bt{b} or \bt{d}:
\begin{enumerate}
	\item \bt{zb}
	\item \bt{zd}:
\end{enumerate}

		\item \bt{S} or \bt{\textctz} followed by a nasal, a liquid, or \bt{v}:
	\begin{enumerate}
		\item \bt{Sm}:
		\item \bt{Sn}:
		\item \bt{Sr}:
		\item \bt{Sl}:
		\item \bt{Sv}:
		\item \bt{zm}:
		\item \bt{zn}:
		\item \bt{zr}:
		\item \bt{zl}:
		\item \bt{zv}:
	\end{enumerate}

		\item \bt{S} before the affricates \bt{\ttb{ts}} or \bt{\ttb{tS}}:
	\begin{enumerate}
			\item \bt{S\ttb{ts}}:
			\item \bt{S\ttb{tS}}
	\end{enumerate}

		\item \bt{S} or \bt{s} before the affricates \bt{x}:
\begin{enumerate}
	\item \bt{Sx}:
	\item \bt{sx}
\end{enumerate}

		\item \bt{v} before the affricates \bt{s} or \bt{S}, \bt{n}, the stops \bt{t}, \bt{k}, or \bt{d}, the liquids \bt{r} or \bt{l}, or the affricate \bt{\ttb{ts}}:
\begin{enumerate}
	\item \bt{vs}:
	\item \bt{vS}:
	\item \bt{vn}:
	\item \bt{vt}:
	\item \bt{vk}:
	\item \bt{vd}:
	\item \bt{vr}:
	\item \bt{vl}:
	\item \bt{v\ttb{ts}}:
\end{enumerate}

		\item \bt{\ttb{tS}} before \bt{k}, \bt{t}, or \bt{l}:\footnote{CC clusters beginning with \bt{\ttb{tS}} have all simplified to \bt{S}.}
\begin{enumerate}
	\item \bt{\ttb{tS}k}:
	\item \bt{\ttb{tS}t}:
	\item \bt{\ttb{tS}l}:
\end{enumerate}

		\item \bt{\ttb{ts}} before \bt{k}, \bt{t}, \bt{l}, \bt{r} \bt{n} or \bt{x}:
\begin{enumerate}
	\item \bt{\ttb{ts}k}:
	\item \bt{\ttb{ts}t}:
	\item \bt{\ttb{ts}r}:
	\item \bt{\ttb{ts}l}:
	\item \bt{\ttb{ts}n}:
	\item \bt{\ttb{ts}x}:
\end{enumerate}

		\item \bt{x} before \bt{t}, \bt{l}, \bt{r} or \bt{v}:
\begin{enumerate}
	\item \bt{xt}:
	\item \bt{xl}:
	\item \bt{xr}:
	\item \bt{xv}:\footnote{This is realized as \bt{\*w}.}
\end{enumerate}

\end{enumerate}

Three-consonant clusters are subject to more constraints.

\begin{table}
	\small
	\caption{Allowed CCC clusters.}
	\begin{tabu} to 0.6\textwidth {Y[2.0]MMMMMM}
		
		\toprule
		&v&z&sz&p&b&k\\
		\midrule
		pr&&&+&&&\\
		pl&&&+&&&\\
		br&&+&+&&&\\
	
		tr&+&&+&&&\\
		tl&&&+&&&\\
		tv&&&&&&\\
		dr&&+&&&&\\
		dv&&&&&&\\
	
		kr&&&+&&&\\
		kl&&&+&&&\\
		kv&&&&&&\\
		gr&&+&&&&\\
		
		sh&+&&&+&&+\\
		
		szp&&&&&&\\
		szt&&&&&&\\
		szk&&&&&&\\
		szh&+&&&+&&+\\
		szr&+&&&+&+&+\\
		szc&+&&&+&+&+\\
		szcs&+&&&+&+&+\\
		
		\bottomrule
	\end{tabu}
\end{table}

\begin{enumerate}
	\item \bt{S}-voiceless stop-liquid clusters
	\begin{enumerate}
		\item \bt{Spr}:
		\item \bt{Str}:
		\item \bt{Skr}:
		\item \bt{Spl}:
		\item \bt{Stl}:
		\item \bt{Skl}:
	\end{enumerate}

	\item \bt{S}, followed by a stop, followed by \bt{v}
\begin{enumerate}
	\item \bt{Skv}:
	\item \bt{Stv}:
	\item \bt{Sdv}:
\end{enumerate}

	\item \bt{\textctz}-voiced stop-\bt{r} clusters
	\begin{enumerate}
		\item \bt{zbr}:
		\item \bt{zdr}:
		\item \bt{zgr}:
	\end{enumerate}

	\item \bt{v} followed by a stop, followed by a liquid:
\begin{enumerate}
	\item \bt{vtr}:
	\item \bt{vdr}:
	\item \bt{vkr}:
\end{enumerate}

	\item \bt{v} followed by \bt{S}, followed by a liquid or a voiceless stop:
\begin{enumerate}
	\item \bt{vSr}:
	\item \bt{vSt}:
	\item \bt{vSk}:
	\item \bt{vSp}:
\end{enumerate}

	\item Stop followed by \bt{S} followed by \bt{t}, \bt{\ttb{ts}} or \bt{\ttb{tS}}:
\begin{enumerate}
	\item \bt{bS\ttb{tS}}:
	\item \bt{bS\ttb{ts}}:
	\item \bt{bSt}:
	\item \bt{pS\ttb{tS}}:
	\item \bt{pS\ttb{tS}}:
	\item \bt{pSt}:
	\item \bt{kS\ttb{tS}}:
	\item \bt{kS\ttb{tS}}:
	\item \bt{kSt}:
\end{enumerate}
\end{enumerate}

\subsection{Nucleus}

\subsection{Coda}


\section{Consonant Alternations}

A large part of consonant palatalization in Iridian is due to palatalization, with a coda consonant getting in contact with \bt{j}, an unrounded front vowel, or a \bt{j}-glide.

\subsection{Simple palatalization}

\subsection{Palatalization}
\par Iridian consonants can either be hard (\textbf{suhne}) or soft (\textbf{gem}). Consonants are hard by default but become soft when followed by the vowels \textbf{i} or \textbf{í}. The vowel \textbf{y} is normally used to indicate non-palatalizing \bt{i}, although it is used to indicate palatalization word-finally or before \textbf{i}.

\par The use of \ird{-y} is a remnant of word final short \rec{i} from Old Iridian that has since disappeared. The same process has caused the shortening of long \rec{i} to \bt{I}. This sound change did not distinguish between palatalizing and non-palatalizing \rec{i} so that \rec{seni} `tooth' and \rec{seny} `blanket' both merged to modern Iridian \ird{seny} \bt{sE\nn}.

\par Softening involves palatal articulation of labial consonants (e.g., \textbf{be} \textipa{[bE]} vs \textbf{bie} \textipa{[b\sx{j}E]}) or the change to a palatal consonant for non-labials (e.g., \textbf{te} \textipa{[tE]} vs \textbf{tie} \textipa{[cE]}). Table \ref{table:softhard} shows how non-labials are affected by palatalization in Iridian.

\begin{table}[ht!]
	\centering \scriptsize
	\caption{Soft and Hard Consonants}\label{table:softhard}
	\begin{tabu} to \textwidth{MM[0.1]MMM[0.1]MM}
		\toprule
		\multirow{2}{*}{\sc \textbf{series}}&&\multicolumn{2}{c}{\sc \textbf{hard}}&&\multicolumn{2}{c}{\sc \textbf{soft}}\\
		\cmidrule{3-4} \cmidrule{6-7}
		&& Unvoiced	& Voiced	&& Unvoiced	& Voiced	\\
		\midrule \addlinespace
		\textit{\textbf{t} series}&& \textbf{t} [t]& \textbf{d} [d]&&\textbf{ty, ti} \textipa{[c]}&\textbf{dy, di} \textipa{[\jjg]}\\ \addlinespace
		\textit{\textbf{k} series}&& \textbf{k} [k]& \textbf{g} \textipa{[g]}&&\textbf{ky, ki} \textipa{[c]}&\textbf{gy, gi} \textipa{[\jjg]}\\ \addlinespace
		\textit{\textbf{s} series}&& \textbf{s} [s]& \textbf{z} \textipa{[z]}&&\textbf{sy, si} \textipa{[C]}&\textbf{zy, zi} \textipa{[\textctz]}\\ \addlinespace
		\textit{\textbf{\v{s}} series}&& \textbf{sz} \textipa{[S]}& \textbf{zs} \textipa{[Z]}&&\textbf{szy, -i} \textipa{[C]}&\textbf{zsy, -i} \textipa{[\textctz]}\\ \addlinespace
		\textit{\textbf{c} series}&& \textbf{c} \textipa{[\ttb{ts}]}& \textbf{dz} \textipa{[\ttb{dz}]}&&\textbf{cy, ci} \textipa{[tC]}&\textbf{dzy, -i} \textipa{[d\textctz]}\\ \addlinespace
		\textit{\textbf{\v{c}} series}&& \textbf{cs} \textipa{[\ttb{tS}]}& \textbf{dc} \textipa{[\ttb{dZ}]}&&\textbf{csy, -i} \textipa{[\ttb{tC}]}&\textbf{dcy, -i} \textipa{[\ttb{d\textctz}]}\\ \addlinespace
		\textit{\textbf{h} series}&& \textbf{h} \textipa{[x]}& ---&&\textbf{hy, hi} \textipa{[ç]}&---\\ \addlinespace
		\textit{\textbf{n} series}&& ---& \textbf{n} \textipa{[n]}&&---&\textbf{ny, ni} \textipa{[\nn]}\\ \addlinespace  
		\textit{\textbf{l} series}&& ---& \textbf{l} \textipa{[l]}&&---&\textbf{ly, li} \textipa{[L]}\\ \addlinespace
		\bottomrule
		
	\end{tabu}
\end{table}

\par Note how sounds produced using the same manner of articulation merge to the corresponding palatal consonant, keeping the voiced/voiceless distinction, such that both sibilant pairs \ird{s-z} and \ird{sz-zs} soften to \bt{C~\textctz}, the plosive pairs \ird{k-g} and \ird{t-d} to \bt{c-\jjg}, and the affricates \ird{c-dz} and \ird{cs-dc} to \bt{\ttb{tC}~\ttb{d\textctz}}.\footnote{This merger and word-final devoicing results, for example, to \ird{-ety}, \ird{-edy}, \ird{-eky}, and \ird{-egy} all being pronounced as \bt{Ec}} Some dialects, however may realize soft \ird{cs-dc} as \bt{c~\jjg}.




\subsection{Mutation of labials}



\subsection{Mutation of dentals and velars}
\pex \bt{k} and \bt{g}
\begin{center}
	\small
	\begin{tabu}to 0.8 \textwidth{Y[0.5]YY}
		k$\sim$c		& \ird{Marek} \trsl{Marek}	& \ird{Marcie} \trsl{Marek-\mk{gen}}\\
		k$\sim$\v{c}	&&\\
		g$\sim$\v{z}	&&\\
		
	\end{tabu}
\end{center}
\xe

\subsection{Compound alternations}

\subsection{Consonant$\sim$zero alternations}

\subsection{Voicing and devoicing}

\subsection{Assimilation of Sibilants}
The sibilants \ird{s, z, \v{s}} and \ird{\v{z}} and the sibilant affricates \ird{c} and \ird{\v{c}} assimilate when forming a cluster, whether in morpheme boundaries or morpheme-internally.

\begin{table}[h!]
	\centering \footnotesize
	\caption{Assimilation of sibilant clusters.}\label{table:sibs}
	\begin{tabu} to 0.8\textwidth{YY[0.8]Y[3]}
		\toprule
		\addlinespace
		{\sc cluster}	&  & {\sc examples}\\
		\midrule
		\addlinespace
		s + \v{s}	& \nt{C}&\\ \addlinespace

		\v{s} + c or \v{c} & \nt{C\jn{tC}} & \\ \addlinespace
		&\nt{Ct}&\\ \addlinespace


		
		 
		\bottomrule
	\end{tabu}
\end{table}


\section{Vowel Alternations}

\subsection{Compensatory vowel lengthening}

\section{Other Phonological Processes}


\section{Phonological Processes}

\subsection{Assimilation of loanwords}


\subsection{Vowel$\sim$zero alternation}

Vowel$\sim$zero alternations refer to an extensive series of morphophonological changes in Iridian causing certain vowels to disappear in certain contexts. Vowels that alternate with zero (i.e., that disappear in certain morphological contexts) are said to be \textit{unstable} vowels.

\par Below is a comprehensive list of environments that trigger vowel zero alternations. Here C represents any phonologically permitted consonant or consonant cluster, V a short vowel and VV a long vowel or a diphthong,

\subsubsection{\_cvcvc stems}
The final V is generally unstable in the following environments

\begin{enumerate}
	\item Stem has the same vowels. Examples: \ird{daman} $\rightarrow$ \ird{damna} `lips'; \ird{ploit} $\rightarrow$ \ird{poilte} `pancake'; \ird{poviasztak} $\rightarrow$ \ird{poviesztkam} `I ate'
	\item V\tss{2} is a short vowel. Examples: \ird{zsedym} $\rightarrow$ \ird{zsedme} `beard'; \ird{elaim} $\rightarrow$ \ird{elme} `fog'
	\item Stressed vowels and most loanwords do not follow this rule. Examples \ird{majoniez} $\rightarrow$ \ird{majonieza} `mayonaise' but \ird{mobil} $\rightarrow$ \ird{mubla} `phone'
	\item Where the deletion would cause the resulting consonant to be geminated or to be a voiced/unvoiced pair of the same consonant, the preceding vowel is lengthened. In the case of voiced/unvoiced pairs, only the voiced consonant is kept. Example: \ird{uidet} $\rightarrow$ \ird{úide}
	\item The presence of a soft consonant in the last or the penultimate consonant position normally inhibit vowel$\sim$zero alternation.
\end{enumerate}


\subsubsection{Stem-final vowel$\sim$zero alternation}

\subsubsection{Suffix-initial vowel$\sim$zero alternation}

\begin{enumerate}
	\item \_CVCVC or \_CVVCVC stems. The final V is generally unstable in the below contexts

	\item 
	\item Suffix-initial vowel$\sim$zero alternation
\end{enumerate}

\subsection{Vowel$\sim$vowel alternation}
\par Vowel$\sim$vowel alternations form an integral part of Iridian morphophonology. These changes can be grouped into two broad categories: (1) pluralizing ablaut, which involves the raising or fronting of stem vowels to form the plural of most native nouns and (2) marginal apophony involving the vowels \bt{E} and \bt{O}.

The first category is one of the most common processes in Iridian, used in the formation of marked plurals. In general, it involves the fronting of back vowels (e.g., o to oi), the raising of low front vowels (ai to oi) and the diphthongization of high front vowels. This change does not affect vowel length, so that long vowels remain long and short vowels remain short. This process is discussed in detail in the chapter on nouns.

The second category involves the short vowels \bt{O} and \bt{E}, and in ome cases \bt{5}. This class of changes is normal observed in the following:

\begin{enumerate}
	\item In \_VC final words, where C is a soft consonant, if followed by a consonat final suffix, or if metathesis or vowel$\sim$zero alternation causes the deletion of the initial vowel of the suffix \bt{5~E~O} become \bt{E~I~U}. The soft consonant remains as soft, although this is not reflected in the orthography
	
		\pex
	\a <ov> +sztraty + ak $\rightarrow$ szovtretka (I) walked
	\xe
	\item Short \bt{O} in a stable position alternates with \bt{U} and short \bt{E} is a stable position after a soft consonant with \bt{I}, when followed by a voiced plosive after the deletion of an unstable vowel.
	
	\pex
	\a \ird{lobek} `apple' $\rightarrow$ \ird{lubka} `apple-\mk{pat}'
	\a \ird{hotel} `hotel' $\rightarrow$ \ird{hotela} `hotel-\mk{pat}'
	\xe

	\item In \_PaC final words, where C is a voiceless obstruent (either phonemically or because of assimilation) or a nasal, \bt{5} becomes \bt{E} and \bt{E} becomes \bt{I} and the voiceless consonant is voiced when followed by a vowel-initial suffix.
	
		\pex
	\a \ird{szviad} `star' $\rightarrow$ \ird{szvieda} `star-\mk{pat}'
	\a \ird{pian} `fire' $\rightarrow$ \ird{piena} `fire-\mk{pat}'
	\xe

	\pex
	\a \ird{miet} `pot' $\rightarrow$ \ird{mida} `pot-\mk{pat}'
	\a \ird{máliek} `bonfire' $\rightarrow$ \ird{máliga} `bonfire-\mk{pat}'
	\xe

\end{enumerate}



\subsection{Reduplication}
\par Reduplication is a process whereby the stem or a part of the stem of a word, or the word itself is repeated with little or no change. 
\par Reduplication is only partially productive in Iridian. Most reduplicated noun forms, for example, have fossilized meanings.

\ex Initial reduplication (CV- prefix)\\
\ird{b\'or\v{z}} `thunder' 	$\rightarrow$ \ird{b\'ob\'or\v{z}} `rumbling sound'\\
\ird{man\'a} \trsl{to drop} $\rightarrow$ \ird{maman\'a} \trsl{to splatter}\\
\xe

\ex Final reduplication (-CV and -CCV suffixes)\\
\ird{b\'or\v{z}} `thunder' 	$\rightarrow$ \ird{b\'ob\'or\v{z}} `rumbling sound'\\
\ird{man\'a} \trsl{to drop} $\rightarrow$ \ird{maman\'a} \trsl{to splatter}\\
\xe

Full reduplication is more common than either initial or final-syllable reduplication, although it is limited (in general) to monosyllabic words with CV, VC or CVC structures.

A possibly grammatically meaningful usage of full reduplication is the repetition of words when answering yes-no questions\index{yes-no questions}. Iridian usually do not use the words for \trsl{yes} or \trsl{no} when responding to yes-no questions, instead repeating the verb. 

\subsection{Metathesis}

\subsubsection{slot a infixes}
%count A to Z prefixes 1 to inf suffixes
Slot A prefixes (grammatical voice and copulative form) metathesize the root when the onset is a cluster of two or more consonants subject to the below rules. In the examples we assume a affix of the type \textbf{\glot VC}. The glottal stop is deleted when the infix is added. The subscripts \textit{n} and \textit{s} are used to for phonemes relating to the infix and the stem respectively.

\begin{enumerate}
	\item Liquid-final clusters: C\tss{s}LV\tss{s} + \glot V\tss{n}C\tss{n} $\rightarrow$ C\tss{s}V\tss{n}LC\tss{s}V\tss{n}
	\begin{center}
	\begin{tabu}to 0.8\textwidth{YM[0.1]Y}
		\textbf{trápe} `cloud'&$\rightarrow$&\textbf{turtápe} `cloudy'\\
		\textbf{tresz} `write (\mk{st})'&$\rightarrow$&\textbf{torveszé}\\
		\textbf{szran} `drink (\mk{st}) &$\rightarrow$&\textbf{szirnaná}\\
	\end{tabu}
	\end{center}

	\item Nasal-final clusters: C\tss{s}NV\tss{s} + \glot V\tss{n}C\tss{n} $\rightarrow$ C\tss{s}V\tss{n}NC\tss{s}V\tss{n}
	\begin{center}
		\begin{tabu}to 0.8\textwidth{YM[0.1]Y}
			\textbf{dnoja} `money'&$\rightarrow$&\textbf{duntoja} `rich'\\
		\end{tabu}
	\end{center}
\end{enumerate}

\section{Prosody}

Stress is not phonemic and generally falls on the penultimate syllable of a word, whether the word is simple, derived or compound. For longer words, secondary stress may be placed on each even-numbered syllable counting backwards from the penultimate one.

\pex
\a \ird{\v{s}tudent}, \trsl{student}\\
\nt{"CtuDEnt}

\a \ird{\v{s}tudenta}, \trsl{student} (pat.)\\
\nt{Ctu"DEnta}

\a \ird{\v{s}tudentr\'ad}, \trsl{university dormitory}\\
\nt{Ctu"DEntra:t}
\xe

A primary exception includes a small class of interjections (most, but not all, of them onomatopoeic), where the stress is placed on the last syllable.

\pex
\a \ird{ah\'a}, \nt{5"xa:}

\a \ird{kab\'um}, \nt{\jn{kx}5"bu:m}
\xe

Loanwords whose stress pattern is different than this general rule are often (but not always) assimilated:


\pex
\a \ird{vizika}, \trsl{physics}\\
\nt{v1"zIx5}

\a \ird{mat\'ematik}, \trsl{mathematics}\\
\nt{m5""te:m5"tIx}
\xe

In addition, the enclitic interrogative particle \ird{no} and inclitic pronouns in general do not affect the location of the stress.

\pex
\a \ird{koma}, \trsl{good}\\
\nt{"\jn{kx}Om5}

\a \ird{Koma-no?}, \trsl{Is (it) good?}\\
\nt{"\jn{kx}Om5""nO} instead of \nt{\jn{kx}O"manO} 
\xe

\pex
\a \ird{kvartir}, \trsl{apartment}\\
\nt{"\*wart1\zz}

\a \ird{kvartirem}, \trsl{my apartment}\\
\nt{"\*wart1""r\~E\~w} instead of \nt{\*war"tIr\~E\~w} 
\xe

\section{Orthographic representation}
\subsection{Alphabet}

\par The Iridian language uses the Latin script with the following 29 letters: \textbf{a, b, c, \v{c}, d, e, f, g, h, i, j, k, l, m, n, o, p, q, r, s, \v{s}, t, u, v, w, x, y, z, \v{z}}.

The language was originally written in its own script but after the Latin alphabet has been adapted and has been in use since the First Bohemian Union in the 14th century. Due to the historical ties with the Kingdom of Bohemia and its historical successors, Czech orthography has had a great influence on the orthography of Iridian.

The last major change in the orthography of the language was during the 1843 reform, when the spellings <h> and <ch>, historically representing the phonemes \bt{h} and \bt{x} have been merged to <h> (representing \bt{x}), as the language lost the distinction between the two.

-ch still used at the end of a word

\par Iridian uses two types of diacritics, the acute accent ( ´ ), which is used to mark long vowels, and the circumflex accent ( ˆ ) used to mark nasal vowels. Accented characters are not considered as separate letter.

\begin{table}
	\small
 	\centering
 	\caption{The Iridian alphabet.}\index{alphabet}
	\begin{tabu}to \textwidth {>{\bfseries}MM[2]M>{\bfseries}MM[2]M}
		\toprule
		{\normalfont{\sc  symbol}} & {\sc name} & {\sc ipa} & {\normalfont{\sc  symbol}} & {\sc name} & {\sc ipa}\\
		\midrule
		\addlinespace
		A a	  		& á 	& \bt{a} 		& O o 		& \'o 		& \bt{O}\\\addlinespace
		B b			& b\'e	& \bt{b}		& P p		& p\'e		& \bt{p}\\\addlinespace
		C c			& c\'et & \bt{\jn{ts}}	& Q q		& kv\'e		& --\\\addlinespace
		\v{C} \v{c} & \v{c}a& \bt{\jn{tC}}	& R r		& er		& \bt{r}\\\addlinespace
		D d			& d\'e	& \bt{d}		& S s		& es		& \bt{s}\\\addlinespace
		E e			& \'e	& \bt{e}		& \v{S} \v{s}& \'e\v{s} & \bt{C}\\\addlinespace
		F f			& f\'i	& --			& T t		& t\'e		& \bt{t}\\\addlinespace
		G g			& g\'e 	& \bt{g}		& U u 		& \'u		& \bt{u}\\\addlinespace
		H h			& há 	& \bt{x}		& V v 		& v\'e 		& \bt{v}\\\addlinespace
		I i			& í 	& \bt{i}		& W w 		& vének		& --\\\addlinespace
		J j			& j\'yt& \bt{j}		& X x 		& iks 		& --\\\addlinespace
		K k 		& ká 	& \bt{k}		& Y y 		& ýpsý\'lon & \bt{y}\\\addlinespace
		L l 		& el 	& \bt{l}		& Z z		& zet 		& \bt{\textctz}\\\addlinespace
		M m			& em 	& \bt{m}		& \v{Z} \v{z}& \v{z}es 	& \bt{\zz} \\			\addlinespace
		N n			& en	&				&			&			&\\ \addlinespace
		\bottomrule	
	\end{tabu}
\end{table}
 
\begin{table}[h!]
	\small
 	\centering
 	\caption{Supplementary characters used in Iridian.}
	\begin{tabu}to \textwidth {>{\bfseries}MM[2]MM[2]}
		
		\toprule
		{\normalfont{\sc  symbol}} & {\sc name} & {\sc ipa} & {\sc name in ipa} \\
		\midrule

		\'A	\'a		& ne\v{c}ko \'a 	& \bt{a:} & \nt{"nE\jn{tC}kOPa:}\\ \addlinespace
		\k{A} \k{a}	& \'a \v{s}e mo\v{z}u & \bt{\~5\~w}&\nt{a:S1"mO\zz u}\\ \addlinespace
		\'E \'e		& ne\v{c}ko \'e 	& \bt{e:} & \nt{"nE\jn{tC}kOPe:}\\ \addlinespace
		\k{E} \k{e}	&\'e \v{s}e mo\v{z}u & \bt{\~E\~w}&\nt{e:S1"mO\zz u}\\ \addlinespace
		\'I \'i		& ne\v{c}ko \'i 	& \bt{i:} & \nt{"nE\jn{tC}kOPi:}\\ \addlinespace
		\'O \'o		& ne\v{c}ko \'o 	& \bt{o:} & \nt{"nE\jn{tC}kOPo:}\\ \addlinespace
		\k{O} \k{o}	&\'o \v{s}e mo\v{z}u & \bt{\~O}&\nt{o:S1"mO\zz u}\\ \addlinespace
		\'U \'u		&ne\v{c}ko \'u 	& \bt{u:} & \nt{"nE\jn{tC}kOPu:}\\ \addlinespace
		\'Y \'y		&ne\v{c}ko \'yps\'il\k{o} 	& \bt{y:} & \nt{"nE\jn{tC}kOP:"y:psi:""l\~O}\\ \addlinespace
		\"Y \"y		& \'yps\'il\k{o} \v{s}e tr\'emu &\bt{y}&\nt{"y:psi:""lOnC1"tRe:m5}\\ \addlinespace
		\bottomrule
	\end{tabu}
\end{table}
 
