\chapter{Verbs}
\section{Categories}
\par Finite verbs (\textbf{lounehlý}) are marked for the following grammatical categories:
\begin{enumerate}
	\item \textit{Aspect}. Iridian has three primary aspects: perfective, imperfective and contemplative; and two secondary ones: retrospective and prospective.
	\item \textit{Focus}. Iridian has a strong tendency to leave the topic of the sentence unmarked, instead encoding the primary information on the verb. Due to this, voice must be explicitly marked on the verb. Iridian has the following grammatical voices:: agentive, patientive, benefactive, instrumental, locative and reflexive.
	\item \textit{Mood}. Besides the unmarked indicative, Iridian has the following grammatical moods: subjunctive, conditional, hortative, optative, abilitative, permissive and non-volitive. In addition, secondary prefixes are used to express what would otherwise could be considered as moods: inceptive, causative and reciprocative.
\end{enumerate}

\par Verbs are also marked for person, although this is done by the addition of clitic pronouns and not through a separate conjugation paradigm. Iridian verbs are not marked for tense, gender, or number.
\par Iridian verbs have four classes of non-finite forms: the gerund, the converb, the supine and the generic nominal formed with \textbf{-ou}. The non-finite verb forms are derived from the uninflected verb stem except the generic nominal in \textbf{-ou} which can only be formed from a fully-inflected verb stem. A fifth class exists--the infinitive--but this form is largely defunct and is only used in certain compound constructions. Infinitives end in \textbf{-á} and is used as the citation form of a verb.

\section{Verb stems and citation forms}

\par The citation form of a verb is the uninflected infinitive, a fossilized form rarely used outside of a very few periphrastic construction. The uninflected infintive is formed by the verb stem and the infinitive markers \textbf{-á} or \textbf{-é}, depending on the quality of the stem vowel.



\section{Agglutination vs conjugation}
\par Iridian is superficially an agglutinative language. However, agglutination only exists in its full form in the indicative mood and in a reduced form in the optative moods. The conditional, quotative and subjunctive has entirely dropped the agglutination, instead using fusional conjugation paradigm. These system are normally called Type I, Type II, and Type III paradigms respectively.
\par Nevertheless, Iridian verbs are analyzed to have eight primary affix slots. Five of these slots are for suffixes and are numbered from one to five counting from the stem. There are three prefix slots, used for both infixes and prefixes, labeled A to C starting from the stem. Table \ref{slots} shows the affix slots used for these three groups of grammatical moods.

\begin{table}[h!]
	\centering \small
	\caption{Verbal affix slots.}
	\begin{tabu} to \textwidth {M[0.3]YYY}
		\toprule
		{\sc slot} & \multicolumn{1}{c}{\sc indicative}&\multicolumn{1}{c}{\sc optative} &\multicolumn{1}{c}{\sc other}\\
		\midrule \addlinespace
		C & Negation&Negation&Negation\\  \addlinespace
		B & Secondary verb prefixes&--&--\\ \addlinespace
		A & Voice&--&-- \\ \addlinespace
		0 & Stem&Stem&Stem\\ \addlinespace
		1 & Secondary pronoun&--&Paradigm ending\\ \addlinespace
		2 & Mood&Voice&--\\ \addlinespace
		3 & Aspect&Aspect&--\\ \addlinespace
		4 & Primary pronoun&Primary Pronoun&--\\ \addlinespace
		5 & Non-finite ending&Mood&--\\ \addlinespace
		\bottomrule
			\label{slots}
	\end{tabu}
\end{table}

\subsection{Stem}
There are two types of verb stems in Iridian: verbal and nominal. Iridian does not have a clear distinction between nominals and verbals. In this sense, verbal stems refer to stems that can be used on their own as imperatives. On the other hand, nominal stems refer to stems that could be used on their own as nouns or adjectives.
\par Examples of verbal stems include \ird{pia\v{s}tá} `to eat', \ird{vyté} `to go', \ird{kravná} `to cry', \ird{cselé} `to leave', etc. Examples of nominal stems include \ird{pledy} `red'; \ird{aro} `water', etc.
\par Some stems contain an unstable vowel


\subsection{Type I paradigm}

\subsubsection{indicative mood}
\par The simplest construction other than the unmarked stem involves a prefix for voice, and suffixes for the modality, aspect and the primary pronoun. Slot 4 suffixes have a strong tendency to be dropped when evident from context. As such in the unmarked indicative mood, most verbs would have only the affixes for voice and aspect.

\ex \ird{povia\v{s}tak}, `(I) ate.'
\begin{center}
	\small
	\begin{tabu}to \textwidth{MMMM[1.5]MMMMM}
		\toprule
		C&B&A&{\sc stem}&1&2&3&4&5\\
		\midrule
		\addlinespace
		-&-&<ov>&pia\v{s}t&-&$\varnothing$&-ak&$\varnothing$&-\\
		\bottomrule
	\end{tabu}
\end{center}
\xe

\par Slot 1 is used for secondary clitic pronouns, i.e., the object of the verb. Secondary clitic pronouns are also often dropped, however, and the inanimate 3rd person pronouns are almost never used.
\ex \ird{voide\v{s}kem}, `I saw you.'
\begin{center}
	\small
	\begin{tabu}to \textwidth{MMMM[1.5]MMMMM}
		\toprule
		C&B&A&{\sc stem}&1&2&3&4&5\\
		\midrule
		\addlinespace
		-&-&<oiv>&vyd&e\v{s}&$\varnothing$&-ek&-em&-\\
		\bottomrule
	\end{tabu}
\end{center}

\xe

\par There is a separate conjugation paradigm for the negative, synthesized with slot 3 aspect suffixes. There are, however, some 60 irregular verbs that use the negative prefix \ird{ná-/nái-} instead of the negative aspectual suffixes, in addition to stative or copulative verbs, which can only be used with the prefix \ird{ná-/nái-}.

\pex 
\a \ird{voide\v{s}oitem}, `I didn't see you.'
\begin{center}
	\small
	\begin{tabu}to \textwidth{MMMM[1.5]MMMMM}
		\toprule
		C&B&A&{\sc stem}&1&2&3&4&5\\
		\midrule
		\addlinespace
		-&-&<oiv>&vyd&-e\v{s}&$\varnothing$&-oit-&-em&-\\
		\bottomrule
	\end{tabu}
\end{center}

\a \ird{nájemnoutalý}, `Your place is closed.'
\begin{center}
	\small
	\begin{tabu}to \textwidth{MMMM[1.5]MMMMM}
		\toprule
		C&B&A&{\sc stem}&1&2&3&4&5\\
		\midrule
		\addlinespace
		ná-&je-&-&mnout&-&$\varnothing$&-&-alý&-\\
		\bottomrule
	\end{tabu}
\end{center}

\xe

\subsubsection{imperative mood}

\par The Iridian imperative mood has two forms: a singular and a plural. Unlike the indicative mood, there is no separate conjugation paradigm for the negative; instead the prefix \ird{ná-/nái-} is used, analyzed as a slot C prefix.

\pex 
\a \ird{pia\v{s}te}, `Eat!'
\begin{center}
	\small
	\begin{tabu}to \textwidth{MMMM[1.5]MMMMM}
		\toprule
		C&B&A&{\sc stem}&1&2&3&4&5\\
		\midrule
		\addlinespace
		-&-&$\varnothing$&pia\v{s}t&-&-o&-&-&-\\
		\bottomrule
	\end{tabu}
\end{center}

\a \ird{nápia\v{s}tet}, `Don't eat!'
\begin{center}
	\small
	\begin{tabu}to \textwidth{MMMM[1.5]MMMMM}
		\toprule
		C&B&A&{\sc stem}&1&2&3&4&5\\
		\midrule
		\addlinespace
		ná&-&$\varnothing$&pia\v{s}t&-&-et&-&-&-\\
		\bottomrule
	\end{tabu}
\end{center}
\xe

\subsubsection{copulative form}
\par Verbs have two copulative forms in Iridian. The copulative in \ird{je-} is a slot B prefix while the copulative in \ird{ut/uit} is a slot A infix. If not followed by he prefix \ird{je-} requires an epenthetic \ird{-o/-eu}, analyzed as a slot 5 suffix.

\ex \ird{jemohlam}, `I am living.'

\begin{center}
	\small
	\begin{tabu}to \textwidth{MMMM[1.5]MMMMM}
		\toprule
		C&B&A&{\sc stem}&1&2&3&4&5\\
		\midrule
		\addlinespace
		-&je-&-&mohl&-&-&-&-am&-\\
		\bottomrule
	\end{tabu}
\end{center}
\xe

\ex \ird{Marek jesorto}, `Marek is standing.'

\begin{center}
	\small
	\begin{tabu}to \textwidth{MMMM[1.5]MMMMM}
		\toprule
		C&B&A&{\sc stem}&1&2&3&4&5\\
		\midrule
		\addlinespace
		-&je-&-&sort&-&-&-&$\varnothing$&-o\\
		\bottomrule
	\end{tabu}
\end{center}
\xe

\section{Voice}

\par Iridian often prefers to encode information on the verb instead of through case marking on nouns. As such, all verbs must be explicitly marked for voice.
\par Voice markers are infixing. They are inserted into the verb stem, following the initial consonant or consonant cluster. 

\begin{table}[h!]
	\small \centering
	\caption{Infixes used in marking voice.}
	\begin{tabu} to 0.8\textwidth{YMM}
		\toprule
		&\multicolumn{2}{c}{\sc stem type}\\ \addlinespace
		\cmidrule{2-3} \addlinespace
		& Verbal&Nominal\\ 
		\midrule
		Agentive	& -ov-/-oiv-&-un-/-uin-\\ \addlinespace
		Patientive	& -in- & -az-/-ez-\\ \addlinespace
		Benefactive	& -ul-/&-us-/-uis-\\ \addlinespace
		Locative	& -át-/-áit- & -át-/-áit-\\ \addlinespace
		Instrumental& -a\v{s}-/-e\v{s}- & -a\v{s}-/-e\v{s}-\\ \addlinespace
		Reflexive	& -iz-&-a\v{s}t-/ -e\v{s}t-\\ \addlinespace
		Reciprocal	& -ol-/-oil- & -olt-/-oilt-\\ \addlinespace
		
		\bottomrule
	\end{tabu}
\end{table}


\subsection{Agentive voice}
\par A verb in the actor focus indicates that the topic of the sentence is the agent of the verb.

\pex 
\begingl
\gla Z povia\v{s}tak.//
\glb already \mk{<av>}eat-\mk{pf}//
\glft `(I) already ate.'//
\endgl
\xe


\subsection{Patientive focus}
\par A verb in the patient focus (glossed \mk{pat}) indicates that the topic of the sentence is the patient of the verb.

\pex
\begingl
\gla Marek vindekem.//
\glb Marek \mk{<pv>}see-\mk{pf-1s}//
\glft `I saw Marek.'//
\endgl
\xe


\subsection{Benefactive focus}
\par The benefactive focus (glossed \mk{ben}) is used when the subject of the sentence is the benefactor or director object of the verb. Verbs often change meaning when used in the benefactive focus.

\pex
\begingl
\gla Maty sega vstulunek.//
\glb mother flower-\mk{pat} \mk{<ben>}buy-\mk{pf}//
\glft `(I) bought my mother flowers.'//
\endgl
\xe


\subsection{Locative Focus}

\subsection{Instrumental Focus}


\subsection{Reflexive Focus}

\par The reflexive focus (glossed \mk{ref}) is used when the patient of the verb is also the agent.

\par The reflexive cannot be used with verbs with the prefix \textbf{ce-}.

\subsection{Usage}

\par The differences 

\section{Grammatical Aspect}
\begin{table}[h!]
	\centering
	\caption{Aspect markers in the indicative mood.}
	\begin{tabu} to 0.8\textwidth{MM}
		\toprule
		{\sc aspect}	& {\sc affix}\\
		\midrule
		Perfective		& -ek\\
		Retrospective	& -an\'i\\
		Imperfective	& -\'al\\
		Contemplative	& -\k{a}c\\
		Progressive		& -\'au \\
		Prospective		& -il\\
		Cessative		& -eic\\
		\bottomrule
	\end{tabu}

\end{table}
\subsection{Perfective aspect}
The perfective aspect (glossed {\sc pf}) indicates an action that has been completed in some specific instance.

\pex
\begingl
\gla Bych na gna\v{z}a Marek vdinek.//
\glb yesterday \mk{loc} school-\mk{pat} Marek see-\mk{pv-pf}//
\glft \trsl{(I) saw Marek at school yesterday.}//
\endgl
\xe

\pex
\begingl
\gla Va\v{s}ko pia\v{s}tnek.//
\glb pastry eat-\mk{pv-pf}//
\glft `(I) ate (the) cake.'//
\endgl
\xe

\par The vowel in the suffix is unstable and the ending would normally collapse to \textbf{-k} when followed by another vowel. Consider the above two sentences followed by the second person singular clitic pronoun \textbf{-a\v{s}/e\v{s}}.

\pex
\begingl
\gla Bych na gnazsa Marek vindeke\v{s}.//
\glb yesterday \mk{loc} school-\mk{pat} Marek \mk{<pv>}see-\mk{pv-pf-2s}//
\glft `You saw Marek at school yesterday.'//
\endgl
\xe

\pex
\begingl
\gla Va\v{s}ko pinia\v{s}tka\v{s}.//
\glb pastry \mk{<pv>}eat-\mk{pf-2s}//
\glft `You ate (the) cake.'//
\endgl
\xe


\par When negated, the perfective indicates something that ought to be done but had not been done. To state that something simply did not happen, the negative of the retrospective is used instead.

\pex
\begingl
\gla Z\'at\'el\'evonirnek.//
\glb \mk{neg}-telephone-\mk{pf}//
\glft `(I) failed to call.' //
\endgl
\xe

\pex
\begingl
\gla Z\'at\'el\'evonirnan\'i.//
\glb \mk{neg}-telephone-\mk{ret}//
\glft `(I) didn't call.' //
\endgl
\xe

\subsection{Retrospective aspect}
\par The retrospective aspect (glossed \mk{ret}) is used for a past action that has a continuing relevance in the presence. Consider, for example, the following sentences: (a) \textit{I went to Amsterdam last week}; and (b) \textit{I have been to France in my childhood}. Iridian would translate the verb in (a) using the perfective and the verb in (b) using the retrospective.

\pex<ret-pres1>
\begingl
\gla Hroná ko tímu na Budape\v{s}ta mohlan\'im.//
\glb three \mk{att} year-\mk{inst} \mk{loc} Budapest-\mk{pat} live-\mk{ret-1s}//
\glft `I have been living in Budapest for three years.'//
\endgl
\xe

\pex<ret-pres>
\begingl
\gla Páku \v{s}avolnan\'ic.//
\glb before-\mk{inst} hurt-\mk{pv-pf-3s.anim}//
\glft `She has been hurt before.' //
\endgl
\xe

\par The retrospective is also often used to imply non-volition or the  accidental/circumstantial nature of an action. Similarly the retrospective is used with verbs of emotion or state (e.g., \ird{cezu\v{s}talá}, ‘to become happy’ from \ird{zu\v{s}tal} ‘happy’). The perfective, on the other hand, is almost exclusively used with the causative in these cases.

\pex
\a	\begingl
\gla Vde\v{s}ek \v{s}e neicezu\v{s}talan\'im.//
\glb see-\mk{2s-pf} with \mk{incep}-be.happy-\mk{ret-1s}//
\glft `I became happy when I saw you.' //
\endgl
\a	\begingl
\gla Do pacezu\v{s}talnike\v{s}.//
\glb \mk{1s.wk} \mk{caus}-be.happy-\mk{pv-pf-2s}//
\glft `You made me happy.' //
\endgl
\xe
\pex<vasebroke>
\begingl
\gla Váz noprizan\'i.//
\glb vase break-\mk{ref-ret}//
\glft `The vase broke (accidentally).' //
\endgl
\xe

\subsection{Imperfective aspect}

\par The imperfect aspect (glossed \mk{ipf}) is used for actions not viewed as being complete or still in the process of happening, as well as habitual actions or general truths. The suffix \textbf{-ê} becomes \textbf{-êm-} before \textbf{a} or \textbf{á} and \textbf{-ên-} before any other vowel. 

\pex
\begingl
\gla Prahá jemohlem.//
\glb Prague-\mk{loc} \mk{cop.stat}-live-\mk{1s}//
\glft `I live in Prague.' //
\endgl
\xe

\pex
\begingl
\gla Vsholu de gnazsa vtênuh. //
\glb daily-\mk{inst} \mk{ill}  school-\mk{pat} go-\mk{ipf-1pl.incl}//
\glft `We go to school everyday.'//
\endgl
\xe

\par To emphasize the habitual nature of an action, the verb is often nominalized.

\pex
\a	\begingl
	\gla Po\v{s}nej radzê. //
	\glb father-\mk{3s.anim} smoke\mk{-ipf}//
	\glft `His father is smoking (right now).' //
	\endgl
\a	\begingl
	\gla Po\v{s}nej radzênou.//
	\glb father-\mk{3s.anim} smoke\mk{-ipf-nz}//
	\glft `His father is a smoker.' //
	\endgl
\xe

\subsection{Progressive aspect}
\par The progressive aspect (glossed \mk{prog}) is used to express an incomplete action that is in progress at a specific point in time.

\pex
\a	\begingl
\gla Csuid etoren. //
\glb history study-\mk{ipf}//
\glft `(I am) studying history.' //
\endgl

\a	\begingl
\gla Óilen etorlava.//
\glb exam study\mk{-ben-prog}//
\glft `(I am) studying for the exam.' //
\endgl
\xe

\par Some verbs, especially stative verbs, change meaning when used in the progressive aspect.

\pex
\a	\begingl
\gla He\v{s} dóilninen.//
\glb reason know\mk{-pv-ipf.neg}//
\glft `(I) don't know the reason.' //
\endgl

\a	\begingl
\gla He\v{s} doilnineve.//
\glb reason know\mk{-pv-prog}//
\glft `(I am) trying to know the reason.' //
\endgl
\xe


\subsection{Prospective aspect}
\par The prospective aspect (glossed {\sc prosp}) is primarily used in secondary clauses to indicate actions that are about to be started in relation to another action. It can also be used in the main clause to indicate an action in the immediate future.

\subsection{Cessative aspect}




\section{Secondary Verbal Prefixes}
In addition to the prefixes used for verbal derivation, Iridian has three prefixes that are analyzed as separate moods.
\subsection{The reciprocative so-}

\section{Grammatical Mood}

\subsection{Indicative}

\subsection{Imperative}
\par Iridian has an imperative mood formed by attaching the suffix \textbf{-e} in the singular and \textbf{-et} in the plural to the verb stem.

\begin{table}[h!]
	\begin{tabular}{ll}
		\textbf{pia\v{s}te} & `eat'\\
		\textbf{pia\v{s}tet} &`eat (\mk{pl})\\		
	\end{tabular}
\end{table}

\par The imperative is largely considered rude and impolite in modern speechc, with the hortative being used even for commands. Nonetheless, the imperative can be often found in literary texts.

\pex
\begingl
\gla Nátiezne.//
\glb \mk{neg}-kill-\mk{imp}//
\glft`Thou shall not kill'//
\endgl
\xe

\par The imperative can be combined with the reciprocative prefix \textbf{so-} to form the adhortative, similar in meaning to the English `Let's\ldots'. Nevertheless, the hortative construction with verbs in \textbf{so-} is still preferred.
\subsection{Subjunctive}
\par The subjunctive mood (glossed \mk{sbj}) is used for actions or events that are not or are not known to be true or factual. The subjunctive has two forms: the perfective subjunctive in \textbf{-á(m)} and the imperfective subjunctive in \textbf{-ú(m)}, glossed as \mk{sbj.pf} and \mk{sbj.ipf} respectively.
\par The copula has the following forms in the subjunctive, all of which are not inflected:

\begin{table}[h!]
	\centering
	\caption{Subjunctive forms of the copula}
	\begin{tabularx}{0.7\textwidth}{YMM}
		\toprule
		&{\sc non-neg}&{\sc neg}\\
		\midrule
		Imperfective&niec &poce\\ \addlinespace
		Perfective& vace & náh\\
		\bottomrule
	\end{tabularx}
\end{table}
\par The perfective-imperfective distinction of the subjunctive is more properly analyzed as a temporal distinction, i.e., past and non-past subjunctive. Iridian does have, howevever has a periphrastic construction for the progressive subjunctive formed by the imperfective converbial in \textbf{-iec} and the perfective subjunctive copula. Similarly, the future subjunctive also uses a periphrastic construction with the imperfective converbial in \textbf{-iec} and the imperfective subjunctive copula. Table \ref{pia\v{s}ta} shows the conjugation paradigm of the verb \textbf{pia\v{s}tá} in the subjunctive mood.
\begin{table}[h!]
	
	\centering \small 
	
	\caption{Conjugation, paradigm, subjunctive mood.}
	\begin{tabu} to 0.7\textwidth{Y>{\bfseries}Y}
		\toprule
		\multicolumn{2}{c}{\textbf{pia\v{s}tá} `to eat'}\\
		\midrule \addlinespace
		 Imperfective& pia\v{s}tú\\ \addlinespace
		 Negative imperfective & nápia\v{s}tú\\ \addlinespace
		 Perfective & pia\v{s}tá\\ \addlinespace
		 Negative perfective & nápia\v{s}tá\\ \addlinespace
		 Progressive&pia\v{s}tiec vace\\ \addlinespace
		 Negative progressive&pia\v{s}tiec náh\\ \addlinespace
		 Future & pia\v{s}tiec niec\\ \addlinespace
		 Negative Future & pia\v{s}tiec poce\\ \addlinespace
		 \bottomrule
	\label{pia\v{s}ta}		 
	\end{tabu}
\end{table}
\par The following are some specific uses of the subjunctive mood in Iridian:
\subsubsection{jussive/desiderative}
\par The subjunctive is used in indirect constructions of verbs for issuing orders, commanding, exhorting, etc.
\pex
\begingl
\gla Martin na America mágnazsivú to csehnêmas.//
\glb Martin \mk{loc} America-\mk{pat} go.to.school-\mk{sbj.ipf} \mk{rz} want-\mk{ipf-3s.anim}//
\glft `He wants Martin to study in America.'//
\endgl
\xe

\pex
\begingl
\gla Marek a\v{s}ná to Tunek dálek.//
\glb Marek sing-\mk{sbj.ipf} \mk{rz} Tunek say-\mk{pf}//
\glft `Tunek told Marek to sing.'//
\endgl
\xe

\subsubsection{dubitative}
\par The subjunctive is used with verbs expressing doubt, uncertainty or disbelief.

\subsubsection{with verbs expressing emotion}


\subsubsection{with the conditional mood}
\par The subjunctive is used in the main clause if the verb in the dependent clause is in the conditional \textit{irrealis} mood.

\pex
\begingl
\gla Dá prezident jenem, //
\glb a//
\glft a//
\endgl
\xe

\subsubsection{irrealis}

\subsection{Conditional}
\par The conditional mood is used for conditional or hypothetical clauses. The table below shows the conjugation paradigm for the conditional mood for both regular verbs and the copula. The Iridian conditional mood is not a true conditional mood grammatically, since it is marked on the verb in the dependent clause (protasis), instead of the main clause.

\begin{table}[h!]
	\centering \small
	\caption{Conjugation paradigm, conditional mood.}
	\begin{tabu} to 0.9 \textwidth	{Y[1.3]MM}
		\toprule
		&{\scshape regular verbs} & {\scshape copula}\\
		\midrule
		
		\textit{Realis} &-ouhna&vsiec\\
		Neg. \textit{Realis}&-ouhnit&vsiemý\\
		
		Non-Past \textit{Irrealis} & -ouc & jenouc\\
		Neg. Non-Past \textit{Irrealis} & -oucit& piêc\\
		
		Past \textit{Irrealis} & -ane & jenem\\
		Neg. Past \textit{Irrealis} & -oucnit & jet\\
		\bottomrule
	\end{tabu}
\end{table}

\subsubsection{conditional realis}

\par The conditional \textit{realis} mood (glossed \mk{cond.rl}) is used in two ways:
\begin{enumerate}
	\item In sentences that express a factual implication rather than a hypothetical situation or a potential future event, e.g., `If you heat water to 100 C, it will boil.'
	\item In `predictive' constructions, i.e., those that concern probable future events.
\end{enumerate} 

\subsubsection{conditional irrealis}
The conditional \textit{irrealis} mood (glossed \mk{cond.irr}) is used with hypothetical, typically counterfactual, events. Iridian distinguishes between past and non-past \textit{irrealis} moods.


\subsection{Hortative}
\par The hortative mood is used for requests. Although Iridian has an imperative form (the unmarked form of the verb), the hortative is normally used in its place. The hortative marker should always appear at the end of the word.

	\pex
\begingl
\gla Jê\v{s}a mine\v{s}ka.//
\glb door.\mk{pat} close-\mk{2s-hort}//
\glft 'Close the door.' \textit{literally,} `May you close the door.'//
\endgl
\xe

\par To soften a command, the expression \textit{am luhninka} (may someone be thanked for\ldots) is normally used.

\pex
\begingl
\gla Jê\v{s}a minke\v{s} ce\v{s} am luhninka.//
\glb door-\mk{pat} close-\mk{pf-2s} \mk{rz.abl} because thank\mk{-pv-hort}//
\glft  `Please close the door.' \textit{literally,} `May (you) be thanked because you closed the door.'//
\endgl
\xe

\par The hortative is used with the reciprocative prefix \textbf{so-} to form the adhortative (similar to the English construction with `Let's + \mk{verb}). This construction cannot be used with \textbf{am luhninka}.

\pex
\begingl
\gla sop//
\glb door-\mk{pat} close-\mk{pf-2s} \mk{rz.abl} because thank\mk{-pv-hort}//
\glft  `Please close the door.' \textit{literally,} `May (you) be thanked because you closed the door.'//
\endgl
\xe

\subsection{Optative}
The optative mood (glossed \mk{opt}) is used for expressing wishes. The optative mood requires two aspect marking, although the primary ending is marked if it is in the imperfective mood.



\subsection{Quotative }	
\par The quotative mood (glossed \mk{quot}) is used to express secondhand information, or when the speaker wishes to make explicit that s/he did not witness the event himself/herself.
\par Clitic pronouns cannot be used with the quotative mood.
\par Table \ref{conj-quot} shows the conjugation paradigm for regular verbs and the copula.


\begin{table}[h!]
	\centering \footnotesize
	\caption{Conjugation paradigm, quotative mood}
	\begin{tabu} to \textwidth{Y[1.3]YY[0.8]}
		\toprule
		&\multicolumn{1}{c}{\textbf{pia\v{s}tá}, `to eat'}& \multicolumn{1}{c}{\sc copula}\\
		\midrule
		Perfective & pia\v{s}tát & vacet\\
		Neg. perfective & nápia\v{s}tát & necê\\
		Retrospective & pia\v{s}tác & ---\\
		Neg. Retrospective & nápia\v{s}tác&---\\
		Imperfective & pia\v{s}tút & ne\v{s}kec \\
		Neg. imperfective & nápia\v{s}tút & po\v{s}nec\\
		Progressive & pia\v{s}tiec ne\v{s}kec&---\\
		Neg. progressive & pia\v{s}tiec po\v{s}nec&---\\
		Future & pia\v{s}tô\v{s} & vacko\\
		Neg. Future & nápia\v{s}tô\v{s} & necko\\
		Subjunctive Non-Past& pia\v{s}tok &necim\\
		Neg. Sub. Non-Pas& nápia\v{s}tok & pocim\\
		Subjunctive Past & pia\v{s}tocke & vacim\\
		Neg. Sub. Past & nápia\v{s}tocke & nêcim\\
		\bottomrule
			\label{conj-quot}
	\end{tabu}

\end{table}


\pex
\begingl
\gla Já na duma ne\v{s}kec to maty dálmek.//
\glb you-\mk{str} \mk{loc} house-\mk{pat} \mk{cop.quot.ipf} \mk{rz} mother say-\mk{1s.pf}//
\glft `(My) mother told me you are at home.'//
\endgl
\xe

\pex
\begingl
\gla Já na duma necim to maty dálmek.//
\glb you-\mk{str} \mk{loc} house-\mk{pat} \mk{cop.quot.sbj.npst} \mk{rz} mother say-\mk{1s.pf}//
\glft `(My) mother told me you might be at home.'//
\endgl
\xe

\pex
\begingl
\gla Mnúcs tiezninát.//
\glb husband kill-\mk{pv-quot.pf}//
\glft `(She) killed (her) husband (or so I heard).'//
\endgl
\xe

\par Direct speech, however, does not use the subjunctive.
\pex
\begingl
\gla ---Tak dá, dálek Tomá\v{s}.//
\glb here \mk{1s.str} say-\mk{pf} Tomá\v{s}//
\glft ```I'm here,'' Tomá\v{s} said.'//
\endgl
\xe


\par The following verbs are considered verba dicendi in Iridian and would trigger the quotative: \textbf{dálá} `to say', \textbf{vadá} `to think', \textbf{kvu\v{s}tá} `to hear', \textbf{vydá} `to see', \textbf{ége\v{s}á} `to ask', \textbf{ohletá} `to remember', \textbf{hová} `to recount, tell a story' . The verb \textbf{vadá} is exclusively used with the subjunctive quotative.

\pex
\begingl
\gla Z \v{s}to óké necim to Luká\v{s} vadê.//
\glb already this OK \mk{cop.quot.sbj.npst} \mk{rz} Luká\v{s} think-\mk{ipf}//
\glft `Luká\v{s} thinks it should be OK by now.'//
\endgl
\xe

\pex
\begingl
\gla Marek bych jsenát to kvu\v{s}tkem.//
\glb Marek yesterday arrive-\mk{quot.pf} \mk{rz} hear-\mk{pf-1s}//
\glft `I heard Marek has arrived.'//
\endgl
\xe


\pex
\begingl
\gla Po\v{s}nelý tajomstác to kvu\v{s}tek.//
\glb father-\mk{2pl} die-\mk{quot.ret} \mk{rz} hear-\mk{pf}//
\glft `(We) heard that your father died.'//
\endgl
\xe

\pex
\begingl
\gla Dá tak bych vacim to náohletê.//
\glb \mk{1s.str} here yesterday \mk{cop.quot.sbj.pst} \mk{rz} \mk{neg}-remember-\mk{ipf}//
\glft `(I) don't remember if I was here yesterday.'//
\endgl
\xe

\par Secondary verba dicendi are formed with an adverbial construction using the imperfective converb in \textbf{-iec}.

\pex
\begingl
\gla Já mnou necim to Martin priviec vadê.//
\glb you correct \mk{cop.quot.sbj.npst} \mk{rz} Martin agree-\mk{cv} think-\mk{ipf}//
\glft `Martin agrees that you are right.'//
\endgl
\xe

\par The quotative is also used emphatically to repeat a quote (often made by the speaker himself or herself), or to express the speaker's frustration or affirmation. When used this way, the verbum dicendi is omitted, and the expletive \textbf{nó} is often added.

\pex
\begingl
\gla Mnou necim to nó!//
\glb correct \mk{cop.quot.sbj.npst} \mk{rz} \mk{expl}//
\glft `(I've been telling you) it is right.'//
\endgl
\xe

\pex
\begingl
\gla Dá roctymút to!//
\glb \mk{1s} dance-\mk{abl-quot.ipf} \mk{rz}//
\glft `(But) I can dance.'//
\endgl
\xe

\par The tense/aspect of the quotative mood follows that of the quoted clause, independent of the tense/aspect of the verbum dicendi.



\subsection{Abilitative and Permissive}
\par The abilitative (glossed \mk{abl}) and permissive (glossed \mk{perm}) are related verbal moods used in expressing the speaker's (or the subject of the sentence's) abililty to do something. The abilitative is used to indicate capability while the permissive is used to indicate whether or not an action is allowed or permitted.

\pex
\begingl
\gla Sa anglecnu nározshovymas.//
\glb \mk{inst} English.language-\mk{inst} \mk{neg}-speak-\mk{abl-3s.anim}//
\glft `He cannot speak English.'//
\endgl
\xe

\pex
\begingl
\gla De rádaka z názahranave\v{s}.//
\glb \mk{ill} building-\mk{1pl.excl-pat} already \mk{neg}-enter-\mk{perm-2s}//
\glft `You're no longer allowed to enter our building.'//
\endgl
\xe

\par The permissive mood is often used for negative commands.

\pex
\begingl
\gla Tak náradzavuj.//
\glb here \mk{neg}-smoke-\mk{perm-4gen}//
\glft `No smoking.' \textit{literally,} `One cannot smoke here.'//
\endgl
\xe

\subsection{Non-Volitive}
\par The non-voliti
\ex[exno={\getref{vasebroke}, rep.}]
ago
\xe
\section{Non-Finite Verb Forms}

\subsection{Copulative form}
\par Verbs have two copulative forms: the adjectival copulative formed with the infix \textbf{-ut-/-uit-} and the stative copulative formed with the prefix \textbf{je-}.
\par The adjective

\begin{table}[h!]
	\small\centering
	\begin{tabularx}{0.8\textwidth}{YMY}
		\textbf{niva} `speed' &$\rightarrow$&\textbf{nutiva} `fast'\\
		\textbf{kiazó} `darkness'&$\rightarrow$&\textbf{kutiazó} `dark'\\
		\textbf{kocska} `distance'&$\rightarrow$&\textbf{kutocska} `far'\\
		\textbf{trápe} `cloud'&$\rightarrow$&\textbf{turtápe} `cloudy'\\
	\end{tabularx}
\end{table}

\par The stative copulative is formed with the prefix \textbf{je-} and can only be used with verbs that express a state, condition or position.

\begin{table}[h!]
	\small\centering
	\begin{tabularx}{0.8\textwidth}{YMY}
		\textbf{sort-} `stand' &$\rightarrow$&\textbf{jesort-} `standing'\\
		\textbf{mohl-} `live'&$\rightarrow$&\textbf{jemohl-} `living'\\
		\textbf{vtiel-} `open'&$\rightarrow$&\textbf{kutocska} `open'\\
	\end{tabularx}
\end{table}


\subsection{Gerund}
\par The gerund (glossed \mk{ger}) refers to the non-finite verb form used as a noun. The gerundive prefix \textbf{po-} is always used with the nominalizing suffix \textit{ou}.

\pex
\begingl
\gla Po\v{s}csênou nauhlý.//
\glb \mk{ger}-forget-\mk{nz} difficult//
\glft `Forgetting is difficult.'//
\endgl
\xe

\subsection{Converbs}
Converbs (glossed \mk{cv}) is a non-finite verb form often used for adverbial constructions. There are two converb forms in Iridian: the imperfective \textbf{-iec} (glossed \mk{cv.ipf}) and the perfective \textbf{-iêce} (glossed \mk{cv.pf}).

\pex
\begingl
\gla Tereza kravniec nóveu cselek. //
\glb Tereza cry-\mk{cv.ipf} room-\mk{abl} leave-\mk{pf}//
\glft `Tereza left the room crying.'//
\endgl
\xe

\pex
\begingl
\gla Nóveu cseliêce Tereza ukravnek.//
\glb room-\mk{abl} leave-\mk{cv.pf} Tereza \mk{incho}-cry-\mk{pf}//
\glft `Having left the room, Tereza started to cry.'//
\endgl
\xe

The perfective \textit{-iêce} is often used in clause linking.

\pex
\begingl
\gla O\v{s}tiêce krazkem.//
\glb read-\mk{cv.pf} understand-\mk{pf-1s}//
\glft `I read and understood.'//
\endgl
\xe

Clauses expressing reason is usually expressed by a converbial construction.

\pex
\begingl
\gla Za eksama názhaziêce, Martin órek.//
\glb for exam-\mk{pat} \mk{neg}-study-\mk{cv.pf} Martin fail-\mk{pf}//
\glft `Martin failed the exam because he didn't study.'//
\endgl
\xe


\subsection{Nominalization}



\subsection{Supine}
The supine is a non-finite verb form formed used to indicate necessity or purpose. There are four forms as shown below:

\begin{table}[h!]
	\centering\small
	\caption{Endings used for the supine}
	\begin{tabularx}{0.8\textwidth}{MMM}
		\toprule
		&{\sc supine of purpose}&{\sc supine of necessity}\\
		\midrule
		Nominal & \textit{-ity} & \textit{-á\v{s}}\\
		\addlinespace
		Non-nominal & \textit{-ice} & \textit{-á\v{s}ce}\\
		\bottomrule
	\end{tabularx}
\end{table}


	\pex
\begingl
\gla >>Ána Karenina<< za gnazsa o\v{s}tá\v{s}ce ko hto\v{s}.//
\glb Anna Karenina for school-\mk{pat} read-\mk{sup} \mk{att} book//
\glft `I have to read \textit{Anna Karenina} for school.'//
\endgl
\xe

	\pex
\begingl
\gla Hto\v{s} vstuninkem to o\v{s}tice.//
\glb book buy-mk{pv-pf-1s} \mk{rz} read-\mk{sup}//
\glft `I bought the book to read.'//
\endgl
\xe

\par The infinitive form of the supine of purpose \textit{-icá} is used with adjectival adverbs:

\pex
\begingl
\gla Just zacep\v{s}csemem to nosiênicá.//
\glb news \mk{caus}-be.sad-\mk{1s} \mk{rz} hear-\mk{sup.inf}//
\glft `I am sad to hear the news.'//
\endgl
\xe

\section{Copular Constructions}
\subsection{Null copula}

Copular sentences are a minor sentence type where the predicate is not a verb. For the purposes of this grammar, we narrow down our definition of copular constructions to the following:
\pex
\a \textit{Equative:} Marek is the doctor (we are talking about).
\a \textit{Inclusive:} Marek is a doctor.
\a \textit{Attributive:} Marek is tall.
\a \textit{Locative:} Marek is in the hospital.
\xe

Iridian does not make a distinction between equative, inclusive and attributive clauses. Locative clauses on the other hand, may be expressed using a copular or an existential construction, as will be discussed in this section.

Iridian is a superficially a zero-copula language and the most common way to form copular sentences is mere juxtaposition.

\pex<cop>
\begingl
\gla Marek doktor.//
\glb Marek doctor//
\glft \eng{Marek (is a/the) doctor.}//
\endgl
\xe

The above example could either be taken to mean (1) Marek is a doctor (inclusive), or (2) Marek is the doctor (equative). Generally, though, Iridian uses word order to distinguish between equative and inclusive clauses.

\pex
\a \textit{Inclusive:} \{item in class\}\tss{N} $\varnothing$ \{class\}\tss{P}
\a \textit{Equative:} \{class\}\tss{N} $\varnothing$ \{item class\}\tss{P}
\xe

To avoid ambiguity, Example \getref{cop} can be reformulated to either of the following sentences:

\pex<cop1>
\a 
\begingl
\gla Marek doktor.//
\glb Marek doctor//
\glft \eng{Marek is a doctor.}//
\endgl

\a 
\begingl
\gla Doktor Marek.//
\glb doctor Marek//
\glft \eng{Marek is the doctor.}//
\endgl

\xe

The inversion of word order is not strongly grammaticalized with NP-NP sentences, i.e., both sentences in Example \getref{cop1} can still be used interchangeably without a change in meaning and preference is given on the one over the other when there is an ambiguity. This is not the case with attributive clauses, i.e., sentences with adjective or adjective phrase predicates. Consider for example the sentence below:

\pex
\begingl
\gla Marek rázym.//
\glb Marek tall//
\glft \eng{Marek is tall.}//
\endgl
\xe

Inverting the word order of the sentence above would change the adjective to a substantive since modifiers cannot occupy the topic position.

\pex
\begingl
\gla Rázym Marek.//
\glb tall Marek//
\glft \eng{The tall one is Marek.}//
\endgl
\xe

Iridian also distinguishes between attributive clauses expressing permanent conditions and clauses expressing temporary conditions, with the latter being expressed using existential constructions in certain adjectives.

\pex
\begingl
\gla *Marek morec.//
\glb Marek hungry//
\glft \eng{Marek is hungry}//
\endgl
\xe


\pex
\begingl
\gla Marka je\v{s} morec.//
\glb Marek-\mk{pat} \mk{exst} hunger//
\glft \eng{Marek is hungry}//
\endgl
\xe

A full list of adjectives/modifiers that use the existential construction can be found in the section~\ref{sec:exst}.

\subsection{Negative copula}

Iridian has the negative copula \ird{csesná}.

\pex
\begingl
\gla Marek doktor csesná.//
\glb Marek doctor \mk{cop.neg}//
\glft \eng{Marek is not (a/the) doctor.}//
\endgl
\xe

\par The inversion of word order may also be used when one wants to avoid ambiguity:

\pex
\begingl
\gla Doktor Marek csesná.//
\glb doctor Marek \mk{cop.neg}//
\glft \eng{Marek is not the doctor.}//
\endgl
\xe

\section{Existential Constructions}
\label{sec:exst}

\section{Formation of Verbs}
\subsection{External Derivation}
\par Loanwords ending in \textbf{-ace} from the Latin change the final e to á:
\begin{table}[h!]
	\centering \small
	\begin{tabu} to 0.9\textwidth{>{\bfseries}YM[0.3]>{\bfseries}YY}
		administrace 	& $\rightarrow$ & administracá 	& `to administrate' \\
		akuzace			& $\rightarrow$ & akuzacá		& `to accuse'\\
		diferenzace		& $\rightarrow$ & diferenzacá	& `to differentiate'\\
		separace		& $\rightarrow$ & separacá		& `to separate'\\
	\end{tabu}
\end{table}
\par Some Latin loanwords are borrowed first from German. Loanwords ending in \textbf{-ieren} become \textbf{-irná}.
\begin{table}[h!]
	\centering \small
	\begin{tabu} to 0.9\textwidth{>{\bfseries}YM[0.3]>{\bfseries}YY}
		akzeptieren 	& $\rightarrow$ & akceptirná 	& `to accept' \\
		konservieren	& $\rightarrow$ & koncervirná	& `to conserve'\\
		produzieren		& $\rightarrow$ & producirná	& `to produce'\\
		vandalieren		& $\rightarrow$ & vandalirná 	& `to deface'\\
	\end{tabu}
\end{table}
\subsection{Internal Derivation}
\begin{center}
	\small
	\begin{longtabu}to \textwidth{Y[0.5]Y}
		
		\caption{Verbal Derivational Affixes}
		\label{verbalder}                             \\
		\toprule
		\multicolumn{1}{c}{\sc affix} & \multicolumn{1}{c}{\sc examples}                      \\
		\midrule
		\endfirsthead
		%---------------------------------------------------------------%
		\caption{Verbal derivational affixes \hfill\textit{(continued)}}            \\
		\toprule
		\multicolumn{1}{c}{\sc affix} & \multicolumn{1}{c}{\sc examples}                      \\
		\midrule
		\endhead
		%---------------------------------------------------------------%
		\bottomrule \addlinespace
		\multicolumn{2}{r}{\footnotesize\textit{continued on the next page}}
		\endfoot
		
		\bottomrule
		\endlastfoot
		
		\textbf{nie-} + {\sc adj}\newline`to cause something to become \mk{adj}' &
		
		\textbf{lo\v{s}} `new' $\rightarrow$ \textbf{nielo\v{s}á} `to renew' \newline
		\textbf{preseh} `young' $\rightarrow$ \textbf{niepreshá} `to rejuvenate' \newline
		\textbf{avic} `long' $\rightarrow$ \textbf{nieavicá} `to lengthen' \newline
		\textbf{gem} `soft' $\rightarrow$ \textbf{niegemá} `to soften'\newline
		\textbf{vyne} `dry' $\rightarrow$ \textbf{nievyneá} `to dry'\\ \addlinespace
		
		\textbf{ce-}\footnote{Verbs in \textbf{ce-} cannot be in the reflexive focus.} + {\sc adj}\newline `to cause oneself to become {\sc adj}' &
		
		\textbf{kdavidy} `clean' $\rightarrow$ \textbf{cekdavicá} `to take a bath' \newline
		\textbf{rum} `old' $\rightarrow$ \textbf{cerumá} `to grow old' \newline
		\textbf{\v{s}eznom} `big' $\rightarrow$ \textbf{ce\v{s}eznomá} `to grow up' \newline
		\textbf{vyne} `dry' $\rightarrow$ \textbf{cevyneá} `to dry oneself'\\ \addlinespace
		
		\textbf{hó-} + {\sc noun}\newline `to use {\sc n} in a particular way' &
		
		\textbf{tvem} `tongue' $\rightarrow$ \textbf{hótvemá} `to lick' \newline
		\textbf{kov} `hammer' $\rightarrow$ \textbf{hóková} `to hammer' \newline
		\textbf{\v{s}eznom} `big' $\rightarrow$ \textbf{ce\v{s}eznomá} `to grow up' \newline
		\textbf{vyne} `dry' $\rightarrow$ \textbf{cevyneá} `to dry oneself'\\ \addlinespace 
		
		\textbf{de\v{s}-} + {\sc noun}\newline `to act in the manner of {\sc n}  &
		
		\textbf{tvem} `tongue' $\rightarrow$ \textbf{hótvemá} `to lick' \newline
		\textbf{rum} `old' $\rightarrow$ \textbf{cerumá} `to grow old' \newline
		\textbf{\v{s}eznom} `big' $\rightarrow$ \textbf{ce\v{s}eznomá} `to grow up' \newline
		\textbf{vyne} `dry' $\rightarrow$ \textbf{cevyneá} `to dry oneself'\\ \addlinespace 
		
		\textbf{má-iv} + {\sc noun}\newline `to so something usually done in {\sc noun}'  &
		
		\textbf{mrc} `market' $\rightarrow$ \textbf{mámrcivá} `to shop' \newline
		\textbf{gnazsa} `school' $\rightarrow$ \textbf{mágnazsivá} `to study in'  \newline
		\textbf{\v{s}eznom} `big' $\rightarrow$ \textbf{ce\v{s}eznomá} `to grow up' \newline
		\textbf{vyne} `dry' $\rightarrow$ \textbf{cevyneá} `to dry oneself'\\ \addlinespace 
		
		
		\textbf{sen-/sem-} + {\sc verb}\newline `to {\sc verb} incorrectly'  &
		
		\textbf{o\v{s}tá} `to read' $\rightarrow$ \textbf{seno\v{s}tá} `to misread' \newline
		\textbf{rum} `old' $\rightarrow$ \textbf{cerumá} `to grow old' \newline
		\textbf{\v{s}eznom} `big' $\rightarrow$ \textbf{ce\v{s}eznomá} `to grow up' \newline
		\textbf{vyne} `dry' $\rightarrow$ \textbf{cevyneá} `to dry oneself'\\ \addlinespace 
	\end{longtabu}
\end{center}