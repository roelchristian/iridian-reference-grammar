\chapter{Verbs}
\section{Categories}
\par Finite verbs (\textbf{lounehlý}) are marked for the following grammatical categories:
\begin{enumerate}
	\item \textit{Aspect}. Iridian has three primary aspects: perfective, imperfective and contemplative; and two secondary ones: retrospective and prospective.
	\item \textit{Voice}. Iridian has a strong tendency to leave the topic of the sentence unmarked, instead encoding the primary information on the verb. Due to this, voice must be explicitly marked on the verb. Iridian has the following grammatical voices:: agentive, patientive, benefactive, instrumental, locative and reflexive.
	\item \textit{Mood}. Besides the unmarked indicative, Iridian has the following grammatical moods: subjunctive, conditional, hortative, optative, abilitative, permissive and non-volitive. In addition, secondary prefixes are used to express what would otherwise could be considered as moods: inceptive, causative and reciprocative.
\end{enumerate}

\par Verbs are also marked for person, although this is done by the addition of clitic pronouns and not through a separate conjugation paradigm. Iridian verbs are not marked for tense, gender, or number.
\par Iridian verbs have four classes of non-finite forms: the gerund, the converb, the supine and the generic nominal formed with \textbf{-ou}. The non-finite verb forms are derived from the uninflected verb stem except the generic nominal in \textbf{-ou} which can only be formed from a fully-inflected verb stem. A fifth class exists--the infinitive--but this form is largely defunct and is only used in certain compound constructions. Infinitives end in \textbf{-á} and is used as the citation form of a verb.

\section{Verb stems and citation forms}\index{citation form}

\par The citation form (or dictionary form) of a verb is the uninflected infinitive\index{infinitive}, a fossilized form rarely used outside of a very few periphrastic construction. The infinitive ends with the vowel \ird{-\'a}, and removing this ending will produce the verb stem. The final consonant (or in rare cases, vowel) of the stem determines the conjugation paradigm the verb follows.


\section{For delrtion}

\par Iridian is superficially an agglutinative language. However, agglutination only exists in its full form in the indicative mood and in a reduced form in the optative moods. The conditional, quotative and subjunctive has entirely dropped the agglutination, instead using fusional conjugation paradigm. These system are normally called Type I, Type II, and Type III paradigms respectively.
\par Nevertheless, Iridian verbs are analyzed to have eight primary affix slots. Five of these slots are for suffixes and are numbered from one to five counting from the stem. There are three prefix slots, used for both infixes and prefixes, labeled A to C starting from the stem. Table \ref{slots} shows the affix slots used for these three groups of grammatical moods.

\begin{table}[h!]
	\centering \small
	\caption{Verbal affix slots.}
	\begin{tabu} to \textwidth {M[0.3]YYY}
		\toprule
		{\sc slot} & \multicolumn{1}{c}{\sc indicative}&\multicolumn{1}{c}{\sc optative} &\multicolumn{1}{c}{\sc other}\\
		\midrule \addlinespace
		C & Negation&Negation&Negation\\  \addlinespace
		B & Secondary verb prefixes&--&--\\ \addlinespace
		A & Voice&--&-- \\ \addlinespace
		0 & Stem&Stem&Stem\\ \addlinespace
		1 & Secondary pronoun&--&Paradigm ending\\ \addlinespace
		2 & Mood&Voice&--\\ \addlinespace
		3 & Aspect&Aspect&--\\ \addlinespace
		4 & Primary pronoun&Primary Pronoun&--\\ \addlinespace
		5 & Non-finite ending&Mood&--\\ \addlinespace
		\bottomrule
			\label{slots}
	\end{tabu}
\end{table}

\subsection{Stem}
There are two types of verb stems in Iridian: verbal and nominal. Iridian does not have a clear distinction between nominals and verbals. In this sense, verbal stems refer to stems that can be used on their own as imperatives. On the other hand, nominal stems refer to stems that could be used on their own as nouns or adjectives.
\par Examples of verbal stems include \ird{pia\v{s}tá} `to eat', \ird{vyté} `to go', \ird{kravná} `to cry', \ird{cselé} `to leave', etc. Examples of nominal stems include \ird{pledy} `red'; \ird{aro} `water', etc.
\par Some stems contain an unstable vowel


\subsection{Type I paradigm}

\subsubsection{indicative mood}
\par The simplest construction other than the unmarked stem involves a prefix for voice, and suffixes for the modality, aspect and the primary pronoun. Slot 4 suffixes have a strong tendency to be dropped when evident from context. As such in the unmarked indicative mood, most verbs would have only the affixes for voice and aspect.

\ex \ird{povia\v{s}tak}, `(I) ate.'
\begin{center}
	\small
	\begin{tabu}to \textwidth{MMMM[1.5]MMMMM}
		\toprule
		C&B&A&{\sc stem}&1&2&3&4&5\\
		\midrule
		\addlinespace
		-&-&<ov>&pia\v{s}t&-&$\varnothing$&-ak&$\varnothing$&-\\
		\bottomrule
	\end{tabu}
\end{center}
\xe

\par Slot 1 is used for secondary clitic pronouns, i.e., the object of the verb. Secondary clitic pronouns are also often dropped, however, and the inanimate 3rd person pronouns are almost never used.
\ex \ird{voide\v{s}kem}, `I saw you.'
\begin{center}
	\small
	\begin{tabu}to \textwidth{MMMM[1.5]MMMMM}
		\toprule
		C&B&A&{\sc stem}&1&2&3&4&5\\
		\midrule
		\addlinespace
		-&-&<oiv>&vyd&e\v{s}&$\varnothing$&-ek&-em&-\\
		\bottomrule
	\end{tabu}
\end{center}

\xe

\par There is a separate conjugation paradigm for the negative, synthesized with slot 3 aspect suffixes. There are, however, some 60 irregular verbs that use the negative prefix \ird{ná-/nái-} instead of the negative aspectual suffixes, in addition to stative or copulative verbs, which can only be used with the prefix \ird{ná-/nái-}.

\pex 
\a \ird{voide\v{s}oitem}, `I didn't see you.'
\begin{center}
	\small
	\begin{tabu}to \textwidth{MMMM[1.5]MMMMM}
		\toprule
		C&B&A&{\sc stem}&1&2&3&4&5\\
		\midrule
		\addlinespace
		-&-&<oiv>&vyd&-e\v{s}&$\varnothing$&-oit-&-em&-\\
		\bottomrule
	\end{tabu}
\end{center}

\a \ird{nájemnoutalý}, `Your place is closed.'
\begin{center}
	\small
	\begin{tabu}to \textwidth{MMMM[1.5]MMMMM}
		\toprule
		C&B&A&{\sc stem}&1&2&3&4&5\\
		\midrule
		\addlinespace
		ná-&je-&-&mnout&-&$\varnothing$&-&-alý&-\\
		\bottomrule
	\end{tabu}
\end{center}

\xe

\subsubsection{imperative mood}\index{imperative mood}

\par The Iridian imperative mood has two forms: a singular and a plural. Unlike the indicative mood, there is no separate conjugation paradigm for the negative; instead the prefix \ird{z\'a} is used.

\pex 
\a \ird{pia\v{s}te}, `Eat!'
\begin{center}
	\small
	\begin{tabu}to \textwidth{MMMM[1.5]MMMMM}
		\toprule
		C&B&A&{\sc stem}&1&2&3&4&5\\
		\midrule
		\addlinespace
		-&-&$\varnothing$&pia\v{s}t&-&-o&-&-&-\\
		\bottomrule
	\end{tabu}
\end{center}

\a \ird{nápia\v{s}tet}, `Don't eat!'
\begin{center}
	\small
	\begin{tabu}to \textwidth{MMMM[1.5]MMMMM}
		\toprule
		C&B&A&{\sc stem}&1&2&3&4&5\\
		\midrule
		\addlinespace
		ná&-&$\varnothing$&pia\v{s}t&-&-et&-&-&-\\
		\bottomrule
	\end{tabu}
\end{center}
\xe

\subsubsection{copulative form}
\par Verbs have two copulative forms in Iridian. The copulative in \ird{je-} is a slot B prefix while the copulative in \ird{ut/uit} is a slot A infix. If not followed by he prefix \ird{je-} requires an epenthetic \ird{-o/-eu}, analyzed as a slot 5 suffix.

\ex \ird{jemo\v{z}lam}, `I am living.'

\begin{center}
	\small
	\begin{tabu}to \textwidth{MMMM[1.5]MMMMM}
		\toprule
		C&B&A&{\sc stem}&1&2&3&4&5\\
		\midrule
		\addlinespace
		-&je-&-&mo\v{z}l&-&-&-&-am&-\\
		\bottomrule
	\end{tabu}
\end{center}
\xe

\ex \ird{Marek jesorto}, `Marek is standing.'

\begin{center}
	\small
	\begin{tabu}to \textwidth{MMMM[1.5]MMMMM}
		\toprule
		C&B&A&{\sc stem}&1&2&3&4&5\\
		\midrule
		\addlinespace
		-&je-&-&sort&-&-&-&$\varnothing$&-o\\
		\bottomrule
	\end{tabu}
\end{center}
\xe

\section{Voice}\index{voice}

Iridian often prefers to encode information on the verb instead of through case marking on nouns. As such, all verbs must be explicitly marked for voice.
\begin{table}[h!]
	\small \centering
	\caption{Suffixes used to mark grammatical voice.}
	\begin{tabu} to 0.8\textwidth{YM}
		\toprule
		&{\sc ending}\\
		\midrule
		Agentive	& -a\v{s}-\\ \addlinespace
		Patientive	& -in-\\ \addlinespace
		Benefactive	& -\'eb-\\ \addlinespace
		Locative	& -á-\\ \addlinespace
		Instrumental& -\\ \addlinespace
		Reflexive	& -\\ \addlinespace
		Reciprocal	& \\ \addlinespace
		\bottomrule
	\end{tabu}
\end{table}


\subsection{Agentive voice}\index{agentive voice}
\par The agentive voice is used if the subject of the verb is the agent of the action.

\pex 
\begingl
\gla Sa pia\v{s}\v{c}ek.//
\glb already eat-\mk{av-pf}//
\glft `(I) already ate.'//
\endgl
\xe

The affix \ird{-a\v{s}-} assimilates to the consonant ending the root, with the vowel \bt{5} normally dropped, subject to the following rules:
\begin{itemize} 
	\item \v{c}: for roots ending with c, \v{c}, k, t
	\begin{itemize}
		\item jelc\'a + -a\v{s}- $\rightarrow$ jel\v{c}-, \trsl{to dance}
		\item zdiek\'a + -a\v{s}- $\rightarrow$ zd\'i\v{c}-, \trsl{to blow}
		\item pia\v{s}t\'a + -a\v{s}- $\rightarrow$ pia\v{s}\v{c}-, \trsl{to eat}
	\end{itemize}
	\item z: for roots ending with b, l, m, n, r\footnote{This change does not involve the deletion of the final consonant in the root.}
	\item \v{z}: for roots ending with d, g, z, \v{z}
	\begin{itemize}
		\item ba\v{z}- + -a\v{s}- $\rightarrow$ b\'a\v{z}-, \trsl{to give}
		\item stoj\'a + -a\v{s}- $\rightarrow$ st\'o\v{z}-, \trsl{to go}
	\end{itemize}
	\item \v{s}: for all other endings\footnote{\ird{-h + -a\v{s}-} , \ird{-s + -a\v{s}-} and \ird{-\v{s} + -a\v{s}-} both simplify to \ird{-\v{s}-}, while the rest retain the final consonant.}
\end{itemize}

Where the assimilation involves the deletion of the final consonant in the root, the preceding vowel is lengthened in compensation if the resulting root would then end in an open syllable.\index{compensatory lengthening}
\pex
\ird{Ud\'u\v{s}ek.}\\
(instead of \ird{*udu\v{s}ek})\\
\trsl{(I) took a shower.}
\xe
\pex
\ird{Pia\v{s}\v{c}ek.}\\
(not \ird{*pi\'a\v{s}\v{c}ek.})\\
\trsl{(I) ate.}
\xe

If the remnant vowel is the i-glide \ird{-ie-} or the diphthongs \ird{-ei-} and \ird{-ou-}, the remaining vowel would simplify to \ird{\'i}, \ird{\'i} and \ird{\'u}, respectively. Consider for example the verb \ird{zdiek\'a} \trsl{to blow}:

\pex
\begingl
\gla Lest zdi\v{c}al\'i.//
\glb wind blow-\mk{av-prog}//
\glft \trsl{The wind is blowing.}//
\endgl
\xe

Nevertheless the vowel \nt{5} in the root resurfaces in the following cases:

\begin{itemize}
	\item Verbs ending in -irn\'a:
	\item Verb root ending in a consonant cluster with a final liquid, nasal, or v
\end{itemize}

\subsection{Patientive focus}
\par A verb in the patient focus (glossed \mk{pat}) indicates that the topic of the sentence is the patient of the verb.

\pex
\begingl
\gla Marek vindekem.//
\glb Marek \mk{<pv>}see-\mk{pf-1s}//
\glft `I saw Marek.'//
\endgl
\xe


\subsection{Benefactive focus}\index{benefactive focus}
\par The benefactive focus (glossed \mk{ben}) is used when the subject of the sentence is the benefactor or director object of the verb. Verbs often change meaning when used in the benefactive focus.

\pex
\begingl
\gla Ma\v{c} sega nazd\'ebik.//
\glb mother flower-\mk{pat} buy-\mk{ben-pf}//
\glft `(I) bought my mother flowers.'//
\endgl
\xe

\pex
\begingl
\gla Kova pia\v{s}t\'ebal\'i.//
\glb cow eat-\mk{ben-prog}//
\glft \trsl{(I am) feeding the cows.}//
\endgl
\xe

The benefactive is also used idiomatically with verbs of judgment including \ird{noviet\'a} \trsl{to like}

\pex
\begingl
\gla D\'a \v{c}eh\'ov\'am z\'anov\'it\'eb\'al.//
\glb \mk{1s} sports-\mk{agt} \mk{neg}-like-\mk{ben-prog}//
\glft \trsl{I don't like sports.}//
\endgl
\xe

\subsection{Locative Focus}

\pex
\begingl
\gla J\'e kopna\v{z}al\'ic.//
\glb you laugh-\mk{loc-prog-3s.anim}//
\glft \trsl{He is laughing at you.}//
\endgl
\xe

\subsection{Instrumental Focus}


\subsection{Reflexive Voice}

The reflexive voice (glossed \mk{ref}) is used when the patient of the verb is also the agent of the action. Morphogically, the reflexive voice is not a separate voice but is derived from the agentive form of the verb and the addition of the prefix \ird{u(d)-}.

\pex
\begingl
\gla Na \v{s}arta uvi\v{z}kem.//
\glb \mk{loc} mirror-\mk{pat} \mk{ref}-see-\mk{av-pf-1s}//
\glft \trsl{I saw myself in the mirror.}//
\endgl
\xe

The use of the reflexive voice is more extensive in Iridian than in English, and is somehow similar to how the reflexive construction is used in Romance languages.

\pex
\begingl
\gla U\v{s}ti\v{z}ek.//
\glb \mk{ref}-take:a:bath-\mk{av-pf}//
\glft \trsl{(I) took a bath.}//
\endgl
\xe

\pex
\begingl
\gla Um\'u\v{s}al\'i.//
\glb \mk{ref}-comb-\mk{av-prog}//
\glft \trsl{(I) am combing my hair.}//
\endgl
\xe

Below is a non-exhaustive list of verbs that are normally used in the reflexive voice:
\bigskip

\noindent 
\ird{du\v{s}\'a} \trsl{to take a shower}\\
\ird{mu\v{s}\'a} \trsl{to comb}\\
\ird{\v{s}a\v{s}t\'a} \trsl{to sit down}\\

Some verbs may change meaning when used in the reflexive voice.


The reflexive voice is also used to imply that an action happened accidentally or involuntary or that the agent of the action is unknown or unimportant.

The reflexive voice may also be used emphatically, especially in spoken Iridian, to express that the action has been performed for the benefit of the actor/agent of the verb.

\pex
\begingl
\gla K\'av\'ea u\v{s}ranz\k{a}cem.//
\glb coffee-\mk{pat} \mk{ref}-drink-\mk{av-ctplv-1s}//
\glft \trsl{I'll drink coffee.} (literally, I'll drink myself coffee)//
\endgl
\xe

\pex
\begingl
\gla Pul\v{s}a uvo\v{s}\v{c}ek.//
\glb soup-\mk{pat} \mk{ref-}cook\mk{-av-pf}//
\glft \trsl{(I) cooked (me) some soup.}//
\endgl
\xe

\subsection{Usage}

\par The differences 

\section{Grammatical Aspect}
\begin{table}[h!]
	\centering
	\caption{Aspect markers in the indicative mood.}
	\begin{tabu} to 0.8\textwidth{MM}
		\toprule
		{\sc aspect}	& {\sc affix}\\
		\midrule
		Perfective		& -ek\\
		Retrospective	& -an\'i\\
		Imperfective	& -\'al\\
		Progressive		& -al\'i \\
		Contemplative	& -\k{a}c\\
		Prospective		& -il\\
		Cessative		& -eic\\
		\bottomrule
	\end{tabu}

\end{table}
\subsection{Perfective aspect}
The perfective aspect (glossed {\sc pf}) indicates an action that has been completed in some specific instance.

\pex
\begingl
\gla Bych na gna\v{z}a Marek vdinek.//
\glb yesterday \mk{loc} school-\mk{pat} Marek see-\mk{pv-pf}//
\glft \trsl{(I) saw Marek at school yesterday.}//
\endgl
\xe

\pex
\begingl
\gla Va\v{s}ko pia\v{s}tnek.//
\glb pastry eat-\mk{pv-pf}//
\glft `(I) ate (the) cake.'//
\endgl
\xe

\par The vowel in the suffix is unstable and the ending would normally collapse to \textbf{-k} when followed by another vowel. Consider the above two sentences followed by the second person singular clitic pronoun \textbf{-a\v{s}/e\v{s}}.

\pex
\begingl
\gla Bych na gnazsa Marek vindeke\v{s}.//
\glb yesterday \mk{loc} school-\mk{pat} Marek \mk{<pv>}see-\mk{pv-pf-2s}//
\glft `You saw Marek at school yesterday.'//
\endgl
\xe

\pex
\begingl
\gla Va\v{s}ko pinia\v{s}tka\v{s}.//
\glb pastry \mk{<pv>}eat-\mk{pf-2s}//
\glft `You ate (the) cake.'//
\endgl
\xe


\par When negated, the perfective indicates something that ought to be done but had not been done. To state that something simply did not happen, the negative of the retrospective is used instead.

\pex
\begingl
\gla Z\'at\'el\'evonirna\v{s}ek.//
\glb \mk{neg}-telephone-\mk{av-pf}//
\glft `(I) failed to call.' //
\endgl
\xe

\pex
\begingl
\gla Z\'at\'el\'evonirna\v{s}an\'i.//
\glb \mk{neg}-telephone-\mk{av-ret}//
\glft `(I) didn't call.' //
\endgl
\xe

\subsection{Retrospective aspect}
\par The retrospective aspect (glossed \mk{ret}) is used for a past action that has a continuing relevance in the presence. Consider, for example, the following sentences: (a) \textit{I went to Amsterdam last week}; and (b) \textit{I have been to France in my childhood}. Iridian would translate the verb in (a) using the perfective and the verb in (b) using the retrospective.

\pex<ret-pres1>
\begingl
\gla Hroná tímu na Budape\v{s}ta mo\v{z}la\v{s}an\'im.//
\glb three year-\mk{inst} \mk{loc} Budapest-\mk{pat} live-\mk{av-ret-1s}//
\glft `I have been living in Budapest for three years.'//
\endgl
\xe

\pex<ret-pres>
\begingl
\gla Páku \v{s}avolnan\'ic.//
\glb before-\mk{inst} hurt-\mk{pv-pf-3s.anim}//
\glft `She has been hurt before.' //
\endgl
\xe

\par The retrospective is also often used to imply non-volition or the  accidental/circumstantial nature of an action. Similarly the retrospective is used with verbs of emotion or state (e.g., \ird{cezu\v{s}talá}, ‘to become happy’ from \ird{zu\v{s}tal} ‘happy’). The perfective, on the other hand, is almost exclusively used with the causative in these cases.

\pex
\a	\begingl
\gla Vde\v{s}ek \v{s}e neicezu\v{s}tala\v{s}an\'im.//
\glb see-\mk{2s-pf} with \mk{incep}-be.happy-\mk{av-ret-1s}//
\glft `I became happy when I saw you.' //
\endgl
\a	\begingl
\gla Do pacezu\v{s}talnike\v{s}.//
\glb \mk{1s.wk} \mk{caus}-be.happy-\mk{pv-pf-2s}//
\glft `You made me happy.' //
\endgl
\xe
\pex<vasebroke>
\begingl
\gla Váz noprizan\'i.//
\glb vase break-\mk{ref-ret}//
\glft `The vase broke (accidentally).' //
\endgl
\xe

\subsection{Continuous and progressive aspects}
Iridian uses the continuous and progressive aspects to denote actions that have not been completed yet and/or are in the process of happening/occuring. The continuous aspect (glossed \mk{cont}) is used to mark a state of being while the progressive aspect (glossed \mk{prog}) is used to mark a dynamic activity.
\pex
\begingl
\gla N\'au uri\v{s}tn\'al.//
\glb clothes \mk{ref-}wear-\mk{pv-cont}//
\glft \trsl{(I'm) wearing clothes.} //
\endgl
\xe

\pex
\begingl
\gla N\'au uri\v{s}tnal\'i.//
\glb clothes \mk{ref-}wear-\mk{pv-prog}//
\glft \trsl{(I'm) putting on clothes.} //
\endgl
\xe

The continuous aspect is also used to denote a habitual action.

\pex
\begingl
\gla Sholu de gna\v{z}a sto\v{z}\'al.//
\glb daily-\mk{inst} \mk{ill} school-\mk{pat} go-\mk{av-cont}//
\glft \trsl{(We) go to school everyday.} //
\endgl
\xe

\pex
\begingl
\gla D\'a na Praha mo\v{z}l\'al.//
\glb \mk{1s.str} \mk{loc} Prague-\mk{pat} live-\mk{cont}//
\glft \trsl{I live in Prague.} //
\endgl
\xe

To emphasize the habitual nature of an action, a nominalized construction is often used.

\pex
\begingl
\gla Na\v{z}em r\k{a}cen\'alou.//
\glb friend-\mk{1s} smoke-\mk{cont-nz}//
\glft \trsl{My friend is a smoker.} //
\endgl
\xe

\subsection{Prospective aspect}
\par The prospective aspect (glossed {\sc prosp}) is primarily used in secondary clauses to indicate actions that are about to be started in relation to another action. It can also be used in the main clause to indicate an action in the immediate future.

\subsection{Cessative aspect}




\section{Secondary Verbal Prefixes}
In addition to the prefixes used for verbal derivation, Iridian has three prefixes that are analyzed as separate moods.
\subsection{The reciprocative so-}

\section{Grammatical Mood}

\subsection{Indicative}

\subsection{Imperative}\index{imperative mood}
The imperative mood has three forms: the singular, formed with the suffix \ird{-e}; the plural, formed with the suffix \ird{\'et}; and the adhortative\index{adhortative}, formed with the suffix \ird{\-i\v{c}e}. The imperative suffix is added directly to the root of the verb as commands are understood implicitly to be in the agentive voice.

\ex
\ird{jelc\'a} \trsl{to dance}\\
\ird{Jelce.}	\trsl{Dance!}\\
\ird{Jelc\'et.} \trsl{Dance! \mk{(pl)}}\\
\ird{Jelci\v{c}e.} \trsl{Let's dance.}
\xe

\ex
\ird{virk\'a} \trsl{to write}\\
\ird{To n\'umer virkne.}	\trsl{Write this number down.}\\
\ird{To n\'umer virkn\'et.} \trsl{Write \mk{(pl)} this number down.}\\
\ird{To n\'umer virkni\v{c}e.} \trsl{Let's write this number down.}
\xe

\medskip

In more formal settings, the imperative may be considered rude or impolite, and speakers would often opt to use the hortative mood\index{hortative} instead when issuing commands. Nonetheless, the imperative is commonly found in the written language. 

\pex
\begingl
\gla Zátiezna\v{s}e.//
\glb \mk{neg}-kill-\mk{agt-imp}//
\glft \trsl{Thou shalt not kill.}//
\endgl
\xe


To negate the imperative, the prefix \ird{z\'a} is used, as can be seen in the example above.

In informal and familiar settings, a version of the imperative is used instead of the hortative which might appear too formal. This version uses the particle \ird{je} (originally a word meaning \trsl{already} but now grammaticalized) as a clitic to `soften' the imperative.

\pex
\begingl
\gla J\'an ba\v{z}ne-je.//
\glb that give-\mk{pv-imp=expl} //
\glft \trsl{Give that to me.}//
\endgl
\xe

\subsection{Subjunctive}

The subjunctive mood (glossed \mk{sbj}) is used for actions or events that are not or are not known to be true or factual. The subjunctive is formed using the suffix \ird{-\'il}

\begin{table}[h!]
	\centering\small
	\caption{Conjugation of the verb \ird{pia\v{s}t\'a} in the subjunctive.}
	\begin{tabularx}{0.7\textwidth}{YYY}
		\toprule
					&\multicolumn{1}{c}{\sc imperfective}&\multicolumn{1}{c}{\sc perfective}\\
		\midrule
		Agentive	& pia\v{s}\v{c}\'ila	& pia\v{s}\v{c}\'il\\
		Patientive	& pia\v{s}tn\'ila		& pia\v{s}tn\'il\\
		Benefactive	& pia\v{s}teb\'ila		& pia\v{s}teb\'il\\
		Locative	& pia\v{s}toun\'ila		& pia\v{s}toun\'il\\
		Instrumental& dopia\v{s}teb\'ila	& dopia\v{s}teb\'il\\
		Reflexive	& upia\v{s}\v{c}\'ila	& upia\v{s}\v{c}\'il\\
		\bottomrule		
	\end{tabularx}
\end{table}

In addition, the copula has two subjunctive forms, the non-negative \ird{niec} and the negative \ird{va\v{s}e}.

Note that the Iridian subjunctive makes neither temporal nor aspectual distinction.

\par The following are some specific uses of the subjunctive mood in Iridian:
\subsubsection{jussive/desiderative}
\par The subjunctive is used in indirect constructions of verbs for issuing orders, commanding, exhorting, etc.
\pex
\begingl
\gla Martin na America \v{z}no\v{z}\'il to \v{c}ezna\v{s}\'alic.//
\glb Martin \mk{loc} America-\mk{pat} study-\mk{av-sbj} \mk{rz} want-\mk{av-cont-3s.anim}//
\glft `He wants Martin to study in America.'//
\endgl
\xe

\pex
\begingl
\gla Beatles-\v{z}e >>Yesterday<< Mark\k{a} z\'a\v{s}n\'il to Tunek dálek.//
\glb Beatles-\mk{gen} ``Yesterday'' Marek-\mk{agt} sing-\mk{sbj} \mk{rz} Tunek say-\mk{pf}//
\glft `Tunek told Marek to sing.'//
\endgl
\xe

\subsubsection{dubitative}
\par The subjunctive is used with verbs expressing doubt, uncertainty or disbelief.

\pex
\begingl
\gla \v{s}e //
\glb Beatles-\mk{gen} ``Yesterday'' Marek-\mk{agt} sing-\mk{sbj} \mk{rz} Tunek say-\mk{pf}//
\glft `Tunek told Marek to sing.'//
\endgl
\xe

\subsubsection{with verbs expressing emotion}

\pex
\begingl
\gla Marek za\v{s}n\'il to Tunek dálek.//
\glb Marek sing-\mk{sbj.ipf} \mk{rz} Tunek say-\mk{pf}//
\glft `Tunek told Marek to sing.'//
\endgl
\xe


\subsubsection{with the conditional mood}
\par The subjunctive is used in the main clause if the verb in the dependent clause is in the conditional \textit{irrealis} mood.

\pex
\begingl
\gla Dá prezident jenem, //
\glb a//
\glft a//
\endgl
\xe

\subsubsection{expressing judgment}

\pex
\begingl
\gla Zavno\v{c}ila\v{s} to t\'ev\'et //
\glb respond-\mk{av-sbj.ipf-2s} \mk{rz} important//
\glft \trsl{It is important that you respond.}//
\endgl
\xe

\subsubsection{irrealis}

\subsection{Conditional}
\par The conditional mood is used for conditional or hypothetical clauses. The table below shows the conjugation paradigm for the conditional mood for both regular verbs and the copula. The Iridian conditional mood is not a true conditional mood grammatically, since it is marked on the verb in the dependent clause (protasis), instead of the main clause.

\begin{table}[h!]
	\centering \small
	\caption{Conjugation paradigm, conditional mood.}
	\begin{tabu} to 0.9 \textwidth	{Y[1.3]MM}
		\toprule
		&{\scshape regular verbs} & {\scshape copula}\\
		\midrule
		
		\textit{Realis} &-ouhn\'a &viec\\
		Neg. \textit{Realis}&-ouhn\'al&ven\\
		
		Non-Past \textit{Irrealis} & -ouc & jenouc\\
		Neg. Non-Past \textit{Irrealis} & -oucik & pi\k{e}c\\
		
		Past \textit{Irrealis} & -\'ane & jenem\\
		Neg. Past \textit{Irrealis} & -oucn\'a & jet\\
		\bottomrule
	\end{tabu}
\end{table}

\subsubsection{conditional realis}

\par The conditional \textit{realis} mood (glossed \mk{cond.rl}) is used in two ways:
\begin{enumerate}
	\item In sentences that express a factual implication rather than a hypothetical situation or a potential future event, e.g., `If you heat water to 100 C, it will boil.'
	\item In `predictive' constructions, i.e., those that concern probable future events.
\end{enumerate} 

\subsubsection{conditional irrealis}
The conditional \textit{irrealis} mood (glossed \mk{cond.irr}) is used with hypothetical, typically counterfactual, events. Iridian distinguishes between past and non-past \textit{irrealis} moods.


\subsection{Hortative}
\par The hortative mood is used for requests. Although Iridian has an imperative form (the unmarked form of the verb), the hortative is normally used in its place. The hortative marker should always appear at the end of the word.

	\pex
\begingl
\gla Jê\v{s}a mine\v{s}ka.//
\glb door.\mk{pat} close-\mk{2s-hort}//
\glft 'Close the door.' \textit{literally,} `May you close the door.'//
\endgl
\xe

\par To soften a command, the expression \textit{am luhninka} (may someone be thanked for\ldots) is normally used.

\pex
\begingl
\gla Jê\v{s}a minke\v{s} ce\v{s} am luhninka.//
\glb door-\mk{pat} close-\mk{pf-2s} \mk{rz.abl} because thank\mk{-pv-hort}//
\glft  `Please close the door.' \textit{literally,} `May (you) be thanked because you closed the door.'//
\endgl
\xe

\par The hortative is used with the reciprocative prefix \textbf{so-} to form the adhortative (similar to the English construction with `Let's + \mk{verb}). This construction cannot be used with \textbf{am luhninka}.

\pex
\begingl
\gla sop//
\glb door-\mk{pat} close-\mk{pf-2s} \mk{rz.abl} because thank\mk{-pv-hort}//
\glft  `Please close the door.' \textit{literally,} `May (you) be thanked because you closed the door.'//
\endgl
\xe

\subsection{Optative}
The optative mood (glossed \mk{opt}) is used for expressing wishes. The optative mood requires two aspect marking, although the primary ending is marked if it is in the imperfective mood.



\subsection{Quotative }	
\par The quotative mood (glossed \mk{quot}) is used to express secondhand information, or when the speaker wishes to make explicit that s/he did not witness the event himself/herself.
\par Clitic pronouns cannot be used with the quotative mood.
\par Table \ref{conj-quot} shows the conjugation paradigm for regular verbs and the copula.


\begin{table}[h!]
	\centering \footnotesize
	\caption{Conjugation paradigm, quotative mood}
	\begin{tabu} to \textwidth{Y[1.3]YY[0.8]}
		\toprule
		&\multicolumn{1}{c}{\textbf{pia\v{s}tá}, `to eat'}& \multicolumn{1}{c}{\sc copula}\\
		\midrule
		Perfective & pia\v{s}tát & vacet\\
		Neg. perfective & nápia\v{s}tát & necê\\
		Retrospective & pia\v{s}tác & ---\\
		Neg. Retrospective & nápia\v{s}tác&---\\
		Imperfective & pia\v{s}tút & ne\v{s}kec \\
		Neg. imperfective & nápia\v{s}tút & po\v{s}nec\\
		Progressive & pia\v{s}tiec ne\v{s}kec&---\\
		Neg. progressive & pia\v{s}tiec po\v{s}nec&---\\
		Future & pia\v{s}tô\v{s} & vacko\\
		Neg. Future & nápia\v{s}tô\v{s} & necko\\
		Subjunctive Non-Past& pia\v{s}tok &necim\\
		Neg. Sub. Non-Pas& nápia\v{s}tok & pocim\\
		Subjunctive Past & pia\v{s}tocke & vacim\\
		Neg. Sub. Past & nápia\v{s}tocke & nêcim\\
		\bottomrule
			\label{conj-quot}
	\end{tabu}

\end{table}


\pex
\begingl
\gla Já na duma ne\v{s}kec to maty dálmek.//
\glb you-\mk{str} \mk{loc} house-\mk{pat} \mk{cop.quot.ipf} \mk{rz} mother say-\mk{1s.pf}//
\glft `(My) mother told me you are at home.'//
\endgl
\xe

\pex
\begingl
\gla Já na duma necim to maty dálmek.//
\glb you-\mk{str} \mk{loc} house-\mk{pat} \mk{cop.quot.sbj.npst} \mk{rz} mother say-\mk{1s.pf}//
\glft `(My) mother told me you might be at home.'//
\endgl
\xe

\pex
\begingl
\gla Mnúcs tiezninát.//
\glb husband kill-\mk{pv-quot.pf}//
\glft `(She) killed (her) husband (or so I heard).'//
\endgl
\xe

\par Direct speech, however, does not use the subjunctive.
\pex
\begingl
\gla ---Tak dá, dálek Tomá\v{s}.//
\glb here \mk{1s.str} say-\mk{pf} Tomá\v{s}//
\glft ```I'm here,'' Tomá\v{s} said.'//
\endgl
\xe


\par The following verbs are considered verba dicendi in Iridian and would trigger the quotative: \textbf{dálá} `to say', \textbf{vadá} `to think', \textbf{kvu\v{s}tá} `to hear', \textbf{vydá} `to see', \textbf{ége\v{s}á} `to ask', \textbf{ohletá} `to remember', \textbf{hová} `to recount, tell a story' . The verb \textbf{vadá} is exclusively used with the subjunctive quotative.

\pex
\begingl
\gla Z \v{s}to óké necim to Luká\v{s} vadê.//
\glb already this OK \mk{cop.quot.sbj.npst} \mk{rz} Luká\v{s} think-\mk{ipf}//
\glft `Luká\v{s} thinks it should be OK by now.'//
\endgl
\xe

\pex
\begingl
\gla Marek bych jsenát to kvu\v{s}tkem.//
\glb Marek yesterday arrive-\mk{quot.pf} \mk{rz} hear-\mk{pf-1s}//
\glft `I heard Marek has arrived.'//
\endgl
\xe


\pex
\begingl
\gla Po\v{s}nelý tajomstác to kvu\v{s}tek.//
\glb father-\mk{2pl} die-\mk{quot.ret} \mk{rz} hear-\mk{pf}//
\glft `(We) heard that your father died.'//
\endgl
\xe

\pex
\begingl
\gla Dá tak bych vacim to náohletê.//
\glb \mk{1s.str} here yesterday \mk{cop.quot.sbj.pst} \mk{rz} \mk{neg}-remember-\mk{ipf}//
\glft `(I) don't remember if I was here yesterday.'//
\endgl
\xe

\par Secondary verba dicendi are formed with an adverbial construction using the imperfective converb in \textbf{-iec}.

\pex
\begingl
\gla Já mnou necim to Martin priviec vadê.//
\glb you correct \mk{cop.quot.sbj.npst} \mk{rz} Martin agree-\mk{cv} think-\mk{ipf}//
\glft `Martin agrees that you are right.'//
\endgl
\xe

\par The quotative is also used emphatically to repeat a quote (often made by the speaker himself or herself), or to express the speaker's frustration or affirmation. When used this way, the verbum dicendi is omitted, and the expletive \textbf{nó} is often added.

\pex
\begingl
\gla Mnou necim to nó!//
\glb correct \mk{cop.quot.sbj.npst} \mk{rz} \mk{expl}//
\glft `(I've been telling you) it is right.'//
\endgl
\xe

\pex
\begingl
\gla Dá roctymút to!//
\glb \mk{1s} dance-\mk{abl-quot.ipf} \mk{rz}//
\glft `(But) I can dance.'//
\endgl
\xe

\par The tense/aspect of the quotative mood follows that of the quoted clause, independent of the tense/aspect of the verbum dicendi.



\subsection{Abilitative and Permissive}
\par The abilitative (glossed \mk{abl}) and permissive (glossed \mk{perm}) are related verbal moods used in expressing the speaker's (or the subject of the sentence's) abililty to do something. The abilitative is used to indicate capability while the permissive is used to indicate whether or not an action is allowed or permitted.

\pex
\begingl
\gla Sa anglecnu nározshovymas.//
\glb \mk{inst} English.language-\mk{inst} \mk{neg}-speak-\mk{abl-3s.anim}//
\glft `He cannot speak English.'//
\endgl
\xe

\pex
\begingl
\gla De rádaka z názahranave\v{s}.//
\glb \mk{ill} building-\mk{1pl.excl-pat} already \mk{neg}-enter-\mk{perm-2s}//
\glft `You're no longer allowed to enter our building.'//
\endgl
\xe

\par The permissive mood is often used for negative commands.

\pex
\begingl
\gla Tak náradzavuj.//
\glb here \mk{neg}-smoke-\mk{perm-4gen}//
\glft `No smoking.' \textit{literally,} `One cannot smoke here.'//
\endgl
\xe

\subsection{Non-Volitive}
\par The non-voliti
\ex[exno={\getref{vasebroke}, rep.}]
ago
\xe
\section{Non-Finite Verb Forms}

\subsection{Gerund}
\par The gerund (glossed \mk{ger}) refers to the non-finite verb form used as a noun. The gerundive prefix \ird{po-} is always used with the nominalizing suffix \ird{-ou}, both of which are added to the uninflected verb root.

\pex
\a
\begingl
\gla \v{S}\v{c}enek.//
\glb forget-\mk{pf}//
\glft \trsl{He forgot (it).}//
\endgl
\a
\begingl
\gla \v{S}\v{c}enekou Jan.//
\glb forget-\mk{pf-nz} Jan//
\glft \trsl{Jan (is) the one who forgot (it).}//
\endgl
\a
\begingl
\gla Po\v{s}\v{c}enou nauhl\'y.//
\glb \mk{ger}-forget-\mk{nz} difficult//
\glft \trsl{Forgetting is difficult.}//
\endgl
\xe

When nominalizing complex clauses, both the agent and the theme are marked in the genitive, with the agent aways appearing first.

\pex
\a
\begingl
\gla P\'a\v{s}ta Jan\k{a} vo\v{s}tnek.//
\glb pasta Jan-\mk{agt} cook-\mk{pv-pf}//
\glft \trsl{Jan cooked (some) pasta.}//
\endgl
\a
\begingl
\gla Jan\'i p\'a\v{s}t\'i povo\v{s}tou//
\glb Jan-\mk{gen} pasta-\mk{gen} \mk{ger-}cook-\mk{nz}//
\glft \trsl{Jan's cooking of the pasta}//
\endgl
\xe

The suffix \ird{-\'al}, used to mark the continuous aspect, may be infixed to the gerund to indicate that the action is repetitive.

\pex
\a
\begingl
\gla Jan nidek.//
\glb Jan stand.up-\mk{pf}//
\glft \trsl{Jan stood up.}//
\endgl
\a
\begingl
\gla Jan\'i ponid\'alou buvec.//
\glb Jan-\mk{gen} \mk{ger}-stand.up-\mk{cont-nz} annoying//
\glft \trsl{Jan's standing up again and again is annoying.}//
\endgl
\xe

\subsection{Converbs}
Converbs (glossed \mk{cv}) is a non-finite verb form often used for adverbial constructions. There are two converb forms in Iridian: the imperfective \textbf{-iec} (glossed \mk{cv.ipf}) and the perfective \textbf{-iêce} (glossed \mk{cv.pf}).

\pex
\begingl
\gla Tereza kravniec nóveu cselek. //
\glb Tereza cry-\mk{cv.ipf} room-\mk{abl} leave-\mk{pf}//
\glft `Tereza left the room crying.'//
\endgl
\xe

\pex
\begingl
\gla Nóveu cseliêce Tereza ukravnek.//
\glb room-\mk{abl} leave-\mk{cv.pf} Tereza \mk{incho}-cry-\mk{pf}//
\glft `Having left the room, Tereza started to cry.'//
\endgl
\xe

The perfective \textit{-iêce} is often used in clause linking.

\pex
\begingl
\gla O\v{s}tiêce krazkem.//
\glb read-\mk{cv.pf} understand-\mk{pf-1s}//
\glft `I read and understood.'//
\endgl
\xe

Clauses expressing reason is usually expressed by a converbial construction.

\pex
\begingl
\gla Za eksama názhaziêce, Martin órek.//
\glb for exam-\mk{pat} \mk{neg}-study-\mk{cv.pf} Martin fail-\mk{pf}//
\glft `Martin failed the exam because he didn't study.'//
\endgl
\xe


\subsection{Nominalization}



\subsection{Supine}
The supine is a non-finite verb form formed used to indicate necessity or purpose. There are four forms as shown below:

\begin{table}[h!]
	\centering\small
	\caption{Endings used for the supine}
	\begin{tabularx}{0.8\textwidth}{MMM}
		\toprule
		&{\sc supine of purpose}&{\sc supine of necessity}\\
		\midrule
		Nominal & \textit{-ity} & \textit{-á\v{s}}\\
		\addlinespace
		Non-nominal & \textit{-ice} & \textit{-á\v{s}ce}\\
		\bottomrule
	\end{tabularx}
\end{table}


	\pex
\begingl
\gla >>Ána Karenina<< za gnazsa o\v{s}tá\v{s}ce ko hto\v{s}.//
\glb Anna Karenina for school-\mk{pat} read-\mk{sup} \mk{att} book//
\glft `I have to read \textit{Anna Karenina} for school.'//
\endgl
\xe

	\pex
\begingl
\gla Hto\v{s} vstuninkem to o\v{s}tice.//
\glb book buy-mk{pv-pf-1s} \mk{rz} read-\mk{sup}//
\glft `I bought the book to read.'//
\endgl
\xe

\par The infinitive form of the supine of purpose \textit{-icá} is used with adjectival adverbs:

\pex
\begingl
\gla Just zacep\v{s}csemem to nosiênicá.//
\glb news \mk{caus}-be.sad-\mk{1s} \mk{rz} hear-\mk{sup.inf}//
\glft `I am sad to hear the news.'//
\endgl
\xe

\section{Copular Constructions}
\subsection{Null copula}

Copular sentences are a minor sentence type where the predicate is not a verb. For the purposes of this grammar, we narrow down our definition of copular constructions to the following:
\pex
\a \textit{Equative:} Marek is the doctor (we are talking about).
\a \textit{Inclusive:} Marek is a doctor.
\a \textit{Attributive:} Marek is tall.
\a \textit{Locative:} Marek is in the hospital.
\xe

Iridian does not make a distinction between equative, inclusive and attributive clauses. Locative clauses on the other hand, may be expressed using a copular or an existential construction, as will be discussed in this section.

Iridian is a superficially a zero-copula language and the most common way to form copular sentences is mere juxtaposition.

\pex<cop>
\begingl
\gla Marek doktor.//
\glb Marek doctor//
\glft \eng{Marek (is a/the) doctor.}//
\endgl
\xe

The above example could either be taken to mean (1) Marek is a doctor (inclusive), or (2) Marek is the doctor (equative). Generally, though, Iridian uses word order to distinguish between equative and inclusive clauses.

\pex
\a \textit{Inclusive:} \{item in class\}\tss{N} $\varnothing$ \{class\}\tss{P}
\a \textit{Equative:} \{class\}\tss{N} $\varnothing$ \{item class\}\tss{P}
\xe

To avoid ambiguity, Example \getref{cop} can be reformulated to either of the following sentences:

\pex<cop1>
\a 
\begingl
\gla Marek doktor.//
\glb Marek doctor//
\glft \eng{Marek is a doctor.}//
\endgl

\a 
\begingl
\gla Doktor Marek.//
\glb doctor Marek//
\glft \eng{Marek is the doctor.}//
\endgl

\xe

The inversion of word order is not strongly grammaticalized with NP-NP sentences, i.e., both sentences in Example \getref{cop1} can still be used interchangeably without a change in meaning and preference is given on the one over the other when there is an ambiguity. This is not the case with attributive clauses, i.e., sentences with adjective or adjective phrase predicates. Consider for example the sentence below:

\pex
\begingl
\gla Marek rázym.//
\glb Marek tall//
\glft \eng{Marek is tall.}//
\endgl
\xe

Inverting the word order of the sentence above would change the adjective to a substantive since modifiers cannot occupy the topic position.

\pex
\begingl
\gla Rázym Marek.//
\glb tall Marek//
\glft \eng{The tall one is Marek.}//
\endgl
\xe

Iridian also distinguishes between attributive clauses expressing permanent conditions and clauses expressing temporary conditions, with the latter being expressed using existential constructions in certain adjectives.

\pex
\begingl
\gla *Marek morec.//
\glb Marek hungry//
\glft \eng{Marek is hungry}//
\endgl
\xe


\pex
\begingl
\gla Marka je\v{s} morec.//
\glb Marek-\mk{pat} \mk{exst} hunger//
\glft \eng{Marek is hungry}//
\endgl
\xe

A full list of adjectives/modifiers that use the existential construction can be found in the section~\ref{sec:exst}.

The copula, however, cannot be ommitted in grammatical moods other than the indicative.

\subsection{Negative copula}

Iridian has the negative copula \ird{\v{c}esná}.

\pex
\begingl
\gla Marek doktor \v{c}esná.//
\glb Marek doctor \mk{cop.neg}//
\glft \eng{Marek is not (a/the) doctor.}//
\endgl
\xe

\par The inversion of word order may also be used when one wants to avoid ambiguity:

\pex
\begingl
\gla Doktor Marek \v{c}esná.//
\glb doctor Marek \mk{cop.neg}//
\glft \eng{Marek is not the doctor.}//
\endgl
\xe

\section{Existential Constructions}
\label{sec:exst}
An existential sentence is a specialized construction used to express the existence or presence of someone or something. The particle \ird{je\v{s}} and its inverse \ird{niho} are used to form existential sentences.

\pex
\begingl
\gla To je\v{s} zarno.//
\glb here \mk{exst} people//
\glft \eng{There are people here.}//
\endgl
\xe

\pex
\begingl
\gla To niho zarno.//
\glb here \mk{exst.neg} people//
\glft \eng{There is no one here.}//
\endgl
\xe

Statements expressing location use a copular construction, although an existential construction is used in the negative.

\pex
\begingl
\gla D\'a na duma.//
\glb \mk{1s.str} \mk{loc} house-\mk{pat}//
\glft \eng{I'm at home.}//
\endgl
\xe

\pex
\begingl
\gla Na duma niho d\'a.//
\glb \mk{loc} house-\mk{pat} \mk{exst.neg} \mk{1s.str}//
\glft \eng{I'm not at home.}//
\endgl
\xe

The particles \ird{je\v{s}} and \ird{niho} must always precede the noun whose presence or existence is being expressed.

\pex
\begingl
\gla Na r\'anema ona je\v{s} hto\v{s}.//
\glb \mk{loc} desk-\mk{1s-pat} one \mk{exst} book//
\glft \eng{There is one book on my desk.}//
\endgl
\xe

\pex
\begingl
\gla M\"y je\v{s} mula\v{z}.//
\glb two \mk{exst} door//
\glft \eng{There are two doors.}//
\endgl
\xe

\subsection{Conjugation paradigm}


\subsection{Possession}
Existential constructions are also used to indicate possession, with the possessor marked in the patientive case.

\pex
\begingl
\gla Marka je\v{s} obla\v{s}c.//
\glb Marek-\mk{pat} \mk{exst} pet//
\glft \eng{Marek has a pet.}//
\endgl
\xe

\pex
\begingl
\gla Tom\'a\v{s}a niho mlaz.//
\glb Tom\'a\v{s}-\mk{pat} \mk{exst} brother//
\glft \eng{Tom\'a\v{s} does not have a brother.}//
\endgl
\xe

\subsection{Impersonal constructions}
\pex
\begingl
\gla Martina je\v{s} tre\v{s}nikou na tropa.//
\glb Martin-\mk{pat} \mk{exst} write-\mk{pv-pf-nz} \mk{loc} wall-\mk{pat}//
\glft \eng{Martin wrote something on the wall.}//
\endgl
\xe

\pex
\begingl
\gla Vo\v{s}tnikouva sa je\v{s} pia\v{s}\v{c}kou?//
\glb cook-\mk{pv-pf-nz-pat} already \mk{exst} eat-\mk{av-pf-nz}//
\glft \eng{Did somebody eat what (I) cooked?}//
\endgl
\xe


\section{Formation of Verbs}
\subsection{External Derivation}
\par Loanwords ending in \textbf{-ace} from the Latin change the final e to á:
\begin{table}[h!]
	\centering \small
	\begin{tabu} to 0.9\textwidth{>{\bfseries}YM[0.3]>{\bfseries}YY}
		administrace 	& $\rightarrow$ & administracá 	& `to administrate' \\
		akuzace			& $\rightarrow$ & akuzacá		& `to accuse'\\
		diferenzace		& $\rightarrow$ & diferenzacá	& `to differentiate'\\
		separace		& $\rightarrow$ & separacá		& `to separate'\\
	\end{tabu}
\end{table}
\par Some Latin loanwords are borrowed first from German. Loanwords ending in \textbf{-ieren} become \textbf{-irná}.
\begin{table}[h!]
	\centering \small
	\begin{tabu} to 0.9\textwidth{>{\bfseries}YM[0.3]>{\bfseries}YY}
		akzeptieren 	& $\rightarrow$ & akceptirná 	& `to accept' \\
		konservieren	& $\rightarrow$ & koncervirná	& `to conserve'\\
		produzieren		& $\rightarrow$ & producirná	& `to produce'\\
		vandalieren		& $\rightarrow$ & vandalirná 	& `to deface'\\
	\end{tabu}
\end{table}
\subsection{Internal Derivation}
\begin{center}
	\small
	\begin{longtabu}to \textwidth{Y[0.5]Y}
		
		\caption{Verbal Derivational Affixes}
		\label{verbalder}                             \\
		\toprule
		\multicolumn{1}{c}{\sc affix} & \multicolumn{1}{c}{\sc examples}                      \\
		\midrule
		\endfirsthead
		%---------------------------------------------------------------%
		\caption{Verbal derivational affixes \hfill\textit{(continued)}}            \\
		\toprule
		\multicolumn{1}{c}{\sc affix} & \multicolumn{1}{c}{\sc examples}                      \\
		\midrule
		\endhead
		%---------------------------------------------------------------%
		\bottomrule \addlinespace
		\multicolumn{2}{r}{\footnotesize\textit{continued on the next page}}
		\endfoot
		
		\bottomrule
		\endlastfoot
		
		\textbf{nie-} + {\sc adj}\newline`to cause something to become \mk{adj}' &
		
		\textbf{lo\v{s}} `new' $\rightarrow$ \textbf{nielo\v{s}á} `to renew' \newline
		\textbf{preseh} `young' $\rightarrow$ \textbf{niepreshá} `to rejuvenate' \newline
		\textbf{avic} `long' $\rightarrow$ \textbf{nieavicá} `to lengthen' \newline
		\textbf{gem} `soft' $\rightarrow$ \textbf{niegemá} `to soften'\newline
		\textbf{vyne} `dry' $\rightarrow$ \textbf{nievyneá} `to dry'\\ \addlinespace
		
		\textbf{ce-}\footnote{Verbs in \textbf{ce-} cannot be in the reflexive focus.} + {\sc adj}\newline `to cause oneself to become {\sc adj}' &
		
		\textbf{kdavidy} `clean' $\rightarrow$ \textbf{cekdavicá} `to take a bath' \newline
		\textbf{rum} `old' $\rightarrow$ \textbf{cerumá} `to grow old' \newline
		\textbf{\v{s}eznom} `big' $\rightarrow$ \textbf{ce\v{s}eznomá} `to grow up' \newline
		\textbf{vyne} `dry' $\rightarrow$ \textbf{cevyneá} `to dry oneself'\\ \addlinespace
		
		\textbf{hó-} + {\sc noun}\newline `to use {\sc n} in a particular way' &
		
		\textbf{tvem} `tongue' $\rightarrow$ \textbf{hótvemá} `to lick' \newline
		\textbf{kov} `hammer' $\rightarrow$ \textbf{hóková} `to hammer' \newline
		\textbf{\v{s}eznom} `big' $\rightarrow$ \textbf{ce\v{s}eznomá} `to grow up' \newline
		\textbf{vyne} `dry' $\rightarrow$ \textbf{cevyneá} `to dry oneself'\\ \addlinespace 
		
		\textbf{de\v{s}-} + {\sc noun}\newline `to act in the manner of {\sc n}  &
		
		\textbf{tvem} `tongue' $\rightarrow$ \textbf{hótvemá} `to lick' \newline
		\textbf{rum} `old' $\rightarrow$ \textbf{cerumá} `to grow old' \newline
		\textbf{\v{s}eznom} `big' $\rightarrow$ \textbf{ce\v{s}eznomá} `to grow up' \newline
		\textbf{vyne} `dry' $\rightarrow$ \textbf{cevyneá} `to dry oneself'\\ \addlinespace 
		
		\textbf{má-iv} + {\sc noun}\newline `to so something usually done in {\sc noun}'  &
		
		\textbf{mrc} `market' $\rightarrow$ \textbf{mámrcivá} `to shop' \newline
		\textbf{gnazsa} `school' $\rightarrow$ \textbf{mágnazsivá} `to study in'  \newline
		\textbf{\v{s}eznom} `big' $\rightarrow$ \textbf{ce\v{s}eznomá} `to grow up' \newline
		\textbf{vyne} `dry' $\rightarrow$ \textbf{cevyneá} `to dry oneself'\\ \addlinespace 
		
		
		\textbf{sen-/sem-} + {\sc verb}\newline `to {\sc verb} incorrectly'  &
		
		\textbf{o\v{s}tá} `to read' $\rightarrow$ \textbf{seno\v{s}tá} `to misread' \newline
		\textbf{rum} `old' $\rightarrow$ \textbf{cerumá} `to grow old' \newline
		\textbf{\v{s}eznom} `big' $\rightarrow$ \textbf{ce\v{s}eznomá} `to grow up' \newline
		\textbf{vyne} `dry' $\rightarrow$ \textbf{cevyneá} `to dry oneself'\\ \addlinespace 
	\end{longtabu}
\end{center}