\chapter{Pronouns}

Pronouns are words that refer to or substitute a noun or a noun phrase

\section{Personal pronouns}

\begin{table}[h!]
	\centering \footnotesize
	\begin{tabularx}{0.7\textwidth}{>{\scshape}YMMM}
		\toprule
		\multicolumn{1}{c}{\textsc{person}} &\textsc{strong} &\textsc{weak}&\textsc{clitic}\\
		\midrule
		1s &dá&do&-em\\ \addlinespace
		2s&já&je&-esz\\ \addlinespace
		3s.anim&szá&sze&-ej\\ \addlinespace
		3s.inan&ta&cej&-as\\ \addlinespace
		4gen&jedá&dien&-uj\\ \addlinespace
		1pl.inc&chec&chce&-uh\\ \addlinespace
		1pl.exc&kiec&kiec&-ak\\ \addlinespace
		2pl&lou&la&-elý\\ \addlinespace
		3pl.anim&dce&dcá&-ac\\ \addlinespace
		3pl.inan&dcej&oce&-et\\ \bottomrule
	\end{tabularx}
\end{table}

\subsection{Strong form}

The strong form of a personal pronoun (glossed \mk{str}) is used when the pronoun is used as the topic of the sentence. The strong form is indeclinable.

\subsection{Weak form}

\subsection{Clitic form}

\subsection{Pronoun dropping}

\section{Possessive pronouns}

\section{Demonstratives}\index{demonstratives}\index{demonstrative pronouns}

Iridian has a three-way distinction between demonstratives, unlike English but similar to Spanish or Japanese: \emph{proximal} demonstratives are used when referring to objects or people that are near the speaker, \emph{medial} demonstratives when referring to those near the listener, and \emph{distal} demonstratives when referring to those that are far from either the listener or speaker.

In addition, Iridian makes an animacy distinction with demonstratives, with one set of demonstratives used with human referents and another with non-human referents, as seen in Table \ref{dem-prons}.


\begin{table}
	\small\centering
	\caption{Demonstrative pronouns in Iridian.}
	\begin{tabu}to 0.8\textwidth{YMMM}
		\toprule
						& {\sc animate}	& {\sc inanimate}	&{\sc locative}\\
		\midrule \addlinespace
		Proximal		& \v{s}a		& to 				& tak\\ \addlinespace
		Medial			& 				& j\'an				& ko\\ \addlinespace
		Distal			& e\v{s}		& j\'on				& u\v{z}e\\ \addlinespace
		\bottomrule
		\label{dem-prons}
	\end{tabu}
\end{table}


\begin{table}[h!]
	\small\centering
	\caption{Conjugation of Iridian demonstrative pronouns.}
	\begin{tabu}to 0.7\textwidth{YMM}
		\toprule
						& {\sc animate}		& {\sc inanimate}\\
		\midrule
		Proximal		& \v{s}a			& to\\ \addlinespace
		Medial			&&j\'an\\ \addlinespace
		Distal			&&j\'on\\ \addlinespace
		\bottomrule
	\end{tabu}
\end{table}

For information about demonstrative adjectives/determiners, see section \ref{dem-adj}.

\section{Indefinite pronouns and quantifiers}


\section{Interrogative pronouns}
