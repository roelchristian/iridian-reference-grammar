\chapter{Pronouns}

Pronouns are words that refer to or substitute a noun or a noun phrase

\section{Personal pronouns}

\begin{table}[h!]
	\centering \footnotesize
	\begin{tabularx}{0.7\textwidth}{>{\scshape}YMMM}
		\toprule
		\multicolumn{1}{c}{\textsc{person}} &\textsc{strong} &\textsc{weak}&\textsc{clitic}\\
		\midrule
		1s &dá&do&-em\\ \addlinespace
		2s&já&je&-esz\\ \addlinespace
		3s.anim&szá&sze&-ej\\ \addlinespace
		3s.inan&ta&cej&-as\\ \addlinespace
		4gen&jedá&dien&-uj\\ \addlinespace
		1pl.inc&chec&chce&-uh\\ \addlinespace
		1pl.exc&kiec&kiec&-ak\\ \addlinespace
		2pl&lou&la&-elý\\ \addlinespace
		3pl.anim&dce&dcá&-ac\\ \addlinespace
		3pl.inan&dcej&oce&-et\\ \bottomrule
	\end{tabularx}
\end{table}

\subsection{Strong form}

The strong form of a personal pronoun (glossed \mk{str}) is used when the pronoun is used as the topic of the sentence. The strong form is indeclinable.

\subsection{Weak form}

\subsection{Clitic form}

\subsection{Pronoun dropping}

\section{Possessive pronouns}

\section{Demonstratives}

\section{Indefinite pronouns and quantifiers}


\section{Interrogative pronouns}
