\chapter{Nouns}

Nominal morphology in Iridian is relatively simpler compared to the corresponding process. 

\section{Grammatical Categories}

\section{Number}

Nouns in Iridian do not inflect for number. Thus the word \ird{byl}, for example, can mean either \trsl{child} or \trsl{children} depending in context. The same form is used when the noun is preceded by a numeral.

\pex
\begingl
\gla hron\'a byl//
\glb three child//
\glft \trsl{three children}//
\endgl
\xe

Nevertheless, the particle \ird{nie} can be used to denote plurality, although this is largely optional and would be often dropped if the grammatical number of the noun is clear from the context. This is more common for inanimate nouns. Note also that \ird{nie} cannot be used for mass nouns.

\pex
\begingl
\gla Nie byl zap\'o\v{c}ek.//
\glb \mk{pl} child laugh-\mk{av-pf}//
\glft \trsl{The children jumped.}//
\endgl
\xe

\pex
\begingl
\gla *Na duma nie je\v{s} pia\v{s}tou.//
\glb \mk{loc} house \mk{pl} \mk{exst} food//
\glft \trsl{There is food in the house.}//
\endgl
\xe

The particle \ird{nie} always precedes the noun it modifies, except in existential clauses where it comes before the existential particle \ird{je\v{s}}\footnote{The sequence is pronounced as if written n\'ije\v{s} \nt{\nn i:jES}}. \ird{Nie} can obviously not be used with the negative particle \ird{niho}.

\pex
\a
\begingl
\gla nie b\v{z}\k{e}//
\glb \mk{pl} bee//
\glft \trsl{bees}//
\endgl
\a
\begingl
\gla Nie je\v{s} b\v{z}\k{e}//
\glb \mk{pl} \mk{exst} bee//
\glft \trsl{There are bees.}//
\endgl
\a
\begingl
\gla *Nie niho b\v{z}\k{e}//
\glb \mk{pl} \mk{exst.neg} bee//
\glft \trsl{There are bees.}//
\endgl
\xe


\section{Nominal declension}

\subsection{Declesion paradigms}



\begin{table}[h!]
	\small \centering
	\begin{tabu} to 0.6\textwidth {YMM}
		\multicolumn{3}{c}{\textbf{dum} `house'}\\
		\addlinespace
		\toprule
		&{\sc singular} &{\sc plural}\\
		\midrule
		Unmarked & dum & duim\\
		Agentive & dumâ & duimê\\
		Accusative & duma& duime\\
		Dative &dumoh&duimih\\
		Ablative & dumosz & duimesz\\
		Instrumental & dumu & duime\\
		Locative & dumá & duimái\\
		Partitive & dumev& duimev\\
		\bottomrule
	\end{tabu}

\end{table}

\begin{table}[h!]
	\small \centering
	\begin{tabu} to 0.6\textwidth {YMM}
		\multicolumn{3}{c}{\textbf{kryszt} `snake'}\\
		\addlinespace
		\toprule
		&{\sc singular} &{\sc plural}\\
		\midrule
		Unmarked & kryszt & kreiszt\\
		Agentive & krysztê & kreisztê\\
		Accusative & kryszte& kreiszte\\
		Dative &krysztih&kreisztih\\
		Ablative & krysztesz & kreistesz\\
		Instrumental & krysztúi & kreiszte\\
		Locative & krysztái & kreisztái\\
		Partitive & krysztev& kreisztev\\
		\bottomrule
	\end{tabu}
	
\end{table}

\subsection{Agentive case}

\subsection{Ablative case}
The ablative case (glossed \mk{abl}) is fo

\subsubsection{ablative of movement}
\subsubsection{ablative of comparison}
\pex
\begingl
\gla Markosz ezuitoizmét.//
\glb Marek-\mk{abl} \mk{comp-adjz}height\mk{1s}//
\glft `I am taller than Marek.'//
\endgl
\xe


\section{Pronouns}

\begin{table}[h!]
	\centering \footnotesize
	\begin{tabularx}{0.7\textwidth}{>{\scshape}YMMM}
		\toprule
		\multicolumn{1}{c}{\textsc{person}} &\textsc{strong} &\textsc{weak}&\textsc{clitic}\\
		\midrule
		1s &dá&do&-em\\ \addlinespace
		2s&já&je&-esz\\ \addlinespace
		3s.anim&szá&sze&-ej\\ \addlinespace
		3s.inan&ta&cej&-as\\ \addlinespace
		4gen&jedá&dien&-uj\\ \addlinespace
		1pl.inc&chec&chce&-uh\\ \addlinespace
		1pl.exc&kiec&kiec&-ak\\ \addlinespace
		2pl&lou&la&-elý\\ \addlinespace
		3pl.anim&dce&dcá&-ac\\ \addlinespace
		3pl.inan&dcej&oce&-et\\ \bottomrule
	\end{tabularx}
\end{table}

\section{Interrogative pronouns}

\section{Numerals}
\par Iridian has a vigesimal number system. Table \ref{one20} shows Iridian numerals from 1 to 20. Numbers from 1 to 10 are given their own name while numbers from 11 to 19 are formed by appending the numbers from one to nine to the clitic \ird{-niem} with the preposition \ird{sze} (with). The clitic \ird{-niem} is derived from the word for number 10, \ird{nau}, which itself comes from the Old Iridian \rec{nagu}, `half.'
\begin{table}[h!]
	\centering
		\caption{Iridian numerals from 1 to 20.}
\begin{tabu}to 0.8 \textwidth {M[0.5]YM[0.5]Y}
	\toprule
	{\sc number} & {\sc iridian} & {\sc number} & {\sc iridian}\\
	\midrule
	1 & on		& 11 & onszeniem\\
	2 & mui		& 12 & muiszeniem\\
	3 & hroná	& 13 & hronaszeniem\\
	4 & dró		& 14 & drószeniem\\
	5 & jed		& 15 & jeceniem\\
	6 &	vú		& 16 & vúszeniem\\
	7 & szcsê	& 17 & szcsêceniem\\
	8 & pesz	& 18 & pêceniem\\
	9 & cam		& 19 & camzeniem\\
	10& nau		& 20 & tydná\\
	
	\bottomrule
	\label{one20}
\end{tabu}
\end{table}
	
For numbers 11 to 19, the words are formed by appending the numbers from one to nine to the suffix \textit{-niem} with the preposition \textit{sze} (with).

\par Numbers from 21 to 99 are first expressed as multiples of 20. Thenceforth, the number system has largely become decimal, due primarily to the inflyence of surrounding Indo-European languages. Old Iridian, however, had a vigesimal system up to the number 8000.

\par Table \ref{3099} shows multiples of 10 from 30 to 100. The numbers are formed by the numeral followed by \ird{tydná}. For bases that are not multiples of 20, the word \ird{nau} \eng{ten} is added first, followed by the conjunction \ird{sze} \eng{with}. 

\begin{table}[h!]
	\centering
	\caption{Iridian numerals from 30 to 100.}
	\begin{tabu}to 0.9 \textwidth {M[0.5]YM[0.5]Y}
		\toprule
		\multicolumn{1}{c}{\sc number} & \multicolumn{1}{c}{\sc iridian} & \multicolumn{1}{c}{\sc number} & \multicolumn{1}{c}{\sc iridian}\\
		\midrule
		30 &	nauszetydná		& 70 	& nauszehronutydná\\
		40 &	muitydná		& 80	& drohutydná\\
		50 &	nauszemuitydná	& 90	& nauszedrohutydná\\
		60 &	hronutydná		& 100	& miesy\\
		\bottomrule
		\label{3099}
	\end{tabu}
\end{table}

Iridian counting starts from the smallest component of the number to the largest. Each component can be simply appended with the conjunction \ird{sze}. Only the numerals in Tables \ref{one20} and \ref{3099}, and the first ten numbers after 100, 500, 1000, etc. appear as single words. Below are some illustrations:

\pex
\a \ird{jecemiesy}\\
	\eng{five with hundred}\\
	105
\a \ird{cam sze drohutydná}\\
	\eng{nine with four twenties}\\
	89
\xe

\begin{table}[h!]
	\centering
	\caption{Iridian numerals from 200 to one billion.}
	\begin{tabu}to 0.9 \textwidth {Y[0.6]Y}
		\toprule
		\multicolumn{1}{c}{\sc number} & \multicolumn{1}{c}{\sc iridian} \\
		\midrule
		200 			&	moig	\\
		300, 400, etc.	& 	hronumiesy, drohumiesy. etc.\\
		1000			& 	nitak\\
		2000, 3000, etc.& 	muiniec, hronuniec, etc.\\
		10.000			&	ohle\\
		20.000, etc.	& 	tydnuniec, etc.\\
		100.000			&	dunie\\
		200.000 etc		&	meguiniec, hronuniec, etc.\\
		1.000.000		&	myliâ\\
		1.000.000.000	&	myliár\\
		1.000.000.000.000	& byliâ\\
		\bottomrule
		\label{3099}
	\end{tabu}
\end{table}

\subsection{Ordinal numbers}

\subsection{Fractions and decimals}

\subsection{Use of numerals}

\section{Derivational Morphology}

\subsection{-ou}
The nominalizing suffix \ird{-ou} is a non-productive affix used to form nouns from certain verbs.

\begin{table}[h!]
	\centering\small
	\caption{Nominal derivation using \ird{-ou}}
	\begin{tabu} to \textwidth{YYY[0.5]YY}
		\toprule
		\multicolumn{2}{c}{\sc verb root}&&\multicolumn{2}{c}{\sc derived noun}\\
		\addlinespace
		\midrule
		\ird{milovan\'a}&\trsl{to learn}&$\rightarrow$& \ird{milovanou} &\trsl{lesson}\\
		\ird{palz\'a}&\trsl{to leave}&$\rightarrow$& \ird{palzou} &\trsl{departure}\\
		\ird{pia\v{s}t\'a}&\trsl{to eat}&$\rightarrow$& \ird{pia\v{s}tou} &\trsl{food}\\
		\ird{scen\'a}&\trsl{to arrive}&$\rightarrow$& \ird{scenou} &\trsl{arrival}\\
		\ird{niek\'a}&\trsl{to open}&$\rightarrow$& \ird{niekou} &\trsl{entrance}\\

		\bottomrule
	
	\end{tabu}

\end{table}

\subsection{-ou\v{s}c}
The suffix \ird{-ou\v{s}c} (pronounced as if written \ird{-\'o\v{s}t} \bt{o:St}, or in some dialects as \ird{-ou\v{s}t} \nt{\dto{}St}) is used to form a noun indicating someone or something associated to a certain thing or performing a certain action.

\begin{table}[h!]
	\centering\small
	\caption{Nominal derivation using \ird{-ou\v{s}c}}
	\begin{tabu} to \textwidth{YYY[0.5]YY}
		\toprule
		\multicolumn{2}{c}{\sc verb root}&&\multicolumn{2}{c}{\sc derived noun}\\
		\addlinespace
		\midrule
		\ird{jork\'a}&\trsl{to travel}&$\rightarrow$& \ird{jorkou\v{s}c} &\trsl{traveller}\\
		\ird{mo\v{z}l\'a}&\trsl{to live}&$\rightarrow$& \ird{mo\v{z}lou\v{s}c} &\trsl{resident}\\
		\ird{umiel\'a}&\trsl{to get drunk}&$\rightarrow$& \ird{um\'ilou\v{s}c} &\trsl{drunkard}\\
		\ird{virk\'a}&\trsl{to write}&$\rightarrow$& \ird{virkou\v{s}c} &\trsl{writer}\\
		\ird{zdiev\'a} &\trsl{to fool (sm.)}&$\rightarrow$& \ird{zd\'ivou\v{s}c} &\trsl{swindler}\\
		\bottomrule
	
	\end{tabu}

\end{table}