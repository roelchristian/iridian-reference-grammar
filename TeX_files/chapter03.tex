\chapter{Nouns}

\section{Grammatical Categories}
\section{Animacy}
Iridian verbs can be grouped into five classes, traditionally identified as \textbf{genders}, although they more correctly pertain to a noun's animacy:

\begin{table}[h!]
	\centering \small
	\caption{Noun classes in Iridian.}
	\begin{tabu} to \textwidth {M[0.6]YY}
		\toprule
		{\sc class} & \multicolumn{1}{c}{\sc used for}& \multicolumn{1}{c}{\sc examples}\\
		\midrule
		Class I & names of persons & \textbf{Marek, Lukász, Tomász, Ána, Tereza, Luda}\\ \addlinespace
		Class II & common nouns referring to humans or groups of humans & \textbf{syn} `son', \textbf{cena} `family', \textbf{}\\ \addlinespace
		Class III & non-human animate nouns, e.g., animals, plants etc. & \textbf{jec} `dog'\\ \addlinespace
		Class IV & countable concrete and abstract inanimate nouns &\\ \addlinespace
		Class V & uncountable and mass concrete inanimate nouns &\\ \addlinespace
		Class VI & abstract inanimate nouns&\\ \bottomrule
	\end{tabu}
\end{table}

\section{Number}
\par Nouns can be marked for number. However, grammatical number does not always correspond with semantic number, as there are several constructions, discussed later in this section, where semantic number is expressed lexically without a corresponding morphological change.

\par There are three ways to form the plural morphologically in Iridian: (1) by a change in the quality of the vowels in the noun, in accordance with certain rules; (2) by suffixation; or (3) irregularly, most likely due to historical sound changes. Some nouns may have variant plural forms, allowing mostly either vowel change or suffixation.

\subsection{Vowel change}

\par Tha most common method of forming the plural is altering the quality of the vowel(s) in the noun stem. The changes are subject to the rules listed below:

\begin{enumerate}
	\item Vowel length is retained.
	\item If the stem is composed only of harmonic vowels or only of disharmonic vowels, all vowels will be changed. However, if the stem is made of both harmonic and disharmonic vowels, only the harmonic vowels are altered.
	\item Back vowels are fronted.
		\begin{table}[h!]
			\centering \small
			\begin{tabu}  to 0.8\textwidth	{>{\bfseries}YM>{\bfseries}YY}
				pápyr & $\rightarrow$ & páipyr & 'papers'\\
				lóbek & $\rightarrow$ & lóibek & 'apples'\\
				
			\end{tabu} 
		\end{table}
	\item Non-high front vowels are raised.
	\item The unrounded high front vowels \bt{Y~y:} are rounded to \bt{I~i:}.
	\item The rounded high front vowels \bt{I~i:} are dipthongized to \bt{e\tsa{I}}.
	\item The front vowels \bt{E~e:} can either be raised to \bt{I~i:} or dipthongized to \bt{e\tsa{I}}.
	\item The diphthong \bt{e\tsa{I}} simplifies to \bt{i:}.
	\item The dipthong \bt{o\dpu} simplifies to either \bt{y:} or \bt{\o:}.
	\item The nasal vowels \bt{\~5~\~O~\~u} becomes \bt{\~E}.
	\item The nasal vowel \bt{\~E} becomes \bt{in}
			\begin{table}[h!]
		\centering \small
		\begin{tabu}  to 0.8\textwidth	{>{\bfseries}YM>{\bfseries}YY}
			szviêce & $\rightarrow$ & szvýnce & 'candles'\\	
		\end{tabu} 
	\end{table}
	
\end{enumerate}

\par Table \ref{vowels-pl} summarize.

\par Class I nouns cannot be pluralized using this method.

\begin{table}[h!]
	\centering
		\caption{Vowel changes used to mark grammatical number}
	\begin{tabu} to \textwidth {MMM[0.1]MMM[0.1]MM}
		\toprule
		\multicolumn{2}{c}{\sc strong}&&\multicolumn{2}{c}{\sc weak}&&\multicolumn{2}{c}{\sc neutral}\\
		\cmidrule{1-2}\cmidrule{4-5}\cmidrule{7-8}
		Sing. & Pl.&&Sing. & Pl.&&Sing. & Pl\\
		\midrule
		a & ai && ai & oi && e & ei, i\\
		á & ái && ái & ói && é & ei, í\\
		o & oi && oi & ui && ei & í\\
		ó & ói && ói & úi && i & ei\\
		u & ui && ui & i && í & ei\\
		ú & úi && úi & í &&  & \\
		â, ô, û & ê && ê & ýn && &\\
		ou & eu && eu & úi && &\\
		\bottomrule
			\label{vowels-pl}		
	\end{tabu}
\end{table}

\subsection{Suffixation}
\par Some nouns can be pluralized using the suffix \textbf{-ac} or \textbf{-ec}, depending on the vowels in the stem.
\par This method is most often used with loanwords

\subsection{Irregular plurals}
\subsection{Variant plurals}

\section{Nominal declension}

\subsection{Declesion paradigms}



\begin{table}[h!]
	\small \centering
	\begin{tabu} to 0.6\textwidth {YMM}
		\multicolumn{3}{c}{\textbf{dum} `house'}\\
		\addlinespace
		\toprule
		&{\sc singular} &{\sc plural}\\
		\midrule
		Unmarked & dum & duim\\
		Agentive & dumâ & duimê\\
		Accusative & duma& duime\\
		Dative &dumoh&duimih\\
		Ablative & dumosz & duimesz\\
		Instrumental & dumu & duime\\
		Locative & dumá & duimái\\
		Partitive & dumev& duimev\\
		\bottomrule
	\end{tabu}

\end{table}

\begin{table}[h!]
	\small \centering
	\begin{tabu} to 0.6\textwidth {YMM}
		\multicolumn{3}{c}{\textbf{kryszt} `snake'}\\
		\addlinespace
		\toprule
		&{\sc singular} &{\sc plural}\\
		\midrule
		Unmarked & kryszt & kreiszt\\
		Agentive & krysztê & kreisztê\\
		Accusative & kryszte& kreiszte\\
		Dative &krysztih&kreisztih\\
		Ablative & krysztesz & kreistesz\\
		Instrumental & krysztúi & kreiszte\\
		Locative & krysztái & kreisztái\\
		Partitive & krysztev& kreisztev\\
		\bottomrule
	\end{tabu}
	
\end{table}

\subsection{Agentive case}

\subsection{Ablative case}
The ablative case (glossed \mk{abl}) is fo

\subsubsection{ablative of movement}
\subsubsection{ablative of comparison}
\pex
\begingl
\gla Markosz ezuitoizmét.//
\glb Marek-\mk{abl} \mk{comp-adjz}height\mk{1s}//
\glft `I am taller than Marek.'//
\endgl
\xe

\section{Pronouns}

\begin{table}[h!]
	\centering \footnotesize
	\begin{tabularx}{0.7\textwidth}{>{\scshape}YMMM}
		\toprule
		\multicolumn{1}{c}{\textsc{person}} &\textsc{strong} &\textsc{weak}&\textsc{clitic}\\
		\midrule
		1s &dá&do&-em\\ \addlinespace
		2s&já&je&-esz\\ \addlinespace
		3s.anim&szá&sze&-ej\\ \addlinespace
		3s.inan&ta&cej&-as\\ \addlinespace
		4gen&jedá&dien&-uj\\ \addlinespace
		1pl.inc&chec&chce&-uh\\ \addlinespace
		1pl.exc&kiec&kiec&-ak\\ \addlinespace
		2pl&lou&la&-elý\\ \addlinespace
		3pl.anim&dce&dcá&-ac\\ \addlinespace
		3pl.inan&dcej&oce&-et\\ \bottomrule
	\end{tabularx}
\end{table}

\section{Interrogative pronouns}

\section{Numerals}
\par Iridian has a vigesimal number system. Table \ref{one20} shows Iridian numerals from 1 to 20. Numbers from 1 to 10 are given their own name while numbers from 11 to 19 are formed by appending the numbers from one to nine to the clitic \ird{-niem} with the preposition \ird{sze} (with). The clitic \ird{-niem} is derived from the word for number 10, \ird{nau}, which itself comes from the Old Iridian \rec{nagu}, `half.'
\begin{table}[h!]
	\centering
		\caption{Iridian numerals from 1 to 20.}
\begin{tabu}to 0.8 \textwidth {M[0.5]YM[0.5]Y}
	\toprule
	{\sc number} & {\sc iridian} & {\sc number} & {\sc iridian}\\
	\midrule
	1 & on		& 11 & onszeniem\\
	2 & mui		& 12 & muiszeniem\\
	3 & hroná	& 13 & hronaszeniem\\
	4 & dró		& 14 & drószeniem\\
	5 & jed		& 15 & jeceniem\\
	6 &	vú		& 16 & vúszeniem\\
	7 & szcsê	& 17 & szcsêceniem\\
	8 & pesz	& 18 & pêceniem\\
	9 & cam		& 19 & camzeniem\\
	10& nau		& 20 & tydná\\
	
	\bottomrule
	\label{one20}
\end{tabu}
\end{table}
	
For numbers 11 to 19, the words are formed by appending the numbers from one to nine to the suffix \textit{-niem} with the preposition \textit{sze} (with).

\par Numbers from 21 to 99 are first expressed as multiples of 20. Thenceforth, the number system has largely become decimal, due primarily to the inflyence of surrounding Indo-European languages. Old Iridian, however, had a vigesimal system up to the number 8000.

\par Table \ref{3099} shows multiples of 10 from 30 to 100. The numbers are formed by the numeral followed by \ird{tydná}. For bases that are not multiples of 20, the word \ird{nau} \eng{ten} is added first, followed by the conjunction \ird{sze} \eng{with}. 

\begin{table}[h!]
	\centering
	\caption{Iridian numerals from 30 to 100.}
	\begin{tabu}to 0.9 \textwidth {M[0.5]YM[0.5]Y}
		\toprule
		\multicolumn{1}{c}{\sc number} & \multicolumn{1}{c}{\sc iridian} & \multicolumn{1}{c}{\sc number} & \multicolumn{1}{c}{\sc iridian}\\
		\midrule
		30 &	nauszetydná		& 70 	& nauszehronutydná\\
		40 &	muitydná		& 80	& drohutydná\\
		50 &	nauszemuitydná	& 90	& nauszedrohutydná\\
		60 &	hronutydná		& 100	& miesy\\
		\bottomrule
		\label{3099}
	\end{tabu}
\end{table}

Iridian counting starts from the smallest component of the number to the largest. Each component can be simply appended with the conjunction \ird{sze}. Only the numerals in Tables \ref{one20} and \ref{3099}, and the first ten numbers after 100, 500, 1000, etc. appear as single words. Below are some illustrations:

\pex
\a \ird{jecemiesy}\\
	\eng{five with hundred}\\
	105
\a \ird{cam sze drohutydná}\\
	\eng{nine with four twenties}\\
	89
\xe

\begin{table}[h!]
	\centering
	\caption{Iridian numerals from 200 to one billion.}
	\begin{tabu}to 0.9 \textwidth {Y[0.6]Y}
		\toprule
		\multicolumn{1}{c}{\sc number} & \multicolumn{1}{c}{\sc iridian} \\
		\midrule
		200 			&	moig	\\
		300, 400, etc.	& 	hronumiesy, drohumiesy. etc.\\
		1000			& 	nitak\\
		2000, 3000, etc.& 	muiniec, hronuniec, etc.\\
		10.000			&	ohle\\
		20.000, etc.	& 	tydnuniec, etc.\\
		100.000			&	dunie\\
		200.000 etc		&	meguiniec, hronuniec, etc.\\
		1.000.000		&	myliâ\\
		1.000.000.000	&	myliár\\
		1.000.000.000.000	& byliâ\\
		\bottomrule
		\label{3099}
	\end{tabu}
\end{table}

\subsection{Ordinal numbers}

\subsection{Fractions and decimals}

\subsection{Use of numerals}

