\chapter{Nouns}

Nominal morphology in Iridian is relatively simpler compared to the corresponding process. 

\section{Grammatical Categories}

\section{Number}\index{grammatical number}\index{plural}

Nouns in Iridian are not formally marked for number. Thus the word \ird{byl}, for example, can mean either \trsl{child} or \trsl{children} depending on the context. The same form is used when the noun is preceded by a numeral.

\pex
\begingl
\gla hron\'a byl//
\glb three child//
\glft \trsl{three children}//
\endgl
\xe

Nevertheless, Iridian can express semantic plurality by using quantifiers, numerals, pluralizing particles or even through context alone. One such particle is \ird{nie}. The use of \ird{nie}, however, is largely optional and where plurality can be implied from context, this particle is seen as redundant and is therefore dropped.

\pex
\begingl
\gla Nie byl zap\'o\v{c}ek.//
\glb \mk{pl} child laugh-\mk{av-pf}//
\glft \trsl{The children jumped.}//
\endgl
\xe

\ird{Nie} cannot be used with mass and uncountable nouns, as well as with abstract nouns.

\pex
\begingl
\gla *Na duma nie je\v{s} pia\v{s}tou.//
\glb \mk{loc} house \mk{pl} \mk{exst} food//
\glft \trsl{There is food in the house.}//
\endgl
\xe

\pex
\begingl
\gla *example.//
\glb food, liberty//
\glft \trsl{There is food in the house.}//
\endgl
\xe

The particle \ird{nie} always precedes the noun it modifies, except in existential clauses where it comes before the existential particle \ird{je\v{s}}\footnote{The sequence is pronounced as if written n\'ije\v{s} \nt{\nn i:jES}}. \ird{Nie} can obviously not be used with the negative particle \ird{niho}.

\pex
\a
\begingl
\gla nie b\v{z}\k{e}//
\glb \mk{pl} bee//
\glft \trsl{bees}//
\endgl
\a
\begingl
\gla Nie je\v{s} b\v{z}\k{e}.//
\glb \mk{pl} \mk{exst} bee//
\glft \trsl{There are bees.}//
\endgl
\a
\begingl
\gla *Nie niho b\v{z}\k{e}.//
\glb \mk{pl} \mk{exst.neg} bee//
\glft \trsl{There are no bees.}//
\endgl
\xe

\index{pluralia tantum}
\ird{Nie} cannot be used as well with a limited number of nouns, mostly referring to paired body parts and related objects, which in the base form is understood to refer to the pair itself and thus cannot be pluralized. If the speaker wishes to explicitly refer to one piece of the pair, the noun noma (an obsolete form of the word for one-half, now surviving only in this construction) and the genitive form of the body part. 

\pex
\begingl
\gla Eg zaromnek.//
\glb eyes close-\mk{pv-pf}//
\glft \trsl{(He) closed (his) eyes.}//
\endgl
\xe
\pex
\begingl
\gla Poh\'ar dievit.//
\glb eyeglasses dirty//
\glft \trsl{(His) eyeglasses are dirty.}//
\endgl
\xe
\pex
\begingl
\gla Ohv\'i noma utie\v{s}\v{c}\'al.//
\glb shoe-\mk{gen} half \mk{ref-}lose-\mk{av-cont}//
\glft \trsl{The other pair of (his) shoe is missing.}//
\endgl
\xe

The base form is also used in generic statements where English would normally use the plural.

\pex
\begingl
\gla Ohv\'i noma utie\v{s}\v{c}\'al.//
\glb shoe-\mk{gen} half \mk{ref-}lose-\mk{av-cont}//
\glft \trsl{The other pair of his shoe is missing.}//
\endgl
\xe

\section{Definiteness}
Iridian does not have definite or indefinite articles

\section{Uninflected form}

\section{Agentive case}

\subsection{Agentive of comparison}
\pex
\begingl
\gla D\'a Mark\k{a} t\'am stroja.//
\glb \mk{1s.str} Marek-\mk{agt} \mk{comp} tall//
\glft \trsl{Marek is taller than me}//
\endgl
\xe

\section{Patientive case}

The patientive case (glossed \mk{pat}) is formed by appending the suffix \ird{-a} to the root of the noun, subject to the following sound changes, notably affecting vowel-final roots for the most part:

\begin{itemize}
	\item Roots ending in e and o replace the final vowel with \ird{-a}: \ird{pivo -- piva} \trsl{beer}, \ird{malno -- malna} \trsl{language}, \ird{\v{s}uze -- \v{s}uza} \trsl{judge}
	\item Roots ending in \'o and ou replace the final vowel with \ird{-\'ova}: \ird{pia\v{s}tou -- pia\v{s}t\'ova} \trsl{food}, \ird{jav\'o -- jav\'ova} \trsl{lizard}, \ird{metr\'o -- metr\'ova} \trsl{subway}
	\item Roots ending in a lengthen the final vowel to \ird{-\'a}: \ird{cigra -- cigr\'a} \trsl{tiger}, \ird{husa -- hus\'a} \trsl{street}
	\item Roots ending in \'a replace the final vowel with \ird{\'anie}: \ird{kom\'a -- kom\'anie} \trsl{boat}, \ird{vietr\'a -- vietr\'anie} \trsl{pants}
	\item Roots ending in \'e, ei and i replace the root with \ird{-\'ena}: \ird{k\'av\'e -- k\'av\'ena} \trsl{coffee}, \ird{matei -- mat\'ena} \trsl{motorbike}
	\item Roots ending in \'i append \ird{na}:
	\item Roots ending in u or \'u append \ird{-\v{s}a}:
\end{itemize}

\subsection{Direct object}
The patientive case is used to mark the direct object of a verb that is in the agentive voice. Note that this usage implies that the direct object is indefinite unless the noun is further qualified (except through a demonstrative).

\pex
\a
\begingl
\gla Va\v{s}ka pia\v{s}\v{c}em.//
\glb cake-\mk{pat} eat-\mk{av-pf-1s}//
\glft \trsl{I ate cake.}//
\endgl
\a
\begingl
\gla Jed\'a va\v{s}ka pia\v{s}\v{c}em.//
\glb that cake-\mk{pat} eat-\mk{av-pf-1s}//
\glft \trsl{I ate from that cake.}//
\endgl
\a
\begingl
\gla Va\v{s}ko pia\v{s}tnikem.//
\glb cake eat-\mk{pv-pf-1s}//
\glft \trsl{I ate the cake.}//
\endgl
\a
\begingl
\gla Jed\'a va\v{s}ko pia\v{s}tnikem.//
\glb that cake eat-\mk{pv-pf-1s}//
\glft \trsl{I ate that cake.}//
\endgl
\a
\begingl
\gla Hron\'a va\v{s}ke vat\'a pia\v{s}\v{c}em.//
\glb three cake-\mk{gen} slice-\mk{pat} eat-\mk{pv-pf-1s}//
\glft \trsl{I ate three slices of cake.}//
\endgl
\xe

The patientive is also used to mark the direct object when the verb is in the benefactive voice.

\pex
\begingl
\gla \v{S}a vitamina pia\v{s}tebik.//
\glb \mk{3s.anim} vitamin-\mk{pat} eat-\mk{ben-pf}//
\glft \trsl{(She) made him take (his) vitamins.}//
\endgl
\xe

\subsection{Locative}

The patientive is used with the particle \ird{na} to form a compound locative case, which is itself used to indicate a general location.

\pex
\begingl
\gla Tom\'a\v{s} na byra.//
\glb Tom\'a\v{s} \mk{loc} office-\mk{pat}//
\glft \trsl{Tom\'a\v{s} is at the office.}//
\endgl
\xe

\subsection{Patientive of purpose}

The patientive is used with the particle \ird{za} to indicate 

\subsection{Lative}
The lative is a compound case indicating movement into or to the direction of something. It is formed using the particle \ird{de} and a noun or noun phrase in the patientive case.

\subsection{Adessive}
The adessive is formed when the particle \ird{u} is used with the patientive. This compound case indicates that the noun being modified by the noun in the adessive is near or in the vicinity of the noun in the adessive. The adessive case behaves synactically in the same manner as the locative case with na in all cases.

\pex
\begingl
\gla Tom\'a\v{s} u byra.//
\glb Tom\'a\v{s} \mk{ade} office-\mk{pat}//
\glft \trsl{Tom\'a\v{s} is somewhere near the office.}//
\endgl
\xe

The adessive case is also used to approximate time.

\pex
\begingl
\gla Ova\v{z} u 19 \'ora.//
\glb dinner \mk{ade} 19 hour-\mk{pat}//
\glft \trsl{Dinner is around seven.}//
\endgl
\xe

\section{Genitive case}

The genitive (glossed \mk{gen}) is formed by appending the suffix \mk{-\'i} to the root of a noun.

Due the palatalizing nature of the suffix, the following sound changes must be noted:

\begin{itemize}
	\item Roots ending in k, h, and t change the final consonant to c and append the glide \ird{-ie} instead: \ird{Marek -- Marcie} \trsl{Marek}, \ird{avt -- avcie} \trsl{car}, \ird{duh -- ducie} \trsl{head}
	\item Roots ending in d and g change the final consonant to \v{z} and append the suffix \ird{-e} instead: \ird{vod -- vo\v{z}e} \trsl{sister}, \ird{seg -- se\v{z}e} \trsl{flower}
	\item Roots ending in the sibilants s, z, \v{s}, \v{z} and the sibilant affricates c and \v{c} append \ird{e} as well:
	\item Roots ending with a palatalized consonant lose the final y (there only for orthographic reasons in any case) before appending the \ird{-\'i}: \ird{kra\v{s}toly -- kra\v{s}tol\'i}
	\item Roots ending in a or o replace the vowel with e, while those ending in \'a and \'o replace the root with \'i
	\item Roots ending in au, ou, or u replace the vowel with -\'ov\'i: \ird{dnou -- dn\'ov\'i} \trsl{front}
	\item Roots ending in \'au, or \'u replace the vowel with -\'ovie
	\item Roots ending in e, i or \"y replace the vowel with -ev\'i
	\item Roots ending in \'e, ei, \'i or \'y replace the vowel with -\'ev\'i
\end{itemize}


\subsection{Possession}
The simplest use of the genitive case is to indicate ownership or possession.

\pex
\begingl
\gla Marcie dum//
\glb Marek-\mk{gen} house//
\glft \trsl{Marek's house}//
\endgl
\xe

\pex
\begingl
\gla vo\v{z}e ohnou//
\glb sister-\mk{gen} pen//
\glft \trsl{(my) sister's pen}//
\endgl
\xe

\subsection{Genitive of material}

\pex
\begingl
\gla kun\'i prosc//
\glb silver\mk{gen} spoon//
\glft \trsl{silver spoon}//
\endgl
\xe

\subsection{Genitive of the whole}
The genitive can also be used to indicate 

\pex
\begingl
\gla na kra\v{s}tol\'i dn\'ova//
\glb \mk{loc} train:station-\mk{gen} front//
\glft \trsl{in front of the train station}//
\endgl
\xe

Note that the patientive and not the genitive case is used when quantifying a part of the whole.

\pex
\a
\begingl
\gla *\v{z}nohou\v{s}ce hron\'a//
\glb student-\mk{gen} three//
\glft \trsl{three of the students}//
\endgl
\a
\begingl
\gla na \v{z}nohou\v{s}ca hron\'a//
\glb \mk{loc} student-\mk{gen} three//
\glft \trsl{three of the students}//
\endgl
\xe

Nevertheless when quantifying a noun per se, and not in relation to a whole, the uninflected form of the quantifier is used (mostly using indefinite quantifiers such as \trsl{many}, \trsl{a lot}, etc.). If however, the quantification involves a countable unit or division of the noun, the genitive is used, but such unit or division must be further quantified by a numeral or an indefinite quantifier.

\pex
\a
\begingl
\gla Na krouma\v{s}ta po zma je\v{s} pivo.//
\glb \mk{loc} refrigerator-\mk{pat} still few \mk{exst} beer//
\glft \trsl{There's still some beer left in the refrigerator.}//
\endgl
\a
\begingl
\gla Ona pive \v{s}tava unar\'i\v{z}\v{c}em.//
\glb one beer-\mk{gen} mug-\mk{pat} \mk{ref-}order-\mk{av-pv-1s}//
\glft \trsl{I ordered a mug of beer.}//
\endgl
\xe

\subsection{Genitive of movement}

The genitive is also used to indicate movement away from somewhere.

\pex
\a
\begingl
\gla Dum\'i pal\v{z}ek.//
\glb house-\mk{gen} leave-\mk{av-pf}//
\glft \trsl{I left the house.}//
\endgl
\a
\begingl
\gla Dum palzinek.//
\glb house leave-\mk{pv-pf}//
\glft \trsl{I left the \emph{house}.}//
\endgl
\xe

\section{Instrumental case}

The instrumental case (glossed \mk{inst})

\subsection{With some prepositions}

The following prepositions take the instrumental case: \ird{\v{s}e} \trsl{with}

\pex
\begingl
\gla Za bolta \v{s}e Janu st\'o\v{z}\k{a}c.//
\glb for party-\mk{pat} with Jan-\mk{inst} go-\mk{av-ctpv}//
\glft \trsl{(I am) coming to the party with Jan.}//
\endgl
\xe

\subsection{With expressions of time and duration}

\section{Numerals}
\par Iridian has a vigesimal number system. Table \ref{one20} shows Iridian numerals from 1 to 20. Numbers from 1 to 10 are given their own name while numbers from 11 to 19 are formed by appending the numbers from one to nine to the clitic \ird{-niem} with the preposition \ird{\v{s}e} (with). The clitic \ird{-niem} is derived from the word for number 10, \ird{nau}, which itself comes from the Old Iridian \rec{nagu}, `half.'
\begin{table}[h!]
	\centering
		\caption{Iridian numerals from 1 to 20.}
\begin{tabu}to 0.8 \textwidth {M[0.5]YM[0.5]Y}
	\toprule
	{\sc number} & {\sc iridian} & {\sc number} & {\sc iridian}\\
	\midrule
	1 & ona			& 11 & on\v{s}eniem\\
	2 & mui			& 12 & mui\v{s}eniem\\
	3 & hroná		& 13 & hrona\v{s}eniem\\
	4 & dró			& 14 & dró\v{s}eniem\\
	5 & jed			& 15 & jeceniem\\
	6 &	vú			& 16 & vú\v{s}eniem\\
	7 & \v{s}\v{c}\k{e}	& 17 & \v{s}\v{c}\k{e}ceniem\\
	8 & pie\v{s}		& 18 & pi\k{e}ceniem\\
	9 & cam			& 19 & camzeniem\\
	10& nau			& 20 & tydná\\
	
	\bottomrule
	\label{one20}
\end{tabu}
\end{table}
	
For numbers 11 to 19, the words are formed by appending the numbers from one to nine to the suffix \textit{-niem} with the preposition \textit{\v{s}e} (with).

\par Numbers from 21 to 99 are first expressed as multiples of 20. Thenceforth, the number system has largely become decimal, due primarily to the inflyence of surrounding Indo-European languages. Old Iridian, however, had a vigesimal system up to the number 8000.

\par Table \ref{3099} shows multiples of 10 from 30 to 100. The numbers are formed by the numeral followed by \ird{tydná}. For bases that are not multiples of 20, the word \ird{nau} \eng{ten} is added first, followed by the conjunction \ird{\v{s}e} \eng{with}. 

\begin{table}[h!]
	\centering
	\caption{Iridian numerals from 30 to 100.}
	\begin{tabu}to 0.9 \textwidth {M[0.5]YM[0.5]Y}
		\toprule
		\multicolumn{1}{c}{\sc number} & \multicolumn{1}{c}{\sc iridian} & \multicolumn{1}{c}{\sc number} & \multicolumn{1}{c}{\sc iridian}\\
		\midrule
		30 &	nau\v{s}etydná		& 70 	& nau\v{s}ehronutydná\\
		40 &	muitydná		& 80	& drohutydná\\
		50 &	nau\v{s}emuitydná	& 90	& nau\v{s}edrohutydná\\
		60 &	hronutydná		& 100	& miesy\\
		\bottomrule
		\label{3099}
	\end{tabu}
\end{table}

Iridian counting starts from the smallest component of the number to the largest. Each component can be simply appended with the conjunction \ird{\v{s}e}. Only the numerals in Tables \ref{one20} and \ref{3099}, and the first ten numbers after 100, 500, 1000, etc. appear as single words. Below are some illustrations:

\pex
\a \ird{jecemiesy}\\
	\eng{five with hundred}\\
	105
\a \ird{cam \v{s}e drohutydná}\\
	\eng{nine with four twenties}\\
	89
\xe

\begin{table}[h!]
	\centering
	\caption{Iridian numerals from 200 to one billion.}
	\begin{tabu}to 0.9 \textwidth {Y[0.6]Y}
		\toprule
		\multicolumn{1}{c}{\sc number} & \multicolumn{1}{c}{\sc iridian} \\
		\midrule
		200 			&	moig	\\
		300, 400, etc.	& 	hronumiesy, drohumiesy. etc.\\
		1000			& 	nitak\\
		2000, 3000, etc.& 	muiniec, hronuniec, etc.\\
		10.000			&	ohle\\
		20.000, etc.	& 	tydnuniec, etc.\\
		100.000			&	dunie\\
		200.000 etc		&	meguiniec, hronuniec, etc.\\
		1.000.000		&	myliâ\\
		1.000.000.000	&	myliár\\
		1.000.000.000.000	& byliâ\\
		\bottomrule
		\label{3099}
	\end{tabu}
\end{table}

\subsection{Ordinal numbers}

\subsection{Fractions and decimals}

\subsection{Use of numerals}

\section{Derivational Morphology}

\subsection{-ma\v{s}t}

\begin{table}[h!]
	\centering\small
	\caption{Nominal derivation using \ird{-ma\v{s}t}}
	\begin{tabu} to \textwidth{YYY[0.5]YY}
		\toprule
		\multicolumn{2}{c}{\sc root}&&\multicolumn{2}{c}{\sc derived noun}\\
		\addlinespace
		\midrule
		\ird{k\'av\'e}&\trsl{coffee}&$\rightarrow$& \ird{k\'av\'ema\v{s}t} &\trsl{caf\'e}\\
		\ird{krou}&\trsl{cold}&$\rightarrow$& \ird{krouma\v{s}t} &\trsl{refrigerator}\\
		\ird{pia\v{s}tou}&\trsl{food}&$\rightarrow$& \ird{pia\v{s}touma\v{s}t} &\trsl{restaurant}\\
		
		\bottomrule
	
	\end{tabu}

\end{table}

\subsection{-ou}
The nominalizing suffix \ird{-ou} is a non-productive affix used to form nouns from certain verbs.

\begin{table}[h!]
	\centering\small
	\caption{Nominal derivation using \ird{-ou}}
	\begin{tabu} to \textwidth{YYY[0.5]YY}
		\toprule
		\multicolumn{2}{c}{\sc verb root}&&\multicolumn{2}{c}{\sc derived noun}\\
		\addlinespace
		\midrule
		\ird{milovan\'a}&\trsl{to learn}&$\rightarrow$& \ird{milovanou} &\trsl{lesson}\\
		\ird{palz\'a}&\trsl{to leave}&$\rightarrow$& \ird{palzou} &\trsl{departure}\\
		\ird{pia\v{s}t\'a}&\trsl{to eat}&$\rightarrow$& \ird{pia\v{s}tou} &\trsl{food}\\
		\ird{scen\'a}&\trsl{to arrive}&$\rightarrow$& \ird{scenou} &\trsl{arrival}\\
		\ird{niek\'a}&\trsl{to open}&$\rightarrow$& \ird{niekou} &\trsl{entrance}\\

		\bottomrule
	
	\end{tabu}

\end{table}

\subsection{-ou\v{s}c}
The suffix \ird{-ou\v{s}c} (pronounced as if written \ird{-\'o\v{s}t} \bt{o:St}, or in some dialects as \ird{-ou\v{s}t} \nt{\dto{}St}) is used to form a noun indicating someone or something associated to a certain thing or performing a certain action.

\begin{table}[h!]
	\centering\small
	\caption{Nominal derivation using \ird{-ou\v{s}c}}
	\begin{tabu} to \textwidth{YYY[0.5]YY}
		\toprule
		\multicolumn{2}{c}{\sc verb root}&&\multicolumn{2}{c}{\sc derived noun}\\
		\addlinespace
		\midrule
		\ird{jork\'a}&\trsl{to travel}&$\rightarrow$& \ird{jorkou\v{s}c} &\trsl{traveller}\\
		\ird{mo\v{z}l\'a}&\trsl{to live}&$\rightarrow$& \ird{mo\v{z}lou\v{s}c} &\trsl{resident}\\
		\ird{umiel\'a}&\trsl{to get drunk}&$\rightarrow$& \ird{um\'ilou\v{s}c} &\trsl{drunkard}\\
		\ird{virk\'a}&\trsl{to write}&$\rightarrow$& \ird{virkou\v{s}c} &\trsl{writer}\\
		\ird{zdiev\'a} &\trsl{to fool (sm.)}&$\rightarrow$& \ird{zd\'ivou\v{s}c} &\trsl{swindler}\\
		\bottomrule
	
	\end{tabu}

\end{table}