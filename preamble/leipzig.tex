\usepackage[nogroupskip,acronym,nomain]{glossaries}
\setglossarysection{section}
\newglossarystyle{myglosses}{%
  \renewenvironment{theglossary}%
	{\begin{multicols}{2}\raggedright}
	{\end{multicols}}

	    \renewcommand*{\glossaryheader}{}
	    \renewcommand*{\glsgroupheading}[1]{}
	    \renewcommand*{\glsgroupskip}{}
	    \renewcommand*{\glsclearpage}{} 

	     % set how each entry should appear:
	      \renewcommand*{\glossentry}[2]{
	       \noindent\makebox[4em][l]{\glstarget{##1}{\CardoSmallCaps{\glossentryname{##1}}}}
	        \glossentrydesc{##1}\par
	    }


	    \renewcommand*{\subglossentry}[3]{%
	        \glossentry{##2}{##3}
	    }
	}
\newglossarystyle{langlist}{%
  \renewenvironment{theglossary}%
	{\begin{multicols}{2}\raggedright}
	{\end{multicols}}

	    \renewcommand*{\glossaryheader}{}
	    \renewcommand*{\glsgroupheading}[1]{}
	    \renewcommand*{\glsgroupskip}{}
	    \renewcommand*{\glsclearpage}{} 

	     % set how each entry should appear:
	      \renewcommand*{\glossentry}[2]{
	       \noindent\makebox[4em][l]{\glstarget{##1}{{\glossentryname{##1}}}}
	        \glossentrydesc{##1}\par
	    }


	    \renewcommand*{\subglossentry}[3]{%
	        \glossentry{##2}{##3}
	    }
	}

\usepackage[glosses,nonumberlist,mcolblock]{leipzig}
\renewcommand*{\leipzigfont}[1]{{\CardoSmallCaps{#1}}}
\usepackage{multicol}

\newleipzig{Av}{av}{active voice}
\newleipzig{Pf}{pf}{perfective}
\newleipzig{Pv}{pv}{passive voice}
\newleipzig{Agt}{agt}{agent}
\newleipzig{Lnk}{lnk}{linking particle}
\newleipzig{Dim}{dim}{diminutive}
\newleipzig{Cont}{cont}{continuous}
\newleipzig{Ctp}{ctpv}{contemplative}
\newleipzig{N}{n}{negative}
\newleipzig{Exst}{exst}{existential particle}
\newleipzig{Quot}{quot}{quotative}
\newleipzig{Qp}{qp}{quotative particle}
\newleipzig{Pot}{pot}{potential mood}
\newleipzig{Sbj}{sbj}{subjunctive mood}
\newleipzig{Ipf}{ipf}{imperfect}
\newleipzig{Cv}{cv}{converb}
\newleipzig{Rz}{rz}{relativizer}
\newleipzig{Nz}{nz}{nominalizer}
\newleipzig{Att}{att}{attributive}
\newleipzig{Sup}{sup}{supine}
\newleipzig{Foc}{foc}{focus}
\newleipzig{Hon}{hon}{honorific}
\newleipzig{Rec}{rec}{reciprocative}
\newleipzig{Comp}{comp}{comparative}
\newleipzig{Ade}{ade}{adessive}
\newleipzig{Med}{med}{medial}
\newleipzig{Dist}{dist}{distal}
\newleipzig{Subj}{subj}{subjunctive}
\newleipzig{Str}{str}{strong form}
\newleipzig{Wk}{wk}{weak form}
\newleipzig{Lat}{lat}{lative}
\newleipzig{Ger}{ger}{gerund}
\newleipzig{Pfv}{pfv}{perfective*}
\newleipzig{Nom}{nom}{nominative*}
\newleipzig{Lv}{lv}{locative voice}
\newleipzig{Third}{3}{third person*}
\newleipzig{Hort}{hort}{hortative}
\newleipzig{Acc}{acc}{accusative}
\newleipzig{SupN}{sup.n}{supine of necessity}
\newleipzig{SupP}{sup.p}{supine of purpose}
\newleipzig{Dub}{dub}{dubitative particle**}
\newleipzig{Aff}{affrm}{affirmative particle**}
\newleipzig{Ret}{ret}{retrospective aspect}
\newleipzig{Rep}{rep}{reportative particle**}
\newleipzig{Infer}{infer}{inferential particle**}
\newleipzig{Deb}{deb}{debitive}
\newleipzig{Incp}{incep}{inceptive}



%% language abbreviations

% macro to call full name of language
\newcommand{\printlang}[1]{\acrlong{#1}}
\newcommand{\citelang}[1]{\acrshort{#1}}

% list of abbreviations used
\newacronym{en}{en}{English}
\newacronym{cs}{cz}{Czech}




\makeglossaries


