\chapter{A Brief History of Iridia}

\section{Pre-history and the Indo-European migrations}

\section{Under Roman rule}

Parts of what is now Iridia belonged to the Roman province of Pannonia, which was established in 9 AD. Old-Iridian-speaking peoples were concentrated in the cities of Aquila (modern-day Roubže), Missia, and Carnuntum. The Roman province of Pannonia was divided into two parts in 285 AD, with the eastern part becoming the province of Dacia Ripensis. The western part was renamed Pannonia Superior, and the eastern part Pannonia Inferior. The Roman province of Pannonia was finally abolished in 395 AD, when the Roman Empire was divided into the Western Roman Empire and the Eastern Roman Empire.

As with much of the Roman lands, the relationship between the native Iridians and the Romans have generally been amicable, with the Romans often being seen as liberators from the Germanic tribes. The concept of an ``Iridian nationhood'' has not yet been developed, and the Iridians have generally been seen as a collection of tribes, with the Romans often referring to them as ``the Iridians'' rather than ``the Iridian people''. The Romans were also the first to use the term ``Iridia'' to refer to the region, and it was not until the 19th century that the term ``Iridia'' was used to refer to the Iridian people. 


\section{The first Iridian kingdoms}


\section{Principality of Iridia}



\section{The first Iridian Republic}

\section{World War II}

\section{The Iridian People's Republic and communist rule}

\section{The fall of communism}

The twelve-year reign of Enta has seen a continuous decline in the country's GDP and standard of living. Throughout the 1970s the IPR is generally considered one of the poorest countries in Europe.

\section{Iridia today}




