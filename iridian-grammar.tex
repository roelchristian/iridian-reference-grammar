\documentclass[10pt,a5paper]{book}

%% PACKAGES USED

%% Page layout and encoding
\usepackage[margin=0.8in]{geometry}
\usepackage[utf8]{inputenc}
\usepackage[T1]{fontenc}

%% Fonts
\usepackage[light]{kpfonts}
\usepackage{biolinum}
\usepackage{savetrees}
\usepackage{microtype}
\usepackage{float,microtype}

%% Lingustics packages
\usepackage{tipa}
\usepackage{expex}

%% Tables and Graphics
\usepackage{graphics,graphicx,amsmath,amssymb}
\usepackage[font=footnotesize,labelfont=bf]{caption}
\usepackage{longtable}
\usepackage{minitoc}
\usepackage{afterpage}
\usepackage{xcolor,colortbl}
\usepackage{fancyhdr,textcase}
\usepackage{siunitx,tabularx,ragged2e,booktabs,tabu,multicol}
\usepackage{titlesec}
\usepackage{etoolbox}
\usepackage{multirow}

%% Indexing and Cross-referencing
\usepackage{imakeidx}
\usepackage[font=footnotesize]{idxlayout}
\usepackage[hidelinks]{hyperref}

\usepackage[authordate,natbib,backend=biber]{biblatex-chicago}
\DeclareFieldFormat[article]{title}{\mkbibquote{#1}} % make article titles in quotes
\DeclareFieldFormat[thesis]{title}{\mkbibemph{#1}} % make theses italics
\addbibresource{bibliography.bib}
\renewcommand*{\postnotedelim}{\addcolon\addspace}
\DeclareFieldFormat{postnote}{#1}
\DeclareFieldFormat{multipostnote}{#1}


\makeindex
\setcounter{secnumdepth}{2}

%% CUSTOM COMMANDS

\newcommand{\zz}{\textctz}								% voiced palatal sibilant
\newcommand{\bt}[1]{/\textipa{#1}/}				% broad transcription
\newcommand{\nt}[1]{[\textipa{#1}]}				% narrow transcription
\newcommand{\sx}[1]{\textsuperscript{#1}}	% superscript phoneme
\newcommand{\mk}[1]{\textsc{#1}}					% glossing abbreviation
\newcommand{\dpu}{\textsubarch{u}}
\newcommand{\nn}{\textltailn}							% palatal n
\newcommand{\jjg}{\textbardotlessj}				% voiced palatal stop
\newcommand{\llt}{\textltilde}						% dark l
\newcommand{\llb}{\textbeltl}							% nahuatl l
\newcommand{\jn}[1]{\texttoptiebar{#1}}		% top tie for affricates
\newcommand{\tsa}[1]{\textsubarch{#1}}
\newcommand{\glot}{\textglotstop}					% glottal stop
\newcommand{\tss}[1]{\textsubscript{#1}}
\newcommand{\tsup}[1]{\textsuperscript{#1}}
\newcommand{\ird}[1]{\textit{#1}}					% text in Iridian
\newcommand{\dto}{o\textsubarch{u}}				% ou diphthong
\newcommand{\dte}{e\textsubarch{I}}				% ei diphthong
\newcommand{\rec}[1]{*\textit{#1}}				% reconstruction in Old Iridian
\newcommand{\trsl}[1]{`#1'}								% translation in quotes
\newcommand{\rrr}{\textinvscr}						% uvular r

% TODO: Rewrite instances of this command and delete
\newcommand{\asp}[1]{{#1}\textsuperscript{h}}

%% Set block quotes to font size \small

\expandafter\def\expandafter\quote\expandafter{\quote\small}

%% Formatting of linguistic glosses

\lingset{everygla=\it,everyglb=\footnotesize,everyglc=\footnotesize,aboveexskip=6pt,aboveglftskip=0pt,belowexskip=3pt}


%%Formatting of section headings

\makeatletter
\def\thickhrulefill{\leavevmode \leaders \hrule height 1ex \hfill \kern \z@}
\def\@makechapterhead#1{%
	\vspace*{10\p@}%
	{\parindent \z@ \centering \reset@font
		{\Large \scshape \thechapter}
		\par
		\vspace*{1\p@}%
		\interlinepenalty\@M
		\setlength{\arrayrulewidth}{2pt}
		\par\noindent
		\rule{24pt}{2pt}
		\\
		\begin{tabular}{@{\qquad}c@{\qquad}}

			\\
			{\LARGE \scshape \MakeLowercase{#1}\par\nobreak} \\
			\\

		\end{tabular}
		\vskip 30\p@
}}
\def\@makeschapterhead#1{%
	\vspace*{\p@}%
	{\parindent \z@ \centering \reset@font
		{\Large \scshape \vphantom{\thechapter}}
		\par\nobreak
		\vspace*{15\p@}%
		\interlinepenalty\@M
		\setlength{\arrayrulewidth}{2pt}
		\par\noindent
		\rule{24pt}{2pt}
		\\
		\begin{tabular}{@{\qquad}c@{\qquad}}

			\\
			{\LARGE \scshape \MakeLowercase{#1}\par\nobreak} \\
			\\

		\end{tabular}
		\vskip 100\p@
}}


%% Formatting of sections
\titleformat{\section}
{\large\sffamily}{\thesection}{1em}{\large}

%% Formatting of subsections
\titleformat{\subsection}
{\normalfont\sffamily}{\thesubsection}{1em}{\small}

%% Formatting of subsubsections
\titleformat{\subsubsection}
{\normalfont\itshape\bfseries}{\thesubsubsection}{1em}{\small}

%% Formatting of page headers
\pagestyle{fancy}
\renewcommand{\chaptermark}[1]{\markboth{#1}{}}
\renewcommand{\headrulewidth}{0pt}
\fancyhf{}
\fancyhead[LE,RO]{\footnotesize\thepage}
\fancyhead[RE]{\small\itshape{\nouppercase\leftmark}}
\fancyhead[LO]{\small\itshape{\nouppercase\rightmark}}
\cfoot{}


%% Custom column types
\newcolumntype{Y}{>{\RaggedRight\arraybackslash}X}
\newcolumntype{M}{>{\centering\arraybackslash}X}
\newcolumntype{T}[1]{S[table-format=#1]}

% Restart numbering each chapter
\pretocmd{\chapter}{\excnt=1}{}{}

\begin{document}

\author{}
\title{A Reference Grammar of the Iridian Language}
\date{}

\frontmatter
\maketitle

\thispagestyle{empty}

    \vspace*{\fill}
\begin{flushleft}

\begin{tabularx}{\textwidth}{Y}
Copyright © 2019 Roel Christian Yambao\\
\addlinespace
Some rights reserved. This work is licensed under a \href{https://creativecommons.org/licenses/by-sa/4.0/}{Creative Commons Attribution-ShareAlike 4.0 International License.}\\
\addlinespace
First published online, 2019

\end{tabularx}

\end{flushleft}
				% Colophon

\tableofcontents

\cleardoublepage

\listoftables

\chapter*{Preface}
\addcontentsline{toc}{chapter}{Preface}

				% Preface
\chapter*{Abbreviations}
\addcontentsline{toc}{chapter}{Abbreviations}

	\begin{longtabu} to 0.8\textwidth {>{\scshape}YY[2.5]}
		1		& first person\\
		2		& second person\\
		3		& third person\\
		4		& fourth person\\
		abl		& abilitative mood\\
		ade		& adessive\\
		agt		& agent\\
		anim	& animate\\
		att		& attributive\\
		ben		& benefactive focus\\
		cond	& conditional\\
		ctpv	& contemplative aspect\\
		cv		& converb\\
		excl	& exclusive\\
		expl	& expletive\\
		g		& generic number\\
		gen 	& genitive\\
		ger		& gerund\\
		hort	& hortative mood\\
		inan	& inanimate\\
		incl	& inclusive\\
		incp	& inceptive\\
		inf		& infinitive\\
		inst	& instrumental case\\
		ipf		& imperfective aspect\\
		iv		& instrumental focus\\
		loc		& locative\\
		lv		& locative focus\\
		neg		& negative\\
		nz		& nominalizer\\
		opt		& optative mood\\
		pat		& patient\\
		perm	& permissive mood\\
		pf		& perfective aspect\\
		pl		& plural\\
		prosp	& prospective aspect\\
		pv		& patientive focus\\
		q		& question particle\\
		quot	& quotative mood\\
		ref		& reflexive focus\\
		ret		& retrospective aspect\\
		rz		& relativizer\\
		s		& singular\\
		sbj		& subjunctive mood\\
		str		& strong form\\
		sup		& supine\\
		wk		& weak form\\
	\end{longtabu}
					% Abbreviations Used

\mainmatter

\chapter{An Overview of Iridian}

\section{Word Classes}
Traditional Iridian grammar classifies words into three types: \textbf{lóihnelý} (verbs), \textbf{zesztelý} (nouns), and \textbf{múisztelý} (function words)			% Overview of Iridian
\chapter{Phonology}\label{ch:phon}
\section{Vowels}\index{vowel}
\subsection{Oral Vowels}\index{vowel!oral}
Iridian has six pairs of corresponding long and short vowels. With the exception of /a\,aː/, long vowels are tenser than their short counterparts. In addition standard Iridian also features the high central vowel [ɨ] as an allophone of /ɛ/ and /ɪ/ and the low central [ɐ] as an allophone of /a/, in unstressed positions.

\begin{table}[h!]\index{vowel!inventory}
	\small
	\caption{Vowel inventory of standard Iridian.}
	\medskip
	\begin{tabularx}{0.7\textwidth}{YMMMM}
		\toprule
		&\multicolumn{2}{c}{\sc front}&\multirow{2}{*}{\sc central}&\multirow{2}{*}{\sc back}\\
		\cmidrule{2-3} &{\sc urd.} &{\sc rnd.}&&\\\midrule
		Close & ɪ\,iː&ʏ\,yː&(ɨ)&ʊ\,uː\\
		Mid &  ɛ\,eː & && ɔ\,oː\\
		Open&&&(ɐ)\,a\,aː&\\
		\bottomrule
		\label{table:vowels}
	\end{tabularx}
\end{table}

Phonetic realization is generally consistent with orthography as seen in Table \ref{table:vowels-orth} below. There a few observations worth nothing, nevertheless. The low vowel /a/ is realized as the open central unrounded vowel /\"a/. In addition, the short high vowel /i/ becomes the lax [ɪ], although Southern dialects eschew this in favor of [i]. Finally, when appearing at the end of a word, \orth{y} does not represent the short /y/ sound but indicates the palatalization of the preceding consonant, e.g., \ird{krastoly} [ˈkɾastɔʎ] and not [ˈkɾastɔly]. Word-final short /y/ is written as \orth{\"y} instead. Note that both [ʏ] and [y] are diphthongized to [ʏɐ̯] and [yːɐ̯] respectively, if followed by a pause (e.g., \irdp{ahl\'y}{juice} pronounced as [ˈaxlyːɐ̯] instead	of [ˈaxlyː]).

\begin{table}
	\small
	\caption{Orthographic representation of vowels.}
	\medskip
	\begin{tabularx}{0.7\textwidth}{YMMYMM}
		\toprule
		& {\sc short} & {\sc long} & & {\sc short} & {\sc long}\\
		\midrule
		/a/ & a &\'a & /o/ & o &\'o \\
		/e/ & e &\'e & /u/ & u &\'u\\
		/i/ & i &\'i & /y/ & y, \"y &\'y\\
		\bottomrule
		\label{table:vowels-orth}
	\end{tabularx}
\end{table}

\subsection{Diphthongs}\index{diphthong}
Iridian has three phonemic oral diphthongs: \ird{au}\,/au̯/, \ird{ei}\,/eɪ̯/ and \ird{ou}\,/ou̯/. In addition, the diphthongs \ird{oi}\,/ɔɪ̯/ and \ird{ui}\,/uɪ̯/  also occur phonetically, but their occurence is marginal, normally appearing only in fixed expressions (mostly interjections and expletives), such as \irdp{Avui}{Damn it!} [ʔɐˈʋuɪ̯ʔ], \irdp{p\v{s}ehui}{annoying} [ˈpɕɛxuɪ̯ʔ] and \irdp{Oi}{Hey!} [ʔɔɪ̯ʔ].

In most dialects the diphthong /eɪ̯/ has almost completely merged with \ird{\'e} /eː/, although some divergent dialects in the south may realize the diphthong as [iː] (e.g., \irdp{neite}{word} /ˈneɪ̯tɛ/ but realized as [ˈneːtɛ] or ['ɲiːtɛ]).

Vowel sequences beginning with \orth{i} are not considered as dipthongs since \orth{i} merely indicates the palatalization of the preceding consonant. The addition of an acute accent to the initial \orth{i} in sequences like this does not lengthen it as it normally would but indicates the addition of an epenthetic /j/: \irdp{si\v{z}molog\'ia}{seismology} [ˈsɪʑmɔlɔˌɣɪjɐ].

\subsection{Nasal and Nasalized Vowels}\index{vowel!nasal}\index{vowel!nasalized}

Iridian has three nasal vowels: \ird{\k{a}} /ɐ̃w̃/, \ird{\k{e}} /ɛ̃w̃/ and \ird{\k{o}} /ɔ̃/ (rarely /ɔ̃w̃/). Nasal vowels are not disinguished for length. In addition, nasal consonants in coda position are normally deleted, and the preceding vowel becomes phonemically nasal. This deletion does not occur however if the preceding vowel is long or is a dipthong. In cases of nasal coda deletion, \ird{a} and \ird{e} are also dipthongized to [ɐ̃w̃] and [ɛ̃w̃] instead of [\~a] and [ɛ̃]. When unstressed [ɐ̃w̃] and [ɛ̃w̃] are further reduced to [ə̃w̃] (cf. \irdp{bi\k{e}c}{cat} [bʲɛ̃w̃t͡s] with \irdp{nie bi\k{e}c}{some cats} [ˈɲɪbʲə̃w̃t͡s]). This brings the inventory of nasal and nasalized consonants in Iridian to the following: [ɐ̃w̃ ɛ̃w̃ ə̃w̃ ɪ̃ ɔ̃ ũ]

\subsection{Vowel Length}\index{vowel length}\index{long vowel|see{vowel length}}

Vowel length is phonemic in Iridian. Length is represented by an acute accent\index{acute accent} over the long vowel. The short-long vowel pairs differ in quality as well as length, with the short vowels being more lax and the long vowels being tenser in addition to being longer. Diphthongs are phonetically considered as long vowels.
ɛɪɔʊʏ
\begin{table}[h!]
	\small
	\caption{Vowel length and quality.}
	\medskip
	\begin{tabu} to 0.7\textwidth{MMM}
		\toprule
		{\sc archiphoneme} & {\sc lax/short} &{\sc tense/long}\\ \midrule

		/a/	& [a]	& [aː]		\\
		/e/	& [ɛ]	& [eː]		\\
		/i/	& [ɪ]	& [iː]		\\
		/o/	& [ɔ]	& [oː]		\\
		/u/	& [ʊ] & [uː]		\\
		/y/	& [ʏ]	& [yː]		\\
		\bottomrule
	\end{tabu}
\end{table}

\subsection{Allophony}\index{allophone}\index{vowel reduction}

Short vowels in Iridian exhibit considerable allophony, influenced by both stress patterns and palatalization.\index{palatalization} Long vowels nevertheless remain generally stable.

Stressed /a/ is realized as [\ae] between palatal consonants, further reduced to [ɪ] when unstressed, e.g., \ird{pia\v{s}t\'a} ['pʲæɕtäː] vs. \ird{nepia\v{s}t\'a} [ˈnɛpʲɪɕtäː]. Elsewhere /a/ is pronounced [ɐ] when in an unstressed position, although some dialects may further reduce it to a [ə].

The short vowels /ɛ/ and /ɪ/ are reduced to \nt{1} in unstressed positions. In less careful speech, this could cause the elision of the vowel and the formation of consonant clusters or the realization of the preceding consonant as syllabic (especially if it is a liquid). Final /ɛ/ is not reduced in a word-final position if preceding a pause.

\section{Consonants}\index{consonants}
Table \ref{table:fullconsonant} shows a complete list of consonant phonemes in Standard Iridian, with the allophones appearing in parentheses. In total, Iridian has 19 consonant phonemes but with 21 additional allophonic variants.
\begin{table}[h!]
	\small
	\caption{Full consonant inventory of standard Iridian.}\label{table:fullconsonant}
	\medskip
	\begin{tabu} to \textwidth{Y[2]YYYY}
		\toprule\addlinespace
											& {\sc labial}	& {\sc alveolar}		& {\sc palatal}	& {\sc velar}	\\
		\addlinespace\midrule\addlinespace
		Plosive					 	& p~b						& t~d								& c~ɟ 					& k~ɡ 		\\
		\addlinespace
		Nasal							& m~(ɱ)					& n									& ɲ							& (ŋ)			\\
		\addlinespace
		Liquid						&								& ɾ~(ʁ)~l						&	ʎ							&					\\
		\addlinespace
		Sib. Fric.				& 							& s~z	  						& ɕ~ʑ						&					\\
		\addlinespace
		Non-Sib. Fric.		& ʋ							&										& (ç) 					& x~ɣ   	\\
		\addlinespace
		Sib. Affricate    &								& t͡s~(d͡z)					& t͡ɕ~(d͡ʑ)			&				  \\
		\addlinespace
		Non-Sib. Aff. 		&								& 									&			  				& (k͡x~g͡ɣ)\\
		\addlinespace
		Approximant 			& (β̞	)  				 & ð̞									& j				 			& (ʍ~w)		\\
		\addlinespace
		\bottomrule
	\end{tabu}
\end{table}


\subsection{Plosives}

\par Initial velar stops are affricated when following a pause, so that the pair /k~ɡ/ is often realized as [k͡x~ɡ͡ɣ]. Some Southeastern dialects, however, normally realize initial velar stops as aspirated [kʰ~ɡʰ] instead. This sound change can be traced to the initial aspirated stops \rec{\asp{k}}, \rec{\asp{g}}, \rec{\asp{t}} and \rec{\asp{d}} in Old Iridian weakening to affricates.\footnote{Old Iridian \rec{\asp{t}} and \rec{\asp{d}} became the Middle Iridian [t̪͡θ̞ ~d̪͡ð̞] but both have since simplified to /t~d/ in modern Iridian.} The labial stops /{p~b}/ are unaffected by this process as most instances of \rec{\asp{p}} and \rec{\asp{b}} have merged to /b/ or /ʋ/ in modern Iridian.

The velar stops /k~ɡ/ are lenited to the velar fricatives [x~ɣ] intervocalically, before a voiceless stop, after a vocalized l if followed by another vowel or a voiceless stop, or before the nasal consonants /n/ or /m/ if following a vowel immediately. This lenition also occurs word-finally unless followed by a voiced obstruent, in which case, subject to word-final devoicing, they merge to [x]. The voiced /ɡ/ itself has a limited distribution, mostly appearing in consonant clusters with liquids or nasals.

This lenition can also be observed with the voiced stops /b/ and /d/ which become the approximants [β̞	] and [ð̞] (written without the diacritic hereafter) intervocalically or between a vocalized /l/ and another vowel.

The glottal stop [ʔ] is often not regarded as a separate phoneme.
It can occur in three cases: (1) before an onset vowel when following a pause, e.g., \irdp{avt}{car} [ʔäft]; (2) between two vowels that do not form a diphthong, e.g., \irdp{naomá}{laundry} ['näʔɔmäː]; or (3) emphatically, especially in interjections, e.g., \irdp{Oi}{Hey!} [ʔɔɪ̯ʔ], \irdp{K\'ap!}{Look out!} \emph{lit.}, \trsl{danger} [k͡xäpʔ].


\subsection{Nasals}
Iridian has three nasal consonants /m~n~ɲ/. /n/ cannot appear before bilabials and similarly /m/ cannot appear before velars. Both /m/ and /n/ are realized as [m] before either /ʋ/ or /f/. Before velars /n/ is consistently realized as [ŋ], although [n] is also possible in emphatic pronunciation or in word boundaries.

The velar [ŋ] is not phonemic in Iridian but can sometimes be observed, especially in loanwords, where it can be realized as nasalization of the preceding vowel when in the syllable coda or as [ŋ] intervocalically, although [ŋɡ] or [ŋk] is also common. Thus, for example, \irdp{anglevn\'i}{English} can be realized as either [ˈɐ̃w̃lɛʋɲiː] or [ˈäŋlɛʋɲiː] or [ˈäŋɡlɛʋɲiː] in order of currency.


\subsection{Liquids}

Iridian has three liquids: the rhotic /r/ and the lateral /l/ and /l/.

The rhotic /r/ is realized in one of three ways. Word-initially it is pronounced as the uvular fricative [ʁ] (or as the uvular trill fricative [ʀ̝], depending on the speaker, but both transcribed here simply as [ʁ]). The realization as [ʁ] is also often used when pronouncing words emphatically. When in the coda position and before a pause /r/ is realized as [ɾʑ] or simply as [ʑ]. This pronunciation was originally that of a voiceless alveolar trill [r̥] but this has simplified to [r̝] and finally to [ɾʑ] or [ʑ] in Standard Iridian. The  pronunciation as [r̥] or [r̝] may nevertheless persist in some southern dialects, primarily due to Czech\index{Czech} influence. Note that [ɾʑ] or [ʑ] is not affected by word-final devoicing. Elsewhere /r/ is realized as the flap [ɾ]. Palatal /rʲ/ is in general more stable, realized simply as [ɾʲ], although when in the coda position and if not followed by a vowel, it may be realized as [ɾʑ] or [ʑ].

The lateral /l/ is actually the velarized alveolar lateral approximant [ɫ]. Nonetheless the sound has been transcribed throughout as /l/. In the coda position /l/ is completely vocalized and is transcribed here as [w] in standard Iridian; most southern dialects nevertheless retain the pronunciation as [ɫ]. The palatalized /lʲ/ is the palatal lateral approximant [ʎ] and is transcribed as such.

\subsection{Fricatives and Affricates}

The palatal sibilants /ɕ~ʑ/ can be realized as either the palatal [ɕ~ʑ] or the post-alveolar [ʃ~ʒ] with the former being more common. The same is true with the palatal affricates /t͡ɕ~d͡ʑ/, realized as either [t͡ɕ~d͡ʑ] or [t͡ʃ~d͡ʒ], with the former also being more prevalent. The voiced affricated /d͡ʑ/, normally written \ird{d\v{z}}, is marginal, and most loanwords originally containing [d͡ʑ] or [d͡ʒ] are assimilated as [ʑ].

The sequence /t͡sɪ/ and /t͡si:/ are realized as [t͡ɕɪ] and [t͡ɕiː] respectively (viz., \irdp{cigra}{tiger} is realized as [ˈt͡ɕɪɣɾɐ] and not [ˈt͡sɪɣɾɐ]). The stop fricative sequence [tɕ] can occur in syllable boundaries, although as form of hypercorrection most speaker may lengthen the initial stop to [tːɕ] or aspirate it (becoming [tʰ.ɕ]) to further distinguish it from /t͡ɕ/.

The voiceless labial fricative /f/ is another marginal phoneme, appearing usually as an allophobe of /ʋ/. Loanwords containing /f/ generally assimilate to /ʋ/, although most recent borrowings tend to keep the marginal /f/ (cf. \irdp{Vranca}{France} [vɾɐ̃w̃t͡sɐ] vs. \irdp{Feizbuk}{Facebook} [feːzbʊx]).

The approximant /ʋ/ is realized as [v] in onsets before vowels and voiced obstruents (e.g., \irdp{vdinice}{I thought I saw.} [ˈvɟɪnɨt͡sɛ]), as [f] in onsets before voiceless obstruents (e.g., \irdp{vternou}{bicycle} [ˈftɛɾnou̯]), and as [ʋ] or [u̯] in coda and elsewhere (e.g., \irdp{pilav}{pilaf} [ˈpʲɪɫäʋ] or [ˈpʲɪɫäu̯]). The sequence /kʋ/ and /ɡʋ/ is further lenited to the labialized velar fricatives [xʷ~ɣʷ]. The voiceless [xʷ] (from both \orth{kv} and \orth{hv}) is in free variation with [ʍ], with the latter being the more common pronunciation, especially among younger speakers. For simplicity both [xʷ] and [ʍ] will be transcribed as [ʍ].

Modern Iridian has lost the distinction between \bt{h} and /x/, with both $\langle$ch$\rangle$ and $\langle$h$\rangle$,\footnote{Most instances of $\langle$ch$\rangle$ have been replaced with $\langle$h$\rangle$ following various spelling reforms.} historically representing /x/ and \bt{h}, respectively, merging to the velar fricative /x/. This becomes \bt{ç} before voiceless stops word-initially or when following a front vowel, or before the front vowels /i/ and /ɪ/. The sequence $\langle$hl$\rangle$ and $\langle$kl$\rangle$ are realized as \bt{\jn{t\llb}}.

\section{Voicing}
\par Iridian consonants are generally affected by two systems of phonological opposition: a primary distinction between voice and unvoiced consonants, and a secondary distinction between hard and soft consonants (i.e., normal and palatalized consonants).
\par Consonant voicing is phonemic. Voiced consonants are called muddy or dark (\textbf{mrknie}) while unvoiced consonants are called clear (\textbf{hocke}). Iridian has a strong tendency to devoice consonants, a process called \textbf{niehockvo} (clearing, lightening).

\par Voiced consonants are devoiced when followed by a voiceless obstruent, or in word-final position, unless followed by a vowel or a voiced obstruent. Conversely, voiceless obstruents become voiced when followed by another voiced obstruent.

\begin{table}[h!]
	\centering \small
	\begin{tabularx}{0.8\textwidth}{>{\bfseries}YYY}
		avt &\textipa{[P5ft]}& `car'\\
		szkad& \textipa{[Sk5t]} & `serious'\\
		kdavidy & \textipa{["gd5v\sx{j}Ic]} & `clean'\\
		ryz &\textipa{[rIs]}& `rice'\\
	\end{tabularx}
\end{table}

\section{Phonotactics}\index{phonotactics}\label{sec:phonotactics}

\subsection{Syllable structure}\index{syllable structure}\label{sec:syllable-structure}

Ignoring the possible complexity of the onset, nucleus or coda, the basic structure of an Iridian syllable is CV(C), with C representing a consonant and V a vowel.\footnote{An alternative view, founded upon the status of the glottal stop as a non-phoneme in Iridian, would be to consider the basic structure as (C)V(C) instead of CV(C), thus allowing for a null onset. This treats the addition of a glottal stop in word-initial syllables starting with a vowel as mere prothesis.} Iridian has relatively few phonotactic constraints, allowing, at a maximum, syllables of the form (C)\tsup{2}CV(C)\tsup{3}. Nevertheless, most syllables fall in either of the four groups CV, CVC, CCV and CVCC

\begin{table}[h!]
	\footnotesize\sffamily
	\caption{Blevin's criteria as they apply to Iridian.}
	\begin{tabularx}{0.6\textwidth}{YM}
		\toprule
		& {\sc parameter}\\
		\midrule
		Obligatory onset & Yes\\
		Coda & No\\
		Complex onset & Yes\\
		Complex nucleus & Yes*\\
		Complex coda & Yes\\
		Edge effect & \\
		\bottomrule
	\end{tabularx}
\end{table}


\subsection{Onset}

\par Iridian does not allow a null onset (vowel in the syllable onset), i.e., the most basic Iridian syllable should be of the form CV. Words that superficially appear as having a null onset syllable in the initial position are actually preceded by a glottal stop. An epenthetic glottal stop is also added between vowels in a sequence that do not otherwise form dipthongs, or before a vowel in a word-initial position in loanwords.

\begin{center} \small
	\begin{tabularx}{0.8\textwidth}{>{\bfseries}YYY}
		Americe & \bt{P5mE"R\sx{j}I\jn{ts}E} &`America'\\
		uide&\bt{PYDE}& `gong'\\
		ekt&\bt{PExt}&`forehead'\\
	\end{tabularx}
\end{center}

\begin{table}[h!]
	\small \centering
	\caption{Allowed word-initial CC clusters}
	\begin{tabularx}{\textwidth}{YMMMMMMMMMMMMMMMMMMMM}
		\toprule
		&p&b&t&d&k&g&m&n&r&l&s&z&\v{s}&\v{z}&v&\v{c}&dc&c&dz&h\\
		\midrule
		p&&&+&&&&&+&+&+&+&&+&&&&&&&\\
		b&&&&&&&&&+&+&&&&&&&&&&\\
		t&&&&&&&+&&+&+&&&&&+&&&&&\\
		d&&&&&&&+&+&+&+&&&&&+&&&&&\\
		k&&&+&+&&&&+&+&+&+&&+&&+&&&&&\\
		g&&&&&&&&+&+&+&&&&&+&&&&&\\
		m&&&&&&&&+&&&&&&&&&&&&\\
		n&&&&&&&&&&+&&&&&&&&&&\\
		r&&&&&&&&&&&&&&&&&&&&\\
		l&&&&&&&&&&&&&&&&&&&&\\
		s&&&&&&&&&&&&&&&&&&&&+\\
		z&&+&&+&&&+&+&+&+&&&&&+&&&&&\\
		\v{s}&+&&+&&+&&+&+&+&+&&&&&+&+&&+&&+\\
		\v{z}&&&&&&&&&&&&&&&&&&&&\\
		v&&&+&+&+&&&+&+&+&+&&+&&&&&+&&\\
		\v{c}&&&+&&+&&&&&+&&&&&&&&&&\\
		c&&&+&&+&&&+&+&+&&&&&&&&&&+\\
		h&&&+&&&&&&+&+&&&&&+&&&&&\\
		\bottomrule

		\multicolumn{21}{l}{\footnotesize + allowed cluster}
	\end{tabularx}
\end{table}

\par The following CC clusters are allowed to be in onset position:

\begin{enumerate}
	\item Stop followed by a liquid:
		\begin{enumerate}
			\item \bt{pr}: \ird{pragy} \bt{pr5c}, `sand'; \ird{pramou} \bt{pr5"mo\dpu}, `petal'
			\item \bt{tr}: \ird{trâ} \bt{tr\~5\~w}, `bread'; \ird{truig} \bt{trYx}, `ball
			\item \bt{kr}: \ird{krova} \bt{"krOv5}, `egg'; \ird{kramy} \bt{kr5m\sx{j}}, `toe'
			\item \bt{pl}: \ird{plan} \bt{pl5n}, `plan'; \ird{ploika}, \bt{"pl\o x5} `knot'
			\item \bt{tl}:\footnote{This is realized as \bt{t\llb} or even \bt{\llb}.} \ird{tlyk} \bt{t\llb Ix}, `pig'; \ird{tlum} \bt{t\llb Um}
			\item \bt{kl}:\footnote{Realized as \bt{\jn{t\llb}} in Standard Iridian or as \bt{k\r*{l}} in some dialects.} \ird{klug} \bt{\jn{t\llb}Ux}, foot; \ird{klúbe} \bt{"\jn{t\llb}u:bE}, `club'
			\item \bt{br}: \ird{brok} \bt{brOx}, `female teenager'; \ird{bremy} \bt{brEm\sx{j}}, `ugly'
			\item \bt{dr}: \ird{drono} \bt{drOnO}, `brother'; \ird{drúi} \bt{dry:} `enemy'
			\item \bt{gr}: \ird{grec} \bt{grE\jn{ts}}, `flag'; \ird{gryny} \bt{grI\nn} `peace'
			\item \bt{bl}: \ird{bloht} \bt{blOxt}, `mud'; \ird{bleu} \bt{bl\o\textsubarch{Y}} `neck'
			\item \bt{dl}\footnote{This has merged to \ird{tl} in Standard Iridian.}: \ird{dleva} \bt{"\jn{t\llb} Ev5}, `low'; \ird{dlouhe} \bt{\jn{t\llb}\dto xE} `duck'
			\item \bt{gl}: \ird{gloibek} \bt{"gl\o bEx}
		\end{enumerate}
	\item Dental or velar stops followed by /ʋ/: \footnote{/ʋ/ is realized as /ʋ/ in this context. See section of stops for details on \ird{kv} and \ird{gv}.}
		\begin{enumerate}
			\item \bt{tv}:
			\item \bt{dv}:
			\item \bt{kv}: \ird{kvártir} \bt{"\*wOrcIr}, apartment; \ird{kveno} \bt{"\*wEnO}, `kitten'
			\item \bt{gv}: \ird{gvarusz} \bt{G\sx{w}5"rUS}, `speech'; \ird{gvecs} \bt{G\sx{w}E\jn{tS}}, `dinner'
		\end{enumerate}
	\item \bt{k} or \bt{p} followed by \bt{t} or its soft counterpart; \bt{k} followed by /d/ or its soft counterpart:
		\begin{enumerate}
			\item \bt{kt}: \ird{kto} \bt{ktO}, `smile'; \ird{ktiesz} \bt{kcES}, `ache'
			\item \bt{pt}: \ird{pteva} \bt{ptEv5}, `leaf'; \ird{ptiará} \bt{pc5R5}, `count'
			\item \bt{kd}:\footnote{This is always realized as \bt{gd}.}
		\end{enumerate}
	\item \bt{k} or \bt{p} before \bt{s} or \bt{S} or their soft counterparts:
	\begin{enumerate}
		\item \bt{ps}:\footnote{This is a marginal cluster, occuring only in mostly Greek loanwords.} \ird{psyhologa} \bt{psIxOlO"G5}, `psychologist';
		\item \bt{pS}: \ird{pszehuj} \bt{"pSExu\tsa{I}}, \trsl{annoyance}; \ird{pszêcem} \bt{"pS\~E\~w\jn{ts}Em}, \trsl{grain}
		\item \bt{ks}:\footnote{This is another marginal cluster, occuring only in mostly Greek loanwords.}
		\item \bt{kS}: \ird{kszêtva} \bt{"kS\~E\~wtv5}, \trsl{chain}; \ird{kszévet} \bt{"kSe:vEt}, \trsl{basket}
	\end{enumerate}
	\item Dental stops followed by /m/:
	\begin{enumerate}
		\item \bt{tm}: \ird{tmeny} \bt{tmE\nn}, \trsl{belt}; \ird{tmou} \bt{tm\dto}, \trsl{waist}
		\item \bt{dm}:
	\end{enumerate}
		\item \bt{p}, /d/, \bt{k} or \bt{g}  followed by /n/:
	\begin{enumerate}
		\item \bt{pn}:
		\item \bt{dn}:
		\item \bt{kn}:
		\item \bt{gn}:\footnote{Realized as \bt{Gn} after a vowel-final word and \bt{kn} elsewhere.} \ird{gnasz} \bt{kn5S}, \trsl{school}; \ird{gnuma} \bt{knUm5}, \trsl{mattress}
	\end{enumerate}

		\item /m/ followed by /n/ or /n/ followed by /l/:
	\begin{enumerate}
		\item \bt{mn}: \ird{mnucs} \bt{mnU\jn{tS}}, \trsl{husband}; \ird{mnouvaty} \bt{"mn\dto v5c}, \trsl{hunchback}
		\item \bt{nl}:\footnote{This is realized as palatal \bt{\nn L}.} \ird{nlâsz} \bt{\nn L\~5\~wS}, \trsl{castle}; \ird{nlúi} \bt{\nn Ly:}, \trsl{horse}
	\end{enumerate}

		\item \bt{S} followed by a voiceless stop:
	\begin{enumerate}
		\item \bt{Sp}:
		\item \bt{St}
		\item \bt{Sk}:
	\end{enumerate}

		\item \bt{\textctz} before /b/ or /d/:
\begin{enumerate}
	\item \bt{zb}
	\item \bt{zd}:
\end{enumerate}

		\item \bt{S} or \bt{\textctz} followed by a nasal, a liquid, or /ʋ/:
	\begin{enumerate}
		\item \bt{Sm}:
		\item \bt{Sn}:
		\item \bt{Sr}:
		\item \bt{Sl}:
		\item \bt{Sv}:
		\item \bt{zm}:
		\item \bt{zn}:
		\item \bt{zr}:
		\item \bt{zl}:
		\item \bt{zv}:
	\end{enumerate}

		\item \bt{S} before the affricates \bt{\jn{ts}} or \bt{\jn{tS}}:
	\begin{enumerate}
			\item \bt{S\jn{ts}}:
			\item \bt{S\jn{tS}}
	\end{enumerate}

		\item \bt{S} or \bt{s} before the affricates /x/:
\begin{enumerate}
	\item \bt{Sx}:
	\item \bt{sx}
\end{enumerate}

		\item /ʋ/ before the affricates \bt{s} or \bt{S}, /n/, the stops \bt{t}, \bt{k}, or /d/, the liquids /r/ or /l/, or the affricate \bt{\jn{ts}}:
\begin{enumerate}
	\item \bt{vs}:
	\item \bt{vS}:
	\item \bt{vn}:
	\item \bt{vt}:
	\item \bt{vk}:
	\item \bt{vd}:
	\item \bt{vr}:
	\item \bt{vl}:
	\item \bt{v\jn{ts}}:
\end{enumerate}

		\item \bt{\jn{tS}} before \bt{k}, \bt{t}, or /l/:\footnote{CC clusters beginning with \bt{\jn{tS}} have all simplified to \bt{S}.}
\begin{enumerate}
	\item \bt{\jn{tS}k}:
	\item \bt{\jn{tS}t}:
	\item \bt{\jn{tS}l}:
\end{enumerate}

		\item \bt{\jn{ts}} before \bt{k}, \bt{t}, /l/, /r/ /n/ or /x/:
\begin{enumerate}
	\item \bt{\jn{ts}k}:
	\item \bt{\jn{ts}t}:
	\item \bt{\jn{ts}r}:
	\item \bt{\jn{ts}l}:
	\item \bt{\jn{ts}n}:
	\item \bt{\jn{ts}x}:
\end{enumerate}

		\item /x/ before \bt{t}, /l/, /r/ or /ʋ/:
\begin{enumerate}
	\item \bt{xt}:
	\item \bt{xl}:
	\item \bt{xr}:
	\item \bt{xv}:\footnote{This is realized as \bt{\*w}.}
\end{enumerate}

\end{enumerate}

Three-consonant clusters are subject to more constraints.

\begin{table}
	\small
	\caption{Allowed CCC clusters.}
	\begin{tabu} to 0.6\textwidth {Y[2.0]MMMMMM}

		\toprule
		&v&z&sz&p&b&k\\
		\midrule
		pr&&&+&&&\\
		pl&&&+&&&\\
		br&&+&+&&&\\

		tr&+&&+&&&\\
		tl&&&+&&&\\
		tv&&&&&&\\
		dr&&+&&&&\\
		dv&&&&&&\\

		kr&&&+&&&\\
		kl&&&+&&&\\
		kv&&&&&&\\
		gr&&+&&&&\\

		sh&+&&&+&&+\\

		szp&&&&&&\\
		szt&&&&&&\\
		szk&&&&&&\\
		szh&+&&&+&&+\\
		szr&+&&&+&+&+\\
		szc&+&&&+&+&+\\
		szcs&+&&&+&+&+\\

		\bottomrule
	\end{tabu}
\end{table}

\begin{enumerate}
	\item \bt{S}-voiceless stop-liquid clusters
	\begin{enumerate}
		\item \bt{Spr}:
		\item \bt{Str}:
		\item \bt{Skr}:
		\item \bt{Spl}:
		\item \bt{Stl}:
		\item \bt{Skl}:
	\end{enumerate}

	\item \bt{S}, followed by a stop, followed by /ʋ/
\begin{enumerate}
	\item \bt{Skv}:
	\item \bt{Stv}:
	\item \bt{Sdv}:
\end{enumerate}

	\item \bt{\textctz}-voiced stop-/r/ clusters
	\begin{enumerate}
		\item \bt{zbr}:
		\item \bt{zdr}:
		\item \bt{zgr}:
	\end{enumerate}

	\item /ʋ/ followed by a stop, followed by a liquid:
\begin{enumerate}
	\item \bt{vtr}:
	\item \bt{vdr}:
	\item \bt{vkr}:
\end{enumerate}

	\item /ʋ/ followed by \bt{S}, followed by a liquid or a voiceless stop:
\begin{enumerate}
	\item \bt{vSr}:
	\item \bt{vSt}:
	\item \bt{vSk}:
	\item \bt{vSp}:
\end{enumerate}

	\item Stop followed by \bt{S} followed by \bt{t}, \bt{\jn{ts}} or \bt{\jn{tS}}:
\begin{enumerate}
	\item \bt{bS\jn{tS}}:
	\item \bt{bS\jn{ts}}:
	\item \bt{bSt}:
	\item \bt{pS\jn{tS}}:
	\item \bt{pS\jn{tS}}:
	\item \bt{pSt}:
	\item \bt{kS\jn{tS}}:
	\item \bt{kS\jn{tS}}:
	\item \bt{kSt}:
\end{enumerate}
\end{enumerate}

\subsection{Nucleus}

\subsection{Coda}


\section{Consonant Alternations}

A large part of consonant palatalization in Iridian is due to palatalization, with a coda consonant getting in contact with \bt{j}, an unrounded front vowel, or a \bt{j}-glide.

\subsection{Simple palatalization}

\subsection{Palatalization}
\par Iridian consonants can either be hard (\textbf{suhne}) or soft (\textbf{gem}). Consonants are hard by default but become soft when followed by the vowels \textbf{i} or \textbf{í}. The vowel \textbf{y} is normally used to indicate non-palatalizing /i/, although it is used to indicate palatalization word-finally or before \textbf{i}.

\par The use of \ird{-y} is a remnant of word final short \rec{i} from Old Iridian that has since disappeared. The same process has caused the shortening of long \rec{i} to /ɪ/. This sound change did not distinguish between palatalizing and non-palatalizing \rec{i} so that \rec{seni} `tooth' and \rec{seny} `blanket' both merged to modern Iridian \ird{seny} \bt{sE\nn}.

\par Softening involves palatal articulation of labial consonants (e.g., \textbf{be} \textipa{[bE]} vs \textbf{bie} \textipa{[b\sx{j}E]}) or the change to a palatal consonant for non-labials (e.g., \textbf{te} \textipa{[tE]} vs \textbf{tie} \textipa{[cE]}). Table \ref{table:softhard} shows how non-labials are affected by palatalization in Iridian.

\begin{table}[ht!]
	\centering \scriptsize
	\caption{Soft and Hard Consonants}\label{table:softhard}
	\begin{tabu} to \textwidth{MM[0.1]MMM[0.1]MM}
		\toprule
		\multirow{2}{*}{\sc \textbf{series}}&&\multicolumn{2}{c}{\sc \textbf{hard}}&&\multicolumn{2}{c}{\sc \textbf{soft}}\\
		\cmidrule{3-4} \cmidrule{6-7}
		&& Unvoiced	& Voiced	&& Unvoiced	& Voiced	\\
		\midrule
		\textit{\textbf{t} series}&& \textbf{t} [t]& \textbf{d} [d]&&\textbf{ty, ti} \textipa{[c]}&\textbf{dy, di} \textipa{[\jjg]}\\
		\textit{\textbf{k} series}&& \textbf{k} [k]& \textbf{g} \textipa{[g]}&&\textbf{ky, ki} \textipa{[c]}&\textbf{gy, gi} \textipa{[\jjg]}\\
		\textit{\textbf{s} series}&& \textbf{s} [s]& \textbf{z} \textipa{[z]}&&\textbf{sy, si} \textipa{[C]}&\textbf{zy, zi} \textipa{[\textctz]}\\
		\textit{\textbf{\v{s}} series}&& \textbf{sz} \textipa{[S]}& \textbf{zs} \textipa{[Z]}&&\textbf{szy, -i} \textipa{[C]}&\textbf{zsy, -i} \textipa{[\textctz]}\\
		\textit{\textbf{c} series}&& \textbf{c} \textipa{[\jn{ts}]}& \textbf{dz} \textipa{[\jn{dz}]}&&\textbf{cy, ci} \textipa{[tC]}&\textbf{dzy, -i} \textipa{[d\textctz]}\\
		\textit{\textbf{\v{c}} series}&& \textbf{cs} \textipa{[\jn{tS}]}& \textbf{dc} \textipa{[\jn{dZ}]}&&\textbf{csy, -i} \textipa{[\jn{tC}]}&\textbf{dcy, -i} \textipa{[\jn{d\textctz}]}\\
		\textit{\textbf{h} series}&& \textbf{h} \textipa{[x]}& ---&&\textbf{hy, hi} \textipa{[ç]}&---\\
		\textit{\textbf{n} series}&& ---& \textbf{n} \textipa{[n]}&&---&\textbf{ny, ni} \textipa{[\nn]}\\
		\textit{\textbf{l} series}&& ---& \textbf{l} \textipa{[l]}&&---&\textbf{ly, li} \textipa{[L]}\\
		\bottomrule

	\end{tabu}
\end{table}

\par Note how sounds produced using the same manner of articulation merge to the corresponding palatal consonant, keeping the voiced/voiceless distinction, such that both sibilant pairs \ird{s-z} and \ird{sz-zs} soften to \bt{C~\textctz}, the plosive pairs \ird{k-g} and \ird{t-d} to \bt{c-\jjg}, and the affricates \ird{c-dz} and \ird{cs-dc} to \bt{\jn{tC}~\jn{d\textctz}}.\footnote{This merger and word-final devoicing results, for example, to \ird{-ety}, \ird{-edy}, \ird{-eky}, and \ird{-egy} all being pronounced as \bt{Ec}} Some dialects, however may realize soft \ird{cs-dc} as \bt{c~\jjg}.




\subsection{Mutation of labials}



\subsection{Mutation of dentals and velars}
\pex \bt{k} and \bt{g}
\begin{center}
	\small
	\begin{tabu}to 0.8 \textwidth{Y[0.5]YY}
		k$\sim$c		& \ird{Marek} \trsl{Marek}	& \ird{Marcie} \trsl{Marek-\mk{gen}}\\
		k$\sim$\v{c}	&&\\
		g$\sim$\v{z}	&&\\

	\end{tabu}
\end{center}
\xe

\subsection{Compound alternations}

\subsection{Consonant$\sim$zero alternations}

\subsection{Voicing and devoicing}

\subsection{Assimilation of Sibilants}
The sibilants \ird{s, z, \v{s}} and \ird{\v{z}} and the sibilant affricates \ird{c} and \ird{\v{c}} assimilate when forming a cluster, whether in morpheme boundaries or morpheme-internally.

\begin{table}[h!]
	\centering \footnotesize
	\caption{Assimilation of sibilant clusters.}\label{table:sibs}
	\begin{tabu} to 0.8\textwidth{YY[0.8]Y[3]}
		\toprule

		{\sc cluster}	&  & {\sc examples}\\
		\midrule

		s + \v{s}	& \nt{C}&\\

		\v{s} + c or \v{c} & \nt{C\jn{tC}} & \\
		&\nt{Ct}&\\




		\bottomrule
	\end{tabu}
\end{table}


\section{Vowel Alternations}

\subsection{Compensatory vowel lengthening}

\section{Other Phonological Processes}


\section{Phonological Processes}

\subsection{Assimilation of loanwords}


\subsection{Vowel$\sim$zero alternation}

Vowel$\sim$zero alternations refer to an extensive series of morphophonological changes in Iridian causing certain vowels to disappear in certain contexts. Vowels that alternate with zero (i.e., that disappear in certain morphological contexts) are said to be \textit{unstable} vowels.

\par Below is a comprehensive list of environments that trigger vowel zero alternations. Here C represents any phonologically permitted consonant or consonant cluster, V a short vowel and VV a long vowel or a diphthong,

\subsubsection{\_cvcvc stems}
The final V is generally unstable in the following environments

\begin{enumerate}
	\item Stem has the same vowels. Examples: \ird{daman} $\rightarrow$ \ird{damna} `lips'; \ird{ploit} $\rightarrow$ \ird{poilte} `pancake'; \ird{poviasztak} $\rightarrow$ \ird{poviesztkam} `I ate'
	\item V\tss{2} is a short vowel. Examples: \ird{zsedym} $\rightarrow$ \ird{zsedme} `beard'; \ird{elaim} $\rightarrow$ \ird{elme} `fog'
	\item Stressed vowels and most loanwords do not follow this rule. Examples \ird{majoniez} $\rightarrow$ \ird{majonieza} `mayonaise' but \ird{mobil} $\rightarrow$ \ird{mubla} `phone'
	\item Where the deletion would cause the resulting consonant to be geminated or to be a voiced/unvoiced pair of the same consonant, the preceding vowel is lengthened. In the case of voiced/unvoiced pairs, only the voiced consonant is kept. Example: \ird{uidet} $\rightarrow$ \ird{úide}
	\item The presence of a soft consonant in the last or the penultimate consonant position normally inhibit vowel$\sim$zero alternation.
\end{enumerate}


\subsubsection{Stem-final vowel$\sim$zero alternation}

\subsubsection{Suffix-initial vowel$\sim$zero alternation}

\begin{enumerate}
	\item \_CVCVC or \_CVVCVC stems. The final V is generally unstable in the below contexts

	\item
	\item Suffix-initial vowel$\sim$zero alternation
\end{enumerate}

\subsection{Vowel$\sim$vowel alternation}
\par Vowel$\sim$vowel alternations form an integral part of Iridian morphophonology. These changes can be grouped into two broad categories: (1) pluralizing ablaut, which involves the raising or fronting of stem vowels to form the plural of most native nouns and (2) marginal apophony involving the vowels /ɛ/ and \bt{O}.

The first category is one of the most common processes in Iridian, used in the formation of marked plurals. In general, it involves the fronting of back vowels (e.g., o to oi), the raising of low front vowels (ai to oi) and the diphthongization of high front vowels. This change does not affect vowel length, so that long vowels remain long and short vowels remain short. This process is discussed in detail in the chapter on nouns.

The second category involves the short vowels \bt{O} and /ɛ/, and in ome cases \bt{5}. This class of changes is normal observed in the following:

\begin{enumerate}
	\item In \_VC final words, where C is a soft consonant, if followed by a consonat final suffix, or if metathesis or vowel$\sim$zero alternation causes the deletion of the initial vowel of the suffix \bt{5~E~O} become \bt{E~I~U}. The soft consonant remains as soft, although this is not reflected in the orthography

		\pex
	\a <ov> +sztraty + ak $\rightarrow$ szovtretka (I) walked
	\xe
	\item Short \bt{O} in a stable position alternates with \bt{U} and short /ɛ/ is a stable position after a soft consonant with /ɪ/, when followed by a voiced plosive after the deletion of an unstable vowel.

	\pex
	\a \ird{lobek} `apple' $\rightarrow$ \ird{lubka} `apple-\mk{pat}'
	\a \ird{hotel} `hotel' $\rightarrow$ \ird{hotela} `hotel-\mk{pat}'
	\xe

	\item In \_PaC final words, where C is a voiceless obstruent (either phonemically or because of assimilation) or a nasal, \bt{5} becomes /ɛ/ and /ɛ/ becomes /ɪ/ and the voiceless consonant is voiced when followed by a vowel-initial suffix.

		\pex
	\a \ird{szviad} `star' $\rightarrow$ \ird{szvieda} `star-\mk{pat}'
	\a \ird{pian} `fire' $\rightarrow$ \ird{piena} `fire-\mk{pat}'
	\xe

	\pex
	\a \ird{miet} `pot' $\rightarrow$ \ird{mida} `pot-\mk{pat}'
	\a \ird{máliek} `bonfire' $\rightarrow$ \ird{máliga} `bonfire-\mk{pat}'
	\xe

\end{enumerate}



\subsection{Reduplication}
\par Reduplication is a process whereby the stem or a part of the stem of a word, or the word itself is repeated with little or no change.
\par Reduplication is only partially productive in Iridian. Most reduplicated noun forms, for example, have fossilized meanings.

\ex Initial reduplication (CV- prefix)\\
\ird{b\'or\v{z}} `thunder' 	$\rightarrow$ \ird{b\'ob\'or\v{z}} `rumbling sound'\\
\ird{man\'a} \trsl{to drop} $\rightarrow$ \ird{maman\'a} \trsl{to splatter}\\
\xe

\ex Final reduplication (-CV and -CCV suffixes)\\
\ird{b\'or\v{z}} `thunder' 	$\rightarrow$ \ird{b\'ob\'or\v{z}} `rumbling sound'\\
\ird{man\'a} \trsl{to drop} $\rightarrow$ \ird{maman\'a} \trsl{to splatter}\\
\xe

Full reduplication is more common than either initial or final-syllable reduplication, although it is limited (in general) to monosyllabic words with CV, VC or CVC structures.

A possibly grammatically meaningful usage of full reduplication is the repetition of words when answering yes-no questions\index{yes-no questions}. Iridian usually do not use the words for \trsl{yes} or \trsl{no} when responding to yes-no questions, instead repeating the verb.

\subsection{Metathesis}

\subsubsection{slot a infixes}
%count A to Z prefixes 1 to inf suffixes
Slot A prefixes (grammatical voice and copulative form) metathesize the root when the onset is a cluster of two or more consonants subject to the below rules. In the examples we assume a affix of the type \textbf{\glot VC}. The glottal stop is deleted when the infix is added. The subscripts \textit{n} and \textit{s} are used to for phonemes relating to the infix and the stem respectively.

\begin{enumerate}
	\item Liquid-final clusters: C\tss{s}LV\tss{s} + \glot V\tss{n}C\tss{n} $\rightarrow$ C\tss{s}V\tss{n}LC\tss{s}V\tss{n}
	\begin{center}
	\begin{tabu}to 0.8\textwidth{YM[0.1]Y}
		\textbf{trápe} `cloud'&$\rightarrow$&\textbf{turtápe} `cloudy'\\
		\textbf{tresz} `write (\mk{st})'&$\rightarrow$&\textbf{torveszé}\\
		\textbf{szran} `drink (\mk{st}) &$\rightarrow$&\textbf{szirnaná}\\
	\end{tabu}
	\end{center}

	\item Nasal-final clusters: C\tss{s}NV\tss{s} + \glot V\tss{n}C\tss{n} $\rightarrow$ C\tss{s}V\tss{n}NC\tss{s}V\tss{n}
	\begin{center}
		\begin{tabu}to 0.8\textwidth{YM[0.1]Y}
			\textbf{dnoja} `money'&$\rightarrow$&\textbf{duntoja} `rich'\\
		\end{tabu}
	\end{center}
\end{enumerate}

\section{Prosody}\index{stress}\index{prosody}

Stress is not phonemic and is almost always fixed on the first syllable of a word.

Another primary exception includes a small class of interjections (most, but not all, of them onomatopoeic), where the stress is placed on the last syllable.

\section{Orthographic representation}
\subsection{Alphabet}

\par The Iridian language uses the Latin script with the following 29 letters: \textbf{a, b, c, \v{c}, d, e, f, g, h, i, j, k, l, m, n, o, p, q, r, s, \v{s}, t, u, v, w, x, y, z, \v{z}}.

The language was originally written in its own script but after the Latin alphabet has been adapted and has been in use since the First Bohemian Union in the 14th century. In addition, for a brief time in the 12th and again in the 15th century, the Cyrillic script was used to write the language. Due to the historical ties with the Kingdom of Bohemia and its historical successors, Czech orthography has had a great influence on the orthography of Iridian.

The last major change in the orthography of the language was during the 1843 reform, when the spellings <h> and <ch>, historically representing the phonemes \bt{h} and /x/ have been merged to <h> (representing /x/), as the language lost the distinction between the two.

-ch still used at the end of a word

\par Iridian uses two types of diacritics, the acute accent ( ´ ), which is used to mark long vowels, and the circumflex accent ( ˆ ) used to mark nasal vowels. Accented characters are not considered as separate letter.

\begin{table}
	\small
 	\caption{The Iridian alphabet.}\index{alphabet}
	\medskip
	\begin{tabu}to 0.8 \textwidth {YY[1.3]YYY[1.3]Y}
		\toprule
		{{\sc  symbol}} & {\sc name} & {\sc ipa} & {{\sc  symbol}} & {\sc name} & {\sc ipa}\\
		\midrule

		A a	  		& á 	& /a/ 		& O o 		& \'o 		& \bt{O}\\
		B b			& b\'e	& /b/		& P p		& p\'e		& \bt{p}\\
		C c			& c\'et & \bt{\jn{ts}}	& Q q		& kv\'e		& --\\
		\v{C} \v{c} & \v{c}a& \bt{\jn{tC}}	& R r		& er		& /r/\\
		D d			& d\'e	& /d/		& S s		& es		& \bt{s}\\
		E e			& \'e	& /e/		& \v{S} \v{s}& \'e\v{s} & \bt{C}\\
		F f			& f\'i	& --			& T t		& t\'e		& \bt{t}\\
		G g			& g\'e 	& \bt{g}		& U u 		& \'u		& /u/\\
		H h			& há 	& /x/		& V v 		& v\'e 		& /ʋ/\\
		I i			& í 	& /i/		& W w 		& vének		& --\\
		J j			& j\'yt& \bt{j}		& X x 		& iks 		& --\\
		K k 		& ká 	& \bt{k}		& Y y 		& ýpsý\'lon & /y/\\
		L l 		& el 	& /l/		& Z z		& zet 		& \bt{\textctz}\\
		M m			& em 	& /m/		& \v{Z} \v{z}& \v{z}es 	& \bt{\zz} \\
		N n			& en	&				&			&			&\\
		\bottomrule
	\end{tabu}
\end{table}

\begin{table}[h!]
	\small
 	\centering
 	\caption{Supplementary characters used in Iridian.}
	\begin{tabu}to \textwidth {YY[2]YY[2]}

		\toprule
		{{\sc  symbol}} & {\sc name} & {\sc ipa} & {\sc name in ipa} \\
		\midrule

		\'A	\'a		& ne\v{c}ko \'a 	& \bt{a:} & \nt{"nE\jn{tC}kOPa:}\\
		\k{A} \k{a}	& \'a \v{s}e mo\v{z}u & \bt{\~5\~w}&\nt{a:S1"mO\zz u}\\
		\'E \'e		& ne\v{c}ko \'e 	& \bt{e:} & \nt{"nE\jn{tC}kOPe:}\\
		\k{E} \k{e}	&\'e \v{s}e mo\v{z}u & \bt{\~E\~w}&\nt{e:S1"mO\zz u}\\
		\'I \'i		& ne\v{c}ko \'i 	& \bt{i:} & \nt{"nE\jn{tC}kOPi:}\\
		\'O \'o		& ne\v{c}ko \'o 	& \bt{o:} & \nt{"nE\jn{tC}kOPo:}\\
		\k{O} \k{o}	&\'o \v{s}e mo\v{z}u & \bt{\~O}&\nt{o:S1"mO\zz u}\\
		\'U \'u		&ne\v{c}ko \'u 	& \bt{u:} & \nt{"nE\jn{tC}kOPu:}\\
		\'Y \'y		&ne\v{c}ko \'yps\'il\k{o} 	& \bt{y:} & \nt{"nE\jn{tC}kOP:"y:psi:""l\~O}\\
		\"Y \"y		& \'yps\'il\k{o} \v{s}e tr\'emu &/y/&\nt{"y:psi:""lOnC1"tRe:m5}\\
		\bottomrule
	\end{tabu}
\end{table}
						% Phonology
\chapter{Verbs}
\section{Introduction}
Verbs (\ird{sládek}) in Iridian are heavily marked. There is a tendency to encode most of the information contained in the sentence on the verb leaving the noun or noun phrase unmarked if possible.

\par Finite verbs are marked\index{markedness} for the following grammatical categories\index{grammatical categories}:
\begin{enumerate}[nosep]
	\item \textit{Aspect}.\index{aspect} Iridian has three primary aspects: perfective, imperfective and contemplative; and two secondary ones: retrospective and prospective.
	\item \textit{Voice}.\index{voice} Iridian has a strong tendency to leave the topic of the sentence unmarked, instead encoding the primary information on the verb. Due to this, voice must be explicitly marked on the verb. Iridian has the following grammatical voices: agentive, patientive, benefactive, instrumental, locative and reflexive.
	\item \textit{Mood}.\index{mood} Besides the unmarked indicative, Iridian has the following grammatical moods: subjunctive, conditional, hortative, optative, abilitative, permissive and non-volitive. In addition, secondary prefixes are used to express what would otherwise could be considered as moods: inceptive, causative and reciprocative.
\end{enumerate}

Verbs are also marked for person, although this is done by the addition of clitic pronouns and not through a separate conjugation paradigm. In most cases, however, this is left out, especially if clear from the context. Iridian verbs are not marked for tense, gender, or number.

\par Iridian verbs have four classes of non-finite forms: the gerund, the converb, the supine and the generic nominal formed with \textbf{-ou}. The non-finite verb forms are derived from the uninflected verb stem except the generic nominal in \textbf{-ou} which can only be formed from a fully-inflected verb stem. A fifth class exists--the infinitive--but this form is largely defunct and is only used in certain compound constructions. Infinitives end in \textbf{-á} and is used as the citation form of a verb.

\section{Verb stem and citation form}\index{citation form}\index{infinitive}

\par The {\sc citation form} (or {\sc dictionary form}\index{dictionary form|see{citation form}} or {\sc lemma}\index{lemma|see{citation form}}) of a verb is the uninflected {\sc infinitive}\index{infinitive}, a fossilised form rarely used outside of a very few periphrastic\index{periphrasis} and historical constructions (see \S\,\ref{sec:infinitive}). The infinitive ends with the vowel \ird{-á}, and removing this ending will produce the verb stem\index{verb stem}. The final consonant  of the stem is called the thematic consonant\index{thematic consonant} and determines the conjugation paradigm the verb follows.

In general the classification of the verb stems is based on how they behave in two phonemic environments: (1) before a phoneme that triggers palatalization such as \ird{-e-}, \ird{-é-}, \ird{-i-} or \ird{-í-} or a \emph{jod}-glide; and (2) before a spirant or an affricate.

\subsection{Type I conjugation (c/č)}
Type I verbs include those whose stems end in \ird{-t}, \ird{-k}, \ird{-c}, \ird{-č},

\subsection{Type II conjugation (z/ž)}
Type II verbs include those whose stems end in \ird{-d}, \ird{-g}, \ird{-z}, \ird{-ž},

\subsection{Type III conjugation (s/š)}
Type II verbs include those whose stems end in \ird{-d}, \ird{-g}, \ird{-z}, \ird{-ž},

\subsection{Type IV conjugation}
Type II verbs include those whose stems end in \ird{-d}, \ird{-g}, \ird{-z}, \ird{-ž},

\subsection{Type V conjugation}
Type II verbs include those whose stems end in \ird{-d}, \ird{-g}, \ird{-z}, \ird{-ž},


\section{Voice}\index{voice}

Iridian often prefers to encode information on the verb instead of through case marking on nouns. As such, all verbs must be explicitly marked for voice.
\begin{table}[!ht]
	\small
	\caption{Suffixes used to mark grammatical voice.}\medskip
	\begin{tabu} to 0.5\textwidth{YY[0.5]}
		\toprule
		&{\sc ending}\\
		\midrule
		Agentive	& \ird{-aš-}\\
		Patientive	& \ird{-in-}\\ 
		Benefactive	& \ird{-éb-}\\ 
		Locative	& \ird{-oun-}\\ 
		Instrumental& \ird{do-\,-oun-}\\ 
		Reflexive	& -\\ 
		Reciprocal	& \\ 
		\bottomrule
	\end{tabu}
\end{table}

\subsection{Morphophonemic changes}


\subsection{Agentive voice}\index{agentive voice}
\par The agentive voice is used if the subject of the verb is the agent of the action.

\pex
\begingl
\gla Sa piašček.//
\glb already eat-\Av{}-\Pf{}//
\glft `(I) already ate.'//
\endgl
\xe

The affix \ird{-aš-} assimilates to the consonant ending the root, with the vowel \bt{5} normally dropped, subject to the following rules:
\begin{itemize}
	\item č: for roots ending with c, č, k, t
	\begin{itemize}
		\item jelcá + -aš- $\rightarrow$ jelč-, \trsl{to dance}
		\item zdieká + -aš- $\rightarrow$ zdíč-, \trsl{to blow}
		\item piaštá + -aš- $\rightarrow$ piašč-, \trsl{to eat}
	\end{itemize}
	\item z: for roots ending with b, l, m, n, r\footnote{This change does not involve the deletion of the final consonant in the root.}
	\item ž: for roots ending with d, g, z, ž
	\begin{itemize}
		\item baž- + -aš- $\rightarrow$ báž-, \trsl{to give}
		\item stojá + -aš- $\rightarrow$ stóž-, \trsl{to go}
	\end{itemize}
	\item š: for all other endings\footnote{\ird{-h + -aš-} , \ird{-s + -aš-} and \ird{-š + -aš-} both simplify to \ird{-š-}, while the rest retain the final consonant.}
\end{itemize}

Where the assimilation involves the deletion of the final consonant in the root, the preceding vowel is lengthened in compensation if the resulting root would then end in an open syllable.\index{compensatory lengthening}
\begin{multicols}{2}
\pex
\ird{Udúšek.}\\
(instead of \ird{*udušek})\\
\trsl{(I) took a shower.}
\xe
\pex
\ird{Piašček.}\\
(not \ird{*piášček.})\\
\trsl{(I) ate.}
\xe
\end{multicols}

If the remnant vowel is the i-glide \ird{-ie-} or the diphthongs \ird{-ei-} and \ird{-ou-}, the remaining vowel would simplify to \ird{í}, \ird{í} and \ird{ú}, respectively. Consider for example the verb \ird{zdieká} \trsl{to blow}:

\pex
\begingl
\gla Lest zdičime.//
\glb wind blow-\Av{}-\Prog{}//
\glft \trsl{The wind is blowing.}//
\endgl
\xe

Nevertheless the vowel \nt{5} in the root resurfaces in the following cases:

\begin{itemize}
	\item Verbs ending in -irná:
	\item Verb root ending in a consonant cluster with a final liquid, nasal, or v
\end{itemize}

\subsection{Patientive voice}
\par A verb in the patient focus (glossed \Acc{}) indicates that the topic of the sentence is the patient of the verb.

\pex
\begingl
\gla Marek vindekem.//
\glb Marek \mk{<pv>}see-\mk{pf-1s}//
\glft `I saw Marek.'//
\endgl
\xe


\subsection{Benefactive voice}\index{benefactive focus}
\par The benefactive focus (glossed \mk{ben}) is used when the subject of the sentence is the benefactor or director object of the verb. Verbs often change meaning when used in the benefactive focus.


\begin{multicols}{2}
\pex
\begingl
\gla Mač sega nazdébik.//
\glb mother flower-\Acc{} buy-\Ben{}-\Pf{}//
\glft `(I) bought my mother flowers.'//
\endgl
\xe

\pex
\begingl
\gla Kova piaštébime.//
\glb cow eat-\Ben{}-\Prog{} //
\glft \trsl{(I am) feeding the cows.}//
\endgl
\xe

\end{multicols}

The benefactive is also used idiomatically with verbs of judgment including \ird{novietá} \trsl{to like}

\pex
\begingl
\gla Dá čehóvám zánovítébime.//
\glb \First\Sg{} sports-\Agt{} \Neg{}-like-\Ben{}-\Prog{}//
\glft \trsl{I don't like sports.}//
\endgl
\xe

\subsection{Locative voice}

\pex
\begingl
\gla Jé kopnažalíc.//
\glb you laugh-\mk{loc-prog-3s.anim}//
\glft \trsl{He is laughing at you.}//
\endgl
\xe

\subsection{Instrumental voice}


\subsection{Reflexive voice}

The reflexive voice (glossed {\Refl}) is used when the patient of the verb is also the agent of the action. Morphogically, the reflexive voice is not a separate voice but is derived from the agentive form of the verb and the addition of the prefix \ird{u(d)-}.

\pex
\begingl
\gla Na šarta uvižek.//
\glb \Loc{} mirror-Pat{} \Refl{}-see-\Av{}-\Pf{}//
\glft \trsl{I saw myself in the mirror.}//
\endgl
\xe

The use of the reflexive voice is more extensive in Iridian than in English\index{English}, and is somehow similar to how the reflexive construction is used in Romance languages.

\pex
\begingl
\gla Uštižek.//
\glb \Refl{}-take:a:bath-\Av{}-\Pf{}//
\glft \trsl{(I) took a bath.}//
\endgl
\xe

\pex
\begingl
\gla Umúšime.//
\glb \Refl{}-comb-\Av{}-\Prog{}//
\glft \trsl{(I) am combing my hair.}//
\endgl
\xe

Below is a non-exhaustive list of verbs that are normally used in the reflexive voice:
\bigskip

\noindent
\ird{dušá} \trsl{to take a shower}\\
\ird{mušá} \trsl{to comb}\\
\ird{šaštá} \trsl{to sit down}\\

Some verbs may change meaning when used in the reflexive voice.


The reflexive voice is also used to imply that an action happened accidentally or involuntary or that the agent of the action is unknown or unimportant.

The reflexive voice may also be used emphatically, especially in spoken Iridian, to express that the action has been performed for the benefit of the actor/agent of the verb.

\pex
\begingl
\gla Kávéa ušranz\k{a}cem.//
\glb coffee-\Acc{} \Refl{}-drink-\mk{av-ctplv-1s}//
\glft \trsl{I'll drink coffee.} (literally, I'll drink myself coffee)//
\endgl
\xe

\pex
\begingl
\gla Pulša uvošček.//
\glb soup-\Acc{} \Refl{}-cook-\Av{}-\Pf{}//
\glft \trsl{(I) cooked (me) some soup.}//
\endgl
\xe


\par The differences

\section{Grammatical aspect}\index{aspect}\index{grammatical aspect|see{aspect}}
\begin{table}[h!]\small
	\caption{Aspect markers in the indicative mood.}
	\medskip
	\begin{tabu} to 0.5\textwidth{YY[0.5]}
		\toprule
		{\sc aspect}	& {\sc affix}\\
		\midrule
		Perfective		& \ird{-ek}\\
		Retrospective	& \ird{-aní}\\
		Imperfective	& \ird{-eví}\\
		Progressive		& \ird{-ime} \\
		Contemplative	& \ird{-ach/-ah}\footnote{Following Iridian orthographic rules, \ird{-ach} is used at the end of a word and \ird{-ah} elsewhere.}\\
		Prospective		& \ird{-ujám}\\
		Cessative		& \ird{-óvít}\\
		\bottomrule
	\end{tabu}

\end{table}
\subsection{Perfective aspect}
The perfective aspect (glossed \Pf{}) indicates an action that has been completed at some specific point in time. The thematic ending for the perfective aspect is \ird{-ek}, but the initial $\langle$e$\rangle$ is rather unstable and often changes depending on the environment. The initial $\langle$e$\rangle$ becomes $\langle$i$\rangle$ when used with \ird{-in} (the suffix indicating the patientive voice), with the initial $\langle$i$\rangle$ in the preceding suffix often dropped or replaced by an $\langle$e$\rangle$. This change also occurs when following the benefactive suffix \ird{-\'eb} and when followed by the quotative suffix \ird{-e} (in which case the final \ird{-k} is fricativised to $\langle$c$\rangle$).

\pex
\a\begingl
\gla Bych na gnaža Marek vdenik.//
\glb yesterday \Loc{} school-\Acc{} Marek see-\Pv{}-\Pf{}//
\glft \trsl{(I) saw Marek at school yesterday.}//
\endgl
\a\begingl
\gla Vaško piaštnik.//
\glb pastry eat-\Pv{}-\Pf{}//
\glft `(I) ate (the) cake.'//
\endgl
\xe


When negated, the perfective indicates something that ought to be done but had not been done. To state that something simply did not happen, the negative of the retrospective is used instead.

\begin{multicols}{2}
\pex
\a\begingl
\gla Zátélévoniržek.//
\glb \Neg{}-telephone-\Av{}-\Pf{}//
\glft `(I) failed to call.' //
\endgl
\a\begingl
\gla Zátélévoniržaní.//
\glb \Neg{}-telephone-\mk{av-ret}//
\glft `(I) didn't call.' //
\endgl
\xe
\end{multicols}

\subsection{Retrospective aspect}
\par The retrospective aspect (glossed \mk{ret}) is used for a past action that has a continuing relevance in the presence. Consider, for example, the following sentences: (a) \textit{I went to Amsterdam last week}; and (b) \textit{I have been to France in my childhood}. Iridian would translate the verb in (a) using the perfective and the verb in (b) using the retrospective.

\pex<ret-pres1>
\begingl
\gla Hroná tímu na Budapešta možlašaním.//
\glb three year-\Ins{} \Loc{} Budapest-\Acc{} live-\mk{av-ret-1s}//
\glft `I have been living in Budapest for three years.'//
\endgl
\xe

\pex<ret-pres>
\begingl
\gla Páku šavolnaníc.//
\glb before-\Ins{} hurt-\mk{pv-pf-3s.anim}//
\glft `She has been hurt before.' //
\endgl
\xe

\par The retrospective is also often used to imply non-volition or the  accidental/circumstantial nature of an action. Similarly the retrospective is used with verbs of emotion or state (e.g., \ird{cezuštalá}, ‘to become happy’ from \ird{zuštal} ‘happy’). The perfective, on the other hand, is almost exclusively used with the causative in these cases.

\pex
\a	\begingl
\gla Vdešek še neicezuštalašaním.//
\glb see-\mk{2s-pf} with \mk{incep}-be.happy-\mk{av-ret-1s}//
\glft `I became happy when I saw you.' //
\endgl
\a	\begingl
\gla Do pacezuštalnikeš.//
\glb \First{}\Sg{}.\Wk{} \Caus{}-be.happy-\mk{pv-pf-2s}//
\glft `You made me happy.' //
\endgl
\xe
\pex<vasebroke>
\begingl
\gla Váz noprizaní.//
\glb vase break-\mk{ref-ret}//
\glft `The vase broke (accidentally).' //
\endgl
\xe

\subsection{Continuous and Progressive Aspects}
Iridian uses the continuous and progressive aspects to denote actions that have not been completed yet and/or are in the process of happening/occuring. The continuous aspect (glossed \mk{cont}) is used to mark a state of being while the progressive aspect (glossed \mk{prog}) is used to mark a dynamic activity.
\pex
\begingl
\gla Nau urištneví.//
\glb clothes \Refl{}-wear-\mk{pv-cont}//
\glft \trsl{(I'm) wearing clothes.} //
\endgl
\xe

\pex
\begingl
\gla Nau urištnime.//
\glb clothes \Refl{}-wear-\mk{pv-prog}//
\glft \trsl{(I'm) putting on clothes.} //
\endgl
\xe

The continuous aspect is also used to denote a habitual action.

\pex
\begingl
\gla Sholu de gnaža stoževí.//
\glb daily-\Ins{} \mk{ill} school-\Acc{} go-\Av{}-\Cont{}//
\glft \trsl{(We) go to school everyday.} //
\endgl
\xe

\pex
\begingl
\gla Dá na Praha možleví.//
\glb \mk{1s.str} \Loc{} Prague-\Acc{} live-\mk{cont}//
\glft \trsl{I live in Prague.} //
\endgl
\xe

To emphasize the habitual nature of an action, a nominalised construction is often used.

\pex
\begingl
\gla Nažem r\k{a}cenživou.//
\glb friend-\First{}\Sg{} smoke-\mk{av-cont-nz}//
\glft \trsl{My friend is a smoker.} //
\endgl
\xe

\subsection{Prospective aspect}
\par The prospective aspect (glossed {\sc prosp}) is primarily used in secondary clauses to indicate actions that are about to be started in relation to another action. It can also be used in the main clause to indicate an action in the immediate future.

\subsection{Cessative aspect}
The cessative aspect

\section{Valency}\index{valency}

Valency (or valence)\index{valency} is the number of overt arguments a verb\index{argument of a verb} can take in a sentence. \textcite[239]{tesniere1965}, in one of the earliest description of the concept, likens valency by comparing it to bonds between atoms:
\begin{quotation}
	\small
The verb may therefore be compared to a sort of atom, susceptible to attracting a greater or lesser number of actants,\footnote{In his work Tesni\`ere used the term \emph{actants} to refer to what we would call here the verb's \trsl{arguments.}} according to the number of bonds the verb has available to keep them as dependents. The number of bonds a verb has constitutes what we call the verb's valency.
\end{quotation}

More rigorous treatments\footnote{\posscite{tesniere1959} definition of valency as \trsl{nombre d'\emph{actants} qu'un verbe est susceptible de régir} (\trsl{number of \emph{actants} which a verb is capable of governing}) essentially frames valency as a function of the verb. More recent definitions however consider valency not just as a property of verbs alone but of any lexical item (cf., e.g., \cite{matthews1997,trask1993}). In addition, in his glossary, he has provided voice (Fr. \emph{voix}) as a synonym for valency; these two terms however we consider as distinct items both in this work and in what I think is the usage of both terms in scholarly literature over the topic.} have of course been published in the years since but we should content ourselves with this definition in our present treatment of Iridian grammar. Instead our primary focus would e

\subsection{Avalent verbs}

Avalent verbs \index{avalent verb} are verbs that have zero core arguments. In Iridian they are limited to a small set of verbs that describe meteorological phenomena, traditionally referred to as `weather verbs' (\ird{plodní sládek}) \index{weather verb}.This term is not wholly accurate, however, as the class includes not just meteorological phenomena but more general natural phenomena as well. When used this way they are marked in the agentive voice\index{agentive voice} and essentially forms topicless sentences\index{topicless sentence} (cf.~\S\,\ref{sec:topicless}). Some common weather verbs in Iridian are listed below.

\pex\deftagex{exw}
\irdp{hravá}{to have the sun shine}\\
\irdp{žužá}{to snow}\\
\irdp{pozbiešá}{to rain}\\
\irdp{néšá}{to rain lightly, to drizzle}\\
\irdp{boboržá}{to have thunder}\\
\irdp{kopriká}{to have lightning}\\
\irdp{dozbuhá}{to have an earthquake}
\xe



\subsection{Passive constructions}


\subsection{Causative constructions}\index{causative}

Causatives may either be lexical, analytical or morphological. Lexical causatives involve the encoding of the causation on the verb itself leading the causative form of the verb to be a different form altogether. An analytical causative, on the other hand uses a different verb (usually a verb like \emph{to do} or \emph{to make}) in conjunction with the main verb, to express the idea of causation (e.g., English\index{English} \trsl{make someone do something.}) Finally, morphological causatives involve morphologically changing the main verb to express the notion of causation. Iridian causative constructions are primarily morphological, formed using the prefix \ird{-ne}.

\begin{table}
\footnotesize
\caption{Causative forms of the verb \irdp{shradá}{to die.}}
\medskip
	\label{tbl:causative}

    \begin{tabu}to \textwidth{Y[0.5]Y[0.6]YY}
         \toprule\addlinespace
		 										& {\sc causative } &{\sc regular meaning} & {\sc causative meaning}\\\addlinespace
												\midrule\addlinespace
				Unmarked				& neshradá									& to die, to be dead 	& \emph{(defective)} \\ \addlinespace
		 		Asgentive				& \ird{neshrážá}			& to kill & to cause someone to kill\\ \addlinespace
		 		Patientive			& \ird{neshradiná}					& to be killed & to be caused to be killed\\\addlinespace
				Benefactive			& \ird{neshradébá}				& to have someone die for oneself	& to have someone be killed for oneself\\\addlinespace
				Locative				& \ird{neshradouná}					& to have someone related die&\emph{(defective)}\\\addlinespace
				Instrumental		& \ird{doneshradouná}&to be the reason for dying&to to be used for killing\\\addlinespace
				Reflexive				& \ird{uneshražá}&to kill oneself&to cause one to commit suicide\\
		 		\addlinespace
				\bottomrule

    \end{tabu}

\end{table}

Due to this suppletive nature, lexical causatives imply a more direct causation, or a tighter link between cause and event\footnote{\textcite{haiman1983} offers a thorough discussion of how the linguistic distance exhibited by the forms of causative constructions existing in a language (e.g., \emph{to cause to die} on one end of the spectrum versus \emph{to kill} on the other) correspond to the conceptual distance between the action of the causer and the result of the action to the causee. In a purely synthetic construction like \emph{kill}, for example, where the linguistic distance is the least, the conceptual distance between the action and the resulting state is also the smallest, with the opposite being true in purely analytical constructions like \emph{to cause to die}.}, than analytical or morphological causatives (\cite{velupillai2012, haiman1983}). Consider for example the three sentences in English\index{English} below:


\pex
\a Joseph \emph{died}.\deftagex{caus}
\a Joseph \emph{killed} the man.\deftagex{caus}\deftaglabel{kill}
\a Joseph \emph{made} the man \emph{die.}\deftagex{caus}\deftaglabel{made}
\xe

The suppletive \emph{kill} in example (\getfullref{caus.kill}) implies more agency on the part of the subject than the more indirect-sounding (\getfullref{caus.made}). In (\getfullref{caus.kill}) the \emph{death} of the patient (\trsl{the man}) is the goal of the act while (\getfullref{caus.made}) it might be inferred that the \emph{dying} was an indirect consequence of an unmentioned second act.


Iridian does not employ lexical causatives as in English\index{English}; instead causatives are formed morphologically by adding the prefix \ird{ne-} (glossed as \Caus{}) to the verb stem. Although \ird{ne-} is required to form the causative morphologically, some verbs, particularly stative verbs like \irdp{shradá}{to die, to be dead} in table \ref{tbl:causative} may already contain the notion of causation in some of its regular conjugated forms. This is because by default stative verbs\index{stative verb} are intransitive (i.e., the only argument required is the actor/agent\index{agent}) while some verbal voices\index{voice} like the patientive\index{patientive voice} and benefactive\index{benefactive voice} inherently imply the existence of a second and a third argument of a verb\index{argument of a verb} respectively.

%% TODO add section reference

Of course Iridian's definition of which verbs are stative and which ones are dynamic\index{dynamic verb} does not neatly align with the definition those classes have in English\index{English} (v. \S\,\ref{sec:statives}). For instance the verbs \emph{to stand} and \emph{to eat} are both dynamic verbs in English\index{English}, while in Iridian \irdp{zdavá}{to stand, to be standing} is stative and only \irdp{piaštá}{to eat} is dynamic. This is why as we see in example (\getfullref{statdyn.1}) below, some forms of the verb \ird{zdavá} already contain the notion of causation in some of its regular conjugated forms.

\pex
\a  \irdp{zdavá}{to be standing}\deftagex{statdyn}\deftaglabel{1}\\
	\irdp{zdavžá}{to stand}\\
    \irdp{zdavná}{to be made standing, to erect}\\
    \irdp{nezdavžá}{to make so./sth. stand}\\
    \irdp{nezdavná}{to be made to make so./sth. standing}
\a  \irdp{piaštá}{to eat}\\
    \irdp{piaštiná}{to be eaten}\\
    \irdp{nepiaščá}{to make someone eat}
\xe

Since causative constructions in Iridian are purely morphological\footnote{To contrast, consider Japanese\index{Japanese} which also forms causative constructions morphologically (using the suffix \emph{-(sa)se}) but which in addition also has synthetic but not fully suppletive forms for some verbs (e.g., \irdp{agaru}{to rise} and \irdp{ageru}{to raise}).} the degree of agency of the causer can be implied from other incidental properties of the verb such as aspect or voice markings.

We pay particular attention first on the interaction of the causative prefix \ird{ne-} with the patientive voice marker \ird{-in} and the benefactive voice marker \ird{-éb}. We begin with stative verbs, since as mentioned earlier and in \S\,XX, most stative verbs will have a causative reading when used with the agentive or benefactive voice. Stative verbs encode the state of the subject and cannot therefore express the idea of an agent nor that of a patient. By conjugating stative verbs for voice, their stative nature is therefore lost; that is why a causative cannot be derived from the unmarked form of a stative verb: a causative construction precludes the existence of a causer and a causee, which at times may be different from the subject, while the unmarked stative only that of the subject itself.


\begin{figure}[H]
	{
	\footnotesize
  \begin{forest}
    [\irdp{shradá}{to die},
		[\ird{shradiná}\\
			patientive\\
			{Arg = 1}
				[$
				\begin{bmatrix}
					\textbf{+ Patient}
				\end{bmatrix}
				$]
				]
      [\ird{shražá}\\
				agentive\\
				{Arg = 2}
					[$
					\begin{bmatrix}
						\textrm{+ Patient}\\
						\textbf{+ Agent}
					\end{bmatrix}
					$]
					]
					[\ird{ushražá}\\
						reflexive\\
						{Arg = 2}
							[$
							\begin{bmatrix}
								\textrm{+ Patient}\\
								\textbf{+ Agent}
							\end{bmatrix}
							$]
							]
			[\ird{shradébá}\\
				benefactive\\
				{Arg = 3}
				[$
				\begin{bmatrix}
					\textrm{+ Agent}\\
					\textrm{+ Patient}\\
					\textbf{+ Benefactor}
				\end{bmatrix}
				$]
			]
		]
  \end{forest}

	}\caption[Voice markings as valence operations in stative verbs.]{Voice markings as valence operations in stative verbs. The number of elements includes all those required to create a well-formed sentence notwithstanding Iridian's tendency to drop elements that can be implied from context, with the element in bold representing whichever element is most likely to surface in speech.}
  \label{causative-reading}
\end{figure}

We see in figure \ref{causative-reading} that this causative reading of the patientive voice with stative verbs is due to properties of stative verbs and not of the patientive voice. We know this is true since this causative reading of the patientive does not exist with non-stative verbs, which are transitive by default in Iridian.

\pex
\a
\begingl
    \gla \ljudge{*}Mámka prehlavnik.//
    \glb mother buy-\Pv{}-\Pf{}//
    \glft \trsl{*I bought my mother.}//
\endgl
\a
\begingl
    \gla Mámka zuštalnik.//
    \glb mother happy-\Pv{}-\Pf{}//
    \glft \trsl{I made my mother happy.}//
\endgl
\xe

The patientive voice only requires a patient as argument; however since this argument does not exist in stative constructions, the role of an agent must first be created for the subject of the stative construction to be able to occupy the role of the patient in the patientive voice. Essentially this means that conjugating a stative verb for the patientive voice is equivalent to creating a biclausal causative construction where the subject becomes the causee and the state the action brought about by the (optionally named) causer. This reading is not possible with dynamic verbs because the patientive voice would only shift the role of the patient to that of the topic without having to create a new role for an agent.

As could have been predicted from \posscite{haiman1983} theory, these indirect forms of the causative express a more direct link between the causer and the action. True morphological causatives, i.e., those formed using the prefix \ird{ne-}, imply that the caused action was brought about by an intermediary.

\begin{multicols}{2}
\pex
\a
\begingl
\gla Váz nopriznek.//
\glb vase break-\Pv{}-\Pf{}//
\glft \trsl{I broke the vase.} (on purpose)//
\endgl
\a
\begingl
\gla Váz nenopriznek.//
\glb vase \Caus{}-break-\Pv{}-\Pf{}//
\glft \trsl{I made someone break the vase.}//
\endgl
\xe
\end{multicols}

If the intermediary appears in the sentence it can be marked either in the genitive or in the patientive. Marking the causee in the genitive is the \trsl{neutral} configuration; using the patientive case on the other hand forms what can be called a \emph{coercive} causative (\cite{shibatani1990,lehmann2006}), which in Iridian\footnote{We can compare this to a similar distinction between a dative causative (formed with the clitic \emph{ni}) and the accusative causative (formed with \emph{o}) in Japanese\index{Japanese}. \textcite{lehmann2006} calls the former a coercive causative construction while the latter a permissive causative construction. There are two main differences between the Japanese\index{Japanese} and Iridian systems however. First the coercive causative in Iridian also implies that the agent has effective control over the action or the causee or both, something not necessarily expressed by the Japanese\index{Japanese} \emph{o}-form; and second, both the patientive and the genitive forms of the causative in Iridian allow `permissive' readings, as we illustrate later in this section.

More importantly however the genitive form is considered the default or neutral form in Iridian, with the patientive form considered as more `marked.' The patientive is often used for emphasis, with the genitive construction replacing it where possible, especially in spoken Iridian, even in places where the use of the patientve would have been in better order.
}
could imply either of two things: (i) that the act was made without or against the consent of the causee or (ii) the causer had direct control over the action and/or the causee. Such distinction however is not possible if the main verb is in the agentive voice since the patientive marking is reserved for the patient of the verb (and thus marking the causee in the patientive will essentially produce a situation where both the agent and the patient of the verb is marked for the same role, which in this case is the patient.)

\pex
\a
\begingl
\gla Váz Jancie nenopriznek.//
\glb vase Janek-\Gen{} \Caus{}-break-\Pv{}-\Pf{}//
\glft \trsl{(I) made John break the vase.}//
\endgl
\a
\begingl
\gla Váz Janka nenopriznek.//
\glb vase Janek-\Acc{} \Caus{}-break-\Pv{}-\Pf{}//
\glft \trsl{(I) made John break the vase.}//
\endgl
\xe


Nevertheless the degree of control exerted by the causer over the action itself may vary between these constructions.

A common way to formally mark the causer's control or lack thereof in Iridian is the opposition between the retrospective aspect and the perfective aspect. Consider for example the two sentences in Iridian below, both of which have the same general translation in English\index{English}.

\pex
\a
\begingl
	\gla Martin nésta najevec shražek.//
	\glb Martin deer-\Acc{} drive-\Cv{} die-\Av{}-\Pf{}//
	\glft \trsl{Martin ran over a deer.} (He did it on purpose)//
\endgl
\a
\begingl
	\gla Martin nésta najevec shražaní.//
	\glb Martin deer-\Acc{} drive-\Cv{} die-\mk{av-ret}//
	\glft \trsl{Martin ran over a deer.} (It was an accident.)//
\endgl
\xe

\subsection{Reflexive and reciprocal constructions}\index{reflexive construction}\index{reflexive voice}\index{reciprocal construction}

The reciprocative prefix \ird{so-} is used with the agentive voice to indicate that an action is performed by the agent and the patient on each other.

\pex
\begingl
\gla Karlu sodalšar\v{z}ím še Marek ščenžek.//
\glb Karel-\Ins{} \Rec{}-talk-\mk{av-prog-1s} with Marek arrive-\Av{}-\Pf{}//
\glft \trsl{Karel and I were talking when Marek arrived.}//
\endgl
\xe

The use of the reciprocative inherently implies plurality on the part of the subject since there are always at least two elements involved (cf. \cite[255]{tesniere1965}). Since Iridian does not often grammaticalize plurality\index{plural}, this means the reciprocative usually won't require additional consideration as to the agreement of the constituents of the sentence; it does, however, mean that this form cannot be used singly with the singular form of pronouns (since pronouns---at least in the first and second persons---formally distinguish between singular and plural) and that most countable nouns would require the use of the particle \ird{nie} or an explicit quantifier.

\pex
\begingl
\gla To na hruma horka sokonížek.//
\glb \Dem{} \Loc{} church-\Acc{} parents \Rec{}-wed-\Av{}-\Pf{}//
\glft \trsl{(My) parents were married in this church.}//
\endgl
\xe

\pex
\begingl
\gla Nie senátor sožubal\v{z}ime to na televiza vížek.//
\glb \Pl{}= senator \Rec{}-shout-\Av{}-\Prog{} \mk{rz} \Loc{} television-\Acc{} see-\Av{}-\Pf{}//
\glft \trsl{(I) saw the senators shouting at each other on tv.}//
\endgl
\xe

Where both elements of the agent-patient pair are present in the sentence, one of them is treated as the agent and left unmarked while the other is marked in the comitative\index{comitative} (i.e., \ird{še} + instrumental). However, since the action itself is reciprocal, which gets marked as the agent is purely a pragmatic choice. Where one of the members of the agent-patient pair is a pronoun, preference is given to marking the pronoun as the agent (in which case \ird{še} is normally ommitted, but with the patient remaining in the instrumental case).

\pex
\begingl
\gla Mišek še Martinu sohévoržime.//
\glb Mišek \Com{} Martin-\Ins{} \Rec{}-know-\Av{}-\Prog{}//
\glft \trsl{Mišek and Martin know each other.}//
\endgl
\xe

\pex
\begingl
\gla Já Mišku sohévoržaní no?.//
\glb \mk{2s.str} Mišek-\Ins{} \Rec{}-know-\mk{av-ret} \mk{q}//
\glft \trsl{You and Mišek already met each other right?}//
\endgl
\xe



\section{Grammatical mood}\index{mood}\index{grammatical mood|see{mood}}

\subsection{Indicative}

\subsection{Imperative and hortative mood}\label{sec:imp-hort}\index{imperative}\index{hortative}

To form commands\index{commands} and requests\index{requests}, the imperative (glossed \mk{imp}) and hortative (\mk{hort}) moods are used in Iridian.

The imperative is formed by replacing the infinitive ending \ird{-á} with the voice marker and the imperative ending \ird{-ím}. The imperative\index{imperative mood} cannot be negated with the prefix \ird{zá-}; instead, to form a negative command the prohibitive\index{prohibitive mood} mood is used (glossed \mk{proh}), formed with the suffix \ird{-éma} instead of \ird{-ím}.

\begin{table}[h!]
\sffamily\footnotesize
	\caption{Conjugation of the verb \ird{piaštá}\\ in the imperative and probihibitive moods.}
	\label{tbl:imperative}

    \begin{tabu}to 0.7\textwidth{YYY}
         \toprule

         &{\sc imperative}&{\sc prohibitive}  \\
         \midrule

         Agentive &
         \ird{piaščím} &
         \ird{piaščéma}\\

         Patientive &
         \ird{piaštním} &
         \ird{piaštnéma}\\

         Benefactive &
         \ird{piaštébím} &
         \ird{piaštébíma}\\

         Locative &
         \ird{piaštouním} &
         \ird{piaštounéma}\\

         Instrumental &
         \ird{dopiaštouním} &
         \ird{dopiaštounima}\\

         Reflexive &
         \ird{upiaščím} &
         \ird{upiaščéma}\\

         \bottomrule
    \end{tabu}

\end{table}

The imperative\index{imperative mood} is used to issue a direct command and the prohibitive to ``signal a prohibition\index{prohibitive mood}'' (SIL). Verbs in the imperative mood do not require an explicit referent, with the addressee or addressees assumed to be the recipient of the command or prohibition. When the addressee is included, it appears in the vocative case if appearing before the verb or unmarked otherwise.\footnote{A comma is placed between the verb and the addressee if the addressee appears after the verb in the sentence but none if it appears before.} Note that both the imperative and the prohibitive do not distinguish number; thus the same form of the verb will be used when giving a command to multiple addressees and to a single one.

\pex
\begingl
    \gla To hrabním.//
    \glb \Dem{} listen-\mk{pv-imp}//
    \glft \trsl{Listen to this.}//
\endgl
\xe
\pex
\a
\begingl
    \gla To hrabním, Marek.//
    \glb \Dem{} listen-\mk{pv-imp} Marek//
    \glft \trsl{Listen to this, Marek.}//
\endgl
\a
\begingl
    \gla Marku to hrabním.//
    \glb Marek-\mk{voc} \Dem{} listen-\mk{pv-imp}//
    \glft \trsl{Listen to this, Marek.}//
\endgl
\xe

\pex
\begingl
    \gla Papír švirkounéma.//
    \glb paper write-\mk{lv-proh}//
    \glft \trsl{Do not write anything on this sheet of paper.}//
\endgl
\xe

When used with verbal adjectives, the suffixes can attach directly to the root without any need for an explicit marker for voice and the addition of a voice marker will in fact change the meaning of the sentence. (The first two sentences below are rather unhelpful given how morphophonemic changes has rendered the imperative form with the voice marker and the one without of the verb \irdp{slouhatá}{to be quiet} identical, but cases like this are common and merit attention.)

\pex
\a
\begingl
    \gla Nie byló slouháčím.//
    \glb \Pl{}= child be:quiet-\mk{imp}//
    \glft \trsl{Keep quiet, children.}//
\endgl
\a
\begingl
    \gla Nie byló uslouháčím.//
    \glb \Pl{}= child \Refl{}-be:quiet-\mk{av-imp}//
    \glft \trsl{Keep quiet, children.}//
\endgl
\xe

\pex
\a
\begingl
    \gla Pitár zuštalébím.//
    \glb Pitár be:happy-\mk{ben-imp}//
    \glft \trsl{Make Pitár happy!}//
\endgl
\a
\begingl
    \gla Zuštalím.//
    \glb be:happy-\mk{imp}//
    \glft \trsl{Be happy!}//
\endgl
\xe


Due to its directness, the use of the imperative or the prohibitive is
considered impolite in most settings, and is often used only when speaking with
friends, family or children. This distinction does not exist in the written
language, where the imperative is used almost exclusively for these functions.
However in signs that give orders or warnings (i.e., `Stop,' `Do not enter')
where English\index{English} may sometimes use imperative constructions, Iridian uses modal
constructions\index{modality} (cf. \S\,\ref{sec:modality}) as they are not treated
 as direct commands or prohibitions.

\pex
\begingl
    \gla Tak slouhatalneví.//
    \glb here be:quiet-\mk{deb-cont}//
    \glft \trsl{Keep quiet.} \textit{Lit.,} \trsl{One must be quiet here.}//
\endgl
\xe

\pex
\begingl
    \gla Tak zahranéčneví.//
    \glb here enter-\mk{npot-cont}//
    \glft \trsl{Do not enter.} \textit{Lit.,} \trsl{One cannot enter here.}//
\endgl
\xe

In spoken Iridian, it is more common and considered more polite to use the
hortative and the negative hortative forms instead of the direct imperative
or prohibitive.

\begin{table}
    \footnotesize\sffamily
		\caption{Conjugation of the verb \ird{piaštá} in the hortative mood.}
		\label{tbl:hortative}
    \begin{tabu}to 0.7\textwidth{YYY}
         \toprule

         &{\sc hortative}&{\sc neg. hortative}  \\
         \midrule
         Agentive &
         \ird{piaščká} &
         \ird{piaščku}\\

         Patientive &
         \ird{piaštniká} &
         \ird{piaštniku}\\

         Benefactive &
         \ird{piaštébká} &
         \ird{piaštébku}\\

         Locative &
         \ird{piaštómká} &
         \ird{piaštómku}\\

         Instrumental &
         \ird{dopiaštómká} &
         \ird{dopiaštómku}\\

         Reflexive &
         \ird{upiaščká} &
         \ird{upiaščku}\\

         \bottomrule
    \end{tabu}

\end{table}

\pex
\begingl
\gla Mina návilastnika.//
\glb door open-\mk{pv-hort}//
\glft \trsl{Open the door.}//
\endgl
\xe

To further soften command, the expression \ird{am luhninká} (from the hortative
form of the verb \irdp{luhná}{to give thanks}, now obsolete except for this
specific usage) and its equivalent negative form \ird{am luhninku} can be used,
with the main verb marked as a perfective converb.\index{converb}\footnote{Cf. the use of the perfective converb with the \textit{merci de} + infinitive construction in French\index{French}. The use of \ird{am luhninká} presupposes that the action being requested has already been done although in fact it hasn't, for which therefore the speaker is giving thanks. Thus, a simple request like \trsl{Please close the door} is expressed in Iridian as \trsl{May you be thanked for having closed the door.}}

\pex
\begingl
\gla Mina se návilastu am luhninka.//
\glb door \Refl{} open-\mk{cv.pf} because thank-\mk{pv-hort}//
\glft \trsl{Please open the door.}//
\endgl
\xe

The adhortative (\trsl{Let's}) is formed using \ird{lidovká} with the imperfective converb form of the main verb. \ird{Lidovká} can also be used by itself where the main verb may be implied from context, or as a reply to the request if the speaker wants to express agreement or assent.

\pex
\begingl
\gla Piaštiec lidovká.//
\glb eat-\mk{cv.ipf} because thank-\mk{pv-hort}//
\glft \trsl{Please open the door.}//
\endgl
\xe

\subsection{Subjunctive}

The subjunctive mood (glossed \mk{sbj}) is used for actions or events that are not or are not known to be true or factual. The subjunctive is formed using the suffix \ird{-íl}

\begin{table}
	\footnotesize\sffamily
	\caption{Conjugation of the verb \ird{piaštá} in the subjunctive.}
	\begin{tabu}to 0.7\textwidth{YYY}
		\toprule
		&{\sc imperfective} &{\sc perfective}\\
		\midrule
		Agentive	& piaščílá	& piaščíš\\
		Patientive	& piaštnílá		& piaštníš\\
		Benefactive	& piaštébílá		& piaštebíš\\
		Locative	& piaštounílá		& piaštouníš\\
		Instrumental& dopiaštébílá	& dopiaštebíš\\
		Reflexive	& upiaščílá	& upiaščíš\\
		\bottomrule
	\end{tabu}
\end{table}

In addition, the copula has two subjunctive forms, the non-negative \ird{niec} and the negative \ird{vaše}.

Note that the Iridian subjunctive makes neither temporal nor aspectual distinction.

\par The following are some specific uses of the subjunctive mood in Iridian:

\subsubsection{Subjunctive of purpose}

Dependent clauses expressing purpose are marked in the subjunctive and normally end in \irdp{te}{in order to} and \irdp{az}{lest}

\pex
\begingl
\gla Traví prehlavnílá te traumašt stojnik.//
\glb bread-\Gen{} buy-\mk{pv-subj.ipf} {so:that} bakery go-\mk{lv-pf}//
\glft \trsl{(I) went to the bakery to buy some bread.}//
\endgl
\xe

\pex
\begingl
\gla Hreščílá te piaščeví.//
\glb be:alive-\mk{av-subj.ipf} {so:that} eat-\mk{lv-cont}//
\glft \trsl{We eat to live.}//
\endgl
\xe

\pex
\a
\begingl
\gla Se vdinílá az varšek.//
\glb \Refl{} see-\mk{pv-subj.ipf} {lest} leave-\Av{}-\Pf{}//
\glft \trsl{(I) left so as not to be seen.}//
\endgl
\a
\begingl
\gla Vdinílá az varšek.//
\glb see-\mk{pv-subj.ipf} {lest} leave-\Av{}-\Pf{}//
\glft \trsl{(I) left so that (it) may not be seen.}//
\endgl
\xe



\subsubsection{jussive/desiderative}
\par The subjunctive is used in indirect constructions of verbs for issuing orders, commanding, exhorting, etc.
\pex
\begingl
\gla Martin na America žnožíl to čeznašálic.//
\glb Martin \Loc{} America-\Acc{} study-\mk{av-sbj} \mk{rz} want-\mk{av-cont-3s.anim}//
\glft `He wants Martin to study in America.'//
\endgl
\xe

\pex
\begingl
\gla Beatles-že >>Yesterday<< Mark\k{a} zášníl to Tunek dálek.//
\glb Beatles-\Gen{} ``Yesterday'' Marek-\Agt{} sing-\mk{pv-sbj} \mk{rz} Tunek say-\mk{pf}//
\glft `Tunek told Marek to sing.'//
\endgl
\xe

\subsubsection{dubitative}
\par The subjunctive is used with verbs expressing doubt, uncertainty or disbelief.

\pex
\begingl
\gla še //
\glb Beatles-\Gen{} ``Yesterday'' Marek-\Agt{} sing-\mk{sbj} \mk{rz} Tunek say-\mk{pf}//
\glft `Tunek told Marek to sing.'//
\endgl
\xe

\subsubsection{with verbs expressing emotion}

\pex
\begingl
\gla Marek zašníl to Tunek dálek.//
\glb Marek sing-\mk{sbj.ipf} \mk{rz} Tunek say-\mk{pf}//
\glft `Tunek told Marek to sing.'//
\endgl
\xe


\subsubsection{with the conditional mood}
\par The subjunctive is used in the main clause if the verb in the dependent clause is in the conditional \textit{irrealis} mood.

\pex
\begingl
\gla Dá prezident jenem, //
\glb a//
\glft a//
\endgl
\xe

\subsubsection{expressing judgment}

\pex
\begingl
\gla Zavnočilaš to tévét //
\glb respond-\mk{av-sbj.ipf-2s} \mk{rz} important//
\glft \trsl{It is important that you respond.}//
\endgl
\xe

\subsubsection{irrealis}

\subsection{Conditional Mood}\index{conditional mood}\label{sec:conditional}
\par The conditional mood is used for conditional or hypothetical clauses. The table below shows the conjugation paradigm for the conditional mood for both regular verbs and the copula. The Iridian conditional mood is not a true conditional mood grammatically, since it is marked on the verb in the dependent clause (protasis), instead of the main clause.

\begin{table}[h!]
	\small
	\caption{Conjugation paradigm in the conditional mood for regular \\verbs, the copula and the existential particle \ird{ješ}.}\medskip
	\begin{tabu} to 0.9 \textwidth	{Y[1.3]Y[1.3]YY}
		\toprule
		&{\scshape regular verbs} & {\scshape copula} & {\scshape existential}\\
		\midrule

		\textit{Realis} 				&\ird{-ič} &víne & jako\\
		Neg. \textit{Realis}		&\ird{-čnie}&ve&neko\\

		\textit{Irrealis} 			& \ird{-išče}& jenem & jenem\\
		Neg. \textit{Irrealis} 	& \ird{-iščenie}& jet & nét\\
		\bottomrule
	\end{tabu}
\end{table}

\subsubsection{Conditional Realis}

\par The conditional \textit{realis} mood (glossed \mk{cond.rl}) is used in two ways:
\begin{enumerate}
	\item In sentences that express a factual implication rather than a hypothetical situation or a potential future event, e.g., `If you heat water to 100 C, it will boil.'
	\item In `predictive' constructions, i.e., those that concern probable future events.
\end{enumerate}

The conditional \emph{realis} mood requires the verb in the main clause to be in the indicative.


\pex
\begingl
\gla Nebo 100 céntigrádu nékrasébič ustručnaševí.//
\glb water 100 Celcius-\Ins{} \Caus{}-heat-\mk{ben-cond.rl} \Refl{}-boil-\Av{}-\Cont{}//
\glft \trsl{If you heat water to 100 C, it will boil.}//
\endgl
\xe

\pex
\begingl
\gla To projekt hlupnič kurvem započneví.//
\glb this project fail-\mk{pv-cond.rl} job-\First{}\Sg{} lose-\mk{pv-cont}//
\glft \trsl{If we lose this project, I will lose my job.}//
\endgl
\xe

\pex
\begingl
\gla Nahte štanžič upíčeví.//
\glb too:much drink-\mk{av-cond.rl} \Refl{}-get:drunk-\Av{}-\Cont{}//
\glft \trsl{If you drink too much, you will get drunk.}//
\endgl
\xe

\pex
\begingl
\gla Mém na prezna víne, dekání byróva stóžka.//
\glb name \Loc{} list-\Acc{} \mk{cop.cond.rl} dean-\Gen{} office-\Acc{} go-\mk{av-hort}//
\glft \trsl{If your name is on the list, please go to the dean's office.}//
\endgl
\xe

\subsubsection{Conditional Irrealis}
The conditional \textit{irrealis} mood (glossed \mk{cond.irr}) is used with hypothetical, typically counterfactual, events. The \emph{irrealis} mood requires the main clause to be in the subjunctive.

\subsection{Quotative}\label{sec:quotative}\index{quotative mood}
\par The quotative mood (glossed \mk{quot}) is used to express secondhand information, or when the speaker wishes to make explicit that s/he did not witness the event himself/herself. This section deals primarily with the morphological properties of the quotative mood. See \S\,\ref{sec:reportedspeech} for a discussion of the syntactical treatment of reported speech in Iridian.

The quotative mood is considered a \emph{secondary} mood in traditional Iridian linguistics since it can only be used in conjunction with other other grammatical moods. The quotative is normally formed with the suffix \ird{-e} in regular verbs, which suppletes the personal pronoun marking (if there are any) in the indicative mood, or word-finally with other moods, subject to the usual morphophonemic changes. Since it must appear as the final suffix at all times, clitic pronouns\index{clitic pronouns} cannot be used with the quotative.

\subsubsection{With the indicatve mood}

\begin{table}
	\small
	\caption{Sound changes used in deriving quotative form of verbs}
	\medskip
	\label{tbl:quotind}
	\begin{tabu} to 0.9\textwidth {Y[0.7]YY}
		\toprule
										&	{\sc sound change}				& {\sc example}\\
		\midrule
				Perfective 		&
				\ird{-ek} $\rightarrow$ \ird{ice}	&
				\ird{piašček} $\rightarrow$ \ird{piaščice}\\
				Retrospective &
				\ird{-aní} $\rightarrow$ \ird{ánie} &
				\ird{piaščaní} $\rightarrow$ \ird{piaščánie}\\
				Continuous &
				\ird{-eví} $\rightarrow$ \ird{íve} &
				\ird{piaščeví} $\rightarrow$ \ird{piaščíve}\\
				Progressive &
				\ird{-ár} $\rightarrow$ \ird{árže} &
				\ird{piaščár} $\rightarrow$ \ird{piaščárže}\\
				Contemplative &
				\ird{-ách} $\rightarrow$ \ird{áže} &
				\ird{piaščách} $\rightarrow$ \ird{piaščáže}\\
				Prospective &
				\ird{-ujám} $\rightarrow$ \ird{-ujime} &
				\ird{piaščujám} $\rightarrow$ \ird{piaščujime}\\
				Cessative &
				\ird{-óvít} $\rightarrow$ \ird{-óvíče} &
				\ird{piaščóvít} $\rightarrow$ \ird{piaščóvíče}\\

			\bottomrule
	\end{tabu}

\end{table}

Where the addition of the quotative suffix \ird{-e} involves the suppletion of a clitic pronoun, the critic pronoun resurfaces elsewhere in the quoted clause, usually in its strong form. Nevertheless, given Iridian's pro-drop tendency, pronouns both in main clauses and in reported clauses are often left out to be inferred from context.

\subsubsection{Quotative forms of the copula}

%% TODO format as table
Copula
Indicative
neví
hvem
Subj
nehlí
niec

Existential
Indicative
jeho
nežní
Subj
houve
hvaš


\section{Modality}\index{modality}\label{sec:modality}

Iridian can express modality either through verbal morphology, using the affixes listed in table \ref{tbl:modality}, or through a periphrastic construction. In general a periphrastic construction is preferred when the verb is non-dynamic, i.e., the sentence is merely descriptive or stative in nature (compare, for example English\index{English} \trsl{Mary can sing} vs. \trsl{Mary was able to finish baking the cake}), while the morphological method is used otherwise.

\begin{table}[h!]
    \small
    \caption{Verbal affixes to express modality.}
    \medskip
    \label{tbl:modality}
    \begin{tabu}to 0.6\textwidth{YYY}
			\toprule
				 {\sc modality} & {\sc positive} & {\sc negative}\\
				 \midrule
         Debitive & \ird{-aln-} & \ird{-išk-}\\
         Desiderative & \ird{-án-}&\ird{-ušh-}\\
         Potential &\ird{-ét-} & \ird{-évn-}\\
			\bottomrule
    \end{tabu}
\end{table}

The affixes used to mark modality as listed in table \ref{tbl:modality} attach directly to the verb stem, subject to the usual morphophonemic rules.

\pex\a \irdp{piaštá}{to eat}
\a \irdp{piaštalná}{to need to eat}
\a \irdp{piaštišká}{to not need to eat}
\a \irdp{piaštáná}{to want to eat}
\a \irdp{piaštušhá}{to not want to eat}
\a \irdp{piaštétá}{to be able to eat}
\a \irdp{piaštévná}{to not be able to eat}
\xe

As in most languages, modal constructions in Iridian exhibit significant {\sc polysemy}\index{polysemy} (i.e. a single construction can have one or more interpretation depending on the context). For example consider the following sentence:

\pex
\begingl
\gla Tomáš rušku zahviržétách.//
\glb Tomáš Russian-\Ins{} speak-\Av{}-\Pot{}-\Ctp{}//
\glft \trsl{Tomáš will be able to speak Russian}//
\endgl
\xe

The following translations are all equally possible without any further contextual clues:

\pex
\a \trsl{Tomáš will be able to speak Russian, if he will study it.} (abilitative)
\a \trsl{Tomáš will be able to speak Russian because he will be allowed to do it.} (permissive)
\a \trsl{Tomáš can speak Russian and he will probably speak it later.} (true potential modality)
\xe

\subsection{Potential modality}\index{potential modality}\index{abilitative}\index{permissive}\index{modality}

Potential modality (glossed as \mk{pot}) is used when, in the speaker's opinion, an event is possible to occur. This definition makes the potential mood in Iridian encompass both the expressions of ability and permissibility.

\pex
\begingl
\gla To švirek moc gruševí še oštinévnílá.//
\glb this handwriting too be:small-\mk{cont} with read-\mk{pv-npot-sbj.ipf}//
\glft \trsl{The handwriting is too small (I) am unable to read it.}//
\endgl
\xe

\subsection{Debitive modality}\index{debitive modality}\index{modality}

The debitive form of a verb expresses necessity. This verb form is now mainly confined in literary usage, and has been entirely replaced in colloquial Iridian by the supine of necessity.\index{supine} The negative debitive form however has survived and is still in common use. The negated form\index{negation} of the positive debitive (in contrast to the negative form) is also no longer used in colloquial Iridian, and the negative debitive coexists instead with the negated form of the supine of necessity, with subtle differences in meaning.

\begin{multicols}{2}
  \pex
  \a
  \begingl
  \gla Tóm zoštináš.//
  \glb book \Neg{}-read-\mk{pv-sup}//
  \glft \trsl{(We) don't need to read the book.}//
  \endgl
  \a \begingl
  \gla Tóm oštniškeví.//
  \glb book read-\mk{pv-ndeb-cont}//
  \glft \trsl{The book should not be read (i.e., it is prohibited).}//
  \endgl
  \xe
\end{multicols}

\subsection{Periphrastic constructions}

\section{Non-finite verb forms}

\subsection{Infinitive}\label{sec:infinitive}\index{infinitive}

The {\sc infinitive} is a fossilised verb form that was used in Old Iridian\index{Old Iridian} (and arguably in Early Middle Iridian\index{Middle Iridian}) as a verbal noun occupying the topic\index{topic} position in a sentence. In Modern Iridian this use has been completely supplanted by the gerund\index{gerund} and the infinitive is only used as the citation form\index{citation form} of verbs. All infinitives in Iridian end in the vowel \ird{-\'a} and the consonant immediately preceding it is called the verb's thematic consonant.\index{thematic consonant}.

\subsection{Nominalised forms and gerunds}\index{gerund}\index{nominalisation}\label{nom-morph}

Nouns can be routinely derived from verbs and verb phrases using the nominalising suffix \ird{-ou} (glossed as \Nz{}). Linguists generally recognize three types of nominalisation: event nominals, which describe an event the same way the parent verb does, and which could either be (1) simple or (2) complex, with {\sc complex event nominals} (CENs)\index{nominalisation!event nominal}\index{event nominal|see{nominalisation, event nominal}} allowing internal arguments and {\sc simple event nominals} (SENs)\index{nominalisation!event nominal} not; and (3) {\sc resultant nominals}\index{nominalisation!resultant nominal}\index{resultant nominal|see{nominalisation, resultant nominal}}, which describe an event similar but not exactly corresponding to the even described by the main verb (\cite{grimshaw1990}; \cite{moulton2014}). In English\index{English}, for example, where verbs can be nominalised using a variety of derivational affixes or with zero derivation, these types are not distinguished, as we see below:

\pex[interpartskip=0pt]
	\a The examination of the students lasted a long time. \hfill {CEN}
	\a The examination lasted a long time.\hfill {SEN}
	\a The examination was photocopied on green paper.\hfill {RN}\\
	\trailingcitation{(\cite[2]{alexiadou2008})}
\xe

Some verbs in Iridian allow the formation of RNs using the suffix \ird{-ou} and the uninflected verb root (e.g., \irdp{pia\v{s}tou}{food} from \irdp{pia\v{s}t\'a}{to eat}). For the vast majority, however, RNs are produced by lexical suppletion, i.e., the RNs are not morphologically derived (or explicitly so, at least) using the nominalising suffix (see \S~\ref{sec:nomder-verb}). As in English, SENs and CENs are not morphologically distinct in Iridian, and are formed with the suffix \ird{-ou} used in conjunction with the prefix \ird{po(d)-}. We call this form the {\sc gerund}.

In addition to these three types of nominalization introduced in \textcite{grimshaw1990}, Iridian recognises a fourth type, which produces a nominal that refers not to the event itself but one of the event's participants, i.e., one of the verbs arguments. We will call this type a {\sc participant nominal} (PN) (cf. \cite[400-5]{schackow2015}; \cite[297-8]{okuna}).

\pex \a Nominalised forms of \irdp{pia\v{s}tou}{to eat} showing a productive morphological RN:\smallskip\\
		\vtop{\halign{%
			#\hfil& \qquad #\hfil\cr
			\quad Infinitive:		& \irdp{pia\v{s}t\'a}{to eat}\cr
			\quad Morphological RN:	& \irdp{pia\v{s}tou}{food}\cr
			\quad Gerund (SEN/CEN):	& \irdp{popia\v{s}tou}{the act of eating}\cr
			\quad PN:				& \irdp{pia\v{s}\v{c}kou}{the person/thing who ate}\cr
		}}
\a Nominalised forms of \irdp{vad\'a}{to think} showing a defective morphological RN and the alternative lexical RN:\smallskip\\
	\vtop{\halign{%
		#\hfil& \qquad #\hfil\cr
			\quad Infinitive:		& \irdp{vad\'a}{to think}\cr
			\quad Morphological RN:	& \ird{*vadou} (ungrammatical)\cr
			\quad Lexical RN:		& \irdp{vied}{thought (n.)}\cr
			\quad Gerund (SEN/CEN):	& \irdp{povadou}{the act of thinking}\cr
			\quad PN:				& \irdp{vadnikou}{that which was thought}\cr
}}\xe


Event nominals (viz., gerunds) are therefore inherently abstract and active in meaning; in addition, they are also understood to be tenseless and aspectless

Gerunds\index{gerund} have an active meaning. The suffix \ird{-ál}, used to mark the continuous aspect, may be infixed to the gerund to indicate that the action is repetitive.

\pex
\a
\begingl
\gla Jan nidek.//
\glb Jan stand.up-\mk{pf}//
\glft \trsl{Jan stood up.}//
\endgl
\a
\begingl
\gla Janí ponidálou buvec.//
\glb Jan-\Gen{} \mk{ger}-stand.up-\mk{cont-nz} annoying//
\glft \trsl{Jan's standing up again and again is annoying.}//
\endgl
\xe

The syntax of event and participant nominals is discussed in further detail in \S~\ref{sec:nomz-syntax}.

\subsection{Converbs}\index{converb}
Converbs (glossed \Cv{}) are a non-finite verb form often used for adverbial constructions. There are two converb forms in Iridian: the imperfective \ird{iec} (glossed \mk{cv.ipf}) and the perfective \ird{-u} (glossed \mk{cv.pf}).

\pex
\begingl
\gla Tereza kravniec nóví palžek. //
\glb Tereza cry-\mk{cv.ipf} room-\Gen{} leave-\Av{}-\Pf{}//
\glft \trsl{Tereza left the room crying.}//
\endgl
\xe

\pex
\begingl
\gla Nóví palzu Tereza neikravnašek.//
\glb room-\Gen{} leave-\mk{cv.pf} Tereza \mk{incho}-cry-\mk{pf}//
\glft `Having left the room, Tereza started to cry.'//
\endgl
\xe

The syntax of converbial constructions and the specific uses of the perfective and imperfective converb form are discussed in detail in \S\,\ref{converbs-syntax}.


\subsection{Supine}

The {\sc supine}\index{supine} is a non-finite verb form formed used to indicate necessity or purpose. Both usage has a nominal and a non-nominal form (used similar to an adverb or an adjective), giving the supine a total of four forms, as shown below:

\begin{table}[h!]
	\small
	\caption{Endings used for the supine.}
	\medskip
	\begin{tabu} to 0.6\textwidth{YYY}
		\toprule
		&{\sc purpose}&{\sc necessity}\\
		\midrule
		Nominal & \textit{-it} & \textit{-áš}\\
		Non-nominal & \textit{-ice} & \textit{-ášce}\\
		\bottomrule
	\end{tabu}
\end{table}

These four forms are invariable. The endings attach to the verb after the root has been conjugated for voice. The use of the non-nominal forms, in addition, does not require the use of the linking particle \ird{ko}.

\pex
\begingl
\gla >>Ána Karenina<< za gnaža oštinášce tóm.//
\glb Anna Karenina for school-\Acc{} read-\mk{pv-sup} book//
\glft `I have to read \textit{Anna Karenina} for school.'//
\endgl
\xe

Although the usage of the supine has evolved to include various other constructions not related to its origins as a verbal noun indicating motion, the supine is still used in Modern Iridian in this original sense, accompanying a main verb to indicate purpose. Both the nominal and the non-nominal form can be used in this construction, with the nominal form (despite being a more recent syntactic innovation) being more common and the non-nominal form considered more archaic, but still more prevalent in literary and formal usage. This usage roughly corresponds to the English infinitive, as in the sentence \trsl{I came here \emph{to bury} Cæsar.} When using the nominal form the clause containing the main verb is first transformed into a \ird{to-}clause and then equated to the nominal supine; when using the non-nominal form, on the other hand, the supine is simply added before the main verb.

\pex
\a
\begingl
\gla Tóm behlenik to oštnit.//
\glb book buy-\Pv{}-\Pf{} \Rz{} read-\Pv{}-\Sup{}//
\glft `I bought the book to read.'//
\endgl
\a
\begingl
\gla Tóm oš\v{c}ice behlenik.//\deftagex{supine}\deftaglabel{lit}
\glb book read-\Av{}-\Sup{} buy-\Pv{}-\Pf{}//
\glft `I bought the book to read.'//
\endgl
\xe

Especially when using the non-nominal construction, the grammatical voice used for the supine does not need to be the same as the one used in the main verb, as we see in example (\getfullref{supine.lit}). Indeed it is quite common to use the active voice for the supine regardless of what voice has been used in the main verb. The genitive is used to mark the object of the supine.

\pex
\begingl
\gla T\v{e}\v{z}\'i probem\'i vedice sto\v{z}ek.//
\glb god-\Gen{} sepulchre-\Gen{} see-\Av{}-\Sup{} go-\Av{}-\Pf{}//
\glft \trsl{She went to see the Lord's sepulchre.}//
\endgl
\xe

In addition to this original usage, and to their use in indicating purpose or necessity, the supine\index{supine} is quite heavily employed idiomatically in Iridian. In colloquial speech, the supine of purpose is often used to express future or probable events as a substitute to the contemplative aspect. In both colloquial and literary registers, it may also be used to indicate a habitual action or a general truth (instead of the continuous or progressive aspect) when the verb implies some sort of purpose or consequentiality, especially in relation to another verb.

\pex
\begingl
\gla Dá to t\'om\'i oščit.//
\glb \First{}\Sg{} this book-\Gen{} read-\mk{av-sup}//
\glft `I will be reading this book.' \emph{Lit.,} `I am someone whose purpose is the reading of this book.'//
\endgl
\xe

\pex
\begingl
\gla Méva dousa ješ me bylu dnou má nemel toha ohlečit.//
\glb all adult-\Acc{} \Exst{} as child-\Ins{} front but few:people \Dem{}.\Prox{}.\Acc{} remember-\Av{}-\Sup{}//
\glft \trsl{All grown-ups were children once but only a few remember it.}//
\endgl
\xe

Another common construction involves the supine of necessity with the words \irdp{shlac}{now} \irdp{mál}{time} (or less frequently \irdp{ór}{hour}). This construction is somehow similar to English\index{English} \trsl{It's time we left} or \trsl{It's time for us to go.} When used this way, the supine is conjugated in the locative voice.\index{supine}

\pex
\begingl
\gla Shlac himatí palzounášce mál.//
\glb now homeland-\Gen{} leave-\mk{lv-sup} time//
\glft `It's time (we) left our homeland.'//
\endgl
\xe
\pex
\begingl
\gla Sa tet. Shlac zdalounášce mál.//
\glb already noon now have:breakfast-\mk{lv-sup} time//
\glft `It's already late (\emph{lit.,} noon). It's time (we) had breakfast.'//
\endgl
\xe

\section{Stative verbs}\index{verbal adjective|see{stative verb}}\index{stative verb}\index{adjectives}\label{sec:statives}

Iridian lacks a distinct class of adjectives.\footnote{There is however a small class of attributives, which includes deictics\index{deictics} and quantifiers\index{quantifiers} among others, which can function as modifiers. They are different in that these words cannot be used as the predicate\index{predicate} of a sentence. They are discussed in detail on Chapter \ref{chap:minor}.} Instead, a special class of verbs called {\sc stative verbs} are used to modify noun or noun-like classes. Unlike most verbs, however, stative verbs can only be marked for aspect, and optionally for voice. In addition to this base form (called the {\sc copulative}), stative verbs also have an {\sc attributive} form (used when the verb is preceding the noun or noun phrase) and {\sc nominative} form (representing a concrete nominalization of the verb), both of which are absent in non-attributives verbs. Consider for example the verb \ird{všihná} \trsl{to be angry}:

\begin{table}[h!]
	\small
	\caption{Conjugation pattern for stative verbs}
	\medskip
	\begin{tabu} to 0.7\textwidth{YY[0.8]Y}
		\toprule
		&{\sc ending}&{\sc example}\\
		\midrule
		Copulative & varies & varies\\
		Attributive & \ird{-í} & \ird{všihní}\\
		Nominative & \ird{-ou}	& \ird{všihnou}\\
		\bottomrule
	\end{tabu}
\end{table}

\subsection{Copulative and attributive forms}\index{stative verb}
The copulative form of stative verbs is used when the verb is the predicate of the sentence. This form is only conjugated for aspect, and optionally for voice. Unlike normal verbs, however, stative verbs cannot be conjugated in the agentive voice since Iridian grammar does not distinguish between agency in an actor and the description of a state in stative verbs, both of which are encoded in the definition of this class. 

\ex
\begingl
\gla Mamka všihneví \emph{(not} *všihnaševí\emph{)}//
\glb mother-\mk{dim} be:angry-\mk{cont} not be:angry-\Av{}-\Cont{}//
\glft \trsl{My mother is angry.}//
\endgl
\xe


The attributive form is derived by replacing the infinitive marker \ird{-á} with \ird{-í}. Other than its conjugated comparative form ending in \ird{-ení}, the attributive form is invariable. The comparative form\index{comparative construction}\index{comparison} is often used, especially in colloquial speech, as an intensifier, even if the stative verb is not actually used in a comparison.

\ex
\begingl
\gla Všihnení mamka t\'el\'evoniržek.//
\glb be:angry-\mk{comp-att} mother-\mk{dim} call-\Av{}-\Pf{}//
\glft \trsl{Mother was fuming (\emph{lit.,} angrier) when she called us.}//
\endgl
\xe

Because of the invariability of the attributive form, the copulative form may sometimes be used as a modifier, similar to a normal verb, separated from the noun it modifies with the particle \ird{ko}. Note, however, that when conjugated in the continuous aspect (except when marked explicitly for voice), such usage is not grammatical, with Iridian only allowing the attributive.

\ex
\begingl
\gla Všihninek ko tieho snov uprožilzách.//
\glb be:angry-\Pv{}-\Pf{} \Att{} god soon \Refl{}-avenge-\mk{av-ctpv} //
\glft \trsl{God whom you have angered will seek vengeance soon.}//
\endgl
\xe


\subsection{Nominative form}\index{stative verb}
The nominative form is derived by replacing the infinitive marker \ird{-á} with the nominalizing suffix \ird{-ou}. This is the same nominalizing suffix used to form nouns from regular verbs, the only difference being that stative verbs allow the suffix to be attached directly on the verb's root.

The copulative form may also be nominalised with \ird{-ou}.\index{nominalisation} However, as with the attributive form, if the copulative stative verb is conjugated in the continuous aspect and is unmarked for voice, the nominal form is used instead of the nominalised copulative form.

\subsection{Stative verbs and voice}\index{voice}\index{stative verb}

In general, stative verbs can also be conjugated for voice, with two main differences: first, as mentioned earlier in this section, the agentive voice cannot be used with stative verbs as Iridian does not distinguish between stative and agentive verbs and such information is considered to be encoded by default in the stative form; and second, in view of the first point, the benefactive gains an ``agentive'' interpretation, as it is used when the subjective is the agent of the action leading to the state being described by the verb, as in the example below:

\ex
\begingl
\gla Zuštal\'ebkou houba.//
\glb be:happy-\mk{ben-pf-nz} gift//
\glft \trsl{What made me happy was (your) gift.}//
\endgl
\xe



\section{Derivational morphology}
\subsection{External derivation}
\par Loanwords ending in \textbf{-ace} from the Latin change the final e to á:
\begin{table}[h!]
	\centering \small
	\begin{tabu} to 0.9\textwidth{>{\bfseries}YM[0.3]>{\bfseries}YY}
		administrace 	& $\rightarrow$ & administracá 	& `to administrate' \\
		akuzace			& $\rightarrow$ & akuzacá		& `to accuse'\\
		diferenzace		& $\rightarrow$ & diferenzacá	& `to differentiate'\\
		separace		& $\rightarrow$ & separacá		& `to separate'\\
	\end{tabu}
\end{table}
\par Some Latin loanwords are borrowed first from German. Loanwords ending in \textbf{-ieren} become \textbf{-irná}.
\begin{table}[h!]
	\centering \small
	\begin{tabu} to 0.9\textwidth{>{\bfseries}YM[0.3]>{\bfseries}YY}
		akzeptieren 	& $\rightarrow$ & akceptirná 	& `to accept' \\
		konservieren	& $\rightarrow$ & koncervirná	& `to conserve'\\
		produzieren		& $\rightarrow$ & producirná	& `to produce'\\
		vandalieren		& $\rightarrow$ & vandalirná 	& `to deface'\\
	\end{tabu}
\end{table}
\subsection{Internal Derivation}
\begin{center}
	\small
	\begin{longtabu}to \textwidth{Y[0.5]Y}

		\caption{Verbal Derivational Affixes}
		\label{verbalder}                             \\
		\toprule
		\multicolumn{1}{c}{\sc affix} & \multicolumn{1}{c}{\sc examples}                      \\
		\midrule
		\endfirsthead
		%---------------------------------------------------------------%
		\caption{Verbal derivational affixes \hfill\textit{(continued)}}            \\
		\toprule
		\multicolumn{1}{c}{\sc affix} & \multicolumn{1}{c}{\sc examples}                      \\
		\midrule
		\endhead
		%---------------------------------------------------------------%
		\bottomrule \addlinespace
		\multicolumn{2}{r}{\footnotesize\textit{continued on the next page}}
		\endfoot

		\bottomrule
		\endlastfoot

		\textbf{nie-} + {\sc adj}\newline`to cause something to become \mk{adj}' &

		\textbf{loš} `new' $\rightarrow$ \textbf{nielošá} `to renew' \newline
		\textbf{preseh} `young' $\rightarrow$ \textbf{niepreshá} `to rejuvenate' \newline
		\textbf{avic} `long' $\rightarrow$ \textbf{nieavicá} `to lengthen' \newline
		\textbf{gem} `soft' $\rightarrow$ \textbf{niegemá} `to soften'\newline
		\textbf{vyne} `dry' $\rightarrow$ \textbf{nievyneá} `to dry'\\ \addlinespace

		\textbf{ce-}\footnote{Verbs in \textbf{ce-} cannot be in the reflexive focus.} + {\sc adj}\newline `to cause oneself to become {\sc adj}' &

		\textbf{kdavidy} `clean' $\rightarrow$ \textbf{cekdavicá} `to take a bath' \newline
		\textbf{rum} `old' $\rightarrow$ \textbf{cerumá} `to grow old' \newline
		\textbf{šeznom} `big' $\rightarrow$ \textbf{cešeznomá} `to grow up' \newline
		\textbf{vyne} `dry' $\rightarrow$ \textbf{cevyneá} `to dry oneself'\\ \addlinespace

		\textbf{hó-} + {\sc noun}\newline `to use {\sc n} in a particular way' &

		\textbf{tvem} `tongue' $\rightarrow$ \textbf{hótvemá} `to lick' \newline
		\textbf{kov} `hammer' $\rightarrow$ \textbf{hóková} `to hammer' \newline
		\textbf{šeznom} `big' $\rightarrow$ \textbf{cešeznomá} `to grow up' \newline
		\textbf{vyne} `dry' $\rightarrow$ \textbf{cevyneá} `to dry oneself'\\ \addlinespace

		\textbf{deš-} + {\sc noun}\newline `to act in the manner of {\sc n}  &

		\textbf{tvem} `tongue' $\rightarrow$ \textbf{hótvemá} `to lick' \newline
		\textbf{rum} `old' $\rightarrow$ \textbf{cerumá} `to grow old' \newline
		\textbf{šeznom} `big' $\rightarrow$ \textbf{cešeznomá} `to grow up' \newline
		\textbf{vyne} `dry' $\rightarrow$ \textbf{cevyneá} `to dry oneself'\\ \addlinespace

		\textbf{má-iv} + {\sc noun}\newline `to so something usually done in {\sc noun}'  &

		\textbf{mrc} `market' $\rightarrow$ \textbf{mámrcivá} `to shop' \newline
		\textbf{gnazsa} `school' $\rightarrow$ \textbf{mágnazsivá} `to study in'  \newline
		\textbf{šeznom} `big' $\rightarrow$ \textbf{cešeznomá} `to grow up' \newline
		\textbf{vyne} `dry' $\rightarrow$ \textbf{cevyneá} `to dry oneself'\\ \addlinespace


		\textbf{sen-/sem-} + {\sc verb}\newline `to {\sc verb} incorrectly'  &

		\textbf{oštá} `to read' $\rightarrow$ \textbf{senoštá} `to misread' \newline
		\textbf{rum} `old' $\rightarrow$ \textbf{cerumá} `to grow old' \newline
		\textbf{šeznom} `big' $\rightarrow$ \textbf{cešeznomá} `to grow up' \newline
		\textbf{vyne} `dry' $\rightarrow$ \textbf{cevyneá} `to dry oneself'\\ \addlinespace
	\end{longtabu}
\end{center}
			% Verbs
\chapter{Nominal Morphology}

Nominal morphology in Iridian is relatively simpler compared to the corresponding process with verbs. Where possible, Iridian sentences are generally constructed to leave the noun or noun phrase unmarked.

\section{Grammatical Categories}

\section{Number}\index{grammatical number}\index{plural}

Nouns in Iridian are not formally marked for number. Thus the word \ird{byl}, for example, can mean either \trsl{child} or \trsl{children} depending on the context. The same form is used when the noun is preceded by a numeral.

\pex
\begingl
\gla hron\'a byl//
\glb three child//
\glft \trsl{three children}//
\endgl
\xe

Nevertheless, Iridian can express semantic plurality by using quantifiers, numerals, pluralizing particles or even through context alone. One such particle is \ird{nie}\label{sec:plurals}\footnote{Cf. Schachter's treatment of Tagalog pluralizing particle \emph{mga}.}. \ird{Nie} is a proclitic and attaches to the left-most part of the noun phrase or the verb phrase it modifies.

\pex
\begingl
    \gla nie \v{s}a zu\v{s}tal\'i byl//
    \glb \mk{pl=} \mk{dem.prox} be:happy-\mk{att} child //
    \glft \trsl{these happy children}//
\endgl
\xe

\ird{Nie} however could be understood to have three distinct uses. The first, as mentioned above, is to mark plurality. Alternatively, \ird{nie} could also be use as an approximative\index{approximation} (roughly equivalent to English \trsl{about}) when used with cardinal numbers or time expressions or as a honorific expletive\index{honorific}\index{expletive} to show politeness when used with proper names\index{proper names} or with some nouns (mostly related to kinship terms\index{kinship terms}). In its use for approximation, \ird{nie} is interchangeable with \irdp{u}{about}, although it is common in spoken speech to combine the two as an intensified construction. Preference is given to \ird{nie}, however, if the noun being modified is the topic of the sentence and must therefore remain unmarked.

\pex
\begingl
    \gla Nie mlaz-no scen\v{z}ek?//
    \glb \mk{hon=} brother\mk{=q} arrive-\mk{av-pf} //
    \glft \trsl{Was my brother the one who arrived?}//
\endgl
\xe
\pex
\begingl
    \gla Nie mlaz-no scen\v{z}ek?//
    \glb \mk{hon=} brother\mk{=q} arrive-\mk{av-pf} //
    \glft \trsl{Was my brother the one who arrived?}//
\endgl
\xe

\pex
\a
\begingl\deftagex{appr}\deftaglabel{1}
    \gla Nie hron\'a byl//
    \glb \mk{approx=} three child //
    \glft \trsl{about three children}//
\endgl
\a
\begingl
    \gla u hron\'a bylu//
    \glb about three child-\mk{inst} //
    \glft \trsl{about three children}//
\endgl
\a
\begingl
    \gla u nie hron\'a bylu//
    \glb about \mk{approx=} three child-\mk{inst}//
    \glft \trsl{about three children}//
\endgl
\xe

Note that when used with a cardinal number, \ird{nie} can only be understood to signify approximation, i.e., (\getfullref{appr.1}) can only mean \trsl{about three children} and not \trsl{three children}, as the latter would only be translated as \ird{hron\'a byl} without the clitic \ird{nie}.

As has been earlier mentioned, \ird{nie} is a proclitic\index{proclisis} and attaches to the left-most part of the noun phrase or verb phrase it modifies, including any modifier no matter how complex but excluding any proposition. In some cases, as can be seen in (b) and (c) below, the use of \ird{nie} to pluralize a noun can imply definiteness\index{definiteness}.

\pex
\a
\begingl\deftagex{pl}\deftaglabel{1}
    \gla \textbf{nie} za byla t\'om//
    \glb \mk{pl=} for child-\mk{pat} child //
    \glft \trsl{books for children}//
\endgl
\a
\begingl\deftaglabel{2}
    \gla za \textbf{nie} byla t\'om//
    \glb for \mk{pl=} child-\mk{pat} child //
    \glft \trsl{a book for (these) children}//
\endgl
\a
\begingl\deftaglabel{3}
    \gla \textbf{nie} za \textbf{nie} byla t\'om//
    \glb \mk{pl=} for \mk{pl=} child-\mk{pat} child //
    \glft \trsl{books for (these) children}//
\endgl
\xe





The use of \ird{nie}, however, is largely optional and where plurality can be implied from context, this particle is seen as redundant and is therefore dropped.

\pex
\begingl
\gla Nie byl zap\'o\v{c}ek.//
\glb \mk{pl} child laugh-\mk{av-pf}//
\glft \trsl{The children jumped.}//
\endgl
\xe

\ird{Nie} cannot be used with mass and uncountable nouns, as well as with abstract nouns.

\pex
\a
\begingl
\gla *Na duma nie je\v{s} pia\v{s}tou.//
\glb \mk{loc} house \mk{pl} \mk{exst} food//
\glft \trsl{There is food in the house.}//
\endgl
\a
\begingl
\gla Na duma tohle je\v{s} pia\v{s}tou.//
\glb \mk{loc} house much \mk{exst} food//
\glft \trsl{There is a lot of food in the house.}//
\endgl
\xe

The particle \ird{nie} always precedes the noun it modifies, except in existential clauses where it comes before the existential particle \ird{je\v{s}}\footnote{The sequence is pronounced as if written n\'ije\v{s} \nt{"ni:jEC}}. \ird{Nie} can obviously not be used with the negative particle \ird{niho}.\index{niho}\index{existential construction}\index{je\v{s}}

\pex
\a
\begingl
\gla nie b\v{z}\k{e}//
\glb \mk{pl} bee//
\glft \trsl{bees}//
\endgl
\a
\begingl
\gla Nie je\v{s} b\v{z}\k{e}.//
\glb \mk{pl} \mk{exst} bee//
\glft \trsl{There are bees.}//
\endgl
\a
\begingl
\gla *Nie niho b\v{z}\k{e}.//
\glb \mk{pl} \mk{exst.neg} bee//
\glft \trsl{There are no bees.}//
\endgl
\xe

\index{pluralia tantum}
\ird{Nie} cannot be used with a limited number of nouns, mostly referring to paired body parts and related objects, which in the base form is understood to refer to the pair itself and thus cannot be pluralized. If the speaker wishes to explicitly refer to one piece of the pair, the noun \ird{noma} (an obsolete form of the word for one-half, now surviving only in this construction) and the genitive form of the body part.

\pex
\begingl
\gla Eg zaromnek.//
\glb eyes close-\mk{pv-pf}//
\glft \trsl{(He) closed (his) eyes.}//
\endgl
\xe
\pex
\begingl
\gla Poh\'ar d\'evit.//
\glb eyeglasses dirty//
\glft \trsl{(Your) eyeglasses are dirty.}//
\endgl
\xe
\pex
\begingl
\gla Ohv\'i noma utie\v{s}\v{c}\'al.//
\glb shoe-\mk{gen} half \mk{ref-}lose-\mk{av-cont}//
\glft \trsl{The other pair of (his) shoe is missing.}//
\endgl
\xe

The base form is also used in generic statements where English would normally use the plural.\index{generic statements}\index{universals}


When used with a proper noun \ird{nie} can be translated with the English construction \trsl{and others}. Note that this is different from the usage of \ird{nie} as a honorific.

\pex
\begingl
    \gla Nie Jancie gna\v{z} uprub\'i\v{z}ice.//
    \glb \mk{pl=} Janek-\mk{gen} school \mk{ref}-burn-{av-pf-quot} //
    \glft \trsl{I heard Janek's school burned down.}//
\endgl
\xe

\pex
\begingl
    \gla Nie Marek z\'azdal\v{s}ek..//
    \glb \mk{pl=} Marek \mk{neg}-have:breakfast-{av-pf} //
    \glft \trsl{Marek and the others did not eat breakfast.}//
\endgl
\xe


\section{Definiteness}
Iridian does not have definite or indefinite articles

\section{Uninflected form}

\section{Agentive case}\index{agentive case}

\subsection{Agentive of comparison}\index{comparison}\index{agentive of comparison}
\pex
\begingl
\gla D\'a Mark\k{a} t\'am stroja.//
\glb \mk{1s.str} Marek-\mk{agt} \mk{comp} tall//
\glft \trsl{Marek is taller than me}//
\endgl
\xe

\section{Patientive case}

The patientive case (glossed \mk{pat}) is formed by appending the suffix \ird{-a} to the root of the noun, subject to the following sound changes, notably affecting vowel-final roots for the most part:

\begin{itemize}
	\item Roots ending in e and o replace the final vowel with \ird{-a}: \ird{pivo -- piva} \trsl{beer}, \ird{malno -- malna} \trsl{language}, \ird{\v{s}uze -- \v{s}uza} \trsl{judge}
	\item Roots ending in \'o and ou replace the final vowel with \ird{-\'ova}: \ird{pia\v{s}tou -- pia\v{s}t\'ova} \trsl{food}, \ird{jav\'o -- jav\'ova} \trsl{lizard}, \ird{metr\'o -- metr\'ova} \trsl{subway}
	\item Roots ending in a lengthen the final vowel to \ird{-\'a}: \ird{cigra -- cigr\'a} \trsl{tiger}, \ird{husa -- hus\'a} \trsl{street}
	\item Roots ending in \'a replace the final vowel with \ird{\'anie}: \ird{kom\'a -- kom\'anie} \trsl{boat}, \ird{vietr\'a -- vietr\'anie} \trsl{pants}
	\item Roots ending in \'e, ei and i replace the root with \ird{-\'ena}: \ird{k\'av\'e -- k\'av\'ena} \trsl{coffee}, \ird{matei -- mat\'ena} \trsl{motorbike}
	\item Roots ending in \'i append \ird{na}:
	\item Roots ending in u or \'u append \ird{-\v{s}a}:
\end{itemize}

\subsection{Direct object}
The patientive case is used to mark the direct object of a verb that is in the agentive voice. Note that this usage implies that the direct object is indefinite unless the noun is further qualified (except through a demonstrative).

\pex
\a
\begingl
\gla Va\v{s}ka pia\v{s}\v{c}em.//
\glb cake-\mk{pat} eat-\mk{av-pf-1s}//
\glft \trsl{I ate cake.}//
\endgl
\a
\begingl
\gla Jed\'a va\v{s}ka pia\v{s}\v{c}em.//
\glb that cake-\mk{pat} eat-\mk{av-pf-1s}//
\glft \trsl{I ate from that cake.}//
\endgl
\a
\begingl
\gla Va\v{s}ko pia\v{s}tnikem.//
\glb cake eat-\mk{pv-pf-1s}//
\glft \trsl{I ate the cake.}//
\endgl
\a
\begingl
\gla Jed\'a va\v{s}ko pia\v{s}tnikem.//
\glb that cake eat-\mk{pv-pf-1s}//
\glft \trsl{I ate that cake.}//
\endgl
\a
\begingl
\gla Hron\'a va\v{s}ke vat\'a pia\v{s}\v{c}em.//
\glb three cake-\mk{gen} slice-\mk{pat} eat-\mk{pv-pf-1s}//
\glft \trsl{I ate three slices of cake.}//
\endgl
\xe

The patientive is also used to mark the direct object when the verb is in the benefactive voice.

\pex
\begingl
\gla \v{S}a vitamina pia\v{s}tebik.//
\glb \mk{3s.anim} vitamin-\mk{pat} eat-\mk{ben-pf}//
\glft \trsl{(She) made him take (his) vitamins.}//
\endgl
\xe

%%%%
% TODO Definiteness and the patientive; use of genitive when the noun marked is indefinite


\subsection{Locative}

The patientive is used with the particle \ird{na} to form a compound locative case, which is itself used to indicate a general location.

\pex
\begingl
\gla Tom\'a\v{s} na byra.//
\glb Tom\'a\v{s} \mk{loc} office-\mk{pat}//
\glft \trsl{Tom\'a\v{s} is at the office.}//
\endgl
\xe

\subsection{Patientive of purpose}

The patientive is used with the particle \ird{za} to indicate

\subsection{Lative}
The lative is a compound case indicating movement into or to the direction of something. It is formed using the particle \ird{de} and a noun or noun phrase in the patientive case.

\subsection{Adessive}
The adessive is formed when the particle \ird{u} is used with the patientive. This compound case indicates that the noun being modified by the noun in the adessive is near or in the vicinity of the noun in the adessive. The adessive case behaves synactically in the same manner as the locative case with na in all cases.

\pex
\begingl
\gla Tom\'a\v{s} u byra.//
\glb Tom\'a\v{s} \mk{ade} office-\mk{pat}//
\glft \trsl{Tom\'a\v{s} is somewhere near the office.}//
\endgl
\xe

The adessive case is also used to approximate time.

\pex
\begingl
\gla Ova\v{z} u 19 \'ora.//
\glb dinner \mk{ade} 19 hour-\mk{pat}//
\glft \trsl{Dinner is around seven.}//
\endgl
\xe

\section{Genitive Case}\index{genitive}

The genitive (glossed \mk{gen}) is formed by appending the suffix \ird{-e} to the root of a noun.

Due the palatalizing nature of the suffix, the following sound changes must be noted:

\begin{itemize}
	\item Roots ending in k, h, and t change the final consonant to c and append the glide \ird{-ie} instead: \ird{Marek -- Marcie} \trsl{Marek}, \ird{avt -- avcie} \trsl{car}, \ird{duh -- ducie} \trsl{head}
	\item Roots ending in d and g change the final consonant to \v{z} and append the suffix \ird{-e} instead: \ird{vod -- vo\v{z}e} \trsl{sister}, \ird{seg -- se\v{z}e} \trsl{flower}
	\item Roots ending in the sibilants s, z, \v{s}, \v{z} and the sibilant affricates c and \v{c} append \ird{e} as well:
	\item Roots ending with a palatalized consonant lose the final y (there only for orthographic reasons in any case) before appending the \ird{-\'i}: \ird{kra\v{s}toly -- kra\v{s}tol\'i}
	\item Roots ending in a or o replace the vowel with e, while those ending in \'a and \'o replace the root with \'i
	\item Roots ending in au, ou, or u replace the vowel with -\'ov\'i: \ird{dnou -- dn\'ov\'i} \trsl{front}
	\item Roots ending in \'au, or \'u replace the vowel with -\'ovie
	\item Roots ending in e, i or \"y replace the vowel with -ev\'i
	\item Roots ending in \'e, ei, \'i or \'y replace the vowel with -\'ev\'i
\end{itemize}


\subsection{Genitive of Possession}\index{possesive}\index{genitive}

The simplest use of the genitive case is to indicate ownership or possession.
When used this way, the noun marked in the genitive must always procede the noun
it modifies.

\pex
\irdp{Marcie dum}{Marek's house}\\
\irdp{m\'amcie ha\v{s}ek}{my mother's bag}\\
\irdp{\v{s}a \v{s}tudencie t\'om}{this bb}
\xe

Demonstratives\index{demonstrative} and other modifers must always come before
the whole noun phrase and cannot split the possessor from the possessee. An
exception to this rule is the clitic \ird{nie}, which comes immediately before
the noun it pluralizes\index{plural}.

\pex
\a  \irdp{\v{s}a \v{s}tudencie t\'om}{the/a book of this student}\\
    \irdp{to \v{s}tudencie t\'om}{this book of the student}
\a  \irdp{nie \v{s}tudencie t\'om}{the students' book}\\
    \irdp{\v{s}tudencie nie t\'om}{the student's books}
\xe

\subsection{Partitive Genitive}\index{partitive}\index{genitive}

\subsection{Genitive of material}

\pex
\begingl
\gla kun\'i prosc//
\glb silver\mk{gen} spoon//
\glft \trsl{silver spoon}//
\endgl
\xe

\subsection{Genitive of the whole}
The genitive can also be used to indicate

\pex
\begingl
\gla na kra\v{s}tol\'i dn\'ova//
\glb \mk{loc} train:station-\mk{gen} front//
\glft \trsl{in front of the train station}//
\endgl
\xe

Note that the patientive and not the genitive case is used when quantifying a part of the whole.

\pex
\a
\begingl
\gla *\v{z}nohou\v{s}ce hron\'a//
\glb student-\mk{gen} three//
\glft \trsl{three of the students}//
\endgl
\a
\begingl
\gla na \v{z}nohou\v{s}ca hron\'a//
\glb \mk{loc} student-\mk{gen} three//
\glft \trsl{three of the students}//
\endgl
\xe

Nevertheless when quantifying a noun per se, and not in relation to a whole, the uninflected form of the quantifier is used (mostly using indefinite quantifiers such as \trsl{many}, \trsl{a lot}, etc.). If however, the quantification involves a countable unit or division of the noun, the genitive is used, but such unit or division must be further quantified by a numeral or an indefinite quantifier.

\pex
\a
\begingl
\gla Na krouma\v{s}ta po zma je\v{s} pivo.//
\glb \mk{loc} refrigerator-\mk{pat} still few \mk{exst} beer//
\glft \trsl{There's still some beer left in the refrigerator.}//
\endgl
\a
\begingl
\gla Ona pive \v{s}tava unar\'i\v{z}\v{c}em.//
\glb one beer-\mk{gen} mug-\mk{pat} \mk{ref-}order-\mk{av-pv-1s}//
\glft \trsl{I ordered a mug of beer.}//
\endgl
\xe

\subsection{Genitive of movement}

The genitive is also used to indicate movement away from somewhere.

\pex
\a
\begingl
\gla Dum\'i pal\v{z}ek.//
\glb house-\mk{gen} leave-\mk{av-pf}//
\glft \trsl{I left the house.}//
\endgl
\a
\begingl
\gla Dum palzinek.//
\glb house leave-\mk{pv-pf}//
\glft \trsl{I left the \emph{house}.}//
\endgl
\xe

\section{Instrumental case}

The instrumental case (glossed \mk{inst})

\subsection{With some prepositions}

The following prepositions take the instrumental case: \ird{\v{s}e} \trsl{with}

\pex
\begingl
\gla Za bolta \v{s}e Janu st\'o\v{z}\k{a}c.//
\glb for party-\mk{pat} with Jan-\mk{inst} go-\mk{av-ctpv}//
\glft \trsl{(I am) coming to the party with Jan.}//
\endgl
\xe

\subsection{With expressions of time and duration}

\section{Vocative Case}


\section{Unmarked Form}


\section{Personal Pronouns}\index{personal pronouns}\index{pronouns}

Personal pronouns are a special class of nouns used to refer and/or replace other nouns or noun phrases. Personal pronouns are marked for person, number and case, and partially for animacy\index{animacy}, although third-person forms are more properly analyzed as demonstratives. In addition, personal pronouns have three forms: (1) an invariable strong form, used when the pronoun is the topic of the sentence; (2) a weak form; and (3) a clitic form.

\begin{table}[h!]
	\caption{Personal pronouns in Iridian}
	\centering\small
	\begin{tabularx}{0.8\textwidth}{>{\scshape}YMMM}

		\toprule
		\multicolumn{1}{c}{\textsc{person}} &\textsc{strong} &\textsc{weak}&\textsc{clitic}\\
		\midrule
		1s &dá&do&-em\\ \addlinespace
		2s&já&je&-e\v{s}\\ \addlinespace
		3s.anim&\v{s}a&\v{s}e&-ic\\ \addlinespace
		3s.inan&to&cej&-as\\ \addlinespace
		4gen&á&dien&-u\v{c}\\ \addlinespace
		1pl.inc&m\'e&chce&-uh\\ \addlinespace
		1pl.exc&tov\'a&kiec&-ak\\ \addlinespace
		2pl&t\'evit&la&-elý\\ \addlinespace
		3pl.anim&o\v{z}e&dcá&-ac\\ \addlinespace
		3pl.inan&\'ima&oce&-et\\ \bottomrule
	\end{tabularx}
\end{table}

\subsection{Grammatical person}\index{person, grammatical}
Iridian pronouns
\subsection{Strong form}\index{strong form}

The strong form of a personal pronoun (glossed \mk{str}) is used when the pronoun is used as the topic of the sentence. The strong form is indeclinable.

\subsection{Weak form}

\subsection{Clitic form}\index{clitic form}

\subsection{Third-Person Pronouns and Demonstratives}


\subsection{Ellipsis}
Iridian is an extremely pro-drop language, with pronouns supplied only if not inferrable from context.


\section{Demonstratives}\index{demonstratives}

\begin{table}
	\small\centering
	\caption{Demonstrative pronouns in Iridian.}
	\begin{tabu}to 0.7\textwidth{YMMM}
		\toprule
						& {\sc animate}	& {\sc inanimate}	&{\sc locative}\\
		\midrule \addlinespace
		Proximal		& \v{s}a		& to 				& tak\\ \addlinespace
		Medial			& \'on				& j\'an				& jen\'i\\ \addlinespace
		Distal			& dn\'i		& j\'on				& jon\'i\\ \addlinespace
		\bottomrule
		\label{dem-prons}
	\end{tabu}
\end{table}

Iridian does not have a separate class of third-person pronouns. Instead it uses a set of demonstrative pronouns, whose deictic\index{deixis} function is both spatial\index{spatial deixis|see{deixis}} and anaphoric\index{anaphora}. Iridian makes a three-way distinction among demonstratives, similar to French or Portuguese for example, distinguishing between proximal (near the speaker), medial (near the addressee) and distal (far from both speaker and addressee) forms. In addition, Iridian makes an animacy distinction with demonstratives, with one set of demonstratives used with human referents and another with non-human referents, as seen in Table \ref{dem-prons}, but are unmarked for either number or gender.

Demonstratives can be used adnominally, to modify a noun phrase, or pronominally, to replace one.

\pex
\a
\begingl
\gla \v{s}a byl//
\glb \mk{dem.prox.anim} child//
\glft \trsl{this child}//
\endgl
\a
\begingl
\gla \v{s}a bylem//
\glb \mk{dem.prox.anim} child-\mk{1s}//
\glft \trsl{this child of mine}//
\endgl
\a
\begingl
\gla \v{S}a bylem.//
\glb \mk{dem.prox.anim} child-\mk{1s}//
\glft \trsl{This (person) is my child.}//
\endgl
\a
\begingl
\gla *To bylem//
\glb \mk{dem.prox.inan} child-\mk{1s}//
\glft \trsl{This (thing) is my child.}//
\endgl
\xe


Unlike true personal pronouns, demonstratives do not have a separate strong form and clitic form. They are fully declined however, with the declined forms being highly irregular, as can be seen in Table \ref{dem-conj}.

\pex
\a\deftagex{obv}
\begingl
\gla ci mlaz a dn\'i maty//
\glft \trsl{this person's brother and that person's mother}//
\endgl
\a\deftaglabel{obv1}
\begingl
\gla D\'a je svou je dnu zapr\'al.//
\glft \trsl{I am as old as either this person or that person.}//
\endgl
\xe

\begin{table}
\footnotesize\sffamily
	\caption{Declension of demonstratives.}
	\begin{tabu}to 0.9\textwidth{Y[1.5]YYYYYY}
		\toprule
						& {\v{s}a}	& {\'on}	&{dn\'i}& {to}	& {j\'an}	&{j\'on}\\
		\midrule \addlinespace
		Agentive&\v{s}em&n\'am&dniem&etom&j\'an&j\'on\\\addlinespace
		Patientive&\v{s}\'a&ona&dn\'a&toha&jina&jin\'ova\\\addlinespace
		Genitive&ci&on\'i&dn\'i&cie&nie&nohe\\\addlinespace
		Instrumental&svou&nu&dnu&etu&nu&nohu\\\addlinespace
		\bottomrule
		\label{dem-conj}
	\end{tabu}
\end{table}

The three-way distinction between demonstratives allows Iridian to disambiguate between an obviative\index{obviation} third person and a proximate third person, using the distal and the proximal demonstrative respectively. Consider for example the two examples in English below:

\pex
\a He saw his dog.
\a He saw his own dog.\smallskip
\xe

The \emph{his} in the first sentence is ambiguous, as it can refer to either the subject or an implied fourth person. That the second \emph{his} refers back to the subject can be made unequivocal by the addition of the word \emph{own}, as in the second sentence. Compare this with the following sentences in Czech:

\pex
\a
\begingl
\gla Vid\v{e}l jeho pes.//
\glft \trsl{He saw his dog.}//
\endgl
\a \begingl
\gla Vid\v{e}l sv\'e pes.//
\glft \trsl{He saw his own dog.}//
\endgl
\xe

Although the English translation of the first sentence may still appear ambiguous, we can see that Czech does away with the ambiguity by using the third person pronoun \ird{jeho} exclusively to signify that the referent is different from the subject, and requiring the use of a separate pronominal form (in this case the reflexive) when the referent and the subject are the same. Iridian, on the other hand, treats this in a diametrically opposite way, i.e., the same pronoun form is used when the subject and the referent are the same, with the obviative form being used otherwise. The sentences in Czech above will therefore be translated in Iridian as follows:

\pex
\a
\begingl
\gla Dn\'i jec vdinek.//
\glb \mk{dem.dist.anim.gen} dog see-\mk{pv-pf}//
\glft \trsl{He saw his (other person's) dog.}//
\endgl
\a \begingl
\gla Ci jec vdinek//
\glb \mk{dem.dist.inan.gen} dog see-\mk{pv-pf}//
\glft \trsl{He saw his own dog.}//
\endgl
\xe

Perhaps we can better understand the distinction between obviative and proximate forms by re-examining example (\getfullref{obv.obv1}) above. The previous examples in Czech remained unambiguous because there are at most two unique arguments in the sentence. In example (\getfullref{obv.obv1}), however, the subject of the sentence is distinct from either the proximate referent or the distal referent.

\ex[exno={\getfullref{obv.obv1}}]
\begingl
\gla D\'a je svou je dnu zapr\'al.//
\glft \trsl{I am as old as either this person or that person.}//
\endgl
\xe

The translation in the gloss demonstrates how idiomatic English uses periphrastic forms to eliminate this ambiguity, although in the spoken language the purely deictic \trsl{I am as old as either him or him} is equally acceptable, with the blanks filled in most likely by non-verbal cues. In Iridian, however, this distinction is not optional, and the following sentence, for example, would be considered ungrammatical:

\ex
\begingl
\gla *D\'a je svou je svou zapr\'al.//
\glft \trsl{I am as old as either him or him.}//
\endgl
\xe

\section{Use of Personal Pronouns}

\subsection{T-V Distinction}\index{politeness}\index{T-V distinction}\index{forms of address}

Iridian has three forms of address: the informal, the polite, and the formal.

The second person singular pronoun \ird{j\'a} is used to address friends, relatives or children. When addressing a stranger or an acquaintance with whom you want to maintain social distance or be polite without being too formal, the second person plural pronoun \ird{t\'evit} is used. The polite form is also used when addressing God/gods. In more formal settings, the third-person plural pronoun \ird{o\v{z}e} is used.



\section{Possessive Pronouns}

\subsection{The reflexive \ird{m\'am}}

\section{Demonstratives}\index{demonstratives}\index{demonstrative pronouns}

Iridian has a three-way distinction between demonstratives, unlike English but similar to Spanish or Japanese: \emph{proximal} demonstratives are used when referring to objects or people that are near the speaker, \emph{medial} demonstratives when referring to those near the listener, and \emph{distal} demonstratives when referring to those that are far from either the listener or speaker.






\begin{table}[h!]
	\small\centering
	\caption{Conjugation of Iridian demonstrative pronouns.}
	\begin{tabu}to 0.7\textwidth{YMM}
		\toprule
						& {\sc animate}		& {\sc inanimate}\\
		\midrule
		Proximal		& \v{s}a			& to\\ \addlinespace
		Medial			&&j\'an\\ \addlinespace
		Distal			&&j\'on\\ \addlinespace
		\bottomrule
	\end{tabu}
\end{table}

For information about demonstrative adjectives/determiners, see section \ref{dem-adj}.

\section{Indefinite pronouns and quantifiers}


\section{Interrogative pronouns}\index{wh- questions}\index{interrogative pronouns}

\begin{table}[h!]
	\small\centering
	\caption{Interrogative pronouns in Iridian.}
	\begin{tabu} to 0.8\textwidth{>{\bfseries}YY>{\bfseries}YY}
		\toprule\addlinespace
		&{\sc english}&&{\sc english}\\ \addlinespace
		\midrule\addlinespace
		jede 		& who &jach &which\\ \addlinespace
		je\v{z}e 	& what 		& zajehu 	&why\\ \addlinespace
		jeh\'at 	& whom		& jik\'a 	&how many\\ \addlinespace
		jehu 		& how		&ji\v{s}k\'a&how much\\ \addlinespace
		jem\'i 		& when 		& jenie 	&to where\\ \addlinespace
		jena 		& where 	& jen\'i 	&from where\\ \addlinespace
		\bottomrule
	\end{tabu}
\end{table}

\section{Negative and Universal Pronouns}\index{negative pronouns}\index{universal pronouns}

Negative pronouns are historically formed by attaching the prefix \ird{\v{z}e} before interrogative pronouns, and universal pronouns by attaching the prefix \ird{n\'i-}

\begin{table}[h!]
	\small\centering
	\caption{Correspondence of interrogative, negative and universal pronouns.}
	\begin{tabu} to \textwidth{>{\bfseries}YY[1.2]>{\bfseries}YY[1.2]>{\bfseries}YY[1.2]}
		\toprule\addlinespace
		\multicolumn{2}{c}{\sc interrogative}& \multicolumn{2}{c}{\sc negative} & \multicolumn{2}{c}{\sc universal}\\ \addlinespace
		\midrule\addlinespace
		jede 		& who & nei\v{z}e & no one & niet & everyone\\ \addlinespace
		je\v{z}e 	& what 		& niho & nothing&n\'i\v{z}e&everything\\ \addlinespace
		jehu 		& how		&\v{z}ehu&by no means&n\'ehu&by all means\\ \addlinespace
		jem\'i 		& when 		& \v{z}emie&never&nimie&always \\\addlinespace
		jena 		& where 	& \v{z}ena&nowhere&nina&everywhere \\ \addlinespace
		jach &which&\v{z}\'e&not one&n\'ach&each\\ \addlinespace
		\bottomrule
	\end{tabu}
\end{table}




\section{Derivational Morphology}

\subsection{-ma\v{s}t}

\begin{table}[h!]
	\centering\small
	\caption{Nominal derivation using \ird{-ma\v{s}t}}
	\begin{tabu} to \textwidth{YYY[0.5]YY}
		\toprule
		\multicolumn{2}{c}{\sc root}&&\multicolumn{2}{c}{\sc derived noun}\\
		\addlinespace
		\midrule
		\ird{k\'av\'e}&\trsl{coffee}&$\rightarrow$& \ird{k\'av\'ema\v{s}t} &\trsl{caf\'e}\\
		\ird{krou}&\trsl{cold}&$\rightarrow$& \ird{krouma\v{s}t} &\trsl{refrigerator}\\
		\ird{pia\v{s}tou}&\trsl{food}&$\rightarrow$& \ird{pia\v{s}touma\v{s}t} &\trsl{restaurant}\\

		\bottomrule

	\end{tabu}

\end{table}

\subsection{-ou}
The nominalizing suffix \ird{-ou} is a non-productive affix used to form nouns from certain verbs.

\begin{table}[h!]
	\centering\small
	\caption{Nominal derivation using \ird{-ou}}
	\begin{tabu} to \textwidth{YYY[0.5]YY}
		\toprule
		\multicolumn{2}{c}{\sc verb root}&&\multicolumn{2}{c}{\sc derived noun}\\
		\addlinespace
		\midrule
		\ird{milovan\'a}&\trsl{to learn}&$\rightarrow$& \ird{milovanou} &\trsl{lesson}\\
		\ird{palz\'a}&\trsl{to leave}&$\rightarrow$& \ird{palzou} &\trsl{departure}\\
		\ird{pia\v{s}t\'a}&\trsl{to eat}&$\rightarrow$& \ird{pia\v{s}tou} &\trsl{food}\\
		\ird{scen\'a}&\trsl{to arrive}&$\rightarrow$& \ird{scenou} &\trsl{arrival}\\
		\ird{niek\'a}&\trsl{to open}&$\rightarrow$& \ird{niekou} &\trsl{entrance}\\

		\bottomrule

	\end{tabu}

\end{table}

\subsection{-ou\v{s}c}
The suffix \ird{-ou\v{s}c} (pronounced as if written \ird{-\'o\v{s}t} \bt{o:St}, or in some dialects as \ird{-ou\v{s}t} \nt{\dto{}St}) is used to form a noun indicating someone or something associated to a certain thing or performing a certain action.

\begin{table}[h!]
	\centering\small
	\caption{Nominal derivation using \ird{-ou\v{s}c}}
	\begin{tabu} to \textwidth{YYY[0.5]YY}
		\toprule
		\multicolumn{2}{c}{\sc verb root}&&\multicolumn{2}{c}{\sc derived noun}\\
		\addlinespace
		\midrule
		\ird{jork\'a}&\trsl{to travel}&$\rightarrow$& \ird{jorkou\v{s}c} &\trsl{traveller}\\
		\ird{mo\v{z}l\'a}&\trsl{to live}&$\rightarrow$& \ird{mo\v{z}lou\v{s}c} &\trsl{resident}\\
		\ird{umiel\'a}&\trsl{to get drunk}&$\rightarrow$& \ird{um\'ilou\v{s}c} &\trsl{drunkard}\\
		\ird{virk\'a}&\trsl{to write}&$\rightarrow$& \ird{virkou\v{s}c} &\trsl{writer}\\
		\ird{zdiev\'a} &\trsl{to fool (sm.)}&$\rightarrow$& \ird{zd\'ivou\v{s}c} &\trsl{swindler}\\
		\bottomrule

	\end{tabu}

\end{table}
 			% Nouns
\chapter{Minor Word Classes}\label{chap:minor}\index{minor word classes}

\section{Prepositions}

\subsection{na}

\subsection{še}

\subsection{\ird{vo}}\index{vo}\index{agentive case}

\ird{Vo} can be translated as \trsl{because of} or \trsl{due to.} This preposition takes the agentive case.

\pex
\begingl
\gla Vo transitám lienu záscenzčem.//
\glb because traffic-\Agt{} on:time-\Ins{} \Neg{}-arrive-\mk{av-pf-1s}//
\glft \trsl{I didn't arrive on time because of the traffic.}//
\endgl
\xe

\subsection{za}


\section{Demonstratives}\label{dem-adj}\index{demonstratives}

\section{Quantifiers}\index{quantifiers}
Iridian has a wide variety of non-numerical/indefinite quantifiers.  Most are actually nouns that used in adjectival or adverbial constructions.


\begin{itemize}
    \item \ird{ošč} \trsl{many} (countable)
    \ex
    \begingl
    \gla Marka ješ naže ošč.//
    \glb Marek-\Pat{} \Exst{} friend-\Gen{} many//
    \glft \trsl{Marek has many friends.}//
    \endgl
    \xe
    \ex
    \begingl
    \gla Za kursa mén ješ ošč oudinášce ko vilm.//
    \glb for class-\Pat{} \mk{1pl.inc.wk} \Exst{} many watch-\Sup{} \Att{} film.//
    \glft \trsl{We have a lot of movies we need to watch for our class.}//
    \endgl
    \xe
    \item \ird{nave} \trsl{too many} (countable)
    \ex
    \begingl
    \gla Marka ješ naže ošš.//
    \glb Marek-\Pat{} \Exst{} friend-\Gen{} many//
    \glft \trsl{Marek has many friends.}//
    \endgl
    \xe
    \item \ird{tohle} \trsl{many} (uncountable)
    \item \ird{nahte} \trsl{too many, too much} (uncountable)
    \ex
    \begingl
    \gla Do ješ nahte kurváš//
    \glb \mk{1s.wk} \Exst{} too:much work-\mk{sup.nom}//
    \glft \trsl{I have so much work to do.}//
    \endgl
    \xe

\end{itemize}

\section{Interjections}

An interjection\index{interjection} is a word or an expression used to express a spontaneous reaction or feeling. We will use the term `interjection' to refer both to the part of speech and to the utterance type that has the same pragmatic function as this part of speech (cf. \cite{ameka1992}).

Interjections can be classifed into two main categories: \emph{primary} interjections, which refer to a word or an utterance that can only be used as an interjection and \emph{secondary} interjections, which refer to forms belonging a different word class but which through its usage, has acquired a new meaning as an interjection.

Although interjections can function as exclamations, not all exclamatory utterances can be considered as interjectons by themselves. As \textcite{jovanovic2004} notes, any word in a language can theoretically become an exclamation. Consider for example this conversation:

\ex (adapted from \cite{jovanovic2004}).\\

  \ird{
  \noindent--- Martin mlaza boulešik.\\
  --- \textbf{Martinám?}
  }\medskip

  \trsl{I heard Martin killed his brother.}\\
  \trsl{Martin?!}
\xe


\section{Discourse Particles}

\subsection{Yes and No}
Iridian has several words for yes and no but their usage in responding to yes-no questions does not exactly align with that of English. This is discussed in detail in \S\,\ref{sec:ansyn}.

There are two main words for \trsl{yes} in Iridian: the affirmative \ird{dé} (\trsl{Did you see it?} \trsl{Yes, I did.}) and the contrastive \ird{če} (\trsl{Did you not see it?} \trsl{Yes, I did.}. The distinction is similar as that between the French \emph{oui} and \emph{si}. Both \ird{dé} and \ird{če} generally appear at the end of a sentence. In colloquial spoken Iridian it is also common to see the form \ird{ja} (most likely from the Czech, and ultimately from the German \emph{ja}) and the more informal \ird{jó}. These forms however are not cliticized to the verb and appear at the start of a sentence, set off from the rest with a commma. Both \ird{ja} and \ird{jó} cannot be used contrastively like \ird{če}. It is also common to use both \ird{ja/jó} at the same time as \ird{dé}.

\pex
\begingl
\gla ---To vdinice? ---Ja vdinek dé.//
\glb this see-\Pv{}-\Pf{}-\Quot{} yes see-\Pv{}-\Pf{} yes//
\glft \trsl{{}``Did you see it?'' ``Yes, I did.''{}}//
\endgl
\xe

When used by themselves, both \ird{ja} and \ird{jó} are often repeated twice or thrice (e.g., \ird{Ja ja ja.})\footnote{Commas are not used to separate each \ird{ja} or \ird{jó} in standard orthography. } even when the usage is not emphatic. \ird{Dé} and \ird{če} cannot be used this way.

\section{Phatic Expressions}\label{sec:phatic expression}\index{phatic expression}

\subsection{Greetings}

\subsection{Apologizing}



\section{Numerals}
\par Iridian has a vigesimal number system. Table \ref{one20} shows Iridian numerals from 1 to 20. Numbers from 1 to 10 are given their own name while numbers from 11 to 19 are formed by appending the numbers from one to nine to the clitic \ird{-niem} with the preposition \ird{še} (with). The clitic \ird{-niem} is derived from the word for number 10, \ird{nau}, which itself comes from the Old Iridian \rec{nagu}, `half.'
\begin{table}[h!]
		\caption{Iridian numerals from 1 to 20.}
		\medskip
		\small

\begin{tabu}to 0.8 \textwidth {Y[0.7]YY[0.7]Y}
	\toprule
	{\sc number} & {\sc iridian} & {\sc number} & {\sc iridian}\\
	\midrule
	1 & ona			& 11 & onšeniem\\
	2 & m\"y			& 12 & myšeniem\\
	3 & hroná		& 13 & hronašeniem\\
	4 & drou			& 14 & drušeniem\\
	5 & jed			& 15 & jecniem\\
	6 &	vou			& 16 & vušeniem\\
	7 & šč\k{e}	& 17 & šč\k{e}ceniem\\
	8 & pieš		& 18 & pi\k{e}ceniem\\
	9 & cam			& 19 & camzeniem\\
	10& nau			& 20 & tydná\\

	\bottomrule
	\label{one20}
\end{tabu}
\end{table}

For numbers 11 to 19, the words are formed by appending the numbers from one to nine to the suffix \textit{-niem} with the preposition \irdp{še}{with}.

Numbers from 21 to 99 are first expressed as multiples of 20. Thenceforth, the number system has largely become decimal, due primarily to the influence of surrounding Indo-European languages. Old Iridian, however, had a vigesimal system up to the number 8000.

Table \ref{3099} shows multiples of 10 from 30 to 100. The numbers are formed by the numeral followed by \ird{tydná}. For bases that are not multiples of 20, the word \irdp{nau}{ten} is added first, followed by the conjunction \irdp{še}{with}.

\begin{table}[h!]
	\caption{Iridian numerals from 30 to 100.}
	\medskip
	\small
	\begin{tabu}to 0.9 \textwidth {Y[0.5]YY[0.5]Y}
		\toprule
		{\sc number} & {\sc iridian} & {\sc number} & {\sc iridian}\\
		\midrule
		30 &	naušetydná		& 70 	& naušehronutydná\\
		40 &	mytydná		& 80	& drohutydná\\
		50 &	naušemytydná	& 90	& naušedrohutydná\\
		60 &	hronutydná		& 100	& miesy\\
		\bottomrule
		\label{3099}
	\end{tabu}
\end{table}

Iridian counting starts from the smallest component of the number to the largest. Each component can be simply appended with the conjunction \ird{še}. Only the numerals in Tables \ref{one20} and \ref{3099}, and the first ten numbers after 100, 500, 1000, etc. appear as single words. Below are some illustrations:

\pex
\a \ird{jecemiesy}\\
	\trsl{five with hundred,} i.e., 105
\a \ird{cam še drohutydná}\\
	\trsl{nine with four twenties,} i.e., 89
\a \ird{pi\k{e}ceniem še hronutydná}\\
	\trsl{eighteen with three twenties,} i.e., 78
\xe

\begin{table}[h!]
	\caption{Iridian numerals from 200 to one billion.}
	\medskip
	\small
	\begin{tabu}to 0.9 \textwidth {Y[0.6]Y}
		\toprule
		{\sc number} & {\sc iridian} \\
		\midrule
		200 			&	mach	\\
		300, 400, etc.	& 	hronumiesy, drohumiesy. etc.\\
		1000			& 	nic\\
		2000, 3000, etc.& 	myniec, hronuniec, etc.\\
		10.000			&	ohle\\
		20.000, etc.	& 	tydnuniec, etc.\\
		100.000			&	hazlek\\
		200.000 etc		&	mehdeniec, hronuniec, etc.\\
		1.000.000		&	miliám\\
		1.000.000.000	&	milár\\
		1.000.000.000.000	& biliám\\
		\bottomrule
		\label{3099}
	\end{tabu}
\end{table}

\subsection{Ordinal numbers}
Except for the first three cardinal numbers that have irregular ordinal forms, ordinals are mostly regular, formed with the suffix \ird{-šle} (or \ird{-išle} after consonants). The ordinal form of the numbers one, two and three are \ird{hezka}, \ird{dviec} and \ird{cehra}, respectively. When written as numerals, a full stop is used as in German (e.g., \irdp{camišle}{ninth} would be written 9.).

The letter n has its own ordinal form (cf. English \trsl{nth} for example), \ird{enišle}, as do the rest of the other letters. These ordinal forms are generally regular. Their usage is confined to mathematical literature, however, with the clear exception of \ird{enišle}, which is often used idiomatically (cf. French \textit{pour la eni\`eme fois}).

\subsection{Fractions and decimals}

As with most languages in Europe, Iridian uses the comma (Iridian \ird{kvá}) to separate whole number from decimals. Numbers after the comma are read in pairs of two, with the first number read separately in case there is an odd number of numerals after the comma (e.g., 3,34 is read as \ird{hroná kvá drušeniem še tydná} while 3,346 is read \ird{hroná kvá hroná vou še mytydná}). If there are seven or more numbers following the come, each is read separately instead.

Fractional forms are also regularly formed using the suffix \ird{-izmek}. The word for half, \ird{niet}, however is irregular. Fractional forms are sometimes used together with the regular decimal forms when dealing with currency. For example, 5,50 kr. can be read as either \ird{jed kvá naušemytydná korun} or more commonly \ird{jed še niet korun}.

\subsection{Date and Time}
Dates are written with the year first, followed by the month, and ultimately by the date. When written in numerals, the numbers are separated by a full stop. When spoken or when written in full, the number representing the year is followed by the word \irdp{hlet}{year}, often in the instrumental case. When followed by the name of the month, \ird{hlet} is declined in the genitive. When the date is included, the ordinal form is used, followed by the word \irdp{ráz}{day,} although the latter may be dropped in casual speech. The inclusion of the date also requires the name of the month to be in the genitive case.

\pex
\a
\begingl
    \gla 1992 hletí julí 15. rázu veštašik //
    \glb 1992 year\mk{-gen} july-\Gen{} 15th day-\Ins{} be:born-\Av{}-\Pf{}//
    \glft \trsl{I was born on 5 July 1992.}//
\endgl
\xe

\begin{table}[h!]
	\caption{Months of the year.}
	\medskip
	\small
	\begin{tabu}to 0.7 \textwidth {YYYY}
		\toprule
		{\sc month} & {\sc iridian} & {\sc month} & {\sc iridian}\\
		\midrule
		January		& jenvár	& July & jul\\
		February	& fevrár 	& August & augošt\\
		March		& merc		& September & seitembár\\
		April		& april 	& October & oktobár\\
		May 		& mai 		& November & novembár\\
		June 		& jón 	& December & dicámbár\\
		\bottomrule
		\label{3099}
	\end{tabu}
\end{table}

			% Minor Word Classes
\chapter{Syntax of Simple Clauses}

\section{Introduction}

The constituent word order of Iridian sentences is SOV, but the agglutinative nature of the language and the presence of case-marking on nouns makes word order typically flexible, with the only universal rule being that the main verb should appear at the end of a sentence.

\section{Topic-Predicate Constructions}\index{topic}\index{predicate}\label{sec:topic-pred}
The Iridian sentence can be divided primarily into a topic part and a predicate
or comment part. The topic is what the sentence is about, while the predicate or comment represents the information presented in the sentence about the topic. While both the topic and the predicate are pragmatic constructs, the topic-predicate construction is important as it determines how the rest of the sentence is structured.


\begin{figure}[H]
  \begin{forest}
    [S,
      [{\sc top}] [{\sc pred}]]
  \end{forest}
  \caption{Nuclear structure of sentences}
  \label{}
\end{figure}

The topic of the sentence does not necessarily coincide with the subject of the sentence. This is true as well in English, as we see in example (\ref{ex:engtop}); although where English allows the topic to appear anywhere in the sentence, as long as the subject is placed first, Iridian, typical of topic-prominent languages. requires the topic to always be introduced first, leaving the rest of the information afterwards.


\pex\label{ex:engtop}
\a Martha saw John.
\a A dog bit \emph{Martha}.
\a It is raining \emph{today},
\xe


\pex

\a
\begingl
\gla \relax[Janek]\tss{\mk{Top}} [mlaza boulešik.]\tss{\mk{Pred}}//
\glft \trsl{As for Janek, he killed his brother}.//
\endgl

\a
\begingl
\gla \relax[Tereza]\tss{\mk{Top}} [jecám nalečnik.]\tss{\mk{Pred}}//
\glft \trsl{As for Tereza, she was bitten by a dog}//
\endgl

\a
\begingl
\gla \relax[Shléd]\tss{\mk{Top}} [zniepšalí.]\tss{\mk{Pred}}//
\glft \trsl{As for today, it is raining.}//
\endgl

\xe


More importantly, the topic of the sentence determines how the main verb, and thus all the other constituents of the sentence, are marked.

\pex
\a
\begingl
\gla Tereza jec\textbf{ám} naleč\textbf{n}ik.//
\glft \trsl{As for Tereza, she was bitten by a dog}//
\endgl

\a
\begingl
\gla Jec Terez\textbf{e} nalč\textbf{eš}ik.//
\glft \trsl{As for the dog, it bit Tereza.}//
\endgl

\xe

As \textcite[9]{kiss2004} notes:

\begin{quote}
  We tend to describe eventsfrom a human perspective, as statements about theirhuman participants – and subjects are more often {\sc[+human]} than objects are. Inthe case of verbs with a {\sc[–human]} subject and a {\sc[+human]} accusative or obliquecomplement,  the  most  common  permutation  is  that  in  which  the  accusative  oroblique complement occupies the topic position\,[.] When the possessor is theonly human involved in an action or state, the possessor is usually topicalized[.]
\end{quote}

\section{The Noun Phrase}

Iridian is a strongly head-final language.

\subsection{Nuclear constructions}
\subsection{With adjectival clauses}
\subsection{Wsith prepositional phrases}
\subsection{With relative clauses}

\section{Topicless Sentences}\label{sec:topicless}\index{topicless sentence}

\section{Definiteness}\index{definiteness}\label{sec:definiteness}

Iridian lacks a specific class of articles\index{articles} such as English
\trsl{a} or \trsl{the} to mark the opposition between definite and indefinite
nouns. For example, the word \ird{jec} can mean both \trsl{a dog} or \trsl{the
dog} depending on the context (or in some environments the same word can be
interpreted as \trsl{dogs,} \trsl{some dogs} or \trsl{the dogs}).

A common way to specificy the definiteness of a noun is to promote it to the
topic position in the sentence. As discussed in \S\,\ref{sec:topic-pred}, the
topic of a sentence must be specific and referential, and therefore it is often,
but not always, definite. Consider for example the two sentences below.

\begin{multicols}{2}
  \pex
  \a
  \begingl
  \gla Pitár pižmo.//
  \glb Pitár farmer//
  \glft \trsl{Pitar is \textbf{a} farmer.}//
  \endgl
  \a
  \begingl
  \gla Pižmo Pitár.//
  \glb farmer Pitár//
  \glft \trsl{Pitár is \textbf{the} farmer.}//
  \endgl
  \xe
\end{multicols}

This can be extended to non-copular constructions.

\begin{multicols}{2}
  \pex
  \a
  \begingl
  \gla Vliče štanžice.//
  \glb milk-\Gen{} drink-\Av{}-\Pf{}-\Quot{}//
  \glft \trsl{(I) drank some milk.}//
  \endgl
  \a
  \begingl
  \gla Vliko štanimce.//
  \glb milk drink-\mk{pv-pf-quot}//
  \glft \trsl{(I) drank the milk.}//
  \endgl
  \xe
\end{multicols}

If the topic is quantified\index{quantifier} by a numeral\index{numeral}, indefiniteness can be expressed by nominalizing\index{nominalization} the main verb and promoting it to topic.

\pex
\a
\begingl
\gla Jaro okrád za propozica niebidček.//
\glb five district for proposal-\Pat{} vote:against-\Av{}-\Pf{}//
\glft \trsl{The five districts voted against the proposal.}//
\endgl
\a
\begingl
\gla Za propozica niebidečkou jaro okrád.//
\glb for proposal-\Pat{} vote:against-\Av{}-\Pf{}-\Nz{} five district//
\glft \trsl{Five districts voted against the proposal.}//
\endgl
\a
\begingl
\gla Za propozica niebidečkou ko okrád jaro.//
\glb for proposal-\Pat{} vote:against-\mk{av-pf-nz} \mk{rz} district five//
\glft \trsl{Five is the number of districts that voted against the proposal.}//
\endgl
\xe

The number one (\ird{oní})

\pex
\a
\begingl
\gla Tóm onaževí.//
\glb book be:lost-\mk{cont}//
\glft \trsl{The book is missing.}//
\endgl
\a
\begingl
\gla Oní tóm onaževí.//
\glb one book be:lost-\mk{cont}//
\glft \trsl{One of the books is missing.}//
\endgl
\a
\begingl
\gla Onaživou pní tóm.//
\glb one book be:lost-\mk{cont}//
\glft \trsl{One of the books is missing.}//
\endgl
\a
\begingl
\gla Onaživou pní tóm.//
\glb one book be:lost-\mk{cont}//
\glft \trsl{One of the books is missing.}//
\endgl
\xe


Note that this rule is not universal and the topic of a sentence does not necessarily have to be definite, especially where the sentence is merely expressing a fact or a general truth:

\pex
\begingl
\gla Jec hvárem.//
\glb dog animal//
\glft \trsl{Dogs are animals.}//
\endgl
\xe



\pex
\begingl
\gla To >>jec<< hvárem že: to robot//
\glb \mk{dem} dog animal \mk{ncop} \mk{dem} robot//
\glft \trsl{The ``dog'' is not a real animal but a robot.}//
\endgl
\xe


\section{Relative and Comparative Constructions}\label{relativecomparative}\index{comparative constructions}

The clitic\index{clitic} \ird{tám} is used to form simple comparative and relative constructions. \ird{Tám} is often ommitted however where the comparison can be implied from context. In this construction, the standard of comparison (the noun preceded by `than' in English) is unmarked and the noun being compared marked in the agentive if it is a positive/negative comparison, or in the instrumental if it is a correlation.


\pex
\a
\begingl
\gla Janek-tám Markám nestačál.//
\glb Janek=\mk{comp} Marek-\Agt{} tall-\mk{cont}//
\glft \trsl{Marek is taller than Janek.}//
\endgl
\a
\begingl
\gla Janek Markám nestačál.//
\glb Janek Marek-\Agt{} tall-\mk{cont}//
\glft \trsl{Marek is taller than Janek.}//
\endgl
\xe

\pex
\a
\begingl
\gla Janek-tám Marku nestačál.//
\glb Janek=\mk{comp} Marek-\Ins{} tall-\mk{cont}//
\glft \trsl{Marek is as tall as Janek.}//
\endgl
\a
\begingl
\gla Janek Marku nestačál.//
\glb Janek Marek-\Ins{} tall-\mk{cont}//
\glft \trsl{Marek is as tall as Janek.}//
\endgl
\xe

Note that \ird{tám} can only be used with the copulative form of the stative verb, as the attributive and nominal forms have separate conjugated comparative forms. When using these forms, however, the standard of comparison is marked in the genitive. In relative constructions, the instrumental is also replaced with the genitive, but the modifier \ird{zní}, \trsl{same} is added before the stative verb.

\pex
\a
\begingl
\gla Jancie nestačení hloc mlazem.//
\glb Janek-\Gen{} tall-\mk{comp-att} boy brother-\mk{1s}//
\glft \trsl{The boy who is taller than Janek is my brother} (\emph{Lit.,} \trsl{The taller-than-Janek boy is my brother.})//
\endgl
\a
\begingl
\gla Jancie zní nestačení hloc mlazem.//
\glb Janek-\Gen{} same tall-\mk{comp-att} boy brother-\mk{1s}//
\glft \trsl{The boy who is as tall as Janek is my brother.}//
\endgl
\xe

\ird{Tám} can be relativized by appending the clitic \ird{to}. When used with \ird{tám-to} the standard of comparison is marked in the patientive case. The use of tám-to in relative clauses is discussed in further detail in the next chapter.

\ex
\begingl
\gla Viktor na shlopa tám-to nestáček.//
\glb Viktor \Loc{} siblings-\Pat{} \mk{comp=rz=} to:be:tall-\Av{}-\Pf{}//
\glft \trsl{Among the siblings, Viktor grew up to be the tallest.}//
\endgl
\xe

\ex
\begingl
\gla Jankám Marka tám-to zuštalébik ko Tereza//
\glb Janek-\Agt{} Marek-\Pat{} \mk{comp=rz=} to:be:happy-\mk{ben-pf} \mk{att} Tereza//
\glft \trsl{Tereza, whom Janek made happier than Marek}//
\endgl
\xe

\ex
\begingl
\gla Marka tám-tóví zuštalébik ko oblašc//
\glb Marek-\Pat{} \mk{comp=rz-gen=} to:be:happy-\mk{ben-pf} \mk{att} pet//
\glft \trsl{the pet [of the person who was made happier than Marek]}//
\endgl
\xe

Iridian does not have a morphologically distinct superlative construction. For example, \ird{pizdení} (from \ird{pizdá}, \trsl{to be big}) can either mean \trsl{bigger} or \trsl{biggest} depending on context. Where the meaning cannot be easily implied from context, the word \ird{ohnu} (derived from the word \ird{ohna}, \trsl{first} in the instrumental case) is often used as quantifier.

\pex
\a
\begingl
\gla Univerzitet na razmeka pizdenou.//
\glb university \Loc{} city-\Pat{} to:be:big-\mk{comp-nz}//
\glft \trsl{(This) university is the biggest in the city.}//
\endgl
\a
\begingl
\gla Univerzitet na razmeka ohnu pizdenou.//
\glb university \Loc{} city-\Pat{} first-\Ins{} to:be:big-\mk{comp-nz}//
\glft \trsl{(This) university is the biggest in the city.}//
\endgl
\xe

When using an adverbial construction with the instrumental case to modify or quantify the comparison, the adverbial phrase must immediately precede the stative verb if in the attributive or nominal form, or the particle \ird{tám} otherwise. The same is true with invariable modifiers like \ird{nahte}, \trsl{too much}, \ird{dnu}, \trsl{a bit}, etc.

\ex
\begingl
\gla To bagáž jánám u 10 kilográmu tám prékvál.//
\glb \mk{dem.prox} baggage \mk{dem.med} around 10 kilogram-\Ins{} \mk{comp=} heavy-\mk{cont}//
\glft \trsl{This baggage is heavier by about 10 kilograms than that one.}//
\endgl
\xe

\ex
\begingl
\gla u 10 kilográmu prékvení bagáž//
\glb around 10 kilogram-\Ins{} heavy-\mk{comp-att} baggage//
\glft \trsl{the baggage, which is heavier by about 10 kilograms}//
\endgl
\xe

\ex
\begingl
\gla Nahte pizdenou zmažnikóveš.//
\glb too:much big-\mk{comp-nz} make-\mk{pv-pf-nz-2s}//
\glft \trsl{The much bigger one is the one you made.}//
\endgl
\xe

\section{Questions}\index{questions!syntax of}\index{interrogative sentence|see{questions!syntax of}}

There are two  main  categories  of  interrogative  sentences in Iridian: yes-no  and  question-word questions (or \emph{wh-} questions).

\subsection{Yes-no questions}\index{questions!yes-no}\index{questions!syntax of}

A declarative sentence can be made into a question by a simple rise in intonation at the end of the phrase:

\pex
\a
\begingl
\gla Janek sa uzdravšek.//
\glb Janek already \Refl{}-sleep-\Av{}-\Pf{}//
\glft \trsl{Janek has fallen asleep.}//
\endgl
\a
\begingl
\gla Janek sa uzdravšek?//
\glb Janek already \Refl{}-sleep-\Av{}-\Pf{}//
\glft \trsl{Has Janek fallen asleep yet?}//
\endgl
\xe

Alternatively the interrogative clitic \ird{no} may be used. When used this way, the base sentence will still feature a clause-final rise in intonation, followed by a falling intonation at the location of the question particle, similar to the intonation structure of tag questions in English. In the written language, the particle \ird{no} may also surface as a clitic, prefixing itself to the verb, which this usage requires to be in the negative.

\pex
\begingl
\gla Janek sa uzdravšek no?//
\glb Janek already \Refl{}-sleep-\Av{}-\Pf{} =\Q{}//
\glft \trsl{Has Janek fallen asleep yet?}//
\endgl
\xe

\pex\deftagex{formalq}
\begingl
\gla Janek sa nozáduzdravšek?//
\glb Janek already \Q{}=\Neg{}=\Refl{}-sleep-\Av{}-\Pf{}//
\glft \trsl{Has Janek fallen asleep yet?}//
\endgl
\xe

The choice between using a simple rise in intonation or the question particle \ird{no} is a personal one, and a speaker may use the one or the other in different situations or shift between them seemingly at random. Both methods in free variation and offer no differences in meaning, formality, etc. However the form in (\getfullref{formalq}) is extremely formal and archaic and rarely (if ever) appears in the spoken language.

Tag questions are formed by affixing the word \irdp{l\'e\v{t}}{truth,} at the end of the sentence. The tag cannot be used with the question particle \ird{no}.

\pex
\begingl
\gla Janek sa uzdravšek, l\'e\v{t}?//
\glft \trsl{Janek has fallen asleep already, right?}//
\endgl
\xe

Although the particle \ird{no} would normally appear after the verb, it can follow other parts of the sentence (except pure function words), but with the effect of changing the emphasis or the nature of the question. When used in this manner, the particle is separated from the word it modifies by a dash. Furthermore, there is a tendency especially in the spoken language to move the cliticized noun to the start of the sentence.

\pex
\a
\begingl
\gla Ivána-no niehu scenžach?//
\glb Ivána=\Q{} later-\Ins{} arrive-\Av{}-\Ctp{}//
\glft \trsl{Is it Ivána who is coming later?}//
\endgl
\a
\begingl
\gla Ivána niehu-no scenžach?//
\glb Ivána later-\Ins{}=\Q{} arrive-\Av{}-\Ctp{}//
\glft \trsl{Will it be later that Ivana is coming?}//
\endgl
\a
\begingl
\gla Niehu-no Ivána scenžach?//
\glb later-\Ins{}=\Q{} Ivána arrive-\Av{}-\Ctp{}//
\glft \trsl{Will it be later that Ivana is coming?}//
\endgl
\xe

To make an existential sentence\index{existential constructions} a yes-no question, it is first transformed to the negative and the particle \ird{no} is then attached to the word \ird{niho}. If however, the theme of the sentence is quantified, the word \ird{ješ}\index{ješ} is kept (but shifted to the front of the quantifier), and \ird{no} is attached to the quantifier. The form \ird{ješ-no} is ungrammatical.

\pex
\begingl
\gla Marka niho-no oblašc?//
\glb Marek-\Pat{} \N{}\Exst{}=\Q{} pet//
\glft \trsl{Does Marek have a pet?}//
\endgl
\xe

\pex
\a
\begingl
\gla Co bibliotécie Marka hroná ješ kupéninkou tóm?//
\glb \mk{abl} library-\Gen{} Marek-\Pat{} three \mk{exst} borrow-\mk{pv-pf-nz} book//
\glft \trsl{Marek borrowed three books from the library.}//
\endgl
\a
\begingl
\gla Co bibliotécie Marka ješ hroná-no kupéninkou tóm?//
\glb \mk{abl} library-\Gen{} Marek-\Pat{} \mk{exst} three\mk{=q} borrow-\mk{pv-pf-nz} book//
\glft \trsl{Did Marek borrow three books from the library.}//
\endgl
\xe

The clitic \ird{no} can of course be moved around, with subtle changes in meaning.

\pex
\a \emph{Neutral form:}\\
\ird{Co bibliotécie Marka ješ hroná-no kupéninkou tóm?}\\
\trsl{Did he borrow \emph{three} books, etc?}
\a \emph{Emphasis on \emph{Marek:}}\\
\ird {Co bibliotécie Marka-no hroná ješ kupéninkou tóm?}\\
\trsl{Did \emph{Marek} borrow them, etc?}
\a \emph{Emphasis on \emph{library:}}\\
\ird {Co bibliotécie-no Marka hroná ješ kupéninkou tóm?}\\
\trsl{Did he borrow them from the \emph{library}, etc?}
\xe

Note that in more complex existential constructions, as the one above which includes a nominalized determiner, the sentence may have to be reconstructed as a non-existential construction if it is the theme (i.e., the object being possessed or whose existence is described) that is in question.

\pex
\a
\begingl
\gla Co bibliotécie Marek hroná tóma kupénžek?//
\glb \mk{abl} library-\Gen{} Marek three book-\Pat{} borrow-\Av{}-\Pf{}//
\glft \trsl{Did Marek \emph{borrow} three books from the library.}//
\endgl
\a
\begingl
\gla Co bibliotécie Marek hroná tóma-no kupénžek?//
\glb \mk{abl} library-\Gen{} Marek three book-\mk{pat=q} borrow-\Av{}-\Pf{}//
\glft \trsl{Did Marek borrow three \emph{books} from the library.}//
\endgl
\xe

%explain further the preference
Note that the first example above is not the neutral word order, given Iridian's preference to use existential constructions in sentences like the ones above. In this case, it would be akin to asking \trsl{Did he borrow them, or did he acquire it by some other means?}


\subsection{\textit{Wh-} questions}\index{wh- questions}\index{information question|see{wh-questions}}\label{sec:ynquestions}
In wh- questions, the interrogative pronoun typically appears after the topic or at the beginning of a sentence if the sentence does not have a topic, and is immediately followed by the clitic \ird{no}.

\pex
\begingl
\gla Karel jena-no možlašál?//
\glb Karel where=\Q{} live-\Av{}-\Cont{}//
\glft \trsl{Where does Karel live?}//
\endgl
\xe

\pex
\begingl
\gla Bych zajehu-no kravnašalí?//
\glb yesterday why=\Q{} cry-\Av{}-\Prog{}//
\glft \trsl{Why was he crying yesterday?}//
\endgl
\xe


\subsection{Echo questions}
\subsection{Indirect questions}

Indirect questions are constructed in the subjunctive, with the addition of the particle \ird{aš}.

\pex
\begingl
\gla Nú aš hošezíla.//
\glb tomorrow \mk{q.ind} rain\mk{av-sbj.ipf}.//
\glft \trsl{I wonder if it's gonna rain tomorrow.}//
\endgl
\xe

\subsection{Answering questions}\label{sec:ansyn}

Most yes-no questions may be answered by repeating the focal word or phrase in the original question or echoing the syntax of the question itself.

\ex
\vtop{\halign{%
#\hfil& \qquad #\hfil\cr
\ird{---\,Kartu\v{s}k\'i tak slouve\v{z}ev\'i?} & \trsl{\small Do they sell potatoes here?}\cr
\ird{---\,Slouve\v{z}ev\'i?} & \trsl{\small They do.}\cr
}}
\xe

Alternatively, the question may be answered by \irdp{da}{yes} or \irdp{ne}{no,} both of which have been adapted from Common Slavic\index{Common Slavic}. In colloquial speech it is also common to use \ird{j\'a} or \ird{j\'o} for \trsl{yes} (this is most likely a borrowing from German\index{German}). These polarity words may be used alone or in combination with the echo response. In general, the order does not matter, although it is more common for the polarity word to appear after the echo response. If the original question was framed in the negative and the response is positive, the contrastive \irdp{ale}{yes} is used instead.

\ex
\vtop{\halign{%
#\hfil\hfil\cr
\ird{---\,Lo\v{s}n\'i Nolan\'i vilm \v{z}a oudnenik?}\cr
\ird{---\,\v{Z}a oudnenik, da. M\'a z\'a\v{c}es\v{c}ik.}\smallskip\cr
\trsl{\small Have you seen Nolan's new film?}\cr
\trsl{\small I've seen it, yes. But I didn't like it.}\cr
}}\xe

\ex\vtop{\halign{%
#\hfil\hfil\cr
\ird{---\,Dan\'i trehlo za banka podarn\'il\'a to-\v{z}e Janek z\'al\'eh\'a\v{c}ek?}\cr
\ird{---\,Leh\'a\v{c}ek, ale. M\'a avtem bych hebo.}\smallskip\cr
\trsl{\small Weren't you advised by Janek to submit your tax return to the bank?}\cr
\trsl{\small He did, yes. But my car broke down yesterday.}\cr
}}
\xe

\ird{Da} may also preface answers to questions as a form of intensifier, or to indicate that the speaker considers the answer to the question as an obvious truth.

\ex
\vtop{\halign{%
#\hfil& \qquad #\hfil\cr
\ird{---\,Na muzla je\v{s} vdenikou.} & \trsl{\small I saw someone at the mall today.}\cr
\ird{---\,Jede?} & \trsl{\small Who?}\cr
\ird{---\,Da Janek.} & \trsl{\small Well, Janek, of course.}\cr
}}\xe


As for questions involving existential constructions


\section{Negation}\index{negation}

In Iridian sentences, negation is performed by the particle \ird{zám}, which attaches to the beginning of the word or phrase  it negates. The default position of the negative particle is usually before the main verb where it surfaces as \ird{z-} before vowels, \ird{ž-} before \emph{i}-glides, and \ird{zá-} elswehere.

\pex
\a
\begingl
    \gla Janek Martina Markám \textbf{zá}hévoržébik.//
    \glb Janek Martin-\Pat{} Marek-\Agt{} \mk{neg=}know-\mk{ben-pf}//
    \glft \trsl{Marek did not introduce Janek to Martin.}//
\endgl
\a
\begingl
    \gla \textbf{Zám} Janek Martina Markám hévoržébik.//
    \glb \mk{neg=} Janek Martin-\Pat{} Marek-\Agt{} know-\mk{ben-pf}//
    \glft \trsl{It was not Janek whom Marek introduced to Martin.}//
\endgl
\a
\begingl
    \gla Janek \textbf{zám} Martina Markám hévoržébik.//
    \glb Janek \mk{neg=} Martin-\Pat{} Marek-\Agt{} know-\mk{ben-pf}//
    \glft \trsl{It was not Martin whom Marek introduced Janek to.}//
\endgl
\a
\begingl
    \gla Janek Martina \textbf{zám} Markám hévoržébik.//
    \glb Janek Martin-\Pat{} \mk{neg=} Marek-\Agt{} know-\mk{ben-pf}//
    \glft \trsl{It was not Marek who introduced Janek to Martin.}//
\endgl
\xe

It is also common, especially in spoken Iridian, to append the clitic \ird{-te}
after the word being negated by \ird{zám} (i.e., if the negative clitic is not
in the default position before the main verb) to provide more emphasis on the
negation.

\pex
\a
\begingl
    \gla \textbf{Zám} Janek\textbf{-te} Martina Markám hévoržébik.//
    \glb \mk{neg=} Janek\mk{=foc} Martin-\Pat{} Marek-\Agt{} know-\mk{ben-pf}//
    \glft \trsl{It was not Janek whom Marek introduced to Martin.}//
\endgl
\a
\begingl
    \gla Janek \textbf{zám} Martina\textbf{-te} Markám hévoržébik.//
    \glb Janek \mk{neg=} Martin-\mk{pat=foc} Marek-\Agt{} know-\mk{ben-pf}//
    \glft \trsl{It was not Martin whom Marek introduced Janek to.}//
\endgl
\a
\begingl
    \gla Janek Martina \textbf{zám} Markám\textbf{-te} hévoržébik.//
    \glb Janek Martin-\Pat{} \mk{neg=} Marek-\mk{agt=foc} know-\mk{ben-pf}//
    \glft \trsl{It was not Marek who introduced Janek to Martin.}//
\endgl
\xe

The different constituents of the sentence can be negated simultaneously; thus,
for example, the sentence below is grammatically permitted:

\pex
\begingl
    \gla \textbf{Zám} Janek \textbf{zám} Martina \textbf{zám} Markám \textbf{zá}hévoržébik.//
    \glb \mk{neg=} Janek \mk{neg=} Martin-\Pat{} \mk{neg=} Marek-\Agt{} \mk{neg=}know-\mk{ben-pf}//
    \glft \trsl{It was not Janek who was not introduced to someone who is not Martin by someone who is not Marek.}//
\endgl
\xe

Nonetheless, due to their general unwieldiness, forms like this are extremely
rare (both in the spoken and the written language), with preference given to
single and double negation instead. Since \ird{-te} can only appear in a
sentence once, where there are more than one negate constituent in a sentence,
\ird{-te} is appended to the element which has the most significance (usually
the topic); or, if there are two constituents negated and one of them is the
main verb, \ird{-te} is appended to that other element.

\pex
\begingl
    \gla \textbf{Zám} Janek\textbf{-te} Martina Markám \textbf{zá}hévoržébik.//
    \glb \mk{neg=} Janek\mk{=foc} Martin-\Pat{} Marek-\Agt{} \mk{neg=}know-\mk{ben-pf}//
    \glft \trsl{It was not Janek who was not introduced to Martin by Marek.}//
\endgl
\xe


\section{Existential Constructions}
\label{sec:exst}
An existential sentence is a specialized construction used to express the existence or presence of someone or something. The particle \ird{ješ} and its inverse \ird{niho} are used to form existential sentences.

\pex
\begingl
\gla Tak ješ zarno.//
\glb here \mk{exst} people//
\glft \trsl{There are people here.}//
\endgl
\xe

\pex
\begingl
\gla Tak niho zarno.//
\glb here \mk{exst.neg} people//
\glft \trsl{There is no one here.}//
\endgl
\xe

Statements expressing location use a copular construction, although an existential construction is used in the negative.

\pex
\begingl
\gla Dá na duma.//
\glb \mk{1s.str} \Loc{} house-\Pat{}//
\glft \trsl{I'm at home.}//
\endgl
\xe

\pex
\begingl
\gla Na duma niho dá.//
\glb \Loc{} house-\Pat{} \mk{exst.neg} \mk{1s.str}//
\glft \trsl{I'm not at home.}//
\endgl
\xe

The particles \ird{ješ} and \ird{niho} must always precede the noun whose presence or existence is being expressed.

\pex
\begingl
\gla Na ránema ona ješ htoš.//
\glb \Loc{} desk-\mk{1s-pat} one \mk{exst} book//
\glft \trsl{There is one book on my desk.}//
\endgl
\xe

\pex
\begingl
\gla M\"y ješ mulaž.//
\glb two \mk{exst} door//
\glft \trsl{There are two doors.}//
\endgl
\xe



\subsection{Possession}
Existential constructions are also used to indicate possession, with the possessor marked in the patientive case.

\pex
\begingl
\gla Marka ješ oblašc.//
\glb Marek-\Pat{} \mk{exst} pet//
\glft \trsl{Marek has a pet.}//
\endgl
\xe

\pex
\begingl
\gla Tomáša niho mlaz.//
\glb Tomáš-\Pat{} \mk{exst} brother//
\glft \trsl{Tomáš does not have a brother.}//
\endgl
\xe

\subsection{Impersonal constructions}
\pex
\begingl
\gla Martina ješ trešnikou na tropa.//
\glb Martin-\Pat{} \mk{exst} write-\mk{pv-pf-nz} \Loc{} wall-\Pat{}//
\glft \trsl{Martin wrote something on the wall.}//
\endgl
\xe

\pex
\begingl
\gla Voštnikouva \v{z}a ješ piaščkou?//
\glb cook-\mk{pv-pf-nz-pat} already \mk{exst} eat-\mk{av-pf-nz}//
\glft \trsl{Did somebody eat what (I) cooked?}//
\endgl
\xe

\section{Copular Constructions}
\subsection{Null copula}

Copular sentences are a minor sentence type where the predicate is not a verb. For the purposes of this grammar, we narrow down our definition of copular constructions to the following:
\pex
\a \textit{Equative:} Marek is the doctor (we are talking about).
\a \textit{Inclusive:} Marek is a doctor.
\a \textit{Attributive:} Marek is tall.
\a \textit{Locative:} Marek is in the hospital.
\xe

Iridian does not make a distinction between equative, inclusive and attributive clauses. Locative clauses on the other hand, may be expressed using a copular or an existential construction, as will be discussed in this section.

Iridian is a superficially a zero-copula language and the most common way to form copular sentences is mere juxtaposition.

\pex<cop>
\begingl
\gla Marek doktor.//
\glb Marek doctor//
\glft \trsl{Marek (is a/the) doctor.}//
\endgl
\xe

The above example could either be taken to mean (1) Marek is a doctor (inclusive), or (2) Marek is the doctor (equative). Generally, though, Iridian uses word order to distinguish between equative and inclusive clauses.

\pex
\a \textit{Inclusive:} \{item in class\}\tss{N} $\varnothing$ \{class\}\tss{P}
\a \textit{Equative:} \{class\}\tss{N} $\varnothing$ \{item class\}\tss{P}
\xe

To avoid ambiguity, Example \getref{cop} can be reformulated to either of the following sentences:

\pex<cop1>
\a
\begingl
\gla Marek doktor.//
\glb Marek doctor//
\glft \trsl{Marek is a doctor.}//
\endgl

\a
\begingl
\gla Doktor Marek.//
\glb doctor Marek//
\glft \trsl{Marek is the doctor.}//
\endgl

\xe

The inversion of word order is not strongly grammaticalized with NP-NP sentences, i.e., both sentences in Example \getref{cop1} can still be used interchangeably without a change in meaning and preference is given on the one over the other when there is an ambiguity. This is not the case with attributive clauses, i.e., sentences with adjective or adjective phrase predicates. Consider for example the sentence below:

\pex
\begingl
\gla Marek rázym.//
\glb Marek tall//
\glft \trsl{Marek is tall.}//
\endgl
\xe

Inverting the word order of the sentence above would change the adjective to a substantive since modifiers cannot occupy the topic position.

\pex
\begingl
\gla Rázym Marek.//
\glb tall Marek//
\glft \trsl{The tall one is Marek.}//
\endgl
\xe

Iridian also distinguishes between attributive clauses expressing permanent conditions and clauses expressing temporary conditions, with the latter being expressed using existential constructions in certain adjectives.

\pex
\begingl
\gla *Marek morec.//
\glb Marek hungry//
\glft \trsl{Marek is hungry}//
\endgl
\xe


\pex
\begingl
\gla Marka ješ morec.//
\glb Marek-\Pat{} \mk{exst} hunger//
\glft \trsl{Marek is hungry}//
\endgl
\xe

A full list of adjectives/modifiers that use the existential construction can be found in the section~\ref{sec:exst}.

The copula, however, cannot be ommitted in grammatical moods other than the indicative.

\subsection{Negative copula}

Iridian has the negative copula \ird{česná}.

\pex
\begingl
\gla Marek doktor česná.//
\glb Marek doctor \mk{cop.neg}//
\glft \trsl{Marek is not (a/the) doctor.}//
\endgl
\xe

\par The inversion of word order may also be used when one wants to avoid ambiguity:

\pex
\begingl
\gla Doktor Marek česná.//
\glb doctor Marek \mk{cop.neg}//
\glft \trsl{Marek is not the doctor.}//
\endgl
\xe


\subsection{Conjugation paradigm}
			% Syntax of Simple Clauses
\chapter{Complex Sentences}

\section{Coordination} \index{coordination}

Iridian has [number here] coordinating conjunctions: \irdp{a}{and},

When coordinating simple noun pairs, however, the particle \irdp{\v{s}e}{with}\index{\v{s}e} is mostly used where English would have used \trsl{and}. The derived construction \ird{a \v{s}e} is also common and has a similar meaning to the English \trsl{and also}.

\pex
\begingl
    \gla M\'amka \textbf{\v{s}e} p\'apku na Prah\'a span\'i\v{c}ek.//
    \glb mother-\mk{dim} \mk{com} father-\mk{dim-inst} \mk{loc} Prague-\mk{pat} vacation-\mk{av-pf}//
    \glft \trsl{Mom and Dad went to Prague for vacation.}//
\endgl
\xe

\pex
\begingl
    \gla Janek \textbf{a} \textbf{\v{s}e} Marku kurs hlupin\v{z}ice.//
    \glb Janek and \mk{com} Marek-\mk{inst} class fail-\mk{av-pf-quot}//
    \glft \trsl{Janek as well as Marek failed the class.}//
\endgl
\xe

In constructions with \ird{\v{s}e} where one of the nouns coordinated is a pronoun or a deictic\index{deictic}, the pronoun or deictic is presented first followed by the other noun in the instrumental case.

\pex
\begingl
    \gla D\'a \textbf{\v{s}e} Ivanu sohladou\v{s}ce.//
    \glb \mk{1s.str} \mk{com} Ivan-\mk{inst} classmate//
    \glft \trsl{Ivan and I are classmates.}//
\endgl
\xe

In a few cases, \ird{a} is used instead of \ird{\v{s}e} where the latter can be interpreted as having an attributive meaning. Where the noun is marked, however, only \ird{a} can be used.

\pex
\a
\begingl
    \gla trava \textbf{\v{s}e} l\'epu//
    \glb bread \mk{com} cheese-\mk{inst}//
    \glft \trsl{bread with cheese} i.e., \trsl{cheese sandwich}//
\endgl
\a
\begingl
    \gla trava \textbf{a} l\'ep//
    \glb bread and cheese//
    \glft \trsl{bread and cheese}//
\endgl
\xe

\pex
\begingl
    \gla To kurs-te Jank\'am \textbf{a} Mark\'am hlupienince.//
    \glb this class-\mk{foc} Janek-\mk{agt} and Marek-\mk{agt} class fail-\mk{pv-pf-quot}//
    \glft \trsl{It was this class that Marek and Janek failed.}//
\endgl
\xe


The bisyndetic coordination (\cite{velupillai2012}) \ird{a} Y \ird{a} Y is also with similar emphatic meaning as \ird{a \v{s}e}.

\pex
\begingl
    \gla \textbf{a} plocem \textbf{a} ploce\v{s}.//
    \glb and family-\mk{1s} and family-\mk{2s}//
    \glft \trsl{both my family and yours}//
\endgl
\xe

\pex
\begingl
    \gla \textbf{a} \v{c}astu \textbf{a} \v{s}e zmenu zoviec hloubi\v{z}al\'i.//
    \glb and suffering-\mk{} and \mk{com} happiness-\mk{inst} remain-\mk{cv} love-\mk{av-prog}//
    \glft \trsl{til death do us part}//
\endgl
\xe

With multiple nouns or noun phrases, especially in serial lists, the coordinating conjunction is often simply dropped.

\pex
\begingl
    \gla Ivan, Jarek, Elena na meza.//
    \glb Ivan Jarek Elena \mk{loc} room-\mk{pat}//
    \glft \trsl{Ivan, Jarek, and Elena are in the room.}//
\endgl
\xe

\pex
\begingl
    \gla Morkve, hlepost, ruk, molec \v{z}a hladni\v{z}\'al.//
    \glb carrot asparagus broccoli cabbage \mk{1s.pat} to:notplease-\mk{av-cont}//
    \glft \trsl{I don't like carrots, asparagus, broccoli or cabbage.}//
\endgl
\xe



\section{Clause-linking with \ird{\v{s}e}}

\section{Converbial Constructions}\label{converbs-syntax}\index{converb}

\subsection{The imperfective in \ird{-iec}}



\subsection{The perfective in \ird{-e}}

The perfective \textit{-iêce} is often used in clause linking.

\pex
\begingl
\gla O\v{s}tiêce krazkem.//
\glb read-\mk{cv.pf} understand-\mk{pf-1s}//
\glft `I read and understood.'//
\endgl
\xe

Clauses expressing reason is usually expressed by a converbial construction.

\pex
\begingl
\gla Za eksama názhaziêce, Martin órek.//
\glb for exam-\mk{pat} \mk{neg}-study-\mk{cv.pf} Martin fail-\mk{pf}//
\glft `Martin failed the exam because he didn't study.'//
\endgl
\xe

\section{Reported Speech}\label{sec:reportedspeech}\index{reported speech}\index{indirect speech|see{reported speech}}

The reported statement and the main clause are separated by the quotative particle \ird{to-\v{z}e}\footnote{This particle will simply be glossed as {\scshape qp} even though it actually consist of two parts: the relativizing particle \ird{to} and the cliticized quotative particle \ird{\v{z}e}.}.

\pex
\begingl
\gla Ma\v{s}a advok\'at nev\'i to-\v{z}e z\'i\v{c}ek.//
\glb Ma\v{s}a lawyer \mk{cop.quot} \mk{qp} say-\mk{av-pf}//
\glft \trsl{(He) said that Ma\v{s}a is a lawyer}//
\endgl
\xe

The reported part is treated as a subordinate clause and must appear before the main clause. In general reported speech takes the form

\ex\deftagex{ex:repstruct}{\small
\bigg[ \Big[ \big[ [TOP*] [PRED in quotative mood] \big] + \big[\ird{to-\v{z}e}*\big] \Big] \bigg] + \bigg[verbum dicendi*\bigg],
}\xe

\ex
\begin{forest}
  [S,
    [{TOP}, [TOP] [VP] ]
    [{PRED}, [QP] [VP,  [NP] [VP]]
    ]]
\end{forest}
\xe
where the elements followed by an asterisk (*) are optional.

The \emph{verbum dicendi}\index{verbum dicendi} (Latin for verb of speech/speaking) is the verb in the main clause that signals that the subordinate clause is a quoted clause and that its main clause should therefore appear in the quotative mood. Examples of \emph{verba dicendi} in Iridian include \irdp{ziek\'a}{to say}; \irdp{vad\'a}{to think}; \irdp{kvu\v{s}t\'a}{to hear}; \irdp{vid\'a}{to see}; \irdp{hloup\'a}{to ask}; \irdp{ohlet\'a}{to remember}; \irdp{shov\'a}{to recount, to tell a story}. Note that although they are called verbs ``of speaking'' they do not necessarily introduce speech as much as function as grammaticalized tags marking the quotative,  which is more properly analyzed to mark not just speech but inferentiality and evidentiality as well.

More complex \emph{verba dicendi} can be formed by using an imperfect converbial construction (the converb form in \ird{-iec}) with a canonical \emph{verbum dicendi}. To understand this consider the following sentences in English:

\pex[*=?*]
\a She said no.\deftagex{vd}\deftaglabel{1}
\a She whispered no.\deftaglabel{2}
\a She said no in a whisper.\deftaglabel{3}
\a \ljudge{??} She said \textbf{in a whisper} no.\deftaglabel{4}
\a \ljudge{??} She said \textbf{whisperingly} no.\deftaglabel{5}
\xe

\smallskip

We see that both \emph{said} (\getfullref{vd.1}) and \emph{whispered} (\getfullref{vd.2}) are \emph{verba dicendi} in English. Nonetheless it's also obvious how \getfullref{vd.2} is simply a function of (\getfullref{vd.1}), i.e., we can express (\getfullref{vd.2}) in terms of (\getfullref{vd.1}), in this case using an adverbial construction (\trsl{in a whisper}) as we see in \getfullref{vd.3} or the more affected \getfullref{vd.4}. Finally using a simple adverbial is theoretically allowed in English (\getfullref{vd.5}), although as we see the resulting construction is rather unwieldy or unnatural-sounding.

In Iridian, however, constructions like (\getfullref{vd.2}) are not permitted, with preference given to adverbial (or more correctly, converbial)\index{converb} constructions. Thus we translate (\getfullref{vd.2}) as:

\pex
\begingl
\gla Ne to-\v{z}e mi\v{s}lec z\'i\v{c}ek.//
\glb no \mk{qp} whisper-\mk{cv} say-\mk{av-pf}//
\glft \trsl{(She) whispered no.}//
\endgl
\xe


This converbial construction is not limited to what is essentially describing how the verbum dicendi was  Other more idiomatic treatments include

%% EXAMPLES HERE

It should be noted as well how the verb \irdp{vad\'a}{to think} and its derived forms, due to their inherent meanings, require the subjunctive to be used in the reported clause. This is true whether or not the subjunctive would have been used had the reported clause been a regular dependent clause.


\pex
\a
\begingl
  \gla Já mnou.//
  \glb you correct//
  \glft \trsl{You're right.}//
\endgl
\a
\begingl
  \gla Já mnou nev\'i.//
  \glb you correct \mk{cop.quot}//
  \glft \trsl{(I heard) you're right}//
\endgl
\a
\begingl
  \gla Já mnou nehl\'i to-\v{z}e Martin spouviec v\'a\v{z}\'al.//
  \glb you correct \mk{cop.quot.sbj} \mk{qp} Martin agree-\mk{cv} think-\mk{av-cont}//
  \glft \trsl{Martin agrees that you are right.}//
\endgl
\xe



We see from (\getfullref{ex:repstruct}) that when it comes to reported speech and similar constructions in Iridian, the \ird{verbum dicendi}\index{verbum dicendi} is not necessary to create a well-formed sentence. The same is true with the quotative particle \ird{to-\v{z}e}. Both can be omitted without making the sentence grammatically incorrect since the quotative particle is enough to identify the reported clause.\index{reported speech}.

In most instances, however, removing either the main verb or the main verb and the quotative particle can cause the resulting sentence to acquire a new meaning. This is especially true when the quotative mood is used not to report speech but to imply a certain unsureness on the part of the speaker about the information being presented, or for the speaker to distance themself by implying through the use of the quotative that the information is secondhand and not theirs.

Generally \ird{to-\v{z}e} is kept when the speaker is quoting themself, to repeat or emphasize what they have said, or expletively, to express their frustration or affirmation.\footnote{When used this way the pronunciation of \ird{to-\v{z}e} is closer to an emphatic \nt{"to\dpu\textctz{}E} or even \nt{"to\dpu:\textctz{}EP}}

%% TODO remove affrication of initial dental stops in Phonology section!!!

\pex
\begingl
\gla Mnou nev\'i to-\v{z}e!//
\glb correct \mk{cop.quot} \mk{qp}//
\glft \trsl{I've been telling you) it is right.}//
\endgl
\xe

\pex
\begingl
\gla Dá roctymút to!//
\glb \mk{1s} dance-\mk{abl-quot.ipf} \mk{rz}//
\glft `(But) I can dance.'//
\endgl
\xe


Interestingly, commands and requests are not treated as reported speech but as regular subordinate clauses governed by \ird{to} and not by \ird{to-\v{z}e}.

When the quoted clause is a question, whether a direct one or not, the quoted clause is preceded by the particle \irdp{a}{and} and the word \irdp{ane}{whether} is used instead of \ird{to-\v{z}e}. The word \ird{ane} is also used for verba dicendi that are interrogative in nature, such as \irdp{pr\'ehoust\'a}{to ask},

\pex
\begingl
  \gla A Janek zdal\v{s}ice ane pr\'ehous\v{c}ek.//
  \glb and Janek have:breakfast-\mk{av-pf-quot} whether ask-\mk{av-pf}//
  \glft \trsl{(He) asked (me) whether Janek has had breakfast yet.}//
\endgl
\xe

\pex
\begingl
  \gla A t\'om to ml\'adu hodina\v{z}e ane, nie svad postup\'al.//
  \glb and book this year-\mk{inst} finish-\mk{pv-ctpv-quot} whether \mk{pl} fan be:excited-\mk{cont}//
  \glft \trsl{His fans are excited to know if he'll finish his book this year.
}//
\endgl
\xe
			% Complex Sentences
\chapter{Semantics and Usage}

\section{Register}
\section{Forms of Address and Treatment}\index{forms of address}

\subsection{Terms of courtesy}

The term \irdp{ma\v{s}e} is exclusively used as a honorific

\subsection{Salutations}\index{salutation}

The general salutation in most formal correspondence uses the honorific \irdp{Staj}{Sir} or \irdp{Nau}{Madame}. The last name of the addressee may also follow, although more often than not, the simple honorific should suffice. When addressing a collegiate entity or a collection of people, the term \irdp{Ma\v{s}e}{crowd} or \irdp{Prehoda\v{s}ce ma\v{s}e}{Esteemed/praiseworthy crowd} is used instead.

If the addressee holds a specific title, the title is included in the salutation. In some cases, the wife of the title-holder may be addressed using \ird{Nau} followed by the title, although this practice is slowly falling out of use, except in most diplomatic correspondence, where it is still considered standard. Below are some examples:


\begin{itemize}[nosep]
	\item \irdp{Staj/Nau Prezident}{Mister/Madame President}
	\item \irdp{Staj/Nau Brac}{Mister/Madame Member of the Parliament}
	\item \irdp{Staj/Nau Kancl\'ar}{Mister/Madame Chancellor}
	\item \irdp{Staj/Nau Holva}{Mister/Madame Chairman/Chairwoman}
	\item \irdp{Staj/Nau Prov\'izor}{Mister/Madame Professor}
\end{itemize}

Where the addressees are multiple individuals who hold specific titles, the honorific \ird{Staj} or \ird{Nau} is replaced with \irdp{prehoda\v{s}ce}{esteemed, praiseworthy}. When used this way, the title is normally not capitalized. Note also that \ird{prehoda\v{s}ce} will only be used in a salutation when there are multiple addressees.

\begin{itemize}[nosep]
	\item \irdp{Staj/Nau brac}{Esteemed members of the Parliament}
	\item \irdp{Prehoda\v{s}ce prov\'izor}{Esteemed members of the faculty}
\end{itemize}

\subsection{Valedictions}\index{valediction}

Standard valedictions used in formal written correspondence\index{correspondence|see{written correspondence}} in Iridian tend to be more complex than the ones used in English. Below is 

\begin{itemize}[nosep]
	\item \irdp{(Staj/Nau) oblostnen\'i mavac/respekt akceptirnik\'a}{Sir/Madame, please accept my sincerest regards (\emph{lit.}, wishes)/respect.}
	\item \irdp{D\'a zespoden\'i/spietnen\'i pok\'ar bile\v{z}it}{I will remain your most humble/loyal servant.}
	\item \irdp{D\'a zespoden\'i/spietnen\'i byl bile\v{z}it}{I will remain your most humble child.}\footnote{This is often used among religious people when writing to members of the clergy.}
	\item \irdp{Oblostnen\'i mavacu/respektu \v{s}e hroznik.}{With the sincerest regards/respect has this letter been sent.}

\end{itemize}

Increasingly, especially in e-mail\index{e-mails} correspondence, it has become more common to use the following valedictions instead:

\begin{itemize}[nosep]
	\item \irdp{Mavac/\v{S}e mavacu}{Regards/with wishes/regards.}
	\item \irdp{Oblostnen\'i}{Most sincere}
\end{itemize}

In more informal situations, such as between close friends and family, the following are used

\begin{itemize}[nosep]
	\item \irdp{D\'a}{I/me}
	\item \irdp{Bes/Mach bes/Nic bes}{Hug/Two hundred hugs/A thousand hugs}
	\item \irdp{Beska/Mach beska/Nic beska}{Little hug/Two hundred little hugs/A thousand little hugs}
	\item \irdp{\v{S}e hloubu/Hloub\v{z}ev\'i}{With love/Loving}
	\item \irdp{\v{Z}u\v{z}/Mach \v{z}u\v{z}/Nic \v{z}u\v{z}}{Kiss/Two hundred kisses/A thousand kisses}
\end{itemize}

\section{Idiomatic Expressions}\index{idiomatic expressions}

\section{Punctuation}			% Semantics and Usage

\appendix

\part*{Appendices}

\chapter{The Dialects of Iridian}					% Dialectology
\chapter{Sample Texts}

\section{The \emph{Pater Noster}}

\section{Milan Kundera, \ttla{A Kidnapped West or the Tragedy of Central Europe}}

{\small
% NOTE ON THE TRANSLATION
The translation is based on the French text of Kundera's essay \ttla{Un Occident kidnappé: ou la tragédie de l'Europe centrale} first published in \ttlb{Le Débat} in 1983. The full text is available online at various websites, with the link I used in the references. Due to copyright considerations, a translation has not been provided, although interlineal glosses and explanatory notes have been added where  I believe they are needed, in addition to the lexicon at the end. The text itself contains its own footnotes however and to distinguish Kundera's notes from those I have added, I have included included his name at their end.
}

\begin{center}1.\end{center}

1956 svemí Septembru Mažarevní Znova Byróví direktorám, byró
nastolám jednočnil ko obiení vniho minutu, ruščevnie uráž
po Budapešta šelčice to-že télexu laska mieta kudní expedica
pashvalébik. Expedice to nie neitu uhožnek: >> Mé za Mažaróma a
za Evropa shražach<<.

Nie neite ježe-no prónesčeví? Mažaróma a še laska Evropu ruščevní šarám zbavujinalu to žvotu prónesčeví. Ma Evropa zbavujinale to ježe prónestu?

Ma žená --- >>za bláha a za Evropa shražá<< --- to že Leningrada že Mušhóva závadnéteví to neite, ma če je Budapešta, če je Varšáva.

\begin{center}2.\end{center}
Vade, Evropa-te ježe-no za ona mažarevna, ona češčevna, ona polščevna?


\section{Written Correspondence}\label{sec:writcorr}\index{written correspondence}

\subsection{Formal Business Letter}
{\small
\begin{flushright}
	Roubže\\
	2019\,h. Mercí 14.\,r.
\end{flushright}

\noindent{Marek Zakár}\\
Ledeman Direkt {\sc m/h}\\
Husplac, \textnumero{} 177\\
Osthalbár\\
86332 Roubže {\sc rb}
}

\subsection{Formal E-mail}




\subsection{Informal Letter}
			% Sample Texts

\cleardoublepage
\nocite{*}
\printbibliography

\cleardoublepage
\printindex

\end{document}
