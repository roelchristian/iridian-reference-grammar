% !Mode:: "TeX:UTF-8"
% encoding:					UTF-8
% compiled with:		luatex
\documentclass[11pt]{scrbook}
\areaset[0.50in]{4.5in}{8in}
% Load packages used

\input{./preamble/common-packages.tex}


%% PACKAGES USED

%% Page layout and encoding
\usepackage[paperwidth=6in,paperheight=9in]{geometry}

\geometry {
    verbose,
    tmargin  =1.1in,
    bmargin  =1.1in,
    lmargin  =0.7in,
    rmargin  =0.7in,
}


%% PACKAGES USED

%% Page layout and encoding

%% Fonts

%\usepackage[oldstylenums]{kpfonts}
\usepackage{ifluatex}
\ifluatex
  \usepackage{fontspec}
  \defaultfontfeatures{Ligatures=TeX,Numbers=OldStyle}
  %\defaultfontfeatures{Ligatures=TeXNumbers=OldStyle}% ,Scale=MatchLowercase} bug in current Biolinum
  \setmainfont{Cardo}[Scale=0.94]
  \setsansfont{Linux Biolinum O}
  \newfontfamily\CardoSmallCaps
  [Ligatures={NoRequired, NoCommon, NoContextual},
  Letters={UppercaseSmallCaps,SmallCaps},Scale=1.05]
  {Cardo}
\else
  \usepackage[utf8]{inputenc}
  \usepackage[T1]{fontenc}
  \usepackage{kpfonts}
  \usepackage{textgreek}
\fi

%\usepackage{savetrees} % for drafts
\usepackage{microtype}

\usepackage{float}
\doublehyphendemerits=1000000000
\widowpenalty=10000
\clubpenalty=10000
\brokenpenalty=10000

%% Lingustics packages
\usepackage{tipa}
\usepackage{expex}

\gathertags

% Third level examples in expex
% http://tex.stackexchange.com/a/77941
\def\beginsubsub{%
\par
\begingroup
\advance\leftskip by 2em
\def\b##1{\par\leavevmode\llap{\hbox to 2em{##1\hfil}}\ignorespaces}}
\def\endsubsub{\par\endgroup}

\usepackage[russian,english]{babel}
\usepackage[german=guillemets]{csquotes}

\usepackage{graphics,graphicx,amsmath,amssymb}
%% Tables and Graphics
\usepackage[justification=raggedright,labelfont=it,font=footnotesize,labelsep=quad]{caption}
\captionsetup{justification=raggedright,singlelinecheck=false}
\usepackage{longtable}
\usepackage{minitoc}
\usepackage{afterpage}
\usepackage{xcolor,colortbl}
\usepackage{fancyhdr,textcase}
\usepackage{siunitx,tabularx,ragged2e,booktabs,tabu,multicol}
\usepackage{titlesec}
\usepackage{etoolbox}
\usepackage{multirow}
\usepackage{enumitem}
\usepackage[british]{datetime2}


\setlist[enumerate]{leftmargin=*}
\setlist[itemize]{leftmargin=*}


%% to draw linguistic trees
\usepackage{tikz}
\usepackage[linguistics]{forest}
\usetikzlibrary{positioning}
\usetikzlibrary{tikzmark}
\forestset{
sn edges/.style={for tree={parent anchor=south, child anchor=north}}}

%% Indexing and Cross-referencing
\usepackage{imakeidx}
\usepackage[font=footnotesize]{idxlayout}
\usepackage[hidelinks,unicode]{hyperref} % for final copy
%\usepackage[unicode]{hyperref} % for drafts
\usepackage{url}
\urlstyle{same}

% fancy footnotes
\usepackage[norule,splitrule]{footmisc}

\RequirePackage[authordate,backend=biber]{biblatex-chicago}
\DeclareFieldFormat[article]{title}{\mkbibquote{#1}} % make article titles in quotes
\DeclareFieldFormat[thesis]{title}{\mkbibemph{#1}} % make theses italics
\addbibresource{./bibliography/bibliography.bib}
\setlength\bibitemsep{0.4\itemsep}
\renewcommand*{\bibfont}{\small}



\renewcommand*{\postnotedelim}{\addcolon\addspace}
\DeclareFieldFormat{postnote}{#1}
\DeclareFieldFormat{multipostnote}{#1}



\makeindex
\setcounter{secnumdepth}{2}
\setcounter{tocdepth}{2}


%% CUSTOM COMMANDS

\newcommand{\zz}{\textctz}								% voiced palatal sibilant
\newcommand{\bt}[1]{/\textipa{#1}/}				% broad transcription
\newcommand{\nt}[1]{[\textipa{#1}]}				% narrow transcription
\newcommand{\sx}[1]{\textsuperscript{#1}}	% superscript phoneme
\newcommand{\mk}[1]{\textsc{#1}}					% glossing abbreviation
\newcommand{\dpu}{\textsubarch{u}}
\newcommand{\nn}{\textltailn}							% palatal n
\newcommand{\jjg}{\textbardotlessj}				% voiced palatal stop
\newcommand{\llt}{\textltilde}						% dark l
\newcommand{\llb}{\textbeltl}							% nahuatl l
\newcommand{\jn}[1]{\texttoptiebar{#1}}		% top tie for affricates
\newcommand{\tsa}[1]{\textsubarch{#1}}
\newcommand{\glot}{\textglotstop}					% glottal stop
\newcommand{\tss}[1]{\textsubscript{#1}}
\newcommand{\nf}[1]{\normalfont {#1}}
\newcommand{\tsup}[1]{\textsuperscript{#1}}
\newcommand{\ird}[1]{\textit{#1}}					% text in Iridian
\newcommand{\foreign}[1]{\textit{#1}}         % text in another language
%\newcommand{\ird}[1]{\colorbox{pink}{\textit{#1}}} % for drafts        % text in Iridian
\newcommand{\dto}{o\textsubarch{u}}				% ou diphthong
\newcommand{\dte}{e\textsubarch{I}}				% ei diphthong
\newcommand{\rec}[1]{*\textit{#1}}				% reconstruction in Old Iridian
\newcommand{\trsl}[1]{`#1'}								% translation in quotes
\newcommand{\rrr}{\textinvscr}						% uvular r
\newcommand{\irdp}[2]{\ird{#1}, \trsl{#2}}
\newcommand{\ttla}[1]{`#1'}								% title of an article, short story setcounter
\newcommand{\ttlb}[1]{\emph{#1}}					% title of a book
\newcommand{\irdt}[3]{\textit{#1} \nt{#2}, \trsl{#3}}
\newcommand{\orth}[1]{$\langle$#1$\rangle$}
\newcommand{\phon}[3]{\ird{#1} [#2] \trsl{#3}}
\newcommand{\su}[1]{$_{#1}$}
\newcommand{\SubjI}{\Subj{}.\Ipf{}}
\newcommand{\SubjP}{\Subj{}.\Pf{}}






% TODO: Rewrite instances of this command and delete
\newcommand{\asp}[1]{{#1}\textsuperscript{h}}

%% Set block quotes to font size \small

\expandafter\def\expandafter\quote\expandafter{\quote\small}

%% Formatting of linguistic glosses

\lingset{Everyex=\small\citereset,everygla=\it,everyglb=\footnotesize,everyglft=\footnotesize,aboveexskip=5pt,aboveglftskip=0pt,belowexskip=5pt}


% Possessive citations like "Doe's (1992)"
% From C Becker
% https://tex.stackexchange.com/a/307461
\DeclareNameFormat{labelname:poss}{% Based on labelname from biblatex.def
  \nameparts{#1}% Not needed if using Biblatex 3.4
  \ifcase\value{uniquename}%
    \usebibmacro{name:family}{\namepartfamily}{\namepartgiven}{\namepartprefix}{\namepartsuffix}%
  \or
    \ifuseprefix
      {\usebibmacro{name:first-last}{\namepartfamily}{\namepartgiveni}{\namepartprefix}{\namepartsuffixi}}
      {\usebibmacro{name:first-last}{\namepartfamily}{\namepartgiveni}{\namepartprefixi}{\namepartsuffixi}}%
  \or
    \usebibmacro{name:first-last}{\namepartfamily}{\namepartgiven}{\namepartprefix}{\namepartsuffix}%
  \fi
  \usebibmacro{name:andothers}%
  \ifnumequal{\value{listcount}}{\value{liststop}}{'s}{}}
\DeclareFieldFormat{shorthand:poss}{%
  \ifnameundef{labelname}{#1's}{#1}}
\DeclareFieldFormat{citetitle:poss}{\mkbibemph{#1}'s}
\DeclareFieldFormat{label:poss}{#1's}
\newrobustcmd*{\posscitealias}{%
  \AtNextCite{%
    \DeclareNameAlias{labelname}{labelname:poss}%
    \DeclareFieldAlias{shorthand}{shorthand:poss}%
    \DeclareFieldAlias{citetitle}{citetitle:poss}%
    \DeclareFieldAlias{label}{label:poss}}}
\newrobustcmd*{\posscite}{%
  \posscitealias%
  \textcite}
\newrobustcmd*{\Posscite}{\bibsentence\posscite}
\newrobustcmd*{\posscites}{%
  \posscitealias%
  \textcites}

%%Formatting of section headings

\makeatletter
\def\thickhrulefill{\leavevmode \leaders \hrule height 1ex \hfill \kern \z@}
\def\@makechapterhead#1{%
	\vspace*{10\p@}%
	{\parindent \z@ \centering \reset@font
		{\Large \CardoSmallCaps \thechapter}
		\par
		\vspace*{1\p@}%
		\interlinepenalty\@M
		\setlength{\arrayrulewidth}{2pt}
		\par\noindent
		\rule{24pt}{2pt}
		\\
		\begin{tabular}{@{\qquad}c@{\qquad}}

			\\
			{\LARGE \CardoSmallCaps \MakeLowercase{#1}\par\nobreak} \\
			\\

		\end{tabular}
		\vskip 90\p@
}}
\def\@makeschapterhead#1{%
	\vspace*{\p@}%
	{\parindent \z@ \centering \reset@font
		{\Large \CardoSmallCaps \vphantom{\thechapter}}
		\par\nobreak
		\vspace*{15\p@}%
		\interlinepenalty\@M
		\setlength{\arrayrulewidth}{2pt}
		\par\noindent
		\rule{24pt}{2pt}
		\\
		\begin{tabular}{@{\qquad}c@{\qquad}}

			\\
			{\LARGE \CardoSmallCaps \MakeLowercase{#1}\par\nobreak} \\
			\\

		\end{tabular}
		\vskip 80\p@
}}


%% Formatting of sections
\titleformat{\section}
{}{\thesection}{1em}{\CardoSmallCaps\large\lowercase}
\titlespacing*{\section}
{0pt}{3ex plus 1ex minus .2ex}{4pt}

%% Formatting of subsections
\titleformat{\subsection}[hang]
{}{\thesubsection}{1em}{\CardoSmallCaps\small\lowercase}
\titlespacing*{\subsection}
{0pt}{3ex plus 1ex minus .2ex}{4pt}


%% Formatting of subsubsections
\titleformat{\subsubsection}
{\small}{\thesubsubsection}{1em}{\small\itshape}

\usepackage{titletoc}

% default definition of the toc format of sections
%\titlecontents{section}
%[3.8em] % left
%{}  % above code
%{\contentslabel{2.3em}}    % numbered-entry-format
%{\hspace*{-2.3em}}         % numberless-entry-format
%{\titlerule*[1pc]{.}\contentspage} % filler-page-format
%[]      % below code

% Define partial toc for part pages
%\contentsmargin{0cm} % Removes the default margin
% Chapter text styling
\titlecontents{chapter}[1cm] % Indentation
{\addvspace{8pt}\scshape} % Spacing and font options for chapters
{\contentslabel[\thecontentslabel]{1cm}} % Chapter number
{}  
{\normalfont\quad\thecontentspage} % Page number
% Section text styling
\titlecontents{section}[2cm] % Indentation
{\addvspace{2pt}\small} % Spacing and font options for sections
{\contentslabel[\scshape{\small\thecontentslabel}]{1.0cm}} % Section number
{}
{\quad\small\thecontentspage} % Page number
[]
% Subsection text styling
\titlecontents{subsection}[3cm] % Indentation
{\addvspace{1pt}\small} % Spacing and font options for subsections
{\contentslabel[\thecontentslabel]{1cm}} % Subsection number
{\quad}
{\quad\small\thecontentspage} % Page number
[] 
%\contentsmargin{2cm}
\titlecontents{figure}[2cm] % Indentation
{\addvspace{2pt}\small} % Spacing and font options for sections
{\contentslabel[\textsc{\small\thecontentslabel}]{1.0cm}} % Section number
{}
{\quad\small\thecontentspage} % Page number
[]
\titlecontents{table}[2cm] % Indentation
{\addvspace{2pt}\small} % Spacing and font options for sections
{\contentslabel[\textsc{\small\thecontentslabel}]{1.0cm}} % Section number
{}
{\quad\small\thecontentspage} % Page number
[]
% Section text styling
\titlecontents{lsection}[0em] % Indendating
{\footnotesize\sffamily} % Font settings
{}
{}
{}
% Subsection text styling

\titlecontents{lsubsection}[.5em] % Indentation
{\normalfont\footnotesize\sffamily} % Font settings
{}
{}
{}
% Thanks to egreg, for providing this code at
% http://tex.stackexchange.com/questions/101773/write-to-back-page-of-part


%% Formatting of page headers

\renewcommand{\chapterpagestyle}{empty}%The first page in each chapter won't have any heading or footer, especially no page number
\pagestyle{fancy}
\renewcommand{\chaptermark}[1]{\markboth{#1}{}}
\renewcommand{\sectionmark}[1]{\markright{#1}}
\renewcommand{\headrulewidth}{0pt}
\fancyhf{}
\fancyhead[RE]{{{\CardoSmallCaps\footnotesize\MakeLowercase{\leftmark}}}\hfill{\footnotesize\thepage}}
\fancyhead[LO]{{\footnotesize\thepage}\hfill{\CardoSmallCaps\footnotesize\MakeLowercase{\rightmark}}}
\cfoot{}


% Glosses in footnotes
\newcounter{fnexno}
\setcounter{fnexno}{1}
\definelingstyle{fnex}{
% exno=\footnotesize\roman{fnexno}\stepcounter{fnexno},
  exno=\roman{fnexno}\stepcounter{fnexno},
  everyex=\footnotesize,
  numoffset=\footnotemargin,
  aboveexskip=2ex,
  belowglpreambleskip=-1ex,
  interpartskip=0.5ex,
  aboveglftskip=-1ex,
  belowexskip=0ex,
}

%% Custom column types
\newcolumntype{Y}{>{\RaggedRight\arraybackslash}X}
\newcolumntype{Z}{>{\RaggedRight\everypar{\hangindent=1em}\arraybackslash}X}
\newcolumntype{M}{>{\centering\arraybackslash}X}
\newcolumntype{T}[1]{S[table-format=#1]}

% Restart numbering each chapter
\pretocmd{\chapter}{\excnt=1}{}{}

\usepackage[nogroupskip,acronym,nomain]{glossaries}
\setglossarysection{section}
\newglossarystyle{myglosses}{%
  \renewenvironment{theglossary}%
	{\begin{multicols}{2}\raggedright}
	{\end{multicols}}

	    \renewcommand*{\glossaryheader}{}
	    \renewcommand*{\glsgroupheading}[1]{}
	    \renewcommand*{\glsgroupskip}{}
	    \renewcommand*{\glsclearpage}{} 

	     % set how each entry should appear:
	      \renewcommand*{\glossentry}[2]{
	       \noindent\makebox[4em][l]{\glstarget{##1}{\CardoSmallCaps{\glossentryname{##1}}}}
	        \glossentrydesc{##1}\par
	    }


	    \renewcommand*{\subglossentry}[3]{%
	        \glossentry{##2}{##3}
	    }
	}
\newglossarystyle{langlist}{%
  \renewenvironment{theglossary}%
	{\begin{multicols}{2}\raggedright}
	{\end{multicols}}

	    \renewcommand*{\glossaryheader}{}
	    \renewcommand*{\glsgroupheading}[1]{}
	    \renewcommand*{\glsgroupskip}{}
	    \renewcommand*{\glsclearpage}{} 

	     % set how each entry should appear:
	      \renewcommand*{\glossentry}[2]{
	       \noindent\makebox[4em][l]{\glstarget{##1}{{\glossentryname{##1}}}}
	        \glossentrydesc{##1}\par
	    }


	    \renewcommand*{\subglossentry}[3]{%
	        \glossentry{##2}{##3}
	    }
	}

\usepackage[glosses,nonumberlist,mcolblock]{leipzig}
\renewcommand*{\leipzigfont}[1]{{\CardoSmallCaps{#1}}}
\usepackage{multicol}

\newleipzig{Av}{av}{active voice}
\newleipzig{Pf}{pf}{perfective}
\newleipzig{Pv}{pv}{passive voice}
\newleipzig{Agt}{agt}{agent}
\newleipzig{Lnk}{lnk}{linking particle}
\newleipzig{Dim}{dim}{diminutive}
\newleipzig{Cont}{cont}{continuous}
\newleipzig{Ctp}{ctpv}{contemplative}
\newleipzig{N}{n}{negative}
\newleipzig{Exst}{exst}{existential particle}
\newleipzig{Quot}{quot}{quotative}
\newleipzig{Qp}{qp}{quotative particle}
\newleipzig{Pot}{pot}{potential mood}
\newleipzig{Sbj}{sbj}{subjunctive mood}
\newleipzig{Ipf}{ipf}{imperfect}
\newleipzig{Cv}{cv}{converb}
\newleipzig{Rz}{rz}{relativizer}
\newleipzig{Nz}{nz}{nominalizer}
\newleipzig{Att}{att}{attributive}
\newleipzig{Sup}{sup}{supine}
\newleipzig{Foc}{foc}{focus}
\newleipzig{Hon}{hon}{honorific}
\newleipzig{Rec}{rcp}{reciprocative}
\newleipzig{Comp}{comp}{comparative}
\newleipzig{Ade}{ade}{adessive}
\newleipzig{Med}{med}{medial}
\newleipzig{Dist}{dist}{distal}
\newleipzig{Subj}{subj}{subjunctive}
\newleipzig{Str}{str}{strong form}
\newleipzig{Wk}{wk}{weak form}
\newleipzig{Lat}{lat}{lative}
\newleipzig{Ger}{ger}{gerund}
\newleipzig{Pfv}{pfv}{perfective*}
\newleipzig{Nom}{nom}{nominative*}
\newleipzig{Lv}{lv}{locative voice}
\newleipzig{Third}{3}{third person*}
\newleipzig{Hort}{hort}{hortative}
\newleipzig{Acc}{acc}{accusative}
\newleipzig{SupN}{sup.n}{supine of necessity}
\newleipzig{SupP}{sup.p}{supine of purpose}
\newleipzig{Dub}{dub}{dubitative particle**}
\newleipzig{Aff}{affrm}{affirmative particle**}
\newleipzig{Ret}{ret}{retrospective aspect}
\newleipzig{Rep}{rep}{reportative particle**}
\newleipzig{Infer}{infer}{inferential particle**}
\newleipzig{Deb}{deb}{debitive}
\newleipzig{Incp}{incep}{inceptive}
\newleipzig{Soc}{soc}{sociative}



%% language abbreviations

% macro to call full name of language
\newcommand{\printlang}[1]{\acrlong{#1}}
\newcommand{\citelang}[1]{\acrshort{#1}}

% list of abbreviations used
\newacronym{en}{en}{English}
\newacronym{cs}{cz}{Czech}




\makeglossaries




\begin{document}

\author{}
\title{A Reference Grammar of the Iridian Language}
\date{}

\frontmatter


\begin{titlepage}
        \centering
        \vspace{4\baselineskip}
        {\Huge
        A Reference Grammar\\of the Iridian Language\par}
        \vspace{\baselineskip}
        {\Large\itshape Ircevní koštgramátik}

        \vfill
        {\em First Edition}\par
        {2020}
\end{titlepage}

\thispagestyle{empty}

    \vspace*{\fill}
{
\small

\noindent Copyright © 2019 Roel Christian Yambao \bigskip

\noindent First published online, July 2019. Some rights reserved. This document is available for use and distribution under the terms of the MIT Free Software License.

\medskip

\noindent Typeset in Cardo and Source Sans Pro using \LuaLaTeX{}. This {\sc pdf} file was generated on \today{}; this is a pre-release build and may differ from the final version. For more information, visit \url{https://lang.roelchristian.com/grammar/git}.

}
				% Colophon

\tableofcontents

\cleardoublepage

\listoftables
\addcontentsline{toc}{chapter}{List of Tables}

\listoffigures
\addcontentsline{toc}{chapter}{List of Figures}


\chapter*{Preface}
\addcontentsline{toc}{chapter}{Preface}

Tolkien called the art of conlanging his ``secret vice'' 

				% Preface

\chapter*{Abbreviations}
\addcontentsline{toc}{chapter}{Abbreviations}
\printglossary[style=myglosses,type=\leipzigtype, title={Glossing abbreviations}]

\smallskip

{\small
\begin{itemize}
\item[*] These are glosses for grammatical terms seen in examples from other languages that are not in use in Iridian.
\item[**] These do not represent grammatical categories used in Iridian but were chosen \emph{hanc ob causam} to approximate the grammatical function of various particles.
\end{itemize}}

\bigskip
\printglossary[style=langlist,type=\acronymtype, title={Abbreviation of language names}]

\clearpage

\section*{Abbreviation of language names}

\begin{longtabu} {YY[2.5]}
	Cz.		& Czech\\
	Eng.	& English\\
	Fr.		& French\\
	Ger.	& German\\
	Gk.		& Greek\\
	Hu.		& Hungarian\\
	It.		& Italian\\
	Lat.	& Latin\\
	OCS		& Old Church Slavonic\\
	Pol.	& Polish\\
	Rus.	& Russian\\
	Sl.		& Common Slavic/Slavonic\\
	Slk.	& Slovakian\\
	Uk.		& Ukrainian\\
\end{longtabu}


\section*{Other Symbols}
\begin{longtabu} {YY[2.5]}
	C 		& consonant\\
	C\sx{j}	& palatalized consonant\\
	D 		& voiced stop\\
	N 		& nasal consonant\\
	P 		& stop\\
	T 		& unvoiced stop\\
	V		& vowel\\
	Ṽ	 	& nasalized vowel\\
	V\tss{u}& unstable vowel\\
	\orth{~}& orthographic representation\\
	/~/		& phonemic transcription\\
	\nt{ }	& phonetic transcription\\
	\sim	& alternates with\\
\end{longtabu}
					% Abbreviations Used

\mainmatter

\chapter{An Overview of Iridian}

\section{A brief history of Iridia}

Iridian is a small republic located in Central Europe, bordering the Czech Republic, Slovakia, Poland, and Ukraine. Home to around 8 million people, Iridia is a relatively small country, but it is a major economic and political power in Central Europe with a long and colorful history. Around 85\% of the population speak Iridian, the national language, and an estimated 94\% of the population speak it as a first or second language. The remaining 15\% of the population, especially near the country's borders, speak various Slavic languages, including Czech, Slovak, Polish, and Ukrainian, while around 20\% of the people in the country's southeast counties (\ird{prest}), near the border with Hungary, speak Hungarian. Immigration from other parts of Europe has also brought speakers of other languages, including Romanian, German, and Albanian, to the country.

The English name Iridia is from the Latinized form of the medieval endonym \ird{Irdzaume} (cf. the modern \ird{Ircome}) meaning ``Land of the Irdz (Irc).''\footnote{
    Translations in Romance and non-European languages usually follow this Latinized version. See for example, French \foreign{Iridie}, Portuguese \foreign{Irídia} or Korean \foreign{{\begin{CJK}{UTF8}{mj}일리지야\end{CJK}} (`Illijiya')}. Most Central and Eastern European languages use an exonym closer to the Iridian endonym: see, for example, German \foreign{Irtzland}, Czech Jírice or Polish \foreign{Ircja}.}
The origin of the word \ird{Irdz} is uncertain, but it has been used to refer to the Iridian people since at least late Roman times, although it is unclear whether the use of the term referred to the actual Iridian people or to the other groups living in that area during that time. The Iridians have been living in the region of what is now Eastern Europe since before the Indo-European migrations and were believed to have been the dominant people in the area until the arrival of the Indo-Europeans. Not much is known about the initial contacts of the two cultures, but there is wide evidence of extensive cultural exchanges between the Iridian tribes and the arriving Indo-European peoples. It is unknown how much Iridian culture has influenced the Indo-European settlers and vice versa but it is clear that the two cultures have influenced each other in many ways simultaneously, with the influence of Iridian more being especially more pronounced in the development of the neighboring Slavic languages, with the Iridian language itself being influenced by the Slavic languages in much the same way.

Iridia as a nation state howver has been a relatively new development. The polity can trace its origins to the Kingdom of Iridia which governed what is now the Iridian Republic, Czech Republic, Slovakia and southern Poland from the tenth to the early 12th century. For the much of the next 500 years, the area was ruled by a series of Germanic and Slavic kingdoms, including the Kingdom of Bohemia, Kingdom of Hungary, Kingdom of Poland, and the Kingdom of Galicia-Volhynia. From the 16th and 17th century the area was ruled as the Principality of Iridia, a vassal state of the Kingdom of Hungary, and then as the Voivodeship of Iridia, within the Austrian and later the Austro-Hungarian Empire.

After the dissolution of the Austro-Hungarian Empire in 1918, most of the area of the present-day Iridian republic declared independence as the Ruginese Republic (from Rugina, the contemporary English name of the nation's capital Roubže). This independence was short-lived and in 1938, the republic was annexed by Germany, a day after the annexation of its neighboring Czechoslovakia.

After the end of the Second World War, the area was occupied by the Soviet Union, which established the Iridian People's Republic (IPR). The republic was a member of the Warsaw Pact and was a satellite state of the Soviet Union. In 1968, the IPR was a part of the invasion of Czechoslovakia by the Soviet Union and its allies, which was known as the Prague Spring. The invasion was met with resistance from the Iridian people and the Soviet Union was forced to withdraw. This however cooled the relation between the IPR and the neighboring Czechoslovakia, an impasse that almost led to a full scale war between the two countries in 1979 after continued provocation from then-president of IPR Jozip Enta (who ruled the country for much of the 1970s). Discontent with the existing regime continued from the 1980s to the 1990s, culminating in the eventual collapse of the Soviet Union and the IPR in 1991 and 1992 republic. The Iridian people then voted to establish a parliamentary republic in 1992, with the first democratic elections being held in 1993. The country, which is officially known as the Iridian Republic (\ird{Ircevní respublika}), has since been a member of the European Union, the Visegrád Group, NATO, and the OECD.

\begin{table}
\centering \begin{tabular}{ll} \hline
Capital & Roubže \\
Major cities & Brest, Preždy, Kum, Štětín, Žilina \\
Official language & Iridian \\
Population & 8 million (2019) \\
Area & 53,244 km$^2$ \\
Water (\%) & 1.5 \\
GDP (nominal) & 422 million USD (2019) \\
GDP (nom. per capita) & 53,000 USD (2019) \\
Currency & Iridian koruna (IRK) \\
HDI & 0.919 - very high (2018) \\
Time zone & UTC+1 \\
Head of State & President Luka Anec \\
Head of Government & Prime Minister Mila Ormi \\
Legislature & Parliament (unicameral)\\
\hline
\end{tabular}
\caption{Basic facts about Iridia}
\label{tab:fact-sheet}
\end{table}


\section{The Iridian Language}

The Iridian language (\ird{ircevní malno}) is a language isolate belonging to the Iridian language family. It is an agglutinative language with a rich inflectional system mainly in the form of suffixes, and a relatively large number of phonemes.


\section{Word Classes}\label{sec:wordclasses}
Traditional Iridian grammar classifies words into four main classes: \irdp{min\v{e}c}{nouns}, \irdp{hlout\v{e}c}{verbs}, \irdp{prid\v{e}c}{modifiers}, and \irdp{zvuk}{function words}. We will follow this system for much of this book, introducing deviations to the system where appropriate to the discussion at hand.


\chapter{Phonology}\label{ch:phon}

\section{Introduction}

This chapter provides an overview of the phonology of Iridian. The phonetic
descriptions provided here are in IPA based on the standard dialect of Iridian
(as spoken in Roubže and surrounding areas). Divergent phonologies, both within
the Roubže dialect itself and the various dialects inside and outside
Iridia, are discussed in detail in Appendix \ref{ch:dialects}.

\section{Phonetic and phonemic notation}\label{sec:notation}

The phonetic descriptions in this chapter are based on the standard dialect of
Iridian, which is itself based on the Roubže dialect. Phonetic notation uses a
single symbol to represent one and only one sound; in this book, it is based on
the International Phonetic Alphabet (IPA) and appears between square brackets,
e.g., [pʲæɕtäː]. Phonemic notation, on the other hand uses a single symbol to
represent one and only one phoneme; in this book, this appears between forward
slashes, e.g., /pʲaɕtaː/. The phonemic transcription is used to represent the
underlying form of a word, while the phonetic transcription is used to represent
the actual pronunciation of a word. Citations in the Iridian language appear
italicized, and their English translations, if given, appear in single quotes
following the Iridian text, e.g., \irdp{piaštá}{to eat.}

\section{Vowels}\index{vowel}\label{sec:vowels}

\subsection{Oral vowels}\index{vowel!oral}

\begin{table}\index{vowel!inventory}
	\footnotesize\sffamily
	\caption{Vowel inventory of standard Iridian.}
	\medskip
	\begin{tblr}{width=0.6\textwidth,colspec={XXXX}}
		\toprule\addlinespace
					& {\sc front}	& {\sc central}	& {\sc back}	\\ \addlinespace
		\midrule\addlinespace
		Close 		& ɪ\,i 			& (ɨ)			& ʊ\,uː			\\ \addlinespace
		Mid 		& ɛ\,eː 		& 				& ɔ\,oː			\\ \addlinespace
		Open 		& 				&(ɐ)			& a\,aː 		\\ \addlinespace
		\bottomrule
		\label{table:vowels}{}
	\end{tblr}
\end{table}{}

Iridian has five pairs of corresponding long and short vowels. With the
exception of /a\,aː/, long vowels are tenser than their short counterparts. In
addition standard Iridian also features the high central vowel [ɨ] as an
allophone of /ɛ/ and /ɪ/ and the low central [ɐ] as an allophone of /a/, in
unstressed positions. Phonetic realization is generally consistent with
orthography as seen in Table \ref{table:vowels-orth} below.

\begin{table}
	\footnotesize\sffamily
	\caption{Orthographic representation of vowels.}
	\medskip
	\begin{tblr}{width=0.7\textwidth,colspec={XXXXXX}}
		\toprule \addlinespace
		& {\sc short} & {\sc long} & & {\sc short} & {\sc long}\\ \addlinespace
		\midrule \addlinespace
		/a/ & a 	&á 			& /o/ 	& o &ó 	\\ \addlinespace
		/e/ & e 	&é 			& /u/ 	& u &ú	\\ \addlinespace
		/i/ & i,\,y &í,\,ý 		& 		& 	&	\\ \addlinespace
		\bottomrule
		\label{table:vowels-orth}
	\end{tblr}
\end{table}


Both ⟨i⟩ and ⟨y⟩ and their long counterparts ⟨í⟩ and ⟨ý⟩ represent the high
front vowel /i/. ⟨y/ý⟩ originally represented the high front rounded vowel /y/
(with the short /y/ realized as the tenser near-close near-front rounded vowel
[ʏ]) but the pronunciation gradually shifted to the central front vowel [ɨ]
before finally settling to /i/ in the 14th or 15th century. As in
Czech\index{Czech} orthography, ⟨i, í⟩ causes the palatalization of the
preceding consonant. The same distinction is found between the palatalising ⟨ě⟩
(another Czech loan originally written in Old Iridian as ⟨je⟩) and the normal
⟨e⟩. This is discussed further in the orthography section (\S\,\ref{sec:ortho}).

The short vowels /ɛ/ and /ɪ/ are reduced to [ɨ] in unstressed positions. In less
careful speech, this could cause the elision of the vowel and the formation of
consonant clusters or the realization of the preceding consonant as syllabic
(especially if it is a liquid). Final /ɛ/ is not reduced in a word-final
position if preceding a pause.

\ex
	\irdp{a mert}{and the dead one} [ˈʔämɨɾt̚ ] or [ˈʔämɾ̩t̚] but\\
	\irdp{akuzace}{accusation} [ˈʔäxʊzɐt͡sɛ]
\xe

The low vowel /a/ is realized as the open central unrounded vowel /ä/.
Stressed /a/ is realized as [\ae] between palatal consonants, further reduced to
[ɨ] when unstressed, e.g., \ird{piaštá} ['pʲæɕtäː] vs. \ird{nepiaštá}
[ˈnɛpʲɨɕtäː]. Elsewhere /a/ is pronounced [ɐ] when in an unstressed position,
although some dialects may further reduce it to a [ə].

In most Eastern dialects, especially those from near the border with Poland, the
long mid vowels /eː/ and /oː/ has merged with /iː/ and /uː/, respectively.

\subsection{Diphthongs}\index{diphthong} Iridian has three phonemic oral
diphthongs: \ird{au}\,/au̯/, \ird{ei}\,/eɪ̯/ and \ird{ou}\,/ou̯/. In addition,
the diphthongs \ird{oi}\,/ɔɪ̯/ and \ird{ui}\,/uɪ̯/  also occur phonetically, but
their occurence is marginal, normally appearing only in fixed expressions
(mostly interjections and expletives), such as \irdp{Avui}{Damn it!} [ʔɐˈʋuɪ̯ʔ],
\irdp{pšehui}{annoying} [ˈpʲɕɛxuɪ̯ʔ] and \irdp{Oi}{Hey!} [ʔɔɪ̯ʔ].

In most dialects the diphthong /eɪ̯/ has almost completely merged with \ird{é}
/eː/, although some divergent dialects in the south may realize the diphthong as
[iː] (e.g., \irdp{neite}{word} /ˈneɪ̯tɛ/ but realized as [ˈneːtɛ] or ['ɲiːtɛ]).

\subsection{Vowel Length}\index{vowel length}\index{long vowel|see{vowel
length}}

Vowel length is phonemic in Iridian. Length is represented by an acute
accent\index{acute accent} over the long vowel. The short-long vowel pairs
differ in quality as well as length, with the short vowels being more lax and
the long vowels being tenser in addition to being longer.

\begin{table}
	\footnotesize\sffamily
	\caption{Vowel length and quality.}
	\medskip
	\begin{tblr}{width=0.7\textwidth,colspec={XXX}}
		\toprule
		{\sc archiphoneme} & {\sc lax/short} &{\sc tense/long}\\ \midrule
		/a/	& [ä]	& [äː]		\\
		/e/	& [ɛ]	& [eː]		\\
		/i/	& [ɪ]	& [iː]		\\
		/o/	& [ɔ]	& [oː]		\\
		/u/	& [ʊ] & [uː]		\\
		\bottomrule
	\end{tblr}
\end{table}

Below are some examples of minimal pairs with long and short vowels.

\pex
\vtop{\halign{%
#\hfil& \qquad  #\hfil\cr
\phon{sam}{säm}{barn}			 & \phon{sám}{säːm}{frog} \cr
\phon{mate}{mätɛ}{spoon}		 & \phon{máte}{mäːtɛ}{check mate} \cr
\phon{se}{sɛ}{glass}			& \phon{sé}{seː}{pulp} \cr
\phon{mel}{mɛw}{honey}			& \phon{mél}{meːw}{straw} \cr
\phon{jite}{jɪtɛ}{sheet}		& \phon{jíte}{jiːtɛ}{shade} \cr
\phon{ton}{tɔn}{tongue}			& \phon{tón}{toːn}{tone} \cr
\phon{mur}{mʊr}{gall}			& \phon{múr}{muːr}{mural} \cr
}}

\xe

\section{Consonants}\index{consonants}\label{sec:consonants}

Table \ref{table:fullconsonant} shows a complete list of consonant phonemes in
Standard Iridian, with the allophones appearing in parentheses. In total,
Iridian has 19 consonant phonemes but with 21 additional allophonic variants.
\begin{table}
	\footnotesize\sffamily
	\caption{Full consonant inventory of standard Iridian.}\label{table:fullconsonant}
	\medskip
	\begin{tblr}{width=\linewidth,colspec={X[1.8]XXXX}}
		\toprule\addlinespace
						& {\sc labial}	& {\sc alveolar}	& {\sc palatal}	& {\sc velar}	\\ \addlinespace
		\midrule\addlinespace
		Plosive			& p~b			& t~d				& c~ɟ 			& k~ɡ 			\\ \addlinespace
		Nasal			& m~(ɱ)			& n					& ɲ				& (ŋ)			\\ \addlinespace
		Liquid			&				& ɾ~(ʁ)~l			&	ʎ			&				\\ \addlinespace
		Sib. Fric.		& 				& s~z	  			& ɕ~ʑ			&				\\ \addlinespace
		Non-Sib. Fric.	& ʋ				&					& (ç) 			& x~ɣ   		\\ \addlinespace
		Sib. Affricate  &				& t͡s~(d͡z)			  & t͡ɕ~(d͡ʑ)		&				\\ \addlinespace
		Non-Sib. Aff. 	&				& 					&			  	& (k͡x~g͡ɣ)		  \\ \addlinespace
		Approximant 	& (β̞)  		& (ð̞)				  & j			  & (ʍ~w)		  \\ \addlinespace
		\bottomrule
	\end{tblr}
\end{table}


\subsection{Plosives}

Initial velar stops are affricated when following a pause, so that the pair
/k~ɡ/ is often realized as [k͡x~ɡ͡ɣ]. Some Southeastern dialects, however,
normally realize initial velar stops as aspirated [kʰ~ɡʰ] instead. This sound
change can be traced to the initial aspirated stops \rec{\asp{k}},
\rec{\asp{g}}, \rec{\asp{t}} and \rec{\asp{d}} in Old Iridian weakening to
affricates.\footnote{Old Iridian \rec{\asp{t}} and \rec{\asp{d}} became the
Middle Iridian [t̪͡θ̞ ~d̪͡ð̞] but both have since simplified to /t~d/ in modern
Iridian.} The labial stops /{p~b}/ are unaffected by this process as most
instances of \rec{\asp{p}} and \rec{\asp{b}} have merged to /b/ or /ʋ/ in modern
Iridian.

The velar stops /k~ɡ/ are lenited to the velar fricatives [x~ɣ]
intervocalically, before a voiceless stop, after a vocalised l if followed by
another vowel or a voiceless stop, or before the nasal consonants /n/ or /m/ if
following a vowel immediately. This lenition also occurs word-finally unless
followed by a voiced obstruent, in which case, subject to word-final devoicing,
they merge to [x]. The voiced /ɡ/ itself has a limited distribution, mostly
appearing in consonant clusters with liquids or nasals. Older loanwords (mainly
Slavic, but to a lesser extent Germanic and Hungarian ones) that contain /ɡ/ in
the original language have often been assimilated as \orth{h} in Iridian. (Cf.
for example, \irdp{hrác}{athlete} and Polish \foreign{grać} or Russian
\foreign{\cyrtext играть}.)

This lenition can also be observed with the voiced stops /b/ and /d/ which
become the approximants [β̞	] and [ð̞] (written without the diacritic hereafter)
intervocalically or between a vocalised /l/ and another vowel. Both /b/ and /d/
and the marginal /g/ may be realized with a nasal release at the beginning of a
word when following a pause, i.e., as [ᵐb], [ⁿd] and [ᵑɡ],
respectively.\footnote{Prenasalized stops are unattested in the Roubže dialect
and there is strong evidence that this process is slowly dying out in the other
dialects as well.} 

The glottal stop [ʔ] is often not regarded as a separate phoneme. It can occur
in three cases: (1) before an onset vowel when following a pause, e.g.,
\irdp{avt}{car} [ʔäft]; (2) between two vowels that do not form a diphthong,
e.g., \irdp{naomá}{laundry} ['näʔɔmäː]; or (3) emphatically, especially in
interjections, e.g., \irdp{Oi}{Hey!} [ʔɔɪ̯ʔ], \irdp{Káp!}{Look out!}
\emph{lit.}, \trsl{danger} [k͡xäpʔ].

\subsection{Nasals}
Iridian has three nasal consonants /m~n~ɲ/. /n/ cannot appear before bilabials
and similarly /m/ cannot appear before velars. Both /m/ and /n/ are realized as
[m] before either /ʋ/ or /f/. Before velars /n/ is consistently realized as [ŋ],
although [n] is also possible in emphatic pronunciation or in word boundaries.
The distribution of word-final /n/ is quite limited when compared to word-final
/m/. When assimilating foreign words with final /n/ or /ŋ/, both nasals usually
surface as an /m/ in the new loanword, e.g., \irdp{bedautum}{definition} from
Ger. \foreign{Bedeutung}. 

The velar [ŋ] is not phonemic in Iridian but can sometimes be observed,
especially in loanwords, where it can be realized as nasalization of the
preceding vowel when in the syllable coda or as [ŋ] intervocalically, although
[ŋɡ] or [ŋk] is also common. Thus, for example, \irdp{anglevní}{English} can be
realized as either [ˈɐ̃w̃lɛʋɲiː] or [ˈäŋlɛʋɲiː] or [ˈäŋɡlɛʋɲiː] in order of
currency.

\subsection{Liquids}

Iridian has two liquids: the rhotic /r/ and the lateral /l/.

The rhotic /r/ is realized as the tap [ɾ], although some speakers may pronounce
it as a trill [r], especially in emphatic pronunciation. Both these
pronunciations are transcribed as [r] in this book. In the coda position /r/ is
devoiced to [r̥].

The lateral /l/ is  the velarised alveolar lateral approximant [ɫ]. Nonetheless
the sound has been transcribed throughout as [l]. In the coda position /l/ is
completely vocalized in standard Iridian, becoming [w]. Most southern dialects
nevertheless retain the pronunciation as [ɫ] in this position. The palatalised
/lʲ/ is the palatal lateral approximant [ʎ] and is transcribed as such.

\subsection{Fricatives and Affricates}

The palatal sibilants /ɕ~ʑ/ can be realized as either the palatal [ɕ~ʑ] or the
post-alveolar [ʃ~ʒ] with the former being more common. The same is true with the
palatal affricates /t͡ɕ~d͡ʑ/, realized as either [t͡ɕ~d͡ʑ] or [t͡ʃ~d͡ʒ], with
the former also being more prevalent. In any case, however, this book treats
/ɕ~ʑ/ and /t͡ɕ~d͡ʑ/ as palatals regardless of the actual realization.

The sequence /t͡sɪ/ and /t͡si:/ are realized as [t͡ɕɪ] and [t͡ɕiː] respectively
(viz., \irdp{cigra}{tiger} is realized as [ˈt͡ɕɪɣɾɐ] and not [ˈt͡sɪɣɾɐ]). The
stop fricative sequence [tɕ] can occur in syllable boundaries, although as form
of hypercorrection most speaker may lengthen the initial stop to [tːɕ] or
aspirate it (becoming [tʰ.ɕ]) to further distinguish it from /t͡ɕ/ (cf. e.g.,
\irdp{otša}{cart} [ˈʔɔtːɕɐ] vs \irdp{oča}{bear} [ˈʔɔt͡ɕɐ]).

The voiced affricates /d͡z/ and /d͡ʑ/, written \orth{dz} and \orth{dž},
respectively, are both marginal phonemes. They normally occur as voiced
allophones of  /t͡s/ and /t͡ɕ/ before voiced obstruents. They do occur
phonemically in a few words, though, mostly in loanwords. Nonetheless, in spoken
Iridian loanwords containing [d͡ʑ] or [d͡ʒ] (mostly from English) are realized
by speakers as [ʑ] (e.g., \irdp{džíns}{jeans} [dʑiːns] or more commonly just
[ʑiːns]).

The voiceless labial fricative /f/ is another marginal phoneme, appearing
usually as an allophobe of /ʋ/. Loanwords containing /f/ generally assimilate to
/ʋ/, although most recent borrowings tend to keep the marginal /f/ (cf.
\irdp{Vranca}{France} [vɾant͡sɐ] vs. \irdp{Feizbuk}{Facebook} [feːzbʊx]).

The approximant /ʋ/ is realized as [v] in onsets before vowels and voiced
obstruents (e.g., \irdp{vdinice}{I thought I saw.} [ˈvɟɪnɨt͡sɛ]), as [f] in
onsets before voiceless obstruents (e.g., \irdp{vternou}{bicycle} [ˈftɛɾnou̯]),
and as [ʋ] or [u̯] in coda and elsewhere (e.g., \irdp{pilav}{pilaf} [ˈpʲɪɫäʋ]
or [ˈpʲɪɫäu̯]). The sequence /kʋ/ and /ɡʋ/ is further lenited to the labialised
velar fricatives [xʷ~ɣʷ]. The voiceless [xʷ] (from both \orth{kv} and \orth{hv})
is in free variation with [ʍ], with the latter being the more common
pronunciation, especially among younger speakers. For simplicity both [xʷ] and
[ʍ] will be transcribed as [ʍ].

Modern Iridian has lost the distinction between /h/ and /x/, with both \orth{ch}
and \orth{h},\footnote{Most instances of \orth{ch} have been replaced with
\orth{h} following various spelling reforms.} historically representing /x/ and
/h/, respectively, merging to the velar fricative /x/. This becomes /ç/ before
voiceless stops word-initially or when following a front vowel, or before the
front vowels /i/ and /ɪ/. The sequence \orth{hl} and \orth{kl} are realized as
/t͡ɬ/.

\section{Phonotactics}\index{phonotactics}\label{sec:phonotactics}

\subsection{Syllable structure}\index{syllable
structure}\label{sec:syllable-structure}

Ignoring the possible complexity of the onset, nucleus or coda, the basic
structure of an Iridian syllable is CV(C), with C representing a consonant and V
a vowel.\footnote{An alternative view, founded upon the status of the glottal
stop as a non-phoneme in Iridian, would be to consider the basic structure as
(C)V(C) instead of CV(C), thus allowing for a null onset. This treats the
addition of a glottal stop in word-initial syllables starting with a vowel as
mere prothesis.} Iridian has relatively few phonotactic constraints, allowing,
at a maximum, syllables of the form CCCCVCCC. Nevertheless, most syllables fall
in either of the five groups CV, CVC, CCV, CCVC and CVCC

\begin{table}
	\footnotesize\sffamily
	\caption{Blevin's criteria as they apply to Iridian.}
	\medskip
	\begin{tblr}{width=0.6\textwidth, colspec={XX}}
		\toprule \addlinespace
		& {\sc parameter}\\ \addlinespace
		\midrule \addlinespace
		Obligatory onset & Yes\\ \addlinespace
		Coda & No\\ \addlinespace
		Complex onset & Yes\\ \addlinespace
		Complex nucleus & Yes*\\ \addlinespace
		Complex coda & Yes\\ \addlinespace
		Edge effect & \\ \addlinespace
		\bottomrule
	\end{tblr}
\end{table}


\subsection{Onset}

All consonant and vowel phonemes can appear in a syllable's onset. Iridian does
not allow a null onset (vowel in the syllable onset), i.e., the most basic
Iridian syllable should be of the form CV. Words that superficially appear as
having a null onset syllable in the initial position are actually preceded by a
glottal stop. An epenthetic glottal stop is also added between vowels in a
sequence that do not otherwise form dipthongs, or before a vowel in a
word-initial position in loanwords. Despite this, vowel-words are significantly
rarer in comparison to consonant-initial ones.

\ex
Prothetic [ʔ] in native Iridian words:\\
\irdp{a}{and} [ˈʔä]\\
\irdp{umielá}{to get drunk} [ˈʔʊmʲɨläː]\\
\irdp{eg}{eyes} [ʔɛx]
\xe

\ex
Prothetic [ʔ] in loanwords:\\
\irdp{Americe}{Amerika} [ˈʔämɨɾʲɪt͡sɛ]\\
\irdp{autobus}{bus} [ˈʔau̯tɔβʊs] \\
\irdp{elefant}{elephant} [ˈɛlɨˌfänt]
\xe

In some eastern dialects, a prothetic [m] is added instead of [ʔ] on words that
begin with vowels after a pause. This never occurs on loanwords or before the
front vowels /e/ and /i/ and has been largely in decline, especially among
younger speakers. With some speakers, the prothetic [m] may be realized as [mw].

\ex
\irdp{umielá}{to get drunk} [ˈmʊmʲɨläː] or [ˈmwʊmʲɨläː]\\
\irdp{očat}{bug} [ˈmɔt͡ɕɐt] or [ˈmwɔt͡ɕɐt]
\xe

A more widespread pattern in colloquial Iridian is the addition of a prothetic
/j/ before the front vowels /e/ and /i/. This phenomenon could be observed in
both native words and loans.

\ex
\irdp{Evrope}{Europe} [ʔɛʋɾɔpɛ], colloq. [jɛʋɾɔpɛ] \\
\irdp{éh}{eyes} [ʔeːx], colloq. [jɛx]\\
\irdp{éšte}{of course} [ˈʔeːɕtɛ], colloq. [ˈjeːɕtɛ]
\xe


The following CC clusters are allowed to be in onset position:

\pex
\a Stop followed by a liquid:\\
/pr/: \irdp{pragy}{sand} [präc]; \irdp{pramou}{petal} [ˈpɾämou̯]\\
/tr/: \irdp{trava}{bread} [ˈtɾävɐ]; \irdp{truk}{ball} [tɾʊx]\\
/kr/: \irdp{krova}{egg} [ˈkɾɔvɐ]; \irdp{kramy}{toe} [kɾämʲ]\\
/pl/: \irdp{plán}{plan} [pläːn]; \irdp{plúka}{knot} [ˈpluːxɐ]\\
/kl/: \irdp{kluk}{foot} [t͡ɬʊx]; \irdp{klúbe}{club} [ˈt͡ɬuːβɛ]\\
/br/: \irdp{bírok}{female teenager} [bʲiːɾɔx]; \irdp{bremy}{prise} [bɾɛmʲ]\\
/dr/: \\
/gr/: \irdp{grec}{flag} [ɣɾɛt͡s]; \irdp{greny}{peace} [ɣɾɛɲ]\\
/bl/: \irdp{bloht}{mud} [blɔxt̚]; \irdp{blau}{neck} [blau̯]\\
/dl/:
\xe

\section{Suprasegmentals}\index{suprasegmentals}

\subsection{Stress}\index{stress} Iridian words generally have a single primary
stress, falling on the first syllable, no matter if the word is simple (e.g.,
\irdp{študent}{student}), derived (e.g., \irdp{študenta}{student, pat.}) or
compound (e.g., \irdp{študentrád}{dormitories}). Most loanwords follow this
general pattern, although more recent borrowings, especially those referring to
proper names, show a greater tendency to keep the phonology of the source
language and not fully assimilate to Iridian's initial stress rule.

\pex
\a Loanwords showing assimilation to word-initial stress:\\
\phon{aristókrat}{ˈäɾɨstoːxɾɐt}{aristocrat}\\
\phon{koruna}{ˈk͡xɔɾʊnä}{crown}

\a Loanwords
\xe

Clitics\index{clitic} are not considered phonologically distinct and are treated
as belonging to the same phonological word as the one after them. These include:

\begin{enumerate}[noitemsep,label=(\alph*)]
	\item Most monosyllabic and some disyllabic prepositions
	\item Most conjunctions:
	\item The pluralizing particle \ird{nie} and the negative particle
	\ird{zám}: 
	\item Demonstratives and the weak form of personal pronouns
\end{enumerate}

\subsection{Intonation}\index{intonation}

\section{Phonological Processes Involving Vowels}

\subsection{Vowel\,\sim\,Zero Alternations}

A vowel\,\sim\,zero alternation occurs when a vowel alternates with zero (i.e.,
gets deleted) in certain morphological contexts. We call this deleted vowel
`unstable' (cf. \cite{siptar2000}, \cite{gussmann2007}). Vowel\,\sim\,zero
alternations in Iridian are virtually all instances of [ɛ] deletion. This
process occurs in roots of the type --C(\sx{j})eC where C is a consonant loses
its [ɛ] when it is followed by a suffix beginning with a vowel.

\ex
\vtop{\halign{%
#\hfil& &\qquad  #\hfil\cr
\irdp{Janek}{Janek} & \irdp{Janka}{Janek (acc.)}	\cr
\irdp{obel}{window} & \irdp{oblu}{window (inst.)} 	\cr
\irdp{pizen}{coin} 	& \irdp{pězní}{coin (inst.)} not \ird{\sx{*}pizní} \cr
}}
\xe

[ɛ] deletion in monosyllabic roots of the type (C)C(\sx{j})eC is subject to
further constraints:

\begin{enumerate}
	\item If the initial consonant is a stop, the [ɛ] is deleted only if the final consonant is a nasal or a fricative:\\
	\vtop{\halign{%
	#\hfil& &\qquad  #\hfil\cr
	\irdp{den}{mirror} & \irdp{dnu}{mirror (inst.)}\cr
	\irdp{pěr}{roof} & \irdp{pěru}{roof (inst.)} not \ird{\sx{*}pru} \cr
	}}

	\item If the initial consonant is a fricative, the [ɛ] is deleted only if the final consonant is a nasal:\\
	\vtop{\halign{%
	#\hfil& &\qquad  #\hfil\cr
	\irdp{ver}{spring}  & \irdp{veru}{spring (inst.)} not \ird{\sx{*}vru} \cr
	\irdp{hen}{crumb}  & \irdp{hnu}{crumb (inst.)} \cr
	}}

	\item The [ɛ] is not deleted in all other cases.
\end{enumerate}

\subsection{Vowel\,\sim\,Vowel Alternations}
Vowel\,\sim\,vowel alternations (also called `ablaut') occurs when one vowel is
substituted for another in some morphophonological contexts. Vowel\,\sim\,vowel
alternations in Iridian can be broadly classified into two types: [ɛ]
substitution and vowel raising.

Roots of the type --C\sx{j}aC(C) and --C\sx{j}oC(C) become --C\sx{j}eC(C) in the
presence of palatalizing suffixes:

\ex
\vtop{\halign{%
#\hfil& &\qquad  #\hfil\cr
\irdp{bial}{money}			& \irdp{bielí}{honey (gen.)}
							& \irdp{biala}{honey (acc.)}\cr
\irdp{šviak}{soldier}	& \irdp{šviecí}{soldier (gen.)}
							& \irdp{šviaka}{soldier (acc.)}\cr
\irdp{pion}{nest}			& \irdp{piení}{nest (gen.)}
							& \irdp{piona}{nest (pat.)}\cr
\irdp{kážol}{threat}  & \irdp{káželí}{threat (gen.)}
							& \irdp{kážola}{threat (pat.)}\cr
}}
\xe

Vowel-raising alternations are generally triggered by the deletion of an
unstable vowel in the final syllable of the root. The front vowels [eː], [eɪ̯]
and [ʲɛ] in the penultimate syllable merge with the high front vowel [iː]. The
back vowels [ɔ] and [ou̯], on the other hand, merge with [ʊ]. The ablaut does
not occur where the penultimate syllable is also the first syllable of the root
and the root has a null onset.

\ex
\vtop{\halign{%
#\hfil& &\qquad  #\hfil\cr
\irdp{lobek}{apple}		& \irdp{lubka}{pat.} 			& not \ird{*lobka}\cr
\irdp{kostel}{fish}		& \irdp{kustlár}{fisherman}	& not \ird{*kostlár}\cr
\irdp{pěštel}{falcon}	& \irdp{píštlár}{falconer}	& not \ird{*pěštlár}\cr
\irdp{obel}{window}	& \irdp{oblí}{window (gen.)}	& not \ird{*obelí} or \ird{ublí}\cr
}}
\xe


\subsection{Compensatory vowel lengthening}

Compensatory vowel lengthening is a process whereby a short vowel is lengthened
to compensate for the loss of another 

\section{Phonological Processes Involving Consonants}

Iridian consonants are generally affected by two systems of phonological
opposition: a primary distinction between voice and unvoiced consonants, and a
secondary distinction between hard and soft consonants (i.e., normal and
palatalised consonants).

\subsection{Voicing}
Consonant voicing is phonemic. Voiced consonants are called muddy or dark
(\ird{měrkní}) while unvoiced consonants are called clear (\ird{hezkní}).
Most of the obstruents in Iridian come in pairs distinguished only by voicing:

\pex
\a /k/ \phon{kapa}{k͡xäpɐ}{cape} vs /g/ \phon{gapa}{ɡ͡xäpɐ}{liquor}
\a /p/ \phon{pac}{pät͡s}{stick} vs /b/ \phon{bac}{bät͡s}{underside}
\a /t/ \phon{tám}{täːm}{more} vs /d/ \phon{dám}{däːm}{by me}
\xe

Another basic rule of consonant voicing is that in a cluster the last consonant
usually determines whether the preceding ones are voiced or not.\index{voicing
assimilation} Note however that although the liquids /r/ and /l/ and the nasals
/m/ and /n/ are intrinsically voiced, they do not cause the preceding consonant
to assimilate.

\pex
\a\phon{nazka}{ˈnäskɐ}{powder (acc.)} \a\phon{nikda}{ˈɲɪɡdɐ}{fever}
\a\phon{zkáte}{skäːte}{patient} \a\phon{slěň}{ɕʎɛɲ}{soup}
\xe

\subsection{Palatalization}

Iridian consonants can either be hard or soft. Consonants are hard by default
but become soft when followed by the vowels \orth{i} or \orth{í}. The vowel
\orth{y} and \orth{ý} on the other hand are used to indicate non-palatalizing
[ɪ] and [iː] respectively. (Compare, for example, \phon{být}{biːt}{cough} and
\phon{bít}{bʲiːt}{cup}.)

Softening involves palatal articulation of labial consonants (e.g.,
\ird{be}~[bɛ] vs \ird{bě}~[bʲɛ] or the change to a palatal consonant for
non-labials (e.g., \ird{te}~[tɛ] vs \ird{tě}~[cɛ]). Table \ref{table:softhard}
shows how non-labials are affected by palatalization in Iridian.

\begin{table}
	\footnotesize\sffamily
	\caption{Alternations caused by consonant softening}
	\medskip
	\begin{tblr}{width=0.7\textwidth,colspec={XXX}}
		\toprule \addlinespace
		{\scshape hard} & {\scshape soft with ě} & {\scshape soft with a}\\ \addlinespace
		\midrule \addlinespace
			\ird{b}~[b] 	& \ird{bě}~[bʲɛ]	&\ird{bia}~[bʲɐ]\\ \addlinespace
			\ird{p}~[p] 	& \ird{pě}~[pʲɛ]	&\ird{pia}~[pʲɐ]\\ \addlinespace
			\ird{d}~[d] 	& \ird{dě}~[ɟɛ]		&\ird{dia}~[ɟɐ]\\ \addlinespace
			\ird{t}~[t] 	& \ird{tě}~[cɛ]		&\ird{tia}~[cɐ]\\ \addlinespace
			\ird{f}~[f] 	& \ird{fě}~[fʲɛ]	&\ird{fia}~[fʲɐ]\\ \addlinespace
			\ird{v}~[v] 	& \ird{vě}~[vʲɛ]	&\ird{via}~[vʲɐ]\\ \addlinespace
			\ird{k}~[k] 	& \ird{kě}~[cɛ]		&\ird{kia}~[cɐ]\\ \addlinespace
			\ird{g}~[ɡ] 	& \ird{gě}~[ɟɛ]		&\ird{gia}~[ɟɐ]\\ \addlinespace
			\ird{s}~[s] 	& \ird{še}~[ɕɛ]		&\ird{ša}~[ɕɐ]\\ \addlinespace
			\ird{z}~[z] 	& \ird{že}~[ʑɛ]		&\ird{ža}~[ʑɐ]\\ \addlinespace
			\ird{h}~[h] 	& \ird{hě}~[çɛ]		&\ird{hia}~[çɐ]\\ \addlinespace
			\ird{c}~[t͡s]	 & \ird{če}~[t͡ɕɛ]	  &\ird{ča}~[t͡ɕɐ]\\ \addlinespace
			\ird{m}~[m] 	& \ird{mě}~[mʲɛ]	&\ird{mia}~[mʲɐ]\\ \addlinespace
			\ird{n}~[n]		& \ird{ňa}~[ɲɛ]		&\ird{ňa}~[ɲɐ]\\ \addlinespace
			\ird{l}~[l] 	& \ird{lě}~[ʎɛ]		&\ird{lia}~[ʎɐ]\\ \addlinespace
			\ird{r}~[r] 	& \ird{rě}~[rʲɛ]	&\ird{ria}~[rʲɐ]\\ \addlinespace
		\bottomrule
	\end{tblr}
	\label{table:softhard}
\end{table}

The assimilation of the preceding consonant to the soft consonant in a consonant
cluster is often observed but it is not represented in the orthography. For
example in \irdp{slěň}{soup} the initial /s/ would assimilate to the soft /l/
and so the word is realized [ɕʎɛɲ] instead of [sʎɛɲ]; the spelling however
remains as \ird{slěň} and not \ird{*šlěň}.

Palatal assimilation only operates leftwards; i.e., a soft consonant at the
start of a cluster would not cause the following consonant to become soft (e.g.,
\irdp{štotnik}{(he) was arrested} is realized as [ɕtɔcɲɪx] and not [ɕcɔcɲɪx]). 


\section{Orthographic representation}\label{sec:ortho}
\subsection{Alphabet}

The Iridian language uses the Latin script with the following 31 letters:
\ird{a, b, c, č, d, e, ě, f, g, h, i, j, k, l, m, n, ň, o, p, q, r, s, š, t, u,
v, w, x, y, ý, z, ž}.

The language was originally written in its own script but after the Latin
alphabet has been adapted and has been in use since the First Bohemian Union in
the 14th century. Due to the historical ties with the Kingdom of Bohemia and its
historical successors, Czech orthography has had a great influence on the
orthography of Iridian and is the direct inspiration for the current
orthography. The main differences between the two include the lack of the
letters ď, ř, ť, and ů. The sound represented by the letter
ř does not exist in Iridian.

The Cyrillic script coexisted with the Iridian Latin alphabet from the 12th
until the early 16th century. Today Cyrillic is still used to write the
Ukrainian dialects of Iridian.

\begin{table}
	\footnotesize\sffamily
 	\caption{The letters of the Iridian alphabet and their corresponding phonemes.}\index{alphabet}
	\medskip
	\begin{tblr}{width=0.9\textwidth,colspec={XX[1.3]XXX[1.3]X}}
		\toprule
		{{\sc  symbol}} & {\sc name} & {\sc ipa} & {{\sc  symbol}} 	& {\sc name}& {\sc ipa}\\
		\midrule
		A a	  			& á 		 & /a/       &  Ň ň				& eň 		& /ɲ/\\
		B b				& bé		 & /b/       &  O o				& ó			& /o/\\
		C c				& cé		 & /t͡s/      &  P p			 & pé		 & /p/\\
		Č č				& čá		 & /t͡ɕ/      &  Q q			 & kú		 & -\\
		D d				& dé		 & /d/       &  R r			 	& er		& /r/\\
		E e				& é		 	 & /ɛ/       &  S s				& es		& /s/\\
		Ě ě				& jé		 & /jɛ/      &  Š š				& eš		& /ɕ/\\
		F f				& fí		 & /f/       &  T t				& té		& /t/\\
		G g				& gá		 & /ɡ/       &  U u				& ú			& /u/\\
		H h				& há		 & /x/       &  V v				& vé		& /v/\\
		I i				& í		 	 & /i/       &  W w				& vênek		& -\\
		J j				& jota		 & /j/       &  X x				& íks		& -\\
		K k				& ká		 & /k/       &  Y y				& ipsylon   & /i/\\
		L l				& el		 & /l/       &  Z z				& zet		& /z/\\
		M m				& em		 & /m/       &  Ž ž				& žeš		& /ʑ/\\
		N n				& en		 & /n/       &    				& 			& \\
		\bottomrule
	\end{tblr}
\end{table}


\begin{table}
	\footnotesize\sffamily
 	\caption{Correspondence between the Iridian Latin and Cyrillic scripts.}\index{alphabet}
	\medskip
	\begin{tblr}{width=0.8\textwidth,colspec={XXXX}}
		\toprule
		{{\sc  latin}} & {\sc cyrillic} & {{\sc  latin}} & {\sc cyrillic} \\
		\midrule\addlinespace
		A a 		& А а	& O o   & О о \\ 
		B b			& Б Б 	& P p 	& П п \\
		C c 		& Ц ц 	& Q q 	& -- \\
		Č č 		& Ч ч 	& R r 	& Р р \\
		D d 		& Д д	& S s 	& С с \\
		E e 		& Е е 	& Š š 	& Ш ш \\
		F f			& Ф ф	& Tt 	& Т т \\
		G g 		& Г г	& Uu	& У у \\
		H h			& Х х	& V v   & В в\\
		I i			& И и	& W w   & --\\
		J j			& --	& X x   & --\\
		K k			& К к	& Y y   & Ы ы\\
		L l			& Л л   & Z z   & З з\\
		M m 		& М м   & Ž ž   & Ж ж\\
		N n 		& Н н   &		&\\\addlinespace
		\SetCell[c=4]{} Letters unique to the Cyrillic script \\\addlinespace
		Dz dz 		& Ѕ ѕ 	& Dž dž & Џ џ\\
		/ja/ 		& Я я	&/je/ & Є є\\
		/jo/		& Ю ю   & &\\
		Ą ą & Ѫ ѫ&Ę ę&Ѧ ѧ\\ \addlinespace
		\bottomrule
	\end{tblr}
\end{table}
 
Iridian uses two diacritics: the caron (◌̌) and the acute accent (◌́). The caron
is used to indicate palatalization of a consonant or, in the context of
\orth{ě}, of the preceding consonant. When the letter \orth{ě} is used after the
consonants \orth{c}, \orth{n}, \orth{s} and \orth{z}, the caron is used only
with \orth{ě} and although they are effectively palatalized, the preceding
consonants are not marked. Thus one writes \orth{ně} instead of \orth{\sx{*}ňe}
or \orth{\sx{*}ňě}. The accute accent is used with vowels and the letter
\orth{y} to indicate a long vowel.

\subsection{Orthographic Conventions}
Iridian spelling is fairly regular.


\chapter{Verbs}

\section{Introduction}


Verbs in Iridian are heavily marked. There is a tendency to encode most of the information contained in the sentence on the verb leaving the noun or noun phrase unmarked if possible. 

\par Finite verbs are marked\index{markedness} for the following grammatical categories\index{grammatical categories}:
\begin{enumerate}[nosep]
	\item {\scshape aspect}.\index{aspect} Iridian has three primary aspects: perfective, imperfective and contemplative; and two secondary ones: retrospective and prospective.
	\item {\scshape voice}.\index{voice} Iridian has a strong tendency to leave the topic of the sentence unmarked, instead encoding the primary information on the verb. Due to this, voice must be explicitly marked on the verb. Iridian has the following grammatical voices: agentive, patientive, benefactive, instrumental, locative and reflexive.
	\item {\scshape mood and modality}.\index{mood} Besides the unmarked indicative, Iridian has the following grammatical moods: subjunctive, conditional, hortative, optative, abilitative, permissive and non-volitive. In addition, secondary prefixes are used to express what would otherwise could be considered as moods: inceptive, causative and reciprocative.
	\item {\scshape evidentiality}.\index{evidentiality} Iridian formally distinguishes between reported and non-reported information, although actual usage of the former often encompasses various semantic and not strictly evidentiality-related applications.
\end{enumerate}

Verbs are also marked for person, although this is done by the addition of clitic pronouns and not through a separate conjugation paradigm. In most cases, however, this is left out, especially if clear from the context. Iridian verbs are not marked for tense, gender, or number.

\par Iridian verbs have four classes of non-finite forms: the gerund, the converb, the supine and the generic nominal formed with \ird{-ou}. The non-finite verb forms are derived from the uninflected verb stem except the generic nominal in \ird{-ou} which can only be formed from a fully-inflected verb stem. A fifth class exists--the infinitive--but this form is largely defunct and is only used in certain compound constructions. Infinitives end in \ird{-á} and is used as the citation form of a verb.

\section{Verb stem and order of inflectional affixes}\index{citation form}\index{infinitive}


\subsection{The verb stem}\index{verb stem}
\par The {\scshape citation form} (or {\scshape dictionary form}\index{dictionary form|see{citation form}} or {\scshape lemma}\index{lemma|see{citation form}}) of a verb is the uninflected {\scshape infinitive}\index{infinitive}, a fossilised form rarely used outside of a very few periphrastic\index{periphrasis} and historical constructions (see \S\,\ref{sec:infinitive}). The infinitive ends with the vowel \ird{-á}, and removing this ending will produce the {\scshape verb stem}\index{verb stem}. The final consonant  of the stem is called the thematic consonant\index{thematic consonant} and determines the conjugation paradigm the verb follows. The verb stem is a bound form and must always appear with at least one inflectional suffix.


\subsection{Sound changes}\label{sec:sound-changes}
Verb stems are normally classified into five groups (called {\scshape classes}) based on how their thematic consonant changes in unstable positions; specifically, since the verb stem most often interacts with the suffixes used in marking grammatical voice, these classes are based on how the stem changes when followed by a sibilant suffix (as in the active voice) or a palatalising suffix (as in the passive voice). The five classes are as follows: 
\begin{enumerate}[nosep]
	\item Class I verbs include verbs with a thematic \ird{-t, -k, -c} and \ird{-\v{c}}. They all merge to \ird{-\v{c}} in the active voice (\ird{pia\v{s}t-}\,$\rightarrow$\,\ird{pia\v{s}\v{c}-}) but remain stable when followed by a palatalising suffix, except \ird{-c} and \ird{-\v{c}} which merge to [t͡ɕ] although this is not reflected orthographically.
	\item Class II verbs include verbs with a thematic \ird{-s} or \ird{\v{s}}, which both merge to [ɕ] in oth the active and passive voice, although only the former is reflected orthographically. 
	\item Class II-A (or Class IV) verbs are the smallest group and include verbs with a thematic \ird{-l} or {-p}. They use the suffix \ird{-\v{s}} in the active voice (\ird{dal-}\,$\rightarrow$\,\ird{dal\v{s}-}) and are stable elsewhere.
	\item Class III verbs include verbs with a thematic \ird{-d, -g, -h, -j, -z} and \ird{-\v{z}}. They all merge to \ird{-\v{z}} in the active voice (\ird{vad-}\,$\rightarrow$\,\ird{v\'a\v{z}-}); they remain stable when followed by a palatalising suffix, except \ird{-z} and \ird{-\v{z}} which merge to [ʑ] although this is again not reflected orthographically.
	\item Class III-A (or Class V) verbs include those ending with the remaining thematic consonants. They use the suffix \ird{-\v{z}} in the active voice (\ird{\v{s}\v{c}en-}\,$\rightarrow$\,\ird{\v{s}\v{c}en\v{z}-}) and are stable elsewhere.
\end{enumerate}

This classification is notwithstanding the fact that if the thematic consonant is immediately after one or more consonants (except a lateral) an epenthetic \ird{-a-} is added and the suffix -\ird{-\v{s}} is used to form the active root regardless of the actual thematic consonant. As such we get \ird{parka\v{s}-} from \irdp{park\'a}{to park} but \ird{p\'al\v{c}-} from \irdp{palk\'a}{to punch.} Moreover German loanwords whose infinitives end in \ird{-irn\'a} behave as if they have a thematic \ird{-r} and so the the active root for \irdp{t\'el\'evonirn\'a}{to call (on the phone)} is \ird{t\'el\'evonir\v{z}-} instead of \ird{*t\'el\'evonirna\v{s}-}.

The suppletion of the original thematic consonant in the first to third classes with the class ending causes the preceding vowel to be lengthened in compensation if the root would have ended in an open syllable or a lateral had the thematic consonant been removed; thus we have, e.g., \ird{d\'u\v{s}-} from \irdp{du\v{s}\'a}{to bathe} but \ird{pia\v{s}\v{c}-} and not \ird{*pi\'a\v{s}\v{c}-} from \irdp{pia\v{s}t\'a}{to eat.} If the remnant vowel is \ird{\v{e}} or the diphthong \ird{ei}, the compensatory lengthening also involves the reduction of the vowel to \ird{-\'i} as in \ird{zd\'i\v{c}-} from \irdp{zd\v{e}k\'a}{to blow.}

\subsection{Finite verb endings}

\section{Voice}\index{voice}

Iridian often prefers to encode information on the verb instead of through case marking on nouns. As such, all verbs must be explicitly marked for voice.
\begin{table}[!ht]
	\sffamily\footnotesize
	\caption{Suffixes used to mark grammatical voice.}\medskip
	\begin{tabu} to 0.5\textwidth{@{}YY[0.5]@{}}
		\toprule
		&{\scshape ending}\\
		\midrule
		Agentive	& {-(a)š-}\\
		Patientive	& {-in-}\\ 
		Benefactive	& {-éb-}\\ 
		Locative	& {-oun-}\\ 
		Instrumental& {do-\,-oun-}\\ 
		Reflexive	& -\\ 
		Reciprocal	& \\ 
		\bottomrule
	\end{tabu}
\end{table}

\subsection{Agentive voice}\index{agentive voice}
\par The agentive voice is used if the subject of the verb is the agent of the action. Unlike English and other languages with a nominative-accusative morphosyntactic alignment, Iridian does not have a ``default'' voice for a verb. As such, the unmarked form of a verb is essentially meaningless. Voice must be explicitly marked on the verb and the agentive does not have any precedence over the other voices. The agentive voice is marked by the suffix \ird{-aš-}, which assimilates to the thematic consonant of the verb, as described in \S\ref{sec:sound-changes}.
\pex
\begingl
\gla Sa piašček.//
\glb already eat-\Av{}-\Pf{}//
\glft `(I) already ate.'//
\endgl
\xe

Where the assimilation involves the deletion of the final consonant in the root, the preceding vowel is lengthened in compensation if the resulting root would then end in an open syllable.\index{compensatory lengthening}
\begin{multicols}{2}
\pex
\ird{Udúšek.}\\
(instead of \ird{*udušek})\\
\trsl{(I) took a shower.}
\xe
\pex
\ird{Piašček.}\\
(not \ird{*piášček.})\\
\trsl{(I) ate.}
\xe
\end{multicols}

If the remnant vowel is the i-glide \ird{-ě-} or the diphthongs \ird{-ei-} and \ird{-ou-}, the remaining vowel would simplify to \ird{í}, \ird{í} and \ird{ú}, respectively. Consider for example the verb \ird{zděká} \trsl{to blow}:

\pex
\begingl
\gla Lest zdičime.//
\glb wind blow-\Av{}-\Prog{}//
\glft \trsl{The wind is blowing.}//
\endgl
\xe

\subsection{Patientive voice}\index{patientive voice}

\par A verb in the patient focus (glossed \Pv{}) indicates that the topic is the thematic patient in the sentence. This is roughly equivalent to the English passive voice, although the patientive voice is more commonly used in Iridian than the passive voice in English. 

\pex
\begingl
\gla Marek vidnek.//
\glb Marek see-\Pv{}-\Pfv{}//
\glft `(I) saw Marek.'//
\endgl
\xe


\subsection{Benefactive voice}\index{benefactive voice}
\par The benefactive voice (glossed \mk{ben}) is used when the subject of the sentence is the benefactor or director object of the verb. Verbs often change meaning when used in the benefactive focus.


\begin{multicols}{2}
\pex
\begingl
\gla Mač sega nazdébik.//
\glb mother flower-\Acc{} buy-\Ben{}-\Pf{}//
\glft `(I) bought my mother flowers.'//
\endgl
\xe

\pex
\begingl
\gla Kova piaštébime.//
\glb cow eat-\Ben{}-\Prog{} //
\glft \trsl{(I am) feeding the cows.}//
\endgl
\xe

\end{multicols}

The benefactive is also used idiomatically with verbs of judgment including \ird{novětá} \trsl{to like}

\pex
\begingl
\gla Dá čehóvám zánovítébime.//
\glb \First\Sg{} sports-\Agt{} \Neg{}-like-\Ben{}-\Prog{}//
\glft \trsl{I don't like sports.}//
\endgl
\xe

\subsection{Locative voice}

\pex
\begingl
\gla Jé kopnažalíc.//
\glb you laugh-\mk{loc-prog-3s.anim}//
\glft \trsl{He is laughing at you.}//
\endgl
\xe

\subsection{Instrumental voice}


\subsection{Reflexive voice}

The reflexive voice (glossed {\Refl}) is used when the patient of the verb is also the agent of the action. Morphogically, the reflexive voice is not a separate voice but is derived from the agentive form of the verb and the addition of the prefix \ird{u(d)-}.

\pex
\begingl
\gla Na šarta uvižek.//
\glb \Loc{} mirror-Pat{} \Refl{}-see-\Av{}-\Pf{}//
\glft \trsl{I saw myself in the mirror.}//
\endgl
\xe

The use of the reflexive voice is more extensive in Iridian than in English\index{English}, and is somehow similar to how the reflexive construction is used in Romance languages.

\pex
\begingl
\gla Uštižek.//
\glb \Refl{}-take:a:bath-\Av{}-\Pf{}//
\glft \trsl{(I) took a bath.}//
\endgl
\xe

\pex
\begingl
\gla Umúšime.//
\glb \Refl{}-comb-\Av{}-\Prog{}//
\glft \trsl{(I) am combing my hair.}//
\endgl
\xe

Below is a non-exhaustive list of verbs that are normally used in the reflexive voice:
\bigskip

\noindent
\ird{dušá} \trsl{to take a shower}\\
\ird{mušá} \trsl{to comb}\\
\ird{šaštá} \trsl{to sit down}\\

Some verbs may change meaning when used in the reflexive voice.


The reflexive voice is also used to imply that an action happened accidentally or involuntary or that the agent of the action is unknown or unimportant.

The reflexive voice may also be used emphatically, especially in spoken Iridian, to express that the action has been performed for the benefit of the actor/agent of the verb.

\pex
\begingl
\gla Kávéa ušranz\k{a}cem.//
\glb coffee-\Acc{} \Refl{}-drink-\mk{av-ctplv-1s}//
\glft \trsl{I'll drink coffee.} (literally, I'll drink myself coffee)//
\endgl
\xe

\pex
\begingl
\gla Pulša uvošček.//
\glb soup-\Acc{} \Refl{}-cook-\Av{}-\Pf{}//
\glft \trsl{(I) cooked (me) some soup.}//
\endgl
\xe


\par The differences

\section{Grammatical aspect}\label{sec:aspect}\index{aspect}\index{grammatical aspect|see{aspect}}

{\scshape grammatical aspect} (or simply {\scshape aspect}) is a category in Iridian that is used to denote how an action or state described by a verb extends over time. Aspect contrasts with {\scshape tense} which situates an action or event as happening or being true at some specific point in time. Iridian does not mark tense grammatically, as does English\index{English}, for example, and so the verb in English sentences \trsl{I am watching a movie on TV now} and \trsl{I was watching a movie on TV when you called} will both be translated using the same verb in the progressive aspect. 

Aspect also contrasts with {\scshape lexical aspect} or \foreign{aktionsart}\index{aktionsart@\emph{aktionsart}} in that the latter, although also describing a verb's structure in relation to time, refers more to an inherent property of the verb itself and is thus often invariant. Iridian also does not grammaticalise \foreign{aktionsart}, and thus the distinction between, say, Polish imperfective \foreign{pisać,} \trsl{to write} and perfective \foreign{napisać,} \trsl{to write down,} is not one made in the language (cf. \cite[9--26]{richardson2007}; \cite{comrie1976}).

Iridian formally distinguishes between seven classes of grammatical aspect, five of which are called {\scshape primary} since they are used to describe the aspect of an independent main verb while two are called {\scshape secondary} since they represent a verb's aspect in relation to some other event.




\begin{table}
	\footnotesize\sffamily
	\caption{Aspect markers in the indicative mood.}
	\medskip
	\begin{tabu} to 0.5\textwidth{@{}YY[0.5]}
		\toprule
		{\sc aspect}	& {\sc affix}\\
		\midrule
		Perfective		& {-ek}\\
		Retrospective	& {-aní}\\
		Imperfective	& {-eví}\\
		Progressive		& {-ime} \\
		Contemplative	& {-\'ach/-\'ah-}\\
		Prospective		& {-ujam}\\
		Cessative		& {-óvít}\\
		\bottomrule
	\end{tabu}

\end{table}


\subsection{Perfective and retrospective aspect}\label{sec:perfective-retrospective}

Both the perfective aspect (glossed \Pf{}) and the retrospectived {glossed \Ret{}} represent actions that have been completed at some point in time.

The perfective aspect indicates an action that has been completed at some specific point in time. The thematic ending for the perfective aspect is \ird{-ek}, but the initial $\langle$e$\rangle$ is rather unstable and often changes depending on the environment. The initial $\langle$e$\rangle$ becomes $\langle$i$\rangle$ when used with \ird{-in} (the suffix indicating the patientive voice), with the initial $\langle$i$\rangle$ in the preceding suffix often dropped or replaced by an $\langle$e$\rangle$. This change also occurs when following the benefactive suffix \ird{-éb} and when followed by the quotative suffix \ird{-e} (in which case the final \ird{-k} is fricativised to $\langle$c$\rangle$).

\pex
\begingl
\gla Bych na gnaža Marek vdenik.//
\glb yesterday \Loc{} school-\Acc{} Marek see-\Pv{}-\Pf{}//
\glft \trsl{(I) saw Marek at school yesterday.}//
\endgl
\xe

When negated, the perfective indicates something that ought to be done but had not been done or otherwise to put an emphasis on the non-completion. To state that something simply did not happen, the negative of the retrospective is used instead.

\begin{multicols}{2}
\pex
\a\begingl
\gla Zátélévoniržek.//
\glb \Neg{}-telephone-\Av{}-\Pf{}//
\glft `(I) failed to call.' //
\endgl
\a\begingl
\gla Zátélévoniržaní.//
\glb \Neg{}-telephone-\mk{av-ret}//
\glft `(I) didn't call.' //
\endgl
\xe
\end{multicols}

\par The retrospective aspect is used for a past action that has a continuing relevance in the presence. Often when using the retrospective aspect, the emphasis is on the resulting state rather than the event itself. Consider, for example, the following sentences: (a) \trsl{I went to Amsterdam last week}; and (b) \trsl{I have been to Berlin in my childhood}. Iridian would translate the verb in (a) using the perfective and the verb in (b) using the retrospective. The retrospective may also be called the {\scshape perfect} in other sources but we shall be solely referring to it as the former in this grammar to avoid any confusion. In addition, the retrospective in Iridian always has a perfective interpretation, i.e., a sentence like \trsl{I have been waiting here an hour} which has a verb in the perfect is translate using the progressive instead of the retrospective.

\pex
\begingl
\gla Hroná tímu na Budapešta možlašime.//
\glb three year-\Ins{} \Loc{} Budapest-\Acc{} live-\mk{av-prog}//
\glft `I have been living in Budapest for three years.'//
\endgl
\xe

\pex
\begingl
\gla Vegetarevn\'i gul\'a\v{s}e stav\'i\v{c}an\'i, ma toleto z\'a\v{c}e\v{s}\v{c}ice.//
\glb vegetarian goulash-\Gen{} eat:a:bit-\Av{}-\Ret{} but \Aff{} \Neg{}-like-\Av{}-\Pf{}-\Quot{}//
\glft \trsl{I have tried vegetarian goulash before but I didn't like it at all.} //
\endgl
\xe

The retrospective is also used to describe an event that is expected or planned to happen before a certain point in the future.

\pex
\begingl
\gla D\'a \v{s}\v{c}en\v{z}it \v{s}e Marek uzdrav\v{z}an\'i.//
\glb \First{}\Sg{}.\Str{} arrive-\Av{}-\SupP{} \Com{} Marek \Refl{}-sleep-\Av{}-\Ret{}//
\glft \trsl{Marek would already have been asleep by the time I arrived.} //
\endgl
\xe

\pex
\begingl
\gla Na arma\v{s}ta \v{s}\v{c}en\v{z}ice dn\'ovim po z\'azdal\v{s}an\'i.//
\glb \Loc{} airport-\Acc{} arrive-\Av{}-\SupP{}-\Att{} front-\Ins{} yet have:breakfast-\Av{}-\Ret{}//
\glft \trsl{We will not have eaten breakfast before we get to the airport tomorrow.} //
\endgl
\xe

Moreover, the retrospective is often used to imply non-volition or the  accidental/circumstantial nature of an action. Similarly the retrospective is used with verbs of emotion or state (e.g., \ird{cezuštalá}, ‘to become happy’ from \ird{zuštal} ‘happy’). The perfective, on the other hand, is almost exclusively used with the causative in these cases.

\pex
\a	\begingl
\gla Vdešek še neicezuštalašaním.//
\glb see-\mk{2s-pf} with \mk{incep}-be.happy-\mk{av-ret-1s}//
\glft `I became happy when I saw you.' //
\endgl
\a	\begingl
\gla Do pacezuštalnikeš.//
\glb \First{}\Sg{}.\Wk{} \Caus{}-be.happy-\mk{pv-pf-2s}//
\glft `You made me happy.' //
\endgl
\xe
\pex<vasebroke>
\begingl
\gla Váz noprizaní.//
\glb vase break-\mk{ref-ret}//
\glft `The vase broke (accidentally).' //
\endgl
\xe

\subsection{Continuous and progressive aspects}
Iridian uses the continuous and progressive aspects to denote actions that have not been completed yet and/or are in the process of happening/occuring. The continuous aspect (glossed \mk{cont}) is used to mark a state of being while the progressive aspect (glossed \mk{prog}) is used to mark a dynamic activity.
\pex
\begingl
\gla Nau urištneví.//
\glb clothes \Refl{}-wear-\mk{pv-cont}//
\glft \trsl{(I'm) wearing clothes.} //
\endgl
\xe

\pex
\begingl
\gla Nau urištnime.//
\glb clothes \Refl{}-wear-\mk{pv-prog}//
\glft \trsl{(I'm) putting on clothes.} //
\endgl
\xe

The continuous aspect is also used to denote a habitual action.

\pex
\begingl
\gla Sholu de gnaža stoževí.//
\glb daily-\Ins{} \mk{ill} school-\Acc{} go-\Av{}-\Cont{}//
\glft \trsl{(We) go to school everyday.} //
\endgl
\xe

\pex
\begingl
\gla Dá na Praha možleví.//
\glb \mk{1s.str} \Loc{} Prague-\Acc{} live-\mk{cont}//
\glft \trsl{I live in Prague.} //
\endgl
\xe

To emphasise the habitual nature of an action, a nominalised construction is often used.

\pex
\begingl
\gla Nažem r\k{a}cenživou.//
\glb friend-\First{}\Sg{} smoke-\mk{av-cont-nz}//
\glft \trsl{My friend is a smoker.} //
\endgl
\xe


\subsection{Contemplative aspect}

The contemplative aspect (glossed \Ctp{}) is used to mark an action that has not been started yet. If the emphasis, however, is on the speaker's intention to do something, and not on the incompleteness or futurity of the action itself, the supine of purpose is used instead of the contemplative. The same is true for events which the speaker thinks is improbable or unlikely to happen but whose non-occurence they are nonetheless unsure of.

\subsection{Secondary aspects}
The prospective aspect (glossed {\scshape prosp}) is primarily used in secondary clauses to indicate actions that are about to be started in relation to another action. It can also be used in the main clause to indicate an action in the immediate future.


The cessative aspect

\section{Valency}\index{valency}\index{valence|see{valency}}

{\scshape valency} (or {\scshape valence})\index{valency} is the number of overt arguments a verb\index{argument of a verb} can take in a sentence. \textcite[239]{tesniere1965}, in one of the earliest description of the concept, likens valency by comparing it to bonds between atoms:
\begin{quotation}
	\small
The verb may therefore be compared to a sort of atom, susceptible to attracting a greater or lesser number of actants,\footnote{In his work Tesni\`ere used the term \emph{actants} to refer to what we would call here the verb's \trsl{arguments.}} according to the number of bonds the verb has available to keep them as dependents. The number of bonds a verb has constitutes what we call the verb's valency.
\end{quotation}

More rigorous treatments\footnote{\posscite{tesniere1959} definition of valency as \trsl{nombre d'\emph{actants} qu'un verbe est susceptible de régir} (\trsl{number of \emph{actants} which a verb is capable of governing}) essentially frames valency as a function of the verb. More recent definitions however consider valency not just as a property of verbs alone but of any lexical item (cf., e.g., \cite{matthews1997,trask1993}). In addition, in his glossary, he has provided voice (Fr. \emph{voix}) as a synonym for valency; these two terms however we consider as distinct items both in this work and in what I think is the usage of both terms in scholarly literature over the topic.} have of course been published in the years since but we should content ourselves with this definition in our present treatment of Iridian grammar. Instead our primary focus would e

\subsection{Avalent verbs}

Avalent verbs \index{avalent verb} are verbs that have zero core arguments. In Iridian they are limited to a small set of verbs that describe meteorological phenomena, traditionally referred to as `weather verbs' (\ird{plodní sládek}) \index{weather verb}.This term is not wholly accurate, however, as the class includes not just meteorological phenomena but more general natural phenomena as well. When used this way they are marked in the agentive voice\index{agentive voice} and essentially forms topicless sentences\index{topicless sentence} (cf.~\S\,\ref{sec:topicless}). Some common weather verbs in Iridian are listed below.

\pex\deftagex{exw}
\irdp{hravá}{to have the sun shine}\\
\irdp{žužá}{to snow}\\
\irdp{pozběšá}{to rain}\\
\irdp{néšá}{to rain lightly, to drizzle}\\
\irdp{boboržá}{to have thunder}\\
\irdp{kopriká}{to have lightning}\\
\irdp{dozbuhá}{to have an earthquake}
\xe



\subsection{Passive constructions}


\subsection{Causative constructions}\index{causative}

Causatives may either be lexical, analytical or morphological. Lexical causatives involve the encoding of the causation on the verb itself leading the causative form of the verb to be a different form altogether. An analytical causative, on the other hand uses a different verb (usually a verb like \emph{to do} or \emph{to make}) in conjunction with the main verb, to express the idea of causation (e.g., English\index{English} \trsl{make someone do something.}) Finally, morphological causatives involve morphologically changing the main verb to express the notion of causation. Iridian causative constructions are primarily morphological, formed using the prefix \ird{ne-}.

\begin{table}
\footnotesize\sffamily
\caption{Causative forms of the verb \irdp{shradá}{to die.}}
\medskip
	\label{tbl:causative}
    \begin{tabu}to \textwidth{@{}Y[0.2]Y[0.5]YY@{}}
        \toprule
		 		{\sc voice}& {\sc causative } &{\sc regular meaning} & {\sc causative meaning}\\
		\midrule
				Inf.				& neshradá									& to die, to be dead 	& \emph{(defective)} \\ 
		 		Agt.				& {neshrážá}			& to kill & to cause someone to kill\\ 
		 		Pat.			& {neshradiná}					& to be killed & to be caused to be killed\\
				Ben.			& {neshradébá}				& to have someone die	& to have someone be killed\\
				Loc.				& {neshradouná}					& to have someone related die&\emph{(defective)}\\
				Ins.		& {doneshradouná}&to be the reason for dying&to be used for killing\\
				Refl.				& {uneshražá}&to kill oneself&to cause one to kill oneself\\
		 		
				\bottomrule

    \end{tabu}

\end{table}

Due to this suppletive nature, lexical causatives imply a more direct causation, or a tighter link between cause and event\footnote{\textcite{haiman1983} offers a thorough discussion of how the linguistic distance exhibited by the forms of causative constructions existing in a language (e.g., \emph{to cause to die} on one end of the spectrum versus \emph{to kill} on the other) correspond to the conceptual distance between the action of the causer and the result of the action to the causee. In a purely synthetic construction like \emph{kill}, for example, where the linguistic distance is the least, the conceptual distance between the action and the resulting state is also the smallest, with the opposite being true in purely analytical constructions like \emph{to cause to die}.}, than analytical or morphological causatives (\cite{velupillai2012, haiman1983}). Consider for example the three sentences in English\index{English} below:


\pex
\a Joseph \emph{died}.\deftagex{caus}
\a Joseph \emph{killed} the man.\deftagex{caus}\deftaglabel{kill}
\a Joseph \emph{made} the man \emph{die.}\deftagex{caus}\deftaglabel{made}
\xe

The suppletive \emph{kill} in example (\getfullref{caus.kill}) implies more agency on the part of the subject than the more indirect-sounding (\getfullref{caus.made}). In (\getfullref{caus.kill}) the \emph{death} of the patient (\trsl{the man}) is the goal of the act while (\getfullref{caus.made}) it might be inferred that the \emph{dying} was an indirect consequence of an unmentioned second act.


Iridian does not employ lexical causatives as in English\index{English}; instead causatives are formed morphologically by adding the prefix \ird{ne-} (glossed as \Caus{}) to the verb stem. Although \ird{ne-} is required to form the causative morphologically, some verbs, particularly stative verbs like \irdp{shradá}{to die, to be dead} in table \ref{tbl:causative} may already contain the notion of causation in some of its regular conjugated forms. This is because by default stative verbs\index{stative verb} are intransitive (i.e., the only argument required is the actor/agent\index{agent}) while some verbal voices\index{voice} like the patientive\index{patientive voice} and benefactive\index{benefactive voice} inherently imply the existence of a second and a third argument of a verb\index{argument of a verb} respectively.

%% TODO add section reference

Of course Iridian's definition of which verbs are stative and which ones are dynamic\index{dynamic verb} does not neatly align with the definition those classes have in English\index{English} (v. \S\,\ref{sec:statives}). For instance the verbs \emph{to stand} and \emph{to eat} are both dynamic verbs in English\index{English}, while in Iridian \irdp{zdavá}{to stand, to be standing} is stative and only \irdp{piaštá}{to eat} is dynamic. This is why as we see in example (\getfullref{statdyn.1}) below, some forms of the verb \ird{zdavá} already contain the notion of causation in some of its regular conjugated forms.

\pex
\a  \irdp{zdavá}{to be standing}\deftagex{statdyn}\deftaglabel{1}\\
	\irdp{zdavžá}{to stand}\\
    \irdp{zdavná}{to be made standing, to erect}\\
    \irdp{nezdavžá}{to make so./sth. stand}\\
    \irdp{nezdavná}{to be made to make so./sth. standing}
\a  \irdp{piaštá}{to eat}\\
    \irdp{piaštiná}{to be eaten}\\
    \irdp{nepiaščá}{to make someone eat}
\xe

Since causative constructions in Iridian are purely morphological\footnote{To contrast, consider Japanese\index{Japanese} which also forms causative constructions morphologically (using the suffix \emph{-(sa)se}) but which in addition also has synthetic but not fully suppletive forms for some verbs (e.g., \irdp{agaru}{to rise} and \irdp{ageru}{to raise}).} the degree of agency of the causer can be implied from other incidental properties of the verb such as aspect or voice markings.

We pay particular attention first on the interaction of the causative prefix \ird{ne-} with the patientive voice marker \ird{-in} and the benefactive voice marker \ird{-éb}. We begin with stative verbs, since as mentioned earlier and in \S\,XX, most stative verbs will have a causative reading when used with the agentive or benefactive voice. Stative verbs encode the state of the subject and cannot therefore express the idea of an agent nor that of a patient. By conjugating stative verbs for voice, their stative nature is therefore lost; that is why a causative cannot be derived from the unmarked form of a stative verb: a causative construction precludes the existence of a causer and a causee, which at times may be different from the subject, while the unmarked stative only that of the subject itself.


\begin{figure}[H]
	{
	\footnotesize
  \begin{forest}
    [\irdp{shradá}{to die},
		[\ird{shradiná}\\
			patientive\\
			{Arg = 1}
				[$
				\begin{bmatrix}
					\textbf{+ Patient}
				\end{bmatrix}
				$]
				]
      [\ird{shražá}\\
				agentive\\
				{Arg = 2}
					[$
					\begin{bmatrix}
						\textrm{+ Patient}\\
						\textbf{+ Agent}
					\end{bmatrix}
					$]
					]
					[\ird{ushražá}\\
						reflexive\\
						{Arg = 2}
							[$
							\begin{bmatrix}
								\textrm{+ Patient}\\
								\textbf{+ Agent}
							\end{bmatrix}
							$]
							]
			[\ird{shradébá}\\
				benefactive\\
				{Arg = 3}
				[$
				\begin{bmatrix}
					\textrm{+ Agent}\\
					\textrm{+ Patient}\\
					\textbf{+ Benefactor}
				\end{bmatrix}
				$]
			]
		]
  \end{forest}

	}\caption[Voice markings as valence operations in stative verbs.]{Voice markings as valence operations in stative verbs. The number of elements includes all those required to create a well-formed sentence notwithstanding Iridian's tendency to drop elements that can be implied from context, with the element in bold representing whichever element is most likely to surface in speech.}
  \label{causative-reading}
\end{figure}

We see in figure \ref{causative-reading} that this causative reading of the patientive voice with stative verbs is due to properties of stative verbs and not of the patientive voice. We know this is true since this causative reading of the patientive does not exist with non-stative verbs, which are transitive by default in Iridian.

\pex
\a
\begingl
    \gla \ljudge{*}Mámka prehlavnik.//
    \glb mother buy-\Pv{}-\Pf{}//
    \glft \trsl{*I bought my mother.}//
\endgl
\a
\begingl
    \gla Mámka zuštalnik.//
    \glb mother happy-\Pv{}-\Pf{}//
    \glft \trsl{I made my mother happy.}//
\endgl
\xe

The patientive voice only requires a patient as argument; however since this argument does not exist in stative constructions, the role of an agent must first be created for the subject of the stative construction to be able to occupy the role of the patient in the patientive voice. Essentially this means that conjugating a stative verb for the patientive voice is equivalent to creating a biclausal causative construction where the subject becomes the causee and the state the action brought about by the (optionally named) causer. This reading is not possible with dynamic verbs because the patientive voice would only shift the role of the patient to that of the topic without having to create a new role for an agent.

As could have been predicted from \posscite{haiman1983} theory, these indirect forms of the causative express a more direct link between the causer and the action. True morphological causatives, i.e., those formed using the prefix \ird{ne-}, imply that the caused action was brought about by an intermediary.

\begin{multicols}{2}
\pex
\a
\begingl
\gla Váz nopriznek.//
\glb vase break-\Pv{}-\Pf{}//
\glft \trsl{I broke the vase.} (on purpose)//
\endgl
\a
\begingl
\gla Váz nenopriznek.//
\glb vase \Caus{}-break-\Pv{}-\Pf{}//
\glft \trsl{I made someone break the vase.}//
\endgl
\xe
\end{multicols}

If the intermediary appears in the sentence it can be marked either in the genitive or in the patientive. Marking the causee in the genitive is the \trsl{neutral} configuration; using the patientive case on the other hand forms what can be called a \emph{coercive} causative (\cite{shibatani1990,lehmann2006}), which in Iridian\footnote{We can compare this to a similar distinction between a dative causative (formed with the clitic \emph{ni}) and the accusative causative (formed with \emph{o}) in Japanese\index{Japanese}. \textcite{lehmann2006} calls the former a coercive causative construction while the latter a permissive causative construction. There are two main differences between the Japanese\index{Japanese} and Iridian systems however. First the coercive causative in Iridian also implies that the agent has effective control over the action or the causee or both, something not necessarily expressed by the Japanese\index{Japanese} \emph{o}-form; and second, both the patientive and the genitive forms of the causative in Iridian allow `permissive' readings, as we illustrate later in this section.

More importantly however the genitive form is considered the default or neutral form in Iridian, with the patientive form considered as more `marked.' The patientive is often used for emphasis, with the genitive construction replacing it where possible, especially in spoken Iridian, even in places where the use of the patientve would have been in better order.
}
could imply either of two things: (i) that the act was made without or against the consent of the causee or (ii) the causer had direct control over the action and/or the causee. Such distinction however is not possible if the main verb is in the agentive voice since the patientive marking is reserved for the patient of the verb (and thus marking the causee in the patientive will essentially produce a situation where both the agent and the patient of the verb is marked for the same role, which in this case is the patient.)

\pex
\a
\begingl
\gla Váz Janc\v{e} nenopriznek.//
\glb vase Janek-\Gen{} \Caus{}-break-\Pv{}-\Pf{}//
\glft \trsl{(I) made John break the vase.}//
\endgl
\a
\begingl
\gla Váz Janka nenopriznek.//
\glb vase Janek-\Acc{} \Caus{}-break-\Pv{}-\Pf{}//
\glft \trsl{(I) made John break the vase.}//
\endgl
\xe


Nevertheless the degree of control exerted by the causer over the action itself may vary between these constructions.

A common way to formally mark the causer's control or lack thereof in Iridian is the opposition between the retrospective aspect and the perfective aspect. Consider for example the two sentences in Iridian below, both of which have the same general translation in English\index{English}.

\pex
\a
\begingl
	\gla Martin nésta najev\v{e}c shražek.//
	\glb Martin deer-\Acc{} drive-\Cv{} die-\Av{}-\Pf{}//
	\glft \trsl{Martin ran over a deer.} (He did it on purpose)//
\endgl
\a
\begingl
	\gla Martin nésta najev\v{e}c shražaní.//
	\glb Martin deer-\Acc{} drive-\Cv{} die-\mk{av-ret}//
	\glft \trsl{Martin ran over a deer.} (It was an accident.)//
\endgl
\xe

\subsection{Reflexive, reciprocal, and sociative constructions}\index{reflexive construction}\index{reflexive voice}\index{reciprocal construction}

The reciprocative prefix \ird{so-} (glossed \Rec{}) is used with the agentive voice to indicate that an action is performed by the agent and the patient on each other.

\pex
\begingl
\gla Karlu sodalšime še Marek ščenžek.//
\glb Karel-\Ins{} \Rec{}-talk-\mk{av-prog} with Marek arrive-\Av{}-\Pf{}//
\glft \trsl{Karel (and I) were talking when Marek arrived.}//
\endgl
\xe

The use of the reciprocative inherently implies plurality on the part of the subject since there are always at least two elements involved (cf. \cite[255]{tesniere1965}). Since Iridian does not often grammaticalise plurality\index{plural}, this means the reciprocative usually won't require additional consideration as to the agreement of the constituents of the sentence; it does, however, mean that this form cannot be used singly with the singular form of pronouns (since pronouns---at least in the first and second persons---formally distinguish between singular and plural) and that most countable nouns would require the use of the particle \ird{ně} or an explicit quantifier.

\pex
\begingl
\gla Na to hruma hurka sokonížek.//
\glb \Loc{} \Dem{} church-\Acc{} parents \Rec{}-wed-\Av{}-\Pf{}//
\glft \trsl{(My) parents were married in this church.}//
\endgl
\xe

\pex
\begingl
\gla N\v{e} senátor sožubalžimej\'i to-\v{z}e na televiza vednik.//
\glb \Pl{}= senator \Rec{}-shout-\Av{}-\Prog{}-\Quot{} \Qp{} \Loc{} television-\Acc{} see-\Pv{}-\Pf{}//
\glft \trsl{(I) saw the senators shouting at each other on TV.}//
\endgl
\xe

Where both elements of the agent-patient pair are present in the sentence, one of them is treated as the agent and left unmarked while the other is marked in the comitative\index{comitative} (i.e., \ird{še} + instrumental). However, since the action itself is reciprocal, which gets marked as the agent is purely a pragmatic choice. Where one of the members of the agent-patient pair is a pronoun, preference is given to marking the pronoun as the agent (in which case \ird{še} is normally ommitted, but with the patient remaining in the instrumental case).

\pex
\begingl
\gla Mišek še Martinu sohévoržev\'i.//
\glb Mišek \Com{} Martin-\Ins{} \Rec{}-know-\Av{}-\Cont{}//
\glft \trsl{Mišek and Martin know each other.}//
\endgl
\xe

\pex
\begingl
\gla No já Mišku sohévoržaní?//
\glb \Q{} \mk{2s.str} Mišek-\Ins{} \Rec{}-know-\Av{}-\Ret{}//
\glft \trsl{You and Mišek already met each other right?}//
\endgl
\xe

Sociative verbs\index{sociative verb}, on the other hand, which are formed with the prefix \ird{se-} (and glossed here as \Soc{}), indicate actions that are performed with another person or other people. These may also represent joining or participating in actions that have been started by somebody. The prefix \ird{se-} is realised as \ird{s-} except before sibilants and the approximant /j/. Sociative verbs are formed with the locative voice, with the noun phrase in the topic representing the person/people with whom an action is done. Note that sociative verb constructions cannot be used with inanimate/non-human noun phrases.

\pex
\begingl
\gla N\v{e} Marek bych sezdravounek.//
\glb \Pl{}= Marek yesterday \Soc{}-sleep-\Lv{}-\Pf{}//
\glft \trsl{I slept at Marek's place yesterday.}//
\endgl
\xe

\pex
\begingl
\gla Kazn\'i hezka lin\v{e} kvu\v{s}tnan\'i \v{s}e m\'e skaznounek.//
\glb song-\Gen{} first line hear-\Pv{}-\Ret{} \Com{} \First\Pl{}.\Str{} \Soc{}-sing-\Lv{}-\Pf{}//
\glft \trsl{Upon hearing the first lines of the song, (he) joined us in singing.}//
\endgl
\xe

\section{Grammatical mood}\index{mood}\index{grammatical mood|see{mood}}

\subsection{Indicative}

\subsection{Imperative and hortative mood}\label{sec:imp-hort}\index{imperative}\index{hortative}

To form commands\index{commands} and requests\index{requests}, the imperative (glossed \mk{imp}) and hortative (\mk{hort}) moods are used in Iridian.

The imperative is formed by replacing the infinitive ending \ird{-á} with the voice marker and the imperative ending \ird{-ím}. The imperative\index{imperative mood} cannot be negated with the prefix \ird{zá-}; instead, to form a negative command the prohibitive\index{prohibitive mood} mood is used (glossed \mk{proh}), formed with the suffix \ird{-éma} instead of \ird{-ím}.

\begin{table}[ht!]
\sffamily\footnotesize
	\caption{Conjugation of the verb \ird{piaštá}\\ in the imperative and probihibitive moods.}
	\label{tbl:imperative}
\medskip
    \begin{tabu}to 0.7\textwidth{@{}YYY@{}}
         \toprule

         {\sc voice}&{\sc imperative}&{\sc prohibitive}  \\
         \midrule

         Agentive &
         {piaščím} &
         {piaščéma}\\

         Patientive &
         {piaštním} &
         {piaštnéma}\\

         Benefactive &
         {piaštébím} &
         {piaštébíma}\\

         Locative &
         {piaštouním} &
         {piaštounéma}\\

         Instrumental &
         {dopiaštouním} &
         {dopiaštounima}\\

         Reflexive &
         {upiaščím} &
         {upiaščéma}\\

         \bottomrule
    \end{tabu}

\end{table}

The imperative\index{imperative mood} is used to issue a direct command and the prohibitive to ``signal a prohibition\index{prohibitive mood}'' (SIL). Verbs in the imperative mood do not require an explicit referent, with the addressee or addressees assumed to be the recipient of the command or prohibition. When the addressee is included, it appears in the vocative case if appearing before the verb or unmarked otherwise.\footnote{A comma is placed between the verb and the addressee if the addressee appears after the verb in the sentence but none if it appears before.} Note that both the imperative and the prohibitive do not distinguish number; thus the same form of the verb will be used when giving a command to multiple addressees and to a single one.

\pex
\begingl
    \gla To hrabním.//
    \glb \Dem{} listen-\mk{pv-imp}//
    \glft \trsl{Listen to this.}//
\endgl
\xe
\pex
\a
\begingl
    \gla To hrabním, Marek.//
    \glb \Dem{} listen-\mk{pv-imp} Marek//
    \glft \trsl{Listen to this, Marek.}//
\endgl
\a
\begingl
    \gla Marku to hrabním.//
    \glb Marek-\mk{voc} \Dem{} listen-\mk{pv-imp}//
    \glft \trsl{Listen to this, Marek.}//
\endgl
\xe

\pex
\begingl
    \gla Papír švirkounéma.//
    \glb paper write-\mk{lv-proh}//
    \glft \trsl{Do not write anything on this sheet of paper.}//
\endgl
\xe

When used with verbal adjectives, the suffixes can attach directly to the root without any need for an explicit marker for voice and the addition of a voice marker will in fact change the meaning of the sentence. (The first two sentences below are rather unhelpful given how morphophonemic changes has rendered the imperative form with the voice marker and the one without of the verb \irdp{slouhatá}{to be quiet} identical, but cases like this are common and merit attention.)

\pex
\a
\begingl
    \gla Nie byló slouháčím.//
    \glb \Pl{}= child be:quiet-\mk{imp}//
    \glft \trsl{Keep quiet, children.}//
\endgl
\a
\begingl
    \gla Nie byló uslouháčím.//
    \glb \Pl{}= child \Refl{}-be:quiet-\mk{av-imp}//
    \glft \trsl{Keep quiet, children.}//
\endgl
\xe

\pex
\a
\begingl
    \gla Pitár zuštalébím.//
    \glb Pitár be:happy-\mk{ben-imp}//
    \glft \trsl{Make Pitár happy!}//
\endgl
\a
\begingl
    \gla Zuštalím.//
    \glb be:happy-\mk{imp}//
    \glft \trsl{Be happy!}//
\endgl
\xe


Due to its directness, the use of the imperative or the prohibitive is
considered impolite in most settings, and is often used only when speaking with
friends, family or children. This distinction does not exist in the written
language, where the imperative is used almost exclusively for these functions.
However in signs that give orders or warnings (i.e., `Stop,' `Do not enter')
where English\index{English} may sometimes use imperative constructions, Iridian uses modal
constructions\index{modality} (cf. \S\,\ref{sec:modality}) as they are not treated
 as direct commands or prohibitions.

\pex
\begingl
    \gla Tak slouhatalneví.//
    \glb here be:quiet-\mk{deb-cont}//
    \glft \trsl{Keep quiet.} \textit{Lit.,} \trsl{One must be quiet here.}//
\endgl
\xe

\pex
\begingl
    \gla Tak zahranéčneví.//
    \glb here enter-\mk{npot-cont}//
    \glft \trsl{Do not enter.} \textit{Lit.,} \trsl{One cannot enter here.}//
\endgl
\xe

In spoken Iridian, it is more common and considered more polite to use the
hortative and the negative hortative forms instead of the direct imperative
or prohibitive.

\begin{table}[ht!]
    \footnotesize\sffamily
		\caption{Conjugation of the verb \ird{piaštá} in the hortative mood.}
		\label{tbl:hortative}
		\medskip
    \begin{tabu}to 0.7\textwidth{@{}YYY@{}}
         \toprule

         &{\sc hortative}&{\sc neg. hortative}  \\
         \midrule
         Agentive &
         {piaščká} &
         {piaščku}\\

         Patientive &
         {piaštniká} &
         {piaštniku}\\

         Benefactive &
         {piaštébká} &
         {piaštébku}\\

         Locative &
         {piaštómká} &
         {piaštómku}\\

         Instrumental &
         {dopiaštómká} &
         {dopiaštómku}\\

         Reflexive &
         {upiaščká} &
         {upiaščku}\\

         \bottomrule
    \end{tabu}

\end{table}

\pex
\begingl
\gla Mina návilastnika.//
\glb door open-\mk{pv-hort}//
\glft \trsl{Open the door.}//
\endgl
\xe

To further soften command, the expression \ird{am luhninká} (from the hortative
form of the verb \irdp{luhná}{to give thanks}, now obsolete except for this
specific usage) and its equivalent negative form \ird{am luhninku} can be used,
with the main verb marked as a perfective converb.\index{converb}\footnote{Cf. the use of the perfective converb with the \textit{merci de} + infinitive construction in French\index{French}. The use of \ird{am luhninká} presupposes that the action being requested has already been done although in fact it hasn't, for which therefore the speaker is giving thanks. Thus, a simple request like \trsl{Please close the door} is expressed in Iridian as \trsl{May you be thanked for having closed the door.}}

\pex
\begingl
\gla Mina se návilastu am luhninka.//
\glb door \Refl{} open-\mk{cv.pf} because thank-\mk{pv-hort}//
\glft \trsl{Please open the door.}//
\endgl
\xe

The adhortative (\trsl{Let's}) is formed using \ird{lidovká} with the imperfective converb form of the main verb. \ird{Lidovká} can also be used by itself where the main verb may be implied from context, or as a reply to the request if the speaker wants to express agreement or assent.

\pex
\begingl
\gla Piaštiec lidovká.//
\glb eat-\mk{cv.ipf} because thank-\mk{pv-hort}//
\glft \trsl{Please open the door.}//
\endgl
\xe

\subsection{Subjunctive}

The subjunctive mood (glossed \mk{sbj}) is used for actions or events that are not or are not known to be true or factual. The subjunctive is formed using the suffix \ird{-íl}

\begin{table}[ht!]
	\footnotesize\sffamily
	\caption{Conjugation of the verb \ird{piaštá} in the subjunctive.}
	\medskip
	\begin{tabu}to 0.7\textwidth{@{}YYY@{}}
		\toprule
		&{\sc imperfective} &{\sc perfective}\\
		\midrule
		Agentive	& piaščílá	& piaščíš\\
		Patientive	& piaštnílá		& piaštníš\\
		Benefactive	& piaštébílá		& piaštebíš\\
		Locative	& piaštounílá		& piaštouníš\\
		Instrumental& dopiaštébílá	& dopiaštebíš\\
		Reflexive	& upiaščílá	& upiaščíš\\
		\bottomrule
	\end{tabu}
\end{table}

In addition, the copula has two subjunctive forms, the non-negative \ird{niec} and the negative \ird{vaše}.

Note that the Iridian subjunctive makes neither temporal nor aspectual distinction.

\par The following are some specific uses of the subjunctive mood in Iridian:

\subsubsection{Subjunctive of purpose}

Dependent clauses expressing purpose are marked in the subjunctive and normally end in \irdp{te}{in order to} and \irdp{az}{lest}

\pex
\begingl
\gla Traví prehlavnílá te traumašt stojnik.//
\glb bread-\Gen{} buy-\mk{pv-subj.ipf} {so:that} bakery go-\mk{lv-pf}//
\glft \trsl{(I) went to the bakery to buy some bread.}//
\endgl
\xe

\pex
\begingl
\gla Hreščílá te piaščeví.//
\glb be:alive-\mk{av-subj.ipf} {so:that} eat-\mk{lv-cont}//
\glft \trsl{We eat to live.}//
\endgl
\xe

\pex
\a
\begingl
\gla Se vdinílá az varšek.//
\glb \Refl{} see-\mk{pv-subj.ipf} {lest} leave-\Av{}-\Pf{}//
\glft \trsl{(I) left so as not to be seen.}//
\endgl
\a
\begingl
\gla Vdinílá az varšek.//
\glb see-\mk{pv-subj.ipf} {lest} leave-\Av{}-\Pf{}//
\glft \trsl{(I) left so that (it) may not be seen.}//
\endgl
\xe



\subsubsection{jussive/desiderative}
\par The subjunctive is used in indirect constructions of verbs for issuing orders, commanding, exhorting, etc.
\pex
\begingl
\gla Martin na America žnožíl to čeznašálic.//
\glb Martin \Loc{} America-\Acc{} study-\mk{av-sbj} \mk{rz} want-\mk{av-cont-3s.anim}//
\glft `He wants Martin to study in America.'//
\endgl
\xe

\pex
\begingl
\gla Beatles-že >>Yesterday<< Mark\k{a} zášníl to Tunek dálek.//
\glb Beatles-\Gen{} ``Yesterday'' Marek-\Agt{} sing-\mk{pv-sbj} \mk{rz} Tunek say-\mk{pf}//
\glft `Tunek told Marek to sing.'//
\endgl
\xe

\subsubsection{dubitative}
\par The subjunctive is used with verbs expressing doubt, uncertainty or disbelief.

\pex
\begingl
\gla še //
\glb Beatles-\Gen{} ``Yesterday'' Marek-\Agt{} sing-\mk{sbj} \mk{rz} Tunek say-\mk{pf}//
\glft `Tunek told Marek to sing.'//
\endgl
\xe

\subsubsection{with verbs expressing emotion}

\pex
\begingl
\gla Marek zašníl to Tunek dálek.//
\glb Marek sing-\mk{sbj.ipf} \mk{rz} Tunek say-\mk{pf}//
\glft `Tunek told Marek to sing.'//
\endgl
\xe


\subsubsection{with the conditional mood}
\par The subjunctive is used in the main clause if the verb in the dependent clause is in the conditional \textit{irrealis} mood.

\pex
\begingl
\gla Dá prezident jenem, //
\glb a//
\glft a//
\endgl
\xe

\subsubsection{expressing judgment}

\pex
\begingl
\gla Zavnočilaš to tévét //
\glb respond-\mk{av-sbj.ipf-2s} \mk{rz} important//
\glft \trsl{It is important that you respond.}//
\endgl
\xe

\subsubsection{irrealis}

\subsection{Conditional Mood}\index{conditional mood}\label{sec:conditional}
\par The conditional mood is used for conditional or hypothetical clauses. The table below shows the conjugation paradigm for the conditional mood for both regular verbs and the copula. The Iridian conditional mood is not a true conditional mood grammatically, since it is marked on the verb in the dependent clause (protasis), instead of the main clause.

\begin{table}[ht!]
	\footnotesize\sffamily
	\caption{Conjugation paradigm in the conditional mood for regular \\verbs, the copula and the existential particle \ird{ješ}.}\medskip
	\begin{tabu} to 0.8 \textwidth	{@{}Y[1.1]Y[1.3]YY@{}}
		\toprule
		&{\sc regular verbs} & {\sc copula} & {\sc existential}\\
		\midrule

		\textit{Realis} 				&{-ič} &víne & jako\\
		Neg. \textit{Realis}		&{-čn\v{e}}&ve&neko\\

		\textit{Irrealis} 			& {-išče}& jenem & jenem\\
		Neg. \textit{Irrealis} 	& {-iščen\v{e}}& jet & nét\\
		\bottomrule
	\end{tabu}
\end{table}

\subsubsection{Conditional Realis}

\par The conditional \textit{realis} mood (glossed \mk{cond.rl}) is used in two ways:
\begin{enumerate}
	\item In sentences that express a factual implication rather than a hypothetical situation or a potential future event, e.g., `If you heat water to 100 C, it will boil.'
	\item In `predictive' constructions, i.e., those that concern probable future events.
\end{enumerate}

The conditional \emph{realis} mood requires the verb in the main clause to be in the indicative.

\subsubsection{Conditional Irrealis}
The conditional \textit{irrealis} mood (glossed \mk{cond.irr}) is used with hypothetical, typically counterfactual, events. The \emph{irrealis} mood requires the main clause to be in the subjunctive.

\section{Evidentiality}\label{sec:quotative}\index{quotative}\index{evidentiality}

Iridian marks {\scshape evidentiality} as a separate grammatical category, distinguishing between a marked {\scshape quotative} or {\scshape reportative} representing secondhand information or hearsay (or more idiomatically when the speaker wishes distance themself from the statement by saying that the information is not coming directly from them) and an unmarked form representing `everything else' (cf. \cite[31-33]{aikhenvald2004}). The quotative form of a finite verb (and of some non-finite verb forms) is seen in Table \ref{tbl:quotative}. The syntax of quotative constructions is discussed in detail in \S\,\ref{sec:quotative-const}.


\begin{table}
\footnotesize \sffamily
	\caption{Sound changes used in deriving quotative form of verbs}
	\medskip
	\label{tbl:quotative}
	\begin{tabu} to \textwidth {@{}Y[0.9]Y[0.7]Y@{}}
		\toprule
		{\sc verbal form}			&	{\sc sound change}				& {\sc example}\\
		\midrule
			\multicolumn{3}{@{}l}{{\sc indicative}}\\
				\quad Perfective 		&
				\ird{-ek} $\rightarrow$ \ird{ice}	&
				\ird{piašček} $\rightarrow$ \ird{piaščice}\\
				\quad Retrospective &
				\ird{-aní} $\rightarrow$ \ird{án\v{e}} &
				\ird{piaščaní} $\rightarrow$ \ird{piaščánie}\\
				\quad Continuous &
				\ird{-eví} $\rightarrow$ \ird{ev\'ije} &
				\ird{piaščeví} $\rightarrow$ \ird{piaščev\'ije}\\
				\quad Progressive &
				\ird{-ime} $\rightarrow$ \ird{imej\'i} &
				\ird{piaščime} $\rightarrow$ \ird{piaščimej\'i}\\
				\quad Contemplative &
				\ird{-ách} $\rightarrow$ \ird{áže} &
				\ird{piaščách} $\rightarrow$ \ird{piaščáže}\\
				\quad Prospective &
				\ird{-ujám} $\rightarrow$ \ird{-ujime} &
				\ird{piaščujám} $\rightarrow$ \ird{piaščujime}\\
				\quad Cessative &
				\ird{-óvít} $\rightarrow$ \ird{-óvíce} &
				\ird{piaščóvít} $\rightarrow$ \ird{piaščóvíce}\\
			\multicolumn{3}{@{}l}{{\sc subjunctive}}\\
				\quad Imperfective &
				\ird{-\'il\'a} $\rightarrow$ \ird{-el\v{e}} &
				\ird{piaščóvít} $\rightarrow$ \ird{piaščóvíce}\\
				\quad Perfective &
				\ird{-i\v{s}} $\rightarrow$ \ird{-i\v{s}ej\'i} &
				\ird{piaščóvít} $\rightarrow$ \ird{piaščóvíce}\\
			\multicolumn{3}{@{}l}{{\sc imperative, \&c.}}\\
				\quad Imperative &
				\ird{-\'im} $\rightarrow$ \ird{-\'imen\'i} &
				\ird{piaščóvít} $\rightarrow$ \ird{piaščóvíce}\\
				\quad Prohibitive &
				\ird{-\'ema} $\rightarrow$ \ird{-\'emn\v{e}} &
				\ird{piaščóvít} $\rightarrow$ \ird{piaščóvíce}\\
				\quad Hortative &
				\ird{-k\'a} $\rightarrow$ \ird{-kaje} &
				\ird{piaščóvít} $\rightarrow$ \ird{piaščóvíce}\\
				\quad Neg. Hortative &
				\ird{-ku} $\rightarrow$ \ird{-kajen\'i} &
				\ird{piaščóvít} $\rightarrow$ \ird{piaščóvíce}\\
			\multicolumn{3}{@{}l}{{\sc other forms}}\\
				\quad Supine of purpose &
				\ird{-it} $\rightarrow$ \ird{-itej\'i} &
				\ird{piaščóvít} $\rightarrow$ \ird{piaščóvíce}\\
				\quad Supine of necessity &
				\ird{-\'a\v{s}} $\rightarrow$ \ird{-\'a\v{s}e} &
				\ird{piaščóvít} $\rightarrow$ \ird{piaščóvíce}\\
				\quad Nominalised form&
				\ird{-ou} $\rightarrow$ \ird{-uje} &
				\ird{piaščóvít} $\rightarrow$ \ird{piaščóvíce}\\
			\bottomrule
	\end{tabu}

\end{table}

\subsubsection{Quotative forms of the copula}

%% TODO format as table
Copula
Indicative
neví
hvem
Subj
nehlí
niec

Existential
Indicative
jeho
nežní
Subj
houve
hvaš


\section{Modality}\index{modality}\label{sec:modality}

Iridian can express modality either through verbal morphology, using the affixes listed in table \ref{tbl:modality}, or through a periphrastic construction. In general a periphrastic construction is preferred when the verb is non-dynamic, i.e., the sentence is merely descriptive or stative in nature (compare, for example English\index{English} \trsl{Mary can sing} vs. \trsl{Mary was able to finish baking the cake}), while the morphological method is used otherwise.

\begin{table}[ht!]
    \footnotesize\sffamily
    \caption{Verbal affixes to express modality.}
    \medskip
    \label{tbl:modality}
    \begin{tabu}to 0.6\textwidth{@{}YYY@{}}
			\toprule
				 {\sc modality} & {\sc positive} & {\sc negative}\\
				 \midrule
         Debitive & {-aln-} & {-išk-}\\
         Desiderative & {-án-}&{-ušh-}\\
         Potential &{-ét-} & {-évn-}\\
			\bottomrule
    \end{tabu}
\end{table}

The affixes used to mark modality as listed in table \ref{tbl:modality} attach directly to the verb stem, subject to the usual morphophonemic rules.

\pex\a \irdp{piaštá}{to eat}
\a \irdp{piaštalná}{to need to eat}
\a \irdp{piaštišká}{to not need to eat}
\a \irdp{piaštáná}{to want to eat}
\a \irdp{piaštušhá}{to not want to eat}
\a \irdp{piaštétá}{to be able to eat}
\a \irdp{piaštévná}{to not be able to eat}
\xe

As in most languages, modal constructions in Iridian exhibit significant {\scshape polysemy}\index{polysemy} (i.e. a single construction can have one or more interpretation depending on the context). For example consider the following sentence:

\pex
\begingl
\gla Tomáš rušku zahviržétách.//
\glb Tomáš Russian-\Ins{} speak-\Av{}-\Pot{}-\Ctp{}//
\glft \trsl{Tomáš will be able to speak Russian}//
\endgl
\xe

The following translations are all equally possible without any further contextual clues:

\pex
\a \trsl{Tomáš will be able to speak Russian, if he will study it.} (abilitative)
\a \trsl{Tomáš will be able to speak Russian because he will be allowed to do it.} (permissive)
\a \trsl{Tomáš can speak Russian and he will probably speak it later.} (true potential modality)
\xe

\subsection{Potential modality}\index{potential modality}\index{abilitative}\index{permissive}\index{modality}

Potential modality (glossed as \mk{pot}) is used when, in the speaker's opinion, an event is possible to occur. This definition makes the potential mood in Iridian encompass both the expressions of ability and permissibility.

\pex
\begingl
\gla To švirek moc gruševí še oštinévnílá.//
\glb this handwriting too be:small-\mk{cont} with read-\mk{pv-npot-sbj.ipf}//
\glft \trsl{The handwriting is too small (I) am unable to read it.}//
\endgl
\xe

\subsection{Debitive modality}\index{debitive modality}\index{modality}

The debitive form of a verb expresses necessity. This verb form is now mainly confined in literary usage, and has been entirely replaced in colloquial Iridian by the supine of necessity.\index{supine} The negative debitive form however has survived and is still in common use. The negated form\index{negation} of the positive debitive (in contrast to the negative form) is also no longer used in colloquial Iridian, and the negative debitive coexists instead with the negated form of the supine of necessity, with subtle differences in meaning.

\begin{multicols}{2}
  \pex
  \a
  \begingl
  \gla Tóm zoštináš.//
  \glb book \Neg{}-read-\SupN{}//
  \glft \trsl{(We) don't need to read the book.}//
  \endgl
  \a \begingl
  \gla Tóm oštniškeví.//
  \glb book read-\Pv{}-\N{}\Deb{}-\Cont{}//
  \glft \trsl{The book should not be read (i.e., it is prohibited).}//
  \endgl
  \xe
\end{multicols}

The negative debitive form is particularly common in written warnings/prohibitions.
\pex \begingl
  \gla Ran\v{z}i\v{s}kev\'i.//
  \glb smoke-\Av{}-\N{}\Deb{}-\Cont{}//
  \glft \trsl{No smoking.}//
  \endgl
\xe

\subsection{Periphrastic constructions}

\section{Non-finite verb forms}

\subsection{Infinitive}\label{sec:infinitive}\index{infinitive}

The {\scshape infinitive} is a fossilised verb form that was used in Old Iridian\index{Old Iridian} (and arguably in Early Middle Iridian\index{Middle Iridian}) as a verbal noun occupying the topic\index{topic} position in a sentence. In Modern Iridian this use has been completely supplanted by the gerund\index{gerund} and the infinitive is only used as the citation form\index{citation form} of verbs. All infinitives in Iridian end in the vowel \ird{-á} and the consonant immediately preceding it is called the verb's thematic consonant.\index{thematic consonant}.

\subsection{Nominalised forms and gerunds}\index{gerund}\index{nominalisation}\label{nom-morph}

Nouns can be routinely derived from verbs and verb phrases using the nominalising suffix \ird{-ou} (glossed as \Nz{}). Linguists generally recognise three types of nominalisation: event nominals, which describe an event the same way the parent verb does, and which could either be (1) simple or (2) complex, with {\scshape complex event nominals} ({\sc cen}s)\index{nominalisation!event nominal}\index{event nominal|see{nominalisation, event nominal}} allowing internal arguments and {\scshape simple event nominals} ({\sc sen}s)\index{nominalisation!event nominal} not; and (3) {\scshape resultant nominals}\index{nominalisation!resultant nominal}\index{resultant nominal|see{nominalisation, resultant nominal}}, which describe an event similar but not exactly corresponding to the even described by the main verb (\cite{grimshaw1990}; \cite{moulton2014}). In English\index{English}, for example, where verbs can be nominalised using a variety of derivational affixes or with zero derivation, these types are not distinguished, as we see below:

\pex[interpartskip=0pt]
	\a The examination of the students lasted a long time. \hfill {\sc cen}
	\a The examination lasted a long time.\hfill {\sc sen}
	\a The examination was photocopied on green paper.\hfill {\sc rn}\\
	\trailingcitation{(\cite[2]{alexiadou2008})}
\xe

Some verbs in Iridian allow the formation of {\sc rn}s using the suffix \ird{-ou} and the uninflected verb root (e.g., \irdp{piaštou}{food} from \irdp{piaštá}{to eat}). For the vast majority, however, {\sc rn}s are produced by lexical suppletion, i.e., the {\sc rn}s are not morphologically derived (or explicitly so, at least) using the nominalising suffix (see \S~\ref{sec:nomder-verb}). As in English, {\sc sen}s and {\sc cen}s are not morphologically distinct in Iridian, and are formed with the suffix \ird{-ou} used in conjunction with the prefix \ird{po(d)-}. We call this form the {\scshape gerund}.

In addition to these three types of nominalisation introduced in \textcite{grimshaw1990}, Iridian recognises a fourth type, which produces a nominal that refers not to the event itself but one of the event's participants, i.e., one of the verbs arguments. We will call this type a {\scshape participant nominal} ({\sc pn}) (cf. \cite[400-5]{schackow2015}; \cite[297-8]{okuna}).

\pex \a Nominalised forms of \irdp{piaštou}{to eat} showing a productive morphological {\sc rn}:\smallskip\\
		\vtop{\halign{%
			#\hfil& \qquad #\hfil\cr
			\quad Infinitive:		& \irdp{piaštá}{to eat}\cr
			\quad Morphological {\sc rn}:	& \irdp{piaštou}{food}\cr
			\quad Gerund ({\sc sen/cen}):	& \irdp{popiaštou}{the act of eating}\cr
			\quad {\sc pn}:				& \irdp{piaščkou}{the person/thing who ate}\cr
		}}
\a Nominalised forms of \irdp{vadá}{to think} showing a defective morphological {\sc rn} and the alternative lexical {\sc rn}:\smallskip\\
	\vtop{\halign{%
		#\hfil& \qquad #\hfil\cr
			\quad Infinitive:		& \irdp{vadá}{to think}\cr
			\quad Morphological {\sc rn}:	& \ird{*vadou} (ungrammatical)\cr
			\quad Lexical {\sc rn}:		& \irdp{vied}{thought (n.)}\cr
			\quad Gerund ({\sc sen/cen}):	& \irdp{povadou}{the act of thinking}\cr
			\quad {\sc pn}:				& \irdp{vadnikou}{that which was thought}\cr
}}\xe


Event nominals (viz., gerunds) are therefore inherently abstract and active in meaning; in addition, they are also understood to be tenseless and aspectless

Gerunds\index{gerund} have an active meaning. The suffix \ird{-ál}, used to mark the continuous aspect, may be infixed to the gerund to indicate that the action is repetitive.

\pex
\a
\begingl
\gla Jan nidek.//
\glb Jan stand.up-\mk{pf}//
\glft \trsl{Jan stood up.}//
\endgl
\a
\begingl
\gla Janí ponidálou buvec.//
\glb Jan-\Gen{} \mk{ger}-stand.up-\mk{cont-nz} annoying//
\glft \trsl{Jan's standing up again and again is annoying.}//
\endgl
\xe

The syntax of event and participant nominals is discussed in further detail in \S~\ref{sec:nomz-syntax}.

\subsection{Converbs}\index{converb}
Converbs (glossed \Cv{}) are a non-finite verb form often used for adverbial constructions. There are two converb forms in Iridian: the imperfective \ird{iec} (glossed \mk{cv.ipf}) and the perfective \ird{-u} (glossed \mk{cv.pf}).

\pex
\begingl
\gla Tereza kravn\v{e}c nóví palžek. //
\glb Tereza cry-\mk{cv.ipf} room-\Gen{} leave-\Av{}-\Pf{}//
\glft \trsl{Tereza left the room crying.}//
\endgl
\xe

\pex
\begingl
\gla Nóví palzu Tereza neikravnašek.//
\glb room-\Gen{} leave-\mk{cv.pf} Tereza \mk{incho}-cry-\mk{pf}//
\glft `Having left the room, Tereza started to cry.'//
\endgl
\xe

The syntax of converbial constructions and the specific uses of the perfective and imperfective converb form are discussed in detail in \S\,\ref{converbs-syntax}.


\subsection{Supine}

The {\scshape supine}\index{supine} is a non-finite verb form formed used to indicate necessity or purpose. Both usage has a nominal and a non-nominal form (used similar to an adverb or an adjective), giving the supine a total of four forms, as shown below:

\begin{table}[ht!]
	\sffamily\footnotesize
	\caption{Endings used for the supine.}
	\medskip
	\begin{tabu} to 0.6\textwidth{@{}YYY@{}}
		\toprule
		&{\sc purpose}&{\sc necessity}\\
		\midrule
		Nominal & {-it} & {-áš}\\
		Non-nominal & {-ice} & {-ášce}\\
		\bottomrule
	\end{tabu}
\end{table}

These four forms are invariable. The endings attach to the verb after the root has been conjugated for voice. The use of the non-nominal forms, in addition, does not require the use of the linking particle \ird{ko}.

\pex
\begingl
\gla >>Ána Karenina<< za gnaža oštinášce tóm.//
\glb Anna Karenina for school-\Acc{} read-\Pv{}-\SupN{} book//
\glft `I have to read \textit{Anna Karenina} for school.'//
\endgl
\xe

Although the usage of the supine has evolved to include various other constructions not related to its origins as a verbal noun indicating motion, the supine is still used in Modern Iridian in this original sense, accompanying a main verb (often a verb of motion) to indicate purpose. Both the nominal and the non-nominal form can be used in this construction, with the nominal form (despite being a more recent syntactic innovation) being more common and the non-nominal form considered more archaic, but still more prevalent in literary and formal usage. This usage roughly corresponds to the English infinitive, as in the sentence \trsl{I came here \emph{to bury} Cæsar.} When using the nominal form the clause containing the main verb is first transformed into a \ird{to-}clause and then equated to the nominal supine; when using the non-nominal form, on the other hand, the supine is simply added before the main verb.

\pex
\a
\begingl
\gla Tóm behlenik to oštnit.//
\glb book buy-\Pv{}-\Pf{} \Rz{} read-\Pv{}-\SupP{}//
\glft `I bought the book to read.'//
\endgl
\a
\begingl
\gla Tóm oščice behlenik.//\deftagex{supine}\deftaglabel{lit}
\glb book read-\Av{}-\SupP{} buy-\Pv{}-\Pf{}//
\glft `I bought the book to read.'//
\endgl
\xe

Especially when using the non-nominal construction, the grammatical voice used for the supine does not need to be the same as the one used in the main verb, as we see in example (\getfullref{supine.lit}). The supine can only take one argument, an object, which is always marked in the genitive regardless of its grammatical voice used to mark the supine governing it.

\pex
\a\begingl
\gla Marjám [těží probem\'i vednice] stožek.//
\glb Mary god-\Gen{} sepulchre-\Gen{} see-\Pv{}-\SupP{} go-\Av{}-\Pf{}//
\glft \trsl{Mary went to see the Lord's sepulchre.}//
\endgl
\a\begingl
\gla Marjám [těží probem\'i vi\v{z}ice] stožek.//
\glb Mary god-\Gen{} sepulchre-\Gen{} see-\Av{}-\SupP{} go-\Av{}-\Pf{}//
\glft \trsl{Mary went to see the Lord's sepulchre.}//
\endgl
\xe

In addition to this original usage, and to their use in indicating purpose or necessity, the supine\index{supine} is quite heavily employed idiomatically in Iridian. In colloquial speech, the supine of purpose is often used to express future or probable events as a substitute to the contemplative aspect. In both colloquial and literary registers, it may also be used to indicate a habitual action or a general truth (instead of the continuous or progressive aspect) when the verb implies some sort of purpose or consequentiality, especially in relation to another verb.

\pex
\begingl
\gla Dá to tómí oščit.//
\glb \First{}\Sg{} this book-\Gen{} read-\Av{}-\SupP{}//
\glft `I will be reading this book.' \emph{Lit.,} `I am someone whose purpose is the reading of this book.'//
\endgl
\xe

\pex
\begingl
\gla Méva dousa ješ me bylu dnou má nemel toha ohlečit.//
\glb all adult-\Acc{} \Exst{} as child-\Ins{} front but few:people this.\Acc{} remember-\Av{}-\SupP{}//
\glft \trsl{All grown-ups were children once but only a few remember it.}//
\endgl
\xe

Another common construction involves the supine of necessity with the words \irdp{shlac}{now} \irdp{mál}{time} (or less frequently \irdp{ór}{hour}). This construction is somehow similar to English\index{English} \trsl{It's time we left} or \trsl{It's time for us to go.} When used this way, the supine is conjugated in the locative voice.\index{supine}

\pex
\begingl
\gla Shlac himatí palzounášce mál.//
\glb now homeland-\Gen{} leave-\Lv{}-\SupN{} time//
\glft `It's time (we) left our homeland.'//
\endgl
\xe
\pex
\begingl
\gla Sa tet. Shlac zdalounášce mál.//
\glb already noon now have:breakfast-\Lv{}-\SupN{} time//
\glft `It's already late (\emph{lit.,} noon). It's time (we) had breakfast.'//
\endgl
\xe

\section{Stative verbs}\index{verbal adjective|see{stative verb}}\index{stative verb}\index{adjectives}\label{sec:statives}

Iridian lacks a distinct class of adjectives.\footnote{There is however a small class of attributives, which includes deictics\index{deictics} and quantifiers\index{quantifiers} among others, which can function as modifiers. They are different in that these words cannot be used as the predicate\index{predicate} of a sentence. They are discussed in detail on Chapter \ref{chap:minor}.} Instead, a special class of verbs called {\scshape stative verbs} are used to modify noun or noun-like classes. Unlike most verbs, however, stative verbs can only be marked for aspect, and optionally for voice. In addition to this base form (called the {\scshape copulative}), stative verbs also have an {\scshape attributive} form (used when the verb is preceding the noun or noun phrase) and {\scshape nominative} form (representing a concrete nominalisation of the verb), both of which are absent in non-attributives verbs. Consider for example the verb \ird{všihná} \trsl{to be angry}:

\begin{table}[ht]
	\sffamily\footnotesize
	\caption{Conjugation pattern for stative verbs}
	\medskip
	\begin{tabu} to 0.7\textwidth{@{}YY[0.8]Y@{}}
		\toprule
		&{\sc ending}&{\sc example}\\
		\midrule
		Copulative & varies & varies\\
		Attributive & {-í} & {všihní}\\
		Nominative & {-ou}	& {všihnou}\\
		\bottomrule
	\end{tabu}
\end{table}

\subsection{Copulative and attributive forms}\index{stative verb}
The copulative form of stative verbs is used when the verb is the predicate of the sentence. This form is only conjugated for aspect, and optionally for voice. Unlike normal verbs, however, stative verbs cannot be conjugated in the agentive voice since Iridian grammar does not distinguish between agency in an actor and the description of a state in stative verbs, both of which are encoded in the definition of this class. 

\ex
\begingl
\gla Mamka všihneví \emph{(not} *všihnaševí\emph{)}//
\glb mother-\mk{dim} be:angry-\mk{cont} not be:angry-\Av{}-\Cont{}//
\glft \trsl{My mother is angry.}//
\endgl
\xe


The attributive form is derived by replacing the infinitive marker \ird{-á} with \ird{-í}. Other than its conjugated comparative form ending in \ird{-ení}, the attributive form is invariable. The comparative form\index{comparative construction}\index{comparison} is often used, especially in colloquial speech, as an intensifier, even if the stative verb is not actually used in a comparison.

\ex
\begingl
\gla Všihnení mamka télévoniržek.//
\glb be:angry-\mk{comp-att} mother-\mk{dim} call-\Av{}-\Pf{}//
\glft \trsl{Mother was fuming (\emph{lit.,} angrier) when she called us.}//
\endgl
\xe

Because of the invariability of the attributive form, the copulative form may sometimes be used as a modifier, similar to a normal verb, separated from the noun it modifies with the particle \ird{ko}. Note, however, that when conjugated in the continuous aspect (except when marked explicitly for voice), such usage is not grammatical, with Iridian only allowing the attributive.

\ex
\begingl
\gla Všihninek ko tieho snov uprožilzách.//
\glb be:angry-\Pv{}-\Pf{} \Att{} god soon \Refl{}-avenge-\mk{av-ctpv} //
\glft \trsl{God whom you have angered will seek vengeance soon.}//
\endgl
\xe


\subsection{Nominative form}\index{stative verb}
The nominative form is derived by replacing the infinitive marker \ird{-á} with the nominalising suffix \ird{-ou}. This is the same nominalising suffix used to form nouns from regular verbs, the only difference being that stative verbs allow the suffix to be attached directly on the verb's root.

The copulative form may also be nominalised with \ird{-ou}.\index{nominalisation} However, as with the attributive form, if the copulative stative verb is conjugated in the continuous aspect and is unmarked for voice, the nominal form is used instead of the nominalised copulative form.

\subsection{Stative verbs and voice}\index{voice}\index{stative verb}

In general, stative verbs can also be conjugated for voice, with two main differences: first, as mentioned earlier in this section, the agentive voice cannot be used with stative verbs as Iridian does not distinguish between stative and agentive verbs and such information is considered to be encoded by default in the stative form; and second, in view of the first point, the benefactive gains an ``agentive'' interpretation, as it is used when the subjective is the agent of the action leading to the state being described by the verb, as in the example below:

\ex
\begingl
\gla Zuštalébkou houba.//
\glb be:happy-\mk{ben-pf-nz} gift//
\glft \trsl{What made me happy was (your) gift.}//
\endgl
\xe



\section{Derivational morphology}
\subsection{External derivation}
\par Loanwords ending in \textbf{-ace} from the Latin change the final e to á:
\begin{table}[h!]
	\centering \small
	\begin{tabu} to 0.9\textwidth{>{\bfseries}YM[0.3]>{\bfseries}YY}
		administrace 	& $\rightarrow$ & administracá 	& `to administrate' \\
		akuzace			& $\rightarrow$ & akuzacá		& `to accuse'\\
		diferenzace		& $\rightarrow$ & diferenzacá	& `to differentiate'\\
		separace		& $\rightarrow$ & separacá		& `to separate'\\
	\end{tabu}
\end{table}
\par Some Latin loanwords are borrowed first from German. Loanwords ending in \textbf{-ieren} become \textbf{-irná}.
\begin{table}[h!]
	\centering \small
	\begin{tabu} to 0.9\textwidth{>{\bfseries}YM[0.3]>{\bfseries}YY}
		akzeptieren 	& $\rightarrow$ & akceptirná 	& `to accept' \\
		konservieren	& $\rightarrow$ & koncervirná	& `to conserve'\\
		produzieren		& $\rightarrow$ & producirná	& `to produce'\\
		vandalieren		& $\rightarrow$ & vandalirná 	& `to deface'\\
	\end{tabu}
\end{table}
\subsection{Internal Derivation}
			
\chapter{Nouns}

Nominal morphology in Iridian is relatively simpler compared to the
corresponding process with verbs. Where possible, Iridian sentences are
generally constructed to leave the noun or noun phrase unmarked.

\section{Grammatical categories}\label{sec:gramm-categories}

\section{Number}\label{sec:number}
\index{grammatical number}\index{plural}

Nouns in Iridian are not formally marked for number. Thus the word \ird{byl},
for example, can mean either \trsl{child} or \trsl{children} depending on the
context. The same form is used when the noun is preceded by a numeral. Thus in
Iridian one says \irdp{ona byl}{one child} and \irdp{hroná byl}{three children}.

Nevertheless, Iridian can express semantic plurality by using quantifiers,
numerals, pluralizing particles or even through context alone. One such particle
is \ird{ně}\label{sec:plurals}\footnote{Cf. \posscite{schachter1983} treatment
of Tagalog\index{Tagalog} pluralizing particle \emph{mga}.}. \ird{Ně} is a
proclitic and attaches to the left-most part of the noun phrase or the verb
phrase it modifies. Throughout this book, \ird{ně} will be glossed as \Pl{}= for
simplicity.

\pex
\begingl
    \gla ně ša zuštalí byl//
    \glb \Pl{}= \Dem{}.\Prox{} be:happy-\Att{} child //
    \glft \trsl{these happy children}//
\endgl
\xe

\ird{Ně} however could be understood to have three distinct uses. The first, as
mentioned above, is to mark plurality. Alternatively, \ird{ně} could also be use
as an approximative\index{approximation} (roughly equivalent to English
\trsl{about}) when used with cardinal numbers or time expressions or as a
honorific expletive\index{honorific}\index{expletive} to show politeness when
used with proper names\index{proper names} or with some nouns (mostly related to
kinship terms\index{kinship terms}). In its use for approximation, \ird{ně} is
interchangeable with \irdp{u}{about}, although it is common in spoken speech to
combine the two as an intensified construction. Preference is given to \ird{ně},
however, if the noun being modified is the topic of the sentence and must
therefore remain unmarked.

\pex
\begingl
    \gla Ně mlazka-no scenžek?//
    \glb \Hon{}= brother\mk{-dim=q} arrive-\Av{}-\Pf{} //
    \glft \trsl{Was my brother the one who arrived?}//
\endgl
\xe

Note that when used with a cardinal number, \ird{ně} can only be understood to
signify approximation, i.e., (\getfullref{appr.1}) can only mean \trsl{about
three children} and not \trsl{three children}, as the latter would only be
translated as \ird{hroná byl} without the clitic \ird{ně}.

As has been earlier mentioned, \ird{ně} is a proclitic\index{proclisis} and
attaches to the left-most part of the noun phrase or verb phrase it modifies,
including any modifier no matter how complex but excluding any proposition. In
some cases, as can be seen in (b) and (c) below, the use of \ird{ně} to
pluralize a noun can imply definiteness\index{definiteness}.

\pex
\a
\begingl\deftagex{pl}\deftaglabel{1}
    \gla {ně} za byla tóm//
    \glb \Pl{}= for child-\Acc{} child //
    \glft \trsl{books for children}//
\endgl
\a
\begingl\deftaglabel{2}
    \gla za {ně} byla tóm//
    \glb for \Pl{}= child-\Acc{} child //
    \glft \trsl{a book for (these) children}//
\endgl
\a
\begingl\deftaglabel{3}
    \gla {ně} za \textbf{ně} byla tóm//
    \glb \Pl{}= for \Pl{}= child-\Acc{} child //
    \glft \trsl{books for (these) children}//
\endgl
\xe

The use of \ird{ně}, however, is largely optional and where plurality can be
implied from context, this particle is seen as redundant and is therefore
dropped.

\ird{Ně} cannot be used with mass and uncountable nouns, as well as with
abstract nouns.

\pex
\a
\begingl
\gla *Na duma ně ješ piaštou.//
\glb \Loc{} house \Pl{} \Exst{} food//
\glft \trsl{There is food in the house.}//
\endgl
\a
\begingl
\gla Na duma tohle ješ piaštou.//
\glb \Loc{} house much \Exst{} food//
\glft \trsl{There is a lot of food in the house.}//
\endgl
\xe

The particle \ird{ně} always precedes the noun it modifies, except in
existential clauses where it comes before the existential particle
\ird{ješ}\footnote{The sequence is pronounced as if written něš [ɲɛɕ]}.
\ird{Ně} can obviously not be used with the negative particle
\ird{niho}.\index{niho}\index{existential construction}\index{ješ}

\pex
\a
\begingl
\gla ně bžem//
\glb \Pl{} bee//
\glft \trsl{bees}//
\endgl
\a
\begingl
\gla Ně ješ bžem.//
\glb \Pl{} \Exst{} bee//
\glft \trsl{There are bees.}//
\endgl
\a
\begingl
\gla *Ně niho bžem.//
\glb \Pl{} \N{}\Exst{} bee//
\glft \trsl{There are no bees.}//
\endgl
\xe

\index{pluralia tantum}
\ird{Ně} cannot be used with a limited number of nouns, mostly referring to
paired body parts and related objects, which in the base form is understood to
refer to the pair itself and thus cannot be pluralized. If the speaker wishes to
explicitly refer to one piece of the pair, the noun \ird{noma} (an obsolete form
of the word for one-half, now surviving only in this construction) and the
genitive form of the body part.

\pex
\begingl
\gla Eg zaromnek.//
\glb eyes close-\Pv{}-\Pf{}//
\glft \trsl{(He) closed (his) eyes.}//
\endgl
\xe
\pex
\begingl
\gla Pohár dévit.//
\glb eyeglasses dirty//
\glft \trsl{(Your) eyeglasses are dirty.}//
\endgl
\xe
\pex
\begingl
\gla Ohví noma utieščál.//
\glb shoe-\Gen{} half \Refl{}-lose-\Av{}-\Cont{}//
\glft \trsl{The other pair of (his) shoe is missing.}//
\endgl
\xe

The base form is also used in generic statements where English would normally
use the plural.\index{generic statements}\index{universals} When used with a
proper noun \ird{ně} can be translated with the English construction \trsl{and
others}. Note that this is different from the usage of \ird{ně} as a honorific.

\pex
\begingl
    \gla Ně Jancě gnaž uprubížice.//
    \glb \Pl{}= Janek-\Gen{} school \Refl{}-burn-\Av{}-\Pf{}-\Quot{} //
    \glft \trsl{I heard Janek's school burned down.}//
\endgl
\xe

\pex
\begingl
    \gla Ně Marek zázdalšek.//
    \glb \Pl{}= Marek \Neg{}-have:breakfast-\Av{}-\Pf{} //
    \glft \trsl{Marek and the others did not eat breakfast.}//
\endgl
\xe


\section{Definiteness}\index{definiteness}

Iridian does not have definite or indefinite articles; instead a noun or a noun
phrase's definiteness is often expressed syntactically. This is discussed in
detail in \S\,\ref{sec:definiteness}.

\section{The case system}

\subsection{Declension patterns}

Nouns in Iridian can end in a consonant or any of \ird{-a}, \ird{-e}, \ird{-ě},
\ird{o} or \ird{ou}. There are seven declension classes, determined by the
ending of the noun. Class I refers to nouns ending in a hard consonant, Class II
to nouns ending in a soft consonant, and Classes III through VII to nouns ending
in \ird{-a}, \ird{-e}, \ird{-ě}, \ird{o} and \ird{ou}, respectively. The
declension classes are summarized in Table~\ref{tab:declension}.


\begin{table}
    \footnotesize\sffamily
    \caption{Paradigm endings for the six declension classes.}\label{tab:declension}
    \medskip
	\begin{tblr}{width=\textwidth,colspec={X[0.2]XXXXXXX}}
    \toprule\addlinespace
            {\sc case}      &{\sc i} &{\sc ii} & {\sc iii} &{\sc iv} &{\sc v} &{\sc vi} & {\sc vii}\\
    \midrule\addlinespace
            Agentive        & -ám    & -ám     & -am       & -em   & -ěm    & -om   & -óvam\\ \addlinespace
            Accusative      & -a     & -a      & -e        & -y    & -y     & -im   & -óva\\ \addlinespace
            Genitive        & -í     & -ý      & -í        & -ý    & -ý     & -e    & -óví\\ \addlinespace
            Instrumental    & -u     & -u      & -u        & -u    & -u     & -u    & -óvím\\ \addlinespace
    \bottomrule
    \end{tblr}
\end{table}

Iridian declension is regular and predictable. There are no irregular nouns, and
the endings of the declension paradigms are the same for all nouns in a given
class. The only notable morphophonemic change is caused by the softening of hard
consonants when followed by the genitive ending. This softening causes the
fricativization of the velar stop /k/ to /t͡ɕ/ which in this case is spelled as
\orth{c} and not \orth{č} as would have been expected. Thus the name \ird{Janek}
is declined as \ird{Janka} in the accusative but \ird{Jancí} in the genitive.

For the purposes of nominal declension, nouns ending in /k/, /g/, /d/, /t/, /f/
and /v/ are considered hard consonants. The sibilants /s/ and /z/ and the
affricate /t͡s/, although technically `hard' consonants, take Case II endings.
Thus \irdp{mez}{room} becomes \ird{mezý} and not \ird{*mezí}. Words ending in
all other consonants also take Case II endings. 

Below are examples showing the declension paradigms in Iridian.

\pex
\a \irdp{viták}{road}\\
\vtop{\halign{%
#\hfil& #\hfil\cr
Unmarked & \ird{viták} \cr
Agentive & \ird{vitákám} \cr
Patientive & \ird{vitáka} \cr
Genitive & \ird{vitácí} \cr
Instrumental & \ird{vitáku} \cr
}}

\a \irdp{slěň}{soup}\\
\vtop{\halign{%
#\hfil& #\hfil\cr
Unmarked & \ird{slěň} \cr
Agentive & \ird{slěňám} \cr
Patientive & \ird{slěňa} \cr
Genitive & \ird{slěňý} \cr
Instrumental & \ird{slěňu} \cr
}}


\a \irdp{prěsta}{neighbor}\\
\vtop{\halign{%
#\hfil& #\hfil\cr
Unmarked & \ird{prěsta} \cr
Agentive & \ird{prěstám} \cr
Patientive & \ird{prěste} \cr
Genitive & \ird{prěstí} \cr
Instrumental & \ird{prěstu} \cr
}}

\a \irdp{vtare}{morning}\\
\vtop{\halign{%
#\hfil& #\hfil\cr
Unmarked & \ird{vtare} \cr
Agentive & \ird{vtarem} \cr
Patientive & \ird{vtary} \cr
Genitive & \ird{vtarý} \cr
Instrumental & \ird{vtaru} \cr
}}

\a \irdp{shorě}{group}\\
\vtop{\halign{%
#\hfil& #\hfil\cr
Unmarked & \ird{shorě} \cr
Agentive & \ird{shorěm} \cr
Patientive & \ird{shorý} \cr
Genitive & \ird{shorí} \cr
Instrumental & \ird{shoru} \cr
}}

\xe 


\subsection{Agentive case}\index{agentive case}

The agentive case (\Agt{}) is used to indicate the agent of an action where the agent is not the topic of the sentence.

\pex
\begingl
\gla Marek Lučkám vidnik.//
\glb Marek Luček-\Agt{} see-\Pv{}-\Pf{}//
\glft \trsl{Marek was seen by Luček.}//
\endgl
\xe

The agentive is also used in comparative constructions, where it indicates the point of reference for the comparison.

\index{comparison}\index{agentive of comparison}
\pex
\begingl
\gla Dá Marką tám stroja.//
\glb \First{}\Sg{} Marek-\Agt{} \Comp{} tall//
\glft \trsl{Marek is taller than me}//
\endgl
\xe

\subsection{Accusative case}\label{sec:accusative-case}
\index{accusative case}

In general, the accusative case is used to mark the direct object of a verb that
is in the agentive voice. Note that this usage implies that the direct object is
indefinite. Where the direct object is definite, the verb is usually in the
accusative voice and the direct object is unmarked.

\pex
\a \begingl
\gla Guláša piašček.//
\glb goulash-\Acc{} eat-\Av{}-\Pf{}//
\glft \trsl{(He) ate goulash.}//
\endgl
\a Compare this to:\\
\begingl
\gla Guláš piaštnik.//
\glb goulash eat-\Pv{}-\Pf{}//
\glft \trsl{(He) ate the goulash.}//
\endgl
\xe

The accusative is also used to mark the direct object when the verb is in the
benefactive voice.

\pex
\begingl
\gla Ša vitamina piaštebik.//
\glb \mk{3s.anim} vitamin-\Acc{} eat-\Ben{}-\Pf{}//
\glft \trsl{(She) made him take (his) vitamins.}//
\endgl
\xe

%%%%
% TODO Definiteness and the accusative; use of genitive when the noun marked is indefinite

When used to mark the direct object, the accusative implies the definiteness of
the noun. Where the noun is indefinite, the genitive is used instead.

\pex
\a\begingl
\gla Vaška piaščem.//
\glb cake-\Acc{} eat-\Av{}-\Pf{}//
\glft \trsl{I ate the cake.}//
\endgl
\a\begingl
\gla Vašcí piašček.//
\glb cake-\Gen{} eat-\Av{}-\Pf{}//
\glft \trsl{I ate some cake.}//
\endgl
\xe

The accusative is used with the particle \ird{na} to form a compound locative
case, which is itself used to indicate a general location.

\pex
\begingl
\gla Tomáš na byra.//
\glb Tomáš \Loc{} office-\Acc{}//
\glft \trsl{Tomáš is at the office.}//
\endgl
\xe

The accusative is also used with some prepositions, often indicating direction
or movement. The most common of these are \ird{na} used to indicate a general
location (\irdp{na byra}{at the office}), \ird{za} which roughly corresponds to
the English `for' or `for the benefit of' (\irdp{za Marka}{for Marek}), \ird{u}
used to indicate proximity (\irdp{u gara}{near the train station}), and \ird{do}
which indicates movement towards a location (\irdp{do byra}{to the office}).

\subsection{Genitive case}\label{sec:genitive-case}\index{genitive}

The simplest use of the genitive case is to indicate ownership or possession.
When used this way, the noun marked in the genitive must always procede the noun
it modifies.

\pex
\irdp{Marcí dum}{Marek's house}\\
\irdp{mámcí hašek}{my mother's bag}\\
\irdp{ša študencí tóm}{this student's book}\\
\xe

Demonstratives\index{demonstrative} and other modifers must always come before
the whole noun phrase and cannot split the possessor from the possessee. An
exception to this rule is the clitic \ird{ně}, which comes immediately before
the noun it pluralizes\index{plural}.

\pex
\a  \irdp{ša študencí tóm}{the/a book of this student}\\
    \irdp{to študencí tóm}{this book of the student}
\a  \irdp{ně študencí tóm}{the students' book}\\
    \irdp{študencí ně tóm}{the student's books}
\xe

The genitive is also used as a partitive 


\subsubsection{Genitive of material}

\ex
\irdp{kuní prosc}{silver spoon}\\
\irdp{be}
\xe

\subsubsection{Genitive of the whole}
The genitive can also be used to indicate

\pex
\begingl
\gla na kraštolí dnóva//
\glb \Loc{} train:station-\Gen{} front//
\glft \trsl{in front of the train station}//
\endgl
\xe

Note that the accusative and not the genitive case is used when quantifying a part of the whole.

\pex
\a
\begingl
\gla *žnohoušce hroná//
\glb student-\Gen{} three//
\glft \trsl{three of the students}//
\endgl
\a
\begingl
\gla na žnohoušca hroná//
\glb \Loc{} student-\Gen{} three//
\glft \trsl{three of the students}//
\endgl
\xe

Nevertheless when quantifying a noun \emph{per se}, and not in relation to a
whole, the uninflected form of the quantifier is used (mostly using indefinite
quantifiers such as \trsl{many}, \trsl{a lot}, etc.). If however, the
quantification involves a countable unit or division of the noun, the genitive
is used, but such unit or division must be further quantified by a numeral or an
indefinite quantifier.

\pex
\a
\begingl
\gla Na kroumašta po zma ješ pivo.//
\glb \Loc{} refrigerator-\Acc{} still few \Exst{} beer//
\glft \trsl{There's still some beer left in the refrigerator.}//
\endgl
\a
\begingl
\gla Ona pive štava unarížčem.//
\glb one beer-\Gen{} mug-\Acc{} \Refl{}-order-\mk{av-pv-1s}//
\glft \trsl{I ordered a mug of beer.}//
\endgl
\xe

\subsubsection{Genitive of movement}

The genitive is also used to indicate movement away from somewhere.

\pex
\a
\begingl
\gla Dumí palžek.//
\glb house-\Gen{} leave-\Av{}-\Pf{}//
\glft \trsl{I left the house.}//
\endgl
\a
\begingl
\gla Dum palzinek.//
\glb house leave-\Pv{}-\Pf{}//
\glft \trsl{I left the \emph{house}.}//
\endgl
\xe

The genitive is also used with certain prepositions, most of which indicate
movement away from somewhere, the source of a movement or the origin of
something, and other similar meanings. The most common of these are \ird{z}
roughly corresponding to the English `from' (\irdp{z Marcí houba}{a gift from
Marek}), \irdp{nam}{without} (\irdp{nam záhárí}{without sugar}),
\irdp{pale}{instead of} (\irdp{pale Jancí}{instead of Janek}) and
\irdp{ahte}{except} (\irdp{ahte Jancí}{except for Janek}).

\subsection{Instrumental case}\label{sec:instrumental-case}
\index{instrumental case}

The instrumental case (glossed \Ins{}) is used to indicate the means by which an
action is performed. It is also used to indicate the instrument used to perform
an action. Some grammar books may also refer to this case as the prepositional
case.

\pex
\begingl
\gla Do byra vternovím stóževí.//
\glb to office-\Acc{} bicycle-\Ins{} go-\Av{}-\Cont{}//
\glft \trsl{I ride my bike to the office.}//
\endgl
\xe

The instrumental is also used with the particle \ird{še} which roughly
corresponds to the English \trsl{with}. In some cases, \ird{še} may also be
dropped altogether (this is especially common with verbs in the sociative form).

\pex
\a\begingl
\gla Za bolte še Janku stóžách.//
\glb for party-\Acc{} with Janek-\Ins{} go-\Av{}-\Ctp{}//
\glft \trsl{(I am) coming to the party with Janek.}//
\endgl
\a\begingl
\gla Terezu skaznašek.//
\glb Tereza-\Ins{} \Soc{}-sing-\Av{}-\Pf{}//
\glft \trsl{I sang with Tereza.} or \trsl{Tereza and I sang together.}//
\endgl
\xe

The instrumental is also used with phrases expressing time (e.g.,
\irdp{mercu}{in March}, \irdp{prědné duhu}{next month}, \irdp{vicé hrona
minutu}{three minutes ago (lit., in the last three minutes.)}) with the
exception of \irdp{bych}{yesterday}, \irdp{hodně}{today} and
\irdp{prohly}{tomorrow} which are not declined. The instrumental case may also
be used for phrases indicating where an action is taking place (i.e., in
contrast with just indicating where something or someone is located, in which
case the accusative with \ird{na} is used). The verb \irdp{možlá}{to live, to
reside}, however, would always trigger the accusative case (e.g., \irdp{na Praha
možlaševí}{(I) live in Prague} and not \irdp{*Prahu možláševí}{(I) live in
Prague}).

\pex
\begingl
\gla Kras prědné nohóru ščenžáže.//
\glb train next-\Att{} half:hour-\Ins{} arrive-\Av{}-\Ctp{}-\Quot{}//
\glft \trsl{The train is arriving in half an hour.}//
\endgl
\xe

\section{Personal pronouns}\index{personal pronouns}\index{pronouns}

\subsection{Personal pronouns in general}

Personal pronouns are a special class of nouns used to refer and/or replace
other nouns or noun phrases. In traditional Iridian grammar, personal pronouns
are called \ird{svědé kaděc} or false nouns. We will follow this analysis and
treat personal pronouns not as a separate grammatical class but as a special
class of nouns since for the most part they are syntactically and
morphologically identical to nouns. Like other nouns, personal pronouns are
marked for person, number and case, and partially for animacy\index{animacy},
although third-person forms are more properly analyzed as demonstratives. In
this section we will only be discussing first and second person forms. Third
person forms are discussed in detail in \S~\ref{sec:demonstratives} with other
demonstratives.

Personal pronouns are declined in the same way as nouns although they are for
the most part more irregular than normal nouns. Table
\ref{tab:personal-pronouns} shows the declension of first- and second-person
personal pronouns in Iridian.

\begin{table}
    \footnotesize\sffamily
	\caption{Personal pronouns in Iridian}\label{tab:personal-pronouns}
	\medskip
	\begin{tblr}{width=0.8\textwidth,colspec={X[2.5]XXXX}}
		\toprule \addlinespace
        {\sc form}      & {\sc 1s}  & {\sc 2s} & {\sc 1pl} & {\sc 2pl}\\ \addlinespace
		\midrule \addlinespace
        Unmarked        & dá        & já      & mé      & tová  \\ \addlinespace
        Agentive        & dám       & jám     & mám     & tám   \\ \addlinespace
        Accusative      & dě        & jí      & mě      & tě    \\ \addlinespace
        Genitive        & že        & je      & mí      & teví  \\ \addlinespace
        Instrumental    & du        & jemu    & mejí    & tvě   \\ \addlinespace
        \bottomrule
	\end{tblr}
\end{table}

Unlike normal nouns, personal pronouns have explicit plural forms. This can be
traced back to their origins as demonstratives. In addition to indicating
number, the plural forms are also used to indicate politeness. This usage is
similar to the T-V distinction found in languages like French or German, but is
more general. The second person plural is used instead of the regular second
person singular forms even when the speaker is referring to a single person to
indicate respect or deference or merely as a way to distance oneself from the
listener, like when talking to a stranger or in situations where a higher level
of formality is required. The first person plural, on the other hand, may be
used in a similar fashion in formal contexts or when the speakers wishes to
communicate their humility. The choice of pronouns and the degree of formality
is determined by the context and the speaker's attitude towards themself and/or
the listener. This is discussed in more detail in \S~\ref{sec:politeness} and in
Chapter 8 in general.

Iridian is an extremely pro-drop language, with pronouns supplied only if not
immediately inferrable from context. In fact, a pronoun does not even have to be
supplied to establish context. Iridian, moreover, tends to favor avoidance not
simply as syntactic strategy but also as a part of its politeness system. Thus,
pronouns may be dropped not just because they are not necessary to establish
context but also because it might be considered impolite to use them in certain
contexts. A common alternative, for example, would be to address the listener
using their name or title or some other common noun as a \emph{quasi} honorific
(e.g., friend, comrade, etc.) either when the ambiguity in the referent would be
too great to simply drop the pronoun, or if, even when the context is clear, the
speaker just so wishes as a stylistic choice. Again, pronoun avoidance as a
politeness strategy is discussed further in \S~\ref{sec:politeness}.

The use of the possessive pronouns (i.e., personal pronouns in the genitive
case) is also very limited, with the possessive dropped in most cases where it
would be used in English. This latter behavior of Iridian would be familiar to
speaker of most Romance or Slavic languages. For example, the English sentence
\trsl{My knees hurt} would be translated in Iridian as \ird{Dliň prozíčime} and
not \ird{*Že dliň prozíčime}. Here \irdp{že}{my} is dropped unless the discourse
has hitherto included references to multiple knees and the speaker wishes to
specify that it is his knees that hurt. Even then, if the speaker wishes to
emphasize the ownership, it is more idiomatic to use what is called an `ethical
dative'\index{ethical dative} construction to indicate possession, viz.,
\ird{Dliň dě prozíčime} where the possessor is marked in the accusative. Compare
this with the Czech \foreign{Bolí mě kolena} `My knees hurt' or Spanish
\foreign{Me duelen las rodillas} `My knees hurt.'

There are no restrictions when it comes to the use of modifiers with personal
pronouns, unlike in, say, English where an adjective is normally not used with a
personal pronoun. For example, \irdp{zuštalé dá}{I who am happy} or more
literally, \trsl{(the) happy I} is a perfectly grammatical construction in
Iridian while the equivalent in English would mostly be reserved in literary
contexts, if at all, or more commonly rephrased with a relative clause \trsl{I,
who am happy, ...} or an apposition \trsl{I, the happy one, ...}.

\subsection{The Reflexive \ird{se}}\label{sec:reflexive-se}

Iridian has a special reflexive particle \ird{se}, which for the purposes of
this grammar we will consider as a personal pronoun. It is a Slavic borrowing,
possibly form Proto-Slavic \foreign{*sę} and is not attested in Old Iridian.
Table \ref{tab:se-declension} shows the declension of \ird{se}.

\begin{table}
    \footnotesize\sffamily
    \caption{Declension of the reflexive pronoun \ird{se}.}\label{tab:se-declension}
    \medskip
    \begin{tblr}{width=0.8\textwidth,colspec={X[1.2]X}}
        \toprule \addlinespace
        {\sc case}      & {\sc declension}\\ \addlinespace
        \midrule \addlinespace
        Unmarked        & se    \\ \addlinespace
        Agentive        & snám  \\ \addlinespace
        Patientive      & semě  \\ \addlinespace
        Genitive        & sní   \\ \addlinespace
        Instrumental    & sem   \\ \addlinespace
        \bottomrule
    \end{tblr}
\end{table}

The reflexive \ird{se} is used to refer back to the topic of the sentence. Se is
often used with the reflexive voice, although the use of \ird{se} often implies
a greater disjunction between the actor and the patient. Where the reflexive
voice has a primarily sociative meaning, as in verbs with a defunct active
voice, \ird{se} is used to form a true reflexive construction.

\pex    \a \irdp{Udúšek}{I took a bath.}
        \a \irdp{Se udúšek}{I bathed myself.}
\xe

\pex
        \a \irdp{Guláše upiašček}{I ate some goulash.}
        \a \ljudge{?} \irdp{Se upiašček}{I ate myself.}
\xe

The genitive and accusative forms of \ird{se} are also used as a proprietary
intensifier, similar to the usage of the English adjective \foreign{own} as in
\foreign{his own worst enemy}.

\pex
\begingl
\gla Bych shradice ko papka sní éhu vednik.//
\glb yesterday be:dead-\Pf{}-\Quot{} \Att{} father-\Dim{} \Refl{}-\Gen{} eye-\Ins{} see-\Pv{}-\Pf{}//
\glft \trsl{I saw your supposedly dead father with my own eyes yesterday.}//
\endgl
\xe

\subsection{Possessive nominals}\label{sec:possessive-nominals}

\section{Demonstratives}\label{sec:demonstratives}
\index{demonstratives}

Iridian does not have a separate class of third-person pronouns. Instead it uses
a set of demonstratives, whose deictic\index{deixis} function is both
spatial\index{spatial deixis|see{deixis}} and anaphoric\index{anaphora}. Iridian
makes a three-way distinction among demonstratives, similar to
French\index{French} or Portuguese\index{Portuguese} for example, distinguishing
between proximal (near the speaker), medial (near the addressee) and distal (far
from both speaker and addressee) forms. In addition, Iridian makes an animacy
distinction with demonstratives, with one set of demonstratives used with human
referents and another with non-human referents, as seen in Table
\ref{tab:dem-prons}. Demonstratives like personal pronouns are not marked for
gender, but unlike personal pronouns they do not have separate plural forms.

\begin{table}
    \footnotesize\sffamily
	\caption{Demonstrative pronouns in Iridian.}
    \medskip
    \begin{tblr}{width=0.7\textwidth,colspec={XXXX}}

		\toprule\addlinespace
						& {\sc animate}	& {\sc inanimate}	&{\sc locative}\\ \addlinespace
		\midrule \addlinespace
		Proximal		& ša		& to 				& tak\\ \addlinespace
		Medial			& kako		& jáne				& jení\\ \addlinespace
		Distal			& dní		& děn				& dně\\ \addlinespace
		\bottomrule
		\label{tab:dem-prons}
	\end{tblr}
\end{table}

Demonstratives can be used adnominally, to modify a noun phrase, or
pronominally, to replace one. In examples (b) and (c) below, for example, the
usage of \ird{ša} can be interpreted can be intrepreted as adnominal, modifying
\ird{že byl}, or pronominal, with the demonstrative as the topic and \ird{že
byl} as the predicate. Note that there are no differences, whether in the
orthography or the intonation, between the phrase \ird{ša že byl} and the
sentence \ird{Ša že byl}.

\pex
    \a
        \begingl
        \gla ša byl//
        \glb \Dem{}.\Prox{}.\Anim{} child//
        \glft \trsl{this child}//
        \endgl
    \a
        \begingl
        \gla ša že byl//
        \glb \Dem{}.\Prox{}.\Anim{} \First\Sg{}.\Gen{} child//
        \glft \trsl{this child of mine}//
        \endgl
    \a
        \begingl
        \gla Ša že byl.//
        \glb \Dem{}.\Prox{}.\Anim{} \First\Sg{}.\Gen{} child//
        \glft \trsl{This (person) is my child.}//
        \endgl
    \a
        \begingl
        \gla \ljudge{*}To že byl.//
        \glb \Dem{}.\Prox{}.\Inan{} \First\Sg{}.\Gen{} child//
        \glft \trsl{This (thing) is my child.}//
        \endgl
\xe

When used with other modifiers, demonstratives appear as the left-most element
of the phrase, but after the clitic \ird{ně}. Thus \irdp{děn tóm}{that book over
there} and \ird{ně děn mordé tóm}{those blue books over there} are valid, but
\ird{*děn mordé ně tóm} is not. Demonstratives cannot modify other
demonstratives; and so forms like \ird{ša dní}, \ird{ša to}, etc. are all
ungrammatical.\footnote{The idiomatic phrase \irdp{tak dně}{here and there} is
not ungrammatical because here \ird{tak} is not really modifying \ird{dně}.
Instead this is more correctly analyzed as the ellipsis of \irdp{a}{and} from
the original phrase \irdp{tak a dně}{here and there}.}

Table \ref{tab:dem-prons} also shows a third set of demonstratives, which are
used to indicate the general location of a referent. In their unmarked forms,
these locative demonstratives can only be used adverbially and not to modify or
replace nouns or noun phrases like animate and inanimate demonstratives.

When used as pronominally, demonstratives are declined in accordance to their role in the sentence. Like personal pronouns, their forms, too, are highly irregular.

\begin{table}
    \footnotesize\sffamily
        \caption{Declension of demonstratives.}
        \medskip
	    \begin{tblr}{width=\textwidth,colspec={X[2]XXXXXX}}
            \toprule
                            & {ša}	& {ón}	&{dní}& {to}	& {ján}	&{jón}\\
            \midrule \addlinespace
            Agentive&šem&nám&dněm&etom&ján&jón\\\addlinespace
            Patientive&šá&ona&dná&toha&jina&jinóva\\\addlinespace
            Genitive&ci&oní&dní&cie&ně&nohe\\\addlinespace
            Instrumental&svou&nu&dnu&etu&nu&nohu\\\addlinespace
            \bottomrule
            \label{dem-conj}
        \end{tblr}
    \end{table}

\pex
\a\deftagex{obv}
\begingl
\gla ci mlaz a dní mač//
\glft \trsl{this person's brother and that person's mother}//
\endgl
\a\deftaglabel{obv1}
\begingl
\gla Dá je svou je dnu zapreví.//
\glft \trsl{I am as old as either this person or that person.}//
\endgl
\xe

The three-way distinction between demonstratives allows Iridian to disambiguate between an obviative\index{obviation} third person and a proximate third person, using the distal and the proximal demonstrative respectively. Consider for example the two sentences in English below:

\pex
\a He saw his dog.
\a He saw his own dog.\smallskip
\xe

The \emph{his} in the first sentence is ambiguous, as it can refer to either the subject or an implied fourth person. That the second \emph{his} refers back to the subject can be made unequivocal by the addition of the word \emph{own}, as in the second sentence. Compare this with the following sentences in Czech:

\begin{multicols}{2}
  \pex
  \a
  \begingl
  \gla Viděl jeho pes.//
  \glft \trsl{He saw his dog.}//
  \endgl
  \a \begingl
  \gla Viděl své pes.//
  \glft \trsl{He saw his own dog.}//
  \endgl
  \xe
\end{multicols}

Although the English translation of the first sentence may still appear ambiguous, we can see that Czech does away with the ambiguity by using the third person pronoun \ird{jeho} exclusively to signify that the referent is different from the subject, and requiring the use of a separate pronominal form (in this case the reflexive) when the referent and the subject are the same. Iridian, on the other hand, treats this in a diametrically opposite way, i.e., the same pronoun form is used when the subject and the referent are the same, with the obviative form being used otherwise. The sentences in Czech above will therefore be translated in Iridian as follows:

\pex
\a
\begingl
\gla Dní jec vdinek.//
\glb \mk{dem.dist.anim.gen} dog see-\Pv{}-\Pf{}//
\glft \trsl{He saw his (other person's) dog.}//
\endgl
\a \begingl
\gla Ci jec vdinek//
\glb \mk{dem.dist.inan.gen} dog see-\Pv{}-\Pf{}//
\glft \trsl{He saw his own dog.}//
\endgl
\xe

Perhaps we can better understand the distinction between obviative and proximate forms by re-examining example (\getfullref{obv.obv1}) above. The previous examples in Czech remained unambiguous because there are at most two unique arguments in the sentence. In example (\getfullref{obv.obv1}), however, the subject of the sentence is distinct from either the proximate referent or the distal referent.

\ex[exno={\getfullref{obv.obv1}}]
\begingl
\gla Dá je svou je dnu zapreví.//
\glft \trsl{I am as old as either this person or that person.}//
\endgl
\xe

The translation in the gloss demonstrates how idiomatic English uses periphrastic forms to eliminate this ambiguity, although in the spoken language the purely deictic \trsl{I am as old as either him or him} is equally acceptable, with the blanks filled in most likely by non-verbal cues. In Iridian, however, this distinction is not optional, and the following sentence, for example, would be considered ungrammatical:

\ex
\begingl
\gla *Dá je svou je svou zapreví.//
\glft \trsl{I am as old as either him or him.}//
\endgl
\xe

\section{Interrogatives and derived forms}\label{sec:int-pron}
\index{wh-question@\emph{wh}-question}\index{interrogative pronoun}

\begin{table}
	\sffamily\footnotesize
    \label{tab:int-pron}
	\caption{Interrogatives in Iridian.}
    \medskip
	\begin{tblr}{width=0.8\textwidth,colspec={XXXX}}
		\toprule \addlinespace
		&{\sc english}&&{\sc english}\\ \addlinespace
		\midrule \addlinespace
		jede 		& who &jak &which\\ \addlinespace
		ježe 	& what 		& zajehu 	&why\\ \addlinespace
		jehát 	& whom		& jiká 	&how many\\ \addlinespace
		jehu 		& how		&jišká&how much\\  \addlinespace
		jemí 		& when 		& jeně 	&to where\\ \addlinespace
		jena 		& where 	& jení 	&from where\\ \addlinespace
		\bottomrule
	\end{tblr}
\end{table}

Interrogatives are a special class of words used to introduce content questions,
i.e., questions for which the speaker expects the listener to supply specific
information. Table \ref{tab:int-pron} shows the interrogatives in Iridian. When
used in questions, an interrogative is required to occupy the topic position in
a sentence. As such, while sentences like \trsl{What did you say?} and \trsl{You
said what?} are valid in English, only the former is grammatical in Iridian.
This phenomenon is called \emph{wh}-fronting and is discussed in detail in
\S\,\ref{sec:content-questions}.

Interrogatives may also be used as determiners before a noun phrase to indicate
that the expected answer is a member of the class described by the noun phrase.
When used in this manner, \irdp{jak}{which} is used when the preferred answer is
from a closed set of alternatives while the rest of the interrogatives
otherwise.

\pex
\begingl
\gla Jede ruščevní svirkošc to tóma svirček?//
\glb who russian-\Att{} author this book-\Acc{} write-\Av{}-\Pf{}//
\glft \trsl{What (Which) Russian author wrote this book?}//
\endgl
\xe

In the example above, we can replace \irdp{jede}{who} with \irdp{jak}{which} to
indicate that the answer is expected to be from a specific set of authors which
have been previously mentioned in the discourse. Another, more marked,
alternative to emphasize the restrictiveness of the expected answer would be to
change the noun phrase to a \ird{na}-phrase first before adding \ird{jak}.

\pex
\a\begingl
\gla Jak ruščevní svirkošc to tóma svirček?//
\glb which russian-\Att{} author this book-\Acc{} write-\Av{}-\Pf{}//
\glft \trsl{Which Russian author wrote this book?}//
\endgl
\a\begingl
\gla Jak na ně ruščevní svirkošta to tóma svirček?//
\glb which \Loc{} \Pl{}= russian-\Att{} author=\Acc{} this book-\Acc{} write-\Av{}-\Pf{}//
\glft \trsl{Which of these Russian authors wrote this book?}//
\endgl
\xe

Notice how in Iridian, when interrogatives are used as determiners before a noun
phrase, the choice is not limited between \ird{jede} and \ird{jak} as in
English. Instead the interrogative qua determiner follows the same role as the
noun phrase it modifies. Thus a sentence like \trsl{What year were you born in?}
is translated in Iridian as \ird{Jemí hletu nebrinek?}, i.e., literally as
\trsl{When year were you born?} with \ird{jemí} used instead of
\ird{jede}.\footnote{Compare this to \irdp{Jak hur?}{What time is it?} where
\ird{jak} is used instead of \ird{jemí} even though the noun \ird{hur} is
temporal, since the number of hours in a day restricts the possible answers to a
closed set.}

In non-question statements, an interrogative may also be used as a determiner
before a participant nominal (i.e., a nominalized verb). Since participant
nominals (v. \S\,\ref{sec:nomz-syntax}) are by nature definite, the use of an
interrogative as a determiner gives the resulting noun phrase an indefinite
meaning. The choice of the interrogative is determined by the voice of the
nominalized verb, and in the case of choosing between \ird{jede} and \ird{ježe},
its animacy as well. Compare, for example,
\irdp{svirčkou}{write-\Av{}-\Pf{}-\Nz{}}, i.e., `the person who wrote' and
\irdp{jede svirčkou}{whoever wrote (it)}.

An interrogative may also be modified by the particle \irdp{každý}{any, each,
even}, in which case the resulting phrase takes an indefinite meaning similar to
English \trsl{anyone,} \trsl{anything,} etc.

\pex
\begingl
\gla Každý jede vtaru do katedrála zahranéževí.//
\glb each who morning-\Ins{} into cathedral-\Acc{} enter-\Pot{}-\Av{}-\Prog{}//
\glft \trsl{Anyone can enter the cathedral in the morning.}//
\endgl
\xe

\section{Indefinite pronouns}\index{indefinite pronoun}\label{sec:indef-pron}

Indefinite pronouns are pronouns that refer to an unspecified referent. They are
of two types in Iridian: universal (like \irdp{nět}{everyone}) and negative
(like \irdp{zide}{no one}). They have historically been formed by attaching the
prefix \ird{že-} and \ird{ní-} to interrogative pronouns respectively. The
forms, as with interrogative pronouns, roughly correspond to the case paradigms
for regular nouns, with additional forms for locative, temporal and distributive
uses (i.e., forms corresponding to \trsl{where}, \trsl{when} and \trsl{which}
respectively). The forms are listed in Table~\ref{tab:indef-pron}.

\begin{table}
	\sffamily\footnotesize
	\caption{Negative and universal pronouns.}
    \label{tab:indef-pron}
    \medskip
	\begin{tblr}{width=0.8\textwidth,colspec={XX[1.2]XX[1.2]}}
		\toprule \addlinespace
		\SetCell[c=2]{}{\sc negative} & \SetCell[c=2]{}{@{}l}{\sc universal}\\ \addlinespace
		\midrule \addlinespace
		zide    & no one        & nět   & everyone \\ \addlinespace
		niho    & nothing       & niže  & everything \\ \addlinespace
		žehu    & by no means   & néhu  & by all means \\ \addlinespace
		žemě    & never         & nimě  & always \\ \addlinespace
		žena    & nowhere       & nina  & everywhere \\ \addlinespace
		žé      & not one       & nách  & each \\ \addlinespace
		\bottomrule
	\end{tblr}
\end{table}

\ird{Nět}, \ird{niže} and \ird{nách} as well as their negative counterparts
\ird{zide}, \ird{niho} and \ird{žé} can also be use adnominally, i.e., to
describe another noun phrase as quantifiers. \ird{Nět} and \ird{zide} are used
with animate, i.e., human, referents while \ird{niže} and \ird{niho} are used
with inanimate, i.e., non-human, referents. There is no such animacy
consideration with \ird{nách} and \ird{žé}. Thus one writes \irdp{nět
študent}{all students} and \irdp{niže jec}{all dogs} and not \ird{*niže študent}
or \ird{*nět jec}. However, both \irdp{nách študent}{each student} and
\irdp{nách jec}{each dog} are considered grammatical

Notice that in Iridian there are no indefinite pronouns that assert the
existence of its referent, similar to English \trsl{someone} or \trsl{something}
or \trsl{somebody}. Instead Iridian uses existential constructions (v.
\S\,\ref{sec:exst}) for the same purpose. For example, an English sentence like
\trsl{Someone is coming} is translated in Iridian as \ird{Semě ješ sčenžách}
(\Refl{}.\Acc{} \Exst{} arrive-\Av{}-\Ctp{}) where \ird{ješ} is the existential
particle. There are also no elective and dubitative existentials\footnote{From
Wikipedia: ``Elective existential pronouns are often used with negatives (I
can't see anyone), while dubitative existential pronouns are used in questions
when there is doubt as to the existence of the pronoun's assumed referent (Is
anybody here a doctor?).'' Both of these examples would use an existential
construction in Iridian.} like English \trsl{anyone} or \trsl{anybody.} Instead
their function is satisfied by the equivalent interrogative pronoun modified by
\ird{každý}, a Slavic borrowing meaning \trsl{any} in Iridian, or by existential
constructions. For more examples of the former, see \S\,\ref{sec:int-pron}.

\section{Names}\index{name}\label{sec:names}
 
\chapter{Minor word classes}\label{chap:minor}
\index{minor word classes}

In the preceding two chapters, we have looked at the three major word classes in Iridian: nouns, verbs, and modifiers. In this chapter we will look at the remaining word class, the function words, which we further divide into adverbial particles, conjunctions, prepositions, discourse markers, interjections, and numerals.

\section{Adverbial particles}\label{sec:adv-particles}

\subsection{In general}\label{sec:adv-particles-general}

Adverbial particles are a small class of words that are similar to one another
in exhibiting proclitic behavior. Notwithstanding certain predictable exceptions
discussed in this section, adverbial particles must obligatorily appear before
the predicate they modify. While we use the term `adverbial particle' to refer
to this class of words, their usage is rather varied and some may function as
discourse markers. A single particle may also be used in multiple ways, as we
will see in the following sections. The full list of adverbial particles is
given in Table~\ref{tab:adv-particles}.

\begin{table}
	\sffamily\scriptsize
	\caption[Adverbial particles]{Adverbial particles. The linguistic glosses, much like the translations, only provide approximations of the meanings of each particle and may not be fully equivalent to the actual meanings of the linguistic categories listed here.}\label{tab:adv-particles}
	\medskip
	\begin{tblr}{width=0.9\textwidth,colspec={X[0.5]X[1.4]X[0.4]X[1.2]}, rowsep=6pt}
		\toprule 
		{\sc particle} &
		{\sc approx. trans.} &
		{\sc gloss} &
		{}\\ 
		\midrule 

		že &
		\trsl{already} &
		\Pfv{} &
		Perfective particle \\ 

		po &
		\trsl{still, yet} &
		\Ipfv{} &
		Imperfective particle \\ 

		lí &
		\trsl{whether, if} &
		\Q{} &
		Question particle \\ 

		može &
		\trsl{also, too} &
		\Add{} &
		Additive particle \\ 

		što &
		\trsl{indeed, truly} &
		\Aff{} &
		Affirmative particle \\ 

		kamo &
		\trsl{apparently, according to} &
		\Rep{} &
		Reportative particle \\ 

		samo &
		\trsl{only, just} &
		\Excl{} &
		Exclusive particle \\ 

		pro &
		\trsl{on the other hand} &
		\Cntr{} &
		Contrastive particle \\ 

		neko &
		\trsl{before} &
		\Antess{} &
		Antessive particle \\ 

		iz &
		\trsl{apparently, may be} &
		\Dub{} &
		Dubitative particle \\ 

		oče &
		\trsl{contrary to my expectations} &
		\Mir{} &
		Mirative particle \\ 

		nadě &
		\trsl{as a result} &
		\Conseq{} &
		Consequential particle \\ 

		delí &
		\trsl{perhaps} &
		\Spec{} &
		Speculative particle\\

		\bottomrule

\end{tblr}
\end{table}

When two or more particles are proclitic to the same predicate, their relative
word order may be described in terms of the following hierarchy, in relation to
their distance from the predicate:
\begin{itemize}[nosep]
	\item Class 1: \ird{že} and \ird{po}
	\item Class 2: \ird{li}, \ird{može}, \ird{što}, \ird{kamo}, \ird{samo}, \ird{pro} and \ird{neko}
	\item Class 3: \ird{iz}, \ird{oče}, \ird{nadě} and \ird{delí}
\end{itemize}

\subsection{Class 1: \ird{že} and \ird{po}}
\label{sec:class1-particles}

The two Class 1 adverbial particles \ird{že} and \ird{po} never occur in
immediate sequence to each other. In general, \ird{že} and \ird{po} carry
aspect-related meanings, with \ird{že} used to indicate the perfective and
\ird{po} the imperfective aspect. This usage, however, does not completely
correspond to the true aspectual suffixes on a verb, as we have seen in
\S~\ref{sec:aspect}. \ird{Že} and \ird{po} can be broadly translated as
\trsl{already} and \trsl{still/yet}, respectively, but their usage as we will
see below is more complex.

In sentences containing a temporal clause expressing a future event, \ird{že}
and \ird{po} are used to indicate the attitude of the speaker towards the time
described in the predicate. \ird{Že} `extends' the perceived time between the
reference point and the time described by the predicate, while \ird{po}
`shortens' it. Thus a neutral sentence such as \irdp{Janek sobotu
ščenžách}{Janek will arrive on Saturday} can be modified as \ird{Janek sobotu že
ščenžách} or as \ird{Janek sobotu po ščenžách.} The former indicates that the
speaker thinks that there is little time left before Janek's arrival on
Saturday, while the latter indicates that the speaker thinks that there is still
a lot of time left before Janek's arrival on Saturday. In contrast to these two,
the original sentence without \ird{že} or \ird{po} does not pass any judgment on
the time left before Janek's arrival on Saturday. With temporal clauses
expressing past events, \ird{po} behaves the same way in `extending' the
perceived time between the reference point and the time described by the
predicate; \ird{že} on the other hand cannot be used in this way. Thus
\irdp{Janek sobotu ščenžek}{Janek arrived on Saturday} can be modified as
\ird{Janek sobotu po ščenžek} which can be interpreted as \trsl{Janek arrived on
Saturday (and it has been quite some time since then).} \ird{Janek sobotu že
ščenžek} is also a valid sentence, but here \ird{že} merely translates as
\trsl{already} and does not have the aspectual connotation it has in the future
tense.

The usage of \ird{že} and \ird{po} described in the previous paragraph is
limited to sentences that satisfy two criteria: (1) the sentence must contain an
explicit temporal clause that specifies the point in time when the action
described in the predicate will take place or has taken place, and (2) the
action or the state must be in the future or past, which means if the predicate
is a verb it must be in the perfective, retrospective or contemplative aspect.
These criteria are required since the length of time upon which the speaker
passes judgment can only be established by first defining a reference point
(i.e., the time of speaking) and another point in time that will serve as the
beginning (in the case of past events) or the end (in the case of future events)
of the time interval. In sentences that do not satisfy these criteria, \ird{že}
and \ird{po} will have different meanings.

Without an explicit temporal clause, \ird{že} is used with a verb in the
perfective aspect to indicate that the action has been completed at some
unspecified time prior to another time. This usage roughly corresponds to the
English present perfect. Compare for example the two sentences below:

\pex
\a
\begingl
	\gla Janek ščenžek.//
	\glb Janek arrive-\Av{}-\Pf{}//
	\glft \trsl{Janek arrived.}//
\endgl
\a
\begingl
	\gla Janek že ščenžek.//
	\glb Janek \Pfv{} arrive-\Av{}-\Pf{}//
	\glft \trsl{Janek has (already) arrived (at some unspecified time in the past).}//
\endgl
\xe

\ird{Že} can alternatively be used with a verb in the retrospective aspect to
indicate that the action has been completed at some unspecified time prior to
another event, as with the perfective, but this usage requires that the
secondary verb that frames the action described by the main verb also be present
in the sentence. This usage can correspond to the English past perfect or the
future perfect depending on the aspect of the verb in the preceding clause
(i.e., before \ird{še}). E.g.,

\pex
\a
\begingl
	\gla Marek ščenžek še Janek že piaščaní.//
	\glb Marek arrive-\Av{}-\Pf{} with Janek \Pfv{} eat-\Av{}-\Ret{}//
	\glft \trsl{Janek had already eaten when Marek arrived.}//
\endgl
\a
\begingl
	\gla Marek ščenžách še Janek že piaščaní.//
	\glb Marek arrive-\Av{}-\Ctp{} with Janek \Pfv{} eat-\Av{}-\Ret{}//
	\glft \trsl{Janek will have already eaten when Marek arrives.}//
\endgl
\xe

When used with the perfective, \ird{po} indicates that an action has been done
together with other actions, with the implication that the speaker regrets or is
annoyed by having to perform the action. Compare this to the use of \trsl{even}
in English:

\pex
\begingl
	\gla Do magazina bych što po štožek.//
	\glb into store-\Acc{} yesterday \Aff{} \Ipfv{} go-\Av{}-\Pf{}//
	\glft \trsl{I even went to the store yesterday.}//
\endgl
\xe

\ird{Po} may also be used to express the meaning \trsl{in addition.} In
questions, \ird{po} corresponds to the English \trsl{else}:

\pex
\begingl
	\gla Jede Marcí dumu po ščenžek?//
	\glb who Marek-\Gen{} house-\Ins{} \Ipfv{} arrive-\Av{}-\Pf{}?//
	\glft \trsl{Who else arrived at Marek's house?}//
\endgl
\xe

\subsection{Class 2 particles}\label{sec:class2-particles}

\paragraph{može} \ird{Može} roughly corresponds to the English \trsl{also} or
\trsl{too.} It is commonly used to express similarity. If the main verb is in
the negative, \ird{može} can be translated as \trsl{either} or \trsl{neither.}
For example:

\pex
\begingl
	\gla Janek može uzdravževí.//
	\glb Janek \Add{} sleep-\Av{}-\Cont{}//
	\glft \trsl{Janek is also sleeping.}//
\endgl
\xe

\pex
\begingl
	\gla Janek može záščenžek.//
	\glb Janek \Add{} \Neg{}-arrive-\Av{}-\Pf{}//
	\glft \trsl{Janek hasn't arrived either.}//
\endgl
\xe

\ird{Može} can also be used idiomatically to mean \trsl{finally} or \trsl{at last.} For example:

\pex
\begingl
	\gla Já može že vednik!//
	\glb \Second{}.\Sg{} \Add{} \Ipfv{} see-\Pv{}-\Pf{}//
	\glft \trsl{I've seen you at last!}//
\endgl
\xe


\paragraph{što} \ird{Što} is used to express emphasis, affirmation or confirmation. For example:

\pex
\begingl
	\gla Janek što že ščenžek.//
	\glb Janek \Aff{} already arrive-\Av{}-\Pf{}//
	\glft \trsl{Janek has indeed arrived (contrary to what I thought).}//
\endgl
\xe

\paragraph{samo} \ird{Samo} generally corresponds to the English \trsl{only.}
This particle is most likely of Slavic origin, as evidenced by cognates like
\foreign{samo} in Czech (meaning \trsl{alone}) or in Croatian (meaning
\trsl{only}) or \foreign{sam} in Polish.


\subsection{Class 3 particles}\label{sec:class3-particles}

\paragraph{iz} \ird{Iz} is used to express doubt or uncertainty. For example:

\pex
\begingl
	\gla Janek iz že ščenžek.//
	\glb Janek \Dub{} \Pfv{} arrive-\Av{}-\Pf{}//
	\glft \trsl{Janek may have already arrived.}//
\endgl
\xe

\subsection{Order of particles}\label{sec:order-of-particles}

As noted in \S~\ref{sec:adv-particles-general} above, the order of adverbial
particles is generally determined by their class, with Class 1 particles
appearing nearest to the predicate and Class 3 particles appearing farthest.
Thus the order of various particles in a sentence like example
(\ref{ex:multiple-particles-diff-class}) where each particle belongs to a
separate class is trivial:

\pex\label{ex:multiple-particles-diff-class}
\begingl
	\gla Janek oče može po uzdravževí.//
	\glb Janek \Mir{} \Add{} \Ipfv{} \Refl{}-sleep-\Av{}-\Cont{}//
	\glft \trsl{To my surprise, Janek is also still asleep.}//
\endgl
\xe

In the above example, the Class 1 particle \ird{po} appears nearest to the
predicate, followed by the Class 2 particle \ird{može}, and finally the Class 3
particle \ird{oče}. The order of particles in Iridian is strict and thus
sentences like the following are not grammatical:

\pex
\a \ljudge{*} \ird{Janek može oče po uzdravževí.}
\a \ljudge{*} \ird{Janek može po oče uzdravževí.}
\a \ljudge{*} \ird{Janek po može oče uzdravževí.}
\a \ljudge{*} \ird{Janek po oče može uzdravževí.}
\a \ljudge{*} \ird{Janek oče po može uzdravževí.}
\xe

The order of particles in sentences where two or more particles belong to the
same class is more complicated. The Class 1 particles \ird{že} and \ird{po}
cannot appear simultaneously in the same sentence, as they carry opposite,
temporal meanings. Thus the position occupied by Class 1 particles can only be
filled by one and only one particle or by none at all. Class 2 particles, on the
other hand, can appear simultaneously in the same sentence; in general, the
order they appear follows the formula below:

\pex
	{pro + neko + lí + kamo + samo + može + što}
\xe

Finally, Class 3 particles appear farthest from the predicate; they may occur in
any order relative to one another. For example, sentences like
(\ref{ex:oce-iz-gramm}) and (\ref{ex:iz-oce-gramm}) are grammatical, but
(\ref{ex:oce-iz-ungramm}) is not. There is no difference in meaning between
(\ref{ex:oce-iz-gramm}) and (\ref{ex:iz-oce-gramm}).

\pex
\a\label{ex:oce-iz-gramm}
\begingl
	\gla Janek oče iz že ščenžek.//
	\glb Janek \Mir{} \Dub{} \Pfv{} arrive-\Av{}-\Pf{}//
	\glft \trsl{Janek may have already arrived apparently.}//
\endgl
\a\label{ex:iz-oce-gramm}
\begingl
	\gla Janek iz oče že ščenžek.//
	\glb Janek \Dub{} \Mir{} \Pfv{} arrive-\Av{}-\Pf{}//
	\glft \trsl{Janek may have already arrived apparently.}//
\endgl
\a\label{ex:oce-iz-ungramm}
\begingl
	\gla \ljudge{*}Janek že oče iz ščenžek.//
	\glb Janek \Pfv{} \Mir{} \Dub{} arrive-\Av{}-\Pf{}//
	\glft \trsl{Janek may have already arrived apparently.}//
\endgl
\xe


\section{Conjunctions}\label{sec:conj}

\subsection{Connective conjunctions}\label{sec:conn-conj}

Sentences of the type

\ex
It is [\mk{adjective}] that[ \mk{subordinate clause}].
\xe

are normally translated in Iridian using an expletive-\ird{a} construction, with the adjective in the attributive form at the start of the phrase, followed by \ird{a}, and then by the rest of the main clause. Normally this construction is used for sentences that pass judgment to the action or state described in the main clause, although in some cases the adjective is simply used for descrciption.

\pex
\begingl
    \gla Interezní a téknik znohouštnilá te prádelnik.//
    \glb interesting-\Att{} and engineering study\mk{-pv-sbj.ipf} \mk{rz} choose-\Pv{}-\Pf{}//
    \glft \trsl{It is interesting that you chose to study engineering.}//
\endgl
\xe
\pex
\begingl
    \gla Komí a já ščenžek.//
    \glb good-\Att{} and \Second{}\Sg{} arrive-\Av{}-\Pf{}//
    \glft \trsl{Good you're here now!}//
\endgl
\xe

Another common use of the expletive \ird{a} is with the word \irdp{shlac}{now} (pronounced [sxlat] instead of the more intuitive [sxlat͡s]) to form the phrase \ird{shlac a}\footnote{This is therefore pronounced [ˈsxlatɐ].}, which is used to introduce a subordinate clause, similar to \trsl{now that} in English.

\pex
\begingl
    \gla Shlac a provísor ščenžek, kurs šelčinách.//
    \glb now and professor arrive-\Av{}-\Pf{} class begin-\mk{pv-ctpv}//
    \glft \trsl{Now that the professor is here, we will begin our class.}//
\endgl
\xe


\section{Prepositions}

\subsection{na}

Iridian has a single locative preposition, \ird{na}, which is used to indicate the location of an object or person. It is used in the same way as the English \trsl{on} or \trsl{in}. \ird{Na} is followed by a noun or a noun phrase in the accusative case.

\pex
\a \ird{na duma}, \trsl{at home}
\a \ird{na škole}, \trsl{at school}
\a \ird{na vele}, \trsl{in the countryside}
\xe

Where English uses specific prepositions such as \trsl{above}, \trsl{under}, \trsl{below}, etc., Iridian uses a compound construction with \ird{na} and another noun indicating the location marked in the accusative. The `object' in the equivalent English construction is marked in the genitive in Iridian.

\pex
\a \ird{na dumí veha}, \trsl{in front of the house}
\a \ird{na bamení pouda}, \trsl{behind the building}
\xe

\subsection{še}

\subsection{\ird{vo}}\index{vo}\index{agentive case}

\ird{Vo} can be translated as \trsl{because of} or \trsl{due to.} This preposition takes the agentive case.

\pex
\begingl
\gla Vo transitám lienu zásčenžek.//
\glb because traffic-\Agt{} on:time-\Ins{} \Neg{}-arrive-\Av{}-\Pf{}//
\glft \trsl{I didn't arrive on time because of the traffic.}//
\endgl
\xe

\subsection{za}

\section{Quantifiers}\index{quantifiers}
Iridian has a wide variety of non-numerical/indefinite quantifiers.  Most are actually nouns that used in adjectival or adverbial constructions.


\begin{itemize}
    \item \ird{ošč} \trsl{many} (countable)
    \ex
    \begingl
    \gla Marka ješ naže ošč.//
    \glb Marek-\Acc{} \Exst{} friend-\Gen{} many//
    \glft \trsl{Marek has many friends.}//
    \endgl
    \xe
    \ex
    \begingl
    \gla Za kursa mén ješ ošč oudinášce ko vilm.//
    \glb for class-\Acc{} \mk{1pl.inc.wk} \Exst{} many watch-\Sup{} \Att{} film.//
    \glft \trsl{We have a lot of movies we need to watch for our class.}//
    \endgl
    \xe
    \item \ird{nave} \trsl{too many} (countable)
    \ex
    \begingl
    \gla Marka ješ naže ošš.//
    \glb Marek-\Acc{} \Exst{} friend-\Gen{} many//
    \glft \trsl{Marek has many friends.}//
    \endgl
    \xe
    \item \ird{tohle} \trsl{many} (uncountable)
    \item \ird{nahte} \trsl{too many, too much} (uncountable)
    \ex
    \begingl
    \gla Do ješ nahte kurváš//
    \glb \First{}\Sg{}.\Acc{} \Exst{} too:much work-\SupN{}//
    \glft \trsl{I have so much work to do.}//
    \endgl
    \xe

\end{itemize}

\section{Interjections}

An interjection\index{interjection} is a word or an expression used to express a spontaneous reaction or feeling. We will use the term `interjection' to refer both to the part of speech and to the utterance type that has the same pragmatic function as this part of speech (cf. \cite{ameka1992}).

Interjections can be classifed into two main categories: \emph{primary} interjections, which refer to a word or an utterance that can only be used as an interjection and \emph{secondary} interjections, which refer to forms belonging a different word class but which through its usage, has acquired a new meaning as an interjection.

Although interjections can function as exclamations, not all exclamatory utterances can be considered as interjectons by themselves. As \textcite{jovanovic2004} notes, any word in a language can theoretically become an exclamation. Consider for example this conversation:

\ex (adapted from \cite{jovanovic2004}).\\

  \ird{
  \noindent--- Martin mlaza boulešik.\\
  --- \textbf{Martinám?}
  }\medskip

  \trsl{I heard Martin killed his brother.}\\
  \trsl{Martin?!}
\xe


\section{Discourse particles}

\subsection{Yes and no}
Iridian has several words for yes and no but their usage in responding to yes-no questions does not exactly align with that of English. This is discussed in detail in \S\,\ref{sec:ansyn}.

There are two main words for \trsl{yes} in Iridian: the affirmative \ird{dé} (\trsl{Did you see it?} \trsl{Yes, I did.}) and the contrastive \ird{če} (\trsl{Did you not see it?} \trsl{Yes, I did.}. The distinction is similar as that between the French \emph{oui} and \emph{si}. Both \ird{dé} and \ird{če} generally appear at the end of a sentence. In colloquial spoken Iridian it is also common to see the form \ird{ja} (most likely from the Czech, and ultimately from the German \emph{ja}) and the more informal \ird{jó}. These forms however are not cliticized to the verb and appear at the start of a sentence, set off from the rest with a commma. Both \ird{ja} and \ird{jó} cannot be used contrastively like \ird{če}. It is also common to use both \ird{ja/jó} at the same time as \ird{dé}.

\pex
\begingl
\gla ---To vdinice? ---Ja vdinek dé.//
\glb this see-\Pv{}-\Pf{}-\Quot{} yes see-\Pv{}-\Pf{} yes//
\glft \trsl{{}``Did you see it?'' ``Yes, I did.''{}}//
\endgl
\xe

When used by themselves, both \ird{ja} and \ird{jó} are often repeated twice or thrice (e.g., \ird{Ja ja ja.})\footnote{Commas are not used to separate each \ird{ja} or \ird{jó} in standard orthography. } even when the usage is not emphatic. \ird{Dé} and \ird{če} cannot be used this way.

\section{Numerals}\label{sec:numerals}

Iridian has a vigesimal number system. Table \ref{tab:nums-one-twenty} shows
Iridian numerals from 1 to 20. Numbers from 1 to 10 are given their own name
while numbers from 11 to 19 are formed by appending the numbers from one to nine
to the clitic \ird{-něm} with the preposition \ird{še} (with). The clitic
\ird{-něm} is derived from the word for number 10, \ird{nau}, which itself
comes from the Old Iridian \rec{nagu}, `half.'

	
\begin{table}
\footnotesize\sffamily
\caption{Iridian numerals from 1 to 20.}
\medskip
\begin{tblr}{width=0.8\textwidth,colspec={X[0.7]XX[0.7]X}}
	\toprule\addlinespace
	{\sc number} & {\sc iridian} & {\sc number} & {\sc iridian}\\ \addlinespace
	\midrule \addlinespace
	1 & ona			& 11 & onšeněm\\ \addlinespace
	2 & vuc			& 12 & myšeněm\\ \addlinespace
	3 & hrona		& 13 & hronašeněm\\ \addlinespace
	4 & drou		& 14 & drušeněm\\ \addlinespace
	5 & jed			& 15 & jecněm\\ \addlinespace
	6 &	dve			& 16 & vušeněm\\ \addlinespace
	7 & šče			& 17 & ščiceněm\\ \addlinespace
	8 & pieš		& 18 & pisčeněm\\ \addlinespace
	9 & cam			& 19 & camšeněm\\ \addlinespace
	10& nou			& 20 & týdna\\ \addlinespace
	\bottomrule
	\label{tab:nums-one-twenty}
\end{tblr}
\end{table}

Numbers from 21 to 99 are first expressed as multiples of 20. Thenceforth, the
number system has largely become decimal, due primarily to the influence of
surrounding Indo-European languages. Old Iridian, however, had a vigesimal
system up to the number 8000.

Table \ref{tab:nums-thirty-one-hundred} shows multiples of 10 from 30 to 100.
The numbers are formed by the numeral followed by \ird{týdna}. For bases that
are not multiples of 20, the word \irdp{nau}{ten} is added first, followed by
the conjunction \irdp{še}{with}.

\begin{table}
	\footnotesize\sffamily
	\caption{Iridian numerals from 30 to 100.}
	\medskip
	\begin{tblr}{width=0.8\textwidth,colspec={X[0.5]XX[0.5]X}}
		\toprule \addlinespace
		{\sc number} & {\sc iridian} & {\sc number} & {\sc iridian}\\ \addlinespace
		\midrule \addlinespace
		30 &	naušetýdna		& 70 	& naušehronutýdna\\ \addlinespace
		40 &	vutýdna			& 80	& drohutýdna\\ \addlinespace
		50 &	nauševutýdna	& 90	& naušedrohutýdna\\ \addlinespace
		60 &	hronutýdna		& 100	& měs\\ \addlinespace
		\bottomrule
		\label{tab:nums-thirty-one-hundred}
	\end{tblr}
\end{table}

Iridian counting starts from the smallest component of the number to the
largest. Each component can be simply appended with the conjunction \ird{a}. The
examples below illustrate the formation of more complex numbers. Table
\ref{tab:nums-two-hundred-one-trillion} shows Iridian numerals from 200 to one
billion.

\pex
\a \irdp{jed a měs}{five and hundred} i.e., 105
\a \irdp{cam a drohutýdna}{nine and four twenties} i.e., 89
\a \irdp{pisčeněm a hronutýdna}{eighteen and three twenties} i.e., 78
\xe

\begin{table}
	\footnotesize\sffamily
	\caption{Iridian numerals from 200 to one trillion.}
	\medskip
	\begin{tblr}{width=0.9\textwidth,colspec={X[0.6]X}}

		\toprule \addlinespace
		{\sc number} & {\sc iridian} \\ \addlinespace
		\midrule \addlinespace
		200 			&	mach	\\ \addlinespace
		300, 400, etc.	& 	hronuměs, drohuměs. etc.\\ \addlinespace
		1000			& 	něk\\ \addlinespace
		2000, 3000, etc.& 	vuněk, hronuněk, etc.\\ \addlinespace
		10.000			&	ohle\\ \addlinespace
		20.000, etc.	& 	t\'ydnuněk, etc.\\ \addlinespace
		100.000			&	hazlek\\ \addlinespace
		200.000 etc		&	mehdeněk, hronuněk, etc.\\ \addlinespace
		1.000.000		&	miliám\\ \addlinespace
		1.000.000.000	&	milár\\ \addlinespace
		1.000.000.000.000	& biliám\\ \addlinespace
		\bottomrule
		\label{tab:nums-two-hundred-one-trillion}
	\end{tblr}
\end{table}

When used attributively, numerals do not require the particle \ird{ty} to be
linked to a noun or a noun phrase. The numerals \ird{týdna}, \ird{měs},
\ird{mach}, \ird{něk}, \ird{ohle}, \ird{hazlek}, \ird{miliám}, \ird{milár} and
\ird{biliám} may reduplicated, separated by \irdp{a}{and} to indicate an
estimate (e.g., \irdp{měs a měs cel}{hundreds of people}). This is similar to
the pluralization of numerals in English for the same purpose, as in, for
example, \trsl{tens} or \trsl{hundreds}. 

Numerals may also be used nominally, in which case they are inflected like
regular nouns.

\subsection{Ordinal numbers}
\label{sec:ordinals}

Except for the first three cardinal numbers that have irregular ordinal forms,
ordinals are mostly regular, formed with the suffix \ird{-šle} (or \ird{-išle}
after consonants). The ordinal form of the numbers one, two and three are
\ird{hezka}, \ird{dvěc} and \ird{cehra}, respectively. The ordinal form is
\ird{měs} is also irregular, being \ird{měšle} and not \ird{*měšišle}. When
written as numerals, a full stop is used followed by a dash (e.g.,
\irdp{camišle}{ninth} would be written 9.-). In compound numbers, only the last
component is inflected with \ird{-šle}; `eighty seventh' for example would be
\ird{šče a drohutýdnišle} and not \ird{*ščišle a drohutýdna}.

The letter n has its own ordinal form (cf. English \trsl{nth} for example),
\ird{enišle}, as do the rest of the other letters. These ordinal forms are
generally regular. Their usage is confined to mathematical literature, however,
with the clear exception of \ird{enišle}, which is often used idiomatically (cf.
French \textit{pour la enième fois}).


\subsection{Fractions, decimals and other derivative forms}
\label{sec:fractions}

As with most languages in Europe, Iridian uses the comma (Iridian \ird{kva}) to
separate whole number from decimals (see \S~\ref{sec:punctuation}). Numbers
after the comma are read in pairs of two, with the first number read separately
in case there is an odd number of numerals after the comma (e.g., 3,34 is read
as \ird{hrona kva drušeněm a týdna} while 3,347 is read \ird{hrona kva hrona šče
a vutýdna}). Generally, if there are seven or more numbers following the come,
each is read separately instead, though this is not a hard and fast rule and the
speaker may read the numbers separately even if there are fewer than seven
decimal places.

Fractional forms are regularly formed using the suffix \ird{-izmek}. The word
for half, \ird{num}, however is irregular. Fractional forms are sometimes used
together with the regular decimal forms when dealing with currency. For example,
5,50 kr. can be read as either \ird{jed kva nauševutýdna korun} or more commonly
\ird{jed a num korun} (cf. English \trsl{five and a half dollars}).

\subsection{Measurements}\label{sec:measurements}

Iridian uses the metric system for most measurements. The cardinal form of the
number is used, followed by the name of the unit. If the unit appears
independently, it is unmarked; if however, the measurement is used
attributively, the unit is marked in the instrumental case: thus one writes
\irdp{hrona měter}{three meters} and \irdp{hrona mětru kuz}{three meters of
silk.} Some commonly used units include \irdp{měter}{meter} for length,
\irdp{gram}{gram} for weight, \irdp{lěter}{liter} for volume, or generic units
like \irdp{procent}{per cent,} \irdp{par}{pair,} \irdp{tuzyn}{dozen} and
\irdp{týdna}{score}.\footnote{The word for \trsl{score} and \trsl{twenty} are
identical, \ird{týdna}. When used as a numeral attributively \ird{týdna} is
indeclinable. When used as a unit, it declines like a regular noun.} SI prefixes
like \ird{kilo-}, \ird{cénti-}, etc. are also commonly used.

\subsection{Date and time}\label{sec:date-time}

Dates are written with the year first, followed by the month, and ultimately by
the date. When written in numerals, the numbers are separated by a full stop.
When spoken or when written in full, the number representing the year is
followed by the word \irdp{hlet}{year}, often in the instrumental case. When
followed by the name of the month, \ird{hlet} is declined in the genitive. When
the date is included, the ordinal form is used, followed by the word
\irdp{ráz}{day,} although the latter may be dropped in casual speech. The
inclusion of the date also requires the name of the month to be in the genitive
case.

\pex
\a
\begingl
    \gla 1992 hletí julí 15. rázu veštašik //
    \glb 1992 year\mk{-gen} july-\Gen{} 15th day-\Ins{} be:born-\Av{}-\Pf{}//
    \glft \trsl{I was born on 5 July 1992.}//
\endgl
\xe

\begin{table}
	\footnotesize\sffamily
	\caption{Months of the year.}
	\medskip
	\begin{tblr}{width=0.7\textwidth,colspec={XXXX}}
		\toprule
		{\sc month} & {\sc iridian} & {\sc month} & {\sc iridian}\\
		\midrule
		January		& jenvár	& July & jul\\
		February	& fevrár 	& August & augošt\\
		March		& merc		& September & seitembár\\
		April		& april 	& October & oktobár\\
		May 		& mai 		& November & novembár\\
		June 		& jón 		& December & dicámbár\\
		\bottomrule
		\label{tab:months}
	\end{tblr}
\end{table}

			
\chapter{Derivational morphology}\index{word}\index{word formation}\index{derivational morphology}

\section{Introduction}

In \S\,\ref{sec:wordclasses} we discussed how Iridian words can be classified into two broad groups: content words\index{content word} and function words\index{function word}. Due to their very nature, function words are largely invariable in form; content words, on the other hand, vary constantly and their form reflect the grammatical information they carry. We call this system of variation {\sc inflection},\index{inflection} and it is one of the ways languages like Iridian form new words from pre-existing ones.\footnote{By ``new" here we mean a form different from the original word; but since inflection is primarily a grammatical operation, the difference in meaning occasioned by inflection is often not significant.}

In this chapter we will discuss two more ways to form new words in Iridian: {\sc derivation}\index{derivation} and {\sc compounding}\index{compound word} (cf. \cite{booij2005}; \cite{velupillai2012}: 115). Compounding involves the amalgamation of multiple words to form a new word; this is discussed in detail in section \S\,\ref{sec:compounding}. Derivation, on the other hand, involves modifying a word with affixes (in a similar way to inflection) to change its meaning. Unlike inflectional affixes, however, derivational affixes do not carry any grammatical information

\section{Nominal derivation}
\subsection{Diminutives and augmentatives}\label{sec:diminutive}
\index{diminutive}\index{augmentative}

Unlike English,\index{English} but similar to most Slavic\index{Slavic languages} and Romance languages\index{Romance languages}, Iridian frequently employs {\sc diminutives} (and to a lesser degree {\sc augmentatives}). The most basic form of the diminutive is formed with the suffix \ird{-ka} (or \ird{-cka} after vowels), which most linguists agree is a non-native morpheme\index{borrowing}, and is most likely borrowed from Slavic.

\ex
\irdp{jec}{dog} $\rightarrow$ \irdp{jecka}{doggy, little dog}\\
\irdp{papír}{paper} $\rightarrow$  \irdp{papírka}{piece of paper}\\
\irdp{dum}{house} $\rightarrow$  \irdp{dumka}{little house}\\
\irdp{kávé}{coffee} $\rightarrow$  \irdp{kávécka}{espresso}
\xe

Diminutives are used to express that something is small or insignificant. In the spoken language\index{spoken Iridian}, however, it is more common to use the diminutive to express endearment or affectation\index{affect}. This same usage makes it possible to use the diminutive patronizingly, to belittle or to be dismissive.\index{pejorative} With mass nouns, the diminutive is also often used to refer to a small quantity of something.

\pex
\a To express affection:\\
\begingl
\gla Jecka do vezdalnik.//
\glb dog-\mk{dim} \mk{1s.pat} to:gift-\Pv{}-\Pf{}//
\glft \trsl{This dog was given to me as a gift.}//
\endgl

\a To dismiss or belittle:\\
\begingl
\gla To na provízorká niho zábor.//
\glb this \Loc{} professor-\mk{dim-pat} \mk{nexst} knowledge//
\glft \trsl{This so-called ``professor'' doesn't know a thing.}//
\endgl

\a To express a small quantity of something:
\xe

When referring to members of one's own family\index{kinship terms}, that of a friend's, or of the person being addressed, the diminutive form is also used. Most kinship terms have irregular forms and are listed in \S\,\ref{sec:nuclear family}. In colloquial Iridian\index{colloquial Iridian} proper names are also often marked as diminutives, with the variant suffix \ird{-ik/-k} being more common. The first-person plural clitic \irdp{-óm}{our} is often used in conjunction with the diminutive. In addition to this, most names also have irregular diminutive forms and variants which are discussed in detail in \S\,\ref{sec:names}.

\ex
\ird{Janek} $\rightarrow$ \ird{Jančik}, \ird{Jančikóm}\\
\ird{Marek} $\rightarrow$ \ird{Marčik}, \ird{Marčikóm}\\
\ird{Tomáš} $\rightarrow$ \ird{Tomášik}, \ird{Tomáškóm}\\
\ird{Tereza} $\rightarrow$ \ird{Terežik}, \ird{Terežkóm}\\
\ird{Agáta} $\rightarrow$ \ird{Agáčik}, \ird{Agáčkóm}
\xe

Double and triple diminutives are also common, formed using \ird{-(i)ška} and \ird{-(i)sička}, respectively. Quadruple and quintuple diminutives are also possible (formed using \ird{-(i)nisička} \ird{-(i)nižesička}, respectively), although their usage is not as neutral, and would often be used to mock or to exaggerate.\footnote{The suffixes \ird{-(i)ška} and \ird{-(i)sička} are of Slavic\index{Slavic languages} origin while \ird{-(i)nisička} \ird{-(i)nižesička} are Iridian innovations.}

Augmentatives\index{augmentative} are also used, although their usage is not as common as diminutives and their usage is often limited as pejoratives\index{pejorative}. Augmentatives are formed with the suffixes \ird{-(ž)ulám} or \ird{-(ž)urnám} or \ird{-(ž)uláhmaš}. These forms are not interchangeable and in general the longer the augmentative suffix is, the more pejorative is its connotation.

\subsection{Nouns from nouns}

The suffix \ird{-(e)vnice} is used in deriving nouns from proper nouns. When used with names of places it generally has the meaning \trsl{resident of} or \trsl{native of}. Countries whose name end in the suffix \ird{-óma} drop the suffix first before adding \ird{-(e)vnice}. The variant \ird{-evnik} has the same meaning as \ird{-evnice} but can only be used derogatorily.

\begin{multicols}{2}
  \ex
  \irdp{ircevnice}{Iridian}\\
  \irdp{mažarevnice}{Hungarian}\\
  \irdp{čiževnice}{Czech}\\
  \irdp{polščevnice}{Polish}\\
  \irdp{mušhouvnice}{Muscovite}\\
  \irdp{néviorčevnice}{New Yorker}\\
  \irdp{turčevnice}{Turk}\\
  \irdp{ruževnice}{Russian}\\
  \irdp{američevnice}{American}\\
  \irdp{anglevnice}{English}
  \xe
\end{multicols}


The suffix \ird{-(h)ár} from the Czech \emph{-ár/-á\v{r}} indicates agency. It is often used to form nouns relating to professions, although it may appear with Latinate loanwords as the assimilated form of the French \emph{-aire}.

\ex
\irdp{revolucehár}{revolutionary} fr. \irdp{revoluce}{revolution}\\
\irdp{milionár}{millionaire} fr. \irdp{milion}{million}\\
\irdp{travár}{baker} fr. \irdp{trava}{bread}\\
\irdp{kostlár}{fisherman} fr. \irdp{kostel}{fish}\\
\irdp{známehár}{smith} fr. \irdp{známe}{metal}\\
\irdp{zakár}{sailor} fr. \irdp{zak}{sea}\\
\irdp{bašketbólár}{basketball player} fr. \irdp{bašketból}{basketball}\\
\irdp{míštár}{warrior} fr. \irdp{miešt}{war}\\
\irdp{ákcehár}{shareholder} fr. \irdp{ákce}{share of stock}\\
\irdp{nepodár}{bureaucrat} fr. \irdp{nepod}{position, rank}
\xe

Variants of \ird{-(h)ár} include \ird{-(h)er} and \ird{-(h)or}, although their usage is much more limited.

\ex
\irdp{senátor}{senator} fr. \irdp{senát}{senate}\\
\irdp{aviátor}{aviator} fr. \irdp{aviace}{aviation}\\
\irdp{helder}{salaryman} fr. \irdp{held}{wage, salary}, itself from German \emph{Geld}
\xe

Another common suffix used to form agent nouns is \ird{-ist}. This suffix is often used on nouns ending in \ird{-ižmus}.

\ex
\irdp{komunist}{communist} fr. \irdp{komunižmus}{communism}\\
\irdp{modernist}{modernist} fr. \irdp{modernižmus}{modernism}\\
\irdp{avtist}{cabdriver} fr. \irdp{avt}{car}\\
\irdp{mašinist}{engineer} fr. \irdp{mašina}{machine, engine}\\
\irdp{bankist}{banker} fr. \irdp{bank}{bank}\\
\irdp{žurnálist}{journalist} fr. \irdp{žurnál}{magazine}
\xe

The most common way of forming abstract nouns is through the suffix \ird{-(i)žnást}.

\ex
\irdp{vidližnást}{slavery} fr. \irdp{videl}{slave}\\
\irdp{tiehožnást}{divinity, holiness} fr. \irdp{tieho}{god}\\
\irdp{\v{c}eližnást}{membership} fr. \irdp{\v{c}elina}{member}\\
\irdp{stultižnást}{puberty} fr. \irdp{stólet}{teenager}
\xe

The suffix \ird{-(i)mašt} forms a place or location associated to a noun.

\ex
\irdp{piaštoumašt}{dining room, pantry} fr. \irdp{piaštou}{food}\\
\irdp{traumašt}{bakery} fr. \irdp{trava}{bread}\\
\irdp{jakomašt}{woods} fr. \irdp{jako}{tree}\\
\irdp{jelcimašt}{jungle} fr. \irdp{jelec}{forest}\\
\irdp{dílmašt}{nursery} fr. \irdp{diel}{infant}\\
\irdp{dohzámašt}{paradise} fr. \irdp{doház}{bliss}\\
\xe

\subsection{Nouns from verbs and adjectives}\label{sec:nomder-verb}

The suffix \ird{-o\v{s}c} (pronounced as if written \ird{o\v{s}t}) is used to form a noun that represents an actor or agent (usually a person) that performs the action denoted by the verb, often habitually, or less commonly, someone or something usually associated with the action described by the verb. The habitual nature of the action may be emphasized by using \ird{-ivo\v{s}c} (which incorporates the aspectual marker \ird{-iv-}) instead of \ird{-o\v{s}c}. 

\ex
\irdp{umilo\v{s}c}{heavy drinker, drunkard} fr. \irdp{um\v{e}l\'a}{to get drunk}\\
\irdp{umilivo\v{s}c}{drunkard} fr. \irdp{um\v{e}l\'a}{to get drunk}\\
\irdp{jelzo\v{s}c}{traveller} fr. \irdp{jelz\'a}{to travel, wander}\\
\irdp{hnervo\v{s}c}{doctor, physician} fr. \irdp{hnerv\'a}{to heal}\\
\xe

Agent/actor nouns can also be formed using \ird{-amite} or \ird{-amnite}. These pair contrasts with \ird{-o\v{s}c} and \ird{-ivo\v{s}c} since they describe actions that are seen to affect someone or something else directly.

\ex
\irdp{nahradamite}{cook, chef} fr. \irdp{nahrad\'a}{to cook}\\
\irdp{pardamite}{guard} fr. \irdp{pard\'a}{to guard, to watch}\\
\irdp{z\v{e}kamite}{spokesperson} fr. \irdp{z\v{e}k\'a}{to say}
\xe

\section{Verbal Derivation}

\section{Compounding}\index{compounding}\index{compound word}\label{sec:compounding}

\section{Linguistic Borrowing}\index{loanword}
A significant portion of the vocabulary of Iridian comes from loanwords from neighbouring languages, especially German\index{German}, Czech\index{Czech} and Polish\index{Polish}, and to a lesser extent Hungarian\index{Hungarian}. Like most languages from the area, Iridian also has a notable portion of its vocabulary derived from French\index{French} and Latin, mostly scientific and academic terms. In addition, after the advent of the internet, there has been an increasing amount derived from English and other world languages as well. Most loanwords are assimilated to conform with Iridian phonological rules, although most recent loanwords generally maintain the phonology of the language they were originally borrowed from.

In most cases, the loanwords or their assimilated forms coexist with their native Iridian counterparts. Often their usage is interchangeable

\subsection{From German and other Germanic languages}\index{German}\index{German loanwords}

Like its neighboring Czech Republic and Slovakia, Iridia has had significant contact with the German-speaking peoples of Central Europe throughout the centuries, leading to a significant German influence on the language's vocabulary. Most of the words of German origin now in Iridian entered the language in the 16th century when the Duchy of Iridia (then a part of the Crown of Bohemia) was absorbed into the Habsburg Monarchy, with the influence continuing into the late 19th century. Starting the 1880s\footnote{Some sources point to the defeat of Austria and the Peace of Prague in 1866 as the beginning of the `de-Germanization' of Iridia. Nevertheless it was not until the Edict of Julmonc (then Olm\"utz) was issued in March 1882 that the de-Germanization of the Iridian language was formalized by Iridian state authorities.} however (in large part due to the spread of Romanticism and nationalism in the region), and until the collapse of the Austro-Hungarian Empire, attempts have been made to `de-Germanize' Iridian vocabulary by replacing German vocabulary with words from the native stock or more often with calques\index{calque}. This `de-Germanization' continued well into the first half of the 20th century, as a result of which, German loanwords in Iridian in constant use have significantly decreased from what they have been in the 16th to the 18th centuries, with most words of Germanic origin now considered archaic and are used primarily as an affectation (cf. English \emph{thou}, \emph{shew} and \emph{methinks}, for example).

Assimilation\index{assimilation of loanwords} of German phonemes that do not exist in Iridian is generally consistent, and is subject to the rules discussed in this section.

German\index{German} has three falling diphthongs (\cite{wiese1996}): /aɪ̯/, /aʊ̯/ and /ɔʏ̯/, none of which have exact equivalents in Iridian. Nonetheless /aʊ̯/ assimilates to Iridian /au̯/ (both spelled $\langle$au$\rangle$). /aɪ̯/ does occur marginally in Iridian, but most instances of /aɪ̯/ in German become either /\"aː/ or /eɪ̯/.\footnote{Or just [\"aː] and [eː] given the monophthongization of /eɪ̯/ in most dialects.} Finally /ɔʏ̯/ is never assimilated to the marginal /ɔɪ̯/ but becomes either /eɪ̯/ or /au̯/.


\ex
Assimilation of German diphthongs:\\
\irdp{Karlštám}{Charles Castle} fr. \emph{Karlstein}\\
\irdp{Bérna}{Bayern} fr. \emph{Bayern}\\
\irdp{bedautum}{significance, importance} fr. \emph{Bedeutung}\\
\irdp{Freid}{Freud} fr. \emph{Freud}
\xe

The raised vowels $\langle$\"a$\rangle$ and $\langle$\"o$\rangle$ become /eː/ $\langle$é$\rangle$ (or sometimes /ɪ/ $\langle$i$\rangle$) in Iridian while $\langle$\"u$\rangle$ become non-palatalising /ɪ/, spelled $\langle$y$\rangle$.


\subsection{From Common Slavic and its descendants}

\subsection{From Latin and Greek}

\subsection{From French, English and other European languages}

\subsection{From other languages} 
\chapter{Clause structure}\label{chap:clause-structure}

Thus far, we have focused our discussion on the behavior of individual words,
examining their functions, forms, and meanings. In this chapter, we will shift
our focus to how these words interact with each other to form sentences and
communicate meaning. Syntax is the set of rules that govern the structure and
arrangement of words in a language, and it plays a crucial role in ensuring the
effectivity of communication. In this chapter, we will examine the basic
syntactic structure of Iridian, then explore the different clause linking
strategies that makes it possible to form more complex sentences, and finally go
over special construction types such as questions, existential constructions,
copular constructions, etc.

\section{Basic clause structure}\label{sec:basic-clause-structure}

The two most salient features of Iridian syntax are the following: (1) SOV word
order and (2) strong head-finality. The constituent word order is
subject-object-verb (SOV); there is little variation in how the constituents of
a sentence are ordered, although the language might allow non-topic NP
constituents to be scrambled for stylistic or other reasons. Iridian is strongly
head-final, meaning that the head of a phrase always appears at the end of the
phrase. This includes all modifiers, including adverbials and determiners, and
non-main clauses, such as relative or subordinate clauses.

While we have been talking about an SOV word order, it would be more accurate to
analyze Iridian sentences to divided primarily into a topic part and a predicate
or comment part. The topic is what the sentence is about, while the predicate or
comment represents the information presented in the sentence about the topic.
While both the topic and the predicate are pragmatic constructs, the
topic-predicate construction is important as it determines how the rest of the
sentence is structured. In general, the topic of the sentence is the constituent
that is most prominent in the sentence and consequently appears first. The topic
of the sentence is also the constituent that determines the case marking of the
main verb and the other constituents of the sentence. The predicate, on the
other hand, provides supplementary information about the topic of the sentence.

\begin{figure}
  \begin{forest}
    [S,
      [{\sc top}] [{\sc pred}]]
  \end{forest}
  \caption{Nuclear structure of sentences}
  \label{fig:topic-pred}
\end{figure}

The topic, despite its prominence, can be considered non-essential, as most of
the information is encoded in the predicate. In fact, the primacy of the topic
and its influence on how the predicate is structured means that for the most
part, where the topic has already been established earlier, only the predicate
is needed to create a well-formed sentence. 

In most topic-prominent languages, the topic does not necessarily coincide with
the subject. For example, the following sentences are grammatical in Korean:

\pex
\a
\begingl
\gla seylinun paykhyellul coahanta.//
\glb Seri-\Top{} Baekhyun-\Obj{} like.//
\glft \trsl{As for Seri, she likes Baekhyun.}//
\endgl
\a
\begingl
\gla seylika paykhyennun coahanta.//
\glb Seri-\Subj{} Baekhyun-\Top{} like.//
\glft \trsl{As for Baekhyun, Seri likes him.}//
\endgl
\xe

In Iridian, however, the topic and the subject are the same. By promoting a noun
phrase as the topic, it automatically becomes the subject of the sentence, as
the voice marking in the main verb is determined by the thematic role of the topic in the sentence.

The topic function is not linked to a particular grammatical function.
Nevertheless, while it is theoretically possible to promote any constituent noun
phrase of a sentence to the topic position, in practice, the topic is usually a
noun phrase that is specific and referential. In most cases, this means that the
topic is a proper noun or definite noun phrase. As \textcite{kiss2004} observes,
\begin{quote}
  We tend to describe events from a human perspective, as statements about their
  human participants – and subjects are more often {\sc[+human]} than objects
  are. In the case of verbs with a {\sc[–human]} subject and a {\sc[+human]}
  accusative or oblique complement,  the  most  common  permutation  is  that
  in  which  the  accusative  or oblique complement occupies the topic
  position\,[.] When the possessor is the only human involved in an action or
  state, the possessor is usually topicalized[.]
\end{quote}

\section{Internal clause structure}\label{sec:internal-clause-structure}

\section{Topicalization}\label{sec:topicalization}

\section{Clause linking strategies}\label{sec:clause-linking}

To communicate more complex ideas, it may sometimes be necessary to link two or
more sentences together in a single sentence or in a sequence of sentences. In
English this is commonly done by using conjunctions such as `and,' `but,' `or,'
etc. or relative pronouns like `that' or `which.' The choice of which connector
to use depends on the type of relationship between the two clauses. We will
consider three broad types of clause linking strategies: 

\pex
\a Relative clause:\\
  {The man [\emph{whom} we saw at the store yesterday] is my brother.}
\a Complementation:\\
  {He asked me [\emph{whether} I had seen the movie.]}
\a Conjunction or clause-linking:\\
  {[\emph{Although} we went to Dobroslau, we did not see him.]}
\xe

We may observe that the sentence constituents enclosed in square brackets can be
freely removed from the above examples without making the sentences
ungrammatical. We will call this remnant part the {\sc main clause} and the
sentence constituent enclosed in square brackets the {\sc secondary clause}. In
general, the verb in the secondary clause will be marked in the conjunctive form
(see \ref{sec:conjunctive-form} in Iridian. For example, the above sentences
would be translated as follows:

\pex
\a
\begingl
  \gla Magazinu vednice ty maša mlazka.//
  \glb store-\Ins{} see-\Pv{}-\Pf{}-\Cnj{} \Rel{} man brother-\Dim{}//
  \glft \trsl{The man whom we saw at the store yesterday is my brother.}//
\endgl
\a
\begingl
  \gla To film vednice by hloupšek.//
  \glb this movie see-\Pv{}-\Pf{}-\Cnj{} \Quot{} ask-\Av{}-\Pf{}//
  \glft \trsl{He asked me whether I had seen the movie.}//
\endgl
\a
\begingl
  \gla Do Dobroslava stožicemá ša závednik.//
  \glb into Dobroslau-\Acc{} go-\Av{}-\Pf{}-\Cnj{}:although \Third{}\Sg{} \Neg{}-see-\Pv{}-\Pf{}//
  \glft \trsl{Although we went to Dobroslau, we did not see him.}//
\endgl
\xe

The secondary clause is terminated by a {\sc conjunctive ending}
\index{conjunctive ending} which indicates the clause's role in the sentence.
With the exception of \ird{ty} (used with relative clauses) and \ird{by} (used
with quotative constructions), the conjunctive ending are fused with the verb in
the secondary clause and do not appear as separate morphemes, as in English. In
general, the main clause follows the secondary clause(s) in a sentence, unless
the secondary clause is expressed as an afterthought.

\pex
\a canonical order: [secondary clause] + [conjunctive ending] + [main clause]
\a afterthought: [main clause] + [secondary clause]
\xe

Iridian has a large number of conjunctive 

\subsection{Temporal succession}
\label{sec:temporal-succession}

The conjunctive ending \irdp{-ní}{and then} is used to link two sequential
clauses; the secondary clause, which occurs first, indicates the first occuring
event while the main clause indicates the second event.

\pex
\begingl
  \gla Marku houčicení tětar zaby stojounek.//
  \glb Mark-\Ins{} meet-\Av{}-\Pf{}-\Cnj{}-and.then theater together
        go-\Lv{}-\Pf{}//
  \glft \trsl{I met Marek and then we went to the theater together.}//
\endgl
\xe

If the secondary clause(s) and the main clause share the same subject, the chain
is usually interpreted as describing a succession of events, as is the case in
the above examples. However, if the secondary clause(s) and the main clause have
different subjects, the chain may also take a `causal' interpretation, i.e., the
action in the secondary clause causes the action in the main clause.

\pex
\begingl
  \gla Mobil Jankam prodnicení mač všihnaševí.//
  \glb mobile.phone Mark-\Agt{} lose-\Pv{}-\Pf{}-\Cnj{}-and.then mother be.angry-\Av{}-\Cont{}//
  \glft \trsl{Mark lost his mobile phone and then/and so his mother got angry.}//
\endgl
\xe

The ending \irdp{-š}{and} has a similar meaning but offers a vaguer
description of the temporal relationship between the two clauses. Consider the
following sentences:

\pex
\a\label{ex:sz-perf}
\begingl
  \gla Marek do Praha stožiceš pislo Jankám prějenik.//
  \glb Marek into Prague-\Acc{} go-\Av{}-\Pf{}-\Cnj{}-and letter Janek-\Agt{}
  send-\Pv{}-\Pf{}//
  \glft \trsl{Marek went to Prague and Janek sent (him) a letter.}//
\endgl
\a\label{ex:sz-prog}
\begingl
  \gla Marek znousčiměš Janek uzdravžime.//
  \glb Marek study-\Av{}-\Prog{}-\Cnj{}-and Janek sleep-\Av{}-\Prog{}//
  \glft \trsl{Marek is studying and Janek is sleeping.}//
\endgl
\xe

Example (\ref{ex:sz-perf}) can be interpreted as either (1) \trsl{Marek went to
Prague \emph{and then} Janek sent him a letter} or \trsl{Marek went to Prague
\emph{at the same time that} Janek sent him a letter.} We can also consider a
third option, where the speaker is neutral as to the temporal relationship
between the two events because it is either unknown or irrelevant to the
conversation. In example (\ref{ex:sz-prog}), on the other hand, while the
clauses are still linked by \ird{-š}, most speakers would likely interpret
this as \trsl{Marek is studying \emph{and at the same time} Janek is sleeping}
as the progressive aspect would make it unlikely that the speaker meant for the
two events to be sequential. Nevertheless, the neutral interpretation
\trsl{Marek is studying \emph{and} Janek is sleeping} is still possible.

If a sentence contains multiple secondary clauses, the order of the clauses
correspond to the order of the events. Only \ird{-š} and not \ird{-ní} can be
used to chain these clauses.

\pex
\a\begingl
  \gla \ljudge{*}Marku houčicení tětar zaby stojounicení zaby prižek.//
  \glb Mark-\Ins{} meet-\Av{}-\Pf{}-\Cnj{}-and.then theater together
        go-\Lv{}-\Pf{}-\Cnj{}-and.then together have.dinner-\Av{}-\Pf{}//
  \glft \trsl{I met Marek and then we went to the theater together and had dinner.}//
\endgl
\a\begingl
\gla Marku houčiceš tětar zaby stojouniceš zaby prižek.//
\glb Mark-\Ins{} meet-\Av{}-\Pf{}-\Cnj{}-and theater together
      go-\Lv{}-\Pf{}-\Cnj{}-and together have.dinner-\Av{}-\Pf{}//
\glft \trsl{I met Marek and then we went to the theater together and had dinner.}//
\endgl
\xe

In brief, the ending \ird{-ní} presumes some level of relationship (whether
causal or not) between the linked clauses and thus cannot be used with patently
divergent or unrelated clauses. The ending \ird{-š} is more neutral and can be
used whether or not the clauses are related temporally or causally.

Other conjunctive endings are available to express more specific temporal
relationships such as \irdp{-mazy}{while}, \irdp{-zak}{until},
\irdp{-škady}{around the time when}, \irdp{-škany}{since}, \irdp{-šhoume}{as
soon as}, \irdp{-šbym}{after}, \irdp{-šdny}{before}, etc. Most of these endings
developed from postpositions, e.g., \ird{-šdny} from \irdp{dnou}{in front of}.

\pex
\a
\begingl
  \gla Zkuzy kadem anuncirnicešdny Janek kourneví.//
  \glb exam-\Gen{} result announce-\Pv{}-\Pf{}-\Cnj{}-before Janek worried-\Cont{}//
  \glft \trsl{Janek was worried before the exam results were announced.}//
\endgl
\a
\begingl
  \gla Zkuzy kadem anuncirnicešhoume Janek vysleví.//
  \glb exam-\Gen{} result announce-\Pv{}-\Pf{}-\Cnj{}-as.soon.as Janek delighted-\Cont{}//
  \glft \trsl{Janek was delighted as soon as the exam results were announced.}//
\endgl
\a
\begingl
  \gla Zkuzy kadem anuncirnicešbym Janek zuštaleví.//
  \glb exam-\Gen{} result announce-\Pv{}-\Pf{}-\Cnj{}-after Janek happy-\Cont{}//
  \glft \trsl{Janek was happy after the exam results were announced.}//
\endgl
\trailingcitation{(adapted from \cite{sohn2009korean})}
\xe

\subsection{Contrasting clauses}
\label{sec:contrast}

Contrast between two clauses is usually expressed by the ending \ird{-má}
usually translated in English as \trsl{although} or \trsl{but}.

\pex
\begingl
  \gla Marek do Praha stožicemá Janek závednaní.//
  \glb Marek into Prague-\Acc{} go-\Av{}-\Pf{}-\Cnj{}-but Janek \Neg{}-see-\Pv{}-\Ret{}//
  \glft \trsl{Although Marek went to Prague, he didn't meet Janek.}//
\endgl
\xe

\subsection{Causality}
\label{sec:causality}

As discussed in \S\,\ref{sec:temporal-succession}, the conjunctive ending
\ird{-ní} often takes a causal reading when the main clause and the secondary
clause have different subjects. Nevertheless, causation can be expressed more
explicitly by the use of \ird{-vlí}.

\pex
\a\begingl
  \gla Zabole zákupébicevlí byl kravnašime.//
  \glb ice:cream-\Acc{} \Neg{}-buy-\Ben{}-\Pf{}-\Cnj{}-because child cry-\Av{}-\Prog{}//
  \glft \trsl{The child is crying because (they) did not buy him ice cream.}//
\endgl
\a\begingl
  \gla Zabole zákupébicení byl kravnašime.//
  \glb ice:cream-\Acc{} \Neg{}-buy-\Ben{}-\Pf{}-\Cnj{}-and.then child cry-\Av{}-\Prog{}//
  \glft \trsl{(They) did not buy the child ice cream and so he is crying.}//
\endgl
\xe

Only \ird{-vlí} can be used to mark the cause or reason used as the basis of an
inference or judgment marked with the inferential particles \ird{izdy} or
\ird{hlavdy}. The verb in the main clause must be in the conditional mood if the
speaker is uncertain about the truth of the reason provided in the main clause,
as in (\ref{ex:causality-inference-uncertain}) below.

\pex
\a\label{ex:causality-inference-uncertain}
\begingl
  \gla Zabole zákupébilevlí(*-ní) byl izdy kravnašime.//
  \glb ice:cream-\Acc{} \Neg{}-buy-\Ben{}-\Cond.\Pf{}-\Cnj{}-because(*-and.then) child \Spec{} cry-\Av{}-\Prog{}//
  \glft \trsl{The child must be crying because (they) did not buy him ice cream.}//
\endgl
\a
\begingl
  \gla Marek do Budapešta stožicevlí magazin izdy zaromenik.//
  \glb Marek into Budapest-\Acc{} go-\Av{}-\Pf{}-\Cnj{}-because store \Spec{} close-\Pv{}-\Pf{}//
  \glft \trsl{Since Marek went to Budapest, his shop must be closed.}//
\endgl
\xe



\subsection{Conditional clauses}
\label{sec:conditional-clauses}
\index{conditional clause}

A conditional sentence is a statement of the form \trsl{If X, then Y.} Here X is
called the protasis or the condition and Y is called the apodosis or the result.
Both verb forms must be marked in the conditional. The protasis corresponds to
the secondary clause and the apodosis to the main clause and thus the former is
additionally marked in the conjunctive form and is commonly terminated by
\irdp{-my}{if} or its negative counterpart \irdp{-zmy}{if not}, or by
\irdp{-bymy}{if} or its negative counterpart \irdp{-byž}{if not}. The ending
\ird{-bymy} and \ird{byž} presuppose that the event described in the protasis
\emph{will} happen, but the exact timing of which is yet uncertain. \ird{-my}
and \ird{-zmy} on the other hand merely state a possibility, i.e., it is
uncertain whether or not the event described in the protasis will happen at all.

\pex\label{ex:conditional-clauses}
	\a
	\begingl
		\gla Piaščejímy, dá može piaščy.//
		\glb eat-\Av{}-\Cond{}.\Ipf{}-\Cnj{}-if I also eat-\Av{}-\Cond{}.\Ipf{}//
		\glft \trsl{If you eat, then I will also eat.}//
	\endgl
	\a
	\begingl
		\gla Nebo 100 centihradu nekraznejíbymy, ustrožy.//
		\glb water 100 celcius-\Ins{} \Caus{}-heat-\Pv{}-\Cond{}.\Ipf{}-\Cnj{}-if \Refl{}-boil-\Av{}-\Cond{}.\Ipf{}//
		\glft \trsl{If you heat the water to 100 degrees Celsius, then it will boil.}//
	\endgl
\xe

In example (\ref{ex:conditional-clauses}b), what is being described
is merely the logical consequence of the protasis happening, i.e., the water
will boil if it is heated to 100 degrees Celsius. In English, this usage of \ird{-bymy/byž} would often be translated as \trsl{when}.

When using \ird{-bymy} or \ird{-byž} for habitual states or actions,
\irdp{ozle}{often}, \irdp{než}{sometimes}, \irdp{pouze}{rarely},
\irdp{žemě}{never} and \irdp{nimě}{always} may be used to indicate the frequency
of the event described in the main clause.

\pex
	\a
	\begingl
		\gla Dá na duma zmy, dá na gnaža.//
		\glb I \Loc{} house if.not I \Loc{} school//
		\glft \trsl{If I'm not at home, then I'm at school.}//
	\endgl
	\a
	\begingl
		\gla Dá na duma byž, dá ozle na gnaža.//
		\glb I \Loc{} house often I \Loc{} school//
		\glft \trsl{When I'm not at home, I'm often at school.}//
	\endgl
\xe

Counterfactuality is expressed by adding the adverbial particle \ird{mlada} to
the protasis or to both clauses. The use of \ird{mlada} is only compatible with
\ird{-my} or \ird{-zmy}. To further emphasize the counterfactual nature of the
sentence, the adverbial particle \irdp{sám}{only} may be used in addition to
\ird{mlada} in the secondary clause.
\pex
	\a
	\begingl
		\gla To bych mlada podatnilemy, prěnžil.//
		\glb this yesterday \Hyp{} submit-\Pv{}-\Cond{}.\Pf{}-\Cnj{}-if pass-\Av{}-\Cond{}.\Pf{}//
		\glft \trsl{If I had submitted this yesterday, I would have passed.}//
	\endgl
	\a
	\begingl
		\gla Dá nesté duhu do Vietnama mlada stožilezmy, Marek vednil.//
		\glb I last-\Att{} month-\Ins{} into Vietnam-\Acc{} \Hyp{} go-\Av{}-\Cond{}.\Pf{}-\Cnj{}-if.not Marek see-\Pv{}-\Cond{}.\Pf{}//
		\glft \trsl{If I hadn't gone to Vietnam last month, I would have seen Marek.}//
	\endgl
  \a
  \begingl
    \gla Dá mlada sám stožilemy, ježe děne po vedny?//
    \glb I \Hyp{} \Excl{} go-\Av{}-\Cond{}.\Pf{}-\Cnj{}-if what \Spec{} \Ipfv{} see-\Pv{}-\Cond{}.\Ipf{}//
    \glft \trsl{I wonder what else I would have seen if only I had gone?}//
  \endgl
  \xe

Concessive clauses are considered a special case of conditional clauses. They
are usually translated in English as \trsl{Unless X, Y}. There are three main
conjunctive endings for consessive clauses in Iridian: \irdp{-kou}{unless},
\irdp{-kuzmy}{as long as} and \irdp{-kazy}{even if}. Some examples are given
below.

\pex
\a
\begingl
  \gla Marek sobotu mlada stožilekazy, opera zaby závednil.//
  \glb Marek saturday-\Ins{} \Hyp{} go-\Av{}-\Cond{}.\Pf{}-\Cnj{}-even.if opera together \Neg{}-see-\Pv{}-\Cond{}.\Pf{}//
  \glft \trsl{Even if Marek had come last Saturday, we wouldn't have been able to go to the opera together anyway.}//
\endgl
\a
\begingl
  \gla To prova Jankám vlastnejíkou, zákabežy.// 
  \glb this exam Janek-\Agt{} pass-\Pv{}-\Cond{}.\Ipf{}-\Cnj{}-unless \Neg{}-pass-\Av{}-\Cond{}.\Ipf{}//
  \glft \trsl{Unless Janek passes this exam, he won't be able to graduate.}//
\endgl
\xe

\subsection{Summary}
\label{sec:conjunctive-summary}

Table \ref{tab:conjunctive-endings} shows a summary of the most common
conjunctive endings in Iridian.

\begin{table}
  \sffamily\footnotesize
  \caption{List of conjunctive endings in Iridian}
  \label{tab:conjunctive-endings}
  \begin{tblr}{width=\textwidth, colspec={X[0.5]XX}}
      \toprule
      {\sc ending} & {\sc usage} & {\sc translation} \\
      \midrule
      -ní & temporal succession & \trsl{and then, and so}\\
  \end{tblr}
\end{table}



\section{Coordination}
\label{sec:coordination}
\index{coordination}

Iridian has three groups of coordinating conjunctions: the additive
\irdp{a}{and} and \irdp{še}{with}; the contrastive \ird{má} and \ird{ozná} (both
translated to \printlang{en}\index{English} as \trsl{but}); and the
disjunctive/correlative \ird{je}, \ird{le} and \ird{ni}.

\ird{A} corresponds to the English \trsl{and.} When coordinating simple noun
pairs, however, \irdp{še} is more often used though. The derived construction
\ird{a še} is also common and has a similar meaning to the English \trsl{and
also}.

\pex
\begingl
    \gla Mámka {še} pápku na Prahe spaníček.//
    \glb mother-\Dim{} \Com{} father-\Dim{}-\Ins{} \Loc{} Prague-\Acc{} vacation-\Av{}-\Pf{}//
    \glft \trsl{Mom and Dad went to Prague for vacation.}//
\endgl
\xe
\pex
\begingl
    \gla Janek {a} {še} Marku kurs hlupinžice.//
    \glb Janek and \Com{} Marek-\Ins{} class fail-\Av{}-\Pf{}-\Quot{}//
    \glft \trsl{Janek as well as Marek failed the class.}//
\endgl
\xe

In constructions with \ird{še} where one of the nouns coordinated is a pronoun
or a deictic\index{deictic}, the pronoun or deictic is presented first followed
by the other noun in the instrumental case\index{isnstrumental case}.

\pex
\begingl
    \gla Dá {še} Ivanu sohladoušce.// \glb \First{}\Sg{} \Com{} Ivan-\Ins{}
    classmate// \glft \trsl{Ivan and I are classmates.}//
\endgl
\xe

In a few cases, \ird{a} is used instead of \ird{še} where the latter can be
interpreted as having an attributive meaning. Where the noun is marked, however,
only \ird{a} can be used.

\begin{multicols}{2}
\pex\a
\begingl
    \gla trava {še} lépu//
    \glb bread \Com{} cheese-\Ins{}//
    \glft \trsl{bread with cheese} i.e., \trsl{cheese sandwich}//
\endgl
\a
\begingl
    \gla trava {a} lép//
    \glb bread and cheese//
    \glft \trsl{bread and cheese}//
\endgl
\xe\end{multicols}

\pex
\begingl
    \gla To kurs-te Jankám {a} Markám hlupienince.//
    \glb this class-\Foc{} Janek-\Agt{} and Marek-\Agt{} class fail-\Pv{}-\Pf{}-\Quot{}//
    \glft \trsl{It was this class that Marek and Janek failed.}//
\endgl
\xe


The bisyndetic coordination (\cite{velupillai2012}) \ird{a} Y \ird{a} Y is also
with similar emphatic meaning as \ird{a še}.

\pex
\begingl
    \gla {a} plocem {a} ploceš.//
    \glb and family-\First{}\Sg{} and family-\Second{}\Sg{}//
    \glft \trsl{both my family and yours}//
\endgl
\xe

\pex
\begingl
    \gla {a} hastu {a} še zmenu zověc hloubižách.//
    \glb and suffering-\mk{} and \Com{} happiness-\Ins{} remain-\Cv{} love-\mk{av-ctpv}//
    \glft \trsl{Til death do us part.} \emph{Lit.,} \trsl{I will love you through both suffering and joy.}//
\endgl
\xe

With multiple nouns or noun phrases, especially in serial lists, the
coordinating conjunction is often simply dropped.

\pex
\begingl
    \gla Ivan, Jarek, Elena na meza.//
    \glb Ivan Jarek Elena \Loc{} room-\Acc{}//
    \glft \trsl{Ivan, Jarek, and Elena are in the room.}//
\endgl
\xe

\pex
\begingl
    \gla Morkve, hlepost, ruk, molec hladniževí.//
    \glb carrot asparagus broccoli cabbage to:displease-\Av{}-\Cont{}//
    \glft \trsl{I don't like carrots, asparagus, broccoli or cabbage.}//
\endgl
\xe

\ird{A} or \ird{še} however is required when two adjectives are used to modify a
noun, with \ird{še} used when the two adjectives describe the same noun and
\ird{a} (or often \ird{a še}) when describing two distinct
objects.\footnote{When used this way, the noun preceding \ird{še} or \ird{a še}
is not declined in the instrumental case.}

\pex
\a
\begingl
    \gla Sodoví {še} ludí kobera tahatnik.//
    \glb black with white shirt bring-\Pv{}-\Pf{}//
    \glft \trsl{I brought the black-and-white shirt.}//
\endgl
\a
\begingl
    \gla Sodoví {a} {(še)} ludí kobera tahatnik.//
    \glb black and with white shirt bring-\Pv{}-\Pf{}//
    \glft \trsl{I brought the black shirt as well as the white one.}//
\endgl
\xe

Other common uses of \ird{a} and \ird{še} are described in detail in section
\S\,\ref{sec:conn-conj}

The particle \irdp{nebí}{also} may take a conjunctive meaning when attached to
multiple elements in a sentence, similar to \irdp{a\ldots{}
a\ldots}{both\ldots{} and\ldots} but more emphatic.

\pex
\begingl
    \gla Lukáš nebí Marek nebí naž//
    \glb Lukáš also Marek also friend//
    \glft \trsl{Lukaš and Marek are also my friends.}//
\endgl
\xe

\ird{Má} and \ird{ozná} are used to express contrast, like the English
\trsl{but}. \ird{Ozná} however is more restrictive, and can only be used if the
first clause is in the negative and the second clause directly contradicts (or
provides an alternative to) the first. The clause introduced by \ird{ozná} must
directly correspond to the element in the first clause being negated. Where the
initial element is inflected, such inflection must also be reflected on the
alternative presented in the \ird{ozná} clause.\footnote{The syntax of the main
clause does not necessarily correspond to how the sentence would have otherwise
been constructed in isolation. For instance, the neutral syntax for example
(\getref{ozna}) without the \ird{ozná} would be: \ird{Bięc záčesčeví.}}

\pex
\begingl
\gla Stožek má na duma niho čast.//
\glb go-\Av{}-\Pf{} but \Loc{} house-\Acc{} \mk{nexst} person//
\glft \trsl{I went but no one was home.}//
\endgl
\xe


\pex[tag=ozna]
\begingl
\gla Zám bięc česčeví ozná jec.//
\glb \Neg{} cat to:please-\Av{}-\Cont{} but dog//
\glft \trsl{(I) don't like cats but I do like dogs.}//
\endgl
\xe

\ird{Ozná} does not allow a negative\index{negation} argument. If the main
clause is positive and the secondary clause is negative, \ird{má} is used
instead.

\pex
\begingl
\gla To jako odpizounilá to hrebe cešceví, má zám jáne.//
\glb \Dem{}.\Prox{} tree to:grow-\Loc{}-\Subj{}.\Ipf{} \Rz{} mushroom-\Acc{} to:please-\Av{}-\Cont{} but \Neg{} \Dem{}.\Med{}//
\glft \trsl{Mushrooms love to grow under this tree, but not under that one.}//
\endgl
\xe

\ird{Má} or its variant \ird{a má} (literally \trsl{and but}) is also used to
introduce exclamatory sentences. This usage is purely idiomatic and does not
require for there to be an actual contrastive meaning in the sentences.

\pex
\begingl
\gla A má duma nahte ašteví!// \glb and but house too:much be:pretty-\Cont{}//
\glft \trsl{Your house is very beautiful!}//
\endgl
\xe

Finally, the disjunctive conjunctions\index{disjunctive conjunction} \ird{je},
\ird{li}, and \ird{ni} are used to join phrases or sentences that are seen as
alternatives to each other. \irdp{Je}{or} may be used to separate the
alternatives proposed, or reduplicated, preceding each of the components of the
sentence (i.e., \irdp{je X je Y}{either X or Y}); this latter use often means
that the options being presented are the only ones available.
\ird{Ni}\footnote{\ird{Ni} is an Indo-European, possibly Slavic,
borrowing.\index{linguistic borrowing}} is the inverse of \ird{je} and must
always be used in pairs (\irdp{ni X ni Y}{neither X nor Y}) as when used alone
it functions as an adverb (similar to English \trsl{not even} or \trsl{at all}).
An obvious exception, however, would be in a conversation, when a speaker would
provide a negative alternative response to an already negative statement (see
example (\getfullref{ni.resp}) below).

\pex
\begingl
\gla Ni ircevní ni ruščevní malnovím zahviržéteví.//
\glb nor Iridian-\Att{} nor Russian-\Att{} tongue-\Ins{} speak-\Av{}-\Pot{}-\Cont{}//
\glft \trsl{I can't speak neither Iridian nor Russian.}//
\endgl
\xe

\pex\a\begingl
\gla Dá ircevní malnovím ni zazahviržéteví.//
\glb \First{}\Sg{}\Str{} Iridian-\Att{} tongue-\Ins{} not:even speak-\Av{}-\Pot{}-\Cont{}//
\glft \trsl{I can't speak any Iridian at all.}//
\endgl
\a\begingl
\gla Ni ircevní ni ruščevní malnovím zahviržéteví.//
\glb nor Iridian-\Att{} nor Russian-\Att{} tongue-\Ins{} speak-\Av{}-\Pot{}-\Cont{}//
\glft \trsl{I can't speak neither Iridian nor Russian.}//
\endgl
\a\vtop{\halign{%
#\hfil& \qquad #\hfil\cr
\ird{---\,Dá ruščevní malnovím zahviržéteví.} & \trsl{I don't speak Russian.}\cr
\ird{---\,Ni dá.} & \trsl{Neither do I.}\cr
}}\deftagex{ni}\deftaglabel{resp}
\xe

\ird{Le} (another possible Slavic\index{Slavic} borrowing\index{linguistic
borrowing}, adopted from Common Slavic \emph{li} or \emph{ili}) has a more
emphatic and contrastive meaning than \ird{je}. It is used when the speaker
thinks that the option being presented is counterfactual or doubtful. Unlike
\ird{je} or \ird{ni}, \ird{le} is added to the end of the word or phrase.
\ird{Le} is most often used in parenthetical statements or in responses; it
cannot be used by itself when both alternatives are present and must be
introduced instead by either \ird{je} or \ird{a}.

\pex\a\begingl
\gla Marek-le ruščevní malnovím zahviržéteví.//
\glb Marek=or Russian-\Att{} tongue-\Ins{} speak-\Av{}-\Pot{}-\Cont{}//
\glft \trsl{Or maybe Marek can speak Iridian.}//
\endgl
\a
\begingl
\gla Já Karlu je Terezu-le de Rume sostožit.//
\glb \Second{}\Sg{}.\Str{} Karel-\Ins{} or Tereza-\Ins{}=or \Lat{} Rome-\Acc{} \Rec{}-go-\Av{}-\Sup{}  //
\glft \trsl{Karel\,---\,or maybe even Tereza\,---\,can come with you to Rome.}//
\endgl\xe

\section{Relative clauses and apposition}
\label{sec:relative-clauses}\index{relative clause}\index{apposition}

Appositive constructions\index{apposition} involve the juxtaposition of two or
more noun phrases that have a single referent. The apposition and the noun or
noun phrase it modifies are linked by the particle \ird{ty}. An apposition can
be nonrestrictive if the appositive can be removed freely without changing the
meaning of a sentence, or restrictive otherwise. This distinction is only
semantic in Iridian, as there are no separate forms for restrictive or
nonrestrictive appositive.

\pex\label{ex:appositive}
\begingl
  \gla Mlazka ty Karel po záščenžaní.//
  \glb brother-\Dim{} \Lnk{} Karel still \Neg{}-arrive-\Av{}-\Ret{}//
  \glft \trsl{My brother Karel hasn't arrived yet.}//
\endgl
\xe

Like other modifiers, appositives must always appear before the noun or noun
phrase they modify. In appositive constructions, however, since both the
modifier and the modified element are noun phrases, it might be possible to
switch their positions, but with slight changes in the meaning. For example, the
example above can also be written as:

\pex\label{ex:appositive-switch}
\begingl
  \gla \ljudge{?}Karel ty mlazka po záščenžaní.//
  \glb Karel \Lnk{} brother-\Dim{} still \Neg{}-arrive-\Av{}-\Ret{}//
  \glft \trsl{My brother Karel hasn't arrived yet.}//
\endgl
\xe

In this second example, \ird{mlazka} is the main noun while \ird{Karel} is the
appositive. While (\ref{ex:appositive-switch}) is still grammatical, a
construction like this where a more specific noun phrase would be used to modify
a more general one would imply that the modified noun phrase refers to a group
with multiple elements, with the appositive referring to a specific member of
that group; i.e., (\ref{ex:appositive-switch}) would imply that the speaker has
more than one brother, and that Karel is one of them. There is no such
implication in (\ref{ex:appositive}), where \ird{mlazka} is merely descriptive
of \ird{Karel}; unless of course the discourse has previously established
multiple \ird{Karel}s, in which case \ird{mlazka} would be more specific than
\ird{Karel} and the same group restriction implication can be drawn from
(\ref{ex:appositive}).

It is possible, especially with longer appositives, for the appositive to appear
after the noun or noun phrase it modifies. In this case, however, the appositive
can more properly be analyzed as a parenthetical. The appositive is introduced
by \ird{to-ty} if the noun phrase being modified is inanimate or by \ird{ša-ty}
if the noun phrase being modified is animate. This parenthetical appositive is
then set off from the rest of the sentence by a pair of commas. While regular
appositives can be restrictive or nonrestrictive, parenthetical appositives are
always nonrestrictive.

\pex
\begingl
  \gla Karel, ša-ty Marcí mlazka, po záščenžaní.//
  \glb Karel \Anim{}.\Lnk{} Marek-\Gen{} brother-\Dim{} still \Neg{}-arrive-\Av{}-\Ret{}//
  \glft \trsl{Karel, Marek's brother, hasn't arrived yet.}//
\endgl
\xe

A relative clause\index{relative clause} is a clause that modifies a noun or
noun phrase. In English\index{English}, a relative clause is introduced by a
relative pronoun (such as ``who,'' ``that,'' ``which,'' ``whose,'' or
``where''), such as in the examples below:
\pex
  \a The book, \emph{which was on the table,} was very interesting.
  \a The man \emph{whom I saw at the store} is my neighbor.
  \a The house \emph{where I grew up} is for sale.
\xe

Relative clauses are used to provide additional information about the noun or
noun phrase that they modify. Like appositive constructions, they can be
restrictive or nonrestrictive. Restrictive relative clauses are necessary to
identify the specific noun or noun phrase that is being referred to, while
nonrestrictive relative clauses provide additional, non-essential information
about the noun or noun phrase, which could be removed from a sentence altogether
without changing its meaning.

Unlike English, however, Iridian does not employ relative pronouns to link
relative clauses with their antecedents. Instead the main verb in the relative
clause is nominalized and is used to directly modify the noun or noun phrase as
a nominal appositive. Unlike simple nominal appositives however, a nominalized
relative clause must be linked to its antecedent with the particle \ird{ty}. 

\pex
\a\begingl
  \gla Ša maša magazinu vednik.//
  \glb this man store-\Ins{} see-\Pv{}-\Pf{}//
  \glft \trsl{(I) saw this man at the store.}//
\endgl

\a\begingl
  \gla Magazinu vednikou ty maša blež.//
  \glb store-\Ins{} see-\Pv{}-\Pf{}-\Nz{} \Lnk{} man neighbor//
  \glft \trsl{The man (I) saw at the store is (my) neighbor.}//
\endgl
\xe

The relative clause must always appear before the noun or noun phrase it
modifies. As with appositives, the relative clause can however come after the
noun it modifies as a parenthetical, introduced by \ird{to-ty} if the noun
phrase being modified is inanimate or by \ird{ša-ty} if the noun phrase being
modified is animate. Since parentheticals are always nonrestrictive, the
relative clause can only be shifted to this position if it is nonrestrictive as
well.

\pex
\a\begingl
  \gla \ljudge{*}Maša ty magazinu vednikou blež.//
  \glb man \Lnk{} store-\Ins{} see-\Pv{}-\Pf{}-\Nz{} neighbor//
  \glft \trsl{The man (I) saw at the store is (my) neighbor.}//
\endgl
\a\begingl
  \gla Marek, ša-ty magazinu vednikou, blež.//
  \glb Marek \Anim{}.\Lnk{} store-\Ins{} see-\Pv{}-\Pf{}-\Nz{} neighbor//
  \glft \trsl{Marek, whom I saw at the store, is my neighbor.}//
\endgl
\xe

Iridian further restricts the formation of relative clauses by requiring that
the shared noun occupy the topic position in the embedded clause. Thus  a
sentence like (num) below would be incorrect:

*Dá magazinu vižkou ty maša


\section{Syntax of event and participant nominals}\index{nominalization!event
nominal}\label{sec:nomz-syntax}

\subsection{Gerunds and event nominals}

As we have established in \S~\ref{sec:nominalized}, Iridian has three forms of
nominalization\index{nominalization}: (1)~the mainly non-productive usage of the
nominalising \ird{-ou} with the verbal stem to form resultant nominals; (2)~the
use of \ird{-ou} together with the gerund-forming suffix\index{gerund} \ird{-c}
to form a verbal noun (which we call an event nominal or simply a gerund) and
which may either include the internal arguments of the parent verb or not; and
(3)~the formation of a participant nominal (cf.~\cite{okuna}) which nominalizes
not the event described by the verb but its participants.

Since gerunds\index{gerund} represent the nominalization of the
event\index{event nominal} described by the verb, they are therefore inherently
abstract and active in meaning. Since the nominalized forms are abstract, it
follows that they are also tenseless and aspectless. Iridian gerunds, however,
may be optionally marked for their lexical aspect or \foreign{aktionsart}
\index{aktionsart@\emph{aktionsart}}\index{lexical
aspect|see{\emph{aktionsart}}} using the continuous aspect suffix \ird{-eví}
(which subsequently becomes \ird{-év-} through sound change). It is important to
note that although a marker for grammatical aspect\index{aspect} is used,
what is being marked is lexical and not grammatical aspect; specifically, the
addition of \ird{-év-} only signifies that the action is iterative in nature and
thus the gerund itself remains tenseless\index{tense} and aspectless.\index{aspect}

\pex
    \a \ird{nidá} → \ird{nidouc}\\
        \trsl{the act of standing up}
    \a \ird{nidá} → \ird{nidévouc}\\
        \trsl{the act of standing up repeatedly}
\xe

In {\sc cen}s, both the agent and the patient are marked in the
genitive.\index{genitive}\footnote{\textcite{serekaite2020} argues that although
(in the case of Lithuanian, at least) the actor and the theme from the original
sentence both become marked in the genitive in the resulting complex event
nominal, the superficially indentical genitives are actually two distinct cases:
a higher genitive ({\sc gen.h}) assigned to agents and possessors and a lower
genitive ({\sc gen.l}) assigned to grammatical objects. Although this argument
is interesting and probably holds true as well in Iridian {\sc cen}s, we will
not make an effort to ascertain whether there is an actual difference in the two
genitive cases in Iridian as this is not needed for the purpose of this
grammar.} If both are present, the agent must always appear first. This
construction is quite common cross-linguistically, as we see in the examples
below.

\pex
\a\begingl
    \gla Mlazcí praví na Mnihe poznohouštou na zahrana nemniček.//
    \glb brother-\Dim{}-\Gen{} law-\Gen{} \Loc{} Munich-\Acc{} \Ger{}-study-\Nz{} \Loc{} beginning-\Acc{} surprise-\Av{}-\Pf{}//
    \glft \trsl{My brother's studying law (i.e., my brother's decision to study law) in Munich surprised us at first.}//
\endgl
\a Lithuanian\index{Lithuanian} (\cite[1]{serekaite2020})\\
\begingl
    \gla Jono augalų sunaikinimas.//
    \glb Jonas-\Gen{} plants-\Gen{} \Pfv{}-destroy-\Caus{}-\Nz-\Nom{}.\M{}.\Sg{}//
    \glft \trsl{Jonas' destruction of plants}//\deftagex{doubgen}\deftaglabel{lithuanian}
\endgl
\a Tagalog\index{Tagalog} (\cite[22]{hsieh2019})\\
\begingl
    \gla (Ang) Pagluluto ni Harvey (ng manok) ang nangyari.//
    \glb \Nom{} \Ger{}$\sim$cook \Gen{} Harvey \Gen{} chicken \Nom{} happen.\Pfv{}//
    \glft \trsl{What happened was Harvey's cooking (of chicken).}//\deftagex{doubgen}\deftaglabel{tagalog}
\endgl
\xe

The use of the genitive\index{genitive} to mark both the actor and the theme in
the original sentence is of course a recipe for ambiguity. When only one of
either the actor or the theme is present in the {\sc cen}, the ambiguity is on
whether the noun marked represents the one or the other, as, e.g., the phrase
\ird{Jancí podohletou} which can be interpreted to mean either \trsl{the act of
remembering Janek} or \trsl{Janek's act of remembering} without any further
information. A second ambiguity arises when both the actor and the theme are in
the sentence as it is unclear, without any context, the genitive is actually
being used to mark their thematic role in the originally or is in fact a
possessive. The same is true in, for example, Lithuanian\index{Lithuanian} where
as \textcite{serekaite2020} points out, sentence
(\getfullref{doubgen.lithuanian}) can also be alternatively translated as
\trsl{[the] destruction of Jonas's plants}.

The first type of ambiguity is resolved in English\index{English} by using word
order: in general, a prepositive genitive (i.e., using the clitic \foreign{'s}
or the possessive form of a pronoun) is used when the noun in the genitive case
in the {\sc cen} represents the actor (e.g., \trsl{John's remembering}) while a
postpositive genitive is used when the noun in the genitive represents the theme
(e.g., \trsl{the remembering of John}). This in turn, can be extended to the
second type, e.g., \trsl{John's remembering of Margaret}. However, the
obligatorily head-final nature of Iridian syntax means that such strategy is not
possible. Instead, the strategy used in Iridian is more similar to the one found
in Tagalog\index{Tagalog} where the theme may be marked using the oblique
\foreign{sa}\footnote{This becomes \foreign{kay} before proper nouns.} instead
of the genitive \foreign{ng}.\footnote{ To call \foreign{ng} (pronounced [nɐŋ])
as a genitive marker is simplistic (even erroneous) but should be enough for the
purpose of our discussion. } Thus we can restate (\getfullref{doubgen.tagalog})
as follows:

\pex{Tagalog\index{Tagalog} (modified from \cite[22]{hsieh2019})}\\
\begingl
    \gla (Ang) Pagluluto ni Harvey {sa} manok ang nangyari.//
    \glb \Nom{} \Ger{}$\sim$cook \Gen{} Harvey \Obl{} chicken \Nom{} happen.\Pfv{}//
    \glft \trsl{What happened was Harvey's cooking of \emph{the} chicken.}//
\endgl
\xe

An immediate consequence of replacing the genitive \foreign{ng} with the oblique
marker \foreign{sa/kay} is that the theme is now interpreted as definite
(cf.~\cite[3,\,40]{kaufman2009}). The use of the oblique to mark the theme can
be used even when only one element is present in the event nominal; in fact,
when the theme is known as definite for a fact (e.g., if it is a person), the
choice between the oblique and the genitive is what distinguishes the actor and
the theme. Thus we have

\pex[interpartskip=0pt]
    \a Choice between \Obl{} and \Gen{} distinguishing actor from theme
    \beginsubsub\index{Tagalog}
        \b{--}{\foreign{pagtawag kay {\nf{[\Obl{}]}} Harvey}\\ \trsl{the act of calling Harvey}}
        \b{--}{\foreign{pagtawag ni {\nf{[\Gen{}]}} Harvey}\\ \trsl{Harvey's act of calling}}
    \endsubsub
    \a Resolving ambiguity by obligatory replacement of \Gen{} by \Obl{} in the theme:
    \beginsubsub
        \b{--}{\foreign{pagtawag ni {\nf{[\Gen{}]}} Harvey sa {\nf{[\Obl{}]}} kasama}\\
        \trsl{Harvey's act of calling his \mbox{colleague}}}
        \b{--}{\foreign{pagtawag ni {\nf{[\Gen{}]}} Harvey ng {\nf{[\Gen{}]}} kasama}\\
        \trsl{Harvey's act of calling a colleague}}
    \endsubsub
    \a New ambiguity introduced by changing the word order:
    \beginsubsub
        \b{--} {\foreign{pagtawag ng {\nf{[\Gen{}]}} kasama ni {\nf{[\Gen{}]}} Harvey}\\
        \trsl{Harvey's act of calling a colleague} or \trsl{The act of calling Harvey's colleague}}
    \endsubsub
    \a Ungrammatical form, with both the theme and actor marked in the oblique:
    \beginsubsub
        \b{--}{\ljudge{*}\foreign{pagtawag kay {\nf{[\Obl{}]}} Harvey sa {\nf{[\Obl{}]}} kasama,}\\
        \trsl{Harvey's act of calling a colleague}}
    \endsubsub
    \a Double genitive, with both indefinite actor and theme:
    \beginsubsub
        \b{--}{\foreign{pagtawag ng {\nf{[\Gen{}]}} tao ng {\nf{[\Gen{}]}} kasama,}\\
        \trsl{a person's act of calling a colleague} or \trsl{a colleague's act of calling of a person}} 
    \endsubsub
\xe


In Iridian, the a \ird{na} clause corresponds to the Tagalog\index{Tagalog} use
of the oblique to indicate a definite theme in a {\sc cen}. 

% nemnetá from CS m{\yer}neti to think + ne not



\subsection{Participant nominals}

Participant nominals are formed by nominalizing a finite verb phrase with the
suffix \ird{-ou}. The resulting noun refers back to a participant in the event
rather than the event itself, with the role determined by the grammatical voice
in which the original verb phrase is marked. Consequently, participant nominals
are inherently definite in meaning.

\pex
\a\begingl
    \gla Jancí materška najevěc shradnaní.//
    \glb Janek-\Gen{} stepmother drive-\Cv{} die-\Pv{}-\Ret{}//
    \glft \trsl{Janek's stepmother was killed in a car crash.}//
\endgl
\a\begingl
    \gla Jancí materšcí najevěc shradněnou policám zánehévorneví.//
    \glb Janek-\Gen{} stepmother-\Gen{} drive-\Cv{} die-\Pv{}-\Ret{}-\Nz{} police-\Agt{} \Neg{}-\Caus{}-know-\Pv{}-\Cont{}//
    \glft \trsl{The police still hasn't identified the person Janek's stepmother has killed in the crash.}//
\endgl
\xe

The creation of participant nominals is a very common strategy in Iridian.
Participant nominalization is also used to shift the focus of the sentence from
the event to the participant. For example, transforming the sentence \irdp{Janek
shražek}{Janek died} into \irdp{Janek shražkou}{It is Janek who died} or more
emphatically, \irdp{Shražkou Janek}{It is Janek who died} changes the emphasis
in the sentence.

\section{Converbial constructions}\label{converbs-syntax}\index{converb}

\subsection{In general}

In \S~\ref{sec:converb}, we have defined a converb as a non-finite verb form
that is often used adverbially. In this section, we will discuss the syntax of
converbial constructions in Iridian.

The most common type of converbial constructions involves the main verb preceded
by the imperfective converbial form of a secondary verb. The secondary verb
normally specifies the manner or the means by which the action described by the
main verb is performed. Adverbial constructions such as these tend to be used
even where English, for example, would use a single verb. For example, in the
sentence \trsl{He cut the branch} would be analyzed in Iridian as \trsl{He
removed the branch by cutting} as Iridian would interpret the verb `cut' as used
in the first sentence as encoding both the action performed and the manner in
which it was performed. Although the second sentence below is not necessarily
incorrect, it would sound unnatural in Iridian.

\pex
\a\begingl
  \gla Platek odněc rutnik.//
  \glb leaf cut-\Cv{}.\Ipf{} remove-\Pv{}-\Pf{}//
  \glft \trsl{(He) removed the leaf by cutting it.}//
\endgl
\a\begingl
  \gla\ljudge{?}Platek odnenik.//
  \glb leaf cut-\Pv{}-\Pf{}//
  \glft \trsl{(He) cut the leaf.}//
\endgl
\xe

\subsection{Temporal constructions}

A converbial construction is often used in temporal clauses\index{temporal
clause}, with the imperfective converbial form used when the action is
unfinished or continuing and the perfective otherwise. When used in a temporal
clause, the converb may sometimes be separated from the main clause by the
particle \ird{si}.\footnote{\ird{Si} is virtually never used in the spoken
language.}

\pex
\begingl
\gla Otvěc (si) na Varšave možlašaní.//
\glb be:young-\Cv{}.\Ipf{} when \Loc{} Warsaw-\Acc{} understand-\Av{}-\Ret{}//
\glft \trsl{When I was young, we used to live in Warsaw.}//
\endgl
\xe

\subsection{Causal clauses}

Clauses expressing reason are usually expressed by a converbial construction.
The antecedent and the main clause may be connected with \irdp{am}{because,}
although this is often dropped in casual speech.

\pex
\begingl
\gla Za prove záznohouštu Martin meštnašek.//
\glb for exam-\Acc{} \Neg{}-study-\Cv{}.\Pf{} Martin fail-\Av{}-\Pf{}//
\glft \trsl{Martin failed the exam because he didn't study.}//
\endgl
\xe


\pex
\begingl
\gla Kinoteka stožílá to všihněc mámka zachovažek.//
\glb cinema-\Acc{} go-\Av{}-\Sbj{}.\Ipf{} \Rz{} be:angry-\Cv{}.\Ipf{} mother-\Dim{} allow-\Av{}-\Pf{}//
\glft \trsl{Since she was still mad at us, Mum did not let us go to the movies.}//
\endgl
\xe


\subsection{Similarities with the Czech and Slovak transgressive}

Converbs in Iridian have parallel usage as the
transgressive\index{transgressive} conjugations in \printlang{cs}\index{Czech}
and Slovak\index{Slovak}. It is the consensus among scholars of the languages,
though, that the converbial forms in Iridian and the transgressive forms in
Czech and Slovak, developed independently of each other; although to what extent
one influenced the other is still the subject of debate. The converbial forms in
Iridian have more varied uses than the transgressives in Czech (Slovak having
kept only the present transgressive form), and whereas the latter forms have
largely fallen in disuse (relegated to the literary register) in both Czech and
Slovak, converbial forms are still widely used in Iridian.

Although Czech grammarians use the terms `past' and `present' to distinguish
between the two forms used in the language, the distinction is actually one of
aspect\index{aspect}, as in Iridian. In general, the past transgressive form
corresponds with the perfect converbial form, and may be used to indicate a
foregoing action; the present transgressive, on the other hand, corresponds to
the imperfect converb and is used to indicate a coincident/contemporaneous
action.

This correspondence is not complete, however. For example, consider this
sentence in Czech\index{Czech}: \foreign{Děti, {vidouce} babičku, vyběhly
ven}{The children, seeing their grandmother, ran outside.} The verb in the
transgressive clause is in the present tense in this case, while in Iridian, the
same sentence will be translated with the perfective as follows:

\pex
\begingl
\gla Šášlika vedu byl naladěc mnilžek.//
\glb grandmother-\Dim{}-\Acc{} see-\Cv{}.\Pf{} children run-\Cv{}.\Ipf{} go:out-\Av{}-\Pf{}//
\glft \trsl{The children, having seen their grandmother, ran outside.}//
\endgl
\xe

The Czech\index{Czech} sentence above can alternatively be translated using the
imperfective converbial form, but this would put a stronger emphasis on the two
actions happening at the same time and so the original construction can be
considered as the more idiomatic one.

\subsection{In fixed expressions}

The past converbial form is used in expressing gratitude, approbation or
condolencess, or in asking for forgiveness. This usage is idiomatic and the
actions do not necessarily need to have been completed. The main clause is often
in the hortative mood\index{hortative mood} and separated from the converb
clause with \irdp{am}{because.} Moreover, this usage, unlike most converbial
constructions, allow the verb of the converb clause to have a different subject
as long as such subject is marked explicitly in the agentive case. However,
since the converbial form of verbs are invariable, if the subordinate clause
requires further complexity when it comes to the verb in the converb clause, a
dependent \ird{še} clause may be use instead of a converb.

\pex
\a Expressing gratitude:\\
\begingl
\gla Stranu am luhninká.//
\glb help-\Cv{}.\Pf{} because thank-\Pv{}-\Hort{}//
\glft `Thank you for helping.'//
\endgl
\a Asking for forgiveness:\\
\begingl
\gla Lěnu záščenu am rozvedniká.//
\glb on:time-\Ins{} \Neg{}-arrive-\Cv{}.\Pf{} because forgive-\Pv{}-\Hort{}//
\glft `Sorry for being late.'//
\endgl
\a Expressing condolences:\footnote{Compare this example to the following, where
a converb clause cannot be used:

\ex[lingstyle=fnex,belowexskip=-1em]
\begingl
\gla Pápka na puvode shradniš to množniká.//
\glb father \Loc{} war-\Acc{} die-\Pv{}-\Subj.\Pf{} \Rz{} with console-\Pv{}-\Hort{}//
\glft `I'm sorry to hear your father died (\emph{lit.,} was killed) in the war.'//
\endgl\xe}\\
\begingl
\gla Pápkám shradu am množniká.//
\glb father-\Dim{}-\Agt{} die-\Cv{}.\Pf{} because console-\Pv{}-\Hort{}//
\glft `I'm sorry for your father's death.'//
\endgl
\a Expressing approbation:\\
\begingl
\gla Prove vlastnu am prehodniká.//
\glb exam-\Acc{} pass-\Cv{}.\Pf{} because praise-\Pv{}-\Hort{}//
\glft \trsl{Congratulations for passing the exam!}//
\endgl
\xe

\section{The conditional mood in subordinate clauses}
\label{sec:conditional-subordinate}
\index{conditional mood}

\section{Quotative constructions and  evidentiality}\label{sec:reportedspeech}
\index{reported speech}\index{indirect speech|see{reported speech}}
\index{evidentiality}

\subsection{Quotative construction in general}

Superficially, the Iridian quotative is used to mark {\sc evidentiality}, a
grammatical category concerned with the explicit encoding of the source of
information or knowledge (i.e., evidence) which the speaker claims to have made
use of for producing the primary proposition of the utterance
(\cite[1-2]{diewald2010}). Iridian is unique among languages of Central Europe
(and of Europe in general) in possessing a grammaticalized evidentiality system.
Even non-Indo European languages in the region such as Hungarian (cf. author) or
Basque (cf. \cite{alcazar2010}) do not possess an overt evidential. Of course a
speaker’s source of information may be expressed through other methods 

The Iridian evidentiality system more or less falls under
\posscite{aikhenvald2004} A3 category, where the distinction is between the
marked quotative form for reported speech/hearsay and the unmarked ‘everything
else’ category which is evidentiality-neutral

In practice, however, the quotative is used in an array of other constructions
that is not necessarily predicated on evidentiality, but might be lexically or
semantically motivated as well, perhaps in the same way the subjunctive in
Romance languages have become grammaticalized into a subordination marker (cf.
\cite{poplacketal}).

\subsection{Quotative constructions and reported
speech}\label{sec:quotative-const}

The principal use of the quotative is to explicitly mark reported speech. The
reported clause is separated from the rest of the sentence by the particle
\ird{cy}. Direct quotations do not require the quotative, although they are
still separated from the main clause by \ird{cy}.

\pex
  \begingl
    \gla Koleč sní polšice cy Lukáš zíček.//
    \glb key \Refl{}.\Acc{} lose-\Av{}-\Pf{}-\Quot{} \Qp{} Lukáš say-\Av{}-\Pf{}//
    \glft ‘Lukáš said he lost his keys.’//
  \endgl
\xe

\pex
  \begingl
    \gla „Záščenžit” cy zíček.//
    \glb \First{}\Sg{} \Neg{}-come-\Av{}-\SupP{} \Qp{} say-\Av{}-\Pf{}//
    \glft \trsl{“I won’t be coming,” (he) said.}//
  \endgl
\xe

The use of pronouns in quoted clauses is similar to English, with the main
exception being the use of the reflexive \ird{se} if the subject of the quoted
clause is the same as the subject of the main clause. This is true even if the
subject of the main clause is a pronoun.

\pex
  \begingl
    \gla Se to obru na večera záščenžite cy Marek (dá) žiček. //
    \glb \Refl{} \Dem{} night-\Ins{} \Loc{} party-\Acc{} \Neg{}-come-\Av{}-\SupP{}-\Quot{} \Qp{} Marek \First\Sg{} say-\Av{}-\Pf{} //
    \glft \trsl{Marek/I said he/I won't be coming to the party tonight.}//
  \endgl
\xe

The verb \irdp{zěká}{to say} is called a \emph{verbum dicendi}\index{verbum
dicendi} from the Latin meaning ‘verb of speech/speaking.’ Other \emph{verba
dicendi} in Iridian include \irdp{vadá}{to think}; \irdp{kvuštá}{to hear};
\irdp{vidá}{to see}; \irdp{hloupá}{to ask}; \irdp{ohletá}{to remember};
\irdp{sehová}{to recount, to tell a story}. Note that although they are called
verbs ``of speaking'' they do not necessarily introduce speech as much as
function as grammaticalized tags marking the quotative,  which is more properly
analyzed to mark not just speech but inferentiality and evidentiality as well.

More complex \emph{verba dicendi} can be formed by using an imperfect converbial
construction (the converb form in \ird{-ěc}) with a canonical \emph{verbum
dicendi}. To illustrate this consider the following sentences in English:

\pex[*=?*,interpartskip=0pt]
\a\label{ex:vd1} She said no.
\a\label{ex:vd2} She whispered no.
\a\label{ex:vd3} She said no \emph{in a whisper}.
\a\label{ex:vd4} \ljudge{?} She said \emph{in a whisper} no.
\a\label{ex:vd5} \ljudge{??} She said \emph{whisperingly} no.
\xe

We see that both \emph{said} (\ref{ex:vd1}) and \emph{whispered} (\ref{ex:vd2})
are \emph{verba dicendi} in English. Nonetheless it's also obvious how
(\ref{ex:vd2}) is simply a function of (\ref{ex:vd1}), i.e., we can express
(\ref{ex:vd2}) in terms of (\ref{ex:vd1}), in this case using an adverbial
construction (\trsl{in a whisper}) as we see in (\ref{ex:vd3}) or the more
affected (\ref{ex:vd4}). Finally using a simple adverbial is theoretically allowed
in English (\ref{ex:vd5}), although as we see the resulting construction is
rather unwieldy or unnatural-sounding.

In Iridian, however, constructions like (\ref{ex:vd2}) are not permitted, with
preference given to adverbial (or more correctly, converbial)\index{converb}
constructions. Thus we translate (\ref{ex:vd2}) as:

\pex
\begingl
\gla Ne cy mišlec zíček.//
\glb no \mk{qp} whisper-\Cv{} say-\Av{}-\Pf{}//
\glft \trsl{(She) whispered no.}//
\endgl
\xe

When using \irdp{vadá}{to think} as the \emph{verbum dicendi} the verb in the
reported clause must be in the conditional. This is true whether or not the verb
in the main clause would otherwise have been in the conditional had it been in
an independent sentence.


The \emph{verbum dicendi} is often marked in the agentive voice, although
Iridian grammar also permits the verb to be marked in the patientive, but with
the resulting construction often having a more explanatory meaning.

\pex
\a
\begingl
  \gla Já mnou nehlí cy Martin spouvěc váževí.//
  \glb you correct \Cop{}.\Sbj{}.\Quot{} \Qp{} Martin agree-\Cv{}.\Ipf{} think-\Av{}-\Cont{}//
  \glft \trsl{Martin agrees that you are right.}//
\endgl
\a
\begingl
  \gla Já mnou nehlí cy Martin spouvěc vadneví.//
  \glb you correct \Cop{}.\Sbj{}.\Quot{} \Qp{} Martin agree-\Cv{}.\Ipf{} think-\Pv{}-\Cont{}//
  \glft \trsl{What Martin agrees to is that you are right.}//
\endgl
\xe

We see from  that when it comes to reported speech and similar constructions in
Iridian, the \ird{verbum dicendi}\index{verbum dicendi} is not necessary to
create a well-formed sentence. The same is true with the quotative particle
\ird{cy}. Both can be omitted without making the sentence grammatically
incorrect since the quotative particle is enough to identify the reported
clause.\index{reported speech}.

In most instances, however, removing either the main verb or the main verb and
the quotative particle can cause the resulting sentence to acquire a new
meaning. This is especially true when the quotative mood is used not to report
speech but to imply a certain unsureness on the part of the speaker about the
information being presented, or for the speaker to distance themself by implying
through the use of the quotative that the information is secondhand and not
theirs. Generally \ird{cy} is kept when the speaker is quoting themself, to
repeat or emphasize what they have said, or expletively, to express their
frustration or affirmation.

Interestingly, commands and requests are not treated as reported speech but as
regular subordinate clauses governed by \ird{to} and not by \ird{cy}.

When the quoted clause is a question, whether a direct one or not, the quoted
clause is preceded by the particle \irdp{a}{and} and the word
\irdp{ane}{whether} is used instead of \ird{cy}. The word \ird{ane} is also
used for verba dicendi that are interrogative in nature, such as
\irdp{préhoustá}{to ask},

\pex
\begingl
  \gla A Janek zdalšice ane préhousček.//
  \glb and Janek have:breakfast-\Av{}-\Pf{}-\Quot{} whether ask-\Av{}-\Pf{}//
  \glft \trsl{(He) asked (me) whether Janek has had breakfast yet.}//
\endgl
\xe

\pex
\begingl
  \gla A tóm to mládu hodinaže ane, ně svad postupeví.//
  \glb and book this year-\Ins{} finish-\mk{pv-ctpv-quot} whether \Pl{} fan be:excited-\Cont{}//
  \glft \trsl{His fans are excited to know if he'll finish his book this year.
}//
\endgl
\xe

The quotative is also triggered by phrases introduced with \irdp{ty}{according
to} or \irdp{záty}{contrary to,} with the latter requiring the subjunctive. 

\pex
\begingl
  \gla Messi a ty Marku debil neví.//
  \glb Messi and according:to Marek-\Ins{} spaz \Cop{}.\Sbj{}//
  \glft \trsl{Marek thinks Messi is a spaz.}//
\endgl
\xe

\pex
\begingl
  \gla Na Vrešlove a záty mamcě čestu papcě vednice stožišejí.//
  \glb \Loc{} Wrocław-\Acc{} and \Neg{}-according:to mother-\Dim{}-\Gen{} desire-\Ins{} father-\Gen{} see-\Pv{}-\SupP{} go-\Av{}-\Subj{}.\Pf{}-\Quot{}//
  \glft \trsl{Against my mother’s wishes, I went to Wrocław to see my father.}//
\endgl
\xe


\subsection{Bare quotatives and clause linking}

Quoted clauses in Iridian may also appear without an overt predicate, as well as
without being signalled by the quotative particle \ird{cy}. We will call this
construction a {\sc bare quotative} after the terminology in
\textcite{tomioka2019} in reference to embedded quotative constructions in
Japanese and Korean without overt predicates. The term as originally used by
these authors refer only to embedded quotatives in Japanese and Korean, but we
will be using it to refer to both an unselected (i.e., predicateless) quotative
in a subordinate clause (which we will call {\sc syntactic}) and in the main
clause (which we will call {\sc semantic}).

The choice to call the second type a semantic bare quotative is motivated by the
fact that an unselected quotative in the main clause is often used not to mark a
speech act but to indicate the epistemic value of (viz., to pass the speaker's
judgement on) a proposition. Nevertheless, we can still see it used as a true
quotative, as when the omission of the predicate or the quotative particle is
through mere ellipsis.

The first type, on the other hand, is mostly used as a clause-linking strategy.
The quotative construction is still considered as a speech act, but, like
converbial constructions or \ird{še} clauses, the relationship between the main
clause and the reported clause becomes interpreted as being one of causality, or
at least of dependency, although of course this causality or dependency is only
indirect, as we see in the examples below, where the embedded quotative and the
simple \ird{še} clause present to different interpretations.

\pex
  \a(adapted from \cite[3]{tomioka2019})\\
  \begingl
    \gla Pizba rážice še sad Markám nakdavtébik.//
    \glb rain stop-\Av{}-\Pf{}-\Quot{} \Com{} garden Marek-\Agt{} \Incp{}-clean-\Ben{}-\Pf{}//
    \glft \trsl{Marek began cleaning the garden, (saying/thinking) it finally stopped raining.}//
  \endgl
  \a\begingl
    \gla Pizba razek še sad Markám nakdavtébik.//
    \glb rain stop-\Av{}.\Pf{} \Com{} garden Marek-\Agt{} \Incp{}-clean-\Ben{}-\Pf{}//
    \glft \trsl{The rain having stopped, Marek began cleaning the garden.}//
  \endgl
\xe


\subsection{Epistemic extensions}

As in most other languages with an overt evidential system, the Iridian
quotative has secondary epistemic extensions. This may be realized either by
using the quotative by itself or through auxiliary epistemic markers. As we have
established in the previous sections, the quotative can be used by a speaker
both to distance themself from the statement on the one hand and to assert their
belief in its truthfulness on the other; the use of a secondary epistemic marker
eliminates this possible confusion in what would otherwise have been a
contradictory usage of the same grammatical category. These auxiliary particles,
nonetheless, may of course be left out in discourse if the speaker thinks the
epistemic usage of the quotative is clear enough from the context.

A speaker’s judgement of the truthfulness of a statement may be made clear by
the dubitative \ird{bude} or the affirmative \ird{toleto}. When using the
quotative to quote oneself, \ird{bude} expresses a disbelief predicated upon
surprise rather than on a judgement of a statement’s veracity; used the same
way, \ird{toleto} acquires a secondary meaning of insistence, even annoyance.

\pex
\begingl
  \gla Sól bude tahatnitejí.//
  \glb peace \Infer{} bring-\Pv{}-\SupP{}-\Quot{}//
  \glft \trsl{They say they come in peace but I don’t believe it.}//
\endgl
\xe

\pex
\begingl
  \gla Ma já bude ža konědnitejí to!//
  \glb but \Second{}\Sg{} \Infer{} already marry-\Pv{}-\SupP{}-\Quot{} \Rel{}//
  \glft \trsl{I still can’t believe you’re already getting married!}//
\endgl
\xe

\pex
\begingl
  \gla Marek toleto poslem všihnébice.//
  \glb Marek \Aff{} message-\Agt{} be:angry-\Ben{}-\Pf{}-\Quot{}//
  \glft \trsl{I’m telling you the message really made Marek angry.}//
\endgl
\xe

\pex
\begingl
  \gla Méva toleto sehovnáně!//
  \glb all \Aff{} recount-\Pv{}-\Ret{}-\Quot{}//
  \glft \trsl{But I’ve told you everything I know already!}//
\endgl
\xe

A speaker’s uncertainty may also be expressed using the quotative even when the
statement directly came from the speaker. The uncertainty may refer to both the
factuality of the statement or to its source. This strategy is used to signal
the speaker’s emotional or cognitive distance from the event. This may be
further complemented by the particle \ird{iz} which we will glossing here as
\Rep{} for reportative but only for the sake of convenience, in order to
distinguish the various auxiliary particles we have introduced here, as the
“reportative” does not exist as a true grammatical category in Iridian for our
purposes. \ird{Iz} implies a greater degree of disjunction between the speaker
and the statement than the plain quotative. Although it does not pass a
judgement on the truth value of the statement as do \ird{dube} or \ird{toleto},
\ird{iz} makes it clear that the statement did not come from the speaker and
that the responsibility for the statement does not lie on them. \ird{Iz} is
particularly common in newscasts or in other formal settings where the speaker
is communicating statements from another speaker or group and the identity of
the speaker or group has already been established earlier in the conversation
and is thus known to everyone.

\pex
\ird{Interiorministerium shléd o senátor Koupárám poto němstministar Novaka
dozakuzacunóvim arklaruma mnilounek. Na Ministerija še Ružómu ty
zěka\-mi\-te\-mu nežni posohredou, a viční němstministarí za Moshóva besuk
{\emph{iz}} Ministerija zázběro\-vnevíje. Akuzace \emph{iz} shlac
investěharnimejí a němstministar \emph{iz} udarklaržice za Ministara breví
paholžáše.}\smallskip\\
{\footnotesize\trsl{The Ministry of the Interior has released a statement today
regarding the accusations of misconduct levelled by Senator Koupár against
Deputy Minister Novak. According to its spokesperson, the ministry is currently
not in talks with Russia and has not sanctioned the reported Deputy Minister's
recent visit to Moscow. It is now investigating the allegations and has asked
Deputy Minister Novak to submit a brief to the Minister to explain his
actions.}}
\xe

Uncertainty on the truthfulness of the statement may also be expressed using the
inferential particles \ird{bylo} and \ird{atole}. Whereas \ird{iz} raises the
question of the character of the source and is neutral as to the speaker’s
commitment to it (although one can be understood simply by pointing out the fact
that the source is something other than oneself to be effectively passing
judgement) both \ird{bylo} and \ird{atole} reflect the speaker’s judgement.
\ird{Bylo} in general is used when the proposition is coming from the speaker
themself while \ird{atole} is used when the speaker thinks that the statement
can be inferred from the surrounding facts.

\pex
\begingl
  \gla Na Hospode bylo milestunitejí.//
  \glb \Loc{} Hospoda-\Acc{} perhaps have:dinner-\Lv{}-\SupP{}-\Quot{}//
  \glft \trsl{Maybe we can have dinner at the \emph{Hospoda} tonight?}//
\endgl
\xe

\pex
\begingl
  \gla Ně ruščevní šar atole na Roubžína ščenžáně.//
  \glb \Pl{} Russian-\Att{} tank \Infer{} \Loc{} Roubže-\Acc{} arrive-\Av{}-\Ret{}-\Quot{}//
  \glft \trsl{The Russian tanks must have reached Roubže by now.}//
\endgl
\xe

\section{Relative and comparative
constructions}\label{relativecomparative}\index{comparative construction}

The clitic\index{clitic} \ird{tám} is used to form simple comparative and
relative constructions. \ird{Tám} is often ommitted where the comparison can be
implied from context. In this construction, the standard of
comparison\index{standard of comparison} (the noun preceded by `than' in
English\index{English}) is unmarked and the noun being compared marked in the
agentive\index{agentive case} if it is a positive/negative comparison, or in the
instrumental\index{instrumental case} if it is a correlation.

\pex
\a\begingl
\gla Janek(-tám) Markám nestaževí.//
\glb Janek Marek-\Agt{} tall-\Cont{}//
\glft \trsl{Marek is taller than Janek.}//
\endgl
\a\begingl
\gla Janek(-tám) Marku nestaževí.//
\glb Janek Marek-\Ins{} tall-\Cont{}//
\glft \trsl{Marek is as tall as Janek.}//
\endgl
\xe

Note that \ird{tám} can only be used with the copulative form of the stative
verb\index{stative verb}, as the attributive and nominal forms have separate
conjugated comparative forms. When using these forms, however, the standard of
comparison is marked in the genitive\index{genitive case}. In relative
constructions, the instrumental\index{instrumental case} is also replaced with
the genitive\index{genitive case}, but the modifier \ird{zní}, \trsl{same} is
added before the stative verb\index{stative verb}.

\pex
\a
\begingl
\gla Jancí nestašení hloc mlazka.//
\glb Janek-\Gen{} tall-\Comp{}-\Att{} boy brother-\Dim{}//
\glft \trsl{The boy who is taller than Janek is my brother} (\emph{Lit.,} \trsl{The taller-than-Janek boy is my brother.})//
\endgl
\a
\begingl
\gla Jancí zní nestažení hloc mlazka.//
\glb Janek-\Gen{} same tall-\Comp{}-\Att{} boy brother-\Dim{}//
\glft \trsl{The boy who is as tall as Janek is my brother.}//
\endgl
\xe

\ird{Tám} can be relativized by appending the clitic\index{clitic} \ird{to}.
When used with \ird{tám-to} the standard of comparison is marked in the
patientive case\index{patientive case}. The use of tám-to in relative clauses is
discussed in further detail in the next chapter.

\ex
\begingl
\gla Viktor na shlopa tám-to nestážek.//
\glb Viktor \Loc{} siblings-\Acc{} \Comp{}=\Rz{} be:tall-\Av{}-\Pf{}//
\glft \trsl{Among the siblings, Viktor grew up to be the tallest.}//
\endgl
\xe

\ex
\begingl
\gla Jankám Marka tám-to zuštalébik ko Tereza//
\glb Janek-\Agt{} Marek-\Acc{} \Comp{}=\Rz{} be:happy-\Ben{}-\Pf{} \Lnk{} Tereza//
\glft \trsl{Tereza, whom Janek made happier than Marek}//
\endgl
\xe

\ex
\begingl
\gla Marka tám-tóví zuštalébik ko oblašc//
\glb Marek-\Acc{} \Comp{}=\Rz{}-\Gen{}= be:happy-\Ben{}-\Pf{} \Lnk{} pet//
\glft \trsl{the pet [of the person who was made happier than Marek]}//
\endgl
\xe

Iridian does not have a morphologically distinct superlative construction. For
example, \ird{pizdení} (from \ird{pizdá}, \trsl{to be big}) can either mean
\trsl{bigger} or \trsl{biggest} depending on context. Where the meaning cannot
be easily implied from context, the word \ird{ohnu} (derived from the word
\ird{ohna}, \trsl{first} in the instrumental case) is often used as quantifier.

\pex
\a
\begingl
\gla Univerzitet na razmeka pizdenou.//
\glb university \Loc{} city-\Acc{} be:big-\Comp{}-\Nz{}//
\glft \trsl{(This) university is the biggest in the city.}//
\endgl
\a
\begingl
\gla Univerzitet na razmeka ohnu pizdenou.//
\glb university \Loc{} city-\Acc{} first-\Ins{} be:big-\Comp{}-\Nz{}//
\glft \trsl{(This) university is the biggest in the city.}//
\endgl
\xe

When using an adverbial construction with the instrumental case to modify or
quantify the comparison, the adverbial phrase must immediately precede the
stative verb if in the attributive or nominal form, or the particle \ird{tám}
otherwise. The same is true with invariable modifiers like \ird{nahte},
\trsl{too much}, \ird{dnu}, \trsl{a bit}, etc.

\ex
\begingl
\gla To bagáž jánám u 10 kilográmu tám prékveví.//
\glb \Dem{}.\Prox{} baggage \Dem{}.\Med{}-\Agt{} around 10 kilogram-\Ins{} \Comp{}= heavy-\Cont{}//
\glft \trsl{This baggage is heavier by about 10 kilograms than that one.}//
\endgl
\xe

\ex
\begingl
\gla u 10 kilográmu prékvení bagáž//
\glb around 10 kilogram-\Ins{} heavy-\Comp{}-\Att{} baggage//
\glft \trsl{the baggage, which is heavier by about 10 kilograms}//
\endgl
\xe

\ex
\begingl
\gla Nahte pizdenou zmažnikóveš.//
\glb too:much big-\Comp{}-\Nz{} make-\Pv{}-\Pf{}-\Nz{}-\Second{}\Sg{}//
\glft \trsl{The much bigger one is the one you made.}//
\endgl
\xe

\section{Questions}\label{sec:questions-syntax}
\index{questions!syntax of}
\index{interrogative sentence|see{questions!syntax of}}

\subsection{Yes-no questions}\label{sec:questions-yesno}

A declarative sentence can be turned into a yes-no question by a simple rise in
intonation, as in English. Alternatively, the interrogative particle \ird{lí}
can be used. \ird{Lí} like all adverbial particles are proclitic and must
necessarily appear before the predicate. While \ird{lí} is optional, its
omission when forming a yes-no question would often be imply some level of
surprise or disbelief on the part of the speaker. Thus a sentence like
\irdp{Janek uzdravževí}{Janek is sleeping} can be transformed into a yes-no
question as either \irdp{Janek uzdravževí?}{Is Janek sleeping?} or \irdp{Janek
lí uzdravževí?}{Is Janek sleeping?} with the choice decided by the current
context. In the written language, especially in longer sentences or in formal
contexts, the interrogative particle is never omitted. The same is true with
copular sentences.

\pex
\begingl
  \gla Balžaróma Europevní Unijí lí čelina?//
  \glb Bulgaria European-\Att{} Union-\Gen{} \Q{} member//
  \glft \trsl{Is Bulgaria a member of the European Union?}//
\endgl
\xe

The particle \ird{děne} (v. \S\,\ref{sec:class3-particles}) is used to form
indirect questions, similar to the English \trsl{I wonder\ldots}. \ird{Děne} and
\ird{lí} may be used together in the same sentence, especially in the written
language, although in colloquial speech, \ird{lí} would often be dropped in such
cases. The use of \ird{děne} may sometimes be used to `soften' an otherwise
direct question by making it less direct.

\pex
\begingl
  \gla Marek na Praha děne lí že ščenžaní?//
  \glb Marek \Loc{} Prague-\Acc{} \Spec{} \Q{} \Pfv{} arrive-\Av{}-\Ret{}//
  \glft \trsl{I wonder if Marek has already arrived in Prague.}//
\endgl
\xe

The scope of a yes-no question is generally understood to be that of the whole
utterance, although the focus may be shifted to any constituent of the sentence,
subject to the usual rules for topicalization discussed in
\S\,\ref{sec:topicalization}.

\pex
\a\begingl
  \gla Stám Kovárž nevo séstu o leguánu děne lí hvaružnašách?//
  \glb mister Kovárž later convention-\Ins{} about iguana-\Ins{} \Spec{} \Q{} give:a:speech-\Av{}-\Ctp{}//
  \glft \trsl{Would Mr Kovárž be giving a speech about iguanas later at the convention?}//
  \endgl
\a\begingl
  \gla Stám Kovárž nevo séstu o leguánu děne lí hvaružnašáchou?//
  \glb mister Kovárž later convention-\Ins{} about iguana-\Ins{} \Spec{} \Q{} give:a:speech-\Av{}-\Ctp{}-\Nz{}//
  \glft \trsl{Would it be Mr Kovárž who would be giving a speech about iguanas later at the convention?}//
  \endgl
\xe

Yes-no questions are similarly formed from existential sentences by the use of
\ird{lí} and/or \ird{děne}. However, only \ird{ješ} accepts this transformation,
and a negative existential sentence would first need to be transformed into a
positive one before \ird{lí} and/or \ird{děne} can be added.

\pex
\a \begingl
  \gla Na to parka niho seh.//
  \glb \Loc{} \Dem{} park-\Acc{} \N{}\Exst{} flower//
  \glft \trsl{There are no flowers in this park.}//
  \endgl
\a \begingl
  \gla Na to parka lí ješ seh?//
  \glb \Loc{} \Dem{} park-\Acc{} \Q{} \Exst{} flower//
  \glft \trsl{Are there flowers in this park?}//
  \endgl
\xe

A sentence like \irdp{Na to parka lí niho seh?}{Aren't there any flowers in this
park?} is not grammatical.

Tag questions\index{tag question} may be formed by appending the phrase \irdp{lí
zám lět}{isn't it the truth?} or variants like \irdp{lí što zám lět}{is it
indeed the truth?} or \irdp{děne lí zám lět}{I wonder if it isn't the truth}
(cf. Russian\index{Russian} \textit{\cyrtext не правда ли}) to the end of the
sentence. In colloquial speech, these forms may be considered too formal or
old-fashioned and alternatives like \irdp{da}{yes,} \irdp{jó}{yeah,}
\irdp{let}{truth} or the Slavic borrowing \irdp{pravda}{truth} may be used
instead.

\pex
\begingl
\gla Traví kupšek, lí zám lět? /da?//
\glb bread-\Gen{} buy-\Av{}-\Pf{} \Q{} \Neg{} truth yes//
\glft \trsl{You bought some bread, didn't you? /right?}//
\endgl
\xe

Unlike in English, tag questions are invariable, i.e., sentences like
\trsl{Janek is studying, isn't he?} and \trsl{Janek studied yesterday, didn't
he?} can be translated as \ird{Janek znohouščime, da?} and \ird{Janek
znohoušček, da?} respectively, with both sentences having an identical tag part.

\subsection{Alternative questions}\label{sec:alternative-questions}
\index{alternative question}\index{choice question|see{alternative question}}

\subsection{Content questions}\label{sec:content-questions}
\index{wh-question@\emph{wh-}question}\index{information
question|see{\emph{wh}-question}}\index{content
question|see{\emph{wh}-question}}

Content questions, also known as \emph{wh}-questions, are formed using the
interrogative pronouns \irdp{jede}{who,} \irdp{ježe}{what,} \irdp{jena}{where,}
etc.\footnote{ A full list of interrogative pronouns can be found in
\S~\ref{sec:int-pron}. } Iridian requires the \emph{wh}-phrase to be moved to
the beginning of the sentence, thus causing it to occupy the topic position.
This \emph{wh}-fronting\index{wh-fronting@\emph{wh}-fronting} consequently
causes the voice of the main verb to be reframed to accomodate the new topic.
More commonly, especially colloquial Iridian\index{colloquial Iridian}, this
also means the nominalization\index{nominalization} of the main verb phrase,
essentially making the question an equational sentence.

\pex
\a\begingl
\gla Karel na Roubžení verštáta možlaševí.//
\glb Karel \Loc{} Roubže-\Gen{} suburbs-\Acc{} live-\Av{}-\Cont{}//
\glft \trsl{Karel lives in the suburbs of Roubže.}//
\endgl
\a\begingl
\gla Jena Karlám možlouneví? /možlounívou?//
\glb where Karel-\Agt{} live-\Lv{}-\Cont{} live-\Lv{}-\Cont{}-\Nz{} //
\glft \trsl{Where does Karel live?}//
\endgl
\xe

Alternatively, the element being questioned may be replaced with a question word
without changing the original word order, in which case the addition of the
clitic \ird{no} is required. Note that questions formed this way generally have
a more emphatic meaning.

\pex
\begingl
\gla Karel jena-no možlaševí?//
\glb Karel where=\Q{} live-\Av{}-\Cont{}//
\glft \trsl{Where did you say Karel lived?}//
\endgl
\xe


\emph{Wh}-fronting may sometimes cause peripheral elements of a phrase to be
moved together with the \emph{wh}-item to the beginning of the sentence, a
phenomenon linguists call `pied-piping' (\cite[263-4]{ross1967}). When this
occurs, Iridian is more conservative than English in that it usually keeps the
same question word instead of replacing it with a specialized one (in English,
normally, `which'); it may, however, use \irdp{jak}{which} if the expected
answer to the question is an element of a class, i.e., not unique. Consider, for
example, the two questions below:

%%%% zuscve, cf. Cz sodestvi, Ru. sodestvo
\pex
\a
\begingl
\gla Jena zuscve možlounívou?//
\glb where neighborhood live-\Lv{}-\Cont{}-\Nz{}//
\glft \trsl{Which (\emph{lit.,} where) neighborhood do you live in?}//
\endgl
\a
\begingl
\gla Jak kvartír možlounívou?//
\glb which apartment live-\Lv{}-\Cont{}-\Nz{}//
\glft \trsl{Which of these apartments is the you live in?}//
\endgl
\xe

In cases where there are multiple \emph{wh}-elements in the sentences, they are
normally all fronted, with the main question word first followed by the rest in
order of importance. Interestingly, too, any or all of the fronted
\emph{wh}-items may be pluralized with \ird{ně} if the speaker expects that the
answer is plural.

\pex
\a\begingl
\gla Jede ježe jena hloupškou?//
\glb who what where ask-\Av{}-\Pf{}-\Nz{}//
\glft \trsl{Who asked what where?}//
\endgl
\a\begingl
\gla Ně jede ježe jena hloupškou?//
\glb \Pl{}= who what where ask-\Av{}-\Pf{}-\Nz{}//
\glft \trsl{Which persons asked what where?}//
\endgl
\a\begingl
\gla Jede ně ježe jena hloupškou?//
\glb who \Pl{}= what where ask-\Av{}-\Pf{}-\Nz{}//
\glft \trsl{Who asked what things where?}//
\endgl
\xe

In the case of more complex \emph{wh}-questions involving the movement of a
  \emph{wh}-item from an embedded clause, Iridian is similar to
  Bulgarian\footnote{ \posscite{rudin1988} description on the nature of multiple
  \emph{wh}-fronting in Bulgarian\index{Bulgarian} involves the movement of the
  \emph{wh}-item to closest interrogative SpecCP, which does not necessarily
  need to occupy the topic position in the sentence. Compare, for example the
  following sentences in Bulgarian and Iridian.

  \ex[lingstyle=fnex,belowexskip=-1em,aboveglftskip=1pt]
  Bulgarian\index{Bulgarian} (\emph{ibid.,} 451)\smallskip\\
    \begingl 
    \gla Boris na kogo kakvo kaza [če šte {dade --- ---]}? //
    \glb Boris to whom what said that will give-\Third{}\Sg{}//
    \glft \trsl{What did Boris say that (he) would give to whom?}//
  \endgl
  \xe
\smallskip
  \ex[lingstyle=fnex,belowexskip=-1em,aboveglftskip=1pt]
    \begingl 
    \gla Ježe jehát Borisám ditnách to zíknou?//
    \glb what to:whom Boris-\Agt{} give-\Pv{}-\Ctp{} \Rz{} say-\Pv{}-\Pf{}-\Nz{}//
    \glft \trsl{What did Boris say that (he) would give to whom?}//
  \endgl
  \xe

} in requiring all the \emph{wh}-items to be fronted (cf.~\cite[450]{rudin1988}).

\pex
\begingl
\gla Ježe jehát dejatnách to zíknou?//
\glb what to:whom give-\Pv{}-\Ctp{} \Rz{} say-\Pv{}-\Pf{}-\Nz{}//
\glft \trsl{What did she say that she will give to whom?}//
\endgl
\xe


\subsection{Answering questions}\label{sec:ansyn}

Most yes-no questions may be answered by repeating the focal word or phrase in
the original question or echoing the syntax of the question itself.

\ex
\vtop{\halign{%
#\hfil& \qquad #\hfil\cr
\ird{---\,Kartuškí tak slouveževí?} & \trsl{\small Do they sell potatoes here?}\cr
\ird{---\,Slouveževí?} & \trsl{\small They do.}\cr
}}
\xe

Alternatively, the question may be answered by \irdp{da}{yes} or \irdp{ne}{no,}
both of which have been adapted from Common Slavic\index{Common Slavic}. In
colloquial speech it is also common to use \ird{já} or \ird{jó} for \trsl{yes}
(most likely borrowings from German\index{German}). These polarity words may be
used alone or in combination with the echo response. In general, the order does
not matter, although it is more common for the polarity word to appear after the
echo response. Unlike English \trsl{yes,} \ird{da} is used when confirming the
question posed by the speaker, whether or not it is in the affirmative or in the
negative. When denying or negating a question, Iridian uses \ird{ne} is used
when the original question was framed in the negative and \ird{ale} otherwise.

\ex
\vtop{\halign{%
#\hfil\hfil\cr
\ird{---\,Lošní Nolaní vilm ža oudnenik?}\cr
\ird{---\,Ža oudnenik, da. Má záčesčik.}\cr
\ird{---\,Ne, po zoudnenik.}\smallskip\cr
\trsl{Have you seen Nolan's new film?}\cr
\trsl{I've seen it, yes. But I didn't like it.}\cr
\trsl{No, I haven't seen it yet.}\cr
}}\xe

\ex\vtop{\halign{%
#\hfil\hfil\cr
\ird{---\,No daní trehlo za banka podarnílá cy Janek záléháček?}\cr
\ird{---\,Léháček, ale. Má avtem bych hebo.}\cr
\ird{---\,Záléháček, da.}\smallskip\cr
\trsl{Weren't you advised by Janek to submit your tax return to the bank?}\cr
\trsl{He did, yes. But my car broke down yesterday.}\cr
\trsl{No, he didn't advise me to.}\cr
}}
\xe

\ird{Da} (or sometimes \ird{a da}) may also preface answers to questions as a
form of intensifier, or to indicate that the speaker considers the answer to the
question as an obvious truth.

\ex
\vtop{\halign{%
#\hfil& \qquad #\hfil\cr
\ird{---\,Na muzla ješ vdenikou.} & \trsl{I saw someone at the mall today.}\cr
\ird{---\,Jede?} & \trsl{Who?}\cr
\ird{---\,Da Janek.} & \trsl{Well, Janek, of course.}\cr
}}\xe

The answer does not need to be positive for \ird{da} or \ird{a da} to be used.

\ex\vtop{\halign{%
#\hfil\hfil\cr
\ird{---\,Šabatu de koncerta stožit?}\cr
\ird{---\,A da ne. To kapela šem záčesčeví.}\smallskip\cr
\trsl{Are you coming to the concert on Saturday?}\cr
\trsl{Well no, I don't even like that band.}\cr
}}
\xe


As for questions involving existential constructions


\section{Negation}\label{sec:negation}
\index{negation}

In Iridian sentences, negation is performed by the particle \ird{zám}, which
attaches to the beginning of the word or phrase  it negates. The default
position of the negative particle is before the main verb where it surfaces as
\ird{z-} before vowels, \ird{ž-} before \emph{i}-glides, and \ird{zá-}
elswehere. This elision does not occur where \ird{zám} appears elswehere in
the sentence.

\pex
\a
\begingl
    \gla Janek Martina Markám {zá}hévoržébik.//
    \glb Janek Martin-\Acc{} Marek-\Agt{} \Neg{}know-\Ben{}-\Pf{}//
    \glft \trsl{Marek did not introduce Janek to Martin.}//
\endgl
\a
\begingl
    \gla {Zám} Janek Martina Markám hévoržébik.//
    \glb \Neg{} Janek Martin-\Acc{} Marek-\Agt{} know-\Ben{}-\Pf{}//
    \glft \trsl{It was not Janek whom Marek introduced to Martin.}//
\endgl
\a
\begingl
    \gla Janek {zám} Martina Markám hévoržébik.//
    \glb Janek \Neg{} Martin-\Acc{} Marek-\Agt{} know-\Ben{}-\Pf{}//
    \glft \trsl{It was not Martin whom Marek introduced Janek to.}//
\endgl
\a
\begingl
    \gla Janek Martina {zám} Markám hévoržébik.//
    \glb Janek Martin-\Acc{} \Neg{} Marek-\Agt{} know-\Ben{}-\Pf{}//
    \glft \trsl{It was not Marek who introduced Janek to Martin.}//
\endgl
\xe

\ird{Zám} attaches directly to the word or phrase it negates, although it is
also common, especially in spoken Iridian, to append the clitic \ird{-te} after
the word being negated by \ird{zám} to provide more emphasis on the negation.
This is a fairly recent development, and is not found in older texts or in the
written language.

\pex
\a
\begingl
    \gla {Zám} Janek{-te} Martina Markám hévoržébik.//
    \glb \Neg{} Janek=\Foc{} Martin-\Acc{} Marek-\Agt{} know-\Ben{}-\Pf{}//
    \glft \trsl{It was not Janek whom Marek introduced to Martin.}//
\endgl
\a
\begingl
    \gla Janek {zám} Martina{-te} Markám hévoržébik.//
    \glb Janek \Neg{} Martin-\Acc{}=\Foc{} Marek-\Agt{} know-\Ben{}-\Pf{}//
    \glft \trsl{It was not Martin whom Marek introduced Janek to.}//
\endgl
\a
\begingl
    \gla Janek Martina {zám} Markám{-te} hévoržébik.//
    \glb Janek Martin-\Acc{} \Neg{} Marek-\Agt{}=\Foc{} know-\Ben{}-\Pf{}//
    \glft \trsl{It was not Marek who introduced Janek to Martin.}//
\endgl
\xe

The different constituents of the sentence can be negated simultaneously; thus,
for example, the sentence below is grammatically permitted:

\pex
\begingl
    \gla {Zám} Janek {zám} Martina {zám} Markám {zá}hévoržébik.//
    \glb \Neg{} Janek \Neg{} Martin-\Acc{} \Neg{} Marek-\Agt{} \Neg{}-know-\Ben{}-\Pf{}//
    \glft \trsl{It was not Janek who was not introduced to someone who is not Martin by someone who is not Marek.}//
\endgl
\xe

Nonetheless, due to their general unwieldiness, forms like this are extremely
rare (both in the spoken and the written language), with preference given to
single and double negation instead. Since \ird{-te} can only appear in a
sentence once, where there are more than one negate constituent in a sentence,
\ird{-te} is appended to the element which has the most significance (usually
the topic); or, if there are two constituents negated and one of them is the
main verb, \ird{-te} is appended to that other element.

\pex
\begingl
    \gla {Zám} Janek{-te} Martina Markám {zá}hévoržébik.//
    \glb \Neg{} Janek=\Foc{} Martin-\Acc{} Marek-\Agt{} \Neg{}know-\Ben{}-\Pf{}//
    \glft \trsl{It was not Janek who was not introduced to Martin by Marek.}//
\endgl
\xe

Alternatively, if there is only one element/phrase negated in the sentence other
than the main verb (which itself may or may not be negated), it is common,
especially in colloquial Iridian\index{colloquial Iridian}, to
nominalize\index{nominalization} the whole verb phrase and transform the
sentence into a copular construction\index{copular construction}, with the
negated phrase as the new topic\index{topic} and the nominalized verb phrase as
the predicate\index{predicate}.

\pex
\begingl
    \gla Zám jájka na Praha zadačkou.//
    \glb \Neg{} daughter-\Dim{} \Loc{} Prague-\Acc{} move-\Av{}-\Pf{}-\Nz{}//
    \glft \trsl{It was not my daughter who moved to Prague.}//
\endgl
\xe




\section{Existential constructions}\index{existential construction}
\label{sec:exst}

\subsubsection{In general}
An existential sentence is a specialized construction used to express the
existence or presence of someone or something. The particle \ird{ješ} and its
inverse \ird{niho} are used to form existential sentences. 
\begin{multicols}{2}
\pex
\a\begingl
\gla Tak ješ zarno.//
\glb here \Exst{} people//
\glft \trsl{There are people here.}//
\endgl
\a\begingl
\gla Tak niho zarno.//
\glb here \N{}\Exst{} people//
\glft \trsl{There is no one here.}//
\endgl
\xe
\end{multicols}

The existential construction in Iridian was originally a
locative\index{locative} one, and this could still be seen in how the use of
\ird{ješ} and \ird{niho} requires both the noun or noun phrase whose existence
is posited and the location where such existence is said to be true to be
explicitly present in the sentence. In true existential sentences (e.g.,
\trsl{There is a God} or \trsl{There is still hope}) where the argument is the
existence of something and not just it's mere presence somewhere, the patientive
form of the reflexive verb \ird{se}, \ird{sní}, is used. In addition, where this
ostensible location is present in the sentence, it would occupy the
topic\index{topic} position\footnote{Although this location (often surfacing as
a \ird{na} clause) appears where the topic of the sentence normally would, it
would be more correct to analyze an existential construction as an inversion of
the regular topic-predicate word order in Iridian. Viewed this way, we can think
of \ird{ješ} or \ird{niho} as a pseudoverb, and the phrase consisting of the
first half of the sentence and ending with this pseudoverb is the predicate
while the unmarked second half is the topic. This approach has the benefit of
keeping the predicate with a verb-final internal word order and the topic as
unmarked, both in accordance with the basic rules of Iridian syntax; however,
this does not account for the use of the dummy \ird{sní} in true existential
clauses.} in the sentence, and unlike in regular sentences, must be explicitly
marked in the patientive.\index{patientive}


\begin{multicols}{2}
\pex
\a\begingl
\gla \ljudge{*}Ješ tieho.//
\glb \Exst{} god//
\glft \trsl{There is a God.}//
\endgl
\a\begingl
\gla Sní ješ tieho.//
\glb \Refl{}.\Acc{} \Exst{} god//
\glft \trsl{There is a God.}//
\endgl
\xe
\end{multicols}

The use of \ird{sní} as a placeholder is not required however if the noun or
noun phrase whose existence is the subject of the sentence is quantified, either
by a numeral or otherwise by an indefinite quantifier.

Statements expressing location use a copular construction, although an
existential construction may be used in the negative to convey an absence of
something, with the normal negative construction used where emphasis on one
element of the sentence is desired by the speaker.

\pex
\begingl
\gla Dá na duma.//
\glb \First{}\Sg{} \Loc{} house-\Acc{}//
\glft \trsl{I'm at home.}//
\endgl
\xe

\pex
\a\begingl
\gla Na duma niho dá.//
\glb \Loc{} house-\Acc{} \N{}\Exst{} \First{}\Sg{}//
\glft \trsl{I'm not at home.}//
\endgl
\a\begingl
\gla Zám dá na duma//
\glb \Neg{} \First{}\Sg{} \Loc{} house-\Acc{}//
\glft \trsl{It is not I who's at home.}//
\endgl
\a\begingl
\gla Dá zám na duma//
\glb \First{}\Sg{} \Neg{} \Loc{} house-\Acc{}//
\glft \trsl{I'm not at home (i.e., I'm somewhere else).}//
\endgl
\xe

The particles \ird{ješ} and \ird{niho} generally proceeds the noun or noun
phrase whose existence is being posited, but in the case of a modified noun or
noun phrase, the existential particle appears before all modifiers. On the other
hand, numerals or indefinite quantifiers appear before the existential particle.

\pex
\begingl
\gla Na duma men ješ mulaž.//
\glb \Loc{} house-\Acc{} two \Exst{} door//
\glft \trsl{There are two doors.}//
\endgl
\xe

\pex
\begingl
\gla Na ránema hroná ješ matematickí tóm.//
\glb \Loc{} desk-\Acc{} three \Exst{} mathematics book//
\glft \trsl{There are three mathematics books on my desk.}//
\endgl
\xe




\subsubsection{Possession}
Existential constructions are also used to indicate possession, with the
possessor marked in the patientive case.

\begin{multicols}{2}
\pex
  \begingl
    \gla Marka ješ oblašc.//
    \glb Marek-\Acc{} \Exst{} pet//
    \glft \trsl{Marek has a pet.}//
  \endgl
\xe
\pex
  \begingl
    \gla Tomáša niho mlaz.//
    \glb Tomáš-\Acc{} \N{}\Exst{} brother//
    \glft \trsl{Tomáš does not have a brother.}//
  \endgl
\xe
\end{multicols}

\subsubsection{Impersonal constructions}\index{impersonal construction}

Iridian prefers using existential constructions where English\index{English} and
other Indo-European languages would use indefinite pronouns. More formally,
sentences of this type are called impersonal constructions.\footnote{See, for
example, \textcite{lawtagalog} where the discussion in this section is largely
based.} In general an impersonal construction in Iridian is produced by
nominalizing\index{nominalization} a verb phrase which would otherwsise have
been the predicate of an indefinite pronoun. We can illustrate this in English
as follows:

\pex
\a  \deftagex{impeng}\deftaglabel{ind}Sentence with an indefinite pronoun as subject:\\
    \emph{Somebody} told me to come here to pick up the dress.
\a  Impersonal construction:\\
    \ljudge{?}\emph{There is somebody} who told me to come here to pick up the dress.
\xe

Sentences of the first type do not exist in Iridian. Instead sentences with an
indefinite element (not necessarily the subject of the sentence) are reframed as
existential constructions. To further illustrate the primacy of impersonal
constructions over indefinite pronouns in Iridian, we can replace the subject of
(\getfullref{impeng.ind}) with a definite noun:

\pex
\a\begingl
    \gla Tak muž nedvačernilá te Tereza ziček.//
    \glb here dress \Caus{}-get-\Pv{}-\Subj{}.\Ipf{} so:that Tereza say-\Av{}-\Pf{}//
    \glft \trsl{Tereza told me to come here to pick up the dress.}//
  \endgl
\a\begingl
    \gla Do ješ tak muž nedvačernilá te zičkou.//
    \glb \First{}\Sg{}.\Acc{} \Exst{} here dress \Caus{}-get-\Pv{}-\Subj{}.\Ipf{} so:that say-\Av{}-\Pf{}-\Nz{}//
    \glft \trsl{Somebody told me to come here to pick up the dress.} (\emph{Lit.,} I have someone who said (I) should come pick up the dress.)//
  \endgl
\xe


\pex
\begingl
\gla Martina ješ trešnikou na tropa.//
\glb Martin-\Acc{} \Exst{} write-\mk{pv-pf-nz} \Loc{} wall-\Acc{}//
\glft \trsl{Martin wrote something on the wall.}//
\endgl
\xe

\pex
\begingl
\gla Voštnikouva ža ješ piaščkou?//
\glb cook-\mk{pv-pf-nz-pat} already \Exst{} eat-\Av{}-\Pf{}-\Nz{}//
\glft \trsl{Did somebody eat what (I) cooked?}//
\endgl
\xe

\section{Copular constructions}
\subsubsection{Null copula}

Copular sentences are a minor sentence type where the predicate is not a verb.
For the purposes of this grammar, we narrow down our definition of copular
constructions to the following:
\pex
\a \textit{Equative:} Marek is the doctor (we are talking about).
\a \textit{Inclusive:} Marek is a doctor.
\a \textit{Attributive:} Marek is tall.
\a \textit{Locative:} Marek is in the hospital.
\xe

Iridian does not make a distinction between equative, inclusive and attributive
clauses. Locative clauses on the other hand, may be expressed using a copular or
an existential construction, as will be discussed in this section.

Iridian is a superficially a zero-copula language and the most common way to
form copular sentences is mere juxtaposition.

\pex<cop>
\begingl
\gla Marek doktor.//
\glb Marek doctor//
\glft \trsl{Marek (is a/the) doctor.}//
\endgl
\xe

The above example could either be taken to mean (1) Marek is a doctor
(inclusive), or (2) Marek is the doctor (equative). Generally, though, Iridian
uses word order to distinguish between equative and inclusive clauses.

\pex
\a \textit{Inclusive:} \{item in class\}\tss{N} $\varnothing$ \{class\}\tss{P}
\a \textit{Equative:} \{class\}\tss{N} $\varnothing$ \{item class\}\tss{P}
\xe

To avoid ambiguity, Example \getref{cop} can be reformulated to either of the
following sentences:

\pex<cop1>
\a
\begingl
\gla Marek doktor.//
\glb Marek doctor//
\glft \trsl{Marek is a doctor.}//
\endgl

\a
\begingl
\gla Doktor Marek.//
\glb doctor Marek//
\glft \trsl{Marek is the doctor.}//
\endgl

\xe

The inversion of word order is not strongly grammaticalized with NP-NP
sentences, i.e., both sentences in Example \getref{cop1} can still be used
interchangeably without a change in meaning and preference is given on the one
over the other when there is an ambiguity. This is not the case with attributive
clauses, i.e., sentences with adjective or adjective phrase predicates. Consider
for example the sentence below:

\pex
\begingl
\gla Marek rázym.//
\glb Marek tall//
\glft \trsl{Marek is tall.}//
\endgl
\xe

Inverting the word order of the sentence above would change the adjective to a
substantive since modifiers cannot occupy the topic position.

\pex
\begingl
\gla Rázym Marek.//
\glb tall Marek//
\glft \trsl{The tall one is Marek.}//
\endgl
\xe

Iridian also distinguishes between attributive clauses expressing permanent
conditions and clauses expressing temporary conditions, with the latter being
expressed using existential constructions in certain adjectives.

\pex
\begingl
\gla *Marek morec.//
\glb Marek hungry//
\glft \trsl{Marek is hungry}//
\endgl
\xe


\pex
\begingl
\gla Marka ješ morec.//
\glb Marek-\Acc{} \Exst{} hunger//
\glft \trsl{Marek is hungry}//
\endgl
\xe

A full list of adjectives/modifiers that use the existential construction can be
found in the section~\ref{sec:exst}.

The copula, however, cannot be ommitted in grammatical moods other than the
indicative.

\subsubsection{Negative copula}

Iridian has the negative copula \ird{česná}.

\pex
\begingl
\gla Marek doktor česná.//
\glb Marek doctor \Cop{}.\Neg{}//
\glft \trsl{Marek is not (a/the) doctor.}//
\endgl
\xe

The inversion of word order may also be used when one wants to avoid ambiguity:

\pex
\begingl
\gla Doktor Marek česná.//
\glb doctor Marek \Cop{}.\Neg{}//
\glft \trsl{Marek is not the doctor.}//
\endgl
\xe


\subsubsection{Conjugation paradigm}

\chapter{Semantics and usage}

\section{Register}
\section{Politeness and forms of address}\index{form of address}\index{politeness}\label{sec:politaddr}

\subsection{Politeness and formality in Iridian}\index{politeness}\index{formality}

Although not as complex and as pervasive as the politeness/formality system found in Japanese\index{Japanese} or Korean\index{Korean}, Iridian formally encodes\index{markedness} more sociolinguistic information than its neigbouring languages such as Czech\index{Czech} or Hungarian\index{Hungarian}.

Broadly speaking, Iridian distinguishes between three levels of speech\index{level of speech}:\footnote{
	The English names are of course imperfect. It would perhaps be more correct,---if not more illustrative of their differences,---to call the polite speech level \emph{formal} and the formal speech level \emph{honorific}. What we call above as polite is more close to what linguists would call `formal' mainly because the strategy is one of distance and not deference. Moreover, although the formal speech level may be used to signal respect and shows a strong tendency to use honorifics and titles, the main usage remains that of showing an even greater detachment on the part of the speaker than would have otherwise have been possible when using the polite speech level.
} (1)~\emph{polite} speech, which serves more or less as the ``default'' level of politeness, as this is the speech level most often used by, say, strangers when talking to each other; (2)~\emph{formal} speech, which is used in more formal settings, where the speaker wants to distance themself from the listener or explicitly signal their politeness, such as in a conversation among business associates or when talking to a divinity; and (3)~\emph{casual} speech, which is used between close friends and family members, or to or among children. These levels of speech are not definite, of course, and politeness is more properly viewed as a spectrum (cf., e.g., \cite{hansonjap}) as speakers would often switch from one level of speech to another even when speaking to the same person, or within a single conversation.

The distinction between politeness\index{politeness} (which for the purpose of this grammar we can define as the psychological or social distance between speakers) on the one hand, and formality (which we can define as situational distance) on the other, is not always one made (or kept) in Iridian. Indeed, more often than not, these categories are often viewed by most speakers as essentially being the same. This is further complicated by the fact that the distinction between the various speech levels is not morphologically marked\index{markedness} but is facilitated instead by the preference for certain constructions and forms of address\index{form of address}.


The choice of which speech level to use with which speaker and in which scenarios is influenced by a lot of factors. It would be helpful, however, to analyse these factors as being influenced by two main considerations: the relationship---more specifically, the familiarity,---between the speakers, and the social setting in which the conversation or interaction is taking place.\footnote{
	One could take a look as well at the dimensions (or `semantics,' to use the authors' term) that influence the formality/politeness distinctions made in a language, proposed by \textcite{browngilman} in their study on the development of second-person pronouns and address forms. Although on the surface, the politeness distinction in Iridian is not dual, we see (as discussed \emph{infra}) that we can in fact classify the speech levels as either familiar (T) and distant (V). Where most Indo-European languages, however, predicate this distinction on the power semantic (i.e., the T-V distinction is made initially when a speaker of one power group speaks to a member of another), Iridian bases this initially on the solidarity semantics, thus creating a T-V distinction first when there is no solidarity (perceived or otherwise) between speakers, and only secondarily on the basis of the power semantic.
}

The first consideration, the relationship between speakers, divides the levels of speech into two groups: {\sc familiar speech}\index{familiar speech}, which consists of the casual speech level and {\sc distant speech}\index{distant speech} which consists of both the formal and polite speech levels. This distinction is perhaps of greater actual importance than that introduced earlier between the levels of speech, as the differences between familiar speech and distant speech are more pronounced than the differences between formal speech and polite speech, which are often more subtler. Distant speech is characterized by a preference to indirect speech acts where possible. For example, direct imperatives or prohibitives are virtually unused in distant speech, replaced instead with hortative constructions, or in more formal situations with questions or optative constructions. Consider for example the following:

\pex
\a Imperative in familiar speech:\\
\ird{Mina návilastním.} \trsl{Open the door!}
\a Alternative constructions in distant speech:
\beginsubsub
  \b{-} {Neutral, using the hortative:\\
  	\ird{Mina návilastniká.} \trsl{Please open the door.}}
  \b{-} {More polite, using \ird{am luhninká}:\\
  	\ird{Mina se návilastu am luhninká}\\\trsl{May (you) be thanked because the door was closed.}}
  \b{-} {More formal and more polite, using a question:\\
  	\ird{Mina návilastníš to mužnali\v{t}?}\\
  	\trsl{Is it possible that the door will be closed?}
  	}
\endsubsub
\xe

Perhaps a direct consequence of this preference for indirect speech acts over direct ones is the strategy of {\sc pronoun avoidance} so heavily employed in distant speech. Pronoun avoidance as it applies to Iridian include not only \posscite[371--2]{velupillai2012} narrow definition of it as the omission and sometimes replacement with a title or other form of address of a pronoun, when addressing or referring to a person, but also the indirect result of Iridian's heavy reliance on context and the resulting tendency to drop elements of the sentence when they can be easily inferred, including pronouns.

In general, familiar speech is indifferent on the use of personal pronouns, with the use or omission dictated by context and not by politeness/formality. Thus both of the following sentences are equally probable in familiar speech:

\pex\a \ird{Avtem bych hebo.} \trsl{My car broke down yesterday}\deftagex{pronavoidance}\deftaglabel{withpron}
	\a \ird{Avt bych hebo.} \trsl{(My) car broke down yesterday}\deftaglabel{sanspron}
\xe

In distant speech, however, sentence (\getfullref{pronavoidance.withpron}) would be largely avoided, or even considered disrespectful or incorrect. When speaking in the polite speech level, the omission of the personal pronoun is often enough; in the formal speech level, especially in writing, this is often complemented by the explicit addition of a referent honorific, even when the context is clear.

\pex
	\a Casual and polite speech:\\
		\ird{Marek záščenžévnik. Avt ce bych hebo.}\footnote{
			The ethical dative as seen in this example is emphatic and can be used in both casual and polite speech.
		}\\
		\trsl{Marek couldn't come yesterday. His car broke down.}
	\a Formal speech:\\
		\ird{Stám Zakár záščenžévnik. Stámí avt bych hebo.}\\
		\trsl{Mr Zakár couldn't come yesterday. His car broke down.}
\xe

The persistence of pronoun avoidance means a person's title or an equivalent honorific will be used in formal speech even when addressing that person directly. Nevertheless, when addressing a listener directly, the formal speech level does allow the use of the distal animate demonstrative \ird{dní} (a stand-in for the third person pronoun, since Iridian does not have one); this is parallel in the polite speech level which allows the use of the second person plural pronoun \ird{tová}\footnote{The use of the plural \ird{tová} has perhaps the closest Iridian is to a true T-V distinction.} in direct addresses. Both ultimately correspond to the use of the second person singular pronoun \ird{já} in casual speech. The use (or omission) of any of these pronouns is as always dependent on actual context.

\pex
	\a Formal speech, using honorifics:\\
	\ird{Stám Zakár bych záščenžévnice to kvušček. Stám jevitébílá te ceščeví?}\\
	\trsl{I heard you were not able to come yesterday. Would you like me to catch you up on what happened?}\medskip\\
	Formal speech, using \ird{dní}:\\
	\ird{Stám Zakár bych záščenžévnice to kvušček. Dní jevitébílá te ceščeví?}\\
	\trsl{I heard you were not able to come yesterday. Would you like me to catch you up on what happened?}\footnote{Note that even when using \ird{dní} instead of honorifics, a honorific would still be used when addressing the listener for the first time, and only on subsequent occurences would the substitution be made.}
	\a Polite speech, using \ird{tová}:\\
	\ird{Tová bych záščenžévnice to kvušček. Jevitébílá te ceščeví?}\\
	\trsl{I heard you were not able to come yesterday. Would you like me to catch you up on what happened?}
	\a Casual speech, using \ird{já}:\\
	\ird{Já bych záščenžévnice. Jevitébílá te ceščeví?}\\
	\trsl{I heard you were not able to come yesterday. Would you like me to catch you up on what happened?}
\xe

The use of bare honorifics\index{honorific} instead of an actual formal/polite second person may seem unwieldy at first, but it is in fact not uncommon. We see similar systems, for example in European Portuguese\index{Portuguese} and Tagalog\index{Tagalog}.

\pex
	\a European Portuguese\index{European Portuguese|see{Portuguese}}\index{Portuguese}
		\beginsubsub
			\b{-}{Explicit V form, honorific used:\\
				\foreign{O senhor sabe onde é que está?} \trsl{Do you know where you are?}}
			\b{-}{Implicit V form, pronoun omitted:\\
				\foreign{Sabe onde é que está?} \trsl{Do you know where you are?}}
			\b{-}{Superficially an explicit V form, but may be interpreted as informal or even rude\footnote{
				Cf. \textcite{laraport}. The peculiar nature of \foreign{voc\^e} is European Portuguese (EP) is quite interesting. Whereas in Brazilian Portuguese\index{Brazilian Portuguese|see{Portuguese}} (BP) \foreign{voc\^e} has almost completely displaced \foreign{tu} as the prevalent T form, in EP it occupies a linguistic limbo between \foreign{tu} (T) and \foreign{o senhor/a senhora} (V), leading to it having quite disparate uses depending on the speaker and the dialect.

				Etymologically, \foreign{voc\^e} shares the same historical development as the Spanish\index{Spanish} \foreign{usted}. They are syncopated versions of the original forms of address \foreign{vossa merc\^e} and \foreign{vuestra merced}, respectively, both of which translate to \trsl{your mercy/grace}. The original pronouns \foreign{vossa/vuestra} persist in both language but are no longer the standard V forms, supplanted instead by developments from the forms of address originally containing them. (Cf., e.g., \cite{hummelport}, which provides an extensive analysis of the diachronic development of both the Spanish \foreign{usted} and the Portuguese \foreign{voc\^e}.)
	
				In most Spanish\index{Spanish} dialects, \foreign{usted} remains the standard V form. In Portuguese\index{Portuguese}, however, \foreign{voc\^e} has itself been supplanted by another form of address used as a pronoun, \foreign{o senhor/a senhora}. In BP, this change coincided (or perhaps caused) \ird{voc\^e} to change from being an intermediate V form to the default T form, with \foreign{tu} (the original T form) and \foreign{vós} (the original V form, and later, intermediate T form) falling out of use. In EP\index{Portuguese}, on the other hand, the T forms (both the original \foreign{tu} and the intermedaite \foreign{vós}) were retained and instead it is \foreign{voc\^e} that fell out of use. (This historical shift of the V form displacing the existing T form and the consequent loss of this original T form, and the grammaticalization of a polite form of address as a new V form, is quite common; in Rioplatense Spanish, one of the more divergent dialects of Spanish\index{Spanish}, for example, we see the V$\rightarrow$T shift started by the grammaticalization of \foreign{usted} completed by the eventual displacement of the original T form \foreign{t\'u} with the original V form \foreign{vos} as the prevalent T form. What is interesting in EP, however, is that V$\rightarrow$T shift was completed, not by the displacement of the original T form \foreign{tu}, but by the loss---or more properly, \emph{obsolescence}---of the intermediate V form \foreign{voc\^e}).

				\foreign{Voc\^e} remains, superficially at least in EP\index{Portuguese}, a V form (cf. \cite[85]{ganhoport}); its actual use, however, is not as clearly defined. As \textcite{laraport} remarks, `not even Portuguese speakers agree in determining the contexts where it can be employed.' The most important development in modern EP\index{Portuguese} with regards to the use of \foreign{voc\^e} is that of conveying anger, sarcasm or annoyance, especially in asymmetric relations, similar to the older \foreign{vós}, the use of V forms between speakers who normally would use T forms to indicate annoyance (cf. \cite{hummelport}). This has led to the ambiguous use of \foreign{voc\^e} both as a polite and an impolite form of address.

				Although this V$\rightarrow$T shift does not directly reflect Iridian's own historical development, it is helpful to understand the fluidity and the inherent arbitrariness of T-V labels in any language. In Iridian, too, this V of annoyance exists marginally both between people who regularly use T forms (i.e., familiar speech) with each other, to indicate sarcasm or displeasure; and those who use V forms (i.e., distant speech) among themselves, to openly signal disrespect.
			}:\\
				\foreign{Voc\^e sabe onde é que está?} \trsl{Do you know where you are?}}
			\b{-}{Explicit T form:\\
				\foreign{(Tu) sabes onde é que estás?} \trsl{Do you know where you are?}}
		\endsubsub
	\a Tagalog\index{Tagalog}
		\beginsubsub
			\b{-}{Explicit V form, 2nd person plural:\\
				\foreign{Alam ba ninyo kung nasaan kayo?} \trsl{Do you know where you are?}}
			\b{-}{Explicit V form, 3rd person plural:\\
				\foreign{Alam ba nila kung nasaan sila?} \trsl{Do you know where you are?}}
			\b{-}{Explicit T form. 2nd person singular:\\
				\foreign{Alam mo ba kung nasaan ka?} \trsl{Do you know where you are?}}
		\endsubsub
\xe

The preference in distant speech for indirect speech acts is also manifested in the extensive use of 

\subsection{Forms of address, titles, and honorifics}\index{honorific}\index{terms of courtesy|see{honorific}}\index{courtesy|see{honorific}}\index{form of address}

A {\cscaps honorific} is a form of address used to indicate respect or courtesy. The most common honorifics in
Iridian are the masculine \ird{Stám} equivalent to the English\index{English} \trsl{Sir} and the feminine
\ird{Nau} equivalent to the English \trsl{Madame/Ma'am.} When addressing a person of an unknown
gender\index{gender}, the term \irdp{Obečne}{mercy/grace} is used.

Both \ird{Stám} and \ird{Nau} may be followed by the addressee's last name. They should never be used with
the first name as it would be considered sarcastic or rude. In writing, these are abbreviated as \ird{S.}
and \ird{N.}, respectively. If the name of the person being addressed is not
known, the placeholders \irdp{vieda}{man} and \irdp{huzak}{woman} are used, thereby producing \ird{Stám Vieda}
and \ird{Nau Huzak}. When writing, these are often abbreviated to {\sc s.v.} and {\sc n.h.}, respectively.
The usage of \ird{Stám Vieda} and \ird{Nau Huzak} is similar to how the third person may sometimes be used
in English\index{English} to politely address someone (e.g., saying, \trsl{Will the gentleman yield?}) but while
it may sometimes appear dated or overly formal in English\index{English}, this practice is still commonly observed
in Iridian, especially when addressing strangers.

Other common titles include \ird{Doktor} used when addressing physicians, \ird{Majestet} or \ird{Kopižnást}
when addressing members of the royalty (with the latter reserved for reigning monarchs), \ird{Eselenc} when
addressing certain high-ranking officials such as senators, governors, and ambassadors, \ird{Eminenc} when addressing
cardinals of the Catholic Church, \ird{Obečne} or \ird{Prac} when addressing judges and magistrates, and
\ird{Tiehožnást} or \ird{Hildažnást} or \ird{Hildení Tá\v{t}}\footnote{This form of address, meaning
\trsl{Holy Father} or more commonly its abbreviation {\sc h.t.}, is used in writing when referring to the Pope in the
third person.} when addressing the Pope or the religious leaders from other traditions.

When addressing or referring to multiple individuals the term \ird{maše} (originally meaning \trsl{crowd}
but now exclusively employed as a honorific) is used. This is often preceded, both in the written and spoken forms,
by the non-nominal supine\index{supine} \irdp{prehodašce}{esteemed/praiseworthy.}

\subsection{Salutations and valedictions in the written language}\index{salutation}\index{valediction}\index{written correspondence}

The general salutation in most formal correspondence uses the honorific \irdp{Stám}{Sir} or \irdp{Nau}{Madame}.
The last name of the addressee may also follow, although more often than not, the simple honorific\index{honorific} 
should suffice. When addressing a collegiate entity or a collection of people, the term \irdp{Maše}{crowd} or
\irdp{Prehodašce maše}{Esteemed/praiseworthy crowd} is used instead.

If the addressee holds a specific title, the title is included in the salutation. In some cases, the wife of the
title-holder may be addressed using \ird{Nau} followed by the title, although this practice is slowly falling out
of use, except in most diplomatic correspondence, where it is still considered standard. Below are some examples:


\begin{itemize}[nosep]
	\item \irdp{Stám/Nau Prezident}{Mister/Madame President}
	\item \irdp{Stám/Nau Brac}{Mister/Madame Member of the Parliament}
	\item \irdp{Stám/Nau Kanclár}{Mister/Madame Chancellor}
	\item \irdp{Stám/Nau Holva}{Mister/Madame Chairman/Chairwoman}
	\item \irdp{Stám/Nau Provízor}{Mister/Madame Professor}
\end{itemize}

Where the addressees are multiple individuals who hold specific titles, the honorific \ird{Stám} or \ird{Nau} is replaced with \irdp{prehodašce}{esteemed, praiseworthy}. When used this way, the title is normally not capitalised. Note also that \ird{prehodašce} will only be used in a salutation when there are multiple addressees.

\begin{itemize}[nosep]
	\item \irdp{Prehodašce brac}{Esteemed members of the Parliament}
	\item \irdp{Prehodašce provízor}{Esteemed members of the faculty}
\end{itemize}

When the addressee is a medical doctor, the salutation \irdp{Doktor}{doctor} is used. When writing to members of the clergy, it is customary to use \irdp{Pápka}{My father} or \irdp{Mlazka}{My brother.}

It is considered rude to use a person's first name by itself in the salutation. A more common way is to add the suffix \irdp{-óm}{our} or \irdp{-(e)m}{my} to the name or the diminutive form of the name. Alternatively the terms \irdp{kamarád}{colleague, comrade} or \irdp{naž}{friend} or their diminutives may also be used. This approach is particularly common in e-mail correspondence between work colleagues.

Standard valedictions used in formal written correspondence\index{correspondence|see{written correspondence}} in Iridian tend to be more complex than the ones used in English. Below is 

\begin{itemize}[nosep]
	\item \irdp{(Stám/Nau) oblostnení mavac/respekt akceptirniká}{Sir/Madame, please accept my sincerest regards (\emph{lit.}, wishes)/respect.}
	\item \irdp{Dá zespodení/spietnení pokárí biležit}{I will remain your most humble/loyal servant.}
	\item \irdp{Dá zespodení/spietnení bylí biležit}{I will remain your most humble child.}\footnote{This is often used among religious people when writing to members of the clergy.}
	\item \irdp{Oblostnení mavacu/respektu še hroznik.}{With the sincerest regards/respect has this letter been sent.}

\end{itemize}

Increasingly, especially in e-mail\index{e-mails} correspondence, it has become more common to use the following valedictions instead:

\begin{itemize}[nosep]
	\item \irdp{Mavac/\v{S}e mavacu}{Regards/with wishes/regards.}
	\item \irdp{Oblostnení}{Most sincere}
\end{itemize}

In more informal situations, such as between close friends and family, the following are used:

\begin{itemize}[nosep]
	\item \irdp{Dá}{I/me}
	\item \irdp{Bes/Mach bes/Nic bes}{Hug/Two hundred hugs/A thousand hugs}
	\item \irdp{Beska/Mach beska/Nic beska}{Little hug/Two hundred little hugs/A thousand little hugs}
	\item \irdp{\v{S}e hloubu/Hloubževí}{With love/Loving}
	\item \irdp{\v{Z}už/Mach žuž/Nic žuž}{Kiss/Two hundred kisses/A thousand kisses}
\end{itemize}

As mentioned earlier, specific examples of written correspondence in Iridian can be found in \S\,\ref{sec:writcorr}.


\section{Phatic Expressions and Social Formulas}

\section{Idiomatic Expressions}\index{idiomatic expressions}

\section{Punctuation}


\appendix
\part*{Appendices}
\chapter{Spoken Iridian}

\section{Introduction}


\chapter{The Dialects of Iridian}


\section{Dialects Outside of Iridia}

\subsection{Ukrainian Dialects}

The Ukrainian dialects of Iridian (\ird{hokránževní mieva}), known locally as \textit{гу\-кр\'ан\-же\-вн\'и мни\-ва} is spoken in the borderlands of Ukraine and Iridia. It forms a dialect continuum with the southeastern dialects of the country and is the dialect with the most number of speakers outside the country. Within the dialect group itself, variations can be observed from the forms spoken from one town to another, mainly because until very recently the language had no official status in Ukraine\footnote{It was recognized as a minority language in 2003.} and its relative isolation from the mainstream dialects of Iridian made it take a path of its own.

As could be expected from its location, this group of dialects has had significant influence from the Ukrainian language (and to a lesser extent, from Russian) and its vocabulary contains more Slavic-derived words than Standard Iridian. These dialects are also written entirely in the Cyrillic script (based on the Iridian Cyrillic alphabet with some spelling conventions adapted from Ukrainian) although in recent years the use of Latin alphabet is becoming more common (primarily due to the rise of text messaging and the internet).

Phonologically, the Ukrainian dialects are perhaps the most divergent. One of the most notable difference is the replacement of nasal vowels with a sequence of an oral vowel and a nasal consonant homorganic with the following stop, or if the nasal vowel was in coda, with an oral vowel and /m/. This also meant that coda /m/ and /n/ no longer nasalizes the preceding vowel (and diphthongize it in case it is an /a/ or an /e/).

\ex
Loss of nasal vowels in Ukrainian Iridian:\\
\irdp{bi\k{e}c}{cat}\quad$\rightarrow$\quad\ird{б\'инц} [bʲiːnt͡s]\\
\irdp{l\k{a}ca}{flatiron}\quad$\rightarrow$\quad\ird{ланца} [ˈlänt͡sɐ]\\
\irdp{bž\k{e}}{bee}\quad$\rightarrow$\quad\ird{бжем} [bʑɛm]
\xe

\ex
Non-nasalization of vowels before /m/ and /n/:\\
\irdp{bi\k{e}cem}{my cat}[ˈbʲɛ̃w̃t͡sə̃w̃]\quad$\rightarrow$\quad\ird{б\'инцем} [bʲiːnt͡sɪm]\\

\xe

\chapter{Lexicon}

\section{Kinship Terms}\label{sec:kinship terms}\index{kinship terms}
\subsection{Nuclear Family}\label{sec:nuclear family}

The diminutive\index{diminutive} form of the nouns relating to the nuclear family are presented here as well since, as discussed in \S\,XX, it is common to use the diminutive instead of the regular form of nouns when to referring to one's own family or that of a socially close one (e.g., a friend's).

%% TODO add section reference above

\begin{table}[h!]
  \caption{Kinship terms, nuclear family.}
  \label{tbl:kinship}
  \footnotesize\sffamily
  \begin{tabu} to 0.8\textwidth{YYY[1.5]}
  \toprule
  {\sc noun}    & {\sc diminutive} & {\sc translation}\\
  \midrule
  \ird{ploc}    & \ird{pluška } & family\\
  \ird{hor}    & \ird{horka}       & parents\\
  \ird{maty}    & \ird{mámka}     & mother\\
  \ird{táty}  & \ird{pápka}     & father\\
  \ird{hrešt}    & \ird{hrištka}     & sibling\\
  \ird{mlaz}    & \ird{mlažka}     & brother\\
  \ird{vod}    & \ird{vodka}     & sister\\
  \ird{proud}    & \ird{prudka}     & oldest sibling/child\\
  \ird{zneibo}    & \ird{zníbka}     & youngest sibling/child\\
  \ird{rohoš}    & \ird{ru{z}ka}     & son\\
  \ird{jaja}    & \ird{jájka}     & daughter\\
  \ird{vremou}    & \ird{vremóvka}     & child\\

  \bottomrule

  \end{tabu}
\end{table}



\section{Place Names}

\subsection{European Countries and Capitals}
{\footnotesize
\begin{longtabu} {ZZZZZ}
\toprule
    {\sc country}             & {\sc trsnslation}      & {\sc capital}       & {\sc demonym}       & {\sc adjective}\\
\midrule
    Albania             & Albánie     & Tirana        & albanice      & albanevní\\
    Andorra             & Andóra      & Andóra la Vella & andórževnice & andórževní\\
    Austria             & Ježiróma& Vína        & ježirževnice & ježirževní\\
    Belarus             & Bielaruz      & Minsk         & bielaruščevnice & bielaruščevní\\
    Belgium             & Belžóma & Brushla       & belževnice & belževní\\
    Bosnia and Herzegovina  & Bošna a Hercegovina   & Sarajevo  & bošnevnice & bošnevní\\
    Bulgaróma         & Bulháróma & Sofía       & bulhárvenice    & bulhárevní\\
    Croatia             & Horvacema     & Zahreb        & horvacevnice  & horvacevní\\
    Czech Republic      & \v{C}estóma & Prah          & češv{c}evnice & češv{c}evní\\
    Denmark             & Dancema       & Kudiena       & dancevnice    & dancevní\\
    Estonia             & Hištuna   & Tálim       & hištunevnice  & hištunevní\\
    Finland             & Vínžóma & Helsinki    & vínževnice    & vinževní\\
    France              & Vranca        & Pariž     & vrancevnice   & vrancevní\\
    Georgia             & Hroužema  & Tablise       & hrouževnice   & hrouževní\\
    Germany             & Némiecema   & Berlim        & némiecevnice    & némiecevní\\
    Greece              & Hiržóma & Atína       & hirževnice    & hirževní\\
    Hungary             & Mažaróma    & Budapešt  & mažarevnice   & mažarevní\\
    Iceland             & Išlám   & Rejkjavik     & išlevnice & išlevní\\
    Ireland             & Irlám       & Doublina      & irlevnice     & irlevní\\
    Italy               & Itálie      & Ruma          & italevnice    & italevní\\
    Latvia              & Lutišema  & Rika          & lutiščevnice  & lutiščevní\\
    Liechtenstein       & Liktánštán & Vaduz    & liktánštevnice & liktánštevní\\
    Lithuania           & Litóma      & Vilnius       & litevnice     & litevní\\
    Malta               & Malta         & Valeta        & malčevnice    & malčevní\\
    Moldova             & Moldávie    & Kišiniev  & moldaževnice  & moldaževní\\
    Monaco              & Monáko      & Monákoštát    & monacevnice   & monacevní\\
    Montenegro          & Sodoví Mel  & Podgorica     & sodovímlevnice  & sodovímlevní\\
    Netherlands         & Kuzní Prava & Amsterdam     & nerlanževnice & nerlanževní\\
    Northern Macedonia  & Roce Makedóma   & Skopie    & (roce)makedonževnice & (roce)makedonževní\\
    Norway              & Nurváž  & Ušla      & nurževnice    & nurževní\\
    Poland              & Pulžóma & Varšáva & polščevnice   & polščevní\\
    Portugal            & Portugál    & Ližbánie    & portoževnice  & portoževní\\
    Romania             & Rumiena       & Buhurešt  & rumínevnice & rumínevní\\
    Russia              & Ružóma  & Mošhou    & ruščevnice & ruščevní\\
    San Marino          & Samarino      & Samarino      & samarinevnice & samarinevní\\
\bottomrule
\end{longtabu}}

\section{Medical Terms}
\subsection{Parts of the Body}
\chapter{Sample Texts}

\section{The \emph{Pater Noster}}

\section{Milan Kundera, \ttla{A Kidnapped West or the Tragedy of Central Europe}}

{\small
% NOTE ON THE TRANSLATION
The translation is based on the French text of Kundera's essay \ttla{Un Occident kidnappé: ou la tragédie de l'Europe centrale} first published in \ttlb{Le Débat} in 1983. The full text is available online at various websites, with the link I used in the references. Due to copyright considerations, a translation has not been provided, although interlineal glosses and explanatory notes have been added where  I believe they are needed, in addition to the lexicon at the end. The text itself contains its own footnotes however and to distinguish Kundera's notes from those I have added, I have included included his name at their end.
}

\begin{center}1.\end{center}

1956 svemí Septembru Mažarevní Znova Byróví direktorám, byró
nastolám jednočnil ko obiení vniho minutu, ruščevnie uráž
po Budapešta šelčice cy télexu laska mieta kudní expedica
pashvalébik. Expedice to nie neitu uhožnek: >> Mé za Mažaróma a
za Evropa shražach<<.

Nie neite ježe-no prónesčeví? Mažaróma a še laska Evropu ruščevní šarám zbavujinalu to žvotu prónesčeví. Ma Evropa zbavujinale to ježe prónestu?

Ma žená --- >>za bláha a za Evropa shražá<< --- to že Leningrada že Mušhóva závadnéteví to neite, ma če je Budapešta, če je Varšáva.

\begin{center}2.\end{center}
Vade, Evropa-te ježe-no za ona mažarevna, ona češčevna, ona polščevna?


\section{From the \emph{Little Prince}}

\subsection{Text and translation}

Za Léon Wertha

Tóm za dousa hledniš to bylám množniká.


\subsection{Glosses}

\pex
	\begingl
	\gla Tóm za dousa hledniš to bylám množniká.//
	\glb book for adult-\Acc{} dedicate-\Pv{}-\Sbj{}.\Pf{} \Rz{} child-\Agt{} forgive-\Pv{}-\Hort{}//
	\glft \trsl{I apologise to children for having dedicated this book to a grown-up.}//
	\endgl
\xe

\section{Written Correspondence}\label{sec:writcorr}\index{written correspondence}

\subsection{Formal Business Letter}
{\small
\begin{flushright}
	Roubže\\
	2019\,h. Mercí 14.\,r.
\end{flushright}

\noindent{Marek Zakár}\\
Ledeman Direkt {\sc m/h}\\
Husplac, \textnumero{} 177\\
Osthalbár\\
86332 Roubže {\sc rb}
}

\subsection{Formal E-mail}




\subsection{Informal Letter}

\chapter{A Brief History of Iridia}

\section{Pre-history and the Indo-European migrations}

\section{Under Roman rule}

Parts of what is now Iridia belonged to the Roman province of Pannonia, which was established in 9 AD. Old-Iridian-speaking peoples were concentrated in the cities of Aquila (modern-day Roubže), Missia, and Carnuntum. The Roman province of Pannonia was divided into two parts in 285 AD, with the eastern part becoming the province of Dacia Ripensis. The western part was renamed Pannonia Superior, and the eastern part Pannonia Inferior. The Roman province of Pannonia was finally abolished in 395 AD, when the Roman Empire was divided into the Western Roman Empire and the Eastern Roman Empire.

As with much of the Roman lands, the relationship between the native Iridians and the Romans have generally been amicable, with the Romans often being seen as liberators from the Germanic tribes. The concept of an ``Iridian nationhood'' has not yet been developed, and the Iridians have generally been seen as a collection of tribes, with the Romans often referring to them as ``the Iridians'' rather than ``the Iridian people''. The Romans were also the first to use the term ``Iridia'' to refer to the region, and it was not until the 19th century that the term ``Iridia'' was used to refer to the Iridian people. 


\section{The first Iridian kingdoms}


\section{Principality of Iridia}



\section{The first Iridian Republic}

\section{World War II}

\section{The Iridian People's Republic and communist rule}

\section{The fall of communism}

The twelve-year reign of Enta has seen a continuous decline in the country's GDP and standard of living. Throughout the 1970s the IPR is generally considered one of the poorest countries in Europe.

\section{Iridia today}






\cleardoublepage
\nocite{*}
\printbibliography
\addcontentsline{toc}{chapter}{Bibliography}
\markboth{Bibliography}{Bibliography}

\cleardoublepage
\printindex
\addcontentsline{toc}{chapter}{Index}
\markboth{Index}{Index}

\end{document}
