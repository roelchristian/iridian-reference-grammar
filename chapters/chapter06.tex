\chapter{Syntax of Simple Clauses}

\section{Introduction}

The constituent word order of Iridian sentences is SOV, but the agglutinative nature of the language and the presence of case-marking on nouns makes word order typically flexible, with the only universal rule being that the main verb should appear at the end of a sentence.

\section{Topic-Predicate Constructions}\index{topic}\index{predicate}\label{sec:topic-pred}
The Iridian sentence can be divided primarily into a topic part and a predicate
or comment part. The topic is what the sentence is about, while the predicate or comment represents the information presented in the sentence about the topic. While both the topic and the predicate are pragmatic constructs, the topic-predicate construction is important as it determines how the rest of the sentence is structured.


\begin{figure}[H]
  \begin{forest}
    [S,
      [{\sc top}] [{\sc pred}]]
  \end{forest}
  \caption{Nuclear structure of sentences}
  \label{}
\end{figure}

The topic of the sentence does not necessarily coincide with the subject of the sentence. This is true as well in English, as we see in example (\ref{ex:engtop}); although where English allows the topic to appear anywhere in the sentence, as long as the subject is placed first, Iridian, typical of topic-prominent languages. requires the topic to always be introduced first, leaving the rest of the information afterwards.


\pex\label{ex:engtop}
\a Martha saw John.
\a A dog bit \emph{Martha}.
\a It is raining \emph{today},
\xe


\pex

\a
\begingl
\gla \relax[Janek]\tss{\mk{Top}} [mlaza boulešik.]\tss{\mk{Pred}}//
\glft \trsl{As for Janek, he killed his brother}.//
\endgl

\a
\begingl
\gla \relax[Tereza]\tss{\mk{Top}} [jecám nalečnik.]\tss{\mk{Pred}}//
\glft \trsl{As for Tereza, she was bitten by a dog}//
\endgl

\a
\begingl
\gla \relax[Shléd]\tss{\mk{Top}} [zniepšalí.]\tss{\mk{Pred}}//
\glft \trsl{As for today, it is raining.}//
\endgl

\xe


More importantly, the topic of the sentence determines how the main verb, and thus all the other constituents of the sentence, are marked.

\pex
\a
\begingl
\gla Tereza jec\textbf{ám} naleč\textbf{n}ik.//
\glft \trsl{As for Tereza, she was bitten by a dog}//
\endgl

\a
\begingl
\gla Jec Terez\textbf{e} nalč\textbf{eš}ik.//
\glft \trsl{As for the dog, it bit Tereza.}//
\endgl

\xe

As \textcite[9]{kiss2004} notes:

\begin{quote}
  We tend to describe eventsfrom a human perspective, as statements about theirhuman participants – and subjects are more often {\sc[+human]} than objects are. Inthe case of verbs with a {\sc[–human]} subject and a {\sc[+human]} accusative or obliquecomplement,  the  most  common  permutation  is  that  in  which  the  accusative  oroblique complement occupies the topic position\,[.] When the possessor is theonly human involved in an action or state, the possessor is usually topicalized[.]
\end{quote}

\section{The Noun Phrase}

Iridian is a strongly head-final language.

\subsection{Nuclear constructions}
\subsection{With adjectival clauses}
\subsection{Wsith prepositional phrases}
\subsection{With relative clauses}

\section{Topicless Sentences}\label{sec:topicless}\index{topicless sentence}

\section{Definiteness}\index{definiteness}\label{sec:definiteness}

Iridian lacks a specific class of articles\index{articles} such as English
\trsl{a} or \trsl{the} to mark the opposition between definite and indefinite
nouns. For example, the word \ird{jec} can mean both \trsl{a dog} or \trsl{the
dog} depending on the context (or in some environments the same word can be
interpreted as \trsl{dogs,} \trsl{some dogs} or \trsl{the dogs}).

A common way to specificy the definiteness of a noun is to promote it to the
topic position in the sentence. As discussed in \S\,\ref{sec:topic-pred}, the
topic of a sentence must be specific and referential, and therefore it is often,
but not always, definite. Consider for example the two sentences below.

\begin{multicols}{2}
  \pex
  \a
  \begingl
  \gla Pitár pižmo.//
  \glb Pitár farmer//
  \glft \trsl{Pitar is \textbf{a} farmer.}//
  \endgl
  \a
  \begingl
  \gla Pižmo Pitár.//
  \glb farmer Pitár//
  \glft \trsl{Pitár is \textbf{the} farmer.}//
  \endgl
  \xe
\end{multicols}

This can be extended to non-copular constructions.

\begin{multicols}{2}
  \pex
  \a
  \begingl
  \gla Vliče štanžice.//
  \glb milk-\Gen{} drink-\Av{}-\Pf{}-\Quot{}//
  \glft \trsl{(I) drank some milk.}//
  \endgl
  \a
  \begingl
  \gla Vliko štanimce.//
  \glb milk drink-\Pv{}-\Pf{}-\Quot{}//
  \glft \trsl{(I) drank the milk.}//
  \endgl
  \xe
\end{multicols}

If the topic is quantified\index{quantifier} by a numeral\index{numeral}, indefiniteness can be expressed by nominalizing\index{nominalization} the main verb and promoting it to topic.

\pex
\a
\begingl
\gla Jaro okrád za propozica niebidček.//
\glb five district for proposal-\Pat{} vote:against-\Av{}-\Pf{}//
\glft \trsl{The five districts voted against the proposal.}//
\endgl
\a
\begingl
\gla Za propozica niebidečkou jaro okrád.//
\glb for proposal-\Pat{} vote:against-\Av{}-\Pf{}-\Nz{} five district//
\glft \trsl{Five districts voted against the proposal.}//
\endgl
\a
\begingl
\gla Za propozica niebidečkou ko okrád jaro.//
\glb for proposal-\Pat{} vote:against-\mk{av-pf-nz} \mk{rz} district five//
\glft \trsl{Five is the number of districts that voted against the proposal.}//
\endgl
\xe

The number one (\ird{oní})

\pex
\a
\begingl
\gla Tóm onaževí.//
\glb book be:lost-\mk{cont}//
\glft \trsl{The book is missing.}//
\endgl
\a
\begingl
\gla Oní tóm onaževí.//
\glb one book be:lost-\mk{cont}//
\glft \trsl{One of the books is missing.}//
\endgl
\a
\begingl
\gla Onaživou pní tóm.//
\glb one book be:lost-\mk{cont}//
\glft \trsl{One of the books is missing.}//
\endgl
\a
\begingl
\gla Onaživou pní tóm.//
\glb one book be:lost-\mk{cont}//
\glft \trsl{One of the books is missing.}//
\endgl
\xe


Note that this rule is not universal and the topic of a sentence does not necessarily have to be definite, especially where the sentence is merely expressing a fact or a general truth:

\pex
\begingl
\gla Jec hvárem.//
\glb dog animal//
\glft \trsl{Dogs are animals.}//
\endgl
\xe



\pex
\begingl
\gla To >>jec<< hvárem že: to robot//
\glb \Dem{} dog animal \mk{ncop} \Dem{} robot//
\glft \trsl{The ``dog'' is not a real animal but a robot.}//
\endgl
\xe


\section{Relative and Comparative Constructions}\label{relativecomparative}\index{comparative constructions}

The clitic\index{clitic} \ird{tám} is used to form simple comparative and relative constructions. \ird{Tám} is often ommitted however where the comparison can be implied from context. In this construction, the standard of comparison (the noun preceded by `than' in English) is unmarked and the noun being compared marked in the agentive if it is a positive/negative comparison, or in the instrumental if it is a correlation.


\pex
\a
\begingl
\gla Janek-tám Markám nestačál.//
\glb Janek=\Comp{} Marek-\Agt{} tall-\mk{cont}//
\glft \trsl{Marek is taller than Janek.}//
\endgl
\a
\begingl
\gla Janek Markám nestačál.//
\glb Janek Marek-\Agt{} tall-\mk{cont}//
\glft \trsl{Marek is taller than Janek.}//
\endgl
\xe

\pex
\a
\begingl
\gla Janek-tám Marku nestačál.//
\glb Janek=\Comp{} Marek-\Ins{} tall-\mk{cont}//
\glft \trsl{Marek is as tall as Janek.}//
\endgl
\a
\begingl
\gla Janek Marku nestačál.//
\glb Janek Marek-\Ins{} tall-\mk{cont}//
\glft \trsl{Marek is as tall as Janek.}//
\endgl
\xe

Note that \ird{tám} can only be used with the copulative form of the stative verb, as the attributive and nominal forms have separate conjugated comparative forms. When using these forms, however, the standard of comparison is marked in the genitive. In relative constructions, the instrumental is also replaced with the genitive, but the modifier \ird{zní}, \trsl{same} is added before the stative verb.

\pex
\a
\begingl
\gla Jancie nestačení hloc mlazem.//
\glb Janek-\Gen{} tall-\mk{comp-att} boy brother-\First{}\Sg{}//
\glft \trsl{The boy who is taller than Janek is my brother} (\emph{Lit.,} \trsl{The taller-than-Janek boy is my brother.})//
\endgl
\a
\begingl
\gla Jancie zní nestačení hloc mlazem.//
\glb Janek-\Gen{} same tall-\mk{comp-att} boy brother-\First{}\Sg{}//
\glft \trsl{The boy who is as tall as Janek is my brother.}//
\endgl
\xe

\ird{Tám} can be relativized by appending the clitic \ird{to}. When used with \ird{tám-to} the standard of comparison is marked in the patientive case. The use of tám-to in relative clauses is discussed in further detail in the next chapter.

\ex
\begingl
\gla Viktor na shlopa tám-to nestáček.//
\glb Viktor \Loc{} siblings-\Pat{} \mk{comp=rz=} to:be:tall-\Av{}-\Pf{}//
\glft \trsl{Among the siblings, Viktor grew up to be the tallest.}//
\endgl
\xe

\ex
\begingl
\gla Jankám Marka tám-to zuštalébik ko Tereza//
\glb Janek-\Agt{} Marek-\Pat{} \mk{comp=rz=} to:be:happy-\Ben{}-\Pf{} \Att{} Tereza//
\glft \trsl{Tereza, whom Janek made happier than Marek}//
\endgl
\xe

\ex
\begingl
\gla Marka tám-tóví zuštalébik ko oblašc//
\glb Marek-\Pat{} \mk{comp=rz-gen=} to:be:happy-\Ben{}-\Pf{} \Att{} pet//
\glft \trsl{the pet [of the person who was made happier than Marek]}//
\endgl
\xe

Iridian does not have a morphologically distinct superlative construction. For example, \ird{pizdení} (from \ird{pizdá}, \trsl{to be big}) can either mean \trsl{bigger} or \trsl{biggest} depending on context. Where the meaning cannot be easily implied from context, the word \ird{ohnu} (derived from the word \ird{ohna}, \trsl{first} in the instrumental case) is often used as quantifier.

\pex
\a
\begingl
\gla Univerzitet na razmeka pizdenou.//
\glb university \Loc{} city-\Pat{} to:be:big-\mk{comp-nz}//
\glft \trsl{(This) university is the biggest in the city.}//
\endgl
\a
\begingl
\gla Univerzitet na razmeka ohnu pizdenou.//
\glb university \Loc{} city-\Pat{} first-\Ins{} to:be:big-\mk{comp-nz}//
\glft \trsl{(This) university is the biggest in the city.}//
\endgl
\xe

When using an adverbial construction with the instrumental case to modify or quantify the comparison, the adverbial phrase must immediately precede the stative verb if in the attributive or nominal form, or the particle \ird{tám} otherwise. The same is true with invariable modifiers like \ird{nahte}, \trsl{too much}, \ird{dnu}, \trsl{a bit}, etc.

\ex
\begingl
\gla To bagáž jánám u 10 kilográmu tám prékvál.//
\glb \Dem{}.\Prox{} baggage \mk{dem.med} around 10 kilogram-\Ins{} \mk{comp=} heavy-\mk{cont}//
\glft \trsl{This baggage is heavier by about 10 kilograms than that one.}//
\endgl
\xe

\ex
\begingl
\gla u 10 kilográmu prékvení bagáž//
\glb around 10 kilogram-\Ins{} heavy-\mk{comp-att} baggage//
\glft \trsl{the baggage, which is heavier by about 10 kilograms}//
\endgl
\xe

\ex
\begingl
\gla Nahte pizdenou zmažnikóveš.//
\glb too:much big-\mk{comp-nz} make-\mk{pv-pf-nz-2s}//
\glft \trsl{The much bigger one is the one you made.}//
\endgl
\xe

\section{Questions}\index{questions!syntax of}\index{interrogative sentence|see{questions!syntax of}}

There are two  main  categories  of  interrogative  sentences in Iridian: yes-no  and  question-word questions (or \emph{wh-} questions).

\subsection{Yes-no questions}\index{questions!yes-no}\index{questions!syntax of}

A declarative sentence can be made into a question by a simple rise in intonation at the end of the phrase:

\pex
\a
\begingl
\gla Janek sa uzdravšek.//
\glb Janek already \Refl{}-sleep-\Av{}-\Pf{}//
\glft \trsl{Janek has fallen asleep.}//
\endgl
\a
\begingl
\gla Janek sa uzdravšek?//
\glb Janek already \Refl{}-sleep-\Av{}-\Pf{}//
\glft \trsl{Has Janek fallen asleep yet?}//
\endgl
\xe

Alternatively the interrogative clitic \ird{no} may be used. When used this way, the base sentence will still feature a clause-final rise in intonation, followed by a falling intonation at the location of the question particle, similar to the intonation structure of tag questions in English. In the written language, the particle \ird{no} may also surface as a clitic, prefixing itself to the verb, which this usage requires to be in the negative.

\pex
\begingl
\gla Janek sa uzdravšek no?//
\glb Janek already \Refl{}-sleep-\Av{}-\Pf{} =\Q{}//
\glft \trsl{Has Janek fallen asleep yet?}//
\endgl
\xe

\pex\deftagex{formalq}
\begingl
\gla Janek sa nozáduzdravšek?//
\glb Janek already \Q{}=\Neg{}=\Refl{}-sleep-\Av{}-\Pf{}//
\glft \trsl{Has Janek fallen asleep yet?}//
\endgl
\xe

The choice between using a simple rise in intonation or the question particle \ird{no} is a personal one, and a speaker may use the one or the other in different situations or shift between them seemingly at random. Both methods in free variation and offer no differences in meaning, formality, etc. However the form in (\getfullref{formalq}) is extremely formal and archaic and rarely (if ever) appears in the spoken language.

Tag questions are formed by affixing the word \irdp{l\'e\v{t}}{truth,} at the end of the sentence. The tag cannot be used with the question particle \ird{no}.

\pex
\begingl
\gla Janek sa uzdravšek, l\'e\v{t}?//
\glft \trsl{Janek has fallen asleep already, right?}//
\endgl
\xe

Although the particle \ird{no} would normally appear after the verb, it can follow other parts of the sentence (except pure function words), but with the effect of changing the emphasis or the nature of the question. When used in this manner, the particle is separated from the word it modifies by a dash. Furthermore, there is a tendency especially in the spoken language to move the cliticized noun to the start of the sentence.

\pex
\a
\begingl
\gla Ivána-no niehu scenžach?//
\glb Ivána=\Q{} later-\Ins{} arrive-\Av{}-\Ctp{}//
\glft \trsl{Is it Ivána who is coming later?}//
\endgl
\a
\begingl
\gla Ivána niehu-no scenžach?//
\glb Ivána later-\Ins{}=\Q{} arrive-\Av{}-\Ctp{}//
\glft \trsl{Will it be later that Ivana is coming?}//
\endgl
\a
\begingl
\gla Niehu-no Ivána scenžach?//
\glb later-\Ins{}=\Q{} Ivána arrive-\Av{}-\Ctp{}//
\glft \trsl{Will it be later that Ivana is coming?}//
\endgl
\xe

To make an existential sentence\index{existential constructions} a yes-no question, it is first transformed to the negative and the particle \ird{no} is then attached to the word \ird{niho}. If however, the theme of the sentence is quantified, the word \ird{ješ}\index{ješ} is kept (but shifted to the front of the quantifier), and \ird{no} is attached to the quantifier. The form \ird{ješ-no} is ungrammatical.

\pex
\begingl
\gla Marka niho-no oblašc?//
\glb Marek-\Pat{} \N{}\Exst{}=\Q{} pet//
\glft \trsl{Does Marek have a pet?}//
\endgl
\xe

\pex
\a
\begingl
\gla Co bibliotécie Marka hroná ješ kupéninkou tóm?//
\glb \mk{abl} library-\Gen{} Marek-\Pat{} three \Exst{} borrow-\mk{pv-pf-nz} book//
\glft \trsl{Marek borrowed three books from the library.}//
\endgl
\a
\begingl
\gla Co bibliotécie Marka ješ hroná-no kupéninkou tóm?//
\glb \mk{abl} library-\Gen{} Marek-\Pat{} \Exst{} three\mk{=q} borrow-\mk{pv-pf-nz} book//
\glft \trsl{Did Marek borrow three books from the library.}//
\endgl
\xe

The clitic \ird{no} can of course be moved around, with subtle changes in meaning.

\pex
\a \emph{Neutral form:}\\
\ird{Co bibliotécie Marka ješ hroná-no kupéninkou tóm?}\\
\trsl{Did he borrow \emph{three} books, etc?}
\a \emph{Emphasis on \emph{Marek:}}\\
\ird {Co bibliotécie Marka-no hroná ješ kupéninkou tóm?}\\
\trsl{Did \emph{Marek} borrow them, etc?}
\a \emph{Emphasis on \emph{library:}}\\
\ird {Co bibliotécie-no Marka hroná ješ kupéninkou tóm?}\\
\trsl{Did he borrow them from the \emph{library}, etc?}
\xe

Note that in more complex existential constructions, as the one above which includes a nominalized determiner, the sentence may have to be reconstructed as a non-existential construction if it is the theme (i.e., the object being possessed or whose existence is described) that is in question.

\pex
\a
\begingl
\gla Co bibliotécie Marek hroná tóma kupénžek?//
\glb \mk{abl} library-\Gen{} Marek three book-\Pat{} borrow-\Av{}-\Pf{}//
\glft \trsl{Did Marek \emph{borrow} three books from the library.}//
\endgl
\a
\begingl
\gla Co bibliotécie Marek hroná tóma-no kupénžek?//
\glb \mk{abl} library-\Gen{} Marek three book-\mk{pat=q} borrow-\Av{}-\Pf{}//
\glft \trsl{Did Marek borrow three \emph{books} from the library.}//
\endgl
\xe

%explain further the preference
Note that the first example above is not the neutral word order, given Iridian's preference to use existential constructions in sentences like the ones above. In this case, it would be akin to asking \trsl{Did he borrow them, or did he acquire it by some other means?}


\subsection{\textit{Wh-} questions}\index{wh- questions}\index{information question|see{wh-questions}}\label{sec:ynquestions}
In wh- questions, the interrogative pronoun typically appears after the topic or at the beginning of a sentence if the sentence does not have a topic, and is immediately followed by the clitic \ird{no}.

\pex
\begingl
\gla Karel jena-no možlašál?//
\glb Karel where=\Q{} live-\Av{}-\Cont{}//
\glft \trsl{Where does Karel live?}//
\endgl
\xe

\pex
\begingl
\gla Bych zajehu-no kravnašalí?//
\glb yesterday why=\Q{} cry-\Av{}-\Prog{}//
\glft \trsl{Why was he crying yesterday?}//
\endgl
\xe


\subsection{Echo questions}
\subsection{Indirect questions}

Indirect questions are constructed in the subjunctive, with the addition of the particle \ird{aš}.

\pex
\begingl
\gla Nú aš hošezíla.//
\glb tomorrow \mk{q.ind} rain\mk{av-sbj.ipf}.//
\glft \trsl{I wonder if it's gonna rain tomorrow.}//
\endgl
\xe

\subsection{Answering questions}\label{sec:ansyn}

Most yes-no questions may be answered by repeating the focal word or phrase in the original question or echoing the syntax of the question itself.

\ex
\vtop{\halign{%
#\hfil& \qquad #\hfil\cr
\ird{---\,Kartu\v{s}k\'i tak slouve\v{z}ev\'i?} & \trsl{\small Do they sell potatoes here?}\cr
\ird{---\,Slouve\v{z}ev\'i?} & \trsl{\small They do.}\cr
}}
\xe

Alternatively, the question may be answered by \irdp{da}{yes} or \irdp{ne}{no,} both of which have been adapted from Common Slavic\index{Common Slavic}. In colloquial speech it is also common to use \ird{j\'a} or \ird{j\'o} for \trsl{yes} (this is most likely a borrowing from German\index{German}). These polarity words may be used alone or in combination with the echo response. In general, the order does not matter, although it is more common for the polarity word to appear after the echo response. If the original question was framed in the negative and the response is positive, the contrastive \irdp{ale}{yes} is used instead.

\ex
\vtop{\halign{%
#\hfil\hfil\cr
\ird{---\,Lo\v{s}n\'i Nolan\'i vilm \v{z}a oudnenik?}\cr
\ird{---\,\v{Z}a oudnenik, da. M\'a z\'a\v{c}es\v{c}ik.}\smallskip\cr
\trsl{\small Have you seen Nolan's new film?}\cr
\trsl{\small I've seen it, yes. But I didn't like it.}\cr
}}\xe

\ex\vtop{\halign{%
#\hfil\hfil\cr
\ird{---\,Dan\'i trehlo za banka podarn\'il\'a to-\v{z}e Janek z\'al\'eh\'a\v{c}ek?}\cr
\ird{---\,Leh\'a\v{c}ek, ale. M\'a avtem bych hebo.}\smallskip\cr
\trsl{Weren't you advised by Janek to submit your tax return to the bank?}\cr
\trsl{He did, yes. But my car broke down yesterday.}\cr
}}
\xe

\ird{Da} (or sometimes \ird{a da}) may also preface answers to questions as a form of intensifier, or to indicate that the speaker considers the answer to the question as an obvious truth.

\ex
\vtop{\halign{%
#\hfil& \qquad #\hfil\cr
\ird{---\,Na muzla je\v{s} vdenikou.} & \trsl{I saw someone at the mall today.}\cr
\ird{---\,Jede?} & \trsl{Who?}\cr
\ird{---\,Da Janek.} & \trsl{Well, Janek, of course.}\cr
}}\xe

The answer does not need to be positive for \ird{da} or \ird{a da} to be used.

\ex\vtop{\halign{%
#\hfil\hfil\cr
\ird{---\,\v{S}abatu de koncerta sto\v{z}it?}\cr
\ird{---\,A da ne. To kapela \v{s}em z\'a\v{c}es\v{c}ev\'i.}\smallskip\cr
\trsl{Are you coming to the concert on Saturday?}\cr
\trsl{Well no, I don't even like that band.}\cr
}}
\xe


As for questions involving existential constructions


\section{Negation}\index{negation}

In Iridian sentences, negation is performed by the particle \ird{zám}, which attaches to the beginning of the word or phrase  it negates. The default position of the negative particle is usually before the main verb where it surfaces as \ird{z-} before vowels, \ird{ž-} before \emph{i}-glides, and \ird{zá-} elswehere.

\pex
\a
\begingl
    \gla Janek Martina Markám \textbf{zá}hévoržébik.//
    \glb Janek Martin-\Pat{} Marek-\Agt{} \Neg{}know-\Ben{}-\Pf{}//
    \glft \trsl{Marek did not introduce Janek to Martin.}//
\endgl
\a
\begingl
    \gla \textbf{Zám} Janek Martina Markám hévoržébik.//
    \glb \Neg{} Janek Martin-\Pat{} Marek-\Agt{} know-\Ben{}-\Pf{}//
    \glft \trsl{It was not Janek whom Marek introduced to Martin.}//
\endgl
\a
\begingl
    \gla Janek \textbf{zám} Martina Markám hévoržébik.//
    \glb Janek \Neg{} Martin-\Pat{} Marek-\Agt{} know-\Ben{}-\Pf{}//
    \glft \trsl{It was not Martin whom Marek introduced Janek to.}//
\endgl
\a
\begingl
    \gla Janek Martina \textbf{zám} Markám hévoržébik.//
    \glb Janek Martin-\Pat{} \Neg{} Marek-\Agt{} know-\Ben{}-\Pf{}//
    \glft \trsl{It was not Marek who introduced Janek to Martin.}//
\endgl
\xe

It is also common, especially in spoken Iridian, to append the clitic \ird{-te}
after the word being negated by \ird{zám} (i.e., if the negative clitic is not
in the default position before the main verb) to provide more emphasis on the
negation.

\pex
\a
\begingl
    \gla \textbf{Zám} Janek\textbf{-te} Martina Markám hévoržébik.//
    \glb \Neg{} Janek=\Foc{} Martin-\Pat{} Marek-\Agt{} know-\Ben{}-\Pf{}//
    \glft \trsl{It was not Janek whom Marek introduced to Martin.}//
\endgl
\a
\begingl
    \gla Janek \textbf{zám} Martina\textbf{-te} Markám hévoržébik.//
    \glb Janek \Neg{} Martin-\Pat{}=\Foc{} Marek-\Agt{} know-\Ben{}-\Pf{}//
    \glft \trsl{It was not Martin whom Marek introduced Janek to.}//
\endgl
\a
\begingl
    \gla Janek Martina \textbf{zám} Markám\textbf{-te} hévoržébik.//
    \glb Janek Martin-\Pat{} \Neg{} Marek-\Agt{}=\Foc{} know-\Ben{}-\Pf{}//
    \glft \trsl{It was not Marek who introduced Janek to Martin.}//
\endgl
\xe

The different constituents of the sentence can be negated simultaneously; thus,
for example, the sentence below is grammatically permitted:

\pex
\begingl
    \gla \textbf{Zám} Janek \textbf{zám} Martina \textbf{zám} Markám \textbf{zá}hévoržébik.//
    \glb \Neg{} Janek \Neg{} Martin-\Pat{} \Neg{} Marek-\Agt{} \Neg{}know-\Ben{}-\Pf{}//
    \glft \trsl{It was not Janek who was not introduced to someone who is not Martin by someone who is not Marek.}//
\endgl
\xe

Nonetheless, due to their general unwieldiness, forms like this are extremely
rare (both in the spoken and the written language), with preference given to
single and double negation instead. Since \ird{-te} can only appear in a
sentence once, where there are more than one negate constituent in a sentence,
\ird{-te} is appended to the element which has the most significance (usually
the topic); or, if there are two constituents negated and one of them is the
main verb, \ird{-te} is appended to that other element.

\pex
\begingl
    \gla \textbf{Zám} Janek\textbf{-te} Martina Markám \textbf{zá}hévoržébik.//
    \glb \Neg{} Janek=\Foc{} Martin-\Pat{} Marek-\Agt{} \Neg{}know-\Ben{}-\Pf{}//
    \glft \trsl{It was not Janek who was not introduced to Martin by Marek.}//
\endgl
\xe


\section{Existential Constructions}
\label{sec:exst}
An existential sentence is a specialized construction used to express the existence or presence of someone or something. The particle \ird{ješ} and its inverse \ird{niho} are used to form existential sentences.

\pex
\begingl
\gla Tak ješ zarno.//
\glb here \Exst{} people//
\glft \trsl{There are people here.}//
\endgl
\xe

\pex
\begingl
\gla Tak niho zarno.//
\glb here \mk{exst.neg} people//
\glft \trsl{There is no one here.}//
\endgl
\xe

Statements expressing location use a copular construction, although an existential construction is used in the negative.

\pex
\begingl
\gla Dá na duma.//
\glb \mk{1s.str} \Loc{} house-\Pat{}//
\glft \trsl{I'm at home.}//
\endgl
\xe

\pex
\begingl
\gla Na duma niho dá.//
\glb \Loc{} house-\Pat{} \mk{exst.neg} \mk{1s.str}//
\glft \trsl{I'm not at home.}//
\endgl
\xe

The particles \ird{ješ} and \ird{niho} must always precede the noun whose presence or existence is being expressed.

\pex
\begingl
\gla Na ránema ona ješ htoš.//
\glb \Loc{} desk-\mk{1s-pat} one \Exst{} book//
\glft \trsl{There is one book on my desk.}//
\endgl
\xe

\pex
\begingl
\gla M\"y ješ mulaž.//
\glb two \Exst{} door//
\glft \trsl{There are two doors.}//
\endgl
\xe



\subsection{Possession}
Existential constructions are also used to indicate possession, with the possessor marked in the patientive case.

\pex
\begingl
\gla Marka ješ oblašc.//
\glb Marek-\Pat{} \Exst{} pet//
\glft \trsl{Marek has a pet.}//
\endgl
\xe

\pex
\begingl
\gla Tomáša niho mlaz.//
\glb Tomáš-\Pat{} \Exst{} brother//
\glft \trsl{Tomáš does not have a brother.}//
\endgl
\xe

\subsection{Impersonal constructions}
\pex
\begingl
\gla Martina ješ trešnikou na tropa.//
\glb Martin-\Pat{} \Exst{} write-\mk{pv-pf-nz} \Loc{} wall-\Pat{}//
\glft \trsl{Martin wrote something on the wall.}//
\endgl
\xe

\pex
\begingl
\gla Voštnikouva \v{z}a ješ piaščkou?//
\glb cook-\mk{pv-pf-nz-pat} already \Exst{} eat-\mk{av-pf-nz}//
\glft \trsl{Did somebody eat what (I) cooked?}//
\endgl
\xe

\section{Copular Constructions}
\subsection{Null copula}

Copular sentences are a minor sentence type where the predicate is not a verb. For the purposes of this grammar, we narrow down our definition of copular constructions to the following:
\pex
\a \textit{Equative:} Marek is the doctor (we are talking about).
\a \textit{Inclusive:} Marek is a doctor.
\a \textit{Attributive:} Marek is tall.
\a \textit{Locative:} Marek is in the hospital.
\xe

Iridian does not make a distinction between equative, inclusive and attributive clauses. Locative clauses on the other hand, may be expressed using a copular or an existential construction, as will be discussed in this section.

Iridian is a superficially a zero-copula language and the most common way to form copular sentences is mere juxtaposition.

\pex<cop>
\begingl
\gla Marek doktor.//
\glb Marek doctor//
\glft \trsl{Marek (is a/the) doctor.}//
\endgl
\xe

The above example could either be taken to mean (1) Marek is a doctor (inclusive), or (2) Marek is the doctor (equative). Generally, though, Iridian uses word order to distinguish between equative and inclusive clauses.

\pex
\a \textit{Inclusive:} \{item in class\}\tss{N} $\varnothing$ \{class\}\tss{P}
\a \textit{Equative:} \{class\}\tss{N} $\varnothing$ \{item class\}\tss{P}
\xe

To avoid ambiguity, Example \getref{cop} can be reformulated to either of the following sentences:

\pex<cop1>
\a
\begingl
\gla Marek doktor.//
\glb Marek doctor//
\glft \trsl{Marek is a doctor.}//
\endgl

\a
\begingl
\gla Doktor Marek.//
\glb doctor Marek//
\glft \trsl{Marek is the doctor.}//
\endgl

\xe

The inversion of word order is not strongly grammaticalized with NP-NP sentences, i.e., both sentences in Example \getref{cop1} can still be used interchangeably without a change in meaning and preference is given on the one over the other when there is an ambiguity. This is not the case with attributive clauses, i.e., sentences with adjective or adjective phrase predicates. Consider for example the sentence below:

\pex
\begingl
\gla Marek rázym.//
\glb Marek tall//
\glft \trsl{Marek is tall.}//
\endgl
\xe

Inverting the word order of the sentence above would change the adjective to a substantive since modifiers cannot occupy the topic position.

\pex
\begingl
\gla Rázym Marek.//
\glb tall Marek//
\glft \trsl{The tall one is Marek.}//
\endgl
\xe

Iridian also distinguishes between attributive clauses expressing permanent conditions and clauses expressing temporary conditions, with the latter being expressed using existential constructions in certain adjectives.

\pex
\begingl
\gla *Marek morec.//
\glb Marek hungry//
\glft \trsl{Marek is hungry}//
\endgl
\xe


\pex
\begingl
\gla Marka ješ morec.//
\glb Marek-\Pat{} \Exst{} hunger//
\glft \trsl{Marek is hungry}//
\endgl
\xe

A full list of adjectives/modifiers that use the existential construction can be found in the section~\ref{sec:exst}.

The copula, however, cannot be ommitted in grammatical moods other than the indicative.

\subsection{Negative copula}

Iridian has the negative copula \ird{česná}.

\pex
\begingl
\gla Marek doktor česná.//
\glb Marek doctor \mk{cop.neg}//
\glft \trsl{Marek is not (a/the) doctor.}//
\endgl
\xe

\par The inversion of word order may also be used when one wants to avoid ambiguity:

\pex
\begingl
\gla Doktor Marek česná.//
\glb doctor Marek \mk{cop.neg}//
\glft \trsl{Marek is not the doctor.}//
\endgl
\xe


\subsection{Conjugation paradigm}
