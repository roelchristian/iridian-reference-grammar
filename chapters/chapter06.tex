\chapter{Syntax of Simple Clauses}

\section{Introduction}

The constituent word order of Iridian sentences is SOV, but the agglutinative nature of the language and the presence of case-marking on nouns makes word order typically flexible, with the only universal rule being that the main verb should appear at the end of a sentence.

\section{Topic-Predicate Constructions}\index{topic}\index{predicate}
The Iridian sentence can be divided primarily into a topic part and a predicate
or comment part. The topic is what the sentence is about, while the predicate or comment represents the information presented in the sentence about the topic. While both the topic and the predicate are pragmatic constructs, the topic-predicate construction is important as it determines how the rest of the sentence is structured.


\begin{figure}[H]
  \centering
  \begin{forest}
    [S,
      [{\sc top}] [{\sc pred}]]
  \end{forest}
  \caption{Nuclear structure of sentences}
  \label{}
\end{figure}

The topic of the sentence does not necessarily coincide with the subject of the sentence. This is true as well in English, as we see in example (\ref{ex:engtop}); although where English allows the topic to appear anywhere in the sentence, as long as the subject is placed first, Iridian, typical of topic-prominent languages. requires the topic to always be introduced first, leaving the rest of the information afterwards.


\pex\label{ex:engtop}
\a Martha saw John.
\a A dog bit \textbf{Martha}.
\a It is raining \textbf{today},
\xe


\pex
\a
\begingl
\gla \relax[Janek]\tss{\mk{Top}} [mlaza boule\v{s}ik.]\tss{\mk{Pred}}//
\glft \trsl{As for Janek, he killed his brother}.//
\endgl

\a
\begingl
\gla \relax[Tereza]\tss{\mk{Top}} [jec\'am nale\v{c}nik.]\tss{\mk{Pred}}//
\glft \trsl{As for Tereza, she was bitten by a dog}//
\endgl

\a
\begingl
\gla \relax[Shl\'ed]\tss{\mk{Top}} [zniep\v{s}al\'i.]\tss{\mk{Pred}}//
\glft \trsl{As for today, it is raining.}//
\endgl

\xe

As \textcite[9]{kiss2004} notes:

\begin{quote}
  We tend to describe eventsfrom a human perspective, as statements about theirhuman participants – and subjects are more often [+human] than objects are. Inthe case of verbs with a [–human] subject and a [+human] accusative or obliquecomplement,  the  most  common  permutation  is  that  in  which  the  accusative  oroblique complement occupies the topic position (3a,b). When the possessor is theonly human involved in an action or state, the possessor is usually topicalized(3c).
\end{quote}

\section{The Noun Phrase}

Iridian is a strongly head-final language.

\subsection{Nuclear constructions}
\subsection{With adjectival clauses}
\subsection{Wsith prepositional phrases}
\subsection{With relative clauses}

\section{Topicless Sentences}\label{sec:topicless}\index{topicless sentence}


\section{Relative and Comparative Constructions}\label{relativecomparative}\index{comparative constructions}

The clitic\index{clitic} \ird{t\'am} is used to form simple comparative and relative constructions. \ird{T\'am} is often ommitted however where the comparison can be implied from context. In this construction, the standard of comparison (the noun preceded by `than' in English) is unmarked and the noun being compared marked in the agentive if it is a positive/negative comparison, or in the instrumental if it is a correlation.


\pex
\a
\begingl
\gla Janek-t\'am Mark\'am nesta\v{c}\'al.//
\glb Janek=\mk{comp} Marek-\mk{agt} tall-\mk{cont}//
\glft \trsl{Marek is taller than Janek.}//
\endgl
\a
\begingl
\gla Janek Mark\'am nesta\v{c}\'al.//
\glb Janek Marek-\mk{agt} tall-\mk{cont}//
\glft \trsl{Marek is taller than Janek.}//
\endgl
\xe
\pex
\a
\begingl
\gla Janek-t\'am Marku nesta\v{c}\'al.//
\glb Janek=\mk{comp} Marek-\mk{inst} tall-\mk{cont}//
\glft \trsl{Marek is as tall as Janek.}//
\endgl
\a
\begingl
\gla Janek Marku nesta\v{c}\'al.//
\glb Janek Marek-\mk{inst} tall-\mk{cont}//
\glft \trsl{Marek is as tall as Janek.}//
\endgl
\xe

Note that \ird{t\'am} can only be used with the copulative form of the stative verb, as the attributive and nominal forms have separate conjugated comparative forms. When using these forms, however, the standard of comparison is marked in the genitive. In relative constructions, the instrumental is also replaced with the genitive, but the modifier \ird{zn\'i}, \trsl{same} is added before the stative verb.

\pex
\a
\begingl
\gla Jancie nesta\v{c}en\'i hloc mlazem.//
\glb Janek-\mk{gen} tall-\mk{comp-att} boy brother-\mk{1s}//
\glft \trsl{The boy who is taller than Janek is my brother} (\emph{Lit.,} \trsl{The taller-than-Janek boy is my brother.})//
\endgl
\a
\begingl
\gla Jancie zn\'i nesta\v{c}en\'i hloc mlazem.//
\glb Janek-\mk{gen} same tall-\mk{comp-att} boy brother-\mk{1s}//
\glft \trsl{The boy who is as tall as Janek is my brother.}//
\endgl
\xe

\ird{T\'am} can be relativized by appending the clitic \ird{to}. When used with \ird{t\'am-to} the standard of comparison is marked in the patientive case. The use of t\'am-to in relative clauses is discussed in further detail in the next chapter.

\ex
\begingl
\gla Viktor na shlopa t\'am-to nest\'a\v{c}ek.//
\glb Viktor \mk{loc} siblings-\mk{pat} \mk{comp=rz=} to:be:tall-\mk{av-pf}//
\glft \trsl{Among the siblings, Viktor grew up to be the tallest.}//
\endgl
\xe

\ex
\begingl
\gla Jank\'am Marka t\'am-to zu\v{s}tal\'ebik ko Tereza//
\glb Janek-\mk{agt} Marek-\mk{pat} \mk{comp=rz=} to:be:happy-\mk{ben-pf} \mk{att} Tereza//
\glft \trsl{Tereza, whom Janek made happier than Marek}//
\endgl
\xe

\ex
\begingl
\gla Marka t\'am-t\'ov\'i zu\v{s}tal\'ebik ko obla\v{s}c//
\glb Marek-\mk{pat} \mk{comp=rz-gen=} to:be:happy-\mk{ben-pf} \mk{att} pet//
\glft \trsl{the pet [of the person who was made happier than Marek]}//
\endgl
\xe

Iridian does not have a morphologically distinct superlative construction. For example, \ird{pizden\'i} (from \ird{pizd\'a}, \trsl{to be big}) can either mean \trsl{bigger} or \trsl{biggest} depending on context. Where the meaning cannot be easily implied from context, the word \ird{ohnu} (derived from the word \ird{ohna}, \trsl{first} in the instrumental case) is often used as quantifier.

\pex
\a
\begingl
\gla Univerzitet na razmeka pizdenou.//
\glb university \mk{loc} city-\mk{pat} to:be:big-\mk{comp-nz}//
\glft \trsl{(This) university is the biggest in the city.}//
\endgl
\a
\begingl
\gla Univerzitet na razmeka ohnu pizdenou.//
\glb university \mk{loc} city-\mk{pat} first-\mk{inst} to:be:big-\mk{comp-nz}//
\glft \trsl{(This) university is the biggest in the city.}//
\endgl
\xe

When using an adverbial construction with the instrumental case to modify or quantify the comparison, the adverbial phrase must immediately precede the stative verb if in the attributive or nominal form, or the particle \ird{t\'am} otherwise. The same is true with invariable modifiers like \ird{nahte}, \trsl{too much}, \ird{dnu}, \trsl{a bit}, etc.

\ex
\begingl
\gla To bag\'a\v{z} j\'an\'am u 10 kilogr\'amu t\'am pr\'ekv\'al.//
\glb \mk{dem.prox} baggage \mk{dem.med} around 10 kilogram-\mk{inst} \mk{comp=} heavy-\mk{cont}//
\glft \trsl{This baggage is heavier by about 10 kilograms than that one.}//
\endgl
\xe

\ex
\begingl
\gla u 10 kilogr\'amu pr\'ekven\'i bag\'a\v{z}//
\glb around 10 kilogram-\mk{inst} heavy-\mk{comp-att} baggage//
\glft \trsl{the baggage, which is heavier by about 10 kilograms}//
\endgl
\xe

\ex
\begingl
\gla Nahte pizdenou zma\v{z}nik\'ove\v{s}.//
\glb too:much big-\mk{comp-nz} make-\mk{pv-pf-nz-2s}//
\glft \trsl{The much bigger one is the one you made.}//
\endgl
\xe

\section{Questions}\index{questions!syntax of}

\subsection{Yes-no questions}\index{yes-no questions}

A declarative sentence can be made into a question by a simple rise in intonation at the end of the phrase:

\pex
\a
\begingl
\gla Janek sa uzdrav\v{s}ek.//
\glb Janek already \mk{ref}-sleep-\mk{av-pf}//
\glft \trsl{Janek has fallen asleep.}//
\endgl
\a
\begingl
\gla Janek sa uzdrav\v{s}ek?//
\glb Janek already \mk{ref}-sleep-\mk{av-pf}//
\glft \trsl{Has Janek fallen asleep yet?}//
\endgl
\xe

Alternatively the interrogative particle \ird{no} may be used. When used this way, the base sentence will still feature a clause-final rise in intonation, followed by a falling intonation at the location of the question particle, similar to the intonation structure of tag questions in English. In the written language, the particle \ird{no} may also surface as a clitic, prefixing itself to the verb, which this usage requires to be in the negative.

\pex
\begingl
\gla Janek sa uzdrav\v{s}ek no?//
\glb Janek already \mk{ref}-sleep-\mk{av-pf} \mk{q}//
\glft \trsl{Has Janek fallen asleep yet?}//
\endgl
\xe

\pex
\begingl
\gla Janek sa noz\'aduzdrav\v{s}ek?//
\glb Janek already \mk{q-neg-ref}-sleep-\mk{av-pf}//
\glft \trsl{Has Janek fallen asleep yet?}//
\endgl
\xe

The choice between using a simple rise in intonation or the question particle \ird{no} is a personal one, and a speaker may use the one or the other in different situations or shift between them seemingly at random. Both methods in free variation and offer no differences in meaning, formality, etc.

Tag questions\index{tag questions} are formed similar to the German use of \emph{`nicht war?'}:

\pex
\begingl
\gla Iv\'ana niegu scen\v{z}ach, z\'ajuda (no)?//
\glb Iv\'ana later-\mk{inst} arrive-\mk{av-ctpv} \mk{neg-}truth \mk{q}//
\glft \trsl{Iv\'ana is coming later, isn't she?}//
\endgl
\xe

Again the tag \ird{z\'ajuda} (\trsl{untrue, not the truth}) can appear with or without the question particle \ird{no}, although here the form without \ird{no} is more common in colloquial speech. Other common ways of forming tag questions include appending (1) the word \ird{juda} \trsl{truth} or (2) the particle \ird{z\'ane}, formed from the negative prefix \ird{z\'a} and the expletive \ird{ne}.

\pex
\a
\begingl
\gla Scen\v{z}ach, juda?//
\glb arrive-\mk{av-ctpv} truth//
\glft \trsl{(She) is coming, isn't she?}//
\endgl

\a
\begingl
\gla Scen\v{z}ach, z\'ane?//
\glb arrive-\mk{av-ctpv} \mk{expl}//
\glft \trsl{(She) is coming, isn't she?}//
\endgl
\xe

Although the particle \ird{no} would normally appear after the verb, it can follow other parts of the sentence (except pure function words), but with the effect of changing the emphasis or the nature of the question. When used in this manner, the particle is treated as a clitic and is separated from the word it modifies by a dash. Furthermore, there is a tendency especially in the spoken language to move the cliticized noun to the start of the sentence.

\pex
\a
\begingl
\gla Iv\'ana-no niegu scen\v{z}ach?//
\glb Iv\'ana=\mk{q} later-\mk{inst} arrive-\mk{av-ctpv} \mk{neg-}truth \mk{q}//
\glft \trsl{Is it Iv\'ana who is coming later?}//
\endgl
\a
\begingl
\gla Iv\'ana niegu-no scen\v{z}ach?//
\glb Iv\'ana later-\mk{inst}=\mk{q} arrive-\mk{av-ctpv} \mk{neg-}truth \mk{q}//
\glft \trsl{Will it be later that Ivana is coming?}//
\endgl
\a
\begingl
\gla Niegu-no Iv\'ana scen\v{z}ach?//
\glb later-\mk{inst}=\mk{q} Iv\'ana arrive-\mk{av-ctpv} \mk{neg-}truth \mk{q}//
\glft \trsl{Will it be later that Ivana is coming?}//
\endgl
\xe

To make an existential sentence\index{existential constructions} a yes-no question, it is first transformed to the negative and the particle \ird{no} is then cliticized to the word \ird{niho}. If however, the theme of the sentence is quantified, the word \ird{je\v{s}}\index{je\v{s}} is kept (but shifted to the front of the quantifier), and \ird{no} is attached to the quantifier. The form \ird{je\v{s}-no} is ungrammatical.

\pex
\begingl
\gla Marka niho-no obla\v{s}c?//
\glb Marek-\mk{pat} \mk{neg.exst=q} pet//
\glft \trsl{Does Marek have a pet?}//
\endgl
\xe

\pex
\a
\begingl
\gla Co bibliot\'ecie Marka hron\'a je\v{s} kup\'eninkou t\'om?//
\glb \mk{abl} library-\mk{gen} Marek-\mk{pat} three \mk{exst} borrow-\mk{pv-pf-nz} book//
\glft \trsl{Marek borrowed three books from the library.}//
\endgl
\a
\begingl
\gla Co bibliot\'ecie Marka je\v{s} hron\'a-no kup\'eninkou t\'om?//
\glb \mk{abl} library-\mk{gen} Marek-\mk{pat} \mk{exst} three\mk{=q} borrow-\mk{pv-pf-nz} book//
\glft \trsl{Did Marek borrow three books from the library.}//
\endgl
\xe

The clitic \ird{no} can of course be moved around, with subtle changes in meaning.

\pex
\a \emph{Neutral form:}\\
\ird{Co bibliot\'ecie Marka je\v{s} hron\'a-no kup\'eninkou t\'om?}\\
\trsl{Did he borrow \emph{three} books, etc?}
\a \emph{Emphasis on \emph{Marek:}}\\
\ird {Co bibliot\'ecie Marka-no hron\'a je\v{s} kup\'eninkou t\'om?}\\
\trsl{Did \emph{Marek} borrow them, etc?}
\a \emph{Emphasis on \emph{library:}}\\
\ird {Co bibliot\'ecie-no Marka hron\'a je\v{s} kup\'eninkou t\'om?}\\
\trsl{Did he borrow them from the \emph{library}, etc?}
\xe

Note that in more complex existential constructions, as the one above which includes a nominalized determiner, the sentence may have to be reconstructed as a non-existential construction if it is the theme (i.e., the object being possessed or whose existence is described) that is in question.

\pex
\a
\begingl
\gla Co bibliot\'ecie Marek hron\'a t\'oma kup\'en\v{z}ek?//
\glb \mk{abl} library-\mk{gen} Marek three book-\mk{pat} borrow-\mk{av-pf}//
\glft \trsl{Did Marek \emph{borrow} three books from the library.}//
\endgl
\a
\begingl
\gla Co bibliot\'ecie Marek hron\'a t\'oma-no kup\'en\v{z}ek?//
\glb \mk{abl} library-\mk{gen} Marek three book-\mk{pat=q} borrow-\mk{av-pf}//
\glft \trsl{Did Marek borrow three \emph{books} from the library.}//
\endgl
\xe

%explain further the preference
Note that the first example above is not the neutral word order, given Iridian's preference to use existential constructions in sentences like the ones above. In this case, it would be akin to asking \trsl{Did he borrow them, or did he acquire it by some other means?}

To change a copular sentence into a question, the clitic \ird{-no} is added to whichever element is in question. Removing \ird{no} would indicate disbelief on the part of the speaker and would imply that the answer \trsl{No} is expected.

\pex
\a
\begingl
\gla Tereza \v{s}tudent-no?//
\glb Tereza student\mk{=q}//
\glft \trsl{Is Tereza a student?}//
\endgl
\a
\begingl
\gla Tereza-no \v{s}tudent?//
\glb Tereza\mk{=q} student//
\glft \trsl{Is Tereza the student?}//
\endgl
\a
\begingl
\gla Tereza \v{s}tudent?//
\glb Tereza student//
\glft \trsl{Is Tereza a student? (I don't think so)}//
\endgl
\xe

\subsection{Wh- questions}\index{wh- questions}\index{information question|see{wh-questions}}
In wh- questions, the interrogative pronoun typically appears after the topic or at the beginning of a sentence if the sentence does not have a topic, and is immediately followed by the clitic \ird{no}.

\pex
\begingl
\gla Karel jena-no mo\v{z}la\v{s}\'al?//
\glb Karel where\mk{=q} live-\mk{av-cont}//
\glft \trsl{Where does Karel live?}//
\endgl
\xe

\pex
\begingl
\gla Bych zajehu-no kravna\v{s}al\'i?//
\glb yesterday why=\mk{q} cry-\mk{av-prog}//
\glft \trsl{Why was he crying yesterday?}//
\endgl
\xe

\subsection{Indirect questions}

Indirect questions are constructed in the subjunctive, with the addition of the particle \ird{a\v{s}}.

\pex
\begingl
\gla N\'u a\v{s} ho\v{s}ez\'ila.//
\glb tomorrow \mk{q.ind} rain\mk{av-sbj.ipf}.//
\glft \trsl{I wonder if it's gonna rain tomorrow.}//
\endgl
\xe

\subsection{Answering questions}

\subsubsection{Answering yes-no questions}\index{yes-no questions}
The most common way to answer a yes-no question is to repeat the main verb in the original question (known formally as an echo response), adding the prefix \ird{z\'a-} if the answer is in the negative.\index{echo response}

\pex
\begingl
\gla ---Na kinot\'eka sto\v{z}ek? ---Sto\v{z}ek.//
\glb \mk{loc} cinema-\mk{pat} go-\mk{av-pf} go-\mk{av-pf}//
\glft \trsl{{}``Did you go to the movies?'' ``Yes, I did.''{}}//
\endgl
\xe

In questions based on copular sentences, this would mean the repetition of the word or phrase where \ird{no} is attached to. In those based on existential constructions, the existential particle is repeated as appropriate (or the word where \ird{no} is attached to, if emphasis has been shifted otherwise).

\pex
\a
\begingl
\gla ---Marek-no doktor? ---Marek.//
\glb Marek\mk{=q} doctor Marek//
\glft \trsl{{}``Is \emph{Marek} a doctor?'' ``Yes, he is.''{}}//
\endgl
\a
\begingl
\gla ---Marek doktor-no? ---Doktor.//
\glb Marek doctor\mk{=q} doctor//
\glft \trsl{{}``Is Marek a \emph{doctor}?'' ``Yes, he is.''{}}//
\endgl
\xe

\pex
\begingl
\gla ---Tak je\v{s} h\'evorn\'al? ---Niho.//
\glb here \mk{exst} know-\mk{pv-cont} \mk{exst.neg}//
\glft \trsl{{}``Do you know anyone here?'' ``No, I don't.''{}}//
\endgl
\xe

Iridian does has separate words for \trsl{yes} and \trsl{no}: \ird{d\'e} (affirmative \trsl{yes}), \ird{\v{c}e} (contrastive \trsl{yes}) and \ird{om\'a} (\trsl{no}). They are, however, rarely used by themselves alone, but would often be added to the echo response as an intensifier.

\pex
\begingl
\gla ---Na kinot\'eka sto\v{z}ek? ---Sto\v{z}ek d\'e.//
\glb \mk{loc} cinema-\mk{pat} go-\mk{av-pf} go-\mk{av-pf} yes//
\glft \trsl{{}``Did you go to the movies?'' ``Yes, I did.''{}}//
\endgl
\xe

The contrastive \ird{\v{c}e} is used if the question has been framed in the negative and the answer \trsl{yes} is an affirmation not of the question but of the statement originally negated. Further emphasis may be added by introducing the answer with the word \ird{ano} (\trsl{but}). \ird{\v{C}e} is also used contrasively to answer questions based on copular constructions that do not contain the clitic \ird{no}, as such would imply that the answer expective is in the negative.

\pex
\begingl
\gla ---Studnik\'ovem z\'ado\v{s}tnik? ---Ano o\v{s}tnik \v{c}e.//
\glb send-\mk{pv-pf-nz-1s} \mk{neg-}read-\mk{pv-pf} but read-\mk{pv-pf} yes//
\glft \trsl{{}``Didn't you read the message I sent?'' ``But I did.''{}}//
\endgl
\xe

\pex
\begingl
\gla ---Marek doktor? ---Doktor \v{c}e.//
\glb Marek doctor doctor yes//
\glft \trsl{{}``\emph{(Dismissively)} Marek is a doctor?'' ``But he is!''{}}//
\endgl
\xe

\ird{\v{C}e} is also required when answering questions that use the proclitic\index{proclisis} \ird{no-} construction (often found in literary registers\index{literary register}), even though the answer does not possess any contrasive meaning.

\pex
\begingl
\gla ---Janek sa noz\'aduzdrav\v{s}ek? Uzdrav\v{s}ek \v{c}e.//
\glb Janek already \mk{q-neg-ref}-sleep-\mk{av-pf} \mk{ref}-sleep-\mk{av-pf} yes//
\glft \trsl{{}``Has Janek fallen asleep yet?'' ``Yes, he has.''}//
\endgl
\xe

To negate an existential question, the negative copula is used. An alternative may also be presented instead to contradict a question more emphatically. Both techniques would often used together in the spoken language.

\pex
\a
\begingl
\gla ---Tereza \v{s}tudent? ---\v{S}tudent \v{c}esn\'a.//
\glb Tereza student student \mk{cop.neg}//
\glft \trsl{{}``Is Tereza a student?'' ``No, she isn't.'{}}//
\endgl
\a
\begingl
\gla ---Tereza \v{s}tudent? ---\'Odilou\v{s}c.//
\glb Tereza student professor//
\glft \trsl{{}``Is Tereza a student?'' ``No, she is a professor.'{}}//
\endgl
\a
\begingl
\gla ---Tereza \v{s}tudent? ---\v{S}tudent \v{c}esn\'a, \'odilou\v{s}c.//
\glb Tereza student student \mk{cop.neg} professor//
\glft \trsl{{}``Is Tereza a student?'' ``No, she isn't. She's a professor'{}}//
\endgl
\xe

Emphatic answers can also be made using \ird{\'e\v{s}te} (\trsl{of course}) and  \ird{\'e\v{s}te om\'a/niho/\v{c}esn\'a} (\trsl{of course not}). \ird{\'E\v{s}te} may also be used with an echo response.

\pex
\a
\begingl
\gla ---T\'om o\v{s}tnik? ---\'E\v{s}te.//
\glb book read-\mk{pv-pf} of:course//
\glft \trsl{{}``Did you read the book?'' ``Of course I did.''{}}//
\endgl
\a
\begingl
\gla ---T\'om o\v{s}tnik? ---\'E\v{s}te z\'ado\v{s}tnik.//
\glb book read-\mk{pv-pf} of:course \mk{neg-}read-\mk{pv-pf}//
\glft \trsl{{}``Did you read the book?'' ``Of course I didn't.''{}}//
\endgl
\xe

\section{Negation}\index{negation}

In Iridian sentences, negation is performed by the particle \ird{z\'am}, which attaches to the beginning of the word or phrase  it negates. The default position of the negative particle is usually before the main verb where it surfaces as \ird{z-} before vowels, \ird{\v{z}-} before \emph{i}-glides, and \ird{z\'a-} elswehere.

\pex
\a
\begingl
    \gla Janek Martina Mark\'am \textbf{z}oplaz\'ebik.//
    \glb Janek Martin-\mk{pat} Marek-\mk{agt} \mk{neg=}know-\mk{ben-pf}//
    \glft \trsl{Marek did not introduce Janek to Martin.}//
\endgl
\a
\begingl
    \gla \textbf{Z\'am} Janek Martina Mark\'am zoplaz\'ebik.//
    \glb \mk{neg=} Janek Martin-\mk{pat} Marek-\mk{agt} \mk{neg=}know-\mk{ben-pf}//
    \glft \trsl{It was not Janek whom Marek introduced to Martin.}//
\endgl
\a
\begingl
    \gla Janek \textbf{z\'am} Martina Mark\'am zoplaz\'ebik.//
    \glb Janek \mk{neg=} Martin-\mk{pat} Marek-\mk{agt} \mk{neg=}know-\mk{ben-pf}//
    \glft \trsl{It was not Martin whom Marek introduced Janek to.}//
\endgl
\a
\begingl
    \gla Janek Martina \textbf{z\'am} Mark\'am zoplaz\'ebik.//
    \glb Janek Martin-\mk{pat} \mk{neg=} Marek-\mk{agt} \mk{neg=}know-\mk{ben-pf}//
    \glft \trsl{It was not Marek who introduced Janek to Martin.}//
\endgl
\xe



\section{Existential Constructions}
\label{sec:exst}
An existential sentence is a specialized construction used to express the existence or presence of someone or something. The particle \ird{je\v{s}} and its inverse \ird{niho} are used to form existential sentences.

\pex
\begingl
\gla Tak je\v{s} zarno.//
\glb here \mk{exst} people//
\glft \trsl{There are people here.}//
\endgl
\xe

\pex
\begingl
\gla Tak niho zarno.//
\glb here \mk{exst.neg} people//
\glft \trsl{There is no one here.}//
\endgl
\xe

Statements expressing location use a copular construction, although an existential construction is used in the negative.

\pex
\begingl
\gla D\'a na duma.//
\glb \mk{1s.str} \mk{loc} house-\mk{pat}//
\glft \trsl{I'm at home.}//
\endgl
\xe

\pex
\begingl
\gla Na duma niho d\'a.//
\glb \mk{loc} house-\mk{pat} \mk{exst.neg} \mk{1s.str}//
\glft \trsl{I'm not at home.}//
\endgl
\xe

The particles \ird{je\v{s}} and \ird{niho} must always precede the noun whose presence or existence is being expressed.

\pex
\begingl
\gla Na r\'anema ona je\v{s} hto\v{s}.//
\glb \mk{loc} desk-\mk{1s-pat} one \mk{exst} book//
\glft \trsl{There is one book on my desk.}//
\endgl
\xe

\pex
\begingl
\gla M\"y je\v{s} mula\v{z}.//
\glb two \mk{exst} door//
\glft \trsl{There are two doors.}//
\endgl
\xe



\subsection{Possession}
Existential constructions are also used to indicate possession, with the possessor marked in the patientive case.

\pex
\begingl
\gla Marka je\v{s} obla\v{s}c.//
\glb Marek-\mk{pat} \mk{exst} pet//
\glft \trsl{Marek has a pet.}//
\endgl
\xe

\pex
\begingl
\gla Tom\'a\v{s}a niho mlaz.//
\glb Tom\'a\v{s}-\mk{pat} \mk{exst} brother//
\glft \trsl{Tom\'a\v{s} does not have a brother.}//
\endgl
\xe

\subsection{Impersonal constructions}
\pex
\begingl
\gla Martina je\v{s} tre\v{s}nikou na tropa.//
\glb Martin-\mk{pat} \mk{exst} write-\mk{pv-pf-nz} \mk{loc} wall-\mk{pat}//
\glft \trsl{Martin wrote something on the wall.}//
\endgl
\xe

\pex
\begingl
\gla Vo\v{s}tnikouva sa je\v{s} pia\v{s}\v{c}kou?//
\glb cook-\mk{pv-pf-nz-pat} already \mk{exst} eat-\mk{av-pf-nz}//
\glft \trsl{Did somebody eat what (I) cooked?}//
\endgl
\xe

\section{Copular Constructions}
\subsection{Null copula}

Copular sentences are a minor sentence type where the predicate is not a verb. For the purposes of this grammar, we narrow down our definition of copular constructions to the following:
\pex
\a \textit{Equative:} Marek is the doctor (we are talking about).
\a \textit{Inclusive:} Marek is a doctor.
\a \textit{Attributive:} Marek is tall.
\a \textit{Locative:} Marek is in the hospital.
\xe

Iridian does not make a distinction between equative, inclusive and attributive clauses. Locative clauses on the other hand, may be expressed using a copular or an existential construction, as will be discussed in this section.

Iridian is a superficially a zero-copula language and the most common way to form copular sentences is mere juxtaposition.

\pex<cop>
\begingl
\gla Marek doktor.//
\glb Marek doctor//
\glft \trsl{Marek (is a/the) doctor.}//
\endgl
\xe

The above example could either be taken to mean (1) Marek is a doctor (inclusive), or (2) Marek is the doctor (equative). Generally, though, Iridian uses word order to distinguish between equative and inclusive clauses.

\pex
\a \textit{Inclusive:} \{item in class\}\tss{N} $\varnothing$ \{class\}\tss{P}
\a \textit{Equative:} \{class\}\tss{N} $\varnothing$ \{item class\}\tss{P}
\xe

To avoid ambiguity, Example \getref{cop} can be reformulated to either of the following sentences:

\pex<cop1>
\a
\begingl
\gla Marek doktor.//
\glb Marek doctor//
\glft \trsl{Marek is a doctor.}//
\endgl

\a
\begingl
\gla Doktor Marek.//
\glb doctor Marek//
\glft \trsl{Marek is the doctor.}//
\endgl

\xe

The inversion of word order is not strongly grammaticalized with NP-NP sentences, i.e., both sentences in Example \getref{cop1} can still be used interchangeably without a change in meaning and preference is given on the one over the other when there is an ambiguity. This is not the case with attributive clauses, i.e., sentences with adjective or adjective phrase predicates. Consider for example the sentence below:

\pex
\begingl
\gla Marek rázym.//
\glb Marek tall//
\glft \trsl{Marek is tall.}//
\endgl
\xe

Inverting the word order of the sentence above would change the adjective to a substantive since modifiers cannot occupy the topic position.

\pex
\begingl
\gla Rázym Marek.//
\glb tall Marek//
\glft \trsl{The tall one is Marek.}//
\endgl
\xe

Iridian also distinguishes between attributive clauses expressing permanent conditions and clauses expressing temporary conditions, with the latter being expressed using existential constructions in certain adjectives.

\pex
\begingl
\gla *Marek morec.//
\glb Marek hungry//
\glft \trsl{Marek is hungry}//
\endgl
\xe


\pex
\begingl
\gla Marka je\v{s} morec.//
\glb Marek-\mk{pat} \mk{exst} hunger//
\glft \trsl{Marek is hungry}//
\endgl
\xe

A full list of adjectives/modifiers that use the existential construction can be found in the section~\ref{sec:exst}.

The copula, however, cannot be ommitted in grammatical moods other than the indicative.

\subsection{Negative copula}

Iridian has the negative copula \ird{\v{c}esná}.

\pex
\begingl
\gla Marek doktor \v{c}esná.//
\glb Marek doctor \mk{cop.neg}//
\glft \trsl{Marek is not (a/the) doctor.}//
\endgl
\xe

\par The inversion of word order may also be used when one wants to avoid ambiguity:

\pex
\begingl
\gla Doktor Marek \v{c}esná.//
\glb doctor Marek \mk{cop.neg}//
\glft \trsl{Marek is not the doctor.}//
\endgl
\xe


\subsection{Conjugation paradigm}
