\chapter{Phonology}\label{ch:phon}

\section{Introduction}

This chapter provides an overview of the phonology of Iridian. The phonetic
descriptions provided here are in IPA based on the standard dialect of Iridian
(as spoken in Roubže and surrounding areas). Divergent phonologies, both within
the Roubže dialect itself and the various dialects inside and outside
Iridia, are discussed in detail in Appendix \ref{ch:dialects}.

\section{Phonetic and phonemic notation}\label{sec:notation}

The phonetic descriptions in this chapter are based on the standard dialect of
Iridian, which is itself based on the Roubže dialect. Phonetic notation uses a
single symbol to represent one and only one sound; in this book, it is based on
the International Phonetic Alphabet (IPA) and appears between square brackets,
e.g., [pʲæɕtäː]. Phonemic notation, on the other hand uses a single symbol to
represent one and only one phoneme; in this book, this appears between forward
slashes, e.g., /pʲaɕtaː/. The phonemic transcription is used to represent the
underlying form of a word, while the phonetic transcription is used to represent
the actual pronunciation of a word. Citations in the Iridian language appear
italicized, and their English translations, if given, appear in single quotes
following the Iridian text, e.g., \irdp{piaštá}{to eat.}

\section{Vowels}\index{vowel}\label{sec:vowels}

\subsection{Oral vowels}\index{vowel!oral}

\begin{table}\index{vowel!inventory}
	\footnotesize\sffamily
	\caption{Vowel inventory of standard Iridian.}
	\medskip
	\begin{tblr}{width=0.6\textwidth,colspec={XXXX}}
		\toprule
					& {\sc front}	& {\sc central}	& {\sc back}	\\ 
		\midrule
		Close 		& ɪ\,i 			& (ɨ)			& ʊ\,uː			\\ 
		Mid 		& ɛ\,eː 		& 				& ɔ\,oː			\\ 
		Open 		& 				&(ɐ)			& a\,aː 		\\ 
		\bottomrule
		\label{table:vowels}{}
	\end{tblr}
\end{table}{}

Iridian has five pairs of corresponding long and short vowels. With the
exception of /a\,aː/, long vowels are tenser than their short counterparts. In
addition standard Iridian also features the high central vowel [ɨ] as an
allophone of /ɛ/ and /ɪ/ and the low central [ɐ] as an allophone of /a/, in
unstressed positions. Phonetic realization is generally consistent with
orthography as seen in Table \ref{table:vowels-orth} below.

\begin{table}
	\footnotesize\sffamily
	\caption{Orthographic representation of vowels.}
	\medskip
	\begin{tblr}{width=0.7\textwidth,colspec={XXXXXX}}
		\toprule 
		& {\sc short} & {\sc long} & & {\sc short} & {\sc long}\\ 
		\midrule 
		/a/ & a 	&á 			& /o/ 	& o &ó 	\\ 
		/e/ & e 	&é 			& /u/ 	& u &ú	\\ 
		/i/ & i,\,y &í,\,ý 		& 		& 	&	\\ 
		\bottomrule
		\label{table:vowels-orth}
	\end{tblr}
\end{table}


Both ⟨i⟩ and ⟨y⟩ and their long counterparts ⟨í⟩ and ⟨ý⟩ represent the high
front vowel /i/. ⟨y/ý⟩ originally represented the high front rounded vowel /y/
(with the short /y/ realized as the tenser near-close near-front rounded vowel
[ʏ]) but the pronunciation gradually shifted to the central front vowel [ɨ]
before finally settling to /i/ in the 14th or 15th century. As in
Czech\index{Czech} orthography, ⟨i, í⟩ causes the palatalization of the
preceding consonant. The same distinction is found between the palatalising ⟨ě⟩
(another Czech loan originally written in Old Iridian as ⟨je⟩) and the normal
⟨e⟩. This is discussed further in the orthography section (\S\,\ref{sec:ortho}).

The short vowels /ɛ/ and /ɪ/ are reduced to [ɨ] in unstressed positions. In less
careful speech, this could cause the elision of the vowel and the formation of
consonant clusters or the realization of the preceding consonant as syllabic
(especially if it is a liquid). Final /ɛ/ is not reduced in a word-final
position if preceding a pause.

\ex
	\irdp{a mert}{and the dead one} [ˈʔämɨɾt̚ ] or [ˈʔämɾ̩t̚] but\\
	\irdp{akuzace}{accusation} [ˈʔäxʊzɐt͡sɛ]
\xe

The low vowel /a/ is realized as the open central unrounded vowel /ä/.
Stressed /a/ is realized as [\ae] between palatal consonants, further reduced to
[ɨ] when unstressed, e.g., \ird{piaštá} ['pʲæɕtäː] vs. \ird{nepiaštá}
[ˈnɛpʲɨɕtäː]. Elsewhere /a/ is pronounced [ɐ] when in an unstressed position,
although some dialects may further reduce it to a [ə].

In most Eastern dialects, especially those from near the border with Poland, the
long mid vowels /eː/ and /oː/ has merged with /iː/ and /uː/, respectively.

\subsection{Diphthongs}\index{diphthong} Iridian has three phonemic oral
diphthongs: \ird{au}\,/au̯/, \ird{ei}\,/eɪ̯/ and \ird{ou}\,/ou̯/. In addition,
the diphthongs \ird{oi}\,/ɔɪ̯/ and \ird{ui}\,/uɪ̯/  also occur phonetically, but
their occurence is marginal, normally appearing only in fixed expressions
(mostly interjections and expletives), such as \irdp{Avui}{Damn it!} [ʔɐˈʋuɪ̯ʔ],
\irdp{pšehui}{annoying} [ˈpʲɕɛxuɪ̯ʔ] and \irdp{Oi}{Hey!} [ʔɔɪ̯ʔ].

In most dialects the diphthong /eɪ̯/ has almost completely merged with \ird{é}
/eː/, although some divergent dialects in the south may realize the diphthong as
[iː] (e.g., \irdp{neite}{word} /ˈneɪ̯tɛ/ but realized as [ˈneːtɛ] or ['ɲiːtɛ]).

\subsection{Vowel Length}\index{vowel length}\index{long vowel|see{vowel
length}}

Vowel length is phonemic in Iridian. Length is represented by an acute
accent\index{acute accent} over the long vowel. The short-long vowel pairs
differ in quality as well as length, with the short vowels being more lax and
the long vowels being tenser in addition to being longer.

\begin{table}
	\footnotesize\sffamily
	\caption{Vowel length and quality.}
	\medskip
	\begin{tblr}{width=0.7\textwidth,colspec={XXX}}
		\toprule
		{\sc archiphoneme} & {\sc lax/short} &{\sc tense/long}\\ \midrule
		/a/	& [ä]	& [äː]		\\
		/e/	& [ɛ]	& [eː]		\\
		/i/	& [ɪ]	& [iː]		\\
		/o/	& [ɔ]	& [oː]		\\
		/u/	& [ʊ] & [uː]		\\
		\bottomrule
	\end{tblr}
\end{table}

Below are some examples of minimal pairs with long and short vowels.

\pex
\vtop{\halign{%
#\hfil& \qquad  #\hfil\cr
\phon{sam}{säm}{barn}			 & \phon{sám}{säːm}{frog} \cr
\phon{mate}{mätɛ}{spoon}		 & \phon{máte}{mäːtɛ}{check mate} \cr
\phon{se}{sɛ}{glass}			& \phon{sé}{seː}{pulp} \cr
\phon{mel}{mɛw}{honey}			& \phon{mél}{meːw}{straw} \cr
\phon{jite}{jɪtɛ}{sheet}		& \phon{jíte}{jiːtɛ}{shade} \cr
\phon{ton}{tɔn}{tongue}			& \phon{tón}{toːn}{tone} \cr
\phon{mur}{mʊr}{gall}			& \phon{múr}{muːr}{mural} \cr
}}

\xe

\section{Consonants}\index{consonants}\label{sec:consonants}

Table \ref{table:fullconsonant} shows a complete list of consonant phonemes in
Standard Iridian, with the allophones appearing in parentheses. In total,
Iridian has 19 consonant phonemes but with 21 additional allophonic variants.
\begin{table}
	\footnotesize\sffamily
	\caption{Full consonant inventory of standard Iridian.}\label{table:fullconsonant}
	\medskip
	\begin{tblr}{width=\linewidth,colspec={X[1.8]XXXX}}
		\toprule
						& {\sc labial}	& {\sc alveolar}	& {\sc palatal}	& {\sc velar}	\\ 
		\midrule
		Plosive			& p~b			& t~d				& c~ɟ 			& k~ɡ 			\\ 
		Nasal			& m~(ɱ)			& n					& ɲ				& (ŋ)			\\ 
		Liquid			&				& ɾ~(ʁ)~l			&	ʎ			&				\\ 
		Sib. Fric.		& 				& s~z	  			& ɕ~ʑ			&				\\ 
		Non-Sib. Fric.	& ʋ				&					& (ç) 			& x~ɣ   		\\ 
		Sib. Affricate  &				& t͡s~(d͡z)			  & t͡ɕ~(d͡ʑ)		&				\\ 
		Non-Sib. Aff. 	&				& 					&			  	& (k͡x~g͡ɣ)		  \\ 
		Approximant 	& (β̞)  		& (ð̞)				  & j			  & (ʍ~w)		  \\ 
		\bottomrule
	\end{tblr}
\end{table}


\subsection{Plosives}

Initial velar stops are affricated when following a pause, so that the pair
/k~ɡ/ is often realized as [k͡x~ɡ͡ɣ]. Some Southeastern dialects, however,
normally realize initial velar stops as aspirated [kʰ~ɡʰ] instead. This sound
change can be traced to the initial aspirated stops \rec{\asp{k}},
\rec{\asp{g}}, \rec{\asp{t}} and \rec{\asp{d}} in Old Iridian weakening to
affricates.\footnote{Old Iridian \rec{\asp{t}} and \rec{\asp{d}} became the
Middle Iridian [t̪͡θ̞ ~d̪͡ð̞] but both have since simplified to /t~d/ in modern
Iridian.} The labial stops /{p~b}/ are unaffected by this process as most
instances of \rec{\asp{p}} and \rec{\asp{b}} have merged to /b/ or /ʋ/ in modern
Iridian.

The velar stops /k~ɡ/ are lenited to the velar fricatives [x~ɣ]
intervocalically, before a voiceless stop, after a vocalised l if followed by
another vowel or a voiceless stop, or before the nasal consonants /n/ or /m/ if
following a vowel immediately. This lenition also occurs word-finally unless
followed by a voiced obstruent, in which case, subject to word-final devoicing,
they merge to [x]. The voiced /ɡ/ itself has a limited distribution, mostly
appearing in consonant clusters with liquids or nasals. Older loanwords (mainly
Slavic, but to a lesser extent Germanic and Hungarian ones) that contain /ɡ/ in
the original language have often been assimilated as \orth{h} in Iridian. (Cf.
for example, \irdp{hrác}{athlete} and Polish \foreign{grać} or Russian
\foreign{\cyrtext играть}.)

This lenition can also be observed with the voiced stops /b/ and /d/ which
become the approximants [β̞	] and [ð̞] (written without the diacritic hereafter)
intervocalically or between a vocalised /l/ and another vowel. Both /b/ and /d/
and the marginal /g/ may be realized with a nasal release at the beginning of a
word when following a pause, i.e., as [ᵐb], [ⁿd] and [ᵑɡ],
respectively.\footnote{Prenasalized stops are unattested in the Roubže dialect
and there is strong evidence that this process is slowly dying out in the other
dialects as well.} 

The glottal stop [ʔ] is often not regarded as a separate phoneme. It can occur
in three cases: (1) before an onset vowel when following a pause, e.g.,
\irdp{avt}{car} [ʔäft]; (2) between two vowels that do not form a diphthong,
e.g., \irdp{naomá}{laundry} ['näʔɔmäː]; or (3) emphatically, especially in
interjections, e.g., \irdp{Oi}{Hey!} [ʔɔɪ̯ʔ], \irdp{Káp!}{Look out!}
\emph{lit.}, \trsl{danger} [k͡xäpʔ].

\subsection{Nasals}
Iridian has three nasal consonants /m~n~ɲ/. /n/ cannot appear before bilabials
and similarly /m/ cannot appear before velars. Both /m/ and /n/ are realized as
[m] before either /ʋ/ or /f/. Before velars /n/ is consistently realized as [ŋ],
although [n] is also possible in emphatic pronunciation or in word boundaries.
The distribution of word-final /n/ is quite limited when compared to word-final
/m/. When assimilating foreign words with final /n/ or /ŋ/, both nasals usually
surface as an /m/ in the new loanword, e.g., \irdp{bedautum}{definition} from
Ger. \foreign{Bedeutung}. 

The velar [ŋ] is not phonemic in Iridian but can sometimes be observed,
especially in loanwords, where it can be realized as nasalization of the
preceding vowel when in the syllable coda or as [ŋ] intervocalically, although
[ŋɡ] or [ŋk] is also common. Thus, for example, \irdp{anglevní}{English} can be
realized as either [ˈɐ̃w̃lɛʋɲiː] or [ˈäŋlɛʋɲiː] or [ˈäŋɡlɛʋɲiː] in order of
currency.

\subsection{Liquids}

Iridian has two liquids: the rhotic /r/ and the lateral /l/.

The rhotic /r/ is realized as the tap [ɾ], although some speakers may pronounce
it as a trill [r], especially in emphatic pronunciation. Both these
pronunciations are transcribed as [r] in this book. In the coda position /r/ is
devoiced to [r̥].

The lateral /l/ is  the velarised alveolar lateral approximant [ɫ]. Nonetheless
the sound has been transcribed throughout as [l]. In the coda position /l/ is
completely vocalized in standard Iridian, becoming [w]. Most southern dialects
nevertheless retain the pronunciation as [ɫ] in this position. The palatalised
/lʲ/ is the palatal lateral approximant [ʎ] and is transcribed as such.

\subsection{Fricatives and Affricates}

The palatal sibilants /ɕ~ʑ/ can be realized as either the palatal [ɕ~ʑ] or the
post-alveolar [ʃ~ʒ] with the former being more common. The same is true with the
palatal affricates /t͡ɕ~d͡ʑ/, realized as either [t͡ɕ~d͡ʑ] or [t͡ʃ~d͡ʒ], with
the former also being more prevalent. In any case, however, this book treats
/ɕ~ʑ/ and /t͡ɕ~d͡ʑ/ as palatals regardless of the actual realization.

The sequence /t͡sɪ/ and /t͡si:/ are realized as [t͡ɕɪ] and [t͡ɕiː] respectively
(viz., \irdp{cigra}{tiger} is realized as [ˈt͡ɕɪɣɾɐ] and not [ˈt͡sɪɣɾɐ]). The
stop fricative sequence [tɕ] can occur in syllable boundaries, although as form
of hypercorrection most speaker may lengthen the initial stop to [tːɕ] or
aspirate it (becoming [tʰ.ɕ]) to further distinguish it from /t͡ɕ/ (cf. e.g.,
\irdp{otša}{cart} [ˈʔɔtːɕɐ] vs \irdp{oča}{bear} [ˈʔɔt͡ɕɐ]).

The voiced affricates /d͡z/ and /d͡ʑ/, written \orth{dz} and \orth{dž},
respectively, are both marginal phonemes. They normally occur as voiced
allophones of  /t͡s/ and /t͡ɕ/ before voiced obstruents. They do occur
phonemically in a few words, though, mostly in loanwords. Nonetheless, in spoken
Iridian loanwords containing [d͡ʑ] or [d͡ʒ] (mostly from English) are realized
by speakers as [ʑ] (e.g., \irdp{džíns}{jeans} [dʑiːns] or more commonly just
[ʑiːns]).

The voiceless labial fricative /f/ is another marginal phoneme, appearing
usually as an allophobe of /ʋ/. Loanwords containing /f/ generally assimilate to
/ʋ/, although most recent borrowings tend to keep the marginal /f/ (cf.
\irdp{Vranca}{France} [vɾant͡sɐ] vs. \irdp{Feizbuk}{Facebook} [feːzbʊx]).

The approximant /ʋ/ is realized as [v] in onsets before vowels and voiced
obstruents (e.g., \irdp{vdinice}{I thought I saw.} [ˈvɟɪnɨt͡sɛ]), as [f] in
onsets before voiceless obstruents (e.g., \irdp{vternou}{bicycle} [ˈftɛɾnou̯]),
and as [ʋ] or [u̯] in coda and elsewhere (e.g., \irdp{pilav}{pilaf} [ˈpʲɪɫäʋ]
or [ˈpʲɪɫäu̯]). The sequence /kʋ/ and /ɡʋ/ is further lenited to the labialised
velar fricatives [xʷ~ɣʷ]. The voiceless [xʷ] (from both \orth{kv} and \orth{hv})
is in free variation with [ʍ], with the latter being the more common
pronunciation, especially among younger speakers. For simplicity both [xʷ] and
[ʍ] will be transcribed as [ʍ].

Modern Iridian has lost the distinction between /h/ and /x/, with both \orth{ch}
and \orth{h},\footnote{Most instances of \orth{ch} have been replaced with
\orth{h} following various spelling reforms.} historically representing /x/ and
/h/, respectively, merging to the velar fricative /x/. This becomes /ç/ before
voiceless stops word-initially or when following a front vowel, or before the
front vowels /i/ and /ɪ/. The sequence \orth{hl} and \orth{kl} are realized as
/t͡ɬ/.

\section{Phonotactics}\index{phonotactics}\label{sec:phonotactics}

\subsection{Syllable structure}\index{syllable
structure}\label{sec:syllable-structure}

Ignoring the possible complexity of the onset, nucleus or coda, the basic
structure of an Iridian syllable is CV(C), with C representing a consonant and V
a vowel.\footnote{An alternative view, founded upon the status of the glottal
stop as a non-phoneme in Iridian, would be to consider the basic structure as
(C)V(C) instead of CV(C), thus allowing for a null onset. This treats the
addition of a glottal stop in word-initial syllables starting with a vowel as
mere prothesis.} Iridian has relatively few phonotactic constraints, allowing,
at a maximum, syllables of the form CCCCVCCC. Nevertheless, most syllables fall
in either of the five groups CV, CVC, CCV, CCVC and CVCC

\begin{table}
	\footnotesize\sffamily
	\caption{Blevin's criteria as they apply to Iridian.}
	\medskip
	\begin{tblr}{width=0.6\textwidth, colspec={XX}}
		\toprule 
		& {\sc parameter}\\ 
		\midrule 
		Obligatory onset & Yes\\ 
		Coda & No\\ 
		Complex onset & Yes\\ 
		Complex nucleus & Yes*\\ 
		Complex coda & Yes\\ 
		Edge effect & \\ 
		\bottomrule
	\end{tblr}
\end{table}


\subsection{Onset}

All consonant and vowel phonemes can appear in a syllable's onset. Iridian does
not allow a null onset (vowel in the syllable onset), i.e., the most basic
Iridian syllable should be of the form CV. Words that superficially appear as
having a null onset syllable in the initial position are actually preceded by a
glottal stop. An epenthetic glottal stop is also added between vowels in a
sequence that do not otherwise form dipthongs, or before a vowel in a
word-initial position in loanwords. Despite this, vowel-words are significantly
rarer in comparison to consonant-initial ones.

\ex
Prothetic [ʔ] in native Iridian words:\\
\irdp{a}{and} [ˈʔä]\\
\irdp{umielá}{to get drunk} [ˈʔʊmʲɨläː]\\
\irdp{eg}{eyes} [ʔɛx]
\xe

\ex
Prothetic [ʔ] in loanwords:\\
\irdp{Americe}{Amerika} [ˈʔämɨɾʲɪt͡sɛ]\\
\irdp{autobus}{bus} [ˈʔau̯tɔβʊs] \\
\irdp{elefant}{elephant} [ˈɛlɨˌfänt]
\xe

In some eastern dialects, a prothetic [m] is added instead of [ʔ] on words that
begin with vowels after a pause. This never occurs on loanwords or before the
front vowels /e/ and /i/ and has been largely in decline, especially among
younger speakers. With some speakers, the prothetic [m] may be realized as [mw].

\ex
\irdp{umielá}{to get drunk} [ˈmʊmʲɨläː] or [ˈmwʊmʲɨläː]\\
\irdp{očat}{bug} [ˈmɔt͡ɕɐt] or [ˈmwɔt͡ɕɐt]
\xe

A more widespread pattern in colloquial Iridian is the addition of a prothetic
/j/ before the front vowels /e/ and /i/. This phenomenon could be observed in
both native words and loans.

\ex
\irdp{Evrope}{Europe} [ʔɛʋɾɔpɛ], colloq. [jɛʋɾɔpɛ] \\
\irdp{éh}{eyes} [ʔeːx], colloq. [jɛx]\\
\irdp{éšte}{of course} [ˈʔeːɕtɛ], colloq. [ˈjeːɕtɛ]
\xe


The following CC clusters are allowed to be in onset position:

\pex
\a Stop followed by a liquid:\\
/pr/: \irdp{pragy}{sand} [präc]; \irdp{pramou}{petal} [ˈpɾämou̯]\\
/tr/: \irdp{trava}{bread} [ˈtɾävɐ]; \irdp{truk}{ball} [tɾʊx]\\
/kr/: \irdp{krova}{egg} [ˈkɾɔvɐ]; \irdp{kramy}{toe} [kɾämʲ]\\
/pl/: \irdp{plán}{plan} [pläːn]; \irdp{plúka}{knot} [ˈpluːxɐ]\\
/kl/: \irdp{kluk}{foot} [t͡ɬʊx]; \irdp{klúbe}{club} [ˈt͡ɬuːβɛ]\\
/br/: \irdp{bírok}{female teenager} [bʲiːɾɔx]; \irdp{bremy}{prise} [bɾɛmʲ]\\
/dr/: \\
/gr/: \irdp{grec}{flag} [ɣɾɛt͡s]; \irdp{greny}{peace} [ɣɾɛɲ]\\
/bl/: \irdp{bloht}{mud} [blɔxt̚]; \irdp{blau}{neck} [blau̯]\\
/dl/:
\xe

\section{Suprasegmentals}\index{suprasegmentals}

\subsection{Stress}\index{stress} Iridian words generally have a single primary
stress, falling on the first syllable, no matter if the word is simple (e.g.,
\irdp{študent}{student}), derived (e.g., \irdp{študenta}{student, pat.}) or
compound (e.g., \irdp{študentrád}{dormitories}). Most loanwords follow this
general pattern, although more recent borrowings, especially those referring to
proper names, show a greater tendency to keep the phonology of the source
language and not fully assimilate to Iridian's initial stress rule.

\pex
\a Loanwords showing assimilation to word-initial stress:\\
\phon{aristókrat}{ˈäɾɨstoːxɾɐt}{aristocrat}\\
\phon{koruna}{ˈk͡xɔɾʊnä}{crown}

\a Loanwords
\xe

Clitics\index{clitic} are not considered phonologically distinct and are treated
as belonging to the same phonological word as the one after them. These include:

\begin{enumerate}[noitemsep,label=(\alph*)]
	\item Most monosyllabic and some disyllabic prepositions
	\item Most conjunctions:
	\item The pluralizing particle \ird{nie} and the negative particle
	\ird{zám}: 
	\item Demonstratives and the weak form of personal pronouns
\end{enumerate}

\subsection{Intonation}\index{intonation}

\section{Phonological Processes Involving Vowels}

\subsection{Vowel\,\sim\,Zero Alternations}

A vowel\,\sim\,zero alternation occurs when a vowel alternates with zero (i.e.,
gets deleted) in certain morphological contexts. We call this deleted vowel
`unstable' (cf. \cite{siptar2000}, \cite{gussmann2007}). Vowel\,\sim\,zero
alternations in Iridian are virtually all instances of [ɛ] deletion. This
process occurs in roots of the type --C(\sx{j})eC where C is a consonant loses
its [ɛ] when it is followed by a suffix beginning with a vowel.

\ex
\vtop{\halign{%
#\hfil& &\qquad  #\hfil\cr
\irdp{Janek}{Janek} & \irdp{Janka}{Janek (acc.)}	\cr
\irdp{obel}{window} & \irdp{oblu}{window (inst.)} 	\cr
\irdp{pizen}{coin} 	& \irdp{pězní}{coin (inst.)} not \ird{\sx{*}pizní} \cr
}}
\xe

[ɛ] deletion in monosyllabic roots of the type (C)C(\sx{j})eC is subject to
further constraints:

\begin{enumerate}
	\item If the initial consonant is a stop, the [ɛ] is deleted only if the final consonant is a nasal or a fricative:\\
	\vtop{\halign{%
	#\hfil& &\qquad  #\hfil\cr
	\irdp{den}{mirror} & \irdp{dnu}{mirror (inst.)}\cr
	\irdp{pěr}{roof} & \irdp{pěru}{roof (inst.)} not \ird{\sx{*}pru} \cr
	}}

	\item If the initial consonant is a fricative, the [ɛ] is deleted only if the final consonant is a nasal:\\
	\vtop{\halign{%
	#\hfil& &\qquad  #\hfil\cr
	\irdp{ver}{spring}  & \irdp{veru}{spring (inst.)} not \ird{\sx{*}vru} \cr
	\irdp{hen}{crumb}  & \irdp{hnu}{crumb (inst.)} \cr
	}}

	\item The [ɛ] is not deleted in all other cases.
\end{enumerate}

\subsection{Vowel\,\sim\,Vowel Alternations}
Vowel\,\sim\,vowel alternations (also called `ablaut') occurs when one vowel is
substituted for another in some morphophonological contexts. Vowel\,\sim\,vowel
alternations in Iridian can be broadly classified into two types: [ɛ]
substitution and vowel raising.

Roots of the type --C\sx{j}aC(C) and --C\sx{j}oC(C) become --C\sx{j}eC(C) in the
presence of palatalizing suffixes:

\ex
\vtop{\halign{%
#\hfil& &\qquad  #\hfil\cr
\irdp{bial}{money}			& \irdp{bielí}{honey (gen.)}
							& \irdp{biala}{honey (acc.)}\cr
\irdp{šviak}{soldier}	& \irdp{šviecí}{soldier (gen.)}
							& \irdp{šviaka}{soldier (acc.)}\cr
\irdp{pion}{nest}			& \irdp{piení}{nest (gen.)}
							& \irdp{piona}{nest (pat.)}\cr
\irdp{kážol}{threat}  & \irdp{káželí}{threat (gen.)}
							& \irdp{kážola}{threat (pat.)}\cr
}}
\xe

Vowel-raising alternations are generally triggered by the deletion of an
unstable vowel in the final syllable of the root. The front vowels [eː], [eɪ̯]
and [ʲɛ] in the penultimate syllable merge with the high front vowel [iː]. The
back vowels [ɔ] and [ou̯], on the other hand, merge with [ʊ]. The ablaut does
not occur where the penultimate syllable is also the first syllable of the root
and the root has a null onset.

\ex
\vtop{\halign{%
#\hfil& &\qquad  #\hfil\cr
\irdp{lobek}{apple}		& \irdp{lubka}{pat.} 			& not \ird{*lobka}\cr
\irdp{kostel}{fish}		& \irdp{kustlár}{fisherman}	& not \ird{*kostlár}\cr
\irdp{pěštel}{falcon}	& \irdp{píštlár}{falconer}	& not \ird{*pěštlár}\cr
\irdp{obel}{window}	& \irdp{oblí}{window (gen.)}	& not \ird{*obelí} or \ird{ublí}\cr
}}
\xe


\subsection{Compensatory vowel lengthening}

Compensatory vowel lengthening is a process whereby a short vowel is lengthened
to compensate for the loss of another 

\section{Phonological Processes Involving Consonants}

Iridian consonants are generally affected by two systems of phonological
opposition: a primary distinction between voice and unvoiced consonants, and a
secondary distinction between hard and soft consonants (i.e., normal and
palatalised consonants).

\subsection{Voicing}
Consonant voicing is phonemic. Voiced consonants are called muddy or dark
(\ird{měrkní}) while unvoiced consonants are called clear (\ird{hezkní}).
Most of the obstruents in Iridian come in pairs distinguished only by voicing:

\pex
\a /k/ \phon{kapa}{k͡xäpɐ}{cape} vs /g/ \phon{gapa}{ɡ͡xäpɐ}{liquor}
\a /p/ \phon{pac}{pät͡s}{stick} vs /b/ \phon{bac}{bät͡s}{underside}
\a /t/ \phon{tám}{täːm}{more} vs /d/ \phon{dám}{däːm}{by me}
\xe

Another basic rule of consonant voicing is that in a cluster the last consonant
usually determines whether the preceding ones are voiced or not.\index{voicing
assimilation} Note however that although the liquids /r/ and /l/ and the nasals
/m/ and /n/ are intrinsically voiced, they do not cause the preceding consonant
to assimilate.

\pex
\a\phon{nazka}{ˈnäskɐ}{powder (acc.)} \a\phon{nikda}{ˈɲɪɡdɐ}{fever}
\a\phon{zkáte}{skäːte}{patient} \a\phon{slěň}{ɕʎɛɲ}{soup}
\xe

\subsection{Palatalization}

Iridian consonants can either be hard or soft. Consonants are hard by default
but become soft when followed by the vowels \orth{i} or \orth{í}. The vowel
\orth{y} and \orth{ý} on the other hand are used to indicate non-palatalizing
[ɪ] and [iː] respectively. (Compare, for example, \phon{být}{biːt}{cough} and
\phon{bít}{bʲiːt}{cup}.)

Softening involves palatal articulation of labial consonants (e.g.,
\ird{be}~[bɛ] vs \ird{bě}~[bʲɛ] or the change to a palatal consonant for
non-labials (e.g., \ird{te}~[tɛ] vs \ird{tě}~[cɛ]). Table \ref{table:softhard}
shows how non-labials are affected by palatalization in Iridian.

\begin{table}
	\footnotesize\sffamily
	\caption{Alternations caused by consonant softening}
	\medskip
	\begin{tblr}{width=0.7\textwidth,colspec={XXX}}
		\toprule 
		{\scshape hard} & {\scshape soft with ě} & {\scshape soft with a}\\ 
		\midrule 
			\ird{b}~[b] 	& \ird{bě}~[bʲɛ]	&\ird{bia}~[bʲɐ]\\ 
			\ird{p}~[p] 	& \ird{pě}~[pʲɛ]	&\ird{pia}~[pʲɐ]\\ 
			\ird{d}~[d] 	& \ird{dě}~[ɟɛ]		&\ird{dia}~[ɟɐ]\\ 
			\ird{t}~[t] 	& \ird{tě}~[cɛ]		&\ird{tia}~[cɐ]\\ 
			\ird{f}~[f] 	& \ird{fě}~[fʲɛ]	&\ird{fia}~[fʲɐ]\\ 
			\ird{v}~[v] 	& \ird{vě}~[vʲɛ]	&\ird{via}~[vʲɐ]\\ 
			\ird{k}~[k] 	& \ird{kě}~[cɛ]		&\ird{kia}~[cɐ]\\ 
			\ird{g}~[ɡ] 	& \ird{gě}~[ɟɛ]		&\ird{gia}~[ɟɐ]\\ 
			\ird{s}~[s] 	& \ird{še}~[ɕɛ]		&\ird{ša}~[ɕɐ]\\ 
			\ird{z}~[z] 	& \ird{že}~[ʑɛ]		&\ird{ža}~[ʑɐ]\\ 
			\ird{h}~[h] 	& \ird{hě}~[çɛ]		&\ird{hia}~[çɐ]\\ 
			\ird{c}~[t͡s]	 & \ird{če}~[t͡ɕɛ]	  &\ird{ča}~[t͡ɕɐ]\\ 
			\ird{m}~[m] 	& \ird{mě}~[mʲɛ]	&\ird{mia}~[mʲɐ]\\ 
			\ird{n}~[n]		& \ird{ňa}~[ɲɛ]		&\ird{ňa}~[ɲɐ]\\ 
			\ird{l}~[l] 	& \ird{lě}~[ʎɛ]		&\ird{lia}~[ʎɐ]\\ 
			\ird{r}~[r] 	& \ird{rě}~[rʲɛ]	&\ird{ria}~[rʲɐ]\\ 
		\bottomrule
	\end{tblr}
	\label{table:softhard}
\end{table}

The assimilation of the preceding consonant to the soft consonant in a consonant
cluster is often observed but it is not represented in the orthography. For
example in \irdp{slěň}{soup} the initial /s/ would assimilate to the soft /l/
and so the word is realized [ɕʎɛɲ] instead of [sʎɛɲ]; the spelling however
remains as \ird{slěň} and not \ird{*šlěň}.

Palatal assimilation only operates leftwards; i.e., a soft consonant at the
start of a cluster would not cause the following consonant to become soft (e.g.,
\irdp{štotnik}{(he) was arrested} is realized as [ɕtɔcɲɪx] and not [ɕcɔcɲɪx]). 


\section{Orthographic representation}\label{sec:ortho}
\subsection{Alphabet}

The Iridian language uses the Latin script with the following 31 letters:
\ird{a, b, c, č, d, e, ě, f, g, h, i, j, k, l, m, n, ň, o, p, q, r, s, š, t, u,
v, w, x, y, ý, z, ž}.

The language was originally written in its own script but after the Latin
alphabet has been adapted and has been in use since the First Bohemian Union in
the 14th century. Due to the historical ties with the Kingdom of Bohemia and its
historical successors, Czech orthography has had a great influence on the
orthography of Iridian and is the direct inspiration for the current
orthography. The main differences between the two include the lack of the
letters ď, ř, ť, and ů. The sound represented by the letter
ř does not exist in Iridian.

The Cyrillic script coexisted with the Iridian Latin alphabet from the 12th
until the early 16th century. Today Cyrillic is still used to write the
Ukrainian dialects of Iridian.

\begin{table}
	\footnotesize\sffamily
 	\caption{The letters of the Iridian alphabet and their corresponding phonemes.}\index{alphabet}
	\medskip
	\begin{tblr}{width=0.9\textwidth,colspec={XX[1.3]XXX[1.3]X}}
		\toprule
		{{\sc  symbol}} & {\sc name} & {\sc ipa} & {{\sc  symbol}} 	& {\sc name}& {\sc ipa}\\
		\midrule
		A a	  			& á 		 & /a/       &  Ň ň				& eň 		& /ɲ/\\
		B b				& bé		 & /b/       &  O o				& ó			& /o/\\
		C c				& cé		 & /t͡s/      &  P p			 & pé		 & /p/\\
		Č č				& čá		 & /t͡ɕ/      &  Q q			 & kú		 & -\\
		D d				& dé		 & /d/       &  R r			 	& er		& /r/\\
		E e				& é		 	 & /ɛ/       &  S s				& es		& /s/\\
		Ě ě				& jé		 & /jɛ/      &  Š š				& eš		& /ɕ/\\
		F f				& fí		 & /f/       &  T t				& té		& /t/\\
		G g				& gá		 & /ɡ/       &  U u				& ú			& /u/\\
		H h				& há		 & /x/       &  V v				& vé		& /v/\\
		I i				& í		 	 & /i/       &  W w				& vênek		& -\\
		J j				& jota		 & /j/       &  X x				& íks		& -\\
		K k				& ká		 & /k/       &  Y y				& ipsylon   & /i/\\
		L l				& el		 & /l/       &  Z z				& zet		& /z/\\
		M m				& em		 & /m/       &  Ž ž				& žeš		& /ʑ/\\
		N n				& en		 & /n/       &    				& 			& \\
		\bottomrule
	\end{tblr}
\end{table}


\begin{table}
	\footnotesize\sffamily
 	\caption{Correspondence between the Iridian Latin and Cyrillic scripts.}\index{alphabet}
	\medskip
	\begin{tblr}{width=0.8\textwidth,colspec={XXXX}}
		\toprule
		{{\sc  latin}} & {\sc cyrillic} & {{\sc  latin}} & {\sc cyrillic} \\
		\midrule
		A a 		& А а	& O o   & О о \\ 
		B b			& Б Б 	& P p 	& П п \\
		C c 		& Ц ц 	& Q q 	& -- \\
		Č č 		& Ч ч 	& R r 	& Р р \\
		D d 		& Д д	& S s 	& С с \\
		E e 		& Е е 	& Š š 	& Ш ш \\
		F f			& Ф ф	& Tt 	& Т т \\
		G g 		& Г г	& Uu	& У у \\
		H h			& Х х	& V v   & В в\\
		I i			& И и	& W w   & --\\
		J j			& --	& X x   & --\\
		K k			& К к	& Y y   & Ы ы\\
		L l			& Л л   & Z z   & З з\\
		M m 		& М м   & Ž ž   & Ж ж\\
		N n 		& Н н   &		&\\
		\SetCell[c=4]{} Letters unique to the Cyrillic script \\
		Dz dz 		& Ѕ ѕ 	& Dž dž & Џ џ\\
		/ja/ 		& Я я	&/je/ & Є є\\
		/jo/		& Ю ю   & &\\
		Ą ą & Ѫ ѫ&Ę ę&Ѧ ѧ\\ 
		\bottomrule
	\end{tblr}
\end{table}
 
Iridian uses two diacritics: the caron (◌̌) and the acute accent (◌́). The caron
is used to indicate palatalization of a consonant or, in the context of
\orth{ě}, of the preceding consonant. When the letter \orth{ě} is used after the
consonants \orth{c}, \orth{n}, \orth{s} and \orth{z}, the caron is used only
with \orth{ě} and although they are effectively palatalized, the preceding
consonants are not marked. Thus one writes \orth{ně} instead of \orth{\sx{*}ňe}
or \orth{\sx{*}ňě}. The accute accent is used with vowels and the letter
\orth{y} to indicate a long vowel.

\subsection{Orthographic Conventions}
Iridian spelling is fairly regular.

