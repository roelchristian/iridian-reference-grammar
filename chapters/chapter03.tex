\chapter{Nominal Morphology}

Nominal morphology in Iridian is relatively simpler compared to the corresponding process with verbs. Where possible, Iridian sentences are generally constructed to leave the noun or noun phrase unmarked.

\section{Grammatical Categories}

\section{Number}\index{grammatical number}\index{plural}

Nouns in Iridian are not formally marked for number. Thus the word \ird{byl}, for example, can mean either \trsl{child} or \trsl{children} depending on the context. The same form is used when the noun is preceded by a numeral.

\pex
\begingl
\gla hron\'a byl//
\glb three child//
\glft \trsl{three children}//
\endgl
\xe

Nevertheless, Iridian can express semantic plurality by using quantifiers, numerals, pluralizing particles or even through context alone. One such particle is \ird{nie}\label{sec:plurals}\footnote{Cf. Schachter's treatment of Tagalog pluralizing particle \emph{mga}.}. \ird{Nie} is a proclitic and attaches to the left-most part of the noun phrase or the verb phrase it modifies.

\pex
\begingl
    \gla nie \v{s}a zu\v{s}tal\'i byl//
    \glb \mk{pl=} \mk{dem.prox} be:happy-\mk{att} child //
    \glft \trsl{these happy children}//
\endgl
\xe

\ird{Nie} however could be understood to have three distinct uses. The first, as mentioned above, is to mark plurality. Alternatively, \ird{nie} could also be use as an approximative\index{approximation} (roughly equivalent to English \trsl{about}) when used with cardinal numbers or time expressions or as a honorific expletive\index{honorific}\index{expletive} to show politeness when used with proper names\index{proper names} or with some nouns (mostly related to kinship terms\index{kinship terms}). In its use for approximation, \ird{nie} is interchangeable with \irdp{u}{about}, although it is common in spoken speech to combine the two as an intensified construction. Preference is given to \ird{nie}, however, if the noun being modified is the topic of the sentence and must therefore remain unmarked.

\pex
\begingl
    \gla Nie mlaz-no scen\v{z}ek?//
    \glb \mk{hon=} brother\mk{=q} arrive-\mk{av-pf} //
    \glft \trsl{Was my brother the one who arrived?}//
\endgl
\xe
\pex
\begingl
    \gla Nie mlaz-no scen\v{z}ek?//
    \glb \mk{hon=} brother\mk{=q} arrive-\mk{av-pf} //
    \glft \trsl{Was my brother the one who arrived?}//
\endgl
\xe

\pex
\a
\begingl\deftagex{appr}\deftaglabel{1}
    \gla Nie hron\'a byl//
    \glb \mk{approx=} three child //
    \glft \trsl{about three children}//
\endgl
\a
\begingl
    \gla u hron\'a bylu//
    \glb about three child-\mk{inst} //
    \glft \trsl{about three children}//
\endgl
\a
\begingl
    \gla u nie hron\'a bylu//
    \glb about \mk{approx=} three child-\mk{inst}//
    \glft \trsl{about three children}//
\endgl
\xe

Note that when used with a cardinal number, \ird{nie} can only be understood to signify approximation, i.e., (\getfullref{appr.1}) can only mean \trsl{about three children} and not \trsl{three children}, as the latter would only be translated as \ird{hron\'a byl} without the clitic \ird{nie}.

As has been earlier mentioned, \ird{nie} is a proclitic\index{proclisis} and attaches to the left-most part of the noun phrase or verb phrase it modifies, including any modifier no matter how complex but excluding any proposition. In some cases, as can be seen in (b) and (c) below, the use of \ird{nie} to pluralize a noun can imply definiteness\index{definiteness}.

\pex
\a
\begingl\deftagex{pl}\deftaglabel{1}
    \gla \textbf{nie} za byla t\'om//
    \glb \mk{pl=} for child-\mk{pat} child //
    \glft \trsl{books for children}//
\endgl
\a
\begingl\deftaglabel{2}
    \gla za \textbf{nie} byla t\'om//
    \glb for \mk{pl=} child-\mk{pat} child //
    \glft \trsl{a book for (these) children}//
\endgl
\a
\begingl\deftaglabel{3}
    \gla \textbf{nie} za \textbf{nie} byla t\'om//
    \glb \mk{pl=} for \mk{pl=} child-\mk{pat} child //
    \glft \trsl{books for (these) children}//
\endgl
\xe





The use of \ird{nie}, however, is largely optional and where plurality can be implied from context, this particle is seen as redundant and is therefore dropped.

\pex
\begingl
\gla Nie byl zap\'o\v{c}ek.//
\glb \mk{pl} child laugh-\mk{av-pf}//
\glft \trsl{The children jumped.}//
\endgl
\xe

\ird{Nie} cannot be used with mass and uncountable nouns, as well as with abstract nouns.

\pex
\a
\begingl
\gla *Na duma nie je\v{s} pia\v{s}tou.//
\glb \mk{loc} house \mk{pl} \mk{exst} food//
\glft \trsl{There is food in the house.}//
\endgl
\a
\begingl
\gla Na duma tohle je\v{s} pia\v{s}tou.//
\glb \mk{loc} house much \mk{exst} food//
\glft \trsl{There is a lot of food in the house.}//
\endgl
\xe

The particle \ird{nie} always precedes the noun it modifies, except in existential clauses where it comes before the existential particle \ird{je\v{s}}\footnote{The sequence is pronounced as if written n\'ije\v{s} \nt{"ni:jEC}}. \ird{Nie} can obviously not be used with the negative particle \ird{niho}.\index{niho}\index{existential construction}\index{je\v{s}}

\pex
\a
\begingl
\gla nie b\v{z}\k{e}//
\glb \mk{pl} bee//
\glft \trsl{bees}//
\endgl
\a
\begingl
\gla Nie je\v{s} b\v{z}\k{e}.//
\glb \mk{pl} \mk{exst} bee//
\glft \trsl{There are bees.}//
\endgl
\a
\begingl
\gla *Nie niho b\v{z}\k{e}.//
\glb \mk{pl} \mk{exst.neg} bee//
\glft \trsl{There are no bees.}//
\endgl
\xe

\index{pluralia tantum}
\ird{Nie} cannot be used with a limited number of nouns, mostly referring to paired body parts and related objects, which in the base form is understood to refer to the pair itself and thus cannot be pluralized. If the speaker wishes to explicitly refer to one piece of the pair, the noun \ird{noma} (an obsolete form of the word for one-half, now surviving only in this construction) and the genitive form of the body part.

\pex
\begingl
\gla Eg zaromnek.//
\glb eyes close-\mk{pv-pf}//
\glft \trsl{(He) closed (his) eyes.}//
\endgl
\xe
\pex
\begingl
\gla Poh\'ar d\'evit.//
\glb eyeglasses dirty//
\glft \trsl{(Your) eyeglasses are dirty.}//
\endgl
\xe
\pex
\begingl
\gla Ohv\'i noma utie\v{s}\v{c}\'al.//
\glb shoe-\mk{gen} half \mk{ref-}lose-\mk{av-cont}//
\glft \trsl{The other pair of (his) shoe is missing.}//
\endgl
\xe

The base form is also used in generic statements where English would normally use the plural.\index{generic statements}\index{universals}


When used with a proper noun \ird{nie} can be translated with the English construction \trsl{and others}. Note that this is different from the usage of \ird{nie} as a honorific.

\pex
\begingl
    \gla Nie Jancie gna\v{z} uprub\'i\v{z}ice.//
    \glb \mk{pl=} Janek-\mk{gen} school \mk{ref}-burn-{av-pf-quot} //
    \glft \trsl{I heard Janek's school burned down.}//
\endgl
\xe

\pex
\begingl
    \gla Nie Marek z\'azdal\v{s}ek..//
    \glb \mk{pl=} Marek \mk{neg}-have:breakfast-{av-pf} //
    \glft \trsl{Marek and the others did not eat breakfast.}//
\endgl
\xe


\section{Definiteness}
Iridian does not have definite or indefinite articles

\section{Uninflected form}

\section{Agentive case}\index{agentive case}

\subsection{Agentive of comparison}\index{comparison}\index{agentive of comparison}
\pex
\begingl
\gla D\'a Mark\k{a} t\'am stroja.//
\glb \mk{1s.str} Marek-\mk{agt} \mk{comp} tall//
\glft \trsl{Marek is taller than me}//
\endgl
\xe

\section{Patientive case}

The patientive case (glossed \mk{pat}) is formed by appending the suffix \ird{-a} to the root of the noun, subject to the following sound changes, notably affecting vowel-final roots for the most part:

\begin{itemize}
	\item Roots ending in e and o replace the final vowel with \ird{-a}: \ird{pivo -- piva} \trsl{beer}, \ird{malno -- malna} \trsl{language}, \ird{\v{s}uze -- \v{s}uza} \trsl{judge}
	\item Roots ending in \'o and ou replace the final vowel with \ird{-\'ova}: \ird{pia\v{s}tou -- pia\v{s}t\'ova} \trsl{food}, \ird{jav\'o -- jav\'ova} \trsl{lizard}, \ird{metr\'o -- metr\'ova} \trsl{subway}
	\item Roots ending in a lengthen the final vowel to \ird{-\'a}: \ird{cigra -- cigr\'a} \trsl{tiger}, \ird{husa -- hus\'a} \trsl{street}
	\item Roots ending in \'a replace the final vowel with \ird{\'anie}: \ird{kom\'a -- kom\'anie} \trsl{boat}, \ird{vietr\'a -- vietr\'anie} \trsl{pants}
	\item Roots ending in \'e, ei and i replace the root with \ird{-\'ena}: \ird{k\'av\'e -- k\'av\'ena} \trsl{coffee}, \ird{matei -- mat\'ena} \trsl{motorbike}
	\item Roots ending in \'i append \ird{na}:
	\item Roots ending in u or \'u append \ird{-\v{s}a}:
\end{itemize}

\subsection{Direct object}
The patientive case is used to mark the direct object of a verb that is in the agentive voice. Note that this usage implies that the direct object is indefinite unless the noun is further qualified (except through a demonstrative).

\pex
\a
\begingl
\gla Va\v{s}ka pia\v{s}\v{c}em.//
\glb cake-\mk{pat} eat-\mk{av-pf-1s}//
\glft \trsl{I ate cake.}//
\endgl
\a
\begingl
\gla Jed\'a va\v{s}ka pia\v{s}\v{c}em.//
\glb that cake-\mk{pat} eat-\mk{av-pf-1s}//
\glft \trsl{I ate from that cake.}//
\endgl
\a
\begingl
\gla Va\v{s}ko pia\v{s}tnikem.//
\glb cake eat-\mk{pv-pf-1s}//
\glft \trsl{I ate the cake.}//
\endgl
\a
\begingl
\gla Jed\'a va\v{s}ko pia\v{s}tnikem.//
\glb that cake eat-\mk{pv-pf-1s}//
\glft \trsl{I ate that cake.}//
\endgl
\a
\begingl
\gla Hron\'a va\v{s}ke vat\'a pia\v{s}\v{c}em.//
\glb three cake-\mk{gen} slice-\mk{pat} eat-\mk{pv-pf-1s}//
\glft \trsl{I ate three slices of cake.}//
\endgl
\xe

The patientive is also used to mark the direct object when the verb is in the benefactive voice.

\pex
\begingl
\gla \v{S}a vitamina pia\v{s}tebik.//
\glb \mk{3s.anim} vitamin-\mk{pat} eat-\mk{ben-pf}//
\glft \trsl{(She) made him take (his) vitamins.}//
\endgl
\xe

%%%%
% TODO Definiteness and the patientive; use of genitive when the noun marked is indefinite


\subsection{Locative}

The patientive is used with the particle \ird{na} to form a compound locative case, which is itself used to indicate a general location.

\pex
\begingl
\gla Tom\'a\v{s} na byra.//
\glb Tom\'a\v{s} \mk{loc} office-\mk{pat}//
\glft \trsl{Tom\'a\v{s} is at the office.}//
\endgl
\xe

\subsection{Patientive of purpose}

The patientive is used with the particle \ird{za} to indicate

\subsection{Lative}
The lative is a compound case indicating movement into or to the direction of something. It is formed using the particle \ird{de} and a noun or noun phrase in the patientive case.

\subsection{Adessive}
The adessive is formed when the particle \ird{u} is used with the patientive. This compound case indicates that the noun being modified by the noun in the adessive is near or in the vicinity of the noun in the adessive. The adessive case behaves synactically in the same manner as the locative case with na in all cases.

\pex
\begingl
\gla Tom\'a\v{s} u byra.//
\glb Tom\'a\v{s} \mk{ade} office-\mk{pat}//
\glft \trsl{Tom\'a\v{s} is somewhere near the office.}//
\endgl
\xe

The adessive case is also used to approximate time.

\pex
\begingl
\gla Ova\v{z} u 19 \'ora.//
\glb dinner \mk{ade} 19 hour-\mk{pat}//
\glft \trsl{Dinner is around seven.}//
\endgl
\xe

\section{Genitive Case}\index{genitive}

The genitive (glossed \mk{gen}) is formed by appending the suffix \ird{-e} to the root of a noun.

Due the palatalizing nature of the suffix, the following sound changes must be noted:

\begin{itemize}
	\item Roots ending in k, h, and t change the final consonant to c and append the glide \ird{-ie} instead: \ird{Marek -- Marcie} \trsl{Marek}, \ird{avt -- avcie} \trsl{car}, \ird{duh -- ducie} \trsl{head}
	\item Roots ending in d and g change the final consonant to \v{z} and append the suffix \ird{-e} instead: \ird{vod -- vo\v{z}e} \trsl{sister}, \ird{seg -- se\v{z}e} \trsl{flower}
	\item Roots ending in the sibilants s, z, \v{s}, \v{z} and the sibilant affricates c and \v{c} append \ird{e} as well:
	\item Roots ending with a palatalized consonant lose the final y (there only for orthographic reasons in any case) before appending the \ird{-\'i}: \ird{kra\v{s}toly -- kra\v{s}tol\'i}
	\item Roots ending in a or o replace the vowel with e, while those ending in \'a and \'o replace the root with \'i
	\item Roots ending in au, ou, or u replace the vowel with -\'ov\'i: \ird{dnou -- dn\'ov\'i} \trsl{front}
	\item Roots ending in \'au, or \'u replace the vowel with -\'ovie
	\item Roots ending in e, i or \"y replace the vowel with -ev\'i
	\item Roots ending in \'e, ei, \'i or \'y replace the vowel with -\'ev\'i
\end{itemize}


\subsection{Genitive of Possession}\index{possesive}\index{genitive}

The simplest use of the genitive case is to indicate ownership or possession.
When used this way, the noun marked in the genitive must always procede the noun
it modifies.

\pex
\irdp{Marcie dum}{Marek's house}\\
\irdp{m\'amcie ha\v{s}ek}{my mother's bag}\\
\irdp{\v{s}a \v{s}tudencie t\'om}{this bb}
\xe

Demonstratives\index{demonstrative} and other modifers must always come before
the whole noun phrase and cannot split the possessor from the possessee. An
exception to this rule is the clitic \ird{nie}, which comes immediately before
the noun it pluralizes\index{plural}.

\pex
\a  \irdp{\v{s}a \v{s}tudencie t\'om}{the/a book of this student}\\
    \irdp{to \v{s}tudencie t\'om}{this book of the student}
\a  \irdp{nie \v{s}tudencie t\'om}{the students' book}\\
    \irdp{\v{s}tudencie nie t\'om}{the student's books}
\xe

\subsection{Partitive Genitive}\index{partitive}\index{genitive}

\subsection{Genitive of material}

\pex
\begingl
\gla kun\'i prosc//
\glb silver\mk{gen} spoon//
\glft \trsl{silver spoon}//
\endgl
\xe

\subsection{Genitive of the whole}
The genitive can also be used to indicate

\pex
\begingl
\gla na kra\v{s}tol\'i dn\'ova//
\glb \mk{loc} train:station-\mk{gen} front//
\glft \trsl{in front of the train station}//
\endgl
\xe

Note that the patientive and not the genitive case is used when quantifying a part of the whole.

\pex
\a
\begingl
\gla *\v{z}nohou\v{s}ce hron\'a//
\glb student-\mk{gen} three//
\glft \trsl{three of the students}//
\endgl
\a
\begingl
\gla na \v{z}nohou\v{s}ca hron\'a//
\glb \mk{loc} student-\mk{gen} three//
\glft \trsl{three of the students}//
\endgl
\xe

Nevertheless when quantifying a noun per se, and not in relation to a whole, the uninflected form of the quantifier is used (mostly using indefinite quantifiers such as \trsl{many}, \trsl{a lot}, etc.). If however, the quantification involves a countable unit or division of the noun, the genitive is used, but such unit or division must be further quantified by a numeral or an indefinite quantifier.

\pex
\a
\begingl
\gla Na krouma\v{s}ta po zma je\v{s} pivo.//
\glb \mk{loc} refrigerator-\mk{pat} still few \mk{exst} beer//
\glft \trsl{There's still some beer left in the refrigerator.}//
\endgl
\a
\begingl
\gla Ona pive \v{s}tava unar\'i\v{z}\v{c}em.//
\glb one beer-\mk{gen} mug-\mk{pat} \mk{ref-}order-\mk{av-pv-1s}//
\glft \trsl{I ordered a mug of beer.}//
\endgl
\xe

\subsection{Genitive of movement}

The genitive is also used to indicate movement away from somewhere.

\pex
\a
\begingl
\gla Dum\'i pal\v{z}ek.//
\glb house-\mk{gen} leave-\mk{av-pf}//
\glft \trsl{I left the house.}//
\endgl
\a
\begingl
\gla Dum palzinek.//
\glb house leave-\mk{pv-pf}//
\glft \trsl{I left the \emph{house}.}//
\endgl
\xe

\section{Instrumental case}

The instrumental case (glossed \mk{inst})

\subsection{With some prepositions}

The following prepositions take the instrumental case: \ird{\v{s}e} \trsl{with}

\pex
\begingl
\gla Za bolta \v{s}e Janu st\'o\v{z}\k{a}c.//
\glb for party-\mk{pat} with Jan-\mk{inst} go-\mk{av-ctpv}//
\glft \trsl{(I am) coming to the party with Jan.}//
\endgl
\xe

\subsection{With expressions of time and duration}

\section{Vocative Case}


\section{Unmarked Form}


\section{Personal Pronouns}\index{personal pronouns}\index{pronouns}

Personal pronouns are a special class of nouns used to refer and/or replace other nouns or noun phrases. Personal pronouns are marked for person, number and case, and partially for animacy\index{animacy}, although third-person forms are more properly analyzed as demonstratives. In addition, personal pronouns have three forms: (1) an invariable strong form, used when the pronoun is the topic of the sentence; (2) a weak form; and (3) a clitic form.

\begin{table}[h!]
	\caption{Personal pronouns in Iridian}
	\centering\small
	\begin{tabularx}{0.8\textwidth}{>{\scshape}YMMM}

		\toprule
		\multicolumn{1}{c}{\textsc{person}} &\textsc{strong} &\textsc{weak}&\textsc{clitic}\\
		\midrule
		1s &dá&do&-em\\ \addlinespace
		2s&já&je&-e\v{s}\\ \addlinespace
		3s.anim&\v{s}a&\v{s}e&-ic\\ \addlinespace
		3s.inan&to&cej&-as\\ \addlinespace
		4gen&á&dien&-u\v{c}\\ \addlinespace
		1pl.inc&m\'e&chce&-uh\\ \addlinespace
		1pl.exc&tov\'a&kiec&-ak\\ \addlinespace
		2pl&t\'evit&la&-elý\\ \addlinespace
		3pl.anim&o\v{z}e&dcá&-ac\\ \addlinespace
		3pl.inan&\'ima&oce&-et\\ \bottomrule
	\end{tabularx}
\end{table}

\subsection{Grammatical person}\index{person, grammatical}
Iridian pronouns
\subsection{Strong form}\index{strong form}

The strong form of a personal pronoun (glossed \mk{str}) is used when the pronoun is used as the topic of the sentence. The strong form is indeclinable.

\subsection{Weak form}

\subsection{Clitic form}\index{clitic form}

\subsection{Third-Person Pronouns and Demonstratives}


\subsection{Ellipsis}
Iridian is an extremely pro-drop language, with pronouns supplied only if not inferrable from context.


\section{Demonstratives}\index{demonstratives}

\begin{table}
	\small\centering
	\caption{Demonstrative pronouns in Iridian.}
	\begin{tabu}to 0.7\textwidth{YMMM}
		\toprule
						& {\sc animate}	& {\sc inanimate}	&{\sc locative}\\
		\midrule \addlinespace
		Proximal		& \v{s}a		& to 				& tak\\ \addlinespace
		Medial			& \'on				& j\'an				& jen\'i\\ \addlinespace
		Distal			& dn\'i		& j\'on				& jon\'i\\ \addlinespace
		\bottomrule
		\label{dem-prons}
	\end{tabu}
\end{table}

Iridian does not have a separate class of third-person pronouns. Instead it uses a set of demonstrative pronouns, whose deictic\index{deixis} function is both spatial\index{spatial deixis|see{deixis}} and anaphoric\index{anaphora}. Iridian makes a three-way distinction among demonstratives, similar to French or Portuguese for example, distinguishing between proximal (near the speaker), medial (near the addressee) and distal (far from both speaker and addressee) forms. In addition, Iridian makes an animacy distinction with demonstratives, with one set of demonstratives used with human referents and another with non-human referents, as seen in Table \ref{dem-prons}, but are unmarked for either number or gender.

Demonstratives can be used adnominally, to modify a noun phrase, or pronominally, to replace one.

\pex
\a
\begingl
\gla \v{s}a byl//
\glb \mk{dem.prox.anim} child//
\glft \trsl{this child}//
\endgl
\a
\begingl
\gla \v{s}a bylem//
\glb \mk{dem.prox.anim} child-\mk{1s}//
\glft \trsl{this child of mine}//
\endgl
\a
\begingl
\gla \v{S}a bylem.//
\glb \mk{dem.prox.anim} child-\mk{1s}//
\glft \trsl{This (person) is my child.}//
\endgl
\a
\begingl
\gla *To bylem//
\glb \mk{dem.prox.inan} child-\mk{1s}//
\glft \trsl{This (thing) is my child.}//
\endgl
\xe


Unlike true personal pronouns, demonstratives do not have a separate strong form and clitic form. They are fully declined however, with the declined forms being highly irregular, as can be seen in Table \ref{dem-conj}.

\pex
\a\deftagex{obv}
\begingl
\gla ci mlaz a dn\'i maty//
\glft \trsl{this person's brother and that person's mother}//
\endgl
\a\deftaglabel{obv1}
\begingl
\gla D\'a je svou je dnu zapr\'al.//
\glft \trsl{I am as old as either this person or that person.}//
\endgl
\xe

\begin{table}
\footnotesize\sffamily
	\caption{Declension of demonstratives.}
	\begin{tabu}to 0.9\textwidth{Y[1.5]YYYYYY}
		\toprule
						& {\v{s}a}	& {\'on}	&{dn\'i}& {to}	& {j\'an}	&{j\'on}\\
		\midrule \addlinespace
		Agentive&\v{s}em&n\'am&dniem&etom&j\'an&j\'on\\\addlinespace
		Patientive&\v{s}\'a&ona&dn\'a&toha&jina&jin\'ova\\\addlinespace
		Genitive&ci&on\'i&dn\'i&cie&nie&nohe\\\addlinespace
		Instrumental&svou&nu&dnu&etu&nu&nohu\\\addlinespace
		\bottomrule
		\label{dem-conj}
	\end{tabu}
\end{table}

The three-way distinction between demonstratives allows Iridian to disambiguate between an obviative\index{obviation} third person and a proximate third person, using the distal and the proximal demonstrative respectively. Consider for example the two examples in English below:

\pex
\a He saw his dog.
\a He saw his own dog.\smallskip
\xe

The \emph{his} in the first sentence is ambiguous, as it can refer to either the subject or an implied fourth person. That the second \emph{his} refers back to the subject can be made unequivocal by the addition of the word \emph{own}, as in the second sentence. Compare this with the following sentences in Czech:

\pex
\a
\begingl
\gla Vid\v{e}l jeho pes.//
\glft \trsl{He saw his dog.}//
\endgl
\a \begingl
\gla Vid\v{e}l sv\'e pes.//
\glft \trsl{He saw his own dog.}//
\endgl
\xe

Although the English translation of the first sentence may still appear ambiguous, we can see that Czech does away with the ambiguity by using the third person pronoun \ird{jeho} exclusively to signify that the referent is different from the subject, and requiring the use of a separate pronominal form (in this case the reflexive) when the referent and the subject are the same. Iridian, on the other hand, treats this in a diametrically opposite way, i.e., the same pronoun form is used when the subject and the referent are the same, with the obviative form being used otherwise. The sentences in Czech above will therefore be translated in Iridian as follows:

\pex
\a
\begingl
\gla Dn\'i jec vdinek.//
\glb \mk{dem.dist.anim.gen} dog see-\mk{pv-pf}//
\glft \trsl{He saw his (other person's) dog.}//
\endgl
\a \begingl
\gla Ci jec vdinek//
\glb \mk{dem.dist.inan.gen} dog see-\mk{pv-pf}//
\glft \trsl{He saw his own dog.}//
\endgl
\xe

Perhaps we can better understand the distinction between obviative and proximate forms by re-examining example (\getfullref{obv.obv1}) above. The previous examples in Czech remained unambiguous because there are at most two unique arguments in the sentence. In example (\getfullref{obv.obv1}), however, the subject of the sentence is distinct from either the proximate referent or the distal referent.

\ex[exno={\getfullref{obv.obv1}}]
\begingl
\gla D\'a je svou je dnu zapr\'al.//
\glft \trsl{I am as old as either this person or that person.}//
\endgl
\xe

The translation in the gloss demonstrates how idiomatic English uses periphrastic forms to eliminate this ambiguity, although in the spoken language the purely deictic \trsl{I am as old as either him or him} is equally acceptable, with the blanks filled in most likely by non-verbal cues. In Iridian, however, this distinction is not optional, and the following sentence, for example, would be considered ungrammatical:

\ex
\begingl
\gla *D\'a je svou je svou zapr\'al.//
\glft \trsl{I am as old as either him or him.}//
\endgl
\xe

\section{Use of Personal Pronouns}

\subsection{T-V Distinction}\index{politeness}\index{T-V distinction}\index{forms of address}

Iridian has three forms of address: the informal, the polite, and the formal.

The second person singular pronoun \ird{j\'a} is used to address friends, relatives or children. When addressing a stranger or an acquaintance with whom you want to maintain social distance or be polite without being too formal, the second person plural pronoun \ird{t\'evit} is used. The polite form is also used when addressing God/gods. In more formal settings, the third-person plural pronoun \ird{o\v{z}e} is used.



\section{Possessive Pronouns}

\subsection{The reflexive \ird{m\'am}}

\section{Demonstratives}\index{demonstratives}\index{demonstrative pronouns}

Iridian has a three-way distinction between demonstratives, unlike English but similar to Spanish or Japanese: \emph{proximal} demonstratives are used when referring to objects or people that are near the speaker, \emph{medial} demonstratives when referring to those near the listener, and \emph{distal} demonstratives when referring to those that are far from either the listener or speaker.






\begin{table}[h!]
	\small\centering
	\caption{Conjugation of Iridian demonstrative pronouns.}
	\begin{tabu}to 0.7\textwidth{YMM}
		\toprule
						& {\sc animate}		& {\sc inanimate}\\
		\midrule
		Proximal		& \v{s}a			& to\\ \addlinespace
		Medial			&&j\'an\\ \addlinespace
		Distal			&&j\'on\\ \addlinespace
		\bottomrule
	\end{tabu}
\end{table}

For information about demonstrative adjectives/determiners, see section \ref{dem-adj}.

\section{Indefinite pronouns and quantifiers}


\section{Interrogative pronouns}\index{wh- questions}\index{interrogative pronouns}

\begin{table}[h!]
	\small\centering
	\caption{Interrogative pronouns in Iridian.}
	\begin{tabu} to 0.8\textwidth{>{\bfseries}YY>{\bfseries}YY}
		\toprule\addlinespace
		&{\sc english}&&{\sc english}\\ \addlinespace
		\midrule\addlinespace
		jede 		& who &jach &which\\ \addlinespace
		je\v{z}e 	& what 		& zajehu 	&why\\ \addlinespace
		jeh\'at 	& whom		& jik\'a 	&how many\\ \addlinespace
		jehu 		& how		&ji\v{s}k\'a&how much\\ \addlinespace
		jem\'i 		& when 		& jenie 	&to where\\ \addlinespace
		jena 		& where 	& jen\'i 	&from where\\ \addlinespace
		\bottomrule
	\end{tabu}
\end{table}

\section{Negative and Universal Pronouns}\index{negative pronouns}\index{universal pronouns}

Negative pronouns are historically formed by attaching the prefix \ird{\v{z}e} before interrogative pronouns, and universal pronouns by attaching the prefix \ird{n\'i-}

\begin{table}[h!]
	\small\centering
	\caption{Correspondence of interrogative, negative and universal pronouns.}
	\begin{tabu} to \textwidth{>{\bfseries}YY[1.2]>{\bfseries}YY[1.2]>{\bfseries}YY[1.2]}
		\toprule\addlinespace
		\multicolumn{2}{c}{\sc interrogative}& \multicolumn{2}{c}{\sc negative} & \multicolumn{2}{c}{\sc universal}\\ \addlinespace
		\midrule\addlinespace
		jede 		& who & nei\v{z}e & no one & niet & everyone\\ \addlinespace
		je\v{z}e 	& what 		& niho & nothing&n\'i\v{z}e&everything\\ \addlinespace
		jehu 		& how		&\v{z}ehu&by no means&n\'ehu&by all means\\ \addlinespace
		jem\'i 		& when 		& \v{z}emie&never&nimie&always \\\addlinespace
		jena 		& where 	& \v{z}ena&nowhere&nina&everywhere \\ \addlinespace
		jach &which&\v{z}\'e&not one&n\'ach&each\\ \addlinespace
		\bottomrule
	\end{tabu}
\end{table}




\section{Derivational Morphology}

\subsection{-ma\v{s}t}

\begin{table}[h!]
	\centering\small
	\caption{Nominal derivation using \ird{-ma\v{s}t}}
	\begin{tabu} to \textwidth{YYY[0.5]YY}
		\toprule
		\multicolumn{2}{c}{\sc root}&&\multicolumn{2}{c}{\sc derived noun}\\
		\addlinespace
		\midrule
		\ird{k\'av\'e}&\trsl{coffee}&$\rightarrow$& \ird{k\'av\'ema\v{s}t} &\trsl{caf\'e}\\
		\ird{krou}&\trsl{cold}&$\rightarrow$& \ird{krouma\v{s}t} &\trsl{refrigerator}\\
		\ird{pia\v{s}tou}&\trsl{food}&$\rightarrow$& \ird{pia\v{s}touma\v{s}t} &\trsl{restaurant}\\

		\bottomrule

	\end{tabu}

\end{table}

\subsection{-ou}
The nominalizing suffix \ird{-ou} is a non-productive affix used to form nouns from certain verbs.

\begin{table}[h!]
	\centering\small
	\caption{Nominal derivation using \ird{-ou}}
	\begin{tabu} to \textwidth{YYY[0.5]YY}
		\toprule
		\multicolumn{2}{c}{\sc verb root}&&\multicolumn{2}{c}{\sc derived noun}\\
		\addlinespace
		\midrule
		\ird{milovan\'a}&\trsl{to learn}&$\rightarrow$& \ird{milovanou} &\trsl{lesson}\\
		\ird{palz\'a}&\trsl{to leave}&$\rightarrow$& \ird{palzou} &\trsl{departure}\\
		\ird{pia\v{s}t\'a}&\trsl{to eat}&$\rightarrow$& \ird{pia\v{s}tou} &\trsl{food}\\
		\ird{scen\'a}&\trsl{to arrive}&$\rightarrow$& \ird{scenou} &\trsl{arrival}\\
		\ird{niek\'a}&\trsl{to open}&$\rightarrow$& \ird{niekou} &\trsl{entrance}\\

		\bottomrule

	\end{tabu}

\end{table}

\subsection{-ou\v{s}c}
The suffix \ird{-ou\v{s}c} (pronounced as if written \ird{-\'o\v{s}t} \bt{o:St}, or in some dialects as \ird{-ou\v{s}t} \nt{\dto{}St}) is used to form a noun indicating someone or something associated to a certain thing or performing a certain action.

\begin{table}[h!]
	\centering\small
	\caption{Nominal derivation using \ird{-ou\v{s}c}}
	\begin{tabu} to \textwidth{YYY[0.5]YY}
		\toprule
		\multicolumn{2}{c}{\sc verb root}&&\multicolumn{2}{c}{\sc derived noun}\\
		\addlinespace
		\midrule
		\ird{jork\'a}&\trsl{to travel}&$\rightarrow$& \ird{jorkou\v{s}c} &\trsl{traveller}\\
		\ird{mo\v{z}l\'a}&\trsl{to live}&$\rightarrow$& \ird{mo\v{z}lou\v{s}c} &\trsl{resident}\\
		\ird{umiel\'a}&\trsl{to get drunk}&$\rightarrow$& \ird{um\'ilou\v{s}c} &\trsl{drunkard}\\
		\ird{virk\'a}&\trsl{to write}&$\rightarrow$& \ird{virkou\v{s}c} &\trsl{writer}\\
		\ird{zdiev\'a} &\trsl{to fool (sm.)}&$\rightarrow$& \ird{zd\'ivou\v{s}c} &\trsl{swindler}\\
		\bottomrule

	\end{tabu}

\end{table}
