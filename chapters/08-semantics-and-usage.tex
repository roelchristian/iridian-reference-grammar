\chapter{Semantics and usage}

\section{Introduction}



\section{Register}
\section{Politeness and forms of address}\index{form of address}\index{politeness}\label{sec:politaddr}

\subsection{Politeness and formality in Iridian}\index{politeness}\index{formality}

Although not as complex and as pervasive as the politeness/formality system
found in Japanese\index{Japanese} or Korean\index{Korean}, Iridian formally
encodes\index{markedness} more sociolinguistic information than its neigbouring
languages such as Czech\index{Czech} or Hungarian\index{Hungarian}.

Broadly speaking, Iridian distinguishes between three levels of
speech\index{level of speech}:\footnote{ The English names are of course
imperfect. It would perhaps be more correct,---if not more illustrative of their
differences,---to call the polite speech level \emph{formal} and the formal
speech level \emph{honorific}. What we call above as polite is more close to
what linguists would call `formal' mainly because the strategy is one of
distance and not deference. Moreover, although the formal speech level may be
used to signal respect and shows a strong tendency to use honorifics and titles,
the main usage remains that of showing an even greater detachment on the part of
the speaker than would have otherwise have been possible when using the polite
speech level. } (1)~\emph{polite} speech, which serves more or less as the
``default'' level of politeness, as this is the speech level most often used by,
say, strangers when talking to each other; (2)~\emph{formal} speech, which is
used in more formal settings, where the speaker wants to distance themself from
the listener or explicitly signal their politeness, such as in a conversation
among business associates or when talking to a divinity; and (3)~\emph{casual}
speech, which is used between close friends and family members, or to or among
children. These levels of speech are not definite, of course, and politeness is
more properly viewed as a spectrum (cf., e.g., \cite{hansonjap}) as speakers
would often switch from one level of speech to another even when speaking to the
same person, or within a single conversation.

The distinction between politeness\index{politeness} (which for the purpose of
this grammar we can define as the psychological or social distance between
speakers) on the one hand, and formality (which we can define as situational
distance) on the other, is not always one made (or kept) in Iridian. Indeed,
more often than not, these categories are often viewed by most speakers as
essentially being the same. This is further complicated by the fact that the
distinction between the various speech levels is not morphologically
marked\index{markedness} but is facilitated instead by the preference for
certain constructions and forms of address\index{form of address}.


The choice of which speech level to use with which speaker and in which
scenarios is influenced by a lot of factors. It would be helpful, however, to
analyse these factors as being influenced by two main considerations: the
relationship---more specifically, the familiarity,---between the speakers, and
the social setting in which the conversation or interaction is taking
place.\footnote{ One could take a look as well at the dimensions (or
`semantics,' to use the authors' term) that influence the formality/politeness
distinctions made in a language, proposed by \textcite{browngilman} in their
study on the development of second-person pronouns and address forms. Although
on the surface, the politeness distinction in Iridian is not dual, we see (as
discussed \emph{infra}) that we can in fact classify the speech levels as either
familiar (T) and distant (V). Where most Indo-European languages, however,
predicate this distinction on the power semantic (i.e., the T-V distinction is
made initially when a speaker of one power group speaks to a member of another),
Iridian bases this initially on the solidarity semantics, thus creating a T-V
distinction first when there is no solidarity (perceived or otherwise) between
speakers, and only secondarily on the basis of the power semantic. }

The first consideration, the relationship between speakers, divides the levels
of speech into two groups: {\sc familiar speech}\index{familiar speech}, which
consists of the casual speech level and {\sc distant speech}\index{distant
speech} which consists of both the formal and polite speech levels. This
distinction is perhaps of greater actual importance than that introduced earlier
between the levels of speech, as the differences between familiar speech and
distant speech are more pronounced than the differences between formal speech
and polite speech, which are often more subtler. Distant speech is characterized
by a preference to indirect speech acts where possible. For example, direct
imperatives or prohibitives are virtually unused in distant speech, replaced
instead with hortative constructions, or in more formal situations with
questions or optative constructions. Consider for example the following:

\pex
\a Imperative in familiar speech:\\
\ird{Mina návilastním.} \trsl{Open the door!}
\a Alternative constructions in distant speech:
\beginsubsub
  \b{-} {Neutral, using the hortative:\\
  	\ird{Mina návilastniká.} \trsl{Please open the door.}}
  \b{-} {More polite, using \ird{am luhninká}:\\
  	\ird{Mina se návilastu am luhninká}\\\trsl{May (you) be thanked because the
  		door was closed.}} \b{-} {More formal and more polite, using a
  		question:\\
  	\ird{Mina návilastníš to mužnali\v{t}?}\\
  	\trsl{Is it possible that the door will be closed?}
  	}
\endsubsub
\xe

Perhaps a direct consequence of this preference for indirect speech acts over
direct ones is the strategy of {\sc pronoun avoidance} so heavily employed in
distant speech. Pronoun avoidance as it applies to Iridian include not only
\posscite[371--2]{velupillai2012} narrow definition of it as the omission and
sometimes replacement with a title or other form of address of a pronoun, when
addressing or referring to a person, but also the indirect result of Iridian's
heavy reliance on context and the resulting tendency to drop elements of the
sentence when they can be easily inferred, including pronouns.

In general, familiar speech is indifferent on the use of personal pronouns, with
the use or omission dictated by context and not by politeness/formality. Thus
both of the following sentences are equally probable in familiar speech:

\pex\a \ird{Avtem bych hebo.} \trsl{My car broke down yesterday}\deftagex{pronavoidance}\deftaglabel{withpron}
	\a \ird{Avt bych hebo.} \trsl{(My) car broke down yesterday}\deftaglabel{sanspron}
\xe

In distant speech, however, sentence (\getfullref{pronavoidance.withpron}) would
be largely avoided, or even considered disrespectful or incorrect. When speaking
in the polite speech level, the omission of the personal pronoun is often
enough; in the formal speech level, especially in writing, this is often
complemented by the explicit addition of a referent honorific, even when the
context is clear.

\pex
	\a Casual and polite speech:\\
		\ird{Marek záščenžévnik. Avt ce bych hebo.}\footnote{ The ethical dative
			as seen in this example is emphatic and can be used in both casual
			and polite speech. }\\
		\trsl{Marek couldn't come yesterday. His car broke down.}
	\a Formal speech:\\
		\ird{Stám Zakár záščenžévnik. Stámí avt bych hebo.}\\
		\trsl{Mr Zakár couldn't come yesterday. His car broke down.}
\xe

The persistence of pronoun avoidance means a person's title or an equivalent
honorific will be used in formal speech even when addressing that person
directly. Nevertheless, when addressing a listener directly, the formal speech
level does allow the use of the distal animate demonstrative \ird{dní} (a
stand-in for the third person pronoun, since Iridian does not have one); this is
parallel in the polite speech level which allows the use of the second person
plural pronoun \ird{tová}\footnote{The use of the plural \ird{tová} has perhaps
the closest Iridian is to a true T-V distinction.} in direct addresses. Both
ultimately correspond to the use of the second person singular pronoun \ird{já}
in casual speech. The use (or omission) of any of these pronouns is as always
dependent on actual context.

\pex
	\a Formal speech, using honorifics:\\
	\ird{Stám Zakár bych záščenžévnice to kvušček. Stám jevitébílá te ceščeví?}\\
	\trsl{I heard you were not able to come yesterday. Would you like me to catch you up on what happened?}\medskip\\
	Formal speech, using \ird{dní}:\\
	\ird{Stám Zakár bych záščenžévnice to kvušček. Dní jevitébílá te ceščeví?}\\
	\trsl{I heard you were not able to come yesterday. Would you like me to catch you up on what happened?}\footnote{Note that even when using \ird{dní} instead of honorifics, a honorific would still be used when addressing the listener for the first time, and only on subsequent occurences would the substitution be made.}
	\a Polite speech, using \ird{tová}:\\
	\ird{Tová bych záščenžévnice to kvušček. Jevitébílá te ceščeví?}\\
	\trsl{I heard you were not able to come yesterday. Would you like me to catch you up on what happened?}
	\a Casual speech, using \ird{já}:\\
	\ird{Já bych záščenžévnice. Jevitébílá te ceščeví?}\\
	\trsl{I heard you were not able to come yesterday. Would you like me to catch you up on what happened?}
\xe

The use of bare honorifics\index{honorific} instead of an actual formal/polite second person may seem unwieldy at first, but it is in fact not uncommon. We see similar systems, for example in European Portuguese\index{Portuguese} and Tagalog\index{Tagalog}.

\pex
	\a European Portuguese\index{European Portuguese|see{Portuguese}}\index{Portuguese}
		\beginsubsub
			\b{-}{Explicit V form, honorific used:\\
				\foreign{O senhor sabe onde é que está?} \trsl{Do you know where you are?}}
			\b{-}{Implicit V form, pronoun omitted:\\
				\foreign{Sabe onde é que está?} \trsl{Do you know where you are?}}
			\b{-}{Superficially an explicit V form, but may be interpreted as informal or even rude\footnote{
				Cf. \textcite{laraport}. The peculiar nature of \foreign{voc\^e} is European Portuguese (EP) is quite interesting. Whereas in Brazilian Portuguese\index{Brazilian Portuguese|see{Portuguese}} (BP) \foreign{voc\^e} has almost completely displaced \foreign{tu} as the prevalent T form, in EP it occupies a linguistic limbo between \foreign{tu} (T) and \foreign{o senhor/a senhora} (V), leading to it having quite disparate uses depending on the speaker and the dialect.

				Etymologically, \foreign{voc\^e} shares the same historical
				development as the Spanish\index{Spanish} \foreign{usted}. They
				are syncopated versions of the original forms of address
				\foreign{vossa merc\^e} and \foreign{vuestra merced},
				respectively, both of which translate to \trsl{your
				mercy/grace}. The original pronouns \foreign{vossa/vuestra}
				persist in both language but are no longer the standard V forms,
				supplanted instead by developments from the forms of address
				originally containing them. (Cf., e.g., \cite{hummelport}, which
				provides an extensive analysis of the diachronic development of
				both the Spanish \foreign{usted} and the Portuguese
				\foreign{voc\^e}.)
	
				In most Spanish\index{Spanish} dialects, \foreign{usted} remains
				the standard V form. In Portuguese\index{Portuguese}, however,
				\foreign{voc\^e} has itself been supplanted by another form of
				address used as a pronoun, \foreign{o senhor/a senhora}. In BP,
				this change coincided (or perhaps caused) \ird{voc\^e} to change
				from being an intermediate V form to the default T form, with
				\foreign{tu} (the original T form) and \foreign{vós} (the
				original V form, and later, intermediate T form) falling out of
				use. In EP\index{Portuguese}, on the other hand, the T forms
				(both the original \foreign{tu} and the intermedaite
				\foreign{vós}) were retained and instead it is \foreign{voc\^e}
				that fell out of use. (This historical shift of the V form
				displacing the existing T form and the consequent loss of this
				original T form, and the grammaticalization of a polite form of
				address as a new V form, is quite common; in Rioplatense
				Spanish, one of the more divergent dialects of
				Spanish\index{Spanish}, for example, we see the V$\rightarrow$T
				shift started by the grammaticalization of \foreign{usted}
				completed by the eventual displacement of the original T form
				\foreign{t\'u} with the original V form \foreign{vos} as the
				prevalent T form. What is interesting in EP, however, is that
				V$\rightarrow$T shift was completed, not by the displacement of
				the original T form \foreign{tu}, but by the loss---or more
				properly, \emph{obsolescence}---of the intermediate V form
				\foreign{voc\^e}).

				\foreign{Voc\^e} remains, superficially at least in
				EP\index{Portuguese}, a V form (cf. \cite[85]{ganhoport}); its
				actual use, however, is not as clearly defined. As
				\textcite{laraport} remarks, `not even Portuguese speakers agree
				in determining the contexts where it can be employed.' The most
				important development in modern EP\index{Portuguese} with
				regards to the use of \foreign{voc\^e} is that of conveying
				anger, sarcasm or annoyance, especially in asymmetric relations,
				similar to the older \foreign{vós}, the use of V forms between
				speakers who normally would use T forms to indicate annoyance
				(cf. \cite{hummelport}). This has led to the ambiguous use of
				\foreign{voc\^e} both as a polite and an impolite form of
				address.

				Although this V$\rightarrow$T shift does not directly reflect
			Iridian's own historical development, it is helpful to understand
			the fluidity and the inherent arbitrariness of T-V labels in any
			language. In Iridian, too, this V of annoyance exists marginally
			both between people who regularly use T forms (i.e., familiar
			speech) with each other, to indicate sarcasm or displeasure; and
			those who use V forms (i.e., distant speech) among themselves, to
			openly signal disrespect. }:\\
				\foreign{Voc\^e sabe onde é que está?} \trsl{Do you know where you are?}}
			\b{-}{Explicit T form:\\
				\foreign{(Tu) sabes onde é que estás?} \trsl{Do you know where you are?}}
		\endsubsub
	\a Tagalog\index{Tagalog}
		\beginsubsub
			\b{-}{Explicit V form, 2nd person plural:\\
				\foreign{Alam ba ninyo kung nasaan kayo?} \trsl{Do you know where you are?}}
			\b{-}{Explicit V form, 3rd person plural:\\
				\foreign{Alam ba nila kung nasaan sila?} \trsl{Do you know where you are?}}
			\b{-}{Explicit T form. 2nd person singular:\\
				\foreign{Alam mo ba kung nasaan ka?} \trsl{Do you know where you are?}}
		\endsubsub
\xe

The preference in distant speech for indirect speech acts is also manifested in
the extensive use of 

\subsection{Forms of address, titles, and honorifics}
\index{honorific}\index{terms of courtesy|see{honorific}}\index{courtesy|see{honorific}}\index{form of address}

A {\scshape honorific} is a form of address used to indicate respect or
courtesy. The most common honorifics in Iridian are the masculine \ird{Stám}
equivalent to the English\index{English} \trsl{Sir} and the feminine \ird{Nau}
equivalent to the English \trsl{Madame/Ma'am.} When addressing a person of an
unknown gender\index{gender}, the term \irdp{Obečne}{mercy/grace} is used.

Both \ird{Stám} and \ird{Nau} may be followed by the addressee's last name. They
should never be used with the first name as it would be considered sarcastic or
rude. In writing, these are abbreviated as \ird{S.} and \ird{N.}, respectively.
If the name of the person being addressed is not known, the placeholders
\irdp{vieda}{man} and \irdp{huzak}{woman} are used, thereby producing \ird{Stám
Vieda} and \ird{Nau Huzak}. When writing, these are often abbreviated to {\sc
s.v.} and {\sc n.h.}, respectively. The usage of \ird{Stám Vieda} and \ird{Nau
Huzak} is similar to how the third person may sometimes be used in
English\index{English} to politely address someone (e.g., saying, \trsl{Will the
gentleman yield?}) but while it may sometimes appear dated or overly formal in
English\index{English}, this practice is still commonly observed in Iridian,
especially when addressing strangers.

Other common titles include \ird{Doktor} used when addressing physicians,
\ird{Majestet} or \ird{Kopižnást} when addressing members of the royalty (with
the latter reserved for reigning monarchs), \ird{Eselenc} when addressing
certain high-ranking officials such as senators, governors, and ambassadors,
\ird{Eminenc} when addressing cardinals of the Catholic Church, \ird{Obečne} or
\ird{Prac} when addressing judges and magistrates, and \ird{Tiehožnást} or
\ird{Hildažnást} or \ird{Hildení Tá\v{t}}\footnote{This form of address, meaning
\trsl{Holy Father} or more commonly its abbreviation {\sc h.t.}, is used in
writing when referring to the Pope in the third person.} when addressing the
Pope or the religious leaders from other traditions.

When addressing or referring to multiple individuals the term \ird{maše}
(originally meaning \trsl{crowd} but now exclusively employed as a honorific) is
used. This is often preceded, both in the written and spoken forms, by the
non-nominal supine\index{supine} \irdp{prehodašce}{esteemed/praiseworthy.}

\subsection{Salutations and valedictions in the written language}\index{salutation}\index{valediction}\index{written correspondence}

The general salutation in most formal correspondence uses the honorific
\irdp{Stám}{Sir} or \irdp{Nau}{Madame}. The last name of the addressee may also
follow, although more often than not, the simple honorific\index{honorific}
should suffice. When addressing a collegiate entity or a collection of people,
the term \irdp{Maše}{crowd} or \irdp{Prehodašce maše}{Esteemed/praiseworthy
crowd} is used instead.

If the addressee holds a specific title, the title is included in the
salutation. In some cases, the wife of the title-holder may be addressed using
\ird{Nau} followed by the title, although this practice is slowly falling out of
use, except in most diplomatic correspondence, where it is still considered
standard. Below are some examples:


\begin{itemize}[nosep]
	\item \irdp{Stám/Nau Prezident}{Mister/Madame President}
	\item \irdp{Stám/Nau Brac}{Mister/Madame Member of the Parliament}
	\item \irdp{Stám/Nau Kanclár}{Mister/Madame Chancellor}
	\item \irdp{Stám/Nau Holva}{Mister/Madame Chairman/Chairwoman}
	\item \irdp{Stám/Nau Provízor}{Mister/Madame Professor}
\end{itemize}

Where the addressees are multiple individuals who hold specific titles, the
honorific \ird{Stám} or \ird{Nau} is replaced with \irdp{prehodašce}{esteemed,
praiseworthy}. When used this way, the title is normally not capitalised. Note
also that \ird{prehodašce} will only be used in a salutation when there are
multiple addressees.

\begin{itemize}[nosep]
	\item \irdp{Prehodašce brac}{Esteemed members of the Parliament}
	\item \irdp{Prehodašce provízor}{Esteemed members of the faculty}
\end{itemize}

When the addressee is a medical doctor, the salutation \irdp{Doktor}{doctor} is
used. When writing to members of the clergy, it is customary to use
\irdp{Pápka}{My father} or \irdp{Mlazka}{My brother.}

It is considered rude to use a person's first name by itself in the salutation.
A more common way is to add the suffix \irdp{-óm}{our} or \irdp{-(e)m}{my} to
the name or the diminutive form of the name. Alternatively the terms
\irdp{kamarád}{colleague, comrade} or \irdp{naž}{friend} or their diminutives
may also be used. This approach is particularly common in e-mail correspondence
between work colleagues.

Standard valedictions used in formal written
correspondence\index{correspondence|see{written correspondence}} in Iridian tend
to be more complex than the ones used in English. Below is 

\begin{itemize}[nosep]
	\item \irdp{(Stám/Nau) oblostnení mavac/respekt akceptirniká}{Sir/Madame, please accept my sincerest regards (\emph{lit.}, wishes)/respect.}
	\item \irdp{Dá zespodení/spietnení pokárí biležit}{I will remain your most humble/loyal servant.}
	\item \irdp{Dá zespodení/spietnení bylí biležit}{I will remain your most humble child.}\footnote{This is often used among religious people when writing to members of the clergy.}
	\item \irdp{Oblostnení mavacu/respektu še hroznik.}{With the sincerest regards/respect has this letter been sent.}

\end{itemize}

Increasingly, especially in e-mail\index{e-mails} correspondence, it has become
more common to use the following valedictions instead:

\begin{itemize}[nosep]
	\item \irdp{Mavac/\v{S}e mavacu}{Regards/with wishes/regards.}
	\item \irdp{Oblostnení}{Most sincere}
\end{itemize}

In more informal situations, such as between close friends and family, the
following are used:

\begin{itemize}[nosep]
	\item \irdp{Dá}{I/me}
	\item \irdp{Bes/Mach bes/Nic bes}{Hug/Two hundred hugs/A thousand hugs}
	\item \irdp{Beska/Mach beska/Nic beska}{Little hug/Two hundred little hugs/A thousand little hugs}
	\item \irdp{\v{S}e hloubu/Hloubževí}{With love/Loving}
	\item \irdp{\v{Z}už/Mach žuž/Nic žuž}{Kiss/Two hundred kisses/A thousand kisses}
\end{itemize}

As mentioned earlier, specific examples of written correspondence in Iridian can
be found in \S\,\ref{sec:writcorr}.


\section{Phatic Expressions and Social Formulas}

Phatic expressions\index{phatic expression} are used to establish and maintain social relationships. They are used 

\section{Idiomatic Expressions}\index{idiomatic expressions}

\section{Punctuation}\label{sec:punctuation}\index{punctuation}

Iridian punctuation is consistent with the punctuation rules of most European languages.

\begin{enumerate}
	\item A full stop/period (.)\index{full stop}\index{period|see{full stop}} is used:
		\begin{enumerate}
			\item to mark the end of a sentence;
			\item with numbers, as thousands separators (e.g, 23.100 vs English 23,100);
			\item with ordinal numbers, especially with dates (e.g., 2009\thinspace{}h. jen. 23.\thinspace{}r. vs English 23 January 2009);
			\item to separate the components in a date when abbreviating (e.g., 2009.\thinspace01.\thinspace23 vs English 23/01/2009) — a half-space is used after the period in this case;
			\item in abbreviations (e.g., \ird{kr.} for \ird{koruna})\footnote{Where the abbreviation appears at the end of a sentence, a single full stop is used for both the abbreviation and the end of the sentence};
			\item 
		\end{enumerate}
	\item A comma\index{comma} is used:
		\begin{enumerate}
			\item to separate items in a list;
			\item with numbers, to separate the decimal part from the whole number (e.g., 23,10~kr. vs English \$~23.10);
		\end{enumerate}
	\item Reverse guillemets\index{quotation marks}\index{guillemets|see{quotation marks}} (»«) are used:
		\begin{enumerate}
			\item to enclose direct quotations\footnote{For longer quotations and in dialogues, an em dash (—) is used instead of quotation marks. See the entry on the em dash for more information.};
			\item with the names of books, films, plays, etc.\footnote{Unlike English,talics are never used for this purpose.}.
		\end{enumerate}
	\item A colon (:) is used to introduce a list or a quotation.
	\item A semicolon (;) is used to separate two independent clauses that are closely related in meaning.
	\item An em dash (—) is used:
		\begin{enumerate}
			\item to mark the beginning of a direct quotation, in which case the quotation should be moved to a new paragraph as an em dash used for this purpose can only be used at the beginning of a paragraph;
			\item in a dialog, to indicate a change of speaker;
			\item a pause by the speaker or an interruption by someone else of a speaker;
			\item in literary works to indicate aposiopesis (a sudden break in the middle of a sentence);
			\item attribution of a quotation to a source, especially in block quotations;
			\item in parenthetical remarks;
			\item when redacting a text, to indicate the omission of a word or phrase;
		\end{enumerate}
	\item A hyphen (-) is used:
\end{enumerate}