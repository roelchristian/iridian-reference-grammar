\chapter{Verbs}
\section{Introduction}
Verbs in Iridian are heavily marked. There is a tendency to encode most of the information contained in the sentence on the verb leaving the noun or noun phrase unmarked if possible.

\par Finite verbs are marked\index{markedness} for the following grammatical categories\index{grammatical categories}:
\begin{enumerate}
	\item \textit{Aspect}.\index{aspect} Iridian has three primary aspects: perfective, imperfective and contemplative; and two secondary ones: retrospective and prospective.
	\item \textit{Voice}.\index{voice} Iridian has a strong tendency to leave the topic of the sentence unmarked, instead encoding the primary information on the verb. Due to this, voice must be explicitly marked on the verb. Iridian has the following grammatical voices: agentive, patientive, benefactive, instrumental, locative and reflexive.
	\item \textit{Mood}.\index{mood} Besides the unmarked indicative, Iridian has the following grammatical moods: subjunctive, conditional, hortative, optative, abilitative, permissive and non-volitive. In addition, secondary prefixes are used to express what would otherwise could be considered as moods: inceptive, causative and reciprocative.
\end{enumerate}

Verbs are also marked for person, although this is done by the addition of clitic pronouns and not through a separate conjugation paradigm. In most cases, however, this is left out, especially if clear from the context. Iridian verbs are not marked for tense, gender, or number.

\par Iridian verbs have four classes of non-finite forms: the gerund, the converb, the supine and the generic nominal formed with \textbf{-ou}. The non-finite verb forms are derived from the uninflected verb stem except the generic nominal in \textbf{-ou} which can only be formed from a fully-inflected verb stem. A fifth class exists--the infinitive--but this form is largely defunct and is only used in certain compound constructions. Infinitives end in \textbf{-á} and is used as the citation form of a verb.

\section{Verb stem and citation form}\index{citation form}\index{infinitive}

\par The citation form (or dictionary form) of a verb is the uninflected infinitive\index{infinitive}, a fossilized form rarely used outside of a very few periphrastic construction. The infinitive ends with the vowel \ird{-\'a}, and removing this ending will produce the verb stem\index{verb stem}. The final consonant  of the stem determines the conjugation paradigm the verb follows.

In general the classification of the verb stems is based on how they behave in two phonemic environments: (1) before a phoneme that triggers palatalization such as \ird{-e-}, \ird{-\'e-}, \ird{-i-} or \ird{-\'i-} or a \emph{jod}-glide; and (2) before a spirant or an affricate.

\subsection{Type I conjugation (c/\v{c})}
Type I verbs include those whose stems end in \ird{-t}, \ird{-k}, \ird{-c}, \ird{-\v{c}},

\subsection{Type II conjugation (z/\v{z})}
Type II verbs include those whose stems end in \ird{-d}, \ird{-g}, \ird{-z}, \ird{-\v{z}},

\subsection{Type III conjugation (s/\v{s})}
Type II verbs include those whose stems end in \ird{-d}, \ird{-g}, \ird{-z}, \ird{-\v{z}},

\subsection{Type IV conjugation}
Type II verbs include those whose stems end in \ird{-d}, \ird{-g}, \ird{-z}, \ird{-\v{z}},

\subsection{Type V conjugation}
Type II verbs include those whose stems end in \ird{-d}, \ird{-g}, \ird{-z}, \ird{-\v{z}},


\section{Voice}\index{voice}

Iridian often prefers to encode information on the verb instead of through case marking on nouns. As such, all verbs must be explicitly marked for voice.
\begin{table}[h!]
	\small \centering
	\caption{Suffixes used to mark grammatical voice.}
	\begin{tabu} to 0.8\textwidth{YM}
		\toprule
		&{\sc ending}\\
		\midrule
		Agentive	& -a\v{s}-\\ \addlinespace
		Patientive	& -in-\\ \addlinespace
		Benefactive	& -\'eb-\\ \addlinespace
		Locative	& -á-\\ \addlinespace
		Instrumental& -\\ \addlinespace
		Reflexive	& -\\ \addlinespace
		Reciprocal	& \\ \addlinespace
		\bottomrule
	\end{tabu}
\end{table}


\subsection{Agentive voice}\index{agentive voice}
\par The agentive voice is used if the subject of the verb is the agent of the action.

\pex
\begingl
\gla Sa pia\v{s}\v{c}ek.//
\glb already eat-\mk{av-pf}//
\glft `(I) already ate.'//
\endgl
\xe

The affix \ird{-a\v{s}-} assimilates to the consonant ending the root, with the vowel \bt{5} normally dropped, subject to the following rules:
\begin{itemize}
	\item \v{c}: for roots ending with c, \v{c}, k, t
	\begin{itemize}
		\item jelc\'a + -a\v{s}- $\rightarrow$ jel\v{c}-, \trsl{to dance}
		\item zdiek\'a + -a\v{s}- $\rightarrow$ zd\'i\v{c}-, \trsl{to blow}
		\item pia\v{s}t\'a + -a\v{s}- $\rightarrow$ pia\v{s}\v{c}-, \trsl{to eat}
	\end{itemize}
	\item z: for roots ending with b, l, m, n, r\footnote{This change does not involve the deletion of the final consonant in the root.}
	\item \v{z}: for roots ending with d, g, z, \v{z}
	\begin{itemize}
		\item ba\v{z}- + -a\v{s}- $\rightarrow$ b\'a\v{z}-, \trsl{to give}
		\item stoj\'a + -a\v{s}- $\rightarrow$ st\'o\v{z}-, \trsl{to go}
	\end{itemize}
	\item \v{s}: for all other endings\footnote{\ird{-h + -a\v{s}-} , \ird{-s + -a\v{s}-} and \ird{-\v{s} + -a\v{s}-} both simplify to \ird{-\v{s}-}, while the rest retain the final consonant.}
\end{itemize}

Where the assimilation involves the deletion of the final consonant in the root, the preceding vowel is lengthened in compensation if the resulting root would then end in an open syllable.\index{compensatory lengthening}
\pex
\ird{Ud\'u\v{s}ek.}\\
(instead of \ird{*udu\v{s}ek})\\
\trsl{(I) took a shower.}
\xe
\pex
\ird{Pia\v{s}\v{c}ek.}\\
(not \ird{*pi\'a\v{s}\v{c}ek.})\\
\trsl{(I) ate.}
\xe

If the remnant vowel is the i-glide \ird{-ie-} or the diphthongs \ird{-ei-} and \ird{-ou-}, the remaining vowel would simplify to \ird{\'i}, \ird{\'i} and \ird{\'u}, respectively. Consider for example the verb \ird{zdiek\'a} \trsl{to blow}:

\pex
\begingl
\gla Lest zdi\v{c}al\'i.//
\glb wind blow-\mk{av-prog}//
\glft \trsl{The wind is blowing.}//
\endgl
\xe

Nevertheless the vowel \nt{5} in the root resurfaces in the following cases:

\begin{itemize}
	\item Verbs ending in -irn\'a:
	\item Verb root ending in a consonant cluster with a final liquid, nasal, or v
\end{itemize}

\subsection{Patientive focus}
\par A verb in the patient focus (glossed \mk{pat}) indicates that the topic of the sentence is the patient of the verb.

\pex
\begingl
\gla Marek vindekem.//
\glb Marek \mk{<pv>}see-\mk{pf-1s}//
\glft `I saw Marek.'//
\endgl
\xe


\subsection{Benefactive focus}\index{benefactive focus}
\par The benefactive focus (glossed \mk{ben}) is used when the subject of the sentence is the benefactor or director object of the verb. Verbs often change meaning when used in the benefactive focus.

\pex
\begingl
\gla Ma\v{c} sega nazd\'ebik.//
\glb mother flower-\mk{pat} buy-\mk{ben-pf}//
\glft `(I) bought my mother flowers.'//
\endgl
\xe

\pex
\begingl
\gla Kova pia\v{s}t\'ebal\'i.//
\glb cow eat-\mk{ben-prog}//
\glft \trsl{(I am) feeding the cows.}//
\endgl
\xe

The benefactive is also used idiomatically with verbs of judgment including \ird{noviet\'a} \trsl{to like}

\pex
\begingl
\gla D\'a \v{c}eh\'ov\'am z\'anov\'it\'eb\'al.//
\glb \mk{1s} sports-\mk{agt} \mk{neg}-like-\mk{ben-prog}//
\glft \trsl{I don't like sports.}//
\endgl
\xe

\subsection{Locative Focus}

\pex
\begingl
\gla J\'e kopna\v{z}al\'ic.//
\glb you laugh-\mk{loc-prog-3s.anim}//
\glft \trsl{He is laughing at you.}//
\endgl
\xe

\subsection{Instrumental Focus}


\subsection{Reflexive Voice}

The reflexive voice (glossed \mk{ref}) is used when the patient of the verb is also the agent of the action. Morphogically, the reflexive voice is not a separate voice but is derived from the agentive form of the verb and the addition of the prefix \ird{u(d)-}.

\pex
\begingl
\gla Na \v{s}arta uvi\v{z}kem.//
\glb \mk{loc} mirror-\mk{pat} \mk{ref}-see-\mk{av-pf-1s}//
\glft \trsl{I saw myself in the mirror.}//
\endgl
\xe

The use of the reflexive voice is more extensive in Iridian than in English, and is somehow similar to how the reflexive construction is used in Romance languages.

\pex
\begingl
\gla U\v{s}ti\v{z}ek.//
\glb \mk{ref}-take:a:bath-\mk{av-pf}//
\glft \trsl{(I) took a bath.}//
\endgl
\xe

\pex
\begingl
\gla Um\'u\v{s}al\'i.//
\glb \mk{ref}-comb-\mk{av-prog}//
\glft \trsl{(I) am combing my hair.}//
\endgl
\xe

Below is a non-exhaustive list of verbs that are normally used in the reflexive voice:
\bigskip

\noindent
\ird{du\v{s}\'a} \trsl{to take a shower}\\
\ird{mu\v{s}\'a} \trsl{to comb}\\
\ird{\v{s}a\v{s}t\'a} \trsl{to sit down}\\

Some verbs may change meaning when used in the reflexive voice.


The reflexive voice is also used to imply that an action happened accidentally or involuntary or that the agent of the action is unknown or unimportant.

The reflexive voice may also be used emphatically, especially in spoken Iridian, to express that the action has been performed for the benefit of the actor/agent of the verb.

\pex
\begingl
\gla K\'av\'ea u\v{s}ranz\k{a}cem.//
\glb coffee-\mk{pat} \mk{ref}-drink-\mk{av-ctplv-1s}//
\glft \trsl{I'll drink coffee.} (literally, I'll drink myself coffee)//
\endgl
\xe

\pex
\begingl
\gla Pul\v{s}a uvo\v{s}\v{c}ek.//
\glb soup-\mk{pat} \mk{ref-}cook\mk{-av-pf}//
\glft \trsl{(I) cooked (me) some soup.}//
\endgl
\xe


\par The differences

\section{Grammatical Aspect}
\begin{table}[h!]
	\footnotesize\sffamily
	%\centering
	\caption{Aspect markers in the indicative mood.}
	\begin{tabu} to 0.5\textwidth{YY[0.5]}
		\toprule
		{\sc aspect}	& {\sc affix}\\
		\midrule
		Perfective		& \ird{-ek}\\
		Retrospective	& \ird{-an\'i}\\
		Imperfective	& \ird{-'al}\\
		Progressive		& \ird{-al\'i} \\
		Contemplative	& \ird{-ach/-ah}\footnote{Following Iridian orthographic rules, \ird{-ach} is used at the end of a word and \ird{-ah} elsewhere.}\\
		Prospective		& \ird{-il}\\
		\bottomrule
	\end{tabu}

\end{table}
\subsection{Perfective aspect}
The perfective aspect (glossed {\sc pf}) indicates an action that has been completed in some specific instance.

\pex
\begingl
\gla Bych na gna\v{z}a Marek vdinek.//
\glb yesterday \mk{loc} school-\mk{pat} Marek see-\mk{pv-pf}//
\glft \trsl{(I) saw Marek at school yesterday.}//
\endgl
\xe

\pex
\begingl
\gla Va\v{s}ko pia\v{s}tnek.//
\glb pastry eat-\mk{pv-pf}//
\glft `(I) ate (the) cake.'//
\endgl
\xe

\par The vowel in the suffix is unstable and the ending would normally collapse to \textbf{-k} when followed by another vowel. Consider the above two sentences followed by the second person singular clitic pronoun \textbf{-a\v{s}/e\v{s}}.

\pex
\begingl
\gla Bych na gnazsa Marek vindeke\v{s}.//
\glb yesterday \mk{loc} school-\mk{pat} Marek \mk{<pv>}see-\mk{pv-pf-2s}//
\glft `You saw Marek at school yesterday.'//
\endgl
\xe

\pex
\begingl
\gla Va\v{s}ko pinia\v{s}tka\v{s}.//
\glb pastry \mk{<pv>}eat-\mk{pf-2s}//
\glft `You ate (the) cake.'//
\endgl
\xe


\par When negated, the perfective indicates something that ought to be done but had not been done. To state that something simply did not happen, the negative of the retrospective is used instead.

\pex
\begingl
\gla Z\'at\'el\'evonirna\v{s}ek.//
\glb \mk{neg}-telephone-\mk{av-pf}//
\glft `(I) failed to call.' //
\endgl
\xe

\pex
\begingl
\gla Z\'at\'el\'evonirna\v{s}an\'i.//
\glb \mk{neg}-telephone-\mk{av-ret}//
\glft `(I) didn't call.' //
\endgl
\xe

\subsection{Retrospective aspect}
\par The retrospective aspect (glossed \mk{ret}) is used for a past action that has a continuing relevance in the presence. Consider, for example, the following sentences: (a) \textit{I went to Amsterdam last week}; and (b) \textit{I have been to France in my childhood}. Iridian would translate the verb in (a) using the perfective and the verb in (b) using the retrospective.

\pex<ret-pres1>
\begingl
\gla Hroná tímu na Budape\v{s}ta mo\v{z}la\v{s}an\'im.//
\glb three year-\mk{inst} \mk{loc} Budapest-\mk{pat} live-\mk{av-ret-1s}//
\glft `I have been living in Budapest for three years.'//
\endgl
\xe

\pex<ret-pres>
\begingl
\gla Páku \v{s}avolnan\'ic.//
\glb before-\mk{inst} hurt-\mk{pv-pf-3s.anim}//
\glft `She has been hurt before.' //
\endgl
\xe

\par The retrospective is also often used to imply non-volition or the  accidental/circumstantial nature of an action. Similarly the retrospective is used with verbs of emotion or state (e.g., \ird{cezu\v{s}talá}, ‘to become happy’ from \ird{zu\v{s}tal} ‘happy’). The perfective, on the other hand, is almost exclusively used with the causative in these cases.

\pex
\a	\begingl
\gla Vde\v{s}ek \v{s}e neicezu\v{s}tala\v{s}an\'im.//
\glb see-\mk{2s-pf} with \mk{incep}-be.happy-\mk{av-ret-1s}//
\glft `I became happy when I saw you.' //
\endgl
\a	\begingl
\gla Do pacezu\v{s}talnike\v{s}.//
\glb \mk{1s.wk} \mk{caus}-be.happy-\mk{pv-pf-2s}//
\glft `You made me happy.' //
\endgl
\xe
\pex<vasebroke>
\begingl
\gla Váz noprizan\'i.//
\glb vase break-\mk{ref-ret}//
\glft `The vase broke (accidentally).' //
\endgl
\xe

\subsection{Continuous and progressive aspects}
Iridian uses the continuous and progressive aspects to denote actions that have not been completed yet and/or are in the process of happening/occuring. The continuous aspect (glossed \mk{cont}) is used to mark a state of being while the progressive aspect (glossed \mk{prog}) is used to mark a dynamic activity.
\pex
\begingl
\gla Nau uri\v{s}tn\'al.//
\glb clothes \mk{ref-}wear-\mk{pv-cont}//
\glft \trsl{(I'm) wearing clothes.} //
\endgl
\xe

\pex
\begingl
\gla Nau uri\v{s}tnal\'i.//
\glb clothes \mk{ref-}wear-\mk{pv-prog}//
\glft \trsl{(I'm) putting on clothes.} //
\endgl
\xe

The continuous aspect is also used to denote a habitual action.

\pex
\begingl
\gla Sholu de gna\v{z}a sto\v{z}\'al.//
\glb daily-\mk{inst} \mk{ill} school-\mk{pat} go-\mk{av-cont}//
\glft \trsl{(We) go to school everyday.} //
\endgl
\xe

\pex
\begingl
\gla D\'a na Praha mo\v{z}l\'al.//
\glb \mk{1s.str} \mk{loc} Prague-\mk{pat} live-\mk{cont}//
\glft \trsl{I live in Prague.} //
\endgl
\xe

To emphasize the habitual nature of an action, a nominalized construction is often used.

\pex
\begingl
\gla Na\v{z}em r\k{a}cen\'alou.//
\glb friend-\mk{1s} smoke-\mk{cont-nz}//
\glft \trsl{My friend is a smoker.} //
\endgl
\xe

\subsection{Prospective aspect}
\par The prospective aspect (glossed {\sc prosp}) is primarily used in secondary clauses to indicate actions that are about to be started in relation to another action. It can also be used in the main clause to indicate an action in the immediate future.

\subsection{Cessative aspect}




\section{Secondary Verbal Prefixes}

\subsection{The reciprocative so-}\index{reciprocative}
The reciprocative prefix \ird{so-} is used with the agentive voice to indicate that an action is performed by the agent and the patient on each other.

\pex
\begingl
\gla Karlu sodal\v{s}al\'im \v{s}e Marek scen\v{z}ek.//
\glb Karel-\mk{inst} \mk{rec}-talk-\mk{av-prog-1s} with Marek arrive-\mk{av-pf}//
\glft \trsl{Karel and I were talking when Marek arrived.}//
\endgl
\xe

\pex
\begingl
\gla O\v{z}e na konzerta-no sovy\v{z}ek?//
\glb \mk{3pl.anim.str} \mk{loc} concert-\mk{pat=q} \mk{rec}-see-\mk{av-pf}//
\glft \trsl{Did they see each other during the concert?}//
\endgl
\xe

The use of the reciprocative inherently implies plurality on the part of the subject (the agent-patient pair). Since Iridian does not often grammaticalize plurality\index{plural}, this means the reciprocative usually won't require additional consideration as to the agreement of the constituents of the sentence; it does, however, mean that this form cannot be used with the singular form of pronouns (since pronouns formally distinguishes between singular and plural) and that most countable nouns would require the use of the particle \ird{nie} or an explicit quantifier.

\pex
\begingl
\gla To na hruma \v{s}ebou sokon\'i\v{z}ek.//
\glb \mk{dem.prox} \mk{loc} church-\mk{pat} parents \mk{rec}-wed-\mk{av-pf}//
\glft \trsl{My parents were married in this church.}//
\endgl
\xe

\pex
\begingl
\gla Nie sen\'ator so\v{z}ubal\v{s}al\'i to na televiza vy\v{z}\v{c}em.//
\glb \mk{pl} senator \mk{rec}-shout-\mk{av-prog} \mk{rz} \mk{loc} televiion-\mk{pat} see-\mk{av-pf-1s}//
\glft \trsl{I saw the senators shouting at each other on tv.}//
\endgl
\xe

Where the agent and the patient are syntactically distinct, the agent is usually presented as the topic of the sentence and the patient is marked in the comitative\index{comitative} (i.e., \ird{\v{s}e} + instrumental). Since the action itself is reciprocal, which gets marked as the agent is purely a pragmatic choice. Where one of the members of the agent-patient pair is a pronoun, preference is given to marking the pronoun as the agent (in which case \ird{\v{s}e} is normally ommitted, but with the patient remaining in the instrumental case).

\pex
\begingl
\gla Mi\v{s}ek \v{s}e Martinu soh\'evor\v{z}\'al.//
\glb Mi\v{s}ek \mk{com} Martin-\mk{inst} \mk{rec}-know-\mk{av-prog}//
\glft \trsl{Mi\v{s}ek and Martin know each other.}//
\endgl
\xe

\pex
\begingl
\gla J\'a Mi\v{s}ku soh\'evor\v{z}\'al.//
\glb \mk{2s.str} Mi\v{s}ek-\mk{inst} \mk{rec}-talk-\mk{av-prog}//
\glft \trsl{You and Mi\v{s}ek know each other.}//
\endgl
\xe

\section{Valence}

\subsection{Passive Constructions}


\subsection{Causative Constructions}\index{causative}
A causative construction is formed by the prefix \ird(ne-)

Causatives may either be lexical, analytical or morphological. Lexical causatives involve the encoding of the causation on the verb itself leading the causative form of the verb to be a different form altogether. An analytical causative, on the other hand uses a different verb (usually a verb like \emph{to do} or \emph{to make}) in conjunction with the main verb, to express the idea of causation (e.g., English \trsl{make someone do something.}) Finally, morphological causatives involve morphologically changing the main verb to express the notion of causation.

\begin{table}[h!]
\sffamily\footnotesize
\caption{Causative forms of the verb \irdp{shrad\'a}{to die.}}
	\label{tbl:causative}

    \begin{tabu}to \textwidth{Y[0.5]Y[0.6]YY}
         \toprule\addlinespace
		 										& {\sc causative } &{\sc regular meaning} & {\sc causative meaning}\\\addlinespace
												\midrule\addlinespace
				unmarked				& neshrad\'a									& to die, to be dead 	& \emph{(defective)} \\ \addlinespace
		 		Agentive				& \ird{neshr\'a\v{z}\'a}			& to kill & to cause someone to kill\\ \addlinespace
		 		Patientive			& \ird{neshradin\'a}					& to be killed & to be caused to be killed\\\addlinespace
				Benefactive			& \ird{neshrad\'eb\'a}				& to have someone die for oneself	& to have someone be killed for oneself\\\addlinespace
				Locative				& \ird{neshradoun\'a}					& to have someone related die&\emph{(defective)}\\\addlinespace
				Instrumental		& \ird{doneshradoun\'a}&to be the reason for dying&to to be used for killing\\\addlinespace
				Reflexive				& \ird{uneshra\v{z}\'a}&to kill oneself&to cause one to commit suicide\\
		 		\addlinespace
				\bottomrule

    \end{tabu}

\end{table}

Due to this suppletive nature, lexical causatives imply a more direct causation, or a tighter link between cause and event\footnote{\textcite{haiman1983} offers a thorough discussion of how the linguistic distance exhibited by the forms of causative constructions existing in a language (e.g., \emph{to cause to die} on one end of the spectrum versus \emph{to kill} on the other) correspond to the conceptual distance between the action of the causer and the result of the action to the causee. In a purely synthetic construction like \emph{kill}, for example, where the linguistic distance is the least, the conceptual distance between the action and the resulting state is also the smallest, with the opposite being true in purely analytical constructions like \emph{to cause to die}.}, than analytical or morphological causatives (\cite{velupillai2012}). Consider for example the three sentences in English below:


\pex
\a Joseph \emph{died}.\deftagex{caus}
\a Joseph \emph{killed} the man.\deftagex{caus}\deftaglabel{kill}
\a Joseph \emph{made} the man \emph{die.}\deftagex{caus}\deftaglabel{made}
\xe

The suppletive \emph{kill} in example (\getfullref{caus.kill}) implies more agency on the part of the subject than the more indirect-sounding (\getfullref{caus.made}). In (\getfullref{caus.kill}) the \emph{death} of the patient (\trsl{the man}) is the goal of the act while (\getfullref{caus.made}) it might be inferred that the \emph{dying} was an indirect consequence of an unmentioned second act.


Iridian does not employ lexical causatives as in English; instead causatives are formed morphologically by adding the prefix \ird{ne-} (glossed as \mk{caus}) to the verb stem. Although \ird{ne-} is required to form the causative morphologically, some verbs, particularly stative verbs like \irdp{shrad\'a}{to die, to be dead} in table \ref{tbl:causative} may already contain the notion of causation in some of its regular conjugated forms. This is because by default stative verbs\index{stative verb} are intransitive (i.e., the only argument required is the actor/agent\index{agent}) while some verbal voices\index{voice} like the patientive\index{patientive voice} and benefactive\index{benefactive voice} inherently imply the existence of a second and a third argument of a verb\index{argument of a verb} respectively.

%% TODO add section reference

Of course Iridian's definition of which verbs are stative and which ones are dynamic\index{dynamic verb} does not neatly align with the definition those classes have in English (v. \S~XX). For instance the verbs \emph{to stand} and \emph{to eat} are both dynamic verbs in English, while in Iridian \irdp{zdav\'a}{to stand, to be standing} is stative and only \irdp{pia\v{s}t\'a}{to eat} is dynamic. This is why as we see in example (\getfullref{statdyn.1}) below, some forms of the verb \ird{zdav\'a} already contain the notion of causation in some of its regular conjugated forms.

\pex
\a  \irdp{zdav\'a}{to be standing}\deftagex{stat-dyn}\\
		\irdp{zdav\'a}{to stand}\deftagex{stat-dyn}\\
    \irdp{zdavn\'a}{to be made standing, to erect}\\
    \irdp{nezdav\v{z}\'a}{to make so./sth. stand}\\
    \irdp{nezdavn\'a}{to be made to make so./sth. standing}
\a  \irdp{pia\v{s}t\'a}{to eat}\\
    \irdp{pia\v{s}tin\'a}{to be eaten}\\
    \irdp{nepia\v{s}\v{c}\'a}{to make someone eat}
\xe

Since causative constructions in Iridian are purely morphological\footnote{To contrast, consider Japanese which also forms causative constructions morphologically (using the suffix \emph{-sase}) but which in addition also has synthetic but not fully suppletive forms for some verbs (e.g., \irdp{agaru}{to rise} and \irdp{ageru}{to raise}).} the degree of agency of the causer can be implied from other incidental properties of the verb such as aspect or voice markings.

We pay particular attention first on the interaction of the causative prefix \ird{ne-} with the patientive voice marker \ird{-in} and the benefactive voice marker \ird{-\'eb}. We begin with stative verbs, since as mentioned earlier and in \S~XX, most stative verbs will have a causative reading when used with the agentive or benefactive voice. Stative verbs encode the state of the subject and cannot therefore express the idea of an agent nor that of a patient. By conjugating stative verbs for voice, their stative nature is therefore lost; that is why a causative cannot be derived from the unmarked form of a stative verb: a causative construction precludes the existence of a causer and a causee, which at times may be different from the subject, while the unmarked stative only that of the subject itself.


\begin{figure}[H]
	{
	\footnotesize
  \begin{forest}
    [\irdp{shrad\'a}{to die},
		[\ird{shradin\'a}\\
			patientive\\
			{Arg = 1}
				[$
				\begin{bmatrix}
					\textbf{+ Patient}
				\end{bmatrix}
				$]
				]
      [\ird{shra\v{z}\'a}\\
				agentive\\
				{Arg = 2}
					[$
					\begin{bmatrix}
						\textrm{+ Patient}\\
						\textbf{+ Agent}
					\end{bmatrix}
					$]
					]
					[\ird{ushra\v{z}\'a}\\
						reflexive\\
						{Arg = 2}
							[$
							\begin{bmatrix}
								\textrm{+ Patient}\\
								\textbf{+ Agent}
							\end{bmatrix}
							$]
							]
			[\ird{shrad\'eb\'a}\\
				benefactive\\
				{Arg = 3}
				[$
				\begin{bmatrix}
					\textrm{+ Agent}\\
					\textrm{+ Patient}\\
					\textbf{+ Benefactor}
				\end{bmatrix}
				$]
			]
		]
  \end{forest}

	}\caption[Voice markings as valence operations in stative verbs.]{Voice markings as valence operations in stative verbs. The number of elements includes all those required to create a well-formed sentence notwithstanding Iridian's tendency to drop elements that can be implied from context, with the element in bold representing whichever element is most likely to surface in speech.}
  \label{causative-reading}
\end{figure}

We see in figure \ref{causative-reading} that this causative reading of the patientive voice with stative verbs is due to properties of stative verbs and not of the patientive voice. We know this is true since this causative reading of the patientive does not exist with non-stative verbs, which are transitive by default in Iridian.

\pex
\a
\begingl
    \gla \ljudge{*}M\'amka prehlavnik.//
    \glb mother buy-\mk{pv-pf}//
    \glft \trsl{*I bought my mother.}//
\endgl
\a
\begingl
    \gla M\'amka zu\v{s}talnik.//
    \glb mother happy-\mk{pv-pf}//
    \glft \trsl{I made my mother happy.}//
\endgl
\xe

The patientive voice only requires a patient as argument; however since this argument does not exist in stative constructions, the role of an agent must first be created for the subject of the stative construction to be able to occupy the role of the patient in the patientive voice. Essentially this means that conjugating a stative verb for the patientive voice is equivalent to creating a biclausal causative construction where the subject becomes the causee and the state the action brought about by the (optionally named) causer. This reading is not possible with dynamic verbs because the patientive voice would only shift the role of the patient to that of the topic without having to create a new role for an agent.

As could have been predicted from \posscite{haiman1983} theory, these indirect forms of the causative express a more direct link between the causer and the action. Nevertheless the degree of control exerted by the causer over the action itself may vary between these constructions.

A common way to formally mark the causer's control or lack thereof in Iridian is the opposition between the retrospective aspect and the perfective aspect. Consider for example the two sentences in Iridian below, both of which have the same general translation in English.

\pex
\a
\begingl
	\gla Martin n\'esta najevec shra\v{z}ek.//
	\glb Martin deer-\mk{pat} drive-\mk{cv} die-\mk{av-pf}//
	\glft \trsl{Martin ran over a deer.} (He did it on purpose)//
\endgl
\a
\begingl
	\gla Martin n\'esta najevec shra\v{z}an\'i.//
	\glb Martin deer-\mk{pat} drive-\mk{cv} die-\mk{av-ret}//
	\glft \trsl{Martin ran over a deer.} (It was an accident.)//
\endgl
\xe

\section{Grammatical Mood}\index{mood}\index{modality|see{mood}}

\subsection{Indicative}

\subsection{Imperative and Hortative Mood}\label{sec:imp-hort}

To form commands\index{commands} and requests\index{requests}, the imperative (glossed \mk{imp}) and hortative (\mk{hort}) moods are used in Iridian.

The imperative is formed by replacing the infinitive ending \ird{-\'a} with the voice marker and the imperative ending \ird{-\'im}. The imperative\index{imperative mood} cannot be negated with the prefix \ird{z\'a-}; instead, to form a negative command the prohibitive\index{prohibitive mood} mood is used (glossed \mk{proh}), formed with the suffix \ird{-\'ema} instead of \ird{-\'im}.

\begin{table}[h!]
\sffamily\footnotesize
	\caption{Conjugation of the verb \ird{pia\v{s}t\'a}\\ in the imperative and probihibitive moods.}
	\label{tbl:imperative}

    \begin{tabu}to 0.7\textwidth{YYY}
         \toprule

         &{\sc imperative}&{\sc prohibitive}  \\
         \midrule

         Agentive &
         \ird{pia\v{s}\v{c}\'im} &
         \ird{pia\v{s}\v{c}\'ema}\\

         Patientive &
         \ird{pia\v{s}tn\'im} &
         \ird{pia\v{s}tn\'ema}\\

         Benefactive &
         \ird{pia\v{s}t\'eb\'im} &
         \ird{pia\v{s}t\'eb\'ima}\\

         Locative &
         \ird{pia\v{s}toun\'im} &
         \ird{pia\v{s}toun\'ema}\\

         Instrumental &
         \ird{dopia\v{s}toun\'im} &
         \ird{dopia\v{s}tounima}\\

         Reflexive &
         \ird{upia\v{s}\v{c}\'im} &
         \ird{upia\v{s}\v{c}\'ema}\\

         \bottomrule
    \end{tabu}

\end{table}

The imperative\index{imperative mood} is used to issue a direct command and the prohibitive to ``signal a prohibition\index{prohibitive mood}'' (SIL). Verbs in the imperative mood do not require an explicit referent, with the addressee or addressees assumed to be the recipient of the command or prohibition. When the addressee is included, it appears in the vocative case if appearing before the verb or unmarked otherwise.\footnote{A comma is placed between the verb and the addressee if the addressee appears after the verb in the sentence but none if it appears before.} Note that both the imperative and the prohibitive do not distinguish number; thus the same form of the verb will be used when giving a command to multiple addressees and to a single one.

\pex
\begingl
    \gla To hrabn\'im.//
    \glb \mk{dem} listen-\mk{pv-imp}//
    \glft \trsl{Listen to this.}//
\endgl
\xe
\pex
\a
\begingl
    \gla To hrabn\'im, Marek.//
    \glb \mk{dem} listen-\mk{pv-imp} Marek//
    \glft \trsl{Listen to this, Marek.}//
\endgl
\a
\begingl
    \gla Mark\'o to hrabn\'im.//
    \glb Marek-\mk{voc} \mk{dem} listen-\mk{pv-imp}//
    \glft \trsl{Listen to this, Marek.}//
\endgl
\xe

\pex
\begingl
    \gla Pap\'ir \v{s}virkoun\'ema.//
    \glb paper write-\mk{lv-proh}//
    \glft \trsl{Do not write anything on this sheet of paper.}//
\endgl
\xe

When used with verbal adjectives, the suffixes can attach directly to the root without any need for an explicit marker for voice and the addition of a voice marker will in fact change the meaning of the sentence. (The first two sentences below are rather unhelpful given how morphophonemic changes has rendered the imperative form with the voice marker and the one without of the verb \irdp{slouhat\'a}{to be quiet} identical, but cases like this are common and merit attention.)

\pex
\a
\begingl
    \gla Nie byl\'o slouh\'a\v{c}\'im.//
    \glb \mk{pl=} child be:quiet-\mk{imp}//
    \glft \trsl{Keep quiet, children.}//
\endgl
\a
\begingl
    \gla Nie byl\'o uslouh\'a\v{c}\'im.//
    \glb \mk{pl=} child \mk{ref}-be:quiet-\mk{av-imp}//
    \glft \trsl{Keep quiet, children.}//
\endgl
\xe

\pex
\a
\begingl
    \gla Pit\'ar zu\v{s}tal\'eb\'im.//
    \glb Pit\'ar be:happy-\mk{ben-imp}//
    \glft \trsl{Make Pit\'ar happy!}//
\endgl
\a
\begingl
    \gla Zu\v{s}tal\'im.//
    \glb be:happy-\mk{imp}//
    \glft \trsl{Be happy!}//
\endgl
\xe


Due to its directness, the use of the imperative or the prohibitive is considered impolite in most settings, and is often used only when speaking with friends, family or children. This distinction does not exist in the written language, where the imperative is used almost exclusively for these functions. However in signs that give orders or warnings (i.e., "Stop," "Do not enter") where English may sometimes use imperative constructions, Iridian uses modal constructions as they are not seen as direct commands or prohibitions.

\pex
\begingl
    \gla Tak hor\v{c}koun\'ema.//
    \glb here kill-\mk{lv-imp}//
    \glft \trsl{Sign here.}//
\endgl
\xe

\pex
\begingl
    \gla Tievna\v{z}\'ema.//
    \glb kill-\mk{av-proh}//
    \glft \trsl{Thou shalt not kill.}//
\endgl
\xe


\begin{table}[h!]
    \centering
    \begin{tabu}to 0.9\textwidth{YYY}
         \toprule
         \addlinespace
         &{\sc imperative}&{\sc prohibitive}  \\\addlinespace
         \midrule
         \addlinespace
         Agentive &
         \ird{pia\v{s}\v{c}ka} &
         \ird{pia\v{s}\v{c}ku}\\ \addlinespace

         Patientive &
         \ird{pia\v{s}tnika} &
         \ird{pia\v{s}tniku}\\ \addlinespace

         Benefactive &
         \ird{pia\v{s}t\'ebka} &
         \ird{pia\v{s}t\'ebku}\\ \addlinespace

         Locative &
         \ird{pia\v{s}tounka} &
         \ird{pia\v{s}tounku}\\ \addlinespace

         Instrumental &
         \ird{dopia\v{s}toun\'im} &
         \ird{dopia\v{s}tounima}\\ \addlinespace

         Reflexive &
         \ird{upia\v{s}\v{c}\'im} &
         \ird{upia\v{s}\v{c}\'ema}\\ \addlinespace

         \bottomrule
    \end{tabu}
    \caption{Conjugation of the verb \ird{pia\v{s}t\'a} in the imperative and probihibitive moods.}
    \label{tbl:hortative}
\end{table}

\subsection{Subjunctive}

The subjunctive mood (glossed \mk{sbj}) is used for actions or events that are not or are not known to be true or factual. The subjunctive is formed using the suffix \ird{-\'il}

\begin{table}[h!]
	\centering\small
	\caption{Conjugation of the verb \ird{pia\v{s}t\'a} in the subjunctive.}
	\begin{tabularx}{0.7\textwidth}{YYY}
		\toprule\addlinespace
					&\multicolumn{1}{c}{\sc perfective}&\multicolumn{1}{c}{\sc imperfective}\\\addlinespace
		\midrule\addlinespace
		Agentive	& pia\v{s}\v{c}\'ila	& pia\v{s}\v{c}\'il\\ \addlinespace
		Patientive	& pia\v{s}tn\'ila		& pia\v{s}tn\'il\\ \addlinespace
		Benefactive	& pia\v{s}teb\'ila		& pia\v{s}teb\'il\\ \addlinespace
		Locative	& pia\v{s}toun\'ila		& pia\v{s}toun\'il\\ \addlinespace
		Instrumental& dopia\v{s}teb\'ila	& dopia\v{s}teb\'il\\ \addlinespace
		Reflexive	& upia\v{s}\v{c}\'ila	& upia\v{s}\v{c}\'il\\ \addlinespace
		\bottomrule
	\end{tabularx}
\end{table}

In addition, the copula has two subjunctive forms, the non-negative \ird{niec} and the negative \ird{va\v{s}e}.

Note that the Iridian subjunctive makes neither temporal nor aspectual distinction.

\par The following are some specific uses of the subjunctive mood in Iridian:
\subsubsection{jussive/desiderative}
\par The subjunctive is used in indirect constructions of verbs for issuing orders, commanding, exhorting, etc.
\pex
\begingl
\gla Martin na America \v{z}no\v{z}\'il to \v{c}ezna\v{s}\'alic.//
\glb Martin \mk{loc} America-\mk{pat} study-\mk{av-sbj} \mk{rz} want-\mk{av-cont-3s.anim}//
\glft `He wants Martin to study in America.'//
\endgl
\xe

\pex
\begingl
\gla Beatles-\v{z}e >>Yesterday<< Mark\k{a} z\'a\v{s}n\'il to Tunek dálek.//
\glb Beatles-\mk{gen} ``Yesterday'' Marek-\mk{agt} sing-\mk{pv-sbj} \mk{rz} Tunek say-\mk{pf}//
\glft `Tunek told Marek to sing.'//
\endgl
\xe

\subsubsection{dubitative}
\par The subjunctive is used with verbs expressing doubt, uncertainty or disbelief.

\pex
\begingl
\gla \v{s}e //
\glb Beatles-\mk{gen} ``Yesterday'' Marek-\mk{agt} sing-\mk{sbj} \mk{rz} Tunek say-\mk{pf}//
\glft `Tunek told Marek to sing.'//
\endgl
\xe

\subsubsection{with verbs expressing emotion}

\pex
\begingl
\gla Marek za\v{s}n\'il to Tunek dálek.//
\glb Marek sing-\mk{sbj.ipf} \mk{rz} Tunek say-\mk{pf}//
\glft `Tunek told Marek to sing.'//
\endgl
\xe


\subsubsection{with the conditional mood}
\par The subjunctive is used in the main clause if the verb in the dependent clause is in the conditional \textit{irrealis} mood.

\pex
\begingl
\gla Dá prezident jenem, //
\glb a//
\glft a//
\endgl
\xe

\subsubsection{expressing judgment}

\pex
\begingl
\gla Zavno\v{c}ila\v{s} to t\'ev\'et //
\glb respond-\mk{av-sbj.ipf-2s} \mk{rz} important//
\glft \trsl{It is important that you respond.}//
\endgl
\xe

\subsubsection{irrealis}

\subsection{Conditional}
\par The conditional mood is used for conditional or hypothetical clauses. The table below shows the conjugation paradigm for the conditional mood for both regular verbs and the copula. The Iridian conditional mood is not a true conditional mood grammatically, since it is marked on the verb in the dependent clause (protasis), instead of the main clause.

\begin{table}[h!]
	\footnotesize\sffamily
	\caption{Conjugation paradigm in the conditional mood for regular \\verbs, the copula and the existential particle \ird{je\v{s}}.}
	\begin{tabu} to 0.8 \textwidth	{Y[1.3]Y[1.3]YY}
		\toprule
		&{\scshape regular verbs} & {\scshape copula} & {\scshape existential}\\
		\midrule

		\textit{Realis} &-ouhn\'a &v\'ine & jako\\
		Neg. \textit{Realis}&-ouhn\'al&ve&neko\\

		\textit{Irrealis} & -\'anie & jenem & jenem\\
		Neg. \textit{Irrealis} & -oucn\'a & jet & n\'et\\
		\bottomrule
	\end{tabu}
\end{table}

\subsubsection{Conditional Realis}

\par The conditional \textit{realis} mood (glossed \mk{cond.rl}) is used in two ways:
\begin{enumerate}
	\item In sentences that express a factual implication rather than a hypothetical situation or a potential future event, e.g., `If you heat water to 100 C, it will boil.'
	\item In `predictive' constructions, i.e., those that concern probable future events.
\end{enumerate}

The conditional \emph{realis} mood requires the verb in the main clause to be in the indicative.


\pex
\begingl
\gla Nebo 100 c\'entigr\'adu kras\'ebouhn\'a ustru\v{c}na\v{s}\'al.//
\glb water 100 Celcius-\mk{inst} heat-\mk{ben-cond.rl} \mk{ref-}boil-\mk{av-cont}//
\glft \trsl{If you heat water to 100 C, it will boil.}//
\endgl
\xe

\pex
\begingl
\gla To projekt hlupinouhn\'a kurvem zapo\v{c}n\'al.//
\glb this project fail-\mk{pv-cond.rl} job-\mk{1s} lose-\mk{pv-cont}//
\glft \trsl{If we lose this project, I will lose my job.}//
\endgl
\xe

\pex
\begingl
\gla Nahte \v{s}t\'anouhn\'a up\'i\v{c}\'al.//
\glb too:much drink-\mk{pv-cond.rl} \mk{ref}-get:drunk-\mk{av-cont}//
\glft \trsl{If you drink too much, you will get drunk.}//
\endgl
\xe

\pex
\begingl
\gla M\'em na prezna v\'ine, dek\'an\'i byr\'ova st\'o\v{z}ka.//
\glb name \mk{loc} list-\mk{pat} \mk{cop.cond.rl} dean-\mk{gen} office-\mk{pat} go-\mk{av-hort}//
\glft \trsl{If your name is on the list, please go to the dean's office.}//
\endgl
\xe

\subsubsection{Conditional Irrealis}
The conditional \textit{irrealis} mood (glossed \mk{cond.irr}) is used with hypothetical, typically counterfactual, events. The \emph{irrealis} mood requires the main clause to be in the subjunctive.


\subsection{Hortative}
\par The hortative mood is used for requests. Although Iridian has an imperative form (the unmarked form of the verb), the hortative is normally used in its place. The hortative marker should always appear at the end of the word.

	\pex
\begingl
\gla Jê\v{s}a mine\v{s}ka.//
\glb door.\mk{pat} close-\mk{2s-hort}//
\glft 'Close the door.' \textit{literally,} `May you close the door.'//
\endgl
\xe

\par To soften a command, the expression \textit{am luhninka} (may someone be thanked for\ldots) is normally used.

\pex
\begingl
\gla Jê\v{s}a minke\v{s} ce\v{s} am luhninka.//
\glb door-\mk{pat} close-\mk{pf-2s} \mk{rz.abl} because thank\mk{-pv-hort}//
\glft  `Please close the door.' \textit{literally,} `May (you) be thanked because you closed the door.'//
\endgl
\xe

\par The hortative is used with the reciprocative prefix \textbf{so-} to form the adhortative (similar to the English construction with `Let's + \mk{verb}). This construction cannot be used with \textbf{am luhninka}.

\pex
\begingl
\gla sop//
\glb door-\mk{pat} close-\mk{pf-2s} \mk{rz.abl} because thank\mk{-pv-hort}//
\glft  `Please close the door.' \textit{literally,} `May (you) be thanked because you closed the door.'//
\endgl
\xe

\subsection{Optative}
The optative mood (glossed \mk{opt}) is used for expressing wishes. The optative mood requires two aspect marking, although the primary ending is marked if it is in the imperfective mood.



\subsection{Quotative}\label{sec:quotative}\index{quotative mood}
\par The quotative mood (glossed \mk{quot}) is used to express secondhand information, or when the speaker wishes to make explicit that s/he did not witness the event himself/herself. This section deals primarily with the morphological properties of the quotative mood. See \S~\ref{sec:reportedspeech} for a discussion of the syntactical treatment of reported speech in Iridian.

The quotative mood is considered a \emph{secondary} mood in traditional Iridian linguistics since it can only be used in conjunction with other other grammatical moods. The quotative is normally formed with the suffix \ird{-e} in regular verbs, which suppletes the personal pronoun marking (if there are any) in the indicative mood, or word-finally with other moods, subject to the usual morphophonemic changes. Since it must appear as the final suffix at all times, clitic pronouns\index{clitic pronouns} cannot be used with the quotative.


\subsubsection{With the indicatve mood}

\begin{table}
	\small\centering
	\caption{Conjugation patterns}
	\label{tbl:quotind}
	\begin{tabu} to 0.8\textwidth {YYY}
		\toprule
										&	{\sc sound change pattern}				& {\sc example}\\
										\addlinespace
		\midrule
			\addlinespace
				Perfective 		&
				\ird{-ek} $\rightarrow$ \ird{ice}	&
				\ird{pia\v{s}\v{c}ek} $\rightarrow$ \ird{pia\v{s}\v{c}ice}\\
			\addlinespace
				Retrospective &
				\ird{-an\'i} $\rightarrow$ \ird{\'anie} &
				pia\v{s}\v{c}an\'i $\rightarrow$ \ird{pia\v{s}\v{c}\'anie}\\
			\addlinespace
				Imperfective &
				\ird{-\'al} $\rightarrow$ \ird{\'ale} &
				pia\v{s}\v{c}an\'i $\rightarrow$ \ird{pia\v{s}\v{c}\'anie}\\
			\addlinespace
				Progressive &
				\ird{-\'al} $\rightarrow$ \ird{\'ale} &
				pia\v{s}\v{c}an\'i $\rightarrow$ \ird{pia\v{s}\v{c}\'anie}\\
			\addlinespace
				Contemplative &
				\ird{-\'al} $\rightarrow$ \ird{\'ale} &
				pia\v{s}\v{c}an\'i $\rightarrow$ \ird{pia\v{s}\v{c}\'anie}\\
			\addlinespace
				Prospective &
				\ird{-\'al} $\rightarrow$ \ird{\'ale} &
				pia\v{s}\v{c}an\'i $\rightarrow$ \ird{pia\v{s}\v{c}\'anie}\\

	\end{tabu}

\end{table}

Where the addition of the quotative suffix \ird{-e} involves the suppletion of a clitic pronoun, the critic pronoun resurfaces elsewhere in the quoted clause, usually in its strong form. Nevertheless, given Iridian's pro-drop tendency, pronouns both in main clauses and in reported clauses are often left out to be inferred from context.

\subsubsection{Quotative forms of the copula}

%% TODO format as table
Copula
Indicative
nev\'i
hvem
Subj
nehl\'i
niec

Existential
Indicative
jeho
ne\v{z}n\'i
Subj
houve
hva\v{s}


\section{Expressing Modality}\index{modality}\label{sec:modality}

Iridian can express modality either through verbal morphology, using the affixes listed in table \ref{tbl:modality}, or through a periphrastic construction. In general a periphrastic construction is preferred when the verb is non-dynamic, i.e., the sentence is merely descriptive or stative in nature (compare, for example English \trsl{Mary can sing} vs. \trsl{Mary was able to finish baking the cake}), while the morphological method is used otherwise.

\begin{table}[h!]
    \footnotesize\sffamily
    \caption{Verbal affixes to express modality.}
    \label{tbl:modality}
    \begin{tabu}to 0.4\textwidth{YY}
			\toprule
				 {\sc modality} & {\sc marking}\\
				 \midrule
         Debitive & \ird{-al}\\
         Desiderative & \ird{-\'an}\\
         Potential &\ird{-\'et}\\
			\bottomrule
    \end{tabu}
\end{table}

The affixes used to mark modality as listed in table \ref{tbl:modality} attach directly to the verb stem, subject to the usual morphophonemic rules.

\pex
\a \irdp{pia\v{s}t\'a}{to eat}
\a \irdp{pia\v{s}tal\'a}{to need to eat}
\a \irdp{pia\v{s}t\'an\'a}{to want to eat}
\a \irdp{pia\v{s}t\'et\'a}{to be able to eat}
\xe

As in most languages, modal constructions in Iridian exhibit significant polysemy\index{polysemy} (i.e. a single construction can have one or more interpretation depending on the context). For example consider the Iridian sentence \irdp{Tom\'a\v{s} ru\v{s}ku zahvir\'etach}{Tom\'a\v{s} will be able to speak Russian,} which is marked in the potential mood. The following translations are all equally possible without any further contextual clues:

\pex
\a \trsl{Tom\'a\v{s} will be able to speak Russian, if he will study it.} (abilitative)
\a \trsl{Tom\'a\v{s} will be able to speak Russian because he will be allowed to do it.} (permissive)
\a \trsl{Tom\'a\v{s} can speak Russian and he will probably speak it later.} (true potential modality)
\xe


\subsection{Potential modality}\index{potential modality}\index{abilitative}\index{permissive}

Potential modality (glossed as \mk{pot}) is used when, in the speaker's opinion, an event is possible to occur. This definition makes the potential mood in Iridian encompass both the expressions of ability and permissibility.



\subsection{Periphrastic constructions}


\section{Non-Finite Verb Forms}

\subsection{Gerund}
\par The gerund (glossed \mk{ger}) refers to the non-finite verb form used as a noun. The gerundive prefix \ird{po-} is always used with the nominalizing suffix \ird{-ou}, both of which are added to the uninflected verb root.

\pex
\a
\begingl
\gla \v{S}\v{c}enek.//
\glb forget-\mk{pf}//
\glft \trsl{He forgot (it).}//
\endgl
\a
\begingl
\gla \v{S}\v{c}enekou Jan.//
\glb forget-\mk{pf-nz} Jan//
\glft \trsl{Jan (is) the one who forgot (it).}//
\endgl
\a
\begingl
\gla Po\v{s}\v{c}enou nauhl\'y.//
\glb \mk{ger}-forget-\mk{nz} difficult//
\glft \trsl{Forgetting is difficult.}//
\endgl
\xe

When nominalizing complex clauses, both the agent and the theme are marked in the genitive, with the agent aways appearing first.

\pex
\a
\begingl
\gla P\'a\v{s}ta Jan\k{a} vo\v{s}tnek.//
\glb pasta Jan-\mk{agt} cook-\mk{pv-pf}//
\glft \trsl{Jan cooked (some) pasta.}//
\endgl
\a
\begingl
\gla Jan\'i p\'a\v{s}t\'i povo\v{s}tou//
\glb Jan-\mk{gen} pasta-\mk{gen} \mk{ger-}cook-\mk{nz}//
\glft \trsl{Jan's cooking of the pasta}//
\endgl
\xe

The suffix \ird{-\'al}, used to mark the continuous aspect, may be infixed to the gerund to indicate that the action is repetitive.

\pex
\a
\begingl
\gla Jan nidek.//
\glb Jan stand.up-\mk{pf}//
\glft \trsl{Jan stood up.}//
\endgl
\a
\begingl
\gla Jan\'i ponid\'alou buvec.//
\glb Jan-\mk{gen} \mk{ger}-stand.up-\mk{cont-nz} annoying//
\glft \trsl{Jan's standing up again and again is annoying.}//
\endgl
\xe

\subsection{Converbs}\index{converb}
Converbs (glossed \mk{cv}) are a non-finite verb form often used for adverbial constructions. There are two converb forms in Iridian: the imperfective \ird{iec} (glossed \mk{cv.ipf}) and the perfective \ird{-e} (glossed \mk{cv.pf}).

\pex
\begingl
\gla Tereza kravniec ce nóve pal\v{z}ek. //
\glb Tereza cry-\mk{cv.ipf} \mk{abl} room-\mk{gen} leave-\mk{av-pf}//
\glft \trsl{Tereza left the room crying.}//
\endgl
\xe

\pex
\begingl
\gla Ce nóve palze Tereza neikravna\v{s}ek.//
\glb \mk{abl} room-\mk{gen} leave-\mk{cv.pf} Tereza \mk{incho}-cry-\mk{pf}//
\glft `Having left the room, Tereza started to cry.'//
\endgl
\xe

The syntax of converbial constructions and the specific uses of the perfective and imperfective converb form are discussed in detail in Section \ref{converbs-syntax}


\subsection{Supine}
The supine is a non-finite verb form formed used to indicate necessity or purpose. There are four forms as shown below:

\begin{table}[h!]
	\centering\small
	\caption{Endings used for the supine}
	\begin{tabularx}{0.8\textwidth}{MMM}
		\toprule
		&{\sc supine of purpose}&{\sc supine of necessity}\\
		\midrule
		Nominal & \textit{-ity} & \textit{-á\v{s}}\\
		\addlinespace
		Non-nominal & \textit{-ice} & \textit{-á\v{s}ce}\\
		\bottomrule
	\end{tabularx}
\end{table}


	\pex
\begingl
\gla >>Ána Karenina<< za gnazsa o\v{s}tá\v{s}ce ko hto\v{s}.//
\glb Anna Karenina for school-\mk{pat} read-\mk{sup} \mk{att} book//
\glft `I have to read \textit{Anna Karenina} for school.'//
\endgl
\xe

	\pex
\begingl
\gla Hto\v{s} vstuninkem to o\v{s}tice.//
\glb book buy-mk{pv-pf-1s} \mk{rz} read-\mk{sup}//
\glft `I bought the book to read.'//
\endgl
\xe

\par The infinitive form of the supine of purpose \textit{-icá} is used with adjectival adverbs:

\pex
\begingl
\gla Just zacep\v{s}csemem to nosiênicá.//
\glb news \mk{caus}-be.sad-\mk{1s} \mk{rz} hear-\mk{sup.inf}//
\glft `I am sad to hear the news.'//
\endgl
\xe


\section{Stative Verbs}\index{verbal adjectives|see{stative verbs}}\index{stative verbs}\index{adjectives}

Iridian lacks a distinct class of adjectives.\footnote{There is however a small class of attributives, which includes deictics\index{deictics} and quantifiers\index{quantifiers} among others, which can function as modifiers. They are different in that these words cannot be used as the predicate\index{predicate} of a sentence. They are discussed in detail on Chapter \ref{chap:minor}.} Instead, a special class of verbs called stative verbs are used to modify noun or noun-like classes. Unlike most verbs, however, stative verbs can only be marked for aspect, and optionally for voice. In addition to this base form (called the copulative), stative verbs also have an attributive form (used when the verb is preceding the noun or noun phrase) and nominative form (representing a concrete nominalization of the verb), both of which are absent in non-attributives verbs. Consider for example the verb \ird{v\v{s}ihn\'a} \trsl{to be angry}:

\pex
\a
\begingl
\gla Maty v\v{s}ihn\'al.//
\glb mother to:be:angry-\mk{cont}//
\glft \trsl{Mother is angry.}//
\endgl

\a
\begingl
\gla V\v{s}ihn\'i r\'am t\'el\'evonirna\v{s}ek.//
\glb to:be:angry-\mk{att} customer telephone-\mk{av-pf}//
\glft \trsl{An angry customer called.}//
\endgl

\a
\begingl
\gla V\v{s}ihnou t\'el\'evonirna\v{s}ek.//
\glb to:be:angry-\mk{nz} telephone-\mk{av-pf}//
\glft \trsl{An angry person called.}//
\endgl

\xe

\subsection{Copulative Form}
The copulative form of a stative verbum

\subsection{Attributive Form}
The attributive form is derived by replacing the infinitive marker \ird{-\'a} with \ird{-\'i}. Other than its conjugated comparative form ending in \ird{-en\'i}, the attributive

\ex
\begingl
\gla V\v{s}ihnou t\'el\'evonirna\v{s}ek.//
\glb to:be:angry-\mk{nz} telephone-\mk{av-pf}//
\glft \trsl{An angry person called.}//
\endgl

\xe


\subsection{Nominal Form}
The nominal form is derived by replacing the infinitive marker \ird{-\'a} with the nominalizing suffix \ird{-ou}.




The use of stative verbs in relative and comparative constructions is discussed in detail in Section \ref{relativecomparative}
\section{Derivational Morphology}
\subsection{External Derivation}
\par Loanwords ending in \textbf{-ace} from the Latin change the final e to á:
\begin{table}[h!]
	\centering \small
	\begin{tabu} to 0.9\textwidth{>{\bfseries}YM[0.3]>{\bfseries}YY}
		administrace 	& $\rightarrow$ & administracá 	& `to administrate' \\
		akuzace			& $\rightarrow$ & akuzacá		& `to accuse'\\
		diferenzace		& $\rightarrow$ & diferenzacá	& `to differentiate'\\
		separace		& $\rightarrow$ & separacá		& `to separate'\\
	\end{tabu}
\end{table}
\par Some Latin loanwords are borrowed first from German. Loanwords ending in \textbf{-ieren} become \textbf{-irná}.
\begin{table}[h!]
	\centering \small
	\begin{tabu} to 0.9\textwidth{>{\bfseries}YM[0.3]>{\bfseries}YY}
		akzeptieren 	& $\rightarrow$ & akceptirná 	& `to accept' \\
		konservieren	& $\rightarrow$ & koncervirná	& `to conserve'\\
		produzieren		& $\rightarrow$ & producirná	& `to produce'\\
		vandalieren		& $\rightarrow$ & vandalirná 	& `to deface'\\
	\end{tabu}
\end{table}
\subsection{Internal Derivation}
\begin{center}
	\small
	\begin{longtabu}to \textwidth{Y[0.5]Y}

		\caption{Verbal Derivational Affixes}
		\label{verbalder}                             \\
		\toprule
		\multicolumn{1}{c}{\sc affix} & \multicolumn{1}{c}{\sc examples}                      \\
		\midrule
		\endfirsthead
		%---------------------------------------------------------------%
		\caption{Verbal derivational affixes \hfill\textit{(continued)}}            \\
		\toprule
		\multicolumn{1}{c}{\sc affix} & \multicolumn{1}{c}{\sc examples}                      \\
		\midrule
		\endhead
		%---------------------------------------------------------------%
		\bottomrule \addlinespace
		\multicolumn{2}{r}{\footnotesize\textit{continued on the next page}}
		\endfoot

		\bottomrule
		\endlastfoot

		\textbf{nie-} + {\sc adj}\newline`to cause something to become \mk{adj}' &

		\textbf{lo\v{s}} `new' $\rightarrow$ \textbf{nielo\v{s}á} `to renew' \newline
		\textbf{preseh} `young' $\rightarrow$ \textbf{niepreshá} `to rejuvenate' \newline
		\textbf{avic} `long' $\rightarrow$ \textbf{nieavicá} `to lengthen' \newline
		\textbf{gem} `soft' $\rightarrow$ \textbf{niegemá} `to soften'\newline
		\textbf{vyne} `dry' $\rightarrow$ \textbf{nievyneá} `to dry'\\ \addlinespace

		\textbf{ce-}\footnote{Verbs in \textbf{ce-} cannot be in the reflexive focus.} + {\sc adj}\newline `to cause oneself to become {\sc adj}' &

		\textbf{kdavidy} `clean' $\rightarrow$ \textbf{cekdavicá} `to take a bath' \newline
		\textbf{rum} `old' $\rightarrow$ \textbf{cerumá} `to grow old' \newline
		\textbf{\v{s}eznom} `big' $\rightarrow$ \textbf{ce\v{s}eznomá} `to grow up' \newline
		\textbf{vyne} `dry' $\rightarrow$ \textbf{cevyneá} `to dry oneself'\\ \addlinespace

		\textbf{hó-} + {\sc noun}\newline `to use {\sc n} in a particular way' &

		\textbf{tvem} `tongue' $\rightarrow$ \textbf{hótvemá} `to lick' \newline
		\textbf{kov} `hammer' $\rightarrow$ \textbf{hóková} `to hammer' \newline
		\textbf{\v{s}eznom} `big' $\rightarrow$ \textbf{ce\v{s}eznomá} `to grow up' \newline
		\textbf{vyne} `dry' $\rightarrow$ \textbf{cevyneá} `to dry oneself'\\ \addlinespace

		\textbf{de\v{s}-} + {\sc noun}\newline `to act in the manner of {\sc n}  &

		\textbf{tvem} `tongue' $\rightarrow$ \textbf{hótvemá} `to lick' \newline
		\textbf{rum} `old' $\rightarrow$ \textbf{cerumá} `to grow old' \newline
		\textbf{\v{s}eznom} `big' $\rightarrow$ \textbf{ce\v{s}eznomá} `to grow up' \newline
		\textbf{vyne} `dry' $\rightarrow$ \textbf{cevyneá} `to dry oneself'\\ \addlinespace

		\textbf{má-iv} + {\sc noun}\newline `to so something usually done in {\sc noun}'  &

		\textbf{mrc} `market' $\rightarrow$ \textbf{mámrcivá} `to shop' \newline
		\textbf{gnazsa} `school' $\rightarrow$ \textbf{mágnazsivá} `to study in'  \newline
		\textbf{\v{s}eznom} `big' $\rightarrow$ \textbf{ce\v{s}eznomá} `to grow up' \newline
		\textbf{vyne} `dry' $\rightarrow$ \textbf{cevyneá} `to dry oneself'\\ \addlinespace


		\textbf{sen-/sem-} + {\sc verb}\newline `to {\sc verb} incorrectly'  &

		\textbf{o\v{s}tá} `to read' $\rightarrow$ \textbf{seno\v{s}tá} `to misread' \newline
		\textbf{rum} `old' $\rightarrow$ \textbf{cerumá} `to grow old' \newline
		\textbf{\v{s}eznom} `big' $\rightarrow$ \textbf{ce\v{s}eznomá} `to grow up' \newline
		\textbf{vyne} `dry' $\rightarrow$ \textbf{cevyneá} `to dry oneself'\\ \addlinespace
	\end{longtabu}
\end{center}
