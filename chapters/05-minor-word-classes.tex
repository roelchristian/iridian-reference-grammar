\chapter{Minor word classes}\label{chap:minor}\index{minor word classes}

\section{Conjunctions}\label{sec:conj}

\subsection{Connective conjunctions}\label{sec:conn-conj}

Sentences of the type

\ex
It is [\mk{adjective}] that[ \mk{subordinate clause}].
\xe

are normally translated in Iridian using an expletive-\ird{a} construction, with the adjective in the attributive form at the start of the phrase, followed by \ird{a}, and then by the rest of the main clause. Normally this construction is used for sentences that pass judgment to the action or state described in the main clause, although in some cases the adjective is simply used for description.

\pex
\begingl
    \gla Interezní a téknik znohouštnilá te prádelnik.//
    \glb interesting-\Att{} and engineering study\mk{-pv-sbj.ipf} \mk{rz} choose-\Pv{}-\Pf{}//
    \glft \trsl{It is interesting that you chose to study engineering.}//
\endgl
\xe
\pex
\begingl
    \gla Komí a já ščenžek.//
    \glb good-\Att{} and \mk{2s.str} arrive-\Av{}-\Pf{}//
    \glft \trsl{Good you're here now!}//
\endgl
\xe

Another common use of the expletive \ird{a} is with the word \irdp{shlac}{now} (pronounced [sxlat] instead of the more intuitive [sxlat͡s]) to form the phrase \ird{shlac a}\footnote{This is therefore pronounced [ˈsxlatɐ].}, which is used to introduce a subordinate clause, similar to \trsl{now that} in English.

\pex
\begingl
    \gla Shlac a provísor ščenžek, kurs šelčinách.//
    \glb now and professor arrive-\Av{}-\Pf{} class begin-\mk{pv-ctpv}//
    \glft \trsl{Now that the professor is here, we will begin our class.}//
\endgl
\xe


\section{Prepositions}

\subsection{na}

\subsection{še}

\subsection{\ird{vo}}\index{vo}\index{agentive case}

\ird{Vo} can be translated as \trsl{because of} or \trsl{due to.} This preposition takes the agentive case.

\pex
\begingl
\gla Vo transitám lienu záscenzčem.//
\glb because traffic-\Agt{} on:time-\Ins{} \Neg{}-arrive-\mk{av-pf-1s}//
\glft \trsl{I didn't arrive on time because of the traffic.}//
\endgl
\xe

\subsection{za}


\section{Demonstratives}\label{dem-adj}\index{demonstratives}

\section{Quantifiers}\index{quantifiers}
Iridian has a wide variety of non-numerical/indefinite quantifiers.  Most are actually nouns that used in adjectival or adverbial constructions.


\begin{itemize}
    \item \ird{ošč} \trsl{many} (countable)
    \ex
    \begingl
    \gla Marka ješ naže ošč.//
    \glb Marek-\Acc{} \Exst{} friend-\Gen{} many//
    \glft \trsl{Marek has many friends.}//
    \endgl
    \xe
    \ex
    \begingl
    \gla Za kursa mén ješ ošč oudinášce ko vilm.//
    \glb for class-\Acc{} \mk{1pl.inc.wk} \Exst{} many watch-\Sup{} \Att{} film.//
    \glft \trsl{We have a lot of movies we need to watch for our class.}//
    \endgl
    \xe
    \item \ird{nave} \trsl{too many} (countable)
    \ex
    \begingl
    \gla Marka ješ naže ošš.//
    \glb Marek-\Acc{} \Exst{} friend-\Gen{} many//
    \glft \trsl{Marek has many friends.}//
    \endgl
    \xe
    \item \ird{tohle} \trsl{many} (uncountable)
    \item \ird{nahte} \trsl{too many, too much} (uncountable)
    \ex
    \begingl
    \gla Do ješ nahte kurváš//
    \glb \mk{1s.wk} \Exst{} too:much work-\mk{sup.nom}//
    \glft \trsl{I have so much work to do.}//
    \endgl
    \xe

\end{itemize}

\section{Interjections}

An interjection\index{interjection} is a word or an expression used to express a spontaneous reaction or feeling. We will use the term `interjection' to refer both to the part of speech and to the utterance type that has the same pragmatic function as this part of speech (cf. \cite{ameka1992}).

Interjections can be classifed into two main categories: \emph{primary} interjections, which refer to a word or an utterance that can only be used as an interjection and \emph{secondary} interjections, which refer to forms belonging a different word class but which through its usage, has acquired a new meaning as an interjection.

Although interjections can function as exclamations, not all exclamatory utterances can be considered as interjectons by themselves. As \textcite{jovanovic2004} notes, any word in a language can theoretically become an exclamation. Consider for example this conversation:

\ex (adapted from \cite{jovanovic2004}).\\

  \ird{
  \noindent--- Martin mlaza boulešik.\\
  --- \textbf{Martinám?}
  }\medskip

  \trsl{I heard Martin killed his brother.}\\
  \trsl{Martin?!}
\xe


\section{Discourse particles}

\subsection{Yes and no}
Iridian has several words for yes and no but their usage in responding to yes-no questions does not exactly align with that of English. This is discussed in detail in \S\,\ref{sec:ansyn}.

There are two main words for \trsl{yes} in Iridian: the affirmative \ird{dé} (\trsl{Did you see it?} \trsl{Yes, I did.}) and the contrastive \ird{če} (\trsl{Did you not see it?} \trsl{Yes, I did.}. The distinction is similar as that between the French \emph{oui} and \emph{si}. Both \ird{dé} and \ird{če} generally appear at the end of a sentence. In colloquial spoken Iridian it is also common to see the form \ird{ja} (most likely from the Czech, and ultimately from the German \emph{ja}) and the more informal \ird{jó}. These forms however are not cliticized to the verb and appear at the start of a sentence, set off from the rest with a commma. Both \ird{ja} and \ird{jó} cannot be used contrastively like \ird{če}. It is also common to use both \ird{ja/jó} at the same time as \ird{dé}.

\pex
\begingl
\gla ---To vdinice? ---Ja vdinek dé.//
\glb this see-\Pv{}-\Pf{}-\Quot{} yes see-\Pv{}-\Pf{} yes//
\glft \trsl{{}``Did you see it?'' ``Yes, I did.''{}}//
\endgl
\xe

When used by themselves, both \ird{ja} and \ird{jó} are often repeated twice or thrice (e.g., \ird{Ja ja ja.})\footnote{Commas are not used to separate each \ird{ja} or \ird{jó} in standard orthography. } even when the usage is not emphatic. \ird{Dé} and \ird{če} cannot be used this way.

\section{Numerals}
\par Iridian has a vigesimal number system. Table \ref{one20} shows Iridian numerals from 1 to 20. Numbers from 1 to 10 are given their own name while numbers from 11 to 19 are formed by appending the numbers from one to nine to the clitic \ird{-niem} with the preposition \ird{še} (with). The clitic \ird{-niem} is derived from the word for number 10, \ird{nau}, which itself comes from the Old Iridian \rec{nagu}, `half.'
\begin{table}[h!]
		\caption{Iridian numerals from 1 to 20.}
		\medskip
		\small

\begin{tabu}to 0.8 \textwidth {Y[0.7]YY[0.7]Y}
	\toprule
	{\sc number} & {\sc iridian} & {\sc number} & {\sc iridian}\\
	\midrule
	1 & ona			& 11 & onšeniem\\
	2 & m\"y			& 12 & myšeniem\\
	3 & hroná		& 13 & hronašeniem\\
	4 & drou			& 14 & drušeniem\\
	5 & jed			& 15 & jecniem\\
	6 &	vou			& 16 & vušeniem\\
	7 & šč\k{e}	& 17 & šč\k{e}ceniem\\
	8 & pieš		& 18 & pi\k{e}ceniem\\
	9 & cam			& 19 & camzeniem\\
	10& nau			& 20 & tydná\\

	\bottomrule
	\label{one20}
\end{tabu}
\end{table}

For numbers 11 to 19, the words are formed by appending the numbers from one to nine to the suffix \textit{-niem} with the preposition \irdp{še}{with}.

Numbers from 21 to 99 are first expressed as multiples of 20. Thenceforth, the number system has largely become decimal, due primarily to the influence of surrounding Indo-European languages. Old Iridian, however, had a vigesimal system up to the number 8000.

Table \ref{3099} shows multiples of 10 from 30 to 100. The numbers are formed by the numeral followed by \ird{tydná}. For bases that are not multiples of 20, the word \irdp{nau}{ten} is added first, followed by the conjunction \irdp{še}{with}.

\begin{table}[h!]
	\caption{Iridian numerals from 30 to 100.}
	\medskip
	\small
	\begin{tabu}to 0.9 \textwidth {Y[0.5]YY[0.5]Y}
		\toprule
		{\sc number} & {\sc iridian} & {\sc number} & {\sc iridian}\\
		\midrule
		30 &	naušetydná		& 70 	& naušehronutydná\\
		40 &	mytydná		& 80	& drohutydná\\
		50 &	naušemytydná	& 90	& naušedrohutydná\\
		60 &	hronutydná		& 100	& miesy\\
		\bottomrule
		\label{3099}
	\end{tabu}
\end{table}

Iridian counting starts from the smallest component of the number to the largest. Each component can be simply appended with the conjunction \ird{še}. Only the numerals in Tables \ref{one20} and \ref{3099}, and the first ten numbers after 100, 500, 1000, etc. appear as single words. Below are some illustrations:

\pex
\a \ird{jecemiesy}\\
	\trsl{five with hundred,} i.e., 105
\a \ird{cam še drohutydná}\\
	\trsl{nine with four twenties,} i.e., 89
\a \ird{pi\k{e}ceniem še hronutydná}\\
	\trsl{eighteen with three twenties,} i.e., 78
\xe

\begin{table}[h!]
	\caption{Iridian numerals from 200 to one billion.}
	\medskip
	\small
	\begin{tabu}to 0.9 \textwidth {Y[0.6]Y}
		\toprule
		{\sc number} & {\sc iridian} \\
		\midrule
		200 			&	mach	\\
		300, 400, etc.	& 	hronumiesy, drohumiesy. etc.\\
		1000			& 	nic\\
		2000, 3000, etc.& 	myniec, hronuniec, etc.\\
		10.000			&	ohle\\
		20.000, etc.	& 	tydnuniec, etc.\\
		100.000			&	hazlek\\
		200.000 etc		&	mehdeniec, hronuniec, etc.\\
		1.000.000		&	miliám\\
		1.000.000.000	&	milár\\
		1.000.000.000.000	& biliám\\
		\bottomrule
		\label{3099}
	\end{tabu}
\end{table}

\subsection{Ordinal numbers}
Except for the first three cardinal numbers that have irregular ordinal forms, ordinals are mostly regular, formed with the suffix \ird{-šle} (or \ird{-išle} after consonants). The ordinal form of the numbers one, two and three are \ird{hezka}, \ird{dviec} and \ird{cehra}, respectively. When written as numerals, a full stop is used as in German (e.g., \irdp{camišle}{ninth} would be written 9.).

The letter n has its own ordinal form (cf. English \trsl{nth} for example), \ird{enišle}, as do the rest of the other letters. These ordinal forms are generally regular. Their usage is confined to mathematical literature, however, with the clear exception of \ird{enišle}, which is often used idiomatically (cf. French \textit{pour la eni\`eme fois}).

\subsection{Fractions and decimals}

As with most languages in Europe, Iridian uses the comma (Iridian \ird{kvá}) to separate whole number from decimals. Numbers after the comma are read in pairs of two, with the first number read separately in case there is an odd number of numerals after the comma (e.g., 3,34 is read as \ird{hroná kvá drušeniem še tydná} while 3,346 is read \ird{hroná kvá hroná vou še mytydná}). If there are seven or more numbers following the come, each is read separately instead.

Fractional forms are also regularly formed using the suffix \ird{-izmek}. The word for half, \ird{niet}, however is irregular. Fractional forms are sometimes used together with the regular decimal forms when dealing with currency. For example, 5,50 kr. can be read as either \ird{jed kvá naušemytydná korun} or more commonly \ird{jed še niet korun}.

\subsection{Date and time}
Dates are written with the year first, followed by the month, and ultimately by the date. When written in numerals, the numbers are separated by a full stop. When spoken or when written in full, the number representing the year is followed by the word \irdp{hlet}{year}, often in the instrumental case. When followed by the name of the month, \ird{hlet} is declined in the genitive. When the date is included, the ordinal form is used, followed by the word \irdp{ráz}{day,} although the latter may be dropped in casual speech. The inclusion of the date also requires the name of the month to be in the genitive case.

\pex
\a
\begingl
    \gla 1992 hletí julí 15. rázu veštašik //
    \glb 1992 year\mk{-gen} july-\Gen{} 15th day-\Ins{} be:born-\Av{}-\Pf{}//
    \glft \trsl{I was born on 5 July 1992.}//
\endgl
\xe

\begin{table}[h!]
	\caption{Months of the year.}
	\medskip
	\small
	\begin{tabu}to 0.7 \textwidth {YYYY}
		\toprule
		{\sc month} & {\sc iridian} & {\sc month} & {\sc iridian}\\
		\midrule
		January		& jenvár	& July & jul\\
		February	& fevrár 	& August & augošt\\
		March		& merc		& September & seitembár\\
		April		& april 	& October & oktobár\\
		May 		& mai 		& November & novembár\\
		June 		& jón 	& December & dicámbár\\
		\bottomrule
		\label{3099}
	\end{tabu}
\end{table}

