\chapter{Minor word classes}\label{chap:minor}
\index{minor word classes}

In the preceding two chapters, we have looked at the three major word classes in Iridian: nouns, verbs, and modifiers. In this chapter we will look at the remaining word class, the function words, which we further divide into adverbial particles, conjunctions, prepositions, discourse markers, interjections, and numerals.

\section{Adverbial particles}\label{sec:adv-particles}

Adverbial particles are a small class of words that are similar to one another
in exhibiting proclitic behavior. Notwithstanding certain predictable exceptions
discussed in this section, adverbial particles must obligatorily appear before
the predicate they modify. While we use the term `adverbial particle' to refer
to this class of words, their usage is rather varied and some may function as
discourse markers. A single particle may also be used in multiple ways, as we
will see in the following sections.

Iridian has the following adverbial particles: 

 
When two or more particles are proclitic to the same predicate, their relative word order may be described in terms of the following hierarchy, in relation to their distance from the predicate:
\begin{itemize}
	\item Class 1: \ird{že} and \ird{po}
	\item Class 2: \ird{li}, \ird{može} and \ird{by}
	\item Class 3: \ird{daw}
\end{itemize}

\subsection{Class 1: \ird{že} and \ird{po}}

The two Class 1 adverbial particles \ird{že} and \ird{po} never occur in
immediate sequence to each other. In general, \ird{že} and \ird{po} carry
aspect-related meanings, with \ird{že} used to indicate the perfective and
\ird{po} the imperfective aspect. This usage, however, does not completely
correspond to the true aspectual suffixes on a verb, as we have seen in
\S~\ref{sec:aspect}. \ird{Že} and \ird{po} can be broadly translated as
\trsl{already} and \trsl{still/yet}, respectively, but their usage as we will
see below is more complex.

In sentences containing a temporal clause expressing a future event, \ird{že}
and \ird{po} are used to indicate the attitude of the speaker towards the time
described in the predicate. \ird{Že} `extends' the perceived time between the
reference point and the time described by the predicate, while \ird{po}
`shortens' it. Thus a neutral sentence such as \irdp{Janek sobotu
ščenžách}{Janek will arrive on Saturday} can be modified as \ird{Janek sobotu že
ščenžách} or as \ird{Janek sobotu po ščenžách.} The former indicates that the
speaker thinks that there is little time left before Janek's arrival on
Saturday, while the latter indicates that the speaker thinks that there is still
a lot of time left before Janek's arrival on Saturday. In contrast to these two,
the original sentence without \ird{že} or \ird{po} does not pass any judgment on
the time left before Janek's arrival on Saturday. With temporal clauses
expressing past events, \ird{po} behaves the same way in `extending' the
perceived time between the reference point and the time described by the
predicate; \ird{že} on the other hand cannot be used in this way. Thus
\irdp{Janek sobotu ščenžek}{Janek arrived on Saturday} can be modified as
\ird{Janek sobotu po ščenžek} which can be interpreted as \trsl{Janek arrived on
Saturday (and it has been quite some time since then).} \ird{Janek sobotu že
ščenžek} is also a valid sentence, but here \ird{že} merely translates as
\trsl{already} and does not have the aspectual connotation it has in the future
tense.

The usage of \ird{že} and \ird{po} described in the previous paragraph is
limited to sentences that satisfy two criteria: (1) the sentence must contain an
explicit temporal clause that specifies the point in time when the action
described in the predicate will take place or has taken place, and (2) the
action or the state must be in the future or past, which means if the predicate
is a verb it must be in the perfective, retrospective or contemplative aspect.
These criteria are required since the length of time upon which the speaker
passes judgment can only be established by first defining a reference point
(i.e., the time of speaking) and another point in time that will serve as the
beginning (in the case of past events) or the end (in the case of future events)
of the time interval. In sentences that do not satisfy these criteria, \ird{že}
and \ird{po} will have different meanings.

\section{Conjunctions}\label{sec:conj}

\subsection{Connective conjunctions}\label{sec:conn-conj}

Sentences of the type

\ex
It is [\mk{adjective}] that[ \mk{subordinate clause}].
\xe

are normally translated in Iridian using an expletive-\ird{a} construction, with the adjective in the attributive form at the start of the phrase, followed by \ird{a}, and then by the rest of the main clause. Normally this construction is used for sentences that pass judgment to the action or state described in the main clause, although in some cases the adjective is simply used for descrciption.

\pex
\begingl
    \gla Interezní a téknik znohouštnilá te prádelnik.//
    \glb interesting-\Att{} and engineering study\mk{-pv-sbj.ipf} \mk{rz} choose-\Pv{}-\Pf{}//
    \glft \trsl{It is interesting that you chose to study engineering.}//
\endgl
\xe
\pex
\begingl
    \gla Komí a já ščenžek.//
    \glb good-\Att{} and \Second{}\Sg{} arrive-\Av{}-\Pf{}//
    \glft \trsl{Good you're here now!}//
\endgl
\xe

Another common use of the expletive \ird{a} is with the word \irdp{shlac}{now} (pronounced [sxlat] instead of the more intuitive [sxlat͡s]) to form the phrase \ird{shlac a}\footnote{This is therefore pronounced [ˈsxlatɐ].}, which is used to introduce a subordinate clause, similar to \trsl{now that} in English.

\pex
\begingl
    \gla Shlac a provísor ščenžek, kurs šelčinách.//
    \glb now and professor arrive-\Av{}-\Pf{} class begin-\mk{pv-ctpv}//
    \glft \trsl{Now that the professor is here, we will begin our class.}//
\endgl
\xe


\section{Prepositions}

\subsection{na}

Iridian has a single locative preposition, \ird{na}, which is used to indicate the location of an object or person. It is used in the same way as the English \trsl{on} or \trsl{in}. \ird{Na} is followed by a noun or a noun phrase in the accusative case.

\pex
\a \ird{na duma}, \trsl{at home}
\a \ird{na škole}, \trsl{at school}
\a \ird{na vele}, \trsl{in the countryside}
\xe

Where English uses specific prepositions such as \trsl{above}, \trsl{under}, \trsl{below}, etc., Iridian uses a compound construction with \ird{na} and another noun indicating the location marked in the accusative. The `object' in the equivalent English construction is marked in the genitive in Iridian.

\pex
\a \ird{na dumí veha}, \trsl{in front of the house}
\a \ird{na bamení pouda}, \trsl{behind the building}
\xe

\subsection{še}

\subsection{\ird{vo}}\index{vo}\index{agentive case}

\ird{Vo} can be translated as \trsl{because of} or \trsl{due to.} This preposition takes the agentive case.

\pex
\begingl
\gla Vo transitám lienu zásčenžek.//
\glb because traffic-\Agt{} on:time-\Ins{} \Neg{}-arrive-\Av{}-\Pf{}//
\glft \trsl{I didn't arrive on time because of the traffic.}//
\endgl
\xe

\subsection{za}

\section{Quantifiers}\index{quantifiers}
Iridian has a wide variety of non-numerical/indefinite quantifiers.  Most are actually nouns that used in adjectival or adverbial constructions.


\begin{itemize}
    \item \ird{ošč} \trsl{many} (countable)
    \ex
    \begingl
    \gla Marka ješ naže ošč.//
    \glb Marek-\Acc{} \Exst{} friend-\Gen{} many//
    \glft \trsl{Marek has many friends.}//
    \endgl
    \xe
    \ex
    \begingl
    \gla Za kursa mén ješ ošč oudinášce ko vilm.//
    \glb for class-\Acc{} \mk{1pl.inc.wk} \Exst{} many watch-\Sup{} \Att{} film.//
    \glft \trsl{We have a lot of movies we need to watch for our class.}//
    \endgl
    \xe
    \item \ird{nave} \trsl{too many} (countable)
    \ex
    \begingl
    \gla Marka ješ naže ošš.//
    \glb Marek-\Acc{} \Exst{} friend-\Gen{} many//
    \glft \trsl{Marek has many friends.}//
    \endgl
    \xe
    \item \ird{tohle} \trsl{many} (uncountable)
    \item \ird{nahte} \trsl{too many, too much} (uncountable)
    \ex
    \begingl
    \gla Do ješ nahte kurváš//
    \glb \First{}\Sg{}.\Acc{} \Exst{} too:much work-\SupN{}//
    \glft \trsl{I have so much work to do.}//
    \endgl
    \xe

\end{itemize}

\section{Interjections}

An interjection\index{interjection} is a word or an expression used to express a spontaneous reaction or feeling. We will use the term `interjection' to refer both to the part of speech and to the utterance type that has the same pragmatic function as this part of speech (cf. \cite{ameka1992}).

Interjections can be classifed into two main categories: \emph{primary} interjections, which refer to a word or an utterance that can only be used as an interjection and \emph{secondary} interjections, which refer to forms belonging a different word class but which through its usage, has acquired a new meaning as an interjection.

Although interjections can function as exclamations, not all exclamatory utterances can be considered as interjectons by themselves. As \textcite{jovanovic2004} notes, any word in a language can theoretically become an exclamation. Consider for example this conversation:

\ex (adapted from \cite{jovanovic2004}).\\

  \ird{
  \noindent--- Martin mlaza boulešik.\\
  --- \textbf{Martinám?}
  }\medskip

  \trsl{I heard Martin killed his brother.}\\
  \trsl{Martin?!}
\xe


\section{Discourse particles}

\subsection{Yes and no}
Iridian has several words for yes and no but their usage in responding to yes-no questions does not exactly align with that of English. This is discussed in detail in \S\,\ref{sec:ansyn}.

There are two main words for \trsl{yes} in Iridian: the affirmative \ird{dé} (\trsl{Did you see it?} \trsl{Yes, I did.}) and the contrastive \ird{če} (\trsl{Did you not see it?} \trsl{Yes, I did.}. The distinction is similar as that between the French \emph{oui} and \emph{si}. Both \ird{dé} and \ird{če} generally appear at the end of a sentence. In colloquial spoken Iridian it is also common to see the form \ird{ja} (most likely from the Czech, and ultimately from the German \emph{ja}) and the more informal \ird{jó}. These forms however are not cliticized to the verb and appear at the start of a sentence, set off from the rest with a commma. Both \ird{ja} and \ird{jó} cannot be used contrastively like \ird{če}. It is also common to use both \ird{ja/jó} at the same time as \ird{dé}.

\pex
\begingl
\gla ---To vdinice? ---Ja vdinek dé.//
\glb this see-\Pv{}-\Pf{}-\Quot{} yes see-\Pv{}-\Pf{} yes//
\glft \trsl{{}``Did you see it?'' ``Yes, I did.''{}}//
\endgl
\xe

When used by themselves, both \ird{ja} and \ird{jó} are often repeated twice or thrice (e.g., \ird{Ja ja ja.})\footnote{Commas are not used to separate each \ird{ja} or \ird{jó} in standard orthography. } even when the usage is not emphatic. \ird{Dé} and \ird{če} cannot be used this way.

\section{Numerals}\label{sec:numerals}

Iridian has a vigesimal number system. Table \ref{tab:nums-one-twenty} shows
Iridian numerals from 1 to 20. Numbers from 1 to 10 are given their own name
while numbers from 11 to 19 are formed by appending the numbers from one to nine
to the clitic \ird{-něm} with the preposition \ird{še} (with). The clitic
\ird{-něm} is derived from the word for number 10, \ird{nau}, which itself
comes from the Old Iridian \rec{nagu}, `half.'

\begin{table}
\footnotesize\sffamily
\caption{Iridian numerals from 1 to 20.}
\medskip
\begin{tblr}{width=0.8\textwidth,colspec={X[0.7]XX[0.7]X}}
	\toprule\addlinespace
	{\sc number} & {\sc iridian} & {\sc number} & {\sc iridian}\\ \addlinespace
	\midrule \addlinespace
	1 & ona			& 11 & onšeněm\\ \addlinespace
	2 & vuc			& 12 & myšeněm\\ \addlinespace
	3 & hrona		& 13 & hronašeněm\\ \addlinespace
	4 & drou		& 14 & drušeněm\\ \addlinespace
	5 & jed			& 15 & jecněm\\ \addlinespace
	6 &	dve			& 16 & vušeněm\\ \addlinespace
	7 & šče			& 17 & ščiceněm\\ \addlinespace
	8 & pieš		& 18 & pisčeněm\\ \addlinespace
	9 & cam			& 19 & camšeněm\\ \addlinespace
	10& nou			& 20 & týdna\\ \addlinespace
	\bottomrule
	\label{tab:nums-one-twenty}
\end{tblr}
\end{table}

Numbers from 21 to 99 are first expressed as multiples of 20. Thenceforth, the
number system has largely become decimal, due primarily to the influence of
surrounding Indo-European languages. Old Iridian, however, had a vigesimal
system up to the number 8000.

Table \ref{tab:nums-thirty-one-hundred} shows multiples of 10 from 30 to 100.
The numbers are formed by the numeral followed by \ird{týdna}. For bases that
are not multiples of 20, the word \irdp{nau}{ten} is added first, followed by
the conjunction \irdp{še}{with}.

\begin{table}
	\footnotesize\sffamily
	\caption{Iridian numerals from 30 to 100.}
	\medskip
	\begin{tblr}{width=0.8\textwidth,colspec={X[0.5]XX[0.5]X}}
		\toprule \addlinespace
		{\sc number} & {\sc iridian} & {\sc number} & {\sc iridian}\\ \addlinespace
		\midrule \addlinespace
		30 &	naušetýdna		& 70 	& naušehronutýdna\\ \addlinespace
		40 &	vutýdna			& 80	& drohutýdna\\ \addlinespace
		50 &	nauševutýdna	& 90	& naušedrohutýdna\\ \addlinespace
		60 &	hronutýdna		& 100	& měs\\ \addlinespace
		\bottomrule
		\label{tab:nums-thirty-one-hundred}
	\end{tblr}
\end{table}

Iridian counting starts from the smallest component of the number to the
largest. Each component can be simply appended with the conjunction \ird{a}. The
examples below illustrate the formation of more complex numbers. Table
\ref{tab:nums-two-hundred-one-trillion} shows Iridian numerals from 200 to one
billion.

\pex
\a \irdp{jed a měs}{five and hundred} i.e., 105
\a \irdp{cam a drohutýdna}{nine and four twenties} i.e., 89
\a \irdp{pisčeněm a hronutýdna}{eighteen and three twenties} i.e., 78
\xe

\begin{table}
	\footnotesize\sffamily
	\caption{Iridian numerals from 200 to one trillion.}
	\medskip
	\begin{tblr}{width=0.9\textwidth,colspec={X[0.6]X}}

		\toprule \addlinespace
		{\sc number} & {\sc iridian} \\ \addlinespace
		\midrule \addlinespace
		200 			&	mach	\\ \addlinespace
		300, 400, etc.	& 	hronuměs, drohuměs. etc.\\ \addlinespace
		1000			& 	něk\\ \addlinespace
		2000, 3000, etc.& 	vuněk, hronuněk, etc.\\ \addlinespace
		10.000			&	ohle\\ \addlinespace
		20.000, etc.	& 	t\'ydnuněk, etc.\\ \addlinespace
		100.000			&	hazlek\\ \addlinespace
		200.000 etc		&	mehdeněk, hronuněk, etc.\\ \addlinespace
		1.000.000		&	miliám\\ \addlinespace
		1.000.000.000	&	milár\\ \addlinespace
		1.000.000.000.000	& biliám\\ \addlinespace
		\bottomrule
		\label{tab:nums-two-hundred-one-trillion}
	\end{tblr}
\end{table}

When used attributively, numerals do not require the particle \ird{ty} to be
linked to a noun or a noun phrase. The numerals \ird{týdna}, \ird{měs},
\ird{mach}, \ird{něk}, \ird{ohle}, \ird{hazlek}, \ird{miliám}, \ird{milár} and
\ird{biliám} may reduplicated, separated by \irdp{a}{and} to indicate an
estimate (e.g., \irdp{měs a měs cel}{hundreds of people}). This is similar to
the pluralization of numerals in English for the same purpose, as in, for
example, \trsl{tens} or \trsl{hundreds}. 

Numerals may also be used nominally, in which case they are inflected like
regular nouns.

\subsection{Ordinal numbers}
\label{sec:ordinals}

Except for the first three cardinal numbers that have irregular ordinal forms,
ordinals are mostly regular, formed with the suffix \ird{-šle} (or \ird{-išle}
after consonants). The ordinal form of the numbers one, two and three are
\ird{hezka}, \ird{dvěc} and \ird{cehra}, respectively. The ordinal form is
\ird{měs} is also irregular, being \ird{měšle} and not \ird{*měšišle}. When
written as numerals, a full stop is used followed by a dash (e.g.,
\irdp{camišle}{ninth} would be written 9.-). In compound numbers, only the last
component is inflected with \ird{-šle}; `eighty seventh' for example would be
\ird{šče a drohutýdnišle} and not \ird{*ščišle a drohutýdna}.

The letter n has its own ordinal form (cf. English \trsl{nth} for example),
\ird{enišle}, as do the rest of the other letters. These ordinal forms are
generally regular. Their usage is confined to mathematical literature, however,
with the clear exception of \ird{enišle}, which is often used idiomatically (cf.
French \textit{pour la enième fois}).


\subsection{Fractions, decimals and other derivative forms}
\label{sec:fractions}

As with most languages in Europe, Iridian uses the comma (Iridian \ird{kva}) to
separate whole number from decimals (see \S~\ref{sec:punctuation}). Numbers
after the comma are read in pairs of two, with the first number read separately
in case there is an odd number of numerals after the comma (e.g., 3,34 is read
as \ird{hrona kva drušeněm a týdna} while 3,347 is read \ird{hrona kva hrona šče
a vutýdna}). Generally, if there are seven or more numbers following the come,
each is read separately instead, though this is not a hard and fast rule and the
speaker may read the numbers separately even if there are fewer than seven
decimal places.

Fractional forms are regularly formed using the suffix \ird{-izmek}. The word
for half, \ird{num}, however is irregular. Fractional forms are sometimes used
together with the regular decimal forms when dealing with currency. For example,
5,50 kr. can be read as either \ird{jed kva nauševutýdna korun} or more commonly
\ird{jed a num korun} (cf. English \trsl{five and a half dollars}).

\subsection{Measurements}\label{sec:measurements}

Iridian uses the metric system for most measurements. The cardinal form of the
number is used, followed by the name of the unit. If the unit appears
independently, it is unmarked; if however, the measurement is used
attributively, the unit is marked in the instrumental case: thus one writes
\irdp{hrona měter}{three meters} and \irdp{hrona mětru kuz}{three meters of
silk.} Some commonly used units include \irdp{měter}{meter} for length,
\irdp{gram}{gram} for weight, \irdp{lěter}{liter} for volume, or generic units
like \irdp{procent}{per cent,} \irdp{par}{pair,} \irdp{tuzyn}{dozen} and
\irdp{týdna}{score}.\footnote{The word for \trsl{score} and \trsl{twenty} are
identical, \ird{týdna}. When used as a numeral attributively \ird{týdna} is
indeclinable. When used as a unit, it declines like a regular noun.} SI prefixes
like \ird{kilo-}, \ird{cénti-}, etc. are also commonly used.

\subsection{Date and time}\label{sec:date-time}

Dates are written with the year first, followed by the month, and ultimately by
the date. When written in numerals, the numbers are separated by a full stop.
When spoken or when written in full, the number representing the year is
followed by the word \irdp{hlet}{year}, often in the instrumental case. When
followed by the name of the month, \ird{hlet} is declined in the genitive. When
the date is included, the ordinal form is used, followed by the word
\irdp{ráz}{day,} although the latter may be dropped in casual speech. The
inclusion of the date also requires the name of the month to be in the genitive
case.

\pex
\a
\begingl
    \gla 1992 hletí julí 15. rázu veštašik //
    \glb 1992 year\mk{-gen} july-\Gen{} 15th day-\Ins{} be:born-\Av{}-\Pf{}//
    \glft \trsl{I was born on 5 July 1992.}//
\endgl
\xe

\begin{table}
	\footnotesize\sffamily
	\caption{Months of the year.}
	\medskip
	\begin{tblr}{width=0.7\textwidth,colspec={XXXX}}
		\toprule
		{\sc month} & {\sc iridian} & {\sc month} & {\sc iridian}\\
		\midrule
		January		& jenvár	& July & jul\\
		February	& fevrár 	& August & augošt\\
		March		& merc		& September & seitembár\\
		April		& april 	& October & oktobár\\
		May 		& mai 		& November & novembár\\
		June 		& jón 		& December & dicámbár\\
		\bottomrule
		\label{tab:months}
	\end{tblr}
\end{table}

