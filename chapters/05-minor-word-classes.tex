\chapter{Minor word classes}\label{chap:minor}
\index{minor word classes}

In the preceding two chapters, we have looked at the three major word classes in Iridian: nouns, verbs, and modifiers. In this chapter we will look at the remaining word class, the function words, which we further divide into adverbial particles, conjunctions, prepositions, discourse markers, interjections, and numerals.

\section{Adverbial particles}\label{sec:adv-particles}

\subsection{In general}\label{sec:adv-particles-general}

Adverbial particles are a small class of words that are similar to one another
in exhibiting proclitic behavior. Notwithstanding certain predictable exceptions
discussed in this section, adverbial particles must obligatorily appear before
the predicate they modify. We use the term `adverbial' as the group generally
behaves like adverbs, i.e., they modify verbs and verbal phrases, other adverbs,
or whole sentences, although most Iridian grammars simply refer to them as
\ird{zemo}, literally meaning `grain' or `particle.' They also display functions
not typically associated with adverbs, such as the ability to express
evidentiality or mirativity, among other things. A single particle may also be
used in multiple ways, as we will see in the following sections. The full list
of adverbial particles is given in Table~\ref{tab:adv-particles}.

\begin{table}
	\sffamily\scriptsize
	\caption[Adverbial particles]{Adverbial particles. The linguistic glosses, much like the translations, only provide approximations of the meanings of each particle and may not be fully equivalent to the actual meanings of the linguistic categories listed here.}\label{tab:adv-particles}
	\medskip
	\begin{tblr}{width=0.9\textwidth,colspec={X[0.5]X[1.4]X[0.4]X[1.2]}}

		\toprule
		{\sc particle} &
		{\sc approx. translation} &
		{\sc gloss} &
		{}\\ 
		\midrule 

		že &
		\trsl{already} &
		\Pfv{} &
		Perfective particle \\ 

		po &
		\trsl{still, yet} &
		\Ipfv{} &
		Imperfective particle \\ 

		lí &
		\trsl{whether, if} &
		\Q{} &
		Question particle \\ 

		može &
		\trsl{also, too} &
		\Add{} &
		Additive particle \\ 

		što &
		\trsl{indeed, truly} &
		\Aff{} &
		Affirmative particle \\ 

		kamo &
		\trsl{apparently, according to} &
		\Rep{} &
		Reportative particle \\ 

		sám &
		\trsl{only, just} &
		\Excl{} &
		Exclusive particle \\ 

		pro &
		\trsl{on the other hand} &
		\Cntr{} &
		Contrastive particle \\ 

		dal &
		\trsl{before} &
		\Antess{} &
		Antessive particle \\ 

		izdy &
		\trsl{apparently, may be} &
		\Infer{} &
		Inferential particle \\ 

		hlavdy &
		\trsl{probably, perhaps} &
		\Infer{} &
		Inferential particle \\

		oče &
		\trsl{contrary to my expectations} &
		\Mir{} &
		Mirative particle \\ 

		nadě &
		\trsl{as a result} &
		\Conseq{} &
		Consequential particle \\ 

		děne &
		\trsl{perhaps} &
		\Spec{} &
		Speculative particle\\

		mlada &
		\trsl{if only} &
		\Hyp{} &
		Hypothetical particle \\

		\bottomrule

\end{tblr}
\end{table}

According to  \posscite{zwicky1983cliticization} criteria for clitichood,
clitics generally exhibit a low degree of selection with respect to their host
while affixes exhibit a high degree of selection with respect to their stems.
Adverbial particles can attach to words of virtually any category, as long as
they appear in the predicate position. For example:
\pex
	\a 	\irdp{Marek \textbf{že=}doktor}{Marek has become a doctor.} (noun)
	\a	\irdp{Marek \textbf{lí=}uzdravževí?}{Is Marek already sleeping.} (verb)
	\a 	\irdp{Marek bych \textbf{po=}na duma}{Marek has been home since
		yesterday.} (preposition)
\xe
Compare this with an affix like \ird{u-} used to mark reflexity, which can only
attach to verbs; thus making a construction like \irdp{Marek uzdravževí}{Marek
is sleeping} grammatical but \ird{Marek udoktor} (presumably for \trsl{Marek
himself is a doctor}) not.

As proclitics, adverbial particles must appear directly before the verb they are
associated with. Thus any other arguments a verb may have will normally appear
before any adverbial particle. Consider for example a sentence like  \irdp{To
tóm Jankám kupenik}{This book was bought by Janek.} Any particle that will be
added to modify the sentence will appear after the subject \ird{To tóm} and the
agent \ird{Jankám} but immediately before the verb \ird{kupenik}. Viz.,

\pex
\a
\begingl
	\gla To tóm Jankám lí=kupenik.//
	\glb this book Janek-\Agt{} \Q{}=buy-\Pv{}-\Pf{}//
	\glft \trsl{Was this book bought by Janek?}//
\endgl
\a
\begingl
	\gla To tóm Jankám sám=lí=kupenik.//
	\glb this book Janek-\Agt{} \Excl{}=\Q{}=buy-\Pv{}-\Pf{}//
	\glft \trsl{Was this book the only thing bought by Janek?}//
\endgl
\a
\begingl
	\gla To tóm Jankám oče=izdy=može=što=kupenik.//
	\glb this book Janek-\Agt{} \Mir{}=\Infer{}=\Add{}=\Aff{}=buy-\Pv{}-\Pf{}//
	\glft \trsl{This looks like what Janek has bought (contrary to what we might
	have thought before, though I'm still not sure).}//
\endgl
\a
\begingl
	\glpreamble Moving the particle to any other position in the sentence other
	than immediately before the verb will result in an ungrammatical sentence,
	like the following://
	\gla \ljudge{*}To tóm može=lí=Jankám kupenik?//
	\glb this book \Add{}=\Q{}=Janek-\Agt{} buy-\Pv{}-\Pf{}//
	\glft \trsl{Is this the book Janek also bought?}//
\endgl
\xe

When two or more particles are proclitic to the same predicate, their relative
word order may be described in terms of the following hierarchy, in relation to
their distance from the predicate:
\begin{itemize}[nosep]
	\item Class 1: \ird{že} and \ird{po}
	\item Class 2: \ird{li}, \ird{može}, \ird{što}, \ird{kamo}, \ird{sám},
	\ird{pro} and \ird{dal}
	\item Class 3: \ird{izdy}, \ird{hlavdy}, \ird{děne}, \ird{oče} and
	\ird{nadě}
\end{itemize}

\subsection{Class 1: \ird{že} and \ird{po}}
\label{sec:class1-particles}

The two Class 1 adverbial particles \ird{že} and \ird{po} never occur in
immediate sequence to each other. In general, \ird{že} and \ird{po} carry
aspect-related meanings, with \ird{že} used to indicate the perfective and
\ird{po} the imperfective aspect. This usage, however, does not completely
correspond to the true aspectual suffixes on a verb, as we have seen in
\S~\ref{sec:aspect}. \ird{Že} and \ird{po} can be broadly translated as
\trsl{already} and \trsl{still/yet}, respectively, but their usage as we will
see below is more complex.

In sentences containing a temporal clause expressing a future event, \ird{že}
and \ird{po} are used to indicate the attitude of the speaker towards the time
described in the predicate. \ird{Že} `extends' the perceived time between the
reference point and the time described by the predicate, while \ird{po}
`shortens' it. Thus a neutral sentence such as \irdp{Janek sobotu
ščenžách}{Janek will arrive on Saturday} can be modified as \ird{Janek sobotu že
ščenžách} or as \ird{Janek sobotu po ščenžách.} The former indicates that the
speaker thinks that there is little time left before Janek's arrival on
Saturday, while the latter indicates that the speaker thinks that there is still
a lot of time left before Janek's arrival on Saturday. In contrast to these two,
the original sentence without \ird{že} or \ird{po} does not pass any judgment on
the time left before Janek's arrival on Saturday. With temporal clauses
expressing past events, \ird{po} behaves the same way in `extending' the
perceived time between the reference point and the time described by the
predicate; \ird{že} on the other hand cannot be used in this way. Thus
\irdp{Janek sobotu ščenžek}{Janek arrived on Saturday} can be modified as
\ird{Janek sobotu po ščenžek} which can be interpreted as \trsl{Janek arrived on
Saturday (and it has been quite some time since then).} \ird{Janek sobotu že
ščenžek} is also a valid sentence, but here \ird{že} merely translates as
\trsl{already} and does not have the aspectual connotation it has in the future
tense.

The usage of \ird{že} and \ird{po} described in the previous paragraph is
limited to sentences that satisfy two criteria: (1) the sentence must contain an
explicit temporal clause that specifies the point in time when the action
described in the predicate will take place or has taken place, and (2) the
action or the state must be in the future or past, which means if the predicate
is a verb it must be in the perfective, retrospective or contemplative aspect.
These criteria are required since the length of time upon which the speaker
passes judgment can only be established by first defining a reference point
(i.e., the time of speaking) and another point in time that will serve as the
beginning (in the case of past events) or the end (in the case of future events)
of the time interval. In sentences that do not satisfy these criteria, \ird{že}
and \ird{po} will have different meanings.

Without an explicit temporal clause, \ird{že} is used with a verb in the
perfective aspect to indicate that the action has been completed at some
unspecified time prior to another time. This usage roughly corresponds to the
English present perfect. Compare for example the two sentences below:

\pex
\a
\begingl
	\gla Janek ščenžek.//
	\glb Janek arrive-\Av{}-\Pf{}//
	\glft \trsl{Janek arrived.}//
\endgl
\a
\begingl
	\gla Janek že ščenžek.//
	\glb Janek \Pfv{} arrive-\Av{}-\Pf{}//
	\glft \trsl{Janek has (already) arrived (at some unspecified time in the past).}//
\endgl
\xe

\ird{Že} can alternatively be used with a verb in the retrospective aspect to
indicate that the action has been completed at some unspecified time prior to
another event, as with the perfective, but this usage requires that the
secondary verb that frames the action described by the main verb also be present
in the sentence. This usage can correspond to the English past perfect or the
future perfect depending on the aspect of the verb in the preceding clause
(i.e., before \ird{še}). E.g.,

\pex
\a
\begingl
	\gla Marek ščenžek še Janek že piaščaní.//
	\glb Marek arrive-\Av{}-\Pf{} with Janek \Pfv{} eat-\Av{}-\Ret{}//
	\glft \trsl{Janek had already eaten when Marek arrived.}//
\endgl
\a
\begingl
	\gla Marek ščenžách še Janek že piaščaní.//
	\glb Marek arrive-\Av{}-\Ctp{} with Janek \Pfv{} eat-\Av{}-\Ret{}//
	\glft \trsl{Janek will have already eaten when Marek arrives.}//
\endgl
\xe

With negated verbs, \ird{že} can be translated as \trsl{no more,} \trsl{not any more,} \trsl{no longer,} etc., while \ird{po} can be translated as \trsl{not yet,} \trsl{at all,} etc. For example,

\pex
\a
\begingl
	\glpreamble with a verbal predicate://
	\gla Janek Prahu že(\#po) zámožlaševí.//
	\glb Janek Prague-\Ins{} \Pfv{}(\#\Ipfv{}) \Neg{}-live-\Av{}-\Cont{}//
	\glft \trsl{Janek is no longer (not yet) living in Prague.}//
\endgl
\a
\begingl
	\glpreamble with an existential construction://
	\gla Na duma že(\#po) niho trava.//
	\glb \Loc{} house-\Acc{} \Pfv{}(\#\Ipfv{}) \N{}\Exst{} bread//
	\glft \trsl{There is no more (not yet any) bread in the house.}//
\endgl
\a
\begingl
	\glpreamble with a copular construction://
	\gla Marek že(po) zám doktor.//
	\glb Marek \Pfv{}(\#\Ipfv{}) \N{}\Cop{} doctor//
	\glft \trsl{Marek is no longer (not yet) a doctor.}//
\endgl
\xe

When used with the perfective, \ird{po} indicates that an action has been done
together with other actions, with the implication that the speaker regrets or is
annoyed by having to perform the action. The implication of annoyance can be
tempered by using \ird{što}. When \ird{po} is used to imply regret or
commiseration, \ird{što} can also be used although it's meaning becomes mainly
emphatic and no longer has the tempering effect it has when \ird{po} is used to
express annoyance. This is comparable to the use of \trsl{even} in English:

\pex
\a
\begingl
	\glpreamble \lingcontext{The speaker is annoyed because he went out of his
	way yesterday to get his friend something he had asked him to get at the
	store, only to find out now that he no longer needs it.}//
	\gla Do magazina bych po štožek!//
	\glb into store-\Acc{} yesterday \Ipfv{} go-\Av{}-\Pf{}//
	\glft \trsl{I even went to the store yesterday!}//
\endgl
\a
\begingl
	\glpreamble \lingcontext{The speaker is telling a story about the places he
	went to yesterday and mentions going to the store, which might not have been
	expected by the listener. In this example, the speaker might not necessarily
	be annoyed.}//
	\gla Do magazina bych što po štožek.//
	\glb into store-\Acc{} yesterday \Aff{} \Ipfv{} go-\Av{}-\Pf{}//
	\glft \trsl{I even went to the store yesterday.}//
\endgl
\a
\begingl
	\glpreamble \lingcontext{The speaker is telling the listener a story about
	how their mutual friend Marek scoured the whole city looking for his lost
	dog. They live in the Bretsko neighborhood, but Marek went all the way to
	the Štěpánou neighborhood on the other side of the city to look for the dog.
	 The speaker is commiserating with Marek about the trouble he went through.
	 Here \ird{po} is used to express commiseration, not annoyance. \ird{Što}
	 may be dropped from the sentence and the meaning will remain the same,
	 albeit becoming less emphatic.}//
	\gla Marek Štěpánovím što po štožek.// 
	\glb Marek Štěpánov-\Ins{} \Aff{} \Ipfv{} go-\Av{}-\Pf{}//
	\glft \trsl{Marek even went to Štěpánov.}//
\endgl
\xe

\ird{Po} may also be used to express the meaning \trsl{in addition.} In
questions, \ird{po} corresponds to the English \trsl{else}:

\pex
\begingl
	\gla Jede Marcí dumu po ščenžek?//
	\glb who Marek-\Gen{} house-\Ins{} \Ipfv{} arrive-\Av{}-\Pf{}?//
	\glft \trsl{Who else arrived at Marek's house?}//
\endgl
\xe

With non-verbal predicates, \ird{po} causes a sentence to take a present perfect
meaning if the sentence contains a past temporal clause. This is an extension of
the general behavior of \ird{po} with temporal clauses discussed earlier in this
chapter.

\pex
\a
\begingl
	\gla Janka bych ješ druh.//
	\glb Janka-\Acc{} yesterday \Exst{} sickness//
	\glft \trsl{Janka was sick yesterday.}//
\endgl
\a
\begingl
	\gla Janka bych po ješ druh.//
	\glb Janka-\Acc{} yesterday \Ipfv{} \Exst{} sickness//
	\glft \trsl{Janka has been sick since yesterday.}//
\endgl
\xe

Without the temporal clause (\ird{bych} in the case of the above examples),
\ird{po} simply means \trsl{still} as seen in (\ref{ex:po-still}). To maintain
this meaning even in the presence of a temporal clause, the temporal clause is
topicalized, as seen in (\ref{ex:po-still-top}).
\pex
\a\label{ex:po-still}
\begingl
	\gla Janka po ješ druh.//
	\glb Janka-\Acc{} \Ipfv{} \Exst{} sickness//
	\glft \trsl{Janka is still sick.}//
\endgl
\a\label{ex:po-still-top}
\begingl
	\gla Bych-te Janka po ješ druh.//
	\glb yesterday-\Foc{} Janka-\Acc{} \Ipfv{} \Exst{} sickness//
	\glft \trsl{Janka was still sick yesterday.}//
\endgl
\xe

Usage of \ird{po} with a non-past temporal clause is generally infelicitous as
seen in (\ref{ex:po-future}). However it can be made felicitous by the addition
of the additive particle \irdp{može}{also} as we see in the example sentences in
(\ref{ex:po-future-moze}).

\pex\label{ex:po-future}
\begingl
	\gla \ljudge{\#}Janka prohlé po ješ druh.//
	\glb Janka-\Acc{} tomorrow \Ipfv{} \Exst{} sickness//
	\glft \trsl{Janka has been sick since tomorrow.}//
\endgl
\xe
\pex\label{ex:po-future-moze}
\a
\begingl
	\gla Janka prohlé može po ješ druh.//
	\glb Janka-\Acc{} tomorrow \Add{} \Ipfv{} \Exst{} sickness//
	\glft \trsl{Janka will still be sick tomorrow.}//
\endgl
\a
\begingl
	\gla Marek vedru može po na duma.//
	\glb Marek-\Acc{} Monday-\Ins{} \Add{} \Ipfv{} \Loc{} house-\Acc{}//
	\glft \trsl{Marek will still be at home on Monday.}//
\endgl
\xe

Both \ird{po} and \ird{že} are required in yes-no questions where the focus of
the question is the completion of an action. Consistent with the use of both
particles with temporal clauses, the use of \ird{že} or \ird{po} usually carries
the speaker's judgment about the length of time the action is performed or the
proximity of completion of the action to now. Some examples of how \ird{že} and
\ird{po} change the meaning yes-no questions are given in Table
\ref{tab:ze-po-questions}.

\begin{table}
\footnotesize\sffamily
\caption{Usage of \ird{že} and \ird{po} in yes-no questions.}
\label{tab:ze-po-questions}
\medskip
\begin{tblr}{width=\textwidth, colspec={X[0.6]XX}}
\toprule
{\sc aspect}  & že & po \\
\midrule

Perfective &
\ird{Janek lí že piašček?} &
\ird{\ldots po piašček?}\\

&
\trsl{Has Janek already eaten?} &
\trsl{Did Janek still manage to eat?}\\

Progressive &
\ird{\ldots že piaščime?} &
\ird{\ldots po piaščime?}\\

&
\trsl{Is Janek already eating?} &
\trsl{Is Janek still eating?}\\

Contemplative &
\ird{\ldots že piaščách?} &
\ird{\ldots po piaščách?}\\

&
\trsl{Is Janek about to eat?} &
\trsl{Will Janek still be eating?}\\

\bottomrule
\end{tblr}
\end{table}

\subsection{Class 2 particles}\label{sec:class2-particles}

\paragraph{može} \ird{Može} roughly corresponds to the English \trsl{also} or
\trsl{too.} It is commonly used to express similarity. If the main verb is in
the negative, \ird{može} can be translated as \trsl{either} or \trsl{neither.}
For example:

\pex
\begingl
	\gla Janek može uzdravževí.//
	\glb Janek \Add{} sleep-\Av{}-\Cont{}//
	\glft \trsl{Janek is also sleeping.}//
\endgl
\xe

\pex
\begingl
	\gla Janek može záščenžek.//
	\glb Janek \Add{} \Neg{}-arrive-\Av{}-\Pf{}//
	\glft \trsl{Janek hasn't arrived either.}//
\endgl
\xe

\ird{Može} can also be used idiomatically to mean \trsl{finally} or \trsl{at
last} with a perfective or a retrospective verb, with or without the particle
\ird{že}. With \ird{po} and a negative verb, \ird{može} usually functions as an
intensifier. 

\pex
\begingl
	\gla Já može že vednik!//
	\glb \Second{}.\Sg{} \Add{} \Ipfv{} see-\Pv{}-\Pf{}//
	\glft \trsl{I've seen you at last!}//
\endgl
\xe

\pex
\begingl
	\gla Janek može po záščenžek.//
	\glb Janek \Add{} \Ipfv{} \Neg{}-arrive-\Av{}-\Pf{}//
	\glft \trsl{Janek hasn't arrived yet (we've been waiting for him for a long
	time).}//
\endgl
\xe


\paragraph{što} \ird{Što} is used to express emphasis, affirmation or
confirmation. For example:

\pex
\begingl
	\glpreamble \lingcontext{A tells B that Janek has arrived. However, B knows
	that Janek is on a trip to Talinn and will not be back until the next day
	and so does not believe A. B then sees Janek himself and confirms that he
	has indeed arrived. B reports back to A, saying:}//
	\gla Janek što že ščenžek.//
	\glb Janek \Aff{} \Pfv{} arrive-\Av{}-\Pf{}//
	\glft \trsl{Janek has indeed arrived.}//
\endgl
\xe

\ird{Što} can also be used in imperative (both the bare-root imperative and the
conjugated imperative forms) and hortative constructions to soften a command.
When used with a bare-root imperative, \ird{što} makes the presence of the
personal pronouns \irdp{jí}{you (acc.sing.)}, \irdp{tě}{you (acc. sing.)} or
\irdp{mě}{us} mandatory. Thus one writes \ird{Piašt!} \trsl{Eat!} and \ird{Mě
što piašt!} \trsl{Let's eat!} or \trsl{I hope we can eat!} but not \ird{*Što
piašt!} With hortative constructions \ird{što} has the implication of asking the
listener's consent in performing the action described by the request. For
example:

\pex
\begingl
	\gla Jancí švirknikou što zoštné.//
	\glb Janek-\Gen{} write-\Pv{}-\Pf{}-\Nz{} \Aff{} read-\Pv{}-\Hort{}//
	\glft \trsl{Let's read what Janek wrote.} Literally: \trsl{I hope we can
	read what Janek has written.}//
\endgl
\xe

\paragraph{sám} \ird{Sám} generally corresponds to the English \trsl{only.}
This particle is most likely of Slavic origin, as evidenced by cognates like
\foreign{samo} in Czech (meaning \trsl{alone}) or in Croatian (meaning
\trsl{only}) or \foreign{sam} in Polish.

\pex
\begingl
	\gla Janek sám lí Roubžu možlaševí?//
	\glb Janek \Excl{} \Q{} Roubže-\Ins{} live-\Av{}-\Cont{}//
	\glft \trsl{Is Janek the only one who lives in Roubže?}//
\endgl
\xe

\paragraph{kamo} \ird{Kamo} is used for indirect quotations and other
reportative or evidential constructions. The use of \ird{kamo} is discussed in
detail in \S\,\ref{sec:reportedspeech}.

\subsection{Class 3 particles}\label{sec:class3-particles}

\paragraph{izdy, hlavdy, děne} \ird{Izdy} and \ird{hlavdy} are used to express
that the speaker has deduced the information expressed in a sentence from some
piece of evidence but is not sure whether or not it is true. They have
essentially the same function, except that \ird{hlavdy} usually denotes that
the speaker has more confidence on the inference than what is conveyed by
\ird{izdy}. Consider the following examples, where either \ird{izdy} or
\ird{hlavdy} would be appropriate, and the choice of which one to use would not
greatly alter the meaning of the sentences.

\pex 
\begingl
	\glpreamble \lingcontext{The speaker saw Janek's car parked in front of the
	house. He infers that Janek must have already come back but is not sure
	since he has not yet seen Janek to confirm his hypothesis.}//
	\gla Janek izdy/hlavdy že ščenžek.//
	\glb Janek \Infer{} \Pfv{} arrive-\Av{}-\Pf{}//
	\glft \trsl{Janek may have already arrived.}//
\endgl
\xe

\pex
\begingl
	\glpreamble \lingcontext{The speaker asks their co-worker Marek to join
	their group on a trip to a ski resort. Marek has not responded yet but the
	speaker infers that he must be going since he has already asked his manager
	for some time off.}//
	\gla Marek na potva izdy/hlavdy svěžách.//
	\glb Marek \Loc{} trip-\Acc{} \Infer{} join-\Av{}-\Ctp{}//
	\glft \trsl{It looks like Marek is going on the trip.}//
\endgl
\xe

The use of \ird{izdy} and \ird{hlavdy} requires two conditions: (1) the
utterance must be directly based on some truth already known to the speaker and
(2) the speaker must be unsure whether or not the utterance itself is true. This
makes the usage of \ird{izdy} and \ird{hlavdy} inferential in nature instead of
merely expressing some doubt or uncertainty. The precondition upon which the
inference is based does not necessarily have to be explicitly stated in the
conversation as the listener and speaker's shared context should by itself
suffice.

Even though \ird{hlavdy} is used to express a higher degree of confidence than
\ird{izdy}, it still presumes some level of uncertainty in an utterance. Thus
the example below, though strictly grammatical in isolation, would be
infelicitous:

\pex
\begingl
	\glpreamble \lingcontext{The speaker asks their co-worker Marek to join
	their group on a trip to a ski resort. Marek then confirms that he will be
	coming.}//
	\gla \ljudge{\#}Marek na potva hlavdy svěžách.//
	\glb Marek \Loc{} trip-\Acc{} \Infer{} join-\Av{}-\Ctp{}//
	\glft \trsl{It looks like Marek is going on the trip.}//
\endgl
\xe

Interestingly, \ird{izdy} and \ird{hlavdy} can be used together in the same
sentence. The resulting sentence is roughly equivalent to using \ird{izdy} alone
in terms of the confidence the speaker has on the inference, or in some cases,
half-way between the confidence conveyed by \ird{izdy} and \ird{hlavdy}. When
used together, \ird{izdy} always precedes \ird{hlavdy}.

\pex
\a
\begingl
	\glpreamble \lingcontext{The speaker has invited their friend Janek to
	watch a movie with them. Janek asks the speaker what movie they are going
	to watch. They mention a sci-fi film by a director they know Janek is a big
	fan of. They then add the following observation after sharing the name of
	the film.}//
	\gla Má to izdy hlavdy že vednaní.//
	\glb but this \Infer{} \Infer{} \Pfv{} see-\Pv{}-\Ret{}//
	\glft \trsl{But you've probably already seen it.}//
\endgl
\a
\begingl
	\gla \ljudge{*}Má to hlavdy izdy že vednaní.//
	\glb but this \Infer{} \Infer{} \Pfv{} see-\Pv{}-\Ret{}//
	\glft \trsl{But you've probably already seen it.}//
\endgl
\xe

\ird{Děne}, on the other hand, is used to express speculation. This particle is
most commonly used in questions to ask for the opinion of the listener. In
information questions (i.e., \emph{wh}-questions), \ird{děne} must appear with
\ird{lí}.

\pex
\a
\begingl
	\gla Jede to tóma děne lí kupšek?//
	\glb who this book-\Acc{} \Spec{} \Q{} buy-\Av{}-\Pf{}//
	\glft \trsl{Who else do you think bought this book?}//
\endgl
\a
\begingl
	\glpreamble{Compare this with:}//
	\gla Janek to tóma děne kupšek?//
	\glb Janek this book-\Acc{} \Spec{} buy-\Av{}-\Pf{}//
	\glft \trsl{Do you suppose it was Janek who bought this book?}//
\endgl
\xe

\ird{Izdy} and \ird{hlavdy} are in complementary distribution with \ird{děne}.
\ird{Izdy} and \ird{hlavdy} can only be used in declarative sentences and are
ungrammatical in interrogative and imperative sentences; \ird{děne}, meanwhile,
is allowed only in the latter and disallowed in the former. While \ird{izdy},
\ird{hlavdy} and \ird{děne} are  allused to express uncertainty, \ird{izdy} and
\ird{hlavdy} also implies that the speaker has some evidence to support the
inference, while \ird{děne} does not. This leads to a proposition anchored by
\ird{izdy} or \ird{hlavdy} to tend to be more `assertive' in nature than an
equivalent proposition anchored by \ird{děne} which passes the burden of
confirmation to the listener, thus it's incompatibility with declarative
sentences.

\pex
\a \begingl
	\gla \ljudge{*}Janek izdy lí že ščenžek?//
	\glb Janek \Infer{} \Q{} \Pfv{} arrive-\Av{}-\Pf{}//
	\glft \trsl{Has Janek already arrived?}//
\endgl
\a \begingl
	\gla Janek děne lí že ščenžek?//
	\glb Janek \Spec{} \Q{} \Pfv{} arrive-\Av{}-\Pf{}//
	\glft \trsl{I wonder if Janek has already arrived?}//
\endgl
\xe


\paragraph{oče} \ird{Oče} is a particle used to express
mirativity\index{mirativity}, a grammatical category first proposed by
\textcite{delancey1997mirativity} which is used to express the speaker's
surprise or unpreparedness with regards to the information presented.
\textcite{aikhenvald2012essence} summarizes the mirative category by listing the
following conceptions of mirativity: (1) new information, (2) sudden discovery,
revelation or realization, (3) surprise, (4) counterexpectation and (5)
unprepared mind.

\pex
\begingl
	\glpreamble \lingcontext{The speaker is surprised to learn that Janek has
	left as he was expecting him to still be there (counterexpectation).}//
	\gla Janek oče že varžek.//
	\glb Janek \Mir{} \Pfv{} leave-\Av{}-\Pf{}//
	\glft \trsl{Janek has already left.}//
\endgl
\xe

\pex
\begingl
	\glpreamble \lingcontext{The speaker knew it was raining, but it has slipped
	his mind. They suddenly remember and say:}//
	\gla Oče što pozebževí.//
	\glb \Mir{} \Aff{} rain-\Av{}-\Cont{}//
	\glft \trsl{Oh, it's raining.}//
\endgl
\trailingcitation{(cf. \cite[2]{anderbois2022pala})}%
\xe

\paragraph{pro} \ird{Pro} is used to express contrast and can often be
translated in English as \trsl{on the other hand}, \trsl{but}, \trsl{however},
\trsl{instead} or \trsl{also.}

\pex\label{ex:contrastive-part-1}
\begingl
	\gla Janek uzdravževí. Marek pro znohouščeví.//
	\glb Janek \Refl{}-sleep-\Av{}-\Cont{} Marek \Cntr{} study-\Av{}-\Cont{}//
	\glft \trsl{Janek is sleeping. Marek, on the other hand, is studying.}//
\endgl
\xe

\pex\label{ex:contrastive-part-2}
\begingl
	\glpreamble \lingcontext{The speaker is talking to their doctor who tells
	them that they should start eating healthier. They respond as follows:}//
	\gla Dá grazy pro piaščeví.//
	\glb \First.\Sg{} vegetables-\Gen{} \Cntr{} eat-\Av{}-\Cont{}//
	\glft \trsl{(But) I do eat vegetables.}//
\endgl
\xe

In (\ref{ex:contrastive-part-1}), the particle \ird{pro} is used to contrast
between the actions performed by Janek and Marek.\footnote{The use of \ird{pro}
in (\ref{ex:contrastive-part-1}) presume some level of disjunction or semantic
opposition between the two actions. As such, the use of \ird{pro} in the example
below would not be grammatical:
\ex[lingstyle=fnex,belowexskip=-1em,aboveglftskip=1pt]
\begingl
	\gla \ljudge{*}Janek uzdravževí. Marek pro uzdravževí.//
	\glb Janek \Refl{}-sleep-\Av{}-\Cont{} Marek \Cntr{} sleep-\Av{}-\Cont{}//
	\glft \trsl{Janek is sleeping. Marek, on the other hand, is sleeping.}//
\endgl
\xe
} In (\ref{ex:contrastive-part-2}), the particle \ird{pro} is used to contrast
between the speaker's current eating habits and the doctor's advice, i.e., the
speaker is saying that they do eat vegetables contrary to what the doctor might
assume. This latter example would normally be translated in English with a
conjunction like \trsl{but} and the emphatic \trsl{do} as in \trsl{I \emph{do}
eat}. The corresponding conjunction \ird{má} can also be used in
(\ref{ex:contrastive-part-2}) resulting to the more forceful \irdp{Má dá grazy
pro piaščeví}{But I do eat vegetables}; however, \ird{pro} alone should be
sufficient to express the same meaning.

The corrective nature of \ird{pro} in (\ref{ex:contrastive-part-2}) in which the
speaker contrasts the truth or their opinion with the listener's opinion or what
the listener might assume is true interestingly extends to sentences where the
speaker in fact agrees with the listener but frames the sentence in a way that
makes it seem like they are `correcting' the listener's assumption that they do
not, as in the following example:

\pex
\begingl
	\glpreamble \lingcontext{The speaker is asked by their friend whether the
	friend can join them on their trip to the mountains. The speaker then
	responds:}//
	\gla (Da) pro svěžalnách.//
	\glb (yes) \Cntr{} join-\Av{}-\Pot{}-\Ctp{}//
	\glft \trsl{Yes, of course, you can join us.}//
\endgl
\xe

In the above example, the question might have been framed by the friend, as a
matter of politeness, in a way that emphasizes their uncertainty on whether they
can join or in a way that suggests they think they would not be able or allowed
to join. The use of \ird{pro} in the speaker's answer then serves to correct the
friend's assumption that they cannot join. This would still be the case even if
the initial question was not framed negatively, with \ird{pro} still providing
contrast by implying that the speaker's answer is an obvious one (much like the
use of \trsl{of course} in the English translation provided in the example).

The use of \ird{pro} to indicate obviousness is not limited to questions and
does not necessarily require a direct negative or positive reference with which
the utterance is contrasted. For example, the following sentence is grammatical:

\pex
\begingl
	\glpreamble \lingcontext{A tells B that she likes chocolates. B cheekily
	responds as follows, implying that A's liking for chocolates is nothing to
	be too excited about:}//
	\gla Nět šokoládám pro novítébeví.//
	\glb everyone chocolate-\Agt{} \Cntr{} like-\Ben{}-\Cont{}//
	\glft \trsl{Everyone likes chocolate (duh).}//
\endgl
\xe

When used with imperatives or hortatives \ird{pro} indicates politeness
compounded with mild annoyance or reproach, as seen in the following examples:

\pex
\begingl
	\glpreamble \lingcontext{A and B are in a library and B has been talking
	loudly for quite some time. Irritated, A tells B to be quiet.}//
	\gla Pro okouzí!//
	\glb \Cntr{} be:noisy-\N{}\Imp{}//
	\glft \trsl{Please don't be too noisy!}//
\endgl
\xe

\pex
\begingl
	\glpreamble \lingcontext{A slipped while walking on the ice and fell. B is
	nearby but does nothing. A then tells B to help them up instead of just
	standing there and watching.}//
	\gla Dá pro stranébé!//
	\glb \First.\Sg{} \Cntr{} help-\Ben{}-\Hort{}//
	\glft \trsl{Come on, help me up!} (Literally: \trsl{May I be helped up!})//
\endgl
\xe

We can analyze the consistency of this behavior with the other uses of \ird{pro}
by first interpreting the command or request as being contrasted with the action
that is currently being performed by the listener. Here, the speaker is
basically implying that it is `obvious' that the action should be performed,
despite it not being so at the moment the command or request was uttered. The
speaker's annoyance or reproach is thus a logical consequence of the listener's
failure to perform the action that is `obvious' to both the speaker and the
listener. The polite nature of this usage stems from the fact that the speaker
is not directly commanding or requesting the listener to perform the action, but
rather referencing the course of action as being `obvious' to both the speaker
and the listener in their shared context.

\paragraph{mlada} \ird{Mlada} is used to express wishes or hypothetical situations.

\subsection{Order of particles}\label{sec:order-of-particles}

As noted in \S~\ref{sec:adv-particles-general} above, the order of adverbial
particles is generally determined by their class, with Class 1 particles
appearing nearest to the predicate and Class 3 particles appearing farthest.
Thus the order of various particles in a sentence like example
(\ref{ex:multiple-particles-diff-class}) where each particle belongs to a
separate class is trivial:

\pex\label{ex:multiple-particles-diff-class}
\begingl
	\gla Janek oče može po uzdravževí.//
	\glb Janek \Mir{} \Add{} \Ipfv{} \Refl{}-sleep-\Av{}-\Cont{}//
	\glft \trsl{To my surprise, Janek is also still asleep.}//
\endgl
\xe

In the above example, the Class 1 particle \ird{po} appears nearest to the
predicate, followed by the Class 2 particle \ird{može}, and finally the Class 3
particle \ird{oče}. The order of particles in Iridian is strict and thus
sentences like the following are not grammatical:

\pex
\a \ljudge{*} \ird{Janek može oče po uzdravževí.}
\a \ljudge{*} \ird{Janek može po oče uzdravževí.}
\a \ljudge{*} \ird{Janek po može oče uzdravževí.}
\a \ljudge{*} \ird{Janek po oče može uzdravževí.}
\a \ljudge{*} \ird{Janek oče po može uzdravževí.}
\xe

The order of particles in sentences where two or more particles belong to the
same class is more complicated. The Class 1 particles \ird{že} and \ird{po}
cannot appear simultaneously in the same sentence, as they carry opposite,
temporal meanings. Thus the position occupied by Class 1 particles can only be
filled by one and only one particle or by none at all. Class 2 particles, on the
other hand, can appear simultaneously in the same sentence; in general, the
order they appear follows the formula below:

\pex\label{ex:order-of-class-2-parts}
	{\small
		kamo > može > sám > dal > pro > lí > što
	}
\xe

Finally, Class 3 particles appear farthest from the predicate; they may occur in
any order relative to one another. For example, sentences like
(\ref{ex:oce-iz-gramm}) and (\ref{ex:iz-oce-gramm}) are grammatical, but
(\ref{ex:oce-iz-ungramm}) is not. There is no difference in meaning between
(\ref{ex:oce-iz-gramm}) and (\ref{ex:iz-oce-gramm}).

\pex
\a\label{ex:oce-iz-gramm}
\begingl
	\gla Janek oče izdy že ščenžek.//
	\glb Janek \Mir{} \Infer{} \Pfv{} arrive-\Av{}-\Pf{}//
	\glft \trsl{Janek may have already arrived apparently.}//
\endgl
\a\label{ex:iz-oce-gramm}
\begingl
	\gla Janek izdy oče že ščenžek.//
	\glb Janek \Infer{} \Mir{} \Pfv{} arrive-\Av{}-\Pf{}//
	\glft \trsl{Janek may have already arrived apparently.}//
\endgl
\a\label{ex:oce-iz-ungramm}
\begingl
	\gla \ljudge{*}Janek že oče izdy ščenžek.//
	\glb Janek \Pfv{} \Mir{} \Infer{} arrive-\Av{}-\Pf{}//
	\glft \trsl{Janek may have already arrived apparently.}//
\endgl
\xe

While the ordering of adverbial particles may at first seem arbitrary, it can be
observed that the order has phonological motivations. All adverbial particles in
Iridian are either have one or two syllables, and we can observe that the
monosyllabic particles appear closer to the host than the disyllabic particles.
This is true even in the case of the Class 2 particles, which consist of a mix
of monosyllabic and disyllabic particles, as we see in
\ref{ex:order-of-class-2-parts}, where \ird{kamo} and \ird{može} appear farther
from the host than the rest of the class. We can thus generalize the ordering of
adverbial particles as follows:

\pex
	{\small
		Class 3 > Class 2 2\sigma{} > Class 1 2\sigma{} > Class 1 > [{\sc host}]
	}
\xe

\section{Movable adverbs}
\label{sec:movable-adverbs}

Movable adverbs, like adverbial particles, are used to modify they modify verbs
and verbal phrases, other adverbs, or whole sentences. Unlike adverbial
particles, however, movable adverbs do not need to be cliticized to the
predicate and may appear in any position in the sentence, subject to the usual
restrictions on word order. For example, the sentences below featuring the
adverb \irdp{bych}{yesterday} are all grammatical:

\pex
\a
	\begingl
	\gla Janek bych Markám magazinu vednik.//
	\glb Janek yesterday Marek-\Agt{} store-\Ins{} see-\Pv{}-\Pf{}//
	\glft \trsl{Marek saw Janek at the store yesterday.}//
	\endgl
\a \ird{Bych Janek Markám magazinu vednik.}
\a \ird{Janek Markám bych magazinu vednik.}
\a \ird{Janek Markám magazinu bych vednik.}
\xe

Movable adverbs can be classified into two main groups: (1) derived adverbs,
which are usually formed with the instrumental case suffix \ird{-u} and (2)
\emph{true} adverbs which are standalone forms that do not require any
additional morphemes. In the examples above, \ird{magazinu}{at the store} is an
example of a derived adverb, while \ird{bych}{yesterday} is a true adverb. Both
groups are movable in the sense that they can appear in any position in the
sentence, but Iridian grammar only classifies the latter as `movable adverbs'.
Nevertheless we will discuss both groups in this section. Iridian further
classifies the second group into: (1) locative adverbs, discussed in
\S\,\ref{sec:locative-adverbs}, (2) temporal adverbs, discussed in
\S\,\ref{sec:temporal-adverbs}, and (3) adverbs of manner, discussed in
\S\,\ref{sec:adverbs-of-manner}.

\subsection{Locative adverbs}
\label{sec:locative-adverbs}

\subsection{Temporal adverbs}
\label{sec:temporal-adverbs}

\subsection{Adverbs of manner}
\label{sec:adverbs-of-manner}

\subsection{Derived adverbs}
\label{sec:derived-adverbs}

The instrumental case ending \ird{-u} or its variant \ird{-óvím} (see
\S\,\ref{sec:declension-patterns}) is productive in deriving adverbs or
adverbial phrases, e.g., \irdp{sobotu}{on Saturday}, \irdp{vtaru}{in the
morning}, \irdp{dumu}{at home}, etc. Compound forms (with the noun in the
instrumental form or otherwise) are also possible with the use of prepositions
such as \irdp{na}{in}, \irdp{za}{for the benefit of}, \irdp{do}{into},
\irdp{o}{about} or modifiers, as in \irdp{nesté duhu}{last month} or \irdp{Marcí
dumu}{at Marek's house}. These compound forms have a fixed word order internally
(i.e., \ird{nesté duhu} will never appear as \ird{duhu nesté}) but at the
sentence level they behave as any other movable adverb.

Note, however, that these derived forms may also be used as noun or noun phrase
modifiers, in which case they are more properly treated as modifiers and not as
movable adverbs. As has been discussed in
\S\,\ref{sec:internal-clause-structure} and elsewhere, modifiers must
obligatorily appear before the noun or noun phrase they modify, and thus do not
enjoy the same freedom of placement as movable adverbs. Consider, for example,
the following sentences which both contain the phrase \ird{za Marka}{for (the
benefit of) Marek}. In the first sentence, \ird{za Marka} is used as a derived
adverb, and thus can appear in any position in the sentence. In the second
sentence, however, \irdp{za Marka} is used as a modifier, and thus cannot be
moved from its position before the noun \ird{houba}, which it modifies.

\pex
	\a\begingl
	\gla Avt za Marka Jankám houbu dítnek.//
	\glb car for Marek Janek-\Agt{} gift-\Ins{} give-\Pv{}-\Pf{}//
	\glft \trsl{Marek gave Janek a car as a gift.}//
	\endgl
	\a\begingl 
	\gla Za Marka houba polnek.//
	\glb for Marek-\Acc{} gift lose-\Pv{}-\Pf{}//
	\glft \trsl{Marek's gift was lost.}//
	\endgl
\xe

\section{Conjunctions}
\label{sec:conj}

\subsection{Connective conjunctions}\label{sec:conn-conj}

Sentences of the type

\ex
It is [\mk{adjective}] that[ \mk{subordinate clause}].
\xe

are normally translated in Iridian using an expletive-\ird{a} construction, with the adjective in the attributive form at the start of the phrase, followed by \ird{a}, and then by the rest of the main clause. Normally this construction is used for sentences that pass judgment to the action or state described in the main clause, although in some cases the adjective is simply used for descrciption.

\pex
\begingl
    \gla Interezní a téknik znohouštnilá te prádelnik.//
    \glb interesting-\Att{} and engineering study\mk{-pv-sbj.ipf} \mk{rz} choose-\Pv{}-\Pf{}//
    \glft \trsl{It is interesting that you chose to study engineering.}//
\endgl
\xe
\pex
\begingl
    \gla Komí a já ščenžek.//
    \glb good-\Att{} and \Second{}\Sg{} arrive-\Av{}-\Pf{}//
    \glft \trsl{Good you're here now!}//
\endgl
\xe

Another common use of the expletive \ird{a} is with the word \irdp{shlac}{now} (pronounced [sxlat] instead of the more intuitive [sxlat͡s]) to form the phrase \ird{shlac a}\footnote{This is therefore pronounced [ˈsxlatɐ].}, which is used to introduce a subordinate clause, similar to \trsl{now that} in English.

\pex
\begingl
    \gla Shlac a provísor ščenžek, kurs šelčinách.//
    \glb now and professor arrive-\Av{}-\Pf{} class begin-\mk{pv-ctpv}//
    \glft \trsl{Now that the professor is here, we will begin our class.}//
\endgl
\xe


\section{Prepositions}

\subsection{na}

Iridian has a single locative preposition, \ird{na}, which is used to indicate the location of an object or person. It is used in the same way as the English \trsl{on} or \trsl{in}. \ird{Na} is followed by a noun or a noun phrase in the accusative case.

\pex
\a \ird{na duma}, \trsl{at home}
\a \ird{na škole}, \trsl{at school}
\a \ird{na vele}, \trsl{in the countryside}
\xe

Where English uses specific prepositions such as \trsl{above}, \trsl{under}, \trsl{below}, etc., Iridian uses a compound construction with \ird{na} and another noun indicating the location marked in the accusative. The `object' in the equivalent English construction is marked in the genitive in Iridian.

\pex
\a \ird{na dumí veha}, \trsl{in front of the house}
\a \ird{na bamení pouda}, \trsl{behind the building}
\xe

\subsection{še}

\subsection{\ird{vo}}\index{vo}\index{agentive case}

\ird{Vo} can be translated as \trsl{because of} or \trsl{due to.} This preposition takes the agentive case.

\pex
\begingl
\gla Vo transitám lienu zásčenžek.//
\glb because traffic-\Agt{} on:time-\Ins{} \Neg{}-arrive-\Av{}-\Pf{}//
\glft \trsl{I didn't arrive on time because of the traffic.}//
\endgl
\xe

\subsection{za}

\section{Quantifiers}\index{quantifiers}
Iridian has a wide variety of non-numerical/indefinite quantifiers.  Most are actually nouns that used in adjectival or adverbial constructions.


\begin{itemize}
    \item \ird{ošč} \trsl{many} (countable)
    \ex
    \begingl
    \gla Marka ješ naže ošč.//
    \glb Marek-\Acc{} \Exst{} friend-\Gen{} many//
    \glft \trsl{Marek has many friends.}//
    \endgl
    \xe
    \ex
    \begingl
    \gla Za kursa mén ješ ošč oudinášce ko vilm.//
    \glb for class-\Acc{} \mk{1pl.inc.wk} \Exst{} many watch-\Sup{} \Att{} film.//
    \glft \trsl{We have a lot of movies we need to watch for our class.}//
    \endgl
    \xe
    \item \ird{nave} \trsl{too many} (countable)
    \ex
    \begingl
    \gla Marka ješ naže ošš.//
    \glb Marek-\Acc{} \Exst{} friend-\Gen{} many//
    \glft \trsl{Marek has many friends.}//
    \endgl
    \xe
    \item \ird{tohle} \trsl{many} (uncountable)
    \item \ird{nahte} \trsl{too many, too much} (uncountable)
    \ex
    \begingl
    \gla Do ješ nahte kurváš//
    \glb \First{}\Sg{}.\Acc{} \Exst{} too:much work-\SupN{}//
    \glft \trsl{I have so much work to do.}//
    \endgl
    \xe

\end{itemize}

\section{Interjections}

An interjection\index{interjection} is a word or an expression used to express a spontaneous reaction or feeling. We will use the term `interjection' to refer both to the part of speech and to the utterance type that has the same pragmatic function as this part of speech (cf. \cite{ameka1992}).

Interjections can be classifed into two main categories: \emph{primary} interjections, which refer to a word or an utterance that can only be used as an interjection and \emph{secondary} interjections, which refer to forms belonging a different word class but which through its usage, has acquired a new meaning as an interjection.

Although interjections can function as exclamations, not all exclamatory utterances can be considered as interjectons by themselves. As \textcite{jovanovic2004} notes, any word in a language can theoretically become an exclamation. Consider for example this conversation:

\ex (adapted from \cite{jovanovic2004}).\\

  \ird{
  \noindent--- Martin mlaza boulešik.\\
  --- \textbf{Martinám?}
  }\medskip

  \trsl{I heard Martin killed his brother.}\\
  \trsl{Martin?!}
\xe


\section{Discourse particles}

\subsection{Yes and no}
Iridian has several words for yes and no but their usage in responding to yes-no questions does not exactly align with that of English. This is discussed in detail in \S\,\ref{sec:ansyn}.

There are two main words for \trsl{yes} in Iridian: the affirmative \ird{dé} (\trsl{Did you see it?} \trsl{Yes, I did.}) and the contrastive \ird{če} (\trsl{Did you not see it?} \trsl{Yes, I did.}. The distinction is similar as that between the French \emph{oui} and \emph{si}. Both \ird{dé} and \ird{če} generally appear at the end of a sentence. In colloquial spoken Iridian it is also common to see the form \ird{ja} (most likely from the Czech, and ultimately from the German \emph{ja}) and the more informal \ird{jó}. These forms however are not cliticized to the verb and appear at the start of a sentence, set off from the rest with a commma. Both \ird{ja} and \ird{jó} cannot be used contrastively like \ird{če}. It is also common to use both \ird{ja/jó} at the same time as \ird{dé}.

\pex
\begingl
\gla ---To vdinice? ---Ja vdinek dé.//
\glb this see-\Pv{}-\Pf{}-\Quot{} yes see-\Pv{}-\Pf{} yes//
\glft \trsl{{}``Did you see it?'' ``Yes, I did.''{}}//
\endgl
\xe

When used by themselves, both \ird{ja} and \ird{jó} are often repeated twice or thrice (e.g., \ird{Ja ja ja.})\footnote{Commas are not used to separate each \ird{ja} or \ird{jó} in standard orthography. } even when the usage is not emphatic. \ird{Dé} and \ird{če} cannot be used this way.

\section{Numerals}\label{sec:numerals}

Iridian has a vigesimal number system. Table \ref{tab:nums-one-twenty} shows
Iridian numerals from 1 to 20. Numbers from 1 to 10 are given their own name
while numbers from 11 to 19 are formed by appending the numbers from one to nine
to the clitic \ird{-něm} with the preposition \ird{še} (with). The clitic
\ird{-něm} is derived from the word for number 10, \ird{nau}, which itself
comes from the Old Iridian \rec{nagu}, `half.'

	
\begin{table}
\footnotesize\sffamily
\caption{Iridian numerals from 1 to 20.}
\medskip
\begin{tblr}{width=0.8\textwidth,colspec={X[0.7]XX[0.7]X}}
	\toprule
	{\sc number} & {\sc iridian} & {\sc number} & {\sc iridian}\\ 
	\midrule 
	1 & ona			& 11 & onšeněm\\ 
	2 & vuc			& 12 & myšeněm\\ 
	3 & hrona		& 13 & hronašeněm\\ 
	4 & drou		& 14 & drušeněm\\ 
	5 & jed			& 15 & jecněm\\ 
	6 &	dve			& 16 & vušeněm\\ 
	7 & šče			& 17 & ščiceněm\\ 
	8 & pieš		& 18 & pisčeněm\\ 
	9 & cam			& 19 & camšeněm\\ 
	10& nou			& 20 & týdna\\ 
	\bottomrule
	\label{tab:nums-one-twenty}
\end{tblr}
\end{table}

Numbers from 21 to 99 are first expressed as multiples of 20. Thenceforth, the
number system has largely become decimal, due primarily to the influence of
surrounding Indo-European languages. Old Iridian, however, had a vigesimal
system up to the number 8000.

Table \ref{tab:nums-thirty-one-hundred} shows multiples of 10 from 30 to 100.
The numbers are formed by the numeral followed by \ird{týdna}. For bases that
are not multiples of 20, the word \irdp{nau}{ten} is added first, followed by
the conjunction \irdp{še}{with}.

\begin{table}
	\footnotesize\sffamily
	\caption{Iridian numerals from 30 to 100.}
	\medskip
	\begin{tblr}{width=0.8\textwidth,colspec={X[0.5]XX[0.5]X}}
		\toprule 
		{\sc number} & {\sc iridian} & {\sc number} & {\sc iridian}\\ 
		\midrule 
		30 &	naušetýdna		& 70 	& naušehronutýdna\\ 
		40 &	vutýdna			& 80	& drohutýdna\\ 
		50 &	nauševutýdna	& 90	& naušedrohutýdna\\ 
		60 &	hronutýdna		& 100	& měs\\ 
		\bottomrule
		\label{tab:nums-thirty-one-hundred}
	\end{tblr}
\end{table}

Iridian counting starts from the smallest component of the number to the
largest. Each component can be simply appended with the conjunction \ird{a}. The
examples below illustrate the formation of more complex numbers. Table
\ref{tab:nums-two-hundred-one-trillion} shows Iridian numerals from 200 to one
billion.

\pex
\a \irdp{jed a měs}{five and hundred} i.e., 105
\a \irdp{cam a drohutýdna}{nine and four twenties} i.e., 89
\a \irdp{pisčeněm a hronutýdna}{eighteen and three twenties} i.e., 78
\xe

\begin{table}
	\footnotesize\sffamily
	\caption{Iridian numerals from 200 to one trillion.}
	\medskip
	\begin{tblr}{width=0.9\textwidth,colspec={X[0.6]X}}

		\toprule 
		{\sc number} & {\sc iridian} \\ 
		\midrule 
		200 			&	mach	\\ 
		300, 400, etc.	& 	hronuměs, drohuměs. etc.\\ 
		1000			& 	něk\\ 
		2000, 3000, etc.& 	vuněk, hronuněk, etc.\\ 
		10.000			&	ohle\\ 
		20.000, etc.	& 	t\'ydnuněk, etc.\\ 
		100.000			&	hazlek\\ 
		200.000 etc		&	mehdeněk, hronuněk, etc.\\ 
		1.000.000		&	miliám\\ 
		1.000.000.000	&	milár\\ 
		1.000.000.000.000	& biliám\\ 
		\bottomrule
		\label{tab:nums-two-hundred-one-trillion}
	\end{tblr}
\end{table}

When used attributively, numerals do not require the particle \ird{ty} to be
linked to a noun or a noun phrase. The numerals \ird{týdna}, \ird{měs},
\ird{mach}, \ird{něk}, \ird{ohle}, \ird{hazlek}, \ird{miliám}, \ird{milár} and
\ird{biliám} may reduplicated, separated by \irdp{a}{and} to indicate an
estimate (e.g., \irdp{měs a měs cel}{hundreds of people}). This is similar to
the pluralization of numerals in English for the same purpose, as in, for
example, \trsl{tens} or \trsl{hundreds}. 

Numerals may also be used nominally, in which case they are inflected like
regular nouns.

\subsection{Ordinal numbers}
\label{sec:ordinals}

Except for the first three cardinal numbers that have irregular ordinal forms,
ordinals are mostly regular, formed with the suffix \ird{-šle} (or \ird{-išle}
after consonants). The ordinal form of the numbers one, two and three are
\ird{hezka}, \ird{dvěc} and \ird{cehra}, respectively. The ordinal form is
\ird{měs} is also irregular, being \ird{měšle} and not \ird{*měšišle}. When
written as numerals, a full stop is used followed by a dash (e.g.,
\irdp{camišle}{ninth} would be written 9.-). In compound numbers, only the last
component is inflected with \ird{-šle}; `eighty seventh' for example would be
\ird{šče a drohutýdnišle} and not \ird{*ščišle a drohutýdna}.

The letter n has its own ordinal form (cf. English \trsl{nth} for example),
\ird{enišle}, as do the rest of the other letters. These ordinal forms are
generally regular. Their usage is confined to mathematical literature, however,
with the clear exception of \ird{enišle}, which is often used idiomatically (cf.
French \textit{pour la enième fois}).


\subsection{Fractions, decimals and other derivative forms}
\label{sec:fractions}

As with most languages in Europe, Iridian uses the comma (Iridian \ird{kva}) to
separate whole number from decimals (see \S~\ref{sec:punctuation}). Numbers
after the comma are read in pairs of two, with the first number read separately
in case there is an odd number of numerals after the comma (e.g., 3,34 is read
as \ird{hrona kva drušeněm a týdna} while 3,347 is read \ird{hrona kva hrona šče
a vutýdna}). Generally, if there are seven or more numbers following the come,
each is read separately instead, though this is not a hard and fast rule and the
speaker may read the numbers separately even if there are fewer than seven
decimal places.

Fractional forms are regularly formed using the suffix \ird{-izmek}. The word
for half, \ird{num}, however is irregular. Fractional forms are sometimes used
together with the regular decimal forms when dealing with currency. For example,
5,50 kr. can be read as either \ird{jed kva nauševutýdna korun} or more commonly
\ird{jed a num korun} (cf. English \trsl{five and a half dollars}).

\subsection{Measurements}\label{sec:measurements}

Iridian uses the metric system for most measurements. The cardinal form of the
number is used, followed by the name of the unit. If the unit appears
independently, it is unmarked; if however, the measurement is used
attributively, the unit is marked in the instrumental case: thus one writes
\irdp{hrona měter}{three meters} and \irdp{hrona mětru kuz}{three meters of
silk.} Some commonly used units include \irdp{měter}{meter} for length,
\irdp{gram}{gram} for weight, \irdp{lěter}{liter} for volume, or generic units
like \irdp{procent}{per cent,} \irdp{par}{pair,} \irdp{tuzyn}{dozen} and
\irdp{týdna}{score}.\footnote{The word for \trsl{score} and \trsl{twenty} are
identical, \ird{týdna}. When used as a numeral attributively \ird{týdna} is
indeclinable. When used as a unit, it declines like a regular noun.} SI prefixes
like \ird{kilo-}, \ird{cénti-}, etc. are also commonly used.

\subsection{Date and time}\label{sec:date-time}

Dates are written with the year first, followed by the month, and ultimately by
the date. When written in numerals, the numbers are separated by a full stop.
When spoken or when written in full, the number representing the year is
followed by the word \irdp{hlet}{year}, often in the instrumental case. When
followed by the name of the month, \ird{hlet} is declined in the genitive. When
the date is included, the ordinal form is used, followed by the word
\irdp{ráz}{day,} although the latter may be dropped in casual speech. The
inclusion of the date also requires the name of the month to be in the genitive
case.

\pex
\a
\begingl
    \gla 1992 hletí julí 15. rázu veštašik //
    \glb 1992 year\mk{-gen} july-\Gen{} 15th day-\Ins{} be:born-\Av{}-\Pf{}//
    \glft \trsl{I was born on 5 July 1992.}//
\endgl
\xe

\begin{table}
	\footnotesize\sffamily
	\caption{Months of the year.}
	\medskip
	\begin{tblr}{width=0.7\textwidth,colspec={XXXX}}
		\toprule
		{\sc month} & {\sc iridian} & {\sc month} & {\sc iridian}\\
		\midrule
		January		& jenvár	& July & jul\\
		February	& fevrár 	& August & augošt\\
		March		& merc		& September & seitembár\\
		April		& april 	& October & oktobár\\
		May 		& mai 		& November & novembár\\
		June 		& jón 		& December & dicámbár\\
		\bottomrule
		\label{tab:months}
	\end{tblr}
\end{table}

