\chapter{Sample Texts}

\section{The \emph{Pater Noster}}

\section{Milan Kundera, \ttla{A Kidnapped West or the Tragedy of Central Europe}}

{\small
% NOTE ON THE TRANSLATION
The translation is based on the French text of Kundera's essay \ttla{Un Occident kidnappé: ou la tragédie de l'Europe centrale} first published in \ttlb{Le Débat} in 1983. The full text is available online at various websites, with the link I used in the references. Due to copyright considerations, a translation has not been provided, although interlineal glosses and explanatory notes have been added where  I believe they are needed, in addition to the lexicon at the end. The text itself contains its own footnotes however and to distinguish Kundera's notes from those I have added, I have included included his name at their end.
}

\begin{center}1.\end{center}

1956 svemí Septembru Mažarevní Znova Byróví direktorám, byró
nastolám jednočnil ko obiení vniho minutu, ruščevnie uráž
po Budapešta šelčice to-že télexu laska mieta kudní expedica
pashvalébik. Expedice to nie neitu uhožnek: >> Mé za Mažaróma a
za Evropa shražach<<.

Nie neite ježe-no prónesčeví? Mažaróma a še laska Evropu ruščevní šarám zbavujinalu to žvotu prónesčeví. Ma Evropa zbavujinale to ježe prónestu?

Ma žená --- >>za bláha a za Evropa shražá<< --- to že Leningrada že Mušhóva závadnéteví to neite, ma če je Budapešta, če je Varšáva.

\begin{center}2.\end{center}
Vade, Evropa-te ježe-no za ona mažarevna, ona češčevna, ona polščevna?


\section{Written Correspondence}\label{sec:writcorr}\index{written correspondence}

\subsection{Formal Business Letter}
{\small
\begin{flushright}
	Roubže\\
	2019\,h. Mercí 14.\,r.
\end{flushright}

\noindent{Marek Zakár}\\
Ledeman Direkt {\sc m/h}\\
Husplac, \textnumero{} 177\\
Osthalbár\\
86332 Roubže {\sc rb}
}

\subsection{Formal E-mail}




\subsection{Informal Letter}
