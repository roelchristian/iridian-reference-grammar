\chapter{Sample Texts}

\section{The \emph{Pater Noster}}

\section{Milan Kundera, \ttla{A Kidnapped West or the Tragedy of Central Europe}}

{\small
% NOTE ON THE TRANSLATION
The translation is based on the French text of Kundera's essay \ttla{Un Occident kidnapp\'e: ou la trag\'edie de l'Europe centrale} first published in \ttlb{Le D\'ebat} in 1983. The full text is available online at various websites, with the link I used in the references. Due to copyright considerations, a translation has not been provided, although interlineal glosses and explanatory notes have been added where  I believe they are needed, in addition to the lexicon at the end. The text itself contains its own footnotes however and to distinguish Kundera's notes from those I have added, I have included included his name at their end.
}

\begin{center}1.\end{center}
En 1956, au mois de septembre, le directeur de líagence de presse de Hongrie, quelques minutes avantque son bureau fÈcrasÈ par líartillerie, envoya par tÈlex dans le monde entier un message dÈsespÈrÈsur líoffensive russe, dÈclenchÈe le matin contre Budapest. La dÈpÍche finit par ces motsNous mour-rons pour la Hongrie et pour líEurope
