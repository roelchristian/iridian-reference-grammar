\chapter{Verbs}

\section{Introduction}


Verbs in Iridian are heavily marked. There is a tendency to encode most of the information contained in the sentence on the verb leaving the noun or noun phrase unmarked if possible. 

\par Finite verbs are marked\index{markedness} for the following grammatical categories\index{grammatical categories}:
\begin{enumerate}[nosep]
	\item {\scshape aspect}.\index{aspect} Iridian has three primary aspects: perfective, imperfective and contemplative; and two secondary ones: retrospective and prospective.
	\item {\scshape voice}.\index{voice} Iridian has a strong tendency to leave the topic of the sentence unmarked, instead encoding the primary information on the verb. Due to this, voice must be explicitly marked on the verb. Iridian has the following grammatical voices: agentive, patientive, benefactive, instrumental, locative and reflexive.
	\item {\scshape mood and modality}.\index{mood} Besides the unmarked indicative, Iridian has the following grammatical moods: subjunctive, conditional, hortative, optative, abilitative, permissive and non-volitive. In addition, secondary prefixes are used to express what would otherwise could be considered as moods: inceptive, causative and reciprocative.
	\item {\scshape evidentiality}.\index{evidentiality} Iridian formally distinguishes between reported and non-reported information, although actual usage of the former often encompasses various semantic and not strictly evidentiality-related applications.
\end{enumerate}

Verbs are also marked for person, although this is done by the addition of clitic pronouns and not through a separate conjugation paradigm. In most cases, however, this is left out, especially if clear from the context. Iridian verbs are not marked for tense, gender, or number.

\par Iridian verbs have four classes of non-finite forms: the gerund, the converb, the supine and the generic nominal formed with \ird{-ou}. The non-finite verb forms are derived from the uninflected verb stem except the generic nominal in \ird{-ou} which can only be formed from a fully-inflected verb stem. A fifth class exists--the infinitive--but this form is largely defunct and is only used in certain compound constructions. Infinitives end in \ird{-á} and is used as the citation form of a verb.

\section{Verb stem and order of inflectional affixes}\index{citation form}\index{infinitive}


\subsection{The verb stem}\index{verb stem}
\par The {\scshape citation form} (or {\scshape dictionary form}\index{dictionary form|see{citation form}} or {\scshape lemma}\index{lemma|see{citation form}}) of a verb is the uninflected {\scshape infinitive}\index{infinitive}, a fossilised form rarely used outside of a very few periphrastic\index{periphrasis} and historical constructions (see \S\,\ref{sec:infinitive}). The infinitive ends with the vowel \ird{-á}, and removing this ending will produce the {\scshape verb stem}\index{verb stem}. The final consonant  of the stem is called the thematic consonant\index{thematic consonant} and determines the conjugation paradigm the verb follows. The verb stem is a bound form and must always appear with at least one inflectional suffix.


\subsection{Sound changes}\label{sec:sound-changes}
Verb stems are normally classified into five groups (called {\scshape classes}) based on how their thematic consonant changes in unstable positions; specifically, since the verb stem most often interacts with the suffixes used in marking grammatical voice, these classes are based on how the stem changes when followed by a sibilant suffix (as in the active voice) or a palatalising suffix (as in the passive voice). The five classes are as follows: 
\begin{enumerate}[nosep]
	\item Class I verbs include verbs with a thematic \ird{-t, -k, -c} and \ird{-\v{c}}. They all merge to \ird{-\v{c}} in the active voice (\ird{pia\v{s}t-}\,$\rightarrow$\,\ird{pia\v{s}\v{c}-}) but remain stable when followed by a palatalising suffix, except \ird{-c} and \ird{-\v{c}} which merge to [t͡ɕ] although this is not reflected orthographically.
	\item Class II verbs include verbs with a thematic \ird{-s} or \ird{\v{s}}, which both merge to [ɕ] in oth the active and passive voice, although only the former is reflected orthographically. 
	\item Class II-A (or Class IV) verbs are the smallest group and include verbs with a thematic \ird{-l} or {-p}. They use the suffix \ird{-\v{s}} in the active voice (\ird{dal-}\,$\rightarrow$\,\ird{dal\v{s}-}) and are stable elsewhere.
	\item Class III verbs include verbs with a thematic \ird{-d, -g, -h, -j, -z} and \ird{-\v{z}}. They all merge to \ird{-\v{z}} in the active voice (\ird{vad-}\,$\rightarrow$\,\ird{v\'a\v{z}-}); they remain stable when followed by a palatalising suffix, except \ird{-z} and \ird{-\v{z}} which merge to [ʑ] although this is again not reflected orthographically.
	\item Class III-A (or Class V) verbs include those ending with the remaining thematic consonants. They use the suffix \ird{-\v{z}} in the active voice (\ird{\v{s}\v{c}en-}\,$\rightarrow$\,\ird{\v{s}\v{c}en\v{z}-}) and are stable elsewhere.
\end{enumerate}

This classification is notwithstanding the fact that if the thematic consonant is immediately after one or more consonants (except a lateral) an epenthetic \ird{-a-} is added and the suffix -\ird{-\v{s}} is used to form the active root regardless of the actual thematic consonant. As such we get \ird{parka\v{s}-} from \irdp{park\'a}{to park} but \ird{p\'al\v{c}-} from \irdp{palk\'a}{to punch.} Moreover German loanwords whose infinitives end in \ird{-irn\'a} behave as if they have a thematic \ird{-r} and so the the active root for \irdp{t\'el\'evonirn\'a}{to call (on the phone)} is \ird{t\'el\'evonir\v{z}-} instead of \ird{*t\'el\'evonirna\v{s}-}.

The suppletion of the original thematic consonant in the first to third classes with the class ending causes the preceding vowel to be lengthened in compensation if the root would have ended in an open syllable or a lateral had the thematic consonant been removed; thus we have, e.g., \ird{d\'u\v{s}-} from \irdp{du\v{s}\'a}{to bathe} but \ird{pia\v{s}\v{c}-} and not \ird{*pi\'a\v{s}\v{c}-} from \irdp{pia\v{s}t\'a}{to eat.} If the remnant vowel is \ird{\v{e}} or the diphthong \ird{ei}, the compensatory lengthening also involves the reduction of the vowel to \ird{-\'i} as in \ird{zd\'i\v{c}-} from \irdp{zd\v{e}k\'a}{to blow.}

\subsection{Finite verb endings}

\section{Voice}\index{voice}

Iridian often prefers to encode information on the verb instead of through case marking on nouns. As such, all verbs must be explicitly marked for voice.
\begin{table}[!ht]
	\sffamily\footnotesize
	\caption{Suffixes used to mark grammatical voice.}\medskip
	\begin{tabu} to 0.5\textwidth{@{}YY[0.5]@{}}
		\toprule
		&{\scshape ending}\\
		\midrule
		Agentive	& {-(a)š-}\\
		Patientive	& {-in-}\\ 
		Benefactive	& {-éb-}\\ 
		Locative	& {-oun-}\\ 
		Instrumental& {do-\,-oun-}\\ 
		Reflexive	& -\\ 
		Reciprocal	& \\ 
		\bottomrule
	\end{tabu}
\end{table}

\subsection{Agentive voice}\index{agentive voice}
\par The agentive voice is used if the subject of the verb is the agent of the action. Unlike English and other languages with a nominative-accusative morphosyntactic alignment, Iridian does not have a ``default'' voice for a verb. As such, the unmarked form of a verb is essentially meaningless. Voice must be explicitly marked on the verb and the agentive does not have any precedence over the other voices. The agentive voice is marked by the suffix \ird{-aš-}, which assimilates to the thematic consonant of the verb, as described in \S\ref{sec:sound-changes}.
\pex
\begingl
\gla Sa piašček.//
\glb already eat-\Av{}-\Pf{}//
\glft `(I) already ate.'//
\endgl
\xe

Where the assimilation involves the deletion of the final consonant in the root, the preceding vowel is lengthened in compensation if the resulting root would then end in an open syllable.\index{compensatory lengthening}
\begin{multicols}{2}
\pex
\ird{Udúšek.}\\
(instead of \ird{*udušek})\\
\trsl{(I) took a shower.}
\xe
\pex
\ird{Piašček.}\\
(not \ird{*piášček.})\\
\trsl{(I) ate.}
\xe
\end{multicols}

If the remnant vowel is the i-glide \ird{-ě-} or the diphthongs \ird{-ei-} and \ird{-ou-}, the remaining vowel would simplify to \ird{í}, \ird{í} and \ird{ú}, respectively. Consider for example the verb \ird{zděká} \trsl{to blow}:

\pex
\begingl
\gla Lest zdičime.//
\glb wind blow-\Av{}-\Prog{}//
\glft \trsl{The wind is blowing.}//
\endgl
\xe

\subsection{Patientive voice}\index{patientive voice}

\par A verb in the patient focus (glossed \Pv{}) indicates that the topic is the thematic patient in the sentence. This is roughly equivalent to the English passive voice, although the patientive voice is more commonly used in Iridian than the passive voice in English. 

\pex
\begingl
\gla Marek vidnek.//
\glb Marek see-\Pv{}-\Pfv{}//
\glft `(I) saw Marek.'//
\endgl
\xe


\subsection{Benefactive voice}\index{benefactive voice}

\par The benefactive voice (glossed \mk{ben}) is used when the subject of the sentence is the benefactor or director object of the verb. Verbs often change meaning when used in the benefactive focus.


\begin{multicols}{2}
\pex
\begingl
\gla Mač sega nazdébik.//
\glb mother flower-\Acc{} buy-\Ben{}-\Pf{}//
\glft `(I) bought my mother flowers.'//
\endgl
\xe

\pex
\begingl
\gla Kova piaštébime.//
\glb cow eat-\Ben{}-\Prog{} //
\glft \trsl{(I am) feeding the cows.}//
\endgl
\xe

\end{multicols}

The benefactive is also used idiomatically with verbs of judgment including \ird{novětá} \trsl{to like}

\pex
\begingl
\gla Dá čehóvám zánovítébime.//
\glb \First\Sg{} sports-\Agt{} \Neg{}-like-\Ben{}-\Prog{}//
\glft \trsl{I don't like sports.}//
\endgl
\xe

\subsection{Locative voice}

The locative voice (glossed \mk{lv}) is used when the topic of the sentence is the location of the action. The locative voice is often used with verbs of motion, but can also be used, similar to the benefactive voice, idiomatically with verbs of perception and other verbs, often with a derogatory connotation.

\pex
\begingl
\gla Bištro piaštounime.//
\glb cafeteria eat-\Loc{}-\Prog{}//
\glft \trsl{(We are) eating in the cafeteria.}//
\endgl
\xe

\pex
\begingl
\gla Já kopounime.//
\glb you laugh-\Lv{}-\Prog{}//
\glft \trsl{He is laughing at you.}//
\endgl
\xe

\subsection{Instrumental voice}


\subsection{Reflexive voice}

The reflexive voice (glossed {\Refl}) is used when the patient of the verb is also the agent of the action. Morphogically, the reflexive voice is not a separate voice but is derived from the agentive form of the verb and the addition of the prefix \ird{u(d)-}.

\pex
\begingl
\gla Na šarta uvižek.//
\glb \Loc{} mirror-Pat{} \Refl{}-see-\Av{}-\Pf{}//
\glft \trsl{I saw myself in the mirror.}//
\endgl
\xe

The use of the reflexive voice is more extensive in Iridian than in English\index{English}, and is somehow similar to how the reflexive construction is used in Romance languages.

\pex
\begingl
\gla Uštižek.//
\glb \Refl{}-take:a:bath-\Av{}-\Pf{}//
\glft \trsl{(I) took a bath.}//
\endgl
\xe

\pex
\begingl
\gla Umúšime.//
\glb \Refl{}-comb-\Av{}-\Prog{}//
\glft \trsl{(I) am combing my hair.}//
\endgl
\xe

Below is a non-exhaustive list of verbs that are normally used in the reflexive voice:
\bigskip

\noindent
\ird{dušá} \trsl{to take a shower}\\
\ird{mušá} \trsl{to comb}\\
\ird{šaštá} \trsl{to sit down}\\

Some verbs may change meaning when used in the reflexive voice.


The reflexive voice is also used to imply that an action happened accidentally or involuntary or that the agent of the action is unknown or unimportant.

The reflexive voice may also be used emphatically, especially in spoken Iridian, to express that the action has been performed for the benefit of the actor/agent of the verb.

\pex
\begingl
\gla Kávéa ušranz\k{a}cem.//
\glb coffee-\Acc{} \Refl{}-drink-\mk{av-ctplv-1s}//
\glft \trsl{I'll drink coffee.} (literally, I'll drink myself coffee)//
\endgl
\xe

\pex
\begingl
\gla Pulša uvošček.//
\glb soup-\Acc{} \Refl{}-cook-\Av{}-\Pf{}//
\glft \trsl{(I) cooked (me) some soup.}//
\endgl
\xe

\section{Grammatical aspect}\label{sec:aspect}\index{aspect}\index{grammatical aspect|see{aspect}}

{\scshape grammatical aspect} (or simply {\scshape aspect}) is a category in Iridian that is used to denote how an action or state described by a verb extends over time. Aspect contrasts with {\scshape tense} which situates an action or event as happening or being true at some specific point in time. Iridian does not mark tense grammatically, as does English\index{English}, for example, and so the verb in English sentences \trsl{I am watching a movie on TV now} and \trsl{I was watching a movie on TV when you called} will both be translated using the same verb in the progressive aspect. 

Aspect also contrasts with {\scshape lexical aspect} or \foreign{aktionsart}\index{aktionsart@\emph{aktionsart}} in that the latter, although also describing a verb's structure in relation to time, refers more to an inherent property of the verb itself and is thus often invariant. Iridian also does not grammaticalise \foreign{aktionsart}, and thus the distinction between, say, Polish imperfective \foreign{pisać,} \trsl{to write} and perfective \foreign{napisać,} \trsl{to write down,} is not one made in the language (cf. \cite[9--26]{richardson2007}; \cite{comrie1976}).

Iridian formally distinguishes between seven classes of grammatical aspect, five of which are called {\scshape primary} since they are used to describe the aspect of an independent main verb while two are called {\scshape secondary} since they represent a verb's aspect in relation to some other event.

\begin{table}
	\footnotesize\sffamily
	\caption{Aspect markers in the indicative mood.}
	\medskip
	\begin{tabu} to 0.5\textwidth{@{}YY[0.5]}
		\toprule
		{\sc aspect}	& {\sc affix}\\
		\midrule
		Perfective		& {-ek}\\
		Retrospective	& {-aní}\\
		Imperfective	& {-eví}\\
		Progressive		& {-ime} \\
		Contemplative	& {-\'ach/-\'ah-}\\
		Prospective		& {-ujam}\\
		Cessative		& {-óvít}\\
		\bottomrule
	\end{tabu}

\end{table}


\subsection{Perfective and retrospective aspect}\label{sec:perfective-retrospective}

Both the perfective aspect (glossed \Pf{}) and the retrospectived {glossed \Ret{}} represent actions that have been completed at some point in time.

The perfective aspect indicates an action that has been completed at some specific point in time. The thematic ending for the perfective aspect is \ird{-ek}, but the initial $\langle$e$\rangle$ is rather unstable and often changes depending on the environment. The initial $\langle$e$\rangle$ becomes $\langle$i$\rangle$ when used with \ird{-in} (the suffix indicating the patientive voice), with the initial $\langle$i$\rangle$ in the preceding suffix often dropped or replaced by an $\langle$e$\rangle$. This change also occurs when following the benefactive suffix \ird{-éb} and when followed by the quotative suffix \ird{-e} (in which case the final \ird{-k} is fricativised to $\langle$c$\rangle$).

\pex
\begingl
\gla Bych na gnaža Marek vdenik.//
\glb yesterday \Loc{} school-\Acc{} Marek see-\Pv{}-\Pf{}//
\glft \trsl{(I) saw Marek at school yesterday.}//
\endgl
\xe

When negated, the perfective indicates something that ought to be done but had not been done or otherwise to put an emphasis on the non-completion. To state that something simply did not happen, the negative of the retrospective is used instead.

\begin{multicols}{2}
\pex
\a\begingl
\gla Zátélévoniržek.//
\glb \Neg{}-telephone-\Av{}-\Pf{}//
\glft `(I) failed to call.' //
\endgl
\a\begingl
\gla Zátélévoniržaní.//
\glb \Neg{}-telephone-\mk{av-ret}//
\glft `(I) didn't call.' //
\endgl
\xe
\end{multicols}

\par The retrospective aspect is used for a past action that has a continuing relevance in the presence. Often when using the retrospective aspect, the emphasis is on the resulting state rather than the event itself. Consider, for example, the following sentences: (a) \trsl{I went to Amsterdam last week}; and (b) \trsl{I have been to Berlin in my childhood}. Iridian would translate the verb in (a) using the perfective and the verb in (b) using the retrospective. The retrospective may also be called the {\scshape perfect} in other sources but we shall be solely referring to it as the former in this grammar to avoid any confusion. In addition, the retrospective in Iridian always has a perfective interpretation, i.e., a sentence like \trsl{I have been waiting here an hour} which has a verb in the perfect is translate using the progressive instead of the retrospective.

\pex
\begingl
\gla Hroná tímu na Budapešta možlašime.//
\glb three year-\Ins{} \Loc{} Budapest-\Acc{} live-\mk{av-prog}//
\glft `I have been living in Budapest for three years.'//
\endgl
\xe

\pex
\begingl
\gla Vegetarevn\'i gul\'a\v{s}e stav\'i\v{c}an\'i, ma toleto z\'a\v{c}e\v{s}\v{c}ice.//
\glb vegetarian goulash-\Gen{} eat:a:bit-\Av{}-\Ret{} but \Aff{} \Neg{}-like-\Av{}-\Pf{}-\Quot{}//
\glft \trsl{I have tried vegetarian goulash before but I didn't like it at all.} //
\endgl
\xe

The retrospective is also used to describe an event that is expected or planned to happen before a certain point in the future.

\pex
\begingl
\gla D\'a \v{s}\v{c}en\v{z}it \v{s}e Marek uzdrav\v{z}an\'i.//
\glb \First{}\Sg{}.\Str{} arrive-\Av{}-\SupP{} \Com{} Marek \Refl{}-sleep-\Av{}-\Ret{}//
\glft \trsl{Marek would already have been asleep by the time I arrived.} //
\endgl
\xe

\pex
\begingl
\gla Na arma\v{s}ta \v{s}\v{c}en\v{z}ice dn\'ovim po z\'azdal\v{s}an\'i.//
\glb \Loc{} airport-\Acc{} arrive-\Av{}-\SupP{}-\Att{} front-\Ins{} yet have:breakfast-\Av{}-\Ret{}//
\glft \trsl{We will not have eaten breakfast before we get to the airport tomorrow.} //
\endgl
\xe

Moreover, the retrospective is often used to imply non-volition or the  accidental/circumstantial nature of an action. Similarly the retrospective is used with verbs of emotion or state (e.g., \ird{cezuštalá}, ‘to become happy’ from \ird{zuštal} ‘happy’). The perfective, on the other hand, is almost exclusively used with the causative in these cases.

\pex
\a	\begingl
\gla Vdešek še neicezuštalašaním.//
\glb see-\mk{2s-pf} with \mk{incep}-be.happy-\mk{av-ret-1s}//
\glft `I became happy when I saw you.' //
\endgl
\a	\begingl
\gla Do pacezuštalnikeš.//
\glb \First{}\Sg{}.\Wk{} \Caus{}-be.happy-\mk{pv-pf-2s}//
\glft `You made me happy.' //
\endgl
\xe
\pex<vasebroke>
\begingl
\gla Váz noprizaní.//
\glb vase break-\mk{ref-ret}//
\glft `The vase broke (accidentally).' //
\endgl
\xe

\subsection{Continuous and progressive aspects}
Iridian uses the continuous and progressive aspects to denote actions that have not been completed yet and/or are in the process of happening/occuring. The continuous aspect (glossed \mk{cont}) is used to mark a state of being while the progressive aspect (glossed \mk{prog}) is used to mark a dynamic activity.
\pex
\begingl
\gla Nau urištneví.//
\glb clothes \Refl{}-wear-\mk{pv-cont}//
\glft \trsl{(I'm) wearing clothes.} //
\endgl
\xe

\pex
\begingl
\gla Nau urištnime.//
\glb clothes \Refl{}-wear-\mk{pv-prog}//
\glft \trsl{(I'm) putting on clothes.} //
\endgl
\xe

The continuous aspect is also used to denote a habitual action.

\pex
\begingl
\gla Sholu de gnaža stoževí.//
\glb daily-\Ins{} \mk{ill} school-\Acc{} go-\Av{}-\Cont{}//
\glft \trsl{(We) go to school everyday.} //
\endgl
\xe

\pex
\begingl
\gla Dá na Praha možleví.//
\glb \mk{1s.str} \Loc{} Prague-\Acc{} live-\mk{cont}//
\glft \trsl{I live in Prague.} //
\endgl
\xe

To emphasise the habitual nature of an action, a nominalised construction is often used.

\pex
\begingl
\gla Nažem r\k{a}cenživou.//
\glb friend-\First{}\Sg{} smoke-\mk{av-cont-nz}//
\glft \trsl{My friend is a smoker.} //
\endgl
\xe


\subsection{Contemplative aspect}

The contemplative aspect (glossed \Ctp{}) is used to mark an action that has not been started yet. If the emphasis, however, is on the speaker's intention to do something, and not on the incompleteness or futurity of the action itself, the supine of purpose is used instead of the contemplative. The same is true for events which the speaker thinks is improbable or unlikely to happen but whose non-occurence they are nonetheless unsure of.

\subsection{Secondary aspects}
The prospective aspect (glossed {\scshape prosp}) is primarily used in secondary clauses to indicate actions that are about to be started in relation to another action. It can also be used in the main clause to indicate an action in the immediate future.


The cessative aspect

\section{Valency}\index{valency}\index{valence|see{valency}}

{\scshape valency} (or {\scshape valence})\index{valency} is the number of overt arguments a verb\index{argument of a verb} can take in a sentence. \textcite[239]{tesniere1965}, in one of the earliest description of the concept, likens valency by comparing it to bonds between atoms:
\begin{quotation}
	\small
The verb may therefore be compared to a sort of atom, susceptible to attracting a greater or lesser number of actants,\footnote{In his work Tesni\`ere used the term \emph{actants} to refer to what we would call here the verb's \trsl{arguments.}} according to the number of bonds the verb has available to keep them as dependents. The number of bonds a verb has constitutes what we call the verb's valency.
\end{quotation}

More rigorous treatments\footnote{\posscite{tesniere1959} definition of valency as \trsl{nombre d'\emph{actants} qu'un verbe est susceptible de régir} (\trsl{number of \emph{actants} which a verb is capable of governing}) essentially frames valency as a function of the verb. More recent definitions however consider valency not just as a property of verbs alone but of any lexical item (cf., e.g., \cite{matthews1997,trask1993}). In addition, in his glossary, he has provided voice (Fr. \emph{voix}) as a synonym for valency; these two terms however we consider as distinct items both in this work and in what I think is the usage of both terms in scholarly literature over the topic.} have of course been published in the years since but we should content ourselves with this definition in our present treatment of Iridian grammar. Instead our primary focus would e

\subsection{Avalent verbs}

Avalent verbs \index{avalent verb} are verbs that have zero core arguments. In Iridian they are limited to a small set of verbs that describe meteorological phenomena, traditionally referred to as `weather verbs' (\ird{plodní sládek}) \index{weather verb}.This term is not wholly accurate, however, as the class includes not just meteorological phenomena but more general natural phenomena as well. When used this way they are marked in the agentive voice\index{agentive voice} and essentially forms topicless sentences\index{topicless sentence} (cf.~\S\,\ref{sec:topicless}). Some common weather verbs in Iridian are listed below.

\pex\deftagex{exw}
\irdp{hravá}{to have the sun shine}\\
\irdp{žužá}{to snow}\\
\irdp{pozběšá}{to rain}\\
\irdp{néšá}{to rain lightly, to drizzle}\\
\irdp{boboržá}{to have thunder}\\
\irdp{kopriká}{to have lightning}\\
\irdp{dozbuhá}{to have an earthquake}
\xe



\subsection{Passive constructions}


\subsection{Causative constructions}\index{causative}

Causatives may either be lexical, analytical or morphological. Lexical causatives involve the encoding of the causation on the verb itself leading the causative form of the verb to be a different form altogether. An analytical causative, on the other hand uses a different verb (usually a verb like \emph{to do} or \emph{to make}) in conjunction with the main verb, to express the idea of causation (e.g., English\index{English} \trsl{make someone do something.}) Finally, morphological causatives involve morphologically changing the main verb to express the notion of causation. Iridian causative constructions are primarily morphological, formed using the prefix \ird{ne-}.

\begin{table}
\footnotesize\sffamily
\caption{Causative forms of the verb \irdp{shradá}{to die.}}
\medskip
	\label{tbl:causative}
    \begin{tabu}to \textwidth{@{}Y[0.2]Y[0.5]YY@{}}
        \toprule
		 		{\sc voice}& {\sc causative } &{\sc regular meaning} & {\sc causative meaning}\\
		\midrule
				Inf.				& neshradá									& to die, to be dead 	& \emph{(defective)} \\ 
		 		Agt.				& {neshrážá}			& to kill & to cause someone to kill\\ 
		 		Pat.			& {neshradiná}					& to be killed & to be caused to be killed\\
				Ben.			& {neshradébá}				& to have someone die	& to have someone be killed\\
				Loc.				& {neshradouná}					& to have someone related die&\emph{(defective)}\\
				Ins.		& {doneshradouná}&to be the reason for dying&to be used for killing\\
				Refl.				& {uneshražá}&to kill oneself&to cause one to kill oneself\\
		 		
				\bottomrule

    \end{tabu}

\end{table}

Due to this suppletive nature, lexical causatives imply a more direct causation, or a tighter link between cause and event\footnote{\textcite{haiman1983} offers a thorough discussion of how the linguistic distance exhibited by the forms of causative constructions existing in a language (e.g., \emph{to cause to die} on one end of the spectrum versus \emph{to kill} on the other) correspond to the conceptual distance between the action of the causer and the result of the action to the causee. In a purely synthetic construction like \emph{kill}, for example, where the linguistic distance is the least, the conceptual distance between the action and the resulting state is also the smallest, with the opposite being true in purely analytical constructions like \emph{to cause to die}.}, than analytical or morphological causatives (\cite{velupillai2012, haiman1983}). Consider for example the three sentences in English\index{English} below:


\pex
\a Joseph \emph{died}.\deftagex{caus}
\a Joseph \emph{killed} the man.\deftagex{caus}\deftaglabel{kill}
\a Joseph \emph{made} the man \emph{die.}\deftagex{caus}\deftaglabel{made}
\xe

The suppletive \emph{kill} in example (\getfullref{caus.kill}) implies more agency on the part of the subject than the more indirect-sounding (\getfullref{caus.made}). In (\getfullref{caus.kill}) the \emph{death} of the patient (\trsl{the man}) is the goal of the act while (\getfullref{caus.made}) it might be inferred that the \emph{dying} was an indirect consequence of an unmentioned second act.


Iridian does not employ lexical causatives as in English\index{English}; instead causatives are formed morphologically by adding the prefix \ird{ne-} (glossed as \Caus{}) to the verb stem. Although \ird{ne-} is required to form the causative morphologically, some verbs, particularly stative verbs like \irdp{shradá}{to die, to be dead} in table \ref{tbl:causative} may already contain the notion of causation in some of its regular conjugated forms. This is because by default stative verbs\index{stative verb} are intransitive (i.e., the only argument required is the actor/agent\index{agent}) while some verbal voices\index{voice} like the patientive\index{patientive voice} and benefactive\index{benefactive voice} inherently imply the existence of a second and a third argument of a verb\index{argument of a verb} respectively.

%% TODO add section reference

Of course Iridian's definition of which verbs are stative and which ones are dynamic\index{dynamic verb} does not neatly align with the definition those classes have in English\index{English} (v. \S\,\ref{sec:statives}). For instance the verbs \emph{to stand} and \emph{to eat} are both dynamic verbs in English\index{English}, while in Iridian \irdp{zdavá}{to stand, to be standing} is stative and only \irdp{piaštá}{to eat} is dynamic. This is why as we see in example (\getfullref{statdyn.1}) below, some forms of the verb \ird{zdavá} already contain the notion of causation in some of its regular conjugated forms.

\pex
\a  \irdp{zdavá}{to be standing}\deftagex{statdyn}\deftaglabel{1}\\
	\irdp{zdavžá}{to stand}\\
    \irdp{zdavná}{to be made standing, to erect}\\
    \irdp{nezdavžá}{to make so./sth. stand}\\
    \irdp{nezdavná}{to be made to make so./sth. standing}
\a  \irdp{piaštá}{to eat}\\
    \irdp{piaštiná}{to be eaten}\\
    \irdp{nepiaščá}{to make someone eat}
\xe

Since causative constructions in Iridian are purely morphological\footnote{To contrast, consider Japanese\index{Japanese} which also forms causative constructions morphologically (using the suffix \emph{-(sa)se}) but which in addition also has synthetic but not fully suppletive forms for some verbs (e.g., \irdp{agaru}{to rise} and \irdp{ageru}{to raise}).} the degree of agency of the causer can be implied from other incidental properties of the verb such as aspect or voice markings.

We pay particular attention first on the interaction of the causative prefix \ird{ne-} with the patientive voice marker \ird{-in} and the benefactive voice marker \ird{-éb}. We begin with stative verbs, since as mentioned earlier and in \S\,XX, most stative verbs will have a causative reading when used with the agentive or benefactive voice. Stative verbs encode the state of the subject and cannot therefore express the idea of an agent nor that of a patient. By conjugating stative verbs for voice, their stative nature is therefore lost; that is why a causative cannot be derived from the unmarked form of a stative verb: a causative construction precludes the existence of a causer and a causee, which at times may be different from the subject, while the unmarked stative only that of the subject itself.


\begin{figure}[H]
	{
	\footnotesize
  \begin{forest}
    [\irdp{shradá}{to die},
		[\ird{shradiná}\\
			patientive\\
			{Arg = 1}
				[$
				\begin{bmatrix}
					\textbf{+ Patient}
				\end{bmatrix}
				$]
				]
      [\ird{shražá}\\
				agentive\\
				{Arg = 2}
					[$
					\begin{bmatrix}
						\textrm{+ Patient}\\
						\textbf{+ Agent}
					\end{bmatrix}
					$]
					]
					[\ird{ushražá}\\
						reflexive\\
						{Arg = 2}
							[$
							\begin{bmatrix}
								\textrm{+ Patient}\\
								\textbf{+ Agent}
							\end{bmatrix}
							$]
							]
			[\ird{shradébá}\\
				benefactive\\
				{Arg = 3}
				[$
				\begin{bmatrix}
					\textrm{+ Agent}\\
					\textrm{+ Patient}\\
					\textbf{+ Benefactor}
				\end{bmatrix}
				$]
			]
		]
  \end{forest}

	}\caption[Voice markings as valence operations in stative verbs.]{Voice markings as valence operations in stative verbs. The number of elements includes all those required to create a well-formed sentence notwithstanding Iridian's tendency to drop elements that can be implied from context, with the element in bold representing whichever element is most likely to surface in speech.}
  \label{causative-reading}
\end{figure}

We see in figure \ref{causative-reading} that this causative reading of the patientive voice with stative verbs is due to properties of stative verbs and not of the patientive voice. We know this is true since this causative reading of the patientive does not exist with non-stative verbs, which are transitive by default in Iridian.

\pex
\a
\begingl
    \gla \ljudge{*}Mámka prehlavnik.//
    \glb mother buy-\Pv{}-\Pf{}//
    \glft \trsl{*I bought my mother.}//
\endgl
\a
\begingl
    \gla Mámka zuštalnik.//
    \glb mother happy-\Pv{}-\Pf{}//
    \glft \trsl{I made my mother happy.}//
\endgl
\xe

The patientive voice only requires a patient as argument; however since this argument does not exist in stative constructions, the role of an agent must first be created for the subject of the stative construction to be able to occupy the role of the patient in the patientive voice. Essentially this means that conjugating a stative verb for the patientive voice is equivalent to creating a biclausal causative construction where the subject becomes the causee and the state the action brought about by the (optionally named) causer. This reading is not possible with dynamic verbs because the patientive voice would only shift the role of the patient to that of the topic without having to create a new role for an agent.

As could have been predicted from \posscite{haiman1983} theory, these indirect forms of the causative express a more direct link between the causer and the action. True morphological causatives, i.e., those formed using the prefix \ird{ne-}, imply that the caused action was brought about by an intermediary.

\begin{multicols}{2}
\pex
\a
\begingl
\gla Váz nopriznek.//
\glb vase break-\Pv{}-\Pf{}//
\glft \trsl{I broke the vase.} (on purpose)//
\endgl
\a
\begingl
\gla Váz nenopriznek.//
\glb vase \Caus{}-break-\Pv{}-\Pf{}//
\glft \trsl{I made someone break the vase.}//
\endgl
\xe
\end{multicols}

If the intermediary appears in the sentence it can be marked either in the genitive or in the patientive. Marking the causee in the genitive is the \trsl{neutral} configuration; using the patientive case on the other hand forms what can be called a \emph{coercive} causative (\cite{shibatani1990,lehmann2006}), which in Iridian\footnote{We can compare this to a similar distinction between a dative causative (formed with the clitic \emph{ni}) and the accusative causative (formed with \emph{o}) in Japanese\index{Japanese}. \textcite{lehmann2006} calls the former a coercive causative construction while the latter a permissive causative construction. There are two main differences between the Japanese\index{Japanese} and Iridian systems however. First the coercive causative in Iridian also implies that the agent has effective control over the action or the causee or both, something not necessarily expressed by the Japanese\index{Japanese} \emph{o}-form; and second, both the patientive and the genitive forms of the causative in Iridian allow `permissive' readings, as we illustrate later in this section.

More importantly however the genitive form is considered the default or neutral form in Iridian, with the patientive form considered as more `marked.' The patientive is often used for emphasis, with the genitive construction replacing it where possible, especially in spoken Iridian, even in places where the use of the patientve would have been in better order.
}
could imply either of two things: (i) that the act was made without or against the consent of the causee or (ii) the causer had direct control over the action and/or the causee. Such distinction however is not possible if the main verb is in the agentive voice since the patientive marking is reserved for the patient of the verb (and thus marking the causee in the patientive will essentially produce a situation where both the agent and the patient of the verb is marked for the same role, which in this case is the patient.)

\pex
\a
\begingl
\gla Váz Janc\v{e} nenopriznek.//
\glb vase Janek-\Gen{} \Caus{}-break-\Pv{}-\Pf{}//
\glft \trsl{(I) made John break the vase.}//
\endgl
\a
\begingl
\gla Váz Janka nenopriznek.//
\glb vase Janek-\Acc{} \Caus{}-break-\Pv{}-\Pf{}//
\glft \trsl{(I) made John break the vase.}//
\endgl
\xe


Nevertheless the degree of control exerted by the causer over the action itself may vary between these constructions.

A common way to formally mark the causer's control or lack thereof in Iridian is the opposition between the retrospective aspect and the perfective aspect. Consider for example the two sentences in Iridian below, both of which have the same general translation in English\index{English}.

\pex
\a
\begingl
	\gla Martin nésta najev\v{e}c shražek.//
	\glb Martin deer-\Acc{} drive-\Cv{} die-\Av{}-\Pf{}//
	\glft \trsl{Martin ran over a deer.} (He did it on purpose)//
\endgl
\a
\begingl
	\gla Martin nésta najev\v{e}c shražaní.//
	\glb Martin deer-\Acc{} drive-\Cv{} die-\mk{av-ret}//
	\glft \trsl{Martin ran over a deer.} (It was an accident.)//
\endgl
\xe

\subsection{Reflexive, reciprocal, and sociative constructions}\index{reflexive construction}\index{reflexive voice}\index{reciprocal construction}

The reciprocative prefix \ird{so-} (glossed \Rec{}) is used with the agentive voice to indicate that an action is performed by the agent and the patient on each other.

\pex
\begingl
\gla Karlu sodalšime še Marek ščenžek.//
\glb Karel-\Ins{} \Rec{}-talk-\mk{av-prog} with Marek arrive-\Av{}-\Pf{}//
\glft \trsl{Karel (and I) were talking when Marek arrived.}//
\endgl
\xe

The use of the reciprocative inherently implies plurality on the part of the subject since there are always at least two elements involved (cf. \cite[255]{tesniere1965}). Since Iridian does not often grammaticalise plurality\index{plural}, this means the reciprocative usually won't require additional consideration as to the agreement of the constituents of the sentence; it does, however, mean that this form cannot be used singly with the singular form of pronouns (since pronouns---at least in the first and second persons---formally distinguish between singular and plural) and that most countable nouns would require the use of the particle \ird{ně} or an explicit quantifier.

\pex
\begingl
\gla Na to hruma hurka sokonížek.//
\glb \Loc{} \Dem{} church-\Acc{} parents \Rec{}-wed-\Av{}-\Pf{}//
\glft \trsl{(My) parents were married in this church.}//
\endgl
\xe

\pex
\begingl
\gla N\v{e} senátor sožubalžimej\'i to-\v{z}e na televiza vednik.//
\glb \Pl{}= senator \Rec{}-shout-\Av{}-\Prog{}-\Quot{} \Qp{} \Loc{} television-\Acc{} see-\Pv{}-\Pf{}//
\glft \trsl{(I) saw the senators shouting at each other on TV.}//
\endgl
\xe

Where both elements of the agent-patient pair are present in the sentence, one of them is treated as the agent and left unmarked while the other is marked in the comitative\index{comitative} (i.e., \ird{še} + instrumental). However, since the action itself is reciprocal, which gets marked as the agent is purely a pragmatic choice. Where one of the members of the agent-patient pair is a pronoun, preference is given to marking the pronoun as the agent (in which case \ird{še} is normally ommitted, but with the patient remaining in the instrumental case).

\pex
\begingl
\gla Mišek še Martinu sohévoržev\'i.//
\glb Mišek \Com{} Martin-\Ins{} \Rec{}-know-\Av{}-\Cont{}//
\glft \trsl{Mišek and Martin know each other.}//
\endgl
\xe

\pex
\begingl
\gla No já Mišku sohévoržaní?//
\glb \Q{} \mk{2s.str} Mišek-\Ins{} \Rec{}-know-\Av{}-\Ret{}//
\glft \trsl{You and Mišek already met each other right?}//
\endgl
\xe

Sociative verbs\index{sociative verb}, on the other hand, which are formed with the prefix \ird{se-} (and glossed here as \Soc{}), indicate actions that are performed with another person or other people. These may also represent joining or participating in actions that have been started by somebody. The prefix \ird{se-} is realised as \ird{s-} except before sibilants and the approximant /j/. Sociative verbs are formed with the locative voice, with the noun phrase in the topic representing the person/people with whom an action is done. Note that sociative verb constructions cannot be used with inanimate/non-human noun phrases.

\pex
\begingl
\gla N\v{e} Marek bych sezdravounek.//
\glb \Pl{}= Marek yesterday \Soc{}-sleep-\Lv{}-\Pf{}//
\glft \trsl{I slept at Marek's place yesterday.}//
\endgl
\xe

\pex
\begingl
\gla Kazn\'i hezka lin\v{e} kvu\v{s}tnan\'i \v{s}e m\'e skaznounek.//
\glb song-\Gen{} first line hear-\Pv{}-\Ret{} \Com{} \First\Pl{}.\Str{} \Soc{}-sing-\Lv{}-\Pf{}//
\glft \trsl{Upon hearing the first lines of the song, (he) joined us in singing.}//
\endgl
\xe

\section{Grammatical mood}\index{mood}\index{grammatical mood|see{mood}}

\subsection{Indicative}

\subsection{Imperative and hortative mood}\label{sec:imp-hort}\index{imperative}\index{hortative}

To form commands\index{commands} and requests\index{requests}, the imperative (glossed \mk{imp}) and hortative (\mk{hort}) moods are used in Iridian.

The imperative is formed by replacing the infinitive ending \ird{-á} with the voice marker and the imperative ending \ird{-ím}. The imperative\index{imperative mood} cannot be negated with the prefix \ird{zá-}; instead, to form a negative command the prohibitive\index{prohibitive mood} mood is used (glossed \mk{proh}), formed with the suffix \ird{-éma} instead of \ird{-ím}.

\begin{table}[ht!]
\sffamily\footnotesize
	\caption{Conjugation of the verb \ird{piaštá}\\ in the imperative and probihibitive moods.}
	\label{tbl:imperative}
\medskip
    \begin{tabu}to 0.7\textwidth{@{}YYY@{}}
         \toprule

         {\sc voice}&{\sc imperative}&{\sc prohibitive}  \\
         \midrule

         Agentive &
         {piaščím} &
         {piaščéma}\\

         Patientive &
         {piaštním} &
         {piaštnéma}\\

         Benefactive &
         {piaštébím} &
         {piaštébíma}\\

         Locative &
         {piaštouním} &
         {piaštounéma}\\

         Instrumental &
         {dopiaštouním} &
         {dopiaštounima}\\

         Reflexive &
         {upiaščím} &
         {upiaščéma}\\

         \bottomrule
    \end{tabu}

\end{table}

The imperative\index{imperative mood} is used to issue a direct command and the prohibitive to ``signal a prohibition\index{prohibitive mood}'' (SIL). Verbs in the imperative mood do not require an explicit referent, with the addressee or addressees assumed to be the recipient of the command or prohibition. When the addressee is included, it appears in the vocative case if appearing before the verb or unmarked otherwise.\footnote{A comma is placed between the verb and the addressee if the addressee appears after the verb in the sentence but none if it appears before.} Note that both the imperative and the prohibitive do not distinguish number; thus the same form of the verb will be used when giving a command to multiple addressees and to a single one.

\pex
\begingl
    \gla To hrabním.//
    \glb \Dem{} listen-\mk{pv-imp}//
    \glft \trsl{Listen to this.}//
\endgl
\xe
\pex
\a
\begingl
    \gla To hrabním, Marek.//
    \glb \Dem{} listen-\mk{pv-imp} Marek//
    \glft \trsl{Listen to this, Marek.}//
\endgl
\a
\begingl
    \gla Marku to hrabním.//
    \glb Marek-\mk{voc} \Dem{} listen-\mk{pv-imp}//
    \glft \trsl{Listen to this, Marek.}//
\endgl
\xe

\pex
\begingl
    \gla Papír švirkounéma.//
    \glb paper write-\mk{lv-proh}//
    \glft \trsl{Do not write anything on this sheet of paper.}//
\endgl
\xe

When used with verbal adjectives, the suffixes can attach directly to the root without any need for an explicit marker for voice and the addition of a voice marker will in fact change the meaning of the sentence. (The first two sentences below are rather unhelpful given how morphophonemic changes has rendered the imperative form with the voice marker and the one without of the verb \irdp{slouhatá}{to be quiet} identical, but cases like this are common and merit attention.)

\pex
\a
\begingl
    \gla Nie byló slouháčím.//
    \glb \Pl{}= child be:quiet-\mk{imp}//
    \glft \trsl{Keep quiet, children.}//
\endgl
\a
\begingl
    \gla Nie byló uslouháčím.//
    \glb \Pl{}= child \Refl{}-be:quiet-\mk{av-imp}//
    \glft \trsl{Keep quiet, children.}//
\endgl
\xe

\pex
\a
\begingl
    \gla Pitár zuštalébím.//
    \glb Pitár be:happy-\mk{ben-imp}//
    \glft \trsl{Make Pitár happy!}//
\endgl
\a
\begingl
    \gla Zuštalím.//
    \glb be:happy-\mk{imp}//
    \glft \trsl{Be happy!}//
\endgl
\xe


Due to its directness, the use of the imperative or the prohibitive is
considered impolite in most settings, and is often used only when speaking with
friends, family or children. This distinction does not exist in the written
language, where the imperative is used almost exclusively for these functions.
However in signs that give orders or warnings (i.e., `Stop,' `Do not enter')
where English\index{English} may sometimes use imperative constructions, Iridian uses modal
constructions\index{modality} (cf. \S\,\ref{sec:modality}) as they are not treated
 as direct commands or prohibitions.

\pex
\begingl
    \gla Tak slouhatalneví.//
    \glb here be:quiet-\mk{deb-cont}//
    \glft \trsl{Keep quiet.} \textit{Lit.,} \trsl{One must be quiet here.}//
\endgl
\xe

\pex
\begingl
    \gla Tak zahranéčneví.//
    \glb here enter-\mk{npot-cont}//
    \glft \trsl{Do not enter.} \textit{Lit.,} \trsl{One cannot enter here.}//
\endgl
\xe

In spoken Iridian, it is more common and considered more polite to use the
hortative and the negative hortative forms instead of the direct imperative
or prohibitive.

\begin{table}[ht!]
    \footnotesize\sffamily
		\caption{Conjugation of the verb \ird{piaštá} in the hortative mood.}
		\label{tbl:hortative}
		\medskip
    \begin{tabu}to 0.7\textwidth{@{}YYY@{}}
         \toprule

         &{\sc hortative}&{\sc neg. hortative}  \\
         \midrule
         Agentive &
         {piaščká} &
         {piaščku}\\

         Patientive &
         {piaštniká} &
         {piaštniku}\\

         Benefactive &
         {piaštébká} &
         {piaštébku}\\

         Locative &
         {piaštómká} &
         {piaštómku}\\

         Instrumental &
         {dopiaštómká} &
         {dopiaštómku}\\

         Reflexive &
         {upiaščká} &
         {upiaščku}\\

         \bottomrule
    \end{tabu}

\end{table}

\pex
\begingl
\gla Mina návilastnika.//
\glb door open-\mk{pv-hort}//
\glft \trsl{Open the door.}//
\endgl
\xe

To further soften command, the expression \ird{am luhninká} (from the hortative
form of the verb \irdp{luhná}{to give thanks}, now obsolete except for this
specific usage) and its equivalent negative form \ird{am luhninku} can be used,
with the main verb marked as a perfective converb.\index{converb}\footnote{Cf. the use of the perfective converb with the \textit{merci de} + infinitive construction in French\index{French}. The use of \ird{am luhninká} presupposes that the action being requested has already been done although in fact it hasn't, for which therefore the speaker is giving thanks. Thus, a simple request like \trsl{Please close the door} is expressed in Iridian as \trsl{May you be thanked for having closed the door.}}

\pex
\begingl
\gla Mina se návilastu am luhninka.//
\glb door \Refl{} open-\mk{cv.pf} because thank-\mk{pv-hort}//
\glft \trsl{Please open the door.}//
\endgl
\xe

The adhortative (\trsl{Let's}) is formed using \ird{lidovká} with the imperfective converb form of the main verb. \ird{Lidovká} can also be used by itself where the main verb may be implied from context, or as a reply to the request if the speaker wants to express agreement or assent.

\pex
\begingl
\gla Piaštiec lidovká.//
\glb eat-\mk{cv.ipf} because thank-\mk{pv-hort}//
\glft \trsl{Please open the door.}//
\endgl
\xe

\subsection{Subjunctive}

The subjunctive mood (glossed \mk{sbj}) is used for actions or events that are not or are not known to be true or factual. The subjunctive is formed using the suffix \ird{-íl}

\begin{table}[ht!]
	\footnotesize\sffamily
	\caption{Conjugation of the verb \ird{piaštá} in the subjunctive.}
	\medskip
	\begin{tabu}to 0.7\textwidth{@{}YYY@{}}
		\toprule
		&{\sc imperfective} &{\sc perfective}\\
		\midrule
		Agentive	& piaščílá	& piaščíš\\
		Patientive	& piaštnílá		& piaštníš\\
		Benefactive	& piaštébílá		& piaštebíš\\
		Locative	& piaštounílá		& piaštouníš\\
		Instrumental& dopiaštébílá	& dopiaštebíš\\
		Reflexive	& upiaščílá	& upiaščíš\\
		\bottomrule
	\end{tabu}
\end{table}

In addition, the copula has two subjunctive forms, the non-negative \ird{niec} and the negative \ird{vaše}.

Note that the Iridian subjunctive makes neither temporal nor aspectual distinction.

\par The following are some specific uses of the subjunctive mood in Iridian:

\subsubsection{Subjunctive of purpose}

Dependent clauses expressing purpose are marked in the subjunctive and normally end in \irdp{te}{in order to} and \irdp{az}{lest}

\pex
\begingl
\gla Traví prehlavnílá te traumašt stojnik.//
\glb bread-\Gen{} buy-\mk{pv-subj.ipf} {so:that} bakery go-\mk{lv-pf}//
\glft \trsl{(I) went to the bakery to buy some bread.}//
\endgl
\xe

\pex
\begingl
\gla Hreščílá te piaščeví.//
\glb be:alive-\mk{av-subj.ipf} {so:that} eat-\mk{lv-cont}//
\glft \trsl{We eat to live.}//
\endgl
\xe

\pex
\a
\begingl
\gla Se vdinílá az varšek.//
\glb \Refl{} see-\mk{pv-subj.ipf} {lest} leave-\Av{}-\Pf{}//
\glft \trsl{(I) left so as not to be seen.}//
\endgl
\a
\begingl
\gla Vdinílá az varšek.//
\glb see-\mk{pv-subj.ipf} {lest} leave-\Av{}-\Pf{}//
\glft \trsl{(I) left so that (it) may not be seen.}//
\endgl
\xe



\subsubsection{jussive/desiderative}
\par The subjunctive is used in indirect constructions of verbs for issuing orders, commanding, exhorting, etc.
\pex
\begingl
\gla Martin na America žnožíl to čeznašálic.//
\glb Martin \Loc{} America-\Acc{} study-\mk{av-sbj} \mk{rz} want-\mk{av-cont-3s.anim}//
\glft `He wants Martin to study in America.'//
\endgl
\xe

\pex
\begingl
\gla Beatles-že >>Yesterday<< Mark\k{a} zášníl to Tunek dálek.//
\glb Beatles-\Gen{} ``Yesterday'' Marek-\Agt{} sing-\mk{pv-sbj} \mk{rz} Tunek say-\mk{pf}//
\glft `Tunek told Marek to sing.'//
\endgl
\xe

\subsubsection{dubitative}
\par The subjunctive is used with verbs expressing doubt, uncertainty or disbelief.

\pex
\begingl
\gla še //
\glb Beatles-\Gen{} ``Yesterday'' Marek-\Agt{} sing-\mk{sbj} \mk{rz} Tunek say-\mk{pf}//
\glft `Tunek told Marek to sing.'//
\endgl
\xe

\subsubsection{with verbs expressing emotion}

\pex
\begingl
\gla Marek zašníl to Tunek dálek.//
\glb Marek sing-\mk{sbj.ipf} \mk{rz} Tunek say-\mk{pf}//
\glft `Tunek told Marek to sing.'//
\endgl
\xe


\subsubsection{with the conditional mood}
\par The subjunctive is used in the main clause if the verb in the dependent clause is in the conditional \textit{irrealis} mood.

\pex
\begingl
\gla Dá prezident jenem, //
\glb a//
\glft a//
\endgl
\xe

\subsubsection{expressing judgment}

\pex
\begingl
\gla Zavnočilaš to tévét //
\glb respond-\mk{av-sbj.ipf-2s} \mk{rz} important//
\glft \trsl{It is important that you respond.}//
\endgl
\xe

\subsubsection{irrealis}

\subsection{Conditional Mood}\index{conditional mood}\label{sec:conditional}
\par The conditional mood is used for conditional or hypothetical clauses. The table below shows the conjugation paradigm for the conditional mood for both regular verbs and the copula. The Iridian conditional mood is not a true conditional mood grammatically, since it is marked on the verb in the dependent clause (protasis), instead of the main clause.

\begin{table}[ht!]
	\footnotesize\sffamily
	\caption{Conjugation paradigm in the conditional mood for regular \\verbs, the copula and the existential particle \ird{ješ}.}\medskip
	\begin{tabu} to 0.8 \textwidth	{@{}Y[1.1]Y[1.3]YY@{}}
		\toprule
		&{\sc regular verbs} & {\sc copula} & {\sc existential}\\
		\midrule

		\textit{Realis} 				&{-ič} &víne & jako\\
		Neg. \textit{Realis}		&{-čn\v{e}}&ve&neko\\

		\textit{Irrealis} 			& {-išče}& jenem & jenem\\
		Neg. \textit{Irrealis} 	& {-iščen\v{e}}& jet & nét\\
		\bottomrule
	\end{tabu}
\end{table}

\subsubsection{Conditional Realis}

\par The conditional \textit{realis} mood (glossed \mk{cond.rl}) is used in two ways:
\begin{enumerate}
	\item In sentences that express a factual implication rather than a hypothetical situation or a potential future event, e.g., `If you heat water to 100 C, it will boil.'
	\item In `predictive' constructions, i.e., those that concern probable future events.
\end{enumerate}

The conditional \emph{realis} mood requires the verb in the main clause to be in the indicative.

\subsubsection{Conditional Irrealis}
The conditional \textit{irrealis} mood (glossed \mk{cond.irr}) is used with hypothetical, typically counterfactual, events. The \emph{irrealis} mood requires the main clause to be in the subjunctive.

\section{Evidentiality}\label{sec:quotative}\index{quotative}\index{evidentiality}

Iridian marks {\scshape evidentiality} as a separate grammatical category, distinguishing between a marked {\scshape quotative} or {\scshape reportative} representing secondhand information or hearsay (or more idiomatically when the speaker wishes distance themself from the statement by saying that the information is not coming directly from them) and an unmarked form representing `everything else' (cf. \cite[31-33]{aikhenvald2004}). The quotative form of a finite verb (and of some non-finite verb forms) is seen in Table \ref{tbl:quotative}. The syntax of quotative constructions is discussed in detail in \S\,\ref{sec:quotative-const}.


\begin{table}
\footnotesize \sffamily
	\caption{Sound changes used in deriving quotative form of verbs}
	\medskip
	\label{tbl:quotative}
	\begin{tabu} to \textwidth {@{}Y[0.9]Y[0.7]Y@{}}
		\toprule
		{\sc verbal form}			&	{\sc sound change}				& {\sc example}\\
		\midrule
			\multicolumn{3}{@{}l}{{\sc indicative}}\\
				\quad Perfective 		&
				\ird{-ek} $\rightarrow$ \ird{ice}	&
				\ird{piašček} $\rightarrow$ \ird{piaščice}\\
				\quad Retrospective &
				\ird{-aní} $\rightarrow$ \ird{án\v{e}} &
				\ird{piaščaní} $\rightarrow$ \ird{piaščánie}\\
				\quad Continuous &
				\ird{-eví} $\rightarrow$ \ird{ev\'ije} &
				\ird{piaščeví} $\rightarrow$ \ird{piaščev\'ije}\\
				\quad Progressive &
				\ird{-ime} $\rightarrow$ \ird{imej\'i} &
				\ird{piaščime} $\rightarrow$ \ird{piaščimej\'i}\\
				\quad Contemplative &
				\ird{-ách} $\rightarrow$ \ird{áže} &
				\ird{piaščách} $\rightarrow$ \ird{piaščáže}\\
				\quad Prospective &
				\ird{-ujám} $\rightarrow$ \ird{-ujime} &
				\ird{piaščujám} $\rightarrow$ \ird{piaščujime}\\
				\quad Cessative &
				\ird{-óvít} $\rightarrow$ \ird{-óvíce} &
				\ird{piaščóvít} $\rightarrow$ \ird{piaščóvíce}\\
			\multicolumn{3}{@{}l}{{\sc subjunctive}}\\
				\quad Imperfective &
				\ird{-\'il\'a} $\rightarrow$ \ird{-el\v{e}} &
				\ird{piaščóvít} $\rightarrow$ \ird{piaščóvíce}\\
				\quad Perfective &
				\ird{-i\v{s}} $\rightarrow$ \ird{-i\v{s}ej\'i} &
				\ird{piaščóvít} $\rightarrow$ \ird{piaščóvíce}\\
			\multicolumn{3}{@{}l}{{\sc imperative, \&c.}}\\
				\quad Imperative &
				\ird{-\'im} $\rightarrow$ \ird{-\'imen\'i} &
				\ird{piaščóvít} $\rightarrow$ \ird{piaščóvíce}\\
				\quad Prohibitive &
				\ird{-\'ema} $\rightarrow$ \ird{-\'emn\v{e}} &
				\ird{piaščóvít} $\rightarrow$ \ird{piaščóvíce}\\
				\quad Hortative &
				\ird{-k\'a} $\rightarrow$ \ird{-kaje} &
				\ird{piaščóvít} $\rightarrow$ \ird{piaščóvíce}\\
				\quad Neg. Hortative &
				\ird{-ku} $\rightarrow$ \ird{-kajen\'i} &
				\ird{piaščóvít} $\rightarrow$ \ird{piaščóvíce}\\
			\multicolumn{3}{@{}l}{{\sc other forms}}\\
				\quad Supine of purpose &
				\ird{-it} $\rightarrow$ \ird{-itej\'i} &
				\ird{piaščóvít} $\rightarrow$ \ird{piaščóvíce}\\
				\quad Supine of necessity &
				\ird{-\'a\v{s}} $\rightarrow$ \ird{-\'a\v{s}e} &
				\ird{piaščóvít} $\rightarrow$ \ird{piaščóvíce}\\
				\quad Nominalised form&
				\ird{-ou} $\rightarrow$ \ird{-uje} &
				\ird{piaščóvít} $\rightarrow$ \ird{piaščóvíce}\\
			\bottomrule
	\end{tabu}

\end{table}

\subsubsection{Quotative forms of the copula}

%% TODO format as table
Copula
Indicative
neví
hvem
Subj
nehlí
niec

Existential
Indicative
jeho
nežní
Subj
houve
hvaš


\section{Modality}\index{modality}\label{sec:modality}

Iridian can express modality either through verbal morphology, using the affixes listed in table \ref{tbl:modality}, or through a periphrastic construction. In general a periphrastic construction is preferred when the verb is non-dynamic, i.e., the sentence is merely descriptive or stative in nature (compare, for example English\index{English} \trsl{Mary can sing} vs. \trsl{Mary was able to finish baking the cake}), while the morphological method is used otherwise.

\begin{table}[ht!]
    \footnotesize\sffamily
    \caption{Verbal affixes to express modality.}
    \medskip
    \label{tbl:modality}
    \begin{tabu}to 0.6\textwidth{@{}YYY@{}}
			\toprule
				 {\sc modality} & {\sc positive} & {\sc negative}\\
				 \midrule
         Debitive & {-aln-} & {-išk-}\\
         Desiderative & {-án-}&{-ušh-}\\
         Potential &{-ét-} & {-évn-}\\
			\bottomrule
    \end{tabu}
\end{table}

The affixes used to mark modality as listed in table \ref{tbl:modality} attach directly to the verb stem, subject to the usual morphophonemic rules.

\pex\a \irdp{piaštá}{to eat}
\a \irdp{piaštalná}{to need to eat}
\a \irdp{piaštišká}{to not need to eat}
\a \irdp{piaštáná}{to want to eat}
\a \irdp{piaštušhá}{to not want to eat}
\a \irdp{piaštétá}{to be able to eat}
\a \irdp{piaštévná}{to not be able to eat}
\xe

As in most languages, modal constructions in Iridian exhibit significant {\scshape polysemy}\index{polysemy} (i.e. a single construction can have one or more interpretation depending on the context). For example consider the following sentence:

\pex
\begingl
\gla Tomáš rušku zahviržétách.//
\glb Tomáš Russian-\Ins{} speak-\Av{}-\Pot{}-\Ctp{}//
\glft \trsl{Tomáš will be able to speak Russian}//
\endgl
\xe

The following translations are all equally possible without any further contextual clues:

\pex
\a \trsl{Tomáš will be able to speak Russian, if he will study it.} (abilitative)
\a \trsl{Tomáš will be able to speak Russian because he will be allowed to do it.} (permissive)
\a \trsl{Tomáš can speak Russian and he will probably speak it later.} (true potential modality)
\xe

\subsection{Potential modality}\index{potential modality}\index{abilitative}\index{permissive}\index{modality}

Potential modality (glossed as \mk{pot}) is used when, in the speaker's opinion, an event is possible to occur. This definition makes the potential mood in Iridian encompass both the expressions of ability and permissibility.

\pex
\begingl
\gla To švirek moc gruševí še oštinévnílá.//
\glb this handwriting too be:small-\mk{cont} with read-\mk{pv-npot-sbj.ipf}//
\glft \trsl{The handwriting is too small (I) am unable to read it.}//
\endgl
\xe

\subsection{Debitive modality}\index{debitive modality}\index{modality}

The debitive form of a verb expresses necessity. This verb form is now mainly confined in literary usage, and has been entirely replaced in colloquial Iridian by the supine of necessity.\index{supine} The negative debitive form however has survived and is still in common use. The negated form\index{negation} of the positive debitive (in contrast to the negative form) is also no longer used in colloquial Iridian, and the negative debitive coexists instead with the negated form of the supine of necessity, with subtle differences in meaning.

\begin{multicols}{2}
  \pex
  \a
  \begingl
  \gla Tóm zoštináš.//
  \glb book \Neg{}-read-\SupN{}//
  \glft \trsl{(We) don't need to read the book.}//
  \endgl
  \a \begingl
  \gla Tóm oštniškeví.//
  \glb book read-\Pv{}-\N{}\Deb{}-\Cont{}//
  \glft \trsl{The book should not be read (i.e., it is prohibited).}//
  \endgl
  \xe
\end{multicols}

The negative debitive form is particularly common in written warnings/prohibitions.
\pex \begingl
  \gla Ran\v{z}i\v{s}kev\'i.//
  \glb smoke-\Av{}-\N{}\Deb{}-\Cont{}//
  \glft \trsl{No smoking.}//
  \endgl
\xe

\subsection{Periphrastic constructions}

\section{Non-finite verb forms}

\subsection{Infinitive}\label{sec:infinitive}\index{infinitive}

The {\scshape infinitive} is a fossilised verb form that was used in Old Iridian\index{Old Iridian} (and arguably in Early Middle Iridian\index{Middle Iridian}) as a verbal noun occupying the topic\index{topic} position in a sentence. In Modern Iridian this use has been completely supplanted by the gerund\index{gerund} and the infinitive is only used as the citation form\index{citation form} of verbs. All infinitives in Iridian end in the vowel \ird{-á} and the consonant immediately preceding it is called the verb's thematic consonant.\index{thematic consonant}.

\subsection{Nominalised forms and gerunds}\index{gerund}\index{nominalisation}\label{nom-morph}

Nouns can be routinely derived from verbs and verb phrases using the nominalising suffix \ird{-ou} (glossed as \Nz{}). Linguists generally recognise three types of nominalisation: event nominals, which describe an event the same way the parent verb does, and which could either be (1) simple or (2) complex, with {\scshape complex event nominals} ({\sc cen}s)\index{nominalisation!event nominal}\index{event nominal|see{nominalisation, event nominal}} allowing internal arguments and {\scshape simple event nominals} ({\sc sen}s)\index{nominalisation!event nominal} not; and (3) {\scshape resultant nominals}\index{nominalisation!resultant nominal}\index{resultant nominal|see{nominalisation, resultant nominal}}, which describe an event similar but not exactly corresponding to the even described by the main verb (\cite{grimshaw1990}; \cite{moulton2014}). In English\index{English}, for example, where verbs can be nominalised using a variety of derivational affixes or with zero derivation, these types are not distinguished, as we see below:

\pex[interpartskip=0pt]
	\a The examination of the students lasted a long time. \hfill {\sc cen}
	\a The examination lasted a long time.\hfill {\sc sen}
	\a The examination was photocopied on green paper.\hfill {\sc rn}\\
	\trailingcitation{(\cite[2]{alexiadou2008})}
\xe

Some verbs in Iridian allow the formation of {\sc rn}s using the suffix \ird{-ou} and the uninflected verb root (e.g., \irdp{piaštou}{food} from \irdp{piaštá}{to eat}). For the vast majority, however, {\sc rn}s are produced by lexical suppletion, i.e., the {\sc rn}s are not morphologically derived (or explicitly so, at least) using the nominalising suffix (see \S~\ref{sec:nomder-verb}). As in English, {\sc sen}s and {\sc cen}s are not morphologically distinct in Iridian, and are formed with the suffix \ird{-ou} used in conjunction with the prefix \ird{po(d)-}. We call this form the {\scshape gerund}.

In addition to these three types of nominalisation introduced in \textcite{grimshaw1990}, Iridian recognises a fourth type, which produces a nominal that refers not to the event itself but one of the event's participants, i.e., one of the verbs arguments. We will call this type a {\scshape participant nominal} ({\sc pn}) (cf. \cite[400-5]{schackow2015}; \cite[297-8]{okuna}).

\pex \a Nominalised forms of \irdp{piaštou}{to eat} showing a productive morphological {\sc rn}:\smallskip\\
		\vtop{\halign{%
			#\hfil& \qquad #\hfil\cr
			\quad Infinitive:		& \irdp{piaštá}{to eat}\cr
			\quad Morphological {\sc rn}:	& \irdp{piaštou}{food}\cr
			\quad Gerund ({\sc sen/cen}):	& \irdp{popiaštou}{the act of eating}\cr
			\quad {\sc pn}:				& \irdp{piaščkou}{the person/thing who ate}\cr
		}}
\a Nominalised forms of \irdp{vadá}{to think} showing a defective morphological {\sc rn} and the alternative lexical {\sc rn}:\smallskip\\
	\vtop{\halign{%
		#\hfil& \qquad #\hfil\cr
			\quad Infinitive:		& \irdp{vadá}{to think}\cr
			\quad Morphological {\sc rn}:	& \ird{*vadou} (ungrammatical)\cr
			\quad Lexical {\sc rn}:		& \irdp{vied}{thought (n.)}\cr
			\quad Gerund ({\sc sen/cen}):	& \irdp{povadou}{the act of thinking}\cr
			\quad {\sc pn}:				& \irdp{vadnikou}{that which was thought}\cr
}}\xe


Event nominals (viz., gerunds) are therefore inherently abstract and active in meaning; in addition, they are also understood to be tenseless and aspectless

Gerunds\index{gerund} have an active meaning. The suffix \ird{-ál}, used to mark the continuous aspect, may be infixed to the gerund to indicate that the action is repetitive.

\pex
\a
\begingl
\gla Jan nidek.//
\glb Jan stand.up-\mk{pf}//
\glft \trsl{Jan stood up.}//
\endgl
\a
\begingl
\gla Janí ponidálou buvec.//
\glb Jan-\Gen{} \mk{ger}-stand.up-\mk{cont-nz} annoying//
\glft \trsl{Jan's standing up again and again is annoying.}//
\endgl
\xe

The syntax of event and participant nominals is discussed in further detail in \S~\ref{sec:nomz-syntax}.

\subsection{Converbs}\index{converb}
Converbs (glossed \Cv{}) are a non-finite verb form often used for adverbial constructions. There are two converb forms in Iridian: the imperfective \ird{iec} (glossed \mk{cv.ipf}) and the perfective \ird{-u} (glossed \mk{cv.pf}).

\pex
\begingl
\gla Tereza kravn\v{e}c nóví palžek. //
\glb Tereza cry-\mk{cv.ipf} room-\Gen{} leave-\Av{}-\Pf{}//
\glft \trsl{Tereza left the room crying.}//
\endgl
\xe

\pex
\begingl
\gla Nóví palzu Tereza neikravnašek.//
\glb room-\Gen{} leave-\mk{cv.pf} Tereza \mk{incho}-cry-\mk{pf}//
\glft `Having left the room, Tereza started to cry.'//
\endgl
\xe

The syntax of converbial constructions and the specific uses of the perfective and imperfective converb form are discussed in detail in \S\,\ref{converbs-syntax}.


\subsection{Supine}

The {\scshape supine}\index{supine} is a non-finite verb form formed used to indicate necessity or purpose. Both usage has a nominal and a non-nominal form (used similar to an adverb or an adjective), giving the supine a total of four forms, as shown below:

\begin{table}[ht!]
	\sffamily\footnotesize
	\caption{Endings used for the supine.}
	\medskip
	\begin{tabu} to 0.6\textwidth{@{}YYY@{}}
		\toprule
		&{\sc purpose}&{\sc necessity}\\
		\midrule
		Nominal & {-it} & {-áš}\\
		Non-nominal & {-ice} & {-ášce}\\
		\bottomrule
	\end{tabu}
\end{table}

These four forms are invariable. The endings attach to the verb after the root has been conjugated for voice. The use of the non-nominal forms, in addition, does not require the use of the linking particle \ird{ko}.

\pex
\begingl
\gla >>Ána Karenina<< za gnaža oštinášce tóm.//
\glb Anna Karenina for school-\Acc{} read-\Pv{}-\SupN{} book//
\glft `I have to read \textit{Anna Karenina} for school.'//
\endgl
\xe

Although the usage of the supine has evolved to include various other constructions not related to its origins as a verbal noun indicating motion, the supine is still used in Modern Iridian in this original sense, accompanying a main verb (often a verb of motion) to indicate purpose. Both the nominal and the non-nominal form can be used in this construction, with the nominal form (despite being a more recent syntactic innovation) being more common and the non-nominal form considered more archaic, but still more prevalent in literary and formal usage. This usage roughly corresponds to the English infinitive, as in the sentence \trsl{I came here \emph{to bury} Cæsar.} When using the nominal form the clause containing the main verb is first transformed into a \ird{to-}clause and then equated to the nominal supine; when using the non-nominal form, on the other hand, the supine is simply added before the main verb.

\pex
\a
\begingl
\gla Tóm behlenik to oštnit.//
\glb book buy-\Pv{}-\Pf{} \Rz{} read-\Pv{}-\SupP{}//
\glft `I bought the book to read.'//
\endgl
\a
\begingl
\gla Tóm oščice behlenik.//\deftagex{supine}\deftaglabel{lit}
\glb book read-\Av{}-\SupP{} buy-\Pv{}-\Pf{}//
\glft `I bought the book to read.'//
\endgl
\xe

Especially when using the non-nominal construction, the grammatical voice used for the supine does not need to be the same as the one used in the main verb, as we see in example (\getfullref{supine.lit}). The supine can only take one argument, an object, which is always marked in the genitive regardless of its grammatical voice used to mark the supine governing it.

\pex
\a\begingl
\gla Marjám [těží probem\'i vednice] stožek.//
\glb Mary god-\Gen{} sepulchre-\Gen{} see-\Pv{}-\SupP{} go-\Av{}-\Pf{}//
\glft \trsl{Mary went to see the Lord's sepulchre.}//
\endgl
\a\begingl
\gla Marjám [těží probem\'i vi\v{z}ice] stožek.//
\glb Mary god-\Gen{} sepulchre-\Gen{} see-\Av{}-\SupP{} go-\Av{}-\Pf{}//
\glft \trsl{Mary went to see the Lord's sepulchre.}//
\endgl
\xe

In addition to this original usage, and to their use in indicating purpose or necessity, the supine\index{supine} is quite heavily employed idiomatically in Iridian. In colloquial speech, the supine of purpose is often used to express future or probable events as a substitute to the contemplative aspect. In both colloquial and literary registers, it may also be used to indicate a habitual action or a general truth (instead of the continuous or progressive aspect) when the verb implies some sort of purpose or consequentiality, especially in relation to another verb.

\pex
\begingl
\gla Dá to tómí oščit.//
\glb \First{}\Sg{} this book-\Gen{} read-\Av{}-\SupP{}//
\glft `I will be reading this book.' \emph{Lit.,} `I am someone whose purpose is the reading of this book.'//
\endgl
\xe

\pex
\begingl
\gla Méva dousa ješ me bylu dnou má nemel toha ohlečit.//
\glb all adult-\Acc{} \Exst{} as child-\Ins{} front but few:people this.\Acc{} remember-\Av{}-\SupP{}//
\glft \trsl{All grown-ups were children once but only a few remember it.}//
\endgl
\xe

Another common construction involves the supine of necessity with the words \irdp{shlac}{now} \irdp{mál}{time} (or less frequently \irdp{ór}{hour}). This construction is somehow similar to English\index{English} \trsl{It's time we left} or \trsl{It's time for us to go.} When used this way, the supine is conjugated in the locative voice.\index{supine}

\pex
\begingl
\gla Shlac himatí palzounášce mál.//
\glb now homeland-\Gen{} leave-\Lv{}-\SupN{} time//
\glft `It's time (we) left our homeland.'//
\endgl
\xe
\pex
\begingl
\gla Sa tet. Shlac zdalounášce mál.//
\glb already noon now have:breakfast-\Lv{}-\SupN{} time//
\glft `It's already late (\emph{lit.,} noon). It's time (we) had breakfast.'//
\endgl
\xe

\section{Stative verbs}\index{verbal adjective|see{stative verb}}\index{stative verb}\index{adjectives}\label{sec:statives}

Iridian lacks a distinct class of adjectives.\footnote{There is however a small class of attributives, which includes deictics\index{deictics} and quantifiers\index{quantifiers} among others, which can function as modifiers. They are different in that these words cannot be used as the predicate\index{predicate} of a sentence. They are discussed in detail on Chapter \ref{chap:minor}.} Instead, a special class of verbs called {\scshape stative verbs} are used to modify noun or noun-like classes. Unlike most verbs, however, stative verbs can only be marked for aspect, and optionally for voice. In addition to this base form (called the {\scshape copulative}), stative verbs also have an {\scshape attributive} form (used when the verb is preceding the noun or noun phrase) and {\scshape nominative} form (representing a concrete nominalisation of the verb), both of which are absent in non-attributives verbs. Consider for example the verb \ird{všihná} \trsl{to be angry}:

\begin{table}[ht]
	\sffamily\footnotesize
	\caption{Conjugation pattern for stative verbs}
	\medskip
	\begin{tabu} to 0.7\textwidth{@{}YY[0.8]Y@{}}
		\toprule
		&{\sc ending}&{\sc example}\\
		\midrule
		Copulative & varies & varies\\
		Attributive & {-í} & {všihní}\\
		Nominative & {-ou}	& {všihnou}\\
		\bottomrule
	\end{tabu}
\end{table}

\subsection{Copulative and attributive forms}\index{stative verb}
The copulative form of stative verbs is used when the verb is the predicate of the sentence. This form is only conjugated for aspect, and optionally for voice. Unlike normal verbs, however, stative verbs cannot be conjugated in the agentive voice since Iridian grammar does not distinguish between agency in an actor and the description of a state in stative verbs, both of which are encoded in the definition of this class. 

\ex
\begingl
\gla Mamka všihneví \emph{(not} *všihnaševí\emph{)}//
\glb mother-\mk{dim} be:angry-\mk{cont} not be:angry-\Av{}-\Cont{}//
\glft \trsl{My mother is angry.}//
\endgl
\xe


The attributive form is derived by replacing the infinitive marker \ird{-á} with \ird{-í}. Other than its conjugated comparative form ending in \ird{-ení}, the attributive form is invariable. The comparative form\index{comparative construction}\index{comparison} is often used, especially in colloquial speech, as an intensifier, even if the stative verb is not actually used in a comparison.

\ex
\begingl
\gla Všihnení mamka télévoniržek.//
\glb be:angry-\mk{comp-att} mother-\mk{dim} call-\Av{}-\Pf{}//
\glft \trsl{Mother was fuming (\emph{lit.,} angrier) when she called us.}//
\endgl
\xe

Because of the invariability of the attributive form, the copulative form may sometimes be used as a modifier, similar to a normal verb, separated from the noun it modifies with the particle \ird{ko}. Note, however, that when conjugated in the continuous aspect (except when marked explicitly for voice), such usage is not grammatical, with Iridian only allowing the attributive.

\ex
\begingl
\gla Všihninek ko tieho snov uprožilzách.//
\glb be:angry-\Pv{}-\Pf{} \Att{} god soon \Refl{}-avenge-\mk{av-ctpv} //
\glft \trsl{God whom you have angered will seek vengeance soon.}//
\endgl
\xe


\subsection{Nominative form}\index{stative verb}
The nominative form is derived by replacing the infinitive marker \ird{-á} with the nominalising suffix \ird{-ou}. This is the same nominalising suffix used to form nouns from regular verbs, the only difference being that stative verbs allow the suffix to be attached directly on the verb's root.

The copulative form may also be nominalised with \ird{-ou}.\index{nominalisation} However, as with the attributive form, if the copulative stative verb is conjugated in the continuous aspect and is unmarked for voice, the nominal form is used instead of the nominalised copulative form.

\subsection{Stative verbs and voice}\index{voice}\index{stative verb}

In general, stative verbs can also be conjugated for voice, with two main differences: first, as mentioned earlier in this section, the agentive voice cannot be used with stative verbs as Iridian does not distinguish between stative and agentive verbs and such information is considered to be encoded by default in the stative form; and second, in view of the first point, the benefactive gains an ``agentive'' interpretation, as it is used when the subjective is the agent of the action leading to the state being described by the verb, as in the example below:

\ex
\begingl
\gla Zuštalébkou houba.//
\glb be:happy-\mk{ben-pf-nz} gift//
\glft \trsl{What made me happy was (your) gift.}//
\endgl
\xe



\section{Derivational morphology}
\subsection{External derivation}
\par Loanwords ending in \textbf{-ace} from the Latin change the final e to á:
\begin{table}[h!]
	\centering \small
	\begin{tabu} to 0.9\textwidth{>{\bfseries}YM[0.3]>{\bfseries}YY}
		administrace 	& $\rightarrow$ & administracá 	& `to administrate' \\
		akuzace			& $\rightarrow$ & akuzacá		& `to accuse'\\
		diferenzace		& $\rightarrow$ & diferenzacá	& `to differentiate'\\
		separace		& $\rightarrow$ & separacá		& `to separate'\\
	\end{tabu}
\end{table}
\par Some Latin loanwords are borrowed first from German. Loanwords ending in \textbf{-ieren} become \textbf{-irná}.
\begin{table}[h!]
	\centering \small
	\begin{tabu} to 0.9\textwidth{>{\bfseries}YM[0.3]>{\bfseries}YY}
		akzeptieren 	& $\rightarrow$ & akceptirná 	& `to accept' \\
		konservieren	& $\rightarrow$ & koncervirná	& `to conserve'\\
		produzieren		& $\rightarrow$ & producirná	& `to produce'\\
		vandalieren		& $\rightarrow$ & vandalirná 	& `to deface'\\
	\end{tabu}
\end{table}
\subsection{Internal Derivation}
