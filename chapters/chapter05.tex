\chapter{Minor Word Classes}\label{chap:minor}\index{minor word classes}

\section{Conjunctions}\index{conjunctions}

\section{Prepositions}

\subsection{na}

\subsection{\v{s}e}

\subsection{\ird{vo}}\index{vo}\index{agentive case}

\ird{Vo} can be translated as \trsl{because of} or \trsl{due to.} This preposition takes the agentive case.

\pex
\begingl
\gla Vo transit\'am lienu z\'ascenz\v{c}em.//
\glb because traffic-\mk{agt} on:time-\mk{inst} \mk{neg}-arrive-\mk{av-pf-1s}//
\glft \trsl{I didn't arrive on time because of the traffic.}//
\endgl
\xe

\subsection{za}

\section{Adjectives}\index{adjectives}
Iridian lacks a true class of adjectives. Instead to modify a noun or a noun phrase, Iridian often uses nominal or verbal constructions.

\pex
\a
\begingl
\gla morc//
\glb black:thing//
\glft \trsl{black thing}//
\endgl
\a
\begingl
\gla Kver\v{s} morc.//
\glb raven black:thing//
\glft \trsl{Ravens are black.} \emph{Literally,} \trsl{(A) raven is (a) black thing.}//
\endgl
\a
\begingl
\gla morcie kver\v{s}.//
\glb black:thing-\mk{gen} raven//
\glft \trsl{black raven}//
\endgl
\xe

\pex
\a
\begingl
\gla na\v{s}t\'a//
\glb fly//
\glft \trsl{to fly}//
\endgl
\a
\begingl
\gla Kver\v{s} na\v{s}\v{c}al\'i.//
\glb raven fly-\mk{av-prog}//
\glft \trsl{(The) raven is flying}//
\endgl
\a
\begingl
\gla na\v{s}\v{c}al\'i ko kver\v{s}.//
\glb fly-\mk{av-prog} \mk{att} raven//
\glft \trsl{flying raven}//
\endgl
\xe

\section{Demonstratives}\label{dem-adj}\index{demonstratives}

\section{Quantifiers}\index{quantifiers}
Iridian has a wide variety of non-numerical/indefinite quantifiers.  Most are actually nouns that used in adjectival or adverbial constructions.


\begin{itemize}
    \item \ird{o\v{s}\v{c}} \trsl{many} (countable)
    \ex
    \begingl
    \gla Marka je\v{s} na\v{z}e o\v{s}\v{c}.//
    \glb Marek-\mk{pat} \mk{exst} friend-\mk{gen} many//
    \glft \trsl{Marek has many friends.}//
    \endgl
    \xe
    \ex
    \begingl
    \gla Za kursa m\'en je\v{s} o\v{s}\v{c} oudin\'a\v{s}ce ko vilm.//
    \glb for class-\mk{pat} \mk{1pl.inc.wk} \mk{exst} many watch-\mk{sup} \mk{att} film.//
    \glft \trsl{We have a lot of movies we need to watch for our class.}//
    \endgl
    \xe
    \item \ird{nave} \trsl{too many} (countable)
    \ex
    \begingl
    \gla Marka je\v{s} na\v{z}e o\v{s}\v{s}.//
    \glb Marek-\mk{pat} \mk{exst} friend-\mk{gen} many//
    \glft \trsl{Marek has many friends.}//
    \endgl
    \xe
    \item \ird{tohle} \trsl{many} (uncountable)
    \item \ird{nahte} \trsl{too many, too much} (uncountable)
    \ex
    \begingl
    \gla Do je\v{s} nahte kurv\'a\v{s}//
    \glb \mk{1s.wk} \mk{exst} too:much work-\mk{sup.nom}//
    \glft \trsl{I have so much work to do.}//
    \endgl
    \xe

\end{itemize}

\section{Interjections}

An interjection\index{interjection} is a word or an expression used to express a spontaneous reaction or feeling. We will use the term `interjection' to refer both to the part of speech and to the utterance type that has the same pragmatic function as this part of speech (cf. \cite{ameka1992}).

Interjections can be classifed into two main categories: \emph{primary} interjections, which refer to a word or an utterance that can only be used as an interjection and \emph{secondary} interjections, which refer to forms belonging a different word class but which through its usage, has acquired a new meaning as an interjection.

Although interjections can function as exclamations, not all exclamatory utterances can be considered as interjectons by themselves. As \textcite{jovanovic2004} notes, any word in a language can theoretically become an exclamation. Consider for example this conversation:

\ex (adapted from \cite{jovanovic2004}).\\

  \ird{
  \noindent--- Martin mlaza boule\v{s}ik.\\
  --- \textbf{Martin\'am?}
  }\medskip

  \trsl{I heard Martin killed his brother.}\\
  \trsl{Martin?!}
\xe
\section{Numerals}
\par Iridian has a vigesimal number system. Table \ref{one20} shows Iridian numerals from 1 to 20. Numbers from 1 to 10 are given their own name while numbers from 11 to 19 are formed by appending the numbers from one to nine to the clitic \ird{-niem} with the preposition \ird{\v{s}e} (with). The clitic \ird{-niem} is derived from the word for number 10, \ird{nau}, which itself comes from the Old Iridian \rec{nagu}, `half.'
\begin{table}[h!]
	\centering
		\caption{Iridian numerals from 1 to 20.}
\begin{tabu}to 0.8 \textwidth {M[0.5]YM[0.5]Y}
	\toprule
	{\sc number} & {\sc iridian} & {\sc number} & {\sc iridian}\\
	\midrule
	1 & ona			& 11 & on\v{s}eniem\\
	2 & m\"y			& 12 & mui\v{s}eniem\\
	3 & hroná		& 13 & hrona\v{s}eniem\\
	4 & dró			& 14 & dró\v{s}eniem\\
	5 & jed			& 15 & jeceniem\\
	6 &	vú			& 16 & vú\v{s}eniem\\
	7 & \v{s}\v{c}\k{e}	& 17 & \v{s}\v{c}\k{e}ceniem\\
	8 & pie\v{s}		& 18 & pi\k{e}ceniem\\
	9 & cam			& 19 & camzeniem\\
	10& d\'ech			& 20 & tydná\\

	\bottomrule
	\label{one20}
\end{tabu}
\end{table}

For numbers 11 to 19, the words are formed by appending the numbers from one to nine to the suffix \textit{-niem} with the preposition \textit{\v{s}e} (with).

\par Numbers from 21 to 99 are first expressed as multiples of 20. Thenceforth, the number system has largely become decimal, due primarily to the inflyence of surrounding Indo-European languages. Old Iridian, however, had a vigesimal system up to the number 8000.

\par Table \ref{3099} shows multiples of 10 from 30 to 100. The numbers are formed by the numeral followed by \ird{tydná}. For bases that are not multiples of 20, the word \ird{nau} \trsl{ten} is added first, followed by the conjunction \ird{\v{s}e} \trsl{with}.

\begin{table}[h!]
	\centering
	\caption{Iridian numerals from 30 to 100.}
	\begin{tabu}to 0.9 \textwidth {M[0.5]YM[0.5]Y}
		\toprule
		\multicolumn{1}{c}{\sc number} & \multicolumn{1}{c}{\sc iridian} & \multicolumn{1}{c}{\sc number} & \multicolumn{1}{c}{\sc iridian}\\
		\midrule
		30 &	nau\v{s}etydná		& 70 	& nau\v{s}ehronutydná\\
		40 &	muitydná		& 80	& drohutydná\\
		50 &	nau\v{s}emuitydná	& 90	& nau\v{s}edrohutydná\\
		60 &	hronutydná		& 100	& miesy\\
		\bottomrule
		\label{3099}
	\end{tabu}
\end{table}

Iridian counting starts from the smallest component of the number to the largest. Each component can be simply appended with the conjunction \ird{\v{s}e}. Only the numerals in Tables \ref{one20} and \ref{3099}, and the first ten numbers after 100, 500, 1000, etc. appear as single words. Below are some illustrations:

\pex
\a \ird{jecemiesy}\\
	\trsl{five with hundred}\\
	105
\a \ird{cam \v{s}e drohutydná}\\
	\trsl{nine with four twenties}\\
	89
\xe

\begin{table}[h!]
	\centering
	\caption{Iridian numerals from 200 to one billion.}
	\begin{tabu}to 0.9 \textwidth {Y[0.6]Y}
		\toprule
		\multicolumn{1}{c}{\sc number} & \multicolumn{1}{c}{\sc iridian} \\
		\midrule
		200 			&	moig	\\
		300, 400, etc.	& 	hronumiesy, drohumiesy. etc.\\
		1000			& 	nitak\\
		2000, 3000, etc.& 	muiniec, hronuniec, etc.\\
		10.000			&	ohle\\
		20.000, etc.	& 	tydnuniec, etc.\\
		100.000			&	dunie\\
		200.000 etc		&	meguiniec, hronuniec, etc.\\
		1.000.000		&	myliâ\\
		1.000.000.000	&	myliár\\
		1.000.000.000.000	& byliâ\\
		\bottomrule
		\label{3099}
	\end{tabu}
\end{table}

\subsection{Ordinal numbers}

\subsection{Fractions and decimals}

\subsection{Use of numerals}
