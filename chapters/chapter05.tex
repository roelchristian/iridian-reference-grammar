\chapter{Minor Word Classes}\label{chap:minor}\index{minor word classes}

\section{Conjunctions}\index{conjunctions}

\section{Prepositions}

\subsection{na}

\subsection{\v{s}e}

\subsection{\ird{vo}}\index{vo}\index{agentive case}

\ird{Vo} can be translated as \trsl{because of} or \trsl{due to.} This preposition takes the agentive case.

\pex
\begingl
\gla Vo transit\'am lienu z\'ascenz\v{c}em.//
\glb because traffic-\mk{agt} on:time-\mk{inst} \mk{neg}-arrive-\mk{av-pf-1s}//
\glft \trsl{I didn't arrive on time because of the traffic.}//
\endgl
\xe

\subsection{za}

\section{Adjectives}\index{adjectives}
Iridian lacks a true class of adjectives. Instead to modify a noun or a noun phrase, Iridian often uses nominal or verbal constructions.

\pex
\a
\begingl
\gla morc//
\glb black:thing//
\glft \trsl{black thing}//
\endgl
\a
\begingl
\gla Kver\v{s} morc.//
\glb raven black:thing//
\glft \trsl{Ravens are black.} \emph{Literally,} \trsl{(A) raven is (a) black thing.}//
\endgl
\a
\begingl
\gla morcie kver\v{s}.//
\glb black:thing-\mk{gen} raven//
\glft \trsl{black raven}//
\endgl
\xe

\pex
\a
\begingl
\gla na\v{s}t\'a//
\glb fly//
\glft \trsl{to fly}//
\endgl
\a
\begingl
\gla Kver\v{s} na\v{s}\v{c}al\'i.//
\glb raven fly-\mk{av-prog}//
\glft \trsl{(The) raven is flying}//
\endgl
\a
\begingl
\gla na\v{s}\v{c}al\'i ko kver\v{s}.//
\glb fly-\mk{av-prog} \mk{att} raven//
\glft \trsl{flying raven}//
\endgl
\xe

\section{Demonstratives}\label{dem-adj}\index{demonstratives}

\section{Quantifiers}\index{quantifiers}
Iridian has a wide variety of non-numerical/indefinite quantifiers.  Most are actually nouns that used in adjectival or adverbial constructions.


\begin{itemize}
    \item \ird{o\v{s}\v{c}} \trsl{many} (countable)
    \ex
    \begingl
    \gla Marka je\v{s} na\v{z}e o\v{s}\v{c}.//
    \glb Marek-\mk{pat} \mk{exst} friend-\mk{gen} many//
    \glft \trsl{Marek has many friends.}//
    \endgl
    \xe
    \ex
    \begingl
    \gla Za kursa m\'en je\v{s} o\v{s}\v{c} oudin\'a\v{s}ce ko vilm.//
    \glb for class-\mk{pat} \mk{1pl.inc.wk} \mk{exst} many watch-\mk{sup} \mk{att} film.//
    \glft \trsl{We have a lot of movies we need to watch for our class.}//
    \endgl
    \xe
    \item \ird{nave} \trsl{too many} (countable)
    \ex
    \begingl
    \gla Marka je\v{s} na\v{z}e o\v{s}\v{s}.//
    \glb Marek-\mk{pat} \mk{exst} friend-\mk{gen} many//
    \glft \trsl{Marek has many friends.}//
    \endgl
    \xe
    \item \ird{tohle} \trsl{many} (uncountable)
    \item \ird{nahte} \trsl{too many, too much} (uncountable)
    \ex
    \begingl
    \gla Do je\v{s} nahte kurv\'a\v{s}//
    \glb \mk{1s.wk} \mk{exst} too:much work-\mk{sup.nom}//
    \glft \trsl{I have so much work to do.}//
    \endgl
    \xe

\end{itemize}

\section{Interjections}

An interjection\index{interjection} is a word or an expression used to express a spontaneous reaction or feeling. We will use the term `interjection' to refer both to the part of speech and to the utterance type that has the same pragmatic function as this part of speech (cf. \cite{ameka1992}).

Interjections can be classifed into two main categories: \emph{primary} interjections, which refer to a word or an utterance that can only be used as an interjection and \emph{secondary} interjections, which refer to forms belonging a different word class but which through its usage, has acquired a new meaning as an interjection.

Although interjections can function as exclamations, not all exclamatory utterances can be considered as interjectons by themselves. As \textcite{jovanovic2004} notes, any word in a language can theoretically become an exclamation. Consider for example this conversation:

\ex (adapted from \cite{jovanovic2004}).\\

  \ird{
  \noindent--- Martin mlaza boule\v{s}ik.\\
  --- \textbf{Martin\'am?}
  }\medskip

  \trsl{I heard Martin killed his brother.}\\
  \trsl{Martin?!}
\xe
