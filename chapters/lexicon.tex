\chapter{Lexicon}

\section{Kinship Terms}\label{sec:kinship terms}\index{kinship terms}
\subsection{Nuclear Family}\label{sec:nuclear family}

The diminutive\index{diminutive} form of the nouns relating to the nuclear family are presented here as well since, as discussed in \S\,XX, it is common to use the diminutive instead of the regular form of nouns when to referring to one's own family or that of a socially close one (e.g., a friend's).

%% TODO add section reference above

\begin{table}[h!]
  \caption{Kinship terms, nuclear family.}
  \label{tbl:kinship}
  \footnotesize\sffamily
  \begin{tabu} to 0.8\textwidth{YYY[1.5]}
  \toprule
  {\sc noun}    & {\sc diminutive} & {\sc translation}\\
  \midrule
  \ird{ploc}    & \ird{plu\v{s}ka } & family\\
  \ird{hor}    & \ird{horka}       & parents\\
  \ird{maty}    & \ird{m\'amka}     & mother\\
  \ird{t\'aty}  & \ird{p\'apka}     & father\\
  \ird{hre\v{s}t}    & \ird{hri\v{s}tka}     & sibling\\
  \ird{mlaz}    & \ird{mla\v{z}ka}     & brother\\
  \ird{vod}    & \ird{vodka}     & sister\\
  \ird{proud}    & \ird{prudka}     & oldest sibling/child\\
  \ird{zneibo}    & \ird{zn\'ibka}     & youngest sibling/child\\
  \ird{roho\v{s}}    & \ird{ru{z}ka}     & son\\
  \ird{jaja}    & \ird{j\'ajka}     & daughter\\
  \ird{vremou}    & \ird{vrem\'ovka}     & child\\

  \bottomrule

  \end{tabu}
\end{table}
