\chapter{An Overview of Iridian}

\section{A brief history of Iridia}

Iridian is a small republic located in Central Europe, bordering the Czech Republic, Slovakia, Poland, and Ukraine. Home to around 8 million people, Iridia is a relatively small country, but it is a major economic and political power in Central Europe with a long and colorful history. Around 85\% of the population speak Iridian, the national language, and an estimated 94\% of the population speak it as a first or second language. The remaining 15\% of the population, especially near the country's borders, speak various Slavic languages, including Czech, Slovak, Polish, and Ukrainian, while around 20\% of the people in the country's southeast counties (\ird{prest}), near the border with Hungary, speak Hungarian. Immigration from other parts of Europe has also brought speakers of other languages, including Romanian, German, and Albanian, to the country.

The English name Iridia is from the Latinized form of the medieval endonym \ird{Irdzaume} (cf. the modern \ird{Ircome}) meaning ``Land of the Irdz (Irc).''\footnote{
    Translations in Romance and non-European languages usually follow this Latinized version. See for example, French \foreign{Iridie}, Portuguese \foreign{Irídia} or Korean \foreign{{\begin{CJK}{UTF8}{mj}일리지야\end{CJK}} (`Illijiya')}. Most Central and Eastern European languages use an exonym closer to the Iridian endonym: see, for example, German \foreign{Irtzland}, Czech Jírice or Polish \foreign{Ircja}.}
The origin of the word \ird{Irdz} is uncertain, but it has been used to refer to the Iridian people since at least late Roman times, although it is unclear whether the use of the term referred to the actual Iridian people or to the other groups living in that area during that time. The Iridians have been living in the region of what is now Eastern Europe since before the Indo-European migrations and were believed to have been the dominant people in the area until the arrival of the Indo-Europeans. Not much is known about the initial contacts of the two cultures, but there is wide evidence of extensive cultural exchanges between the Iridian tribes and the arriving Indo-European peoples. It is unknown how much Iridian culture has influenced the Indo-European settlers and vice versa but it is clear that the two cultures have influenced each other in many ways simultaneously, with the influence of Iridian more being especially more pronounced in the development of the neighboring Slavic languages, with the Iridian language itself being influenced by the Slavic languages in much the same way.

Iridia as a nation state howver has been a relatively new development. The polity can trace its origins to the Kingdom of Iridia which governed what is now the Iridian Republic, Czech Republic, Slovakia and southern Poland from the tenth to the early 12th century. For the much of the next 500 years, the area was ruled by a series of Germanic and Slavic kingdoms, including the Kingdom of Bohemia, Kingdom of Hungary, Kingdom of Poland, and the Kingdom of Galicia-Volhynia. From the 16th and 17th century the area was ruled as the Principality of Iridia, a vassal state of the Kingdom of Hungary, and then as the Voivodeship of Iridia, within the Austrian and later the Austro-Hungarian Empire.

After the dissolution of the Austro-Hungarian Empire in 1918, most of the area of the present-day Iridian republic declared independence as the Ruginese Republic (from Rugina, the contemporary English name of the nation's capital Roubže). This independence was short-lived and in 1938, the republic was annexed by Germany, a day after the annexation of its neighboring Czechoslovakia.

After the end of the Second World War, the area was occupied by the Soviet Union, which established the Iridian People's Republic (IPR). The republic was a member of the Warsaw Pact and was a satellite state of the Soviet Union. In 1968, the IPR was a part of the invasion of Czechoslovakia by the Soviet Union and its allies, which was known as the Prague Spring. The invasion was met with resistance from the Iridian people and the Soviet Union was forced to withdraw. This however cooled the relation between the IPR and the neighboring Czechoslovakia, an impasse that almost led to a full scale war between the two countries in 1979 after continued provocation from then-president of IPR Jozip Enta (who ruled the country for much of the 1970s). Discontent with the existing regime continued from the 1980s to the 1990s, culminating in the eventual collapse of the Soviet Union and the IPR in 1991 and 1992 republic. The Iridian people then voted to establish a parliamentary republic in 1992, with the first democratic elections being held in 1993. The country, which is officially known as the Iridian Republic (\ird{Ircevní respublika}), has since been a member of the European Union, the Visegrád Group, NATO, and the OECD.

\begin{table}
\centering \begin{tabular}{ll} \hline
Capital & Roubže \\
Major cities & Brest, Preždy, Kum, Štětín, Žilina \\
Official language & Iridian \\
Population & 8 million (2019) \\
Area & 53,244 km$^2$ \\
Water (\%) & 1.5 \\
GDP (nominal) & 422 million USD (2019) \\
GDP (nom. per capita) & 53,000 USD (2019) \\
Currency & Iridian koruna (IRK) \\
HDI & 0.919 - very high (2018) \\
Time zone & UTC+1 \\
Head of State & President Luka Anec \\
Head of Government & Prime Minister Mila Ormi \\
Legislature & Parliament (unicameral)\\
\hline
\end{tabular}
\caption{Basic facts about Iridia}
\label{tab:fact-sheet}
\end{table}


\section{The Iridian Language}

The Iridian language (\ird{ircevní malno}) is a language isolate belonging to the Iridian language family. It is an agglutinative language with a rich inflectional system mainly in the form of suffixes, and a relatively large number of phonemes.


\section{Word Classes}\label{sec:wordclasses}
Traditional Iridian grammar classifies words into four main classes: \irdp{min\v{e}c}{nouns}, \irdp{hlout\v{e}c}{verbs}, \irdp{prid\v{e}c}{modifiers}, and \irdp{zvuk}{function words}. We will follow this system for much of this book, introducing deviations to the system where appropriate to the discussion at hand.

