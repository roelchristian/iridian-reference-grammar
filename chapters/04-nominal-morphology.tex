\chapter{Nouns}

Nominal morphology in Iridian is relatively simpler compared to the corresponding process with verbs. Where possible, Iridian sentences are generally constructed to leave the noun or noun phrase unmarked.

\section{Grammatical categories}

\section{Number}\index{grammatical number}\index{plural}

Nouns in Iridian are not formally marked for number. Thus the word \ird{byl}, for example, can mean either \trsl{child} or \trsl{children} depending on the context. The same form is used when the noun is preceded by a numeral. Thus in Iridian one says \irdp{ona byl}{one child} and \irdp{hroná byl}{three children}.

Nevertheless, Iridian can express semantic plurality by using quantifiers, numerals, pluralizing particles or even through context alone. One such particle is \ird{ně}\label{sec:plurals}\footnote{Cf. \posscite{schachter1983} treatment of Tagalog\index{Tagalog} pluralizing particle \emph{mga}.}. \ird{Ně} is a proclitic and attaches to the left-most part of the noun phrase or the verb phrase it modifies. Throughout this book, \ird{ně} will be glossed as \Pl{}= for simplicity.

\pex
\begingl
    \gla ně ša zuštalí byl//
    \glb \Pl{}= \Dem{}.\Prox{} be:happy-\Att{} child //
    \glft \trsl{these happy children}//
\endgl
\xe

\ird{Ně} however could be understood to have three distinct uses. The first, as mentioned above, is to mark plurality. Alternatively, \ird{ně} could also be use as an approximative\index{approximation} (roughly equivalent to English \trsl{about}) when used with cardinal numbers or time expressions or as a honorific expletive\index{honorific}\index{expletive} to show politeness when used with proper names\index{proper names} or with some nouns (mostly related to kinship terms\index{kinship terms}). In its use for approximation, \ird{ně} is interchangeable with \irdp{u}{about}, although it is common in spoken speech to combine the two as an intensified construction. Preference is given to \ird{ně}, however, if the noun being modified is the topic of the sentence and must therefore remain unmarked.

\pex
\begingl
    \gla Ně mlazka-no scenžek?//
    \glb \Hon{}= brother\mk{-dim=q} arrive-\Av{}-\Pf{} //
    \glft \trsl{Was my brother the one who arrived?}//
\endgl
\xe

Note that when used with a cardinal number, \ird{ně} can only be understood to signify approximation, i.e., (\getfullref{appr.1}) can only mean \trsl{about three children} and not \trsl{three children}, as the latter would only be translated as \ird{hroná byl} without the clitic \ird{ně}.

As has been earlier mentioned, \ird{ně} is a proclitic\index{proclisis} and attaches to the left-most part of the noun phrase or verb phrase it modifies, including any modifier no matter how complex but excluding any proposition. In some cases, as can be seen in (b) and (c) below, the use of \ird{ně} to pluralize a noun can imply definiteness\index{definiteness}.

\pex
\a
\begingl\deftagex{pl}\deftaglabel{1}
    \gla {ně} za byla tóm//
    \glb \Pl{}= for child-\Acc{} child //
    \glft \trsl{books for children}//
\endgl
\a
\begingl\deftaglabel{2}
    \gla za {ně} byla tóm//
    \glb for \Pl{}= child-\Acc{} child //
    \glft \trsl{a book for (these) children}//
\endgl
\a
\begingl\deftaglabel{3}
    \gla {ně} za \textbf{ně} byla tóm//
    \glb \Pl{}= for \Pl{}= child-\Acc{} child //
    \glft \trsl{books for (these) children}//
\endgl
\xe

The use of \ird{ně}, however, is largely optional and where plurality can be implied from context, this particle is seen as redundant and is therefore dropped.

\ird{Ně} cannot be used with mass and uncountable nouns, as well as with abstract nouns.

\pex
\a
\begingl
\gla *Na duma ně ješ piaštou.//
\glb \Loc{} house \Pl{} \Exst{} food//
\glft \trsl{There is food in the house.}//
\endgl
\a
\begingl
\gla Na duma tohle ješ piaštou.//
\glb \Loc{} house much \Exst{} food//
\glft \trsl{There is a lot of food in the house.}//
\endgl
\xe

The particle \ird{ně} always precedes the noun it modifies, except in
existential clauses where it comes before the existential particle
\ird{ješ}\footnote{The sequence is pronounced as if written něš [ɲɛɕ]}.
\ird{Ně} can obviously not be used with the negative particle
\ird{niho}.\index{niho}\index{existential construction}\index{ješ}

\pex
\a
\begingl
\gla ně bžem//
\glb \Pl{} bee//
\glft \trsl{bees}//
\endgl
\a
\begingl
\gla Ně ješ bžem.//
\glb \Pl{} \Exst{} bee//
\glft \trsl{There are bees.}//
\endgl
\a
\begingl
\gla *Ně niho bžem.//
\glb \Pl{} \N{}\Exst{} bee//
\glft \trsl{There are no bees.}//
\endgl
\xe

\index{pluralia tantum}
\ird{Ně} cannot be used with a limited number of nouns, mostly referring to paired body parts and related objects, which in the base form is understood to refer to the pair itself and thus cannot be pluralized. If the speaker wishes to explicitly refer to one piece of the pair, the noun \ird{noma} (an obsolete form of the word for one-half, now surviving only in this construction) and the genitive form of the body part.

\pex
\begingl
\gla Eg zaromnek.//
\glb eyes close-\Pv{}-\Pf{}//
\glft \trsl{(He) closed (his) eyes.}//
\endgl
\xe
\pex
\begingl
\gla Pohár dévit.//
\glb eyeglasses dirty//
\glft \trsl{(Your) eyeglasses are dirty.}//
\endgl
\xe
\pex
\begingl
\gla Ohví noma utieščál.//
\glb shoe-\Gen{} half \Refl{}-lose-\Av{}-\Cont{}//
\glft \trsl{The other pair of (his) shoe is missing.}//
\endgl
\xe

The base form is also used in generic statements where English would normally
use the plural.\index{generic statements}\index{universals} When used with a
proper noun \ird{ně} can be translated with the English construction \trsl{and
others}. Note that this is different from the usage of \ird{ně} as a honorific.

\pex
\begingl
    \gla Ně Jancě gnaž uprubížice.//
    \glb \Pl{}= Janek-\Gen{} school \Refl{}-burn-\Av{}-\Pf{}-\Quot{} //
    \glft \trsl{I heard Janek's school burned down.}//
\endgl
\xe

\pex
\begingl
    \gla Ně Marek zázdalšek.//
    \glb \Pl{}= Marek \Neg{}-have:breakfast-\Av{}-\Pf{} //
    \glft \trsl{Marek and the others did not eat breakfast.}//
\endgl
\xe


\section{Definiteness}\index{definiteness}
Iridian does not have definite or indefinite articles; instead a noun or a noun phrase's definiteness is often expressed syntactically. This is discussed in detail in \S\,\ref{sec:definiteness}.

\section{The case system}

\subsection{Declension patterns}

Nouns in Iridian can end in a consonant or any of \ird{-a}, \ird{-e}, \ird{-ě},
\ird{o} or \ird{ou}. There are seven declension classes, determined by the
ending of the noun. Class I refers to nouns ending in a hard consonant, Class II
to nouns ending in a soft consonant, and Classes III through VII to nouns ending
in \ird{-a}, \ird{-e}, \ird{-ě}, \ird{o} and \ird{ou}, respectively. The
declension classes are summarized in Table~\ref{tab:declension}.


\begin{table}[h]
    \footnotesize\sffamily
    \caption{Paradigm endings for the six declension classes.}\label{tab:declension}
    \medskip
    \begin{tabu} to \textwidth {@{}Y[2]YYYYYYY@{}}
    \toprule\addlinespace
            {\sc case}      &{\sc i} &{\sc ii} & {\sc iii} &{\sc iv} &{\sc v} &{\sc vi} & {\sc vii}\\
    \midrule\addlinespace
            Agentive        & -ám    & -ám     & -am       & -em   & -ěm    & -om   & -óvam\\ \addlinespace
            Accusative      & -a     & -a      & -e        & -y    & -y     & -im   & -óva\\ \addlinespace
            Genitive        & -í     & -ý      & -\'i      & -ý    & -ý     & -e    & -óví\\ \addlinespace
            Instrumental    & -u     & -u      & -u        & -u    & -u     & -u    & -óvím\\ \addlinespace
    \bottomrule
    \end{tabu}
\end{table}

Iridian declension is regular and predictable. There are no irregular nouns, and
the endings of the declension paradigms are the same for all nouns in a given
class. The only notable morphophonemic change is caused by the softening of hard
consonants when followed by the genitive ending. This softening causes the
fricativization of the velar stop /k/ to /t͡ɕ/ which in this case is spelled as
\orth{c} and not \orth{č} as would have been expected. Thus the name \ird{Janek}
is declined as \ird{Janka} in the accusative but \ird{Jancí} in the genitive.

For the purposes of nominal declension, nouns ending in /k/, /g/, /d/, /t/, /f/
and /v/ are considered hard consonants. The sibilants /s/ and /z/ and the
affricate /t͡s/, although technically `hard' consonants, take Case II endings.
Thus \irdp{mez}{room} becomes \ird{mezý} and not \ird{*mezí}. Words ending in
all other consonants also take Case II endings. 

Below are examples showing the declension paradigms in Iridian.

\pex
\a \irdp{viták}{road}\\
\vtop{\halign{%
#\hfil& #\hfil\cr
Unmarked & \ird{viták} \cr
Agentive & \ird{vitákám} \cr
Patientive & \ird{vitáka} \cr
Genitive & \ird{vitácí} \cr
Instrumental & \ird{vitáku} \cr
}}

\a \irdp{slěň}{soup}\\
\vtop{\halign{%
#\hfil& #\hfil\cr
Unmarked & \ird{slěň} \cr
Agentive & \ird{slěňám} \cr
Patientive & \ird{slěňa} \cr
Genitive & \ird{slěňý} \cr
Instrumental & \ird{slěňu} \cr
}}


\a \irdp{prěsta}{neighbor}\\
\vtop{\halign{%
#\hfil& #\hfil\cr
Unmarked & \ird{prěsta} \cr
Agentive & \ird{prěstám} \cr
Patientive & \ird{prěste} \cr
Genitive & \ird{prěstí} \cr
Instrumental & \ird{prěstu} \cr
}}

\a \irdp{vtare}{morning}\\
\vtop{\halign{%
#\hfil& #\hfil\cr
Unmarked & \ird{vtare} \cr
Agentive & \ird{vtarem} \cr
Patientive & \ird{vtary} \cr
Genitive & \ird{vtarý} \cr
Instrumental & \ird{vtaru} \cr
}}

\a \irdp{shorě}{group}\\
\vtop{\halign{%
#\hfil& #\hfil\cr
Unmarked & \ird{shorě} \cr
Agentive & \ird{shorěm} \cr
Patientive & \ird{shorý} \cr
Genitive & \ird{shorí} \cr
Instrumental & \ird{shoru} \cr
}}

\xe 


\subsection{Agentive case}\index{agentive case}

The agentive case (\Agt{}) is used to indicate the agent of an action where the agent is not the topic of the sentence.

\pex
\begingl
\gla Marek Lučkám vidnik.//
\glb Marek Luček-\Agt{} see-\Pv{}-\Pf{}//
\glft \trsl{Marek was seen by Luček.}//
\endgl
\xe

The agentive is also used in comparative constructions, where it indicates the point of reference for the comparison.

\index{comparison}\index{agentive of comparison}
\pex
\begingl
\gla Dá Mark\k{a} tám stroja.//
\glb \mk{1s.str} Marek-\Agt{} \Comp{} tall//
\glft \trsl{Marek is taller than me}//
\endgl
\xe

\subsection{Patientive case}

In general, the accusative case is used to mark the direct object of a verb that is in the agentive voice. Note that this usage implies that the direct object is indefinite. Where the direct object is definite, the verb is usually in the accusative voice and the direct object is unmarked.

\pex
\a \begingl
\gla Guláša piašček.//
\glb goulash-\Acc{} eat-\Av{}-\Pf{}//
\glft \trsl{(He) ate goulash.}//
\endgl
\a Compare this to:\\
\begingl
\gla Guláš piaštnik.//
\glb goulash eat-\Pv{}-\Pf{}//
\glft \trsl{(He) ate the goulash.}//
\endgl
\xe

The accusative is also used to mark the direct object when the verb is in the benefactive voice.

\pex
\begingl
\gla Ša vitamina piaštebik.//
\glb \mk{3s.anim} vitamin-\Acc{} eat-\Ben{}-\Pf{}//
\glft \trsl{(She) made him take (his) vitamins.}//
\endgl
\xe

%%%%
% TODO Definiteness and the accusative; use of genitive when the noun marked is indefinite

When used to mark the direct object, the accusative implies the definiteness of the noun. Where the noun is indefinite, the genitive is used instead.

\pex
\a\begingl
\gla Vaška piaščem.//
\glb cake-\Acc{} eat-\Av{}-\Pf{}//
\glft \trsl{I ate the cake.}//
\endgl
\a\begingl
\gla Vašcí piašček.//
\glb cake-\Gen{} eat-\Av{}-\Pf{}//
\glft \trsl{I ate some cake.}//
\endgl
\xe

The accusative is used with the particle \ird{na} to form a compound locative case, which is itself used to indicate a general location.

\pex
\begingl
\gla Tomáš na byra.//
\glb Tomáš \Loc{} office-\Acc{}//
\glft \trsl{Tomáš is at the office.}//
\endgl
\xe

The accusative is also used with some prepositions, often indicating direction or movement. The most common of these are \ird{na} used to indicate a general location (\irdp{na byra}{at the office}), \ird{za} which roughly corresponds to the English `for' or `for the benefit of' (\irdp{za Marka}{for Marek}), \ird{u} used to indicate proximity (\irdp{u gara}{near the train station}), and \ird{do} which indicates movement towards a location (\irdp{do byra}{to the office}).

\subsection{Genitive case}\ref{sec:genitive-case}\index{genitive}

The simplest use of the genitive case is to indicate ownership or possession.
When used this way, the noun marked in the genitive must always procede the noun
it modifies.

\pex
\irdp{Marcí dum}{Marek's house}\\
\irdp{mámcí hašek}{my mother's bag}\\
\irdp{ša študencí tóm}{this student's book}\\
\xe

Demonstratives\index{demonstrative} and other modifers must always come before
the whole noun phrase and cannot split the possessor from the possessee. An
exception to this rule is the clitic \ird{ně}, which comes immediately before
the noun it pluralizes\index{plural}.

\pex
\a  \irdp{ša študencí tóm}{the/a book of this student}\\
    \irdp{to študencí tóm}{this book of the student}
\a  \irdp{ně študencí tóm}{the students' book}\\
    \irdp{študencí ně tóm}{the student's books}
\xe

The genitive is also used as a partitive 


\subsubsection{Genitive of material}

\ex
\irdp{kuní prosc}{silver spoon}\\
\irdp{be}
\xe

\subsubsection{Genitive of the whole}
The genitive can also be used to indicate

\pex
\begingl
\gla na kraštolí dnóva//
\glb \Loc{} train:station-\Gen{} front//
\glft \trsl{in front of the train station}//
\endgl
\xe

Note that the accusative and not the genitive case is used when quantifying a part of the whole.

\pex
\a
\begingl
\gla *žnohoušce hroná//
\glb student-\Gen{} three//
\glft \trsl{three of the students}//
\endgl
\a
\begingl
\gla na žnohoušca hroná//
\glb \Loc{} student-\Gen{} three//
\glft \trsl{three of the students}//
\endgl
\xe

Nevertheless when quantifying a noun per se, and not in relation to a whole, the uninflected form of the quantifier is used (mostly using indefinite quantifiers such as \trsl{many}, \trsl{a lot}, etc.). If however, the quantification involves a countable unit or division of the noun, the genitive is used, but such unit or division must be further quantified by a numeral or an indefinite quantifier.

\pex
\a
\begingl
\gla Na kroumašta po zma ješ pivo.//
\glb \Loc{} refrigerator-\Acc{} still few \Exst{} beer//
\glft \trsl{There's still some beer left in the refrigerator.}//
\endgl
\a
\begingl
\gla Ona pive štava unarížčem.//
\glb one beer-\Gen{} mug-\Acc{} \Refl{}-order-\mk{av-pv-1s}//
\glft \trsl{I ordered a mug of beer.}//
\endgl
\xe

\subsubsection{Genitive of movement}

The genitive is also used to indicate movement away from somewhere.

\pex
\a
\begingl
\gla Dumí palžek.//
\glb house-\Gen{} leave-\Av{}-\Pf{}//
\glft \trsl{I left the house.}//
\endgl
\a
\begingl
\gla Dum palzinek.//
\glb house leave-\Pv{}-\Pf{}//
\glft \trsl{I left the \emph{house}.}//
\endgl
\xe

The genitive is also used with certain prepositions, most of which indicate movement away from somewhere, the source of a movement or the origin of something, and other similar meanings. The most common of these are \ird{z} roughly corresponding to the English `from' (\irdp{z Marcí houba}{a gift from Marek}), \irdp{nam}{without} (\irdp{nam záhárí}{without sugar}), \irdp{pale}{instead of} (\irdp{pale Jancí}{instead of Janek}) and \irdp{ahte}{except} (\irdp{ahte Jancí}{except for Janek}).

\subsection{Instrumental case}\ref{sec:instrumental-case}
\index{instrumental case}

The instrumental case (glossed \Ins{}) is used to indicate the means by which an
action is performed. It is also used to indicate the instrument used to perform
an action. Some grammar books may also refer to this case as the prepositional
case.

\pex
\begingl
\gla Do byra vternovím stóževí.//
\glb to office-\Acc{} bicycle-\Ins{} go-\Av{}-\Cont{}//
\glft \trsl{I ride my bike to the office.}//
\endgl
\xe

The instrumental is also used with the particle \ird{še} which roughly
corresponds to the English \trsl{with}. In some cases, \ird{še} may also be
dropped altogether.

\pex
\a\begingl
\gla Za bolte še Janku stóžách.//
\glb for party-\Acc{} with Janek-\Ins{} go-\Av{}-\Ctp{}//
\glft \trsl{(I am) coming to the party with Janek.}//
\endgl
\a\begingl
\gla Terezu skaznašek.//
\glb Tereza-\Ins{} \Soc{}-sing-\Av{}-\Pf{}//
\glft \trsl{I sang with Tereza.} or \trsl{Tereza and I sang together.}//
\endgl
\xe

\section{Personal pronouns}\index{personal pronouns}\index{pronouns}

\subsection{Personal pronouns in general}

Personal pronouns are a special class of nouns used to refer and/or replace
other nouns or noun phrases. In traditional Iridian grammar, personal pronouns
are called \ird{svědé kaděc} or false nouns. We will follow this analysis and
treat personal pronouns not as a separate grammatical class but as a special
class of nouns since for the most part they are syntactically and
morphologically identical to nouns. Like other nouns, personal pronouns are
marked for person, number and case, and partially for animacy\index{animacy},
although third-person forms are more properly analyzed as demonstratives. In
this section we will only be discussing first and second person forms. Third
person forms are discussed in detail in \S~\ref{sec:demonstratives} with other
demonstratives.

Personal pronouns are declined in the same way as nouns although they are for
the most part more irregular than normal nouns. Table
\ref{tab:personal-pronouns} shows the declension of first- and second-person
personal pronouns in Iridian.

\begin{table}
    \footnotesize\sffamily
	\caption{Personal pronouns in Iridian}\label{tab:personal-pronouns}
	\medskip
	\begin{tabu} to 0.8 \textwidth {@{}Y[2.5]YYYY@{}}
		\toprule \addlinespace
        {\sc form}      & {\sc 1s}  & {\sc 2s} & {\sc 1pl} & {\sc 2pl}\\ \addlinespace
		\midrule \addlinespace
        Unmarked        & dá        & já      & mé      & tová  \\ \addlinespace
        Agentive        & dám       & jám     & mám     & tám   \\ \addlinespace
        Accusative      & dě        & jí      & mě      & tě    \\ \addlinespace
        Genitive        & že        & je      & mí      & teví  \\ \addlinespace
        Instrumental    & du        & jemu    & mejí    & tvě   \\ \addlinespace
        \bottomrule
	\end{tabu}
\end{table}

Unlike normal nouns, personal pronouns have explicit plural forms. This can be
traced back to their origins as demonstratives. In addition to indicating
number, the plural forms are also used to indicate politeness. This usage is
similar to the T-V distinction found in languages like French or German, but is
more general. The second person plural is used instead of the regular second
person singular forms even when the speaker is referring to a single person to
indicate respect or deference or merely as a way to distance oneself from the
listener, like when talking to a stranger or in situations where a higher level
of formality is required. The first person plural, on the other hand, may be
used in a similar fashion in formal contexts or when the speakers wishes to
communicate their humility. The choice of pronouns and the degree of formality
is determined by the context and the speaker's attitude towards themself and/or
the listener. This is discussed in more detail in \S~\ref{sec:politeness} and in
Chapter 8 in general.

Iridian is an extremely pro-drop language, with pronouns supplied only if not
immediately inferrable from context. In fact, a pronoun does not even have to be
supplied to establish context. Iridian, moreover, tends to favor avoidance not
simply as syntactic strategy but also as a part of its politeness system. Thus,
pronouns may be dropped not just because they are not necessary to establish
context but also because it might be considered impolite to use them in certain
contexts. A common alternative, for example, would be to address the listener
using their name or title or some other common noun as a \emph{quasi} honorific
(e.g., friend, comrade, etc.) either when the ambiguity in the referent would be
too great to simply drop the pronoun, or if, even when the context is clear, the
speaker just so wishes as a stylistic choice. Again, pronoun avoidance as a
politeness strategy is discussed further in \S~\ref{sec:politeness}.

The use of the possessive pronouns (i.e., personal pronouns in the genitive
case) is also very limited, with the possessive dropped in most cases where it
would be used in English. This latter behavior of Iridian would be familiar to
speaker of most Romance or Slavic languages. For example, the English sentence
\trsl{My knees hurt} would be translated in Iridian as \ird{Dliň prozíčime} and
not \ird{*Že dliň prozíčime}. Here \irdp{že}{my} is dropped unless the discourse
has hitherto included references to multiple knees and the speaker wishes to
specify that it is his knees that hurt. Even then, if the speaker wishes to
emphasize the ownership, it is more idiomatic to use what is called an `ethical
dative'\index{ethical dative} construction to indicate possession, viz.,
\ird{Dliň dě prozíčime} where the possessor is marked in the accusative. Compare
this with the Czech \foreign{Bolí mě kolena} `My knees hurt' or Spanish
\foreign{Me duelen las rodillas} `My knees hurt.'

There are no restrictions when it comes to the use of modifiers with personal
pronouns, unlike in, say, English where an adjective is normally not used with a
personal pronoun. For example, \irdp{zuštalé dá}{I who am happy} or more
literally, \trsl{(the) happy I} is a perfectly grammatical construction in
Iridian while the equivalent in English would mostly be reserved in literary
contexts, if at all, or more commonly rephrased with a relative clause \trsl{I,
who am happy, ...} or an apposition \trsl{I, the happy one, ...}.

\subsection{The Reflexive \ird{se}}\label{sec:reflexive-se}

Iridian has a special reflexive particle \ird{se}, which for the purposes of
this grammar we will consider as a personal pronoun. It is a Slavic borrowing,
possibly form Proto-Slavic \foreign{*sę} and is not attested in Old Iridian.
Table \ref{tab:se-declension} shows the declension of \ird{se}.

\begin{table}
    \footnotesize\sffamily
    \caption{Declension of the reflexive pronoun \ird{se}.}\label{tab:se-declension}
    \medskip
    \begin{tabu} to 0.5 \textwidth {@{}Y[1.2]Y@{}}
        \toprule \addlinespace
        {\sc case}      & {\sc declension}\\ \addlinespace
        \midrule \addlinespace
        Unmarked        & se    \\ \addlinespace
        Agentive        & snám  \\ \addlinespace
        Patientive      & semě  \\ \addlinespace
        Genitive        & sní   \\ \addlinespace
        Instrumental    & sem   \\ \addlinespace
        \bottomrule
    \end{tabu}
\end{table}

The reflexive \ird{se} is used to refer back to the topic of the sentence. Se is
often used with the reflexive voice, although the use of \ird{se} often implies
a greater disjunction between the actor and the patient. Where the reflexive
voice has a primarily sociative meaning, as in verbs with a defunct active
voice, \ird{se} is used to form a true reflexive construction.

\pex    \a \irdp{Udúšek}{I took a bath.}
        \a \irdp{Se udúšek}{I bathed myself.}
\xe

\pex
        \a \irdp{Guláše upiašček}{I ate some goulash.}
        \a \ljudge{?} \irdp{Se upiašček}{I ate myself.}
\xe

The genitive and accusative forms of \ird{se} are also used as a proprietary
intensifier, similar to the usage of the English adjective \foreign{own} as in
\foreign{his own worst enemy}.

\pex
\begingl
\gla Bych shradice ko papka sní éhu vednik.//
\glb yesterday be:dead-\Pf{}-\Quot{} \Att{} father-\Dim{} \Refl{}-\Gen{} eye-\Ins{} see-\Pv{}-\Pf{}//
\glft \trsl{I saw your supposedly dead father with my own eyes yesterday.}//
\endgl
\xe

\subsection{Possessive nominals}\label{sec:possessive-nominals}

\section{Demonstratives}\label{sec:demonstratives}
\index{demonstratives}

Iridian does not have a separate class of third-person pronouns. Instead it uses
a set of demonstratives, whose deictic\index{deixis} function is both
spatial\index{spatial deixis|see{deixis}} and anaphoric\index{anaphora}. Iridian
makes a three-way distinction among demonstratives, similar to
French\index{French} or Portuguese\index{Portuguese} for example, distinguishing
between proximal (near the speaker), medial (near the addressee) and distal (far
from both speaker and addressee) forms. In addition, Iridian makes an animacy
distinction with demonstratives, with one set of demonstratives used with human
referents and another with non-human referents, as seen in Table
\ref{tab:dem-prons}. Demonstratives like personal pronouns are not marked for
gender, but unlike personal pronouns they do not have separate plural forms.

\begin{table}
    \footnotesize\sffamily
	\caption{Demonstrative pronouns in Iridian.}
    \medskip
	\begin{tabu}to 0.7\textwidth{@{}YYYY@{}}
		\toprule\addlinespace
						& {\sc animate}	& {\sc inanimate}	&{\sc locative}\\ \addlinespace
		\midrule \addlinespace
		Proximal		& ša		& to 				& tak\\ \addlinespace
		Medial			& kako		& jáne				& jení\\ \addlinespace
		Distal			& dní		& děn				& dně\\ \addlinespace
		\bottomrule
		\label{tab:dem-prons}
	\end{tabu}
\end{table}

Demonstratives can be used adnominally, to modify a noun phrase, or
pronominally, to replace one. In examples (b) and (c) below, for example, the
usage of \ird{ša} can be interpreted can be intrepreted as adnominal, modifying
\ird{že byl}, or pronominal, with the demonstrative as the topic and \ird{že
byl} as the predicate. Note that there are no differences, whether in the
orthography or the intonation, between the phrase \ird{ša že byl} and the
sentence \ird{Ša že byl}.

\pex
    \a
        \begingl
        \gla ša byl//
        \glb \Dem{}.\Prox{}.\Anim{} child//
        \glft \trsl{this child}//
        \endgl
    \a
        \begingl
        \gla ša že byl//
        \glb \Dem{}.\Prox{}.\Anim{} \First\Sg{}.\Gen{} child//
        \glft \trsl{this child of mine}//
        \endgl
    \a
        \begingl
        \gla Ša že byl.//
        \glb \Dem{}.\Prox{}.\Anim{} \First\Sg{}.\Gen{} child//
        \glft \trsl{This (person) is my child.}//
        \endgl
    \a
        \begingl
        \gla \ljudge{*}To že byl.//
        \glb \Dem{}.\Prox{}.\Inan{} \First\Sg{}.\Gen{} child//
        \glft \trsl{This (thing) is my child.}//
        \endgl
\xe

When used with other modifiers, demonstratives appear as the left-most element
of the phrase, but after the clitic \ird{ně}. Thus \irdp{děn tóm}{that book over
there} and \ird{ně děn mordé tóm}{those blue books over there} are valid, but
\ird{*děn mordé ně tóm} is not. Demonstratives cannot modify other
demonstratives; and so forms like \ird{ša dní}, \ird{ša to}, etc. are all
ungrammatical.\footnote{The idiomatic phrase \irdp{tak dně}{here and there} is
not ungrammatical because here \ird{tak} is not really modifying \ird{dně}.
Instead this is more correctly analyzed as the ellipsis of \irdp{a}{and} from
the original phrase \irdp{tak a dně}{here and there}.}

Table \ref{tab:dem-prons} also shows a third set of demonstratives, which are
used to indicate the general location of a referent. In their unmarked forms,
these locative demonstratives can only be used adverbially and not to modify or
replace nouns or noun phrases like animate and inanimate demonstratives.

When used as pronominally, demonstratives are declined in accordance to their role in the sentence. Like personal pronouns, their forms, too, are highly irregular.

\begin{table}
    \footnotesize\sffamily
        \caption{Declension of demonstratives.}
        \medskip
        \begin{tabu}to \textwidth{@{}Y[2]YYYYYY@{}}
            \toprule
                            & {ša}	& {ón}	&{dní}& {to}	& {ján}	&{jón}\\
            \midrule \addlinespace
            Agentive&šem&nám&dněm&etom&ján&jón\\\addlinespace
            Patientive&šá&ona&dná&toha&jina&jinóva\\\addlinespace
            Genitive&ci&oní&dní&cie&ně&nohe\\\addlinespace
            Instrumental&svou&nu&dnu&etu&nu&nohu\\\addlinespace
            \bottomrule
            \label{dem-conj}
        \end{tabu}
    \end{table}

\pex
\a\deftagex{obv}
\begingl
\gla ci mlaz a dní mač//
\glft \trsl{this person's brother and that person's mother}//
\endgl
\a\deftaglabel{obv1}
\begingl
\gla Dá je svou je dnu zapreví.//
\glft \trsl{I am as old as either this person or that person.}//
\endgl
\xe

The three-way distinction between demonstratives allows Iridian to disambiguate between an obviative\index{obviation} third person and a proximate third person, using the distal and the proximal demonstrative respectively. Consider for example the two sentences in English below:

\pex
\a He saw his dog.
\a He saw his own dog.\smallskip
\xe

The \emph{his} in the first sentence is ambiguous, as it can refer to either the subject or an implied fourth person. That the second \emph{his} refers back to the subject can be made unequivocal by the addition of the word \emph{own}, as in the second sentence. Compare this with the following sentences in Czech:

\begin{multicols}{2}
  \pex
  \a
  \begingl
  \gla Viděl jeho pes.//
  \glft \trsl{He saw his dog.}//
  \endgl
  \a \begingl
  \gla Viděl své pes.//
  \glft \trsl{He saw his own dog.}//
  \endgl
  \xe
\end{multicols}

Although the English translation of the first sentence may still appear ambiguous, we can see that Czech does away with the ambiguity by using the third person pronoun \ird{jeho} exclusively to signify that the referent is different from the subject, and requiring the use of a separate pronominal form (in this case the reflexive) when the referent and the subject are the same. Iridian, on the other hand, treats this in a diametrically opposite way, i.e., the same pronoun form is used when the subject and the referent are the same, with the obviative form being used otherwise. The sentences in Czech above will therefore be translated in Iridian as follows:

\pex
\a
\begingl
\gla Dní jec vdinek.//
\glb \mk{dem.dist.anim.gen} dog see-\Pv{}-\Pf{}//
\glft \trsl{He saw his (other person's) dog.}//
\endgl
\a \begingl
\gla Ci jec vdinek//
\glb \mk{dem.dist.inan.gen} dog see-\Pv{}-\Pf{}//
\glft \trsl{He saw his own dog.}//
\endgl
\xe

Perhaps we can better understand the distinction between obviative and proximate forms by re-examining example (\getfullref{obv.obv1}) above. The previous examples in Czech remained unambiguous because there are at most two unique arguments in the sentence. In example (\getfullref{obv.obv1}), however, the subject of the sentence is distinct from either the proximate referent or the distal referent.

\ex[exno={\getfullref{obv.obv1}}]
\begingl
\gla Dá je svou je dnu zapreví.//
\glft \trsl{I am as old as either this person or that person.}//
\endgl
\xe

The translation in the gloss demonstrates how idiomatic English uses periphrastic forms to eliminate this ambiguity, although in the spoken language the purely deictic \trsl{I am as old as either him or him} is equally acceptable, with the blanks filled in most likely by non-verbal cues. In Iridian, however, this distinction is not optional, and the following sentence, for example, would be considered ungrammatical:

\ex
\begingl
\gla *Dá je svou je svou zapreví.//
\glft \trsl{I am as old as either him or him.}//
\endgl
\xe

\section{Interrogative pronouns}\index{wh-question@\emph{wh}-question}\index{interrogative pronoun}\label{sec:int-pron}

\begin{table}[h!]
	\sffamily\footnotesize
	\caption{Interrogative pronouns in Iridian.}
    \medskip
	\begin{tabu} to 0.8\textwidth{@{}YYYY@{}}
		\toprule
		&{\sc english}&&{\sc english}\\ 
		\midrule
		jede 		& who &jach &which\\ 
		ježe 	& what 		& zajehu 	&why\\ 
		jehát 	& whom		& jiká 	&how many\\ 
		jehu 		& how		&jišká&how much\\ 
		jemí 		& when 		& jeně 	&to where\\ 
		jena 		& where 	& jení 	&from where\\ 
		\bottomrule
	\end{tabu}
\end{table}

\section{Indefinite pronouns}\index{indefinite pronoun}\label{sec:indef-pron}

Indefinite pronouns are pronouns that refer to an unspecified referent. They are
of two types in Iridian: universal (like \irdp{nět}{everyone}) and negative
(like \irdp{zide}{no one}). They have historically been formed by attaching the
prefix \ird{že-} and \ird{ní-} to interrogative pronouns respectively. The
forms, as with interrogative pronouns, roughly correspond to the case paradigms
for regular nouns, with additional forms for locative, temporal and distributive
uses (i.e., forms corresponding to \trsl{where}, \trsl{when} and \trsl{which}
respectively). The forms are listed in Table~\ref{tab:indef-pron}.

\begin{table}[h!]
	\sffamily\footnotesize
	\caption{Negative and universal pronouns.}
    \label{tab:indef-pron}
    \medskip
	\begin{tabu} to 0.9 \textwidth{@{}YY[1.2]YY[1.2]@{}}
		\toprule \addlinespace
		\multicolumn{2}{@{}l}{\sc negative} & \multicolumn{2}{@{}l}{\sc universal}\\ \addlinespace
		\midrule \addlinespace
		zide    & no one        & nět   & everyone \\ \addlinespace
		niho    & nothing       & niže  & everything \\ \addlinespace
		žehu    & by no means   & néhu  & by all means \\ \addlinespace
		žemě    & never         & nimě  & always \\ \addlinespace
		žena    & nowhere       & nina  & everywhere \\ \addlinespace
		žé      & not one       & nách  & each \\ \addlinespace
		\bottomrule
	\end{tabu}
\end{table}

\ird{Nět}, \ird{niže} and \ird{nách} as well as their negative counterparts
\ird{zide}, \ird{niho} and \ird{žé} can also be use adnominally, i.e., to
describe another noun phrase as quantifiers. \ird{Nět} and \ird{zide} are used
with animate, i.e., human, referents while \ird{niže} and \ird{niho} are used
with inanimate, i.e., non-human, referents. There is no such animacy
consideration with \ird{nách} and \ird{žé}. Thus one writes \irdp{nět
študent}{all students} and \irdp{niže jec}{all dogs} and not \ird{*niže študent}
or \ird{*nět jec}. However, both \irdp{nách študent}{each student} and
\irdp{nách jec}{each dog} are considered grammatical

Notice that in Iridian there are no indefinite pronouns that assert the
existence of its referent, similar to English \trsl{someone} or \trsl{something}
or \trsl{somebody}. Instead Iridian uses existential constructions (v.
\S\,\ref{sec:exst}) for the same purpose. For example, an English sentence like
\trsl{Someone is coming} is translated in Iridian as \ird{Semě ješ sčenžách}
(\Refl{}.\Acc{} \Exst{} arrive-\Av{}-\Ctp{}) where \ird{ješ} is the existential
particle. There are also no elective and dubitative existentials\footnote{From
Wikipedia: ``Elective existential pronouns are often used with negatives (I
can't see anyone), while dubitative existential pronouns are used in questions
when there is doubt as to the existence of the pronoun's assumed referent (Is
anybody here a doctor?).'' Both of these examples would use an existential
construction in Iridian.} like English \trsl{anyone} or \ird{anybody.} Instead
their function is satisfied by the equivalent interrogative pronoun modified by
\ird{každý}, a Slavic borrowing meaning \trsl{any} in Iridian, or by existential
constructions. For more examples of the former, see \S\,\ref{sec:int-pron}.

\section{Names}\index{name}\label{sec:names}
