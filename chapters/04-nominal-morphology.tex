\chapter{Nouns}

Nominal morphology in Iridian is relatively simpler compared to the corresponding process with verbs. Where possible, Iridian sentences are generally constructed to leave the noun or noun phrase unmarked.

\section{Grammatical categories}

\section{Number}\index{grammatical number}\index{plural}

Nouns in Iridian are not formally marked for number. Thus the word \ird{byl}, for example, can mean either \trsl{child} or \trsl{children} depending on the context. The same form is used when the noun is preceded by a numeral. Thus in Iridian one says \irdp{ona byl}{one child} and \irdp{hroná byl}{three children}.

Nevertheless, Iridian can express semantic plurality by using quantifiers, numerals, pluralizing particles or even through context alone. One such particle is \ird{ně}\label{sec:plurals}\footnote{Cf. \posscite{schachter1983} treatment of Tagalog\index{Tagalog} pluralizing particle \emph{mga}.}. \ird{Ně} is a proclitic and attaches to the left-most part of the noun phrase or the verb phrase it modifies. Throughout this book, \ird{ně} will be glossed as \Pl{}= for simplicity.

\pex
\begingl
    \gla ně ša zuštalí byl//
    \glb \Pl{}= \Dem{}.\Prox{} be:happy-\Att{} child //
    \glft \trsl{these happy children}//
\endgl
\xe

\ird{Ně} however could be understood to have three distinct uses. The first, as mentioned above, is to mark plurality. Alternatively, \ird{ně} could also be use as an approximative\index{approximation} (roughly equivalent to English \trsl{about}) when used with cardinal numbers or time expressions or as a honorific expletive\index{honorific}\index{expletive} to show politeness when used with proper names\index{proper names} or with some nouns (mostly related to kinship terms\index{kinship terms}). In its use for approximation, \ird{ně} is interchangeable with \irdp{u}{about}, although it is common in spoken speech to combine the two as an intensified construction. Preference is given to \ird{ně}, however, if the noun being modified is the topic of the sentence and must therefore remain unmarked.

\pex
\begingl
    \gla Ně mlazka-no scenžek?//
    \glb \Hon{}= brother\mk{-dim=q} arrive-\Av{}-\Pf{} //
    \glft \trsl{Was my brother the one who arrived?}//
\endgl
\xe

Note that when used with a cardinal number, \ird{ně} can only be understood to signify approximation, i.e., (\getfullref{appr.1}) can only mean \trsl{about three children} and not \trsl{three children}, as the latter would only be translated as \ird{hroná byl} without the clitic \ird{ně}.

As has been earlier mentioned, \ird{ně} is a proclitic\index{proclisis} and attaches to the left-most part of the noun phrase or verb phrase it modifies, including any modifier no matter how complex but excluding any proposition. In some cases, as can be seen in (b) and (c) below, the use of \ird{ně} to pluralize a noun can imply definiteness\index{definiteness}.

\pex
\a
\begingl\deftagex{pl}\deftaglabel{1}
    \gla {ně} za byla tóm//
    \glb \Pl{}= for child-\Acc{} child //
    \glft \trsl{books for children}//
\endgl
\a
\begingl\deftaglabel{2}
    \gla za {ně} byla tóm//
    \glb for \Pl{}= child-\Acc{} child //
    \glft \trsl{a book for (these) children}//
\endgl
\a
\begingl\deftaglabel{3}
    \gla {ně} za \textbf{ně} byla tóm//
    \glb \Pl{}= for \Pl{}= child-\Acc{} child //
    \glft \trsl{books for (these) children}//
\endgl
\xe

The use of \ird{ně}, however, is largely optional and where plurality can be implied from context, this particle is seen as redundant and is therefore dropped.

\ird{Ně} cannot be used with mass and uncountable nouns, as well as with abstract nouns.

\pex
\a
\begingl
\gla *Na duma ně ješ piaštou.//
\glb \Loc{} house \Pl{} \Exst{} food//
\glft \trsl{There is food in the house.}//
\endgl
\a
\begingl
\gla Na duma tohle ješ piaštou.//
\glb \Loc{} house much \Exst{} food//
\glft \trsl{There is a lot of food in the house.}//
\endgl
\xe

The particle \ird{ně} always precedes the noun it modifies, except in existential clauses where it comes before the existential particle \ird{ješ}\footnote{The sequence is pronounced as if written níješ \nt{"ni:jEC}}. \ird{Ně} can obviously not be used with the negative particle \ird{niho}.\index{niho}\index{existential construction}\index{ješ}

\pex
\a
\begingl
\gla ně bžem//
\glb \Pl{} bee//
\glft \trsl{bees}//
\endgl
\a
\begingl
\gla Ně ješ bžem.//
\glb \Pl{} \Exst{} bee//
\glft \trsl{There are bees.}//
\endgl
\a
\begingl
\gla *Ně niho bžem.//
\glb \Pl{} \N{}\Exst{} bee//
\glft \trsl{There are no bees.}//
\endgl
\xe

\index{pluralia tantum}
\ird{Ně} cannot be used with a limited number of nouns, mostly referring to paired body parts and related objects, which in the base form is understood to refer to the pair itself and thus cannot be pluralized. If the speaker wishes to explicitly refer to one piece of the pair, the noun \ird{noma} (an obsolete form of the word for one-half, now surviving only in this construction) and the genitive form of the body part.

\pex
\begingl
\gla Eg zaromnek.//
\glb eyes close-\Pv{}-\Pf{}//
\glft \trsl{(He) closed (his) eyes.}//
\endgl
\xe
\pex
\begingl
\gla Pohár dévit.//
\glb eyeglasses dirty//
\glft \trsl{(Your) eyeglasses are dirty.}//
\endgl
\xe
\pex
\begingl
\gla Ohví noma utieščál.//
\glb shoe-\Gen{} half \Refl{}-lose-\Av{}-\Cont{}//
\glft \trsl{The other pair of (his) shoe is missing.}//
\endgl
\xe

The base form is also used in generic statements where English would normally use the plural.\index{generic statements}\index{universals} When used with a proper noun \ird{ně} can be translated with the English construction \trsl{and others}. Note that this is different from the usage of \ird{ně} as a honorific.

\pex
\begingl
    \gla Ně Janc\v{e} gnaž uprubížice.//
    \glb \Pl{}= Janek-\Gen{} school \Refl{}-burn-\Av{}-\Pf{}-\Quot{} //
    \glft \trsl{I heard Janek's school burned down.}//
\endgl
\xe

\pex
\begingl
    \gla Ně Marek zázdalšek.//
    \glb \Pl{}= Marek \Neg{}-have:breakfast-\Av{}-\Pf{} //
    \glft \trsl{Marek and the others did not eat breakfast.}//
\endgl
\xe


\section{Definiteness}\index{definiteness}
Iridian does not have definite or indefinite articles; instead a noun or a noun phrase's definiteness is often expressed syntactically. This is discussed in detail in \S\,\ref{sec:definiteness}.

\section{The case system}

\subsection{Declension patterns}

There are four basic declension classes (or simply declensions) in Iridian, distinguished from one another by the final letter of the stem. Most Iridian words end with the stem final vowels -a, -e, -o, -ó or -i, or with a consonantal stem.


\begin{table}[h]
    \footnotesize\sffamily
    \caption{Paradigm endings for the six declension classes.}
    \medskip
    \begin{tabu} to \textwidth {@{}Y[2]YYYYYY@{}}
    \toprule\addlinespace
            {\sc case}          &{\sc i} &{\sc ii} & {\sc iii} &{\sc iv} &{\sc v} &{\sc vi} \\ \addlinespace
    \midrule\addlinespace
        Agentive    & -ám   & -em     & -am    & -óvam  & -ínam    & -ám\\ \addlinespace
        Patientive  & -e      & -ína  & -ie    & -óva   & -ína     & -a\\ \addlinespace
        Genitive    & -í    & -ení  & -e & -óví & -ení     & -í\\ \addlinespace
        Instrumental& -u      & -emu    & -u     & -óvím  & -imu       & -u\\ \addlinespace
    \bottomrule
    \end{tabu}
\end{table}

\subsection{Agentive case}\index{agentive case}

The agentive case (\Agt{}) is used to indicate the agent of an action where the agent is not the topic of the sentence.

\pex
\begingl
\gla Marek Lučkám vidnik.//
\glb Marek Luček-\Agt{} see-\Pv{}-\Pf{}//
\glft \trsl{Marek was seen by Luček.}//
\endgl
\xe

The agentive is also used in comparative constructions, where it indicates the point of reference for the comparison.

\index{comparison}\index{agentive of comparison}
\pex
\begingl
\gla Dá Mark\k{a} tám stroja.//
\glb \mk{1s.str} Marek-\Agt{} \Comp{} tall//
\glft \trsl{Marek is taller than me}//
\endgl
\xe

\subsection{Patientive case}

In general, the patientive case is used to mark the direct object of a verb that is in the agentive voice. Note that this usage implies that the direct object is indefinite. Where the direct object is definite, the verb is usually in the patientive voice and the direct object is unmarked.

\pex
\a \begingl
\gla Guláša piašček.//
\glb goulash-\Acc{} eat-\Av{}-\Pf{}//
\glft \trsl{(He) ate goulash.}//
\endgl
\a Compare this to:\\
\begingl
\gla Guláš piaštnik.//
\glb goulash eat-\Pv{}-\Pf{}//
\glft \trsl{(He) ate the goulash.}//
\endgl
\xe

The patientive is also used to mark the direct object when the verb is in the benefactive voice.

\pex
\begingl
\gla Ša vitamina piaštebik.//
\glb \mk{3s.anim} vitamin-\Acc{} eat-\Ben{}-\Pf{}//
\glft \trsl{(She) made him take (his) vitamins.}//
\endgl
\xe

%%%%
% TODO Definiteness and the patientive; use of genitive when the noun marked is indefinite

When used to mark the direct object, the patientive implies the definiteness of the noun. Where the noun is indefinite, the genitive is used instead.

\pex
\a\begingl
\gla Vaška piaščem.//
\glb cake-\Acc{} eat-\Av{}-\Pf{}//
\glft \trsl{I ate the cake.}//
\endgl
\a\begingl
\gla Vašcí piašček.//
\glb cake-\Gen{} eat-\Av{}-\Pf{}//
\glft \trsl{I ate some cake.}//
\endgl
\xe

The patientive is used with the particle \ird{na} to form a compound locative case, which is itself used to indicate a general location.

\pex
\begingl
\gla Tomáš na byra.//
\glb Tomáš \Loc{} office-\Acc{}//
\glft \trsl{Tomáš is at the office.}//
\endgl
\xe

The patientive is also used with some prepositions, often indicating direction or movement. The most common of these are \ird{na} used to indicate a general location (\irdp{na byra}{at the office}), \ird{za} which roughly corresponds to the English `for' or `for the benefit of' (\irdp{za Marka}{for Marek}), \ird{u} used to indicate proximity (\irdp{u gara}{near the train station}), and \ird{do} which indicates movement towards a location (\irdp{do byra}{to the office}).

\subsection{Genitive Case}\index{genitive}

The simplest use of the genitive case is to indicate ownership or possession.
When used this way, the noun marked in the genitive must always procede the noun
it modifies.

\pex
\irdp{Marcí dum}{Marek's house}\\
\irdp{mámcí hašek}{my mother's bag}\\
\irdp{ša študencí tóm}{this student's book}\\
\xe

Demonstratives\index{demonstrative} and other modifers must always come before
the whole noun phrase and cannot split the possessor from the possessee. An
exception to this rule is the clitic \ird{ně}, which comes immediately before
the noun it pluralizes\index{plural}.

\pex
\a  \irdp{ša študencí tóm}{the/a book of this student}\\
    \irdp{to študencí tóm}{this book of the student}
\a  \irdp{ně študencí tóm}{the students' book}\\
    \irdp{študencí ně tóm}{the student's books}
\xe

The genitive is also used as a partitive 


\subsubsection{Genitive of material}

\ex
\irdp{kuní prosc}{silver spoon}\\
\irdp{be}
\xe

\subsubsection{Genitive of the whole}
The genitive can also be used to indicate

\pex
\begingl
\gla na kraštolí dnóva//
\glb \Loc{} train:station-\Gen{} front//
\glft \trsl{in front of the train station}//
\endgl
\xe

Note that the patientive and not the genitive case is used when quantifying a part of the whole.

\pex
\a
\begingl
\gla *žnohoušce hroná//
\glb student-\Gen{} three//
\glft \trsl{three of the students}//
\endgl
\a
\begingl
\gla na žnohoušca hroná//
\glb \Loc{} student-\Gen{} three//
\glft \trsl{three of the students}//
\endgl
\xe

Nevertheless when quantifying a noun per se, and not in relation to a whole, the uninflected form of the quantifier is used (mostly using indefinite quantifiers such as \trsl{many}, \trsl{a lot}, etc.). If however, the quantification involves a countable unit or division of the noun, the genitive is used, but such unit or division must be further quantified by a numeral or an indefinite quantifier.

\pex
\a
\begingl
\gla Na kroumašta po zma ješ pivo.//
\glb \Loc{} refrigerator-\Acc{} still few \Exst{} beer//
\glft \trsl{There's still some beer left in the refrigerator.}//
\endgl
\a
\begingl
\gla Ona pive štava unarížčem.//
\glb one beer-\Gen{} mug-\Acc{} \Refl{}-order-\mk{av-pv-1s}//
\glft \trsl{I ordered a mug of beer.}//
\endgl
\xe

\subsubsection{Genitive of movement}

The genitive is also used to indicate movement away from somewhere.

\pex
\a
\begingl
\gla Dumí palžek.//
\glb house-\Gen{} leave-\Av{}-\Pf{}//
\glft \trsl{I left the house.}//
\endgl
\a
\begingl
\gla Dum palzinek.//
\glb house leave-\Pv{}-\Pf{}//
\glft \trsl{I left the \emph{house}.}//
\endgl
\xe

\subsection{Instrumental case}

The instrumental case (glossed \Ins{})

The following prepositions take the instrumental case: \ird{še} \trsl{with}

\pex
\begingl
\gla Za bolta še Janu stóž\k{a}c.//
\glb for party-\Acc{} with Jan-\Ins{} go-\mk{av-ctpv}//
\glft \trsl{(I am) coming to the party with Jan.}//
\endgl
\xe

\section{Personal Pronouns}\index{personal pronouns}\index{pronouns}

Personal pronouns are a special class of nouns used to refer and/or replace other nouns or noun phrases. Personal pronouns are marked for person, number and case, and partially for animacy\index{animacy}, although third-person forms are more properly analyzed as demonstratives\index{demonstrative}. In addition, personal pronouns have three forms: (1) an invariable strong form, used when the pronoun is the topic of the sentence; (2) a weak form; and (3) a clitic form.

\begin{table}[h!]
    \footnotesize\sffamily
	\caption{Personal pronouns in Iridian}
	\medskip
	\begin{tabu} to 0.8 \textwidth {@{}Y[2.5]YYYY@{}}
		\toprule 
        {\sc form} & {\sc 1s} & {\sc 2s} & {\sc 1pl} & {\sc 2pl}\\
		\midrule
        {\sc strong form}  & dá      & já      & mé      & tová \\
        {\sc weak form}    &           &           &           & \\
        \quad Agentive      & dám     & jám     & mám     & tám\\
        \quad Patientive    & do        & jí      & mně      & te\\
        \quad Genitive      & že    & jení    & mneví   & teví\\
        \quad Instrumental  & du        & jemu      & mo        & tve\\
        \bottomrule
	\end{tabu}

\end{table}

\subsection{Grammatical person}\index{person, grammatical}
Iridian pronouns

\subsection{Strong form}\index{strong form}

The strong form of a personal pronoun (glossed \mk{str}) is used when the pronoun is used as the topic of the sentence. The strong form is indeclinable.

\subsection{Weak form}

\subsection{Clitic form}\index{clitic form}

\subsection{Third-Person Pronouns and Demonstratives}


\subsection{Ellipsis}
Iridian is an extremely pro-drop language, with pronouns supplied only if not inferrable from context.


\subsection{The Reflexive \ird{se}}

The reflexive \ird{se} is used to refer back to the topic of the sentence. Se is often used with the reflexive voice, although the use of \ird{se} often implies a greater disjunction between the actor and the patient. Where the reflexive voice has a primarily sociative meaning, as in verbs with a defunct active voice, \ird{se} is used to form a true reflexive construction. In cases where the use of the reflexive is not syntactically required, \ird{se} may nevertheless still be used as a form of emphasis, similar to the use of `own' in English.

\pex    \a \irdp{Udušek}{I took a bath.}
        \a \irdp{Se udušek}{I bathed myself.}
\xe

\pex
        \a \irdp{Guláše upiašček}{I ate some goulash.}
        \a \ljudge{?} \irdp{Se upiašček}{I ate myself.}
\xe


\begin{table}
    \footnotesize\sffamily
    \caption{Declension of the reflexive pronoun \ird{se}.}
    \medskip
    \begin{tabu} to 0.5 \textwidth {@{}Y[3]Y@{}}
        \toprule
        Unmarked        & se\\
        Agentive        & snám\\
        Patientive      & sní\\
        Genitive        & si\\
        Instrumental    & sem\\
        \bottomrule
    \end{tabu}
\end{table}


\section{Demonstratives}\index{demonstratives}

\begin{table}
    \footnotesize\sffamily
	\caption{Demonstrative pronouns in Iridian.}
    \medskip
	\begin{tabu}to 0.7\textwidth{@{}YYYY@{}}
		\toprule
						& {\sc animate}	& {\sc inanimate}	&{\sc locative}\\ 
		\midrule 
		Proximal		& ša		& to 				& tak\\ 
		Medial			& ón				& ján				& jení\\ 
		Distal			& dní		& jón				& joní\\ 
		\bottomrule
		\label{dem-prons}
	\end{tabu}
\end{table}

Iridian does not have a separate class of third-person pronouns. Instead it uses a set of demonstrative pronouns, whose deictic\index{deixis} function is both spatial\index{spatial deixis|see{deixis}} and anaphoric\index{anaphora}. Iridian makes a three-way distinction among demonstratives, similar to French\index{French} or Portuguese\index{Portuguese} for example, distinguishing between proximal (near the speaker), medial (near the addressee) and distal (far from both speaker and addressee) forms. In addition, Iridian makes an animacy distinction with demonstratives, with one set of demonstratives used with human referents and another with non-human referents, as seen in Table \ref{dem-prons}, but are unmarked for either number or gender.

Demonstratives can be used adnominally, to modify a noun phrase, or pronominally, to replace one.

\begin{multicols}{2}
    \pex
    \a
    \begingl
    \gla ša byl//
    \glb \mk{dem.prox.anim} child//
    \glft \trsl{this child}//
    \endgl
    \a
    \begingl
    \gla ša bylem//
    \glb \mk{dem.prox.anim} child-\First{}\Sg{}//
    \glft \trsl{this child of mine}//
    \endgl
    \a
    \begingl
    \gla Ša bylem.//
    \glb \mk{dem.prox.anim} child-\First{}\Sg{}//
    \glft \trsl{This (person) is my child.}//
    \endgl
    \a
    \begingl
    \gla *To bylem//
    \glb \mk{dem.prox.inan} child-\First{}\Sg{}//
    \glft \trsl{This (thing) is my child.}//
    \endgl
    \xe
\end{multicols}

Unlike true personal pronouns, demonstratives do not have a separate strong form and clitic form. They are fully declined however, with the declined forms being highly irregular, as can be seen in Table \ref{dem-conj}.

\pex
\a\deftagex{obv}
\begingl
\gla ci mlaz a dní maty//
\glft \trsl{this person's brother and that person's mother}//
\endgl
\a\deftaglabel{obv1}
\begingl
\gla Dá je svou je dnu zapreví.//
\glft \trsl{I am as old as either this person or that person.}//
\endgl
\xe

\begin{table}
\footnotesize\sffamily
	\caption{Declension of demonstratives.}
    \medskip
	\begin{tabu}to \textwidth{@{}Y[2]YYYYYY@{}}
		\toprule
						& {ša}	& {ón}	&{dní}& {to}	& {ján}	&{jón}\\
		\midrule \addlinespace
		Agentive&šem&nám&dněm&etom&ján&jón\\\addlinespace
		Patientive&šá&ona&dná&toha&jina&jinóva\\\addlinespace
		Genitive&ci&oní&dní&cie&ně&nohe\\\addlinespace
		Instrumental&svou&nu&dnu&etu&nu&nohu\\\addlinespace
		\bottomrule
		\label{dem-conj}
	\end{tabu}
\end{table}

The three-way distinction between demonstratives allows Iridian to disambiguate between an obviative\index{obviation} third person and a proximate third person, using the distal and the proximal demonstrative respectively. Consider for example the two sentences in English below:

\pex
\a He saw his dog.
\a He saw his own dog.\smallskip
\xe

The \emph{his} in the first sentence is ambiguous, as it can refer to either the subject or an implied fourth person. That the second \emph{his} refers back to the subject can be made unequivocal by the addition of the word \emph{own}, as in the second sentence. Compare this with the following sentences in Czech:

\begin{multicols}{2}
  \pex
  \a
  \begingl
  \gla Viděl jeho pes.//
  \glft \trsl{He saw his dog.}//
  \endgl
  \a \begingl
  \gla Viděl své pes.//
  \glft \trsl{He saw his own dog.}//
  \endgl
  \xe
\end{multicols}

Although the English translation of the first sentence may still appear ambiguous, we can see that Czech does away with the ambiguity by using the third person pronoun \ird{jeho} exclusively to signify that the referent is different from the subject, and requiring the use of a separate pronominal form (in this case the reflexive) when the referent and the subject are the same. Iridian, on the other hand, treats this in a diametrically opposite way, i.e., the same pronoun form is used when the subject and the referent are the same, with the obviative form being used otherwise. The sentences in Czech above will therefore be translated in Iridian as follows:

\pex
\a
\begingl
\gla Dní jec vdinek.//
\glb \mk{dem.dist.anim.gen} dog see-\Pv{}-\Pf{}//
\glft \trsl{He saw his (other person's) dog.}//
\endgl
\a \begingl
\gla Ci jec vdinek//
\glb \mk{dem.dist.inan.gen} dog see-\Pv{}-\Pf{}//
\glft \trsl{He saw his own dog.}//
\endgl
\xe

Perhaps we can better understand the distinction between obviative and proximate forms by re-examining example (\getfullref{obv.obv1}) above. The previous examples in Czech remained unambiguous because there are at most two unique arguments in the sentence. In example (\getfullref{obv.obv1}), however, the subject of the sentence is distinct from either the proximate referent or the distal referent.

\ex[exno={\getfullref{obv.obv1}}]
\begingl
\gla Dá je svou je dnu zapreví.//
\glft \trsl{I am as old as either this person or that person.}//
\endgl
\xe

The translation in the gloss demonstrates how idiomatic English uses periphrastic forms to eliminate this ambiguity, although in the spoken language the purely deictic \trsl{I am as old as either him or him} is equally acceptable, with the blanks filled in most likely by non-verbal cues. In Iridian, however, this distinction is not optional, and the following sentence, for example, would be considered ungrammatical:

\ex
\begingl
\gla *Dá je svou je svou zapreví.//
\glft \trsl{I am as old as either him or him.}//
\endgl
\xe

\section{Use of Personal Pronouns}

\subsection{T-V Distinction}\index{politeness}\index{T-V distinction}\index{forms of address}

Iridian has three forms of address: the informal, the polite, and the formal.

The second person singular pronoun \ird{já} is used to address friends, relatives or children. When addressing a stranger or an acquaintance with whom you want to maintain social distance or be polite without being too formal, the second person plural pronoun \ird{tévit} is used. The polite form is also used when addressing God/gods. In more formal settings, the third-person plural pronoun \ird{ože} is used.


\section{Indefinite pronouns and quantifiers}


\section{Interrogative pronouns}\index{wh-question@\emph{wh}-question}\index{interrogative pronoun}\label{sec:int-pron}

\begin{table}[h!]
	\sffamily\footnotesize
	\caption{Interrogative pronouns in Iridian.}
    \medskip
	\begin{tabu} to 0.8\textwidth{@{}YYYY@{}}
		\toprule
		&{\sc english}&&{\sc english}\\ 
		\midrule
		jede 		& who &jach &which\\ 
		ježe 	& what 		& zajehu 	&why\\ 
		jehát 	& whom		& jiká 	&how many\\ 
		jehu 		& how		&jišká&how much\\ 
		jemí 		& when 		& jeně 	&to where\\ 
		jena 		& where 	& jení 	&from where\\ 
		\bottomrule
	\end{tabu}
\end{table}

\section{Negative and Universal Pronouns}\index{negative pronouns}\index{universal pronouns}

Negative pronouns are historically formed by attaching the prefix \ird{že} before interrogative pronouns, and universal pronouns by attaching the prefix \ird{ní-}

\begin{table}[h!]
	\sffamily\footnotesize
	\caption{Correspondence of interrogative, negative and universal pronouns.}
    \medskip
	\begin{tabu} to \textwidth{@{}YY[1.2]YY[1.2]YY[1.2]@{}}
		\toprule
		\multicolumn{2}{@{}l}{\sc interrogative}& \multicolumn{2}{@{}l}{\sc negative} & \multicolumn{2}{@{}l}{\sc universal}\\ 
		\midrule
		jede 		& who & neiže & no one & nět & everyone\\ 
		ježe 	& what 		& niho & nothing&níže&everything\\ 
		jehu 		& how		&žehu&by no means&néhu&by all means\\ 
		jemí 		& when 		& žemie&never&nimie&always \\
		jena 		& where 	& žena&nowhere&nina&everywhere \\ 
		jach &which&žé&not one&nách&each\\ 
		\bottomrule
	\end{tabu}
\end{table}


\section{Names}\index{name}\label{sec:names}
