\chapter{Lexicon and Derivation}

\section{Nominal Derivation}
\section{Verbal Derivation}
\section{Linguistic Borrowing}
A significant portion of the vocabulary of Iridian comes from loanwords from neighbouring languages, especially German, Czech and Polish, and to a lesser extent Hungarian. Like most languages from the area, Iridian also has a notable portion of its vocabulary derived from French and Latin, mostly scientific and academic terms. In addition, after the advent of the internet, there has been an increasing amount derived from English and other world languages as well. Most loanwords are assimilated to conform with Iridian phonological rules, although most recent loanwords generally maintain the phonology of the language they were originally borrowed from.

In most cases, the loanwords or their assimilated forms coexist with their native Iridian counterparts. Often their usage is interchangeable

\subsection{Czech, Polish, and Other Slavic Languages}
Due to their proximiity, Iridian has had the most contact with

\subsection{Latin}

\subsection{Calques}\label{sec:calques}\index{calque}
