\chapter{Lexicon and Derivation}

\section{Nominal Derivation}
\section{Verbal Derivation}
\section{Linguistic Borrowing}\index{loanword}
A significant portion of the vocabulary of Iridian comes from loanwords from neighbouring languages, especially German\index{German}, Czech\index{Czech} and Polish\index{Polish}, and to a lesser extent Hungarian\index{Hungarian}. Like most languages from the area, Iridian also has a notable portion of its vocabulary derived from French\index{French} and Latin, mostly scientific and academic terms. In addition, after the advent of the internet, there has been an increasing amount derived from English and other world languages as well. Most loanwords are assimilated to conform with Iridian phonological rules, although most recent loanwords generally maintain the phonology of the language they were originally borrowed from.

In most cases, the loanwords or their assimilated forms coexist with their native Iridian counterparts. Often their usage is interchangeable

\subsection{German and Other Germanic Languages}

Assimilation\index{assimilation of loanwords} of German phonemes that do not exist in Iridian is generally consistent, and is subject to the rules discussed in this section.

German has three falling diphthongs (\cite{wiese1996}): \bt{a\tsa{I}}, \bt{a\tsa{U}} and \bt{O\tsa{Y}}, none of which have exact equivalents in Iridian. Nonetheless \bt{a\tsa{U}} assimilates to Iridian \bt{a\tsa{u}} (both spelled $\langle$au$\rangle$). \bt{a\tsa{I}} does occur marginally in Iridian, but most instances of \bt{a\tsa{I}} in German become either \bt{a:} or \bt{e\tsa{I}}.\footnote{Or just \nt{\"a:} and \nt{e:} given the monophthongization of \bt{e\tsa{I}} in most dialects.} Finally \bt{O\tsa{Y}} is never assimilated to the marginal \bt{O\tsa{I}} but becomes either \bt{e\tsa{I}} or \bt{a\tsa{u}}.

\ex
Assimilation of German diphthongs:\\
\irdp{Karl\v{s}t\'an}{Charles Castle} fr. \emph{Karlstein}\\
\irdp{B\'erna}{Bayern} fr. \emph{Bayern}\\
\irdp{bedautum}{significance, importance} fr. \emph{Bedeutung}\\
\irdp{Freid}{Freud} fr. \emph{Freud}
\xe

The raised vowels $\langle$\"a$\rangle$ and $\langle$\"o$\rangle$ become \bt{e:} $\langle$\'e$\rangle$ (or sometimes \bt{I} $\langle$i$\rangle$) in Iridian while $\langle$\"u$\rangle$ become \bt{y} $\langle$y$\rangle$.

\ex
Assimilation of German \emph{umlaut} vowels:\\
\irdp{Karl\v{s}t\'an}{Charles Castle} fr. \emph{Karlstein}\\
\irdp{B\'erna}{Bayern} fr. \emph{Bayern}\\
\irdp{bedautum}{significance, importance} fr. \emph{Bedeutung}\\
\irdp{Freid}{Freud} fr. \emph{Freud}
\xe


\subsection{Czech, Polish, and Other Slavic Languages}
Due to their proximiity, Iridian has had the most contact with

\subsection{Romance Languages and Greek}

Greek/Latin \emph{-ismus/-ismos} becomes Iridian \ird{-i\v{z}mus}. Most are derived indirectly from the German.

\ex
\irdp{turi\v{z}mus}{tourism}\\
\irdp{komuni\v{z}mus}{communism}\\
\irdp{bavti\v{z}mus}{baptism}\\
\irdp{hri\v{s}tani\v{z}mus}{Christianity}\\
\irdp{anarhi\v{z}mus}{anarchism}
\xe

\subsection{English}


\subsection{Calques}\label{sec:calques}\index{calque}
