\chapter{Words and Word-Formation}

\section{Introduction}

In \S\,\ref{sec:wordclasses} we discussed how Iridian words can be classified into two broad groups: content words and function words. Due to their very nature nature, function words are largely invariable in form; content words, on the other hand, vary constantly and their form reflect the grammatical information they carry. We call this system of variation {\sc inflection}, and it is one of the ways languages like Iridian form new words from pre-existing ones.\footnote{By ``new" here we mean a form different from the original word; but since inflection is primarily a grammatical operation, the difference in meaning occasioned by inflection is often not significant.}

In this chapter we will discuss two more ways to form new words in Iridian: {\sc derivation} and {\sc compounding} (cf. \cite{booij2005}; \cite{velupillai2012}: 115). Compounding involves the amalgamation of multiple words to form a new word; this is discussed in detail in section \S\,\ref{sec:compounding}. Derivation, on the other hand, involves modifying a word with affixes (in a similar way to inflection) to change its meaning. Unlike inflectional affixes, however, derivational affixes do not carry any grammatical information

\section{Nominal Derivation}
\subsection{Diminutives and Augmentatives}\label{sec:diminutive}
\index{diminutive}\index{augmentative}

Unlike English, but similar to most Slavic and Romance languages, Iridian frequently employs {\sc diminutives} (and to a lesser degree {\sc augmentatives}). The most basic form of the diminutive is formed with the suffix \ird{-ka} (or \ird{-cka} after vowels), which most linguists agree is a non-native morpheme, and is most likely borrowed from Slavic.

\ex
\irdp{jec}{dog} $\rightarrow$ \irdp{jecka}{doggy, little dog}\\
\irdp{pap\'ir}{paper} $\rightarrow$  \irdp{pap\'irka}{piece of paper}\\
\irdp{dum}{house} $\rightarrow$  \irdp{dumka}{little house}\\
\irdp{k\'av\'e}{coffee} $\rightarrow$  \irdp{k\'av\'ecka}{espresso}
\xe

Diminutives are used to express that something is small or insignificant. In the spoken language, however, it is more common to use the diminutive to express endearment or affectation. This same usage makes it possible to use the diminutive patronizingly, to belittle or to be dismissive. With mass nouns, the diminutive is also often used to refer to a small quantity of something.

\pex
\a To express affection:\\
\begingl
\gla Jecka do vezdalnik.//
\glb dog-\mk{dim} \mk{1s.pat} to:gift-\mk{pv-pf}//
\glft \trsl{This dog was given to me as a gift.}//
\endgl

\a To dismiss or belittle:\\
\begingl
\gla To na prov\'izork\'a niho z\'abor.//
\glb this \mk{loc} professor-\mk{dim-pat} \mk{nexst} knowledge//
\glft \trsl{This so-called ``professor'' doesn't know a thing.}//
\endgl

\a To express a small quantity of something:
\xe

When referring to members of one's own family, that of a friend's, or of the person being addressed, the diminutive form is also used. Most kinship terms have irregular forms and are listed in \S\,\ref{sec:nuclear family}. In colloquial Iridian\index{colloquial Iridian} proper names are also often marked as diminutives, with the variant suffix \ird{-ik/-k} being more common. The first-person plural clitic \irdp{-\'om}{our} is often used in conjunction with the diminutive. In addition to this, most names also have irregular diminutive forms and variants which are discussed in detail in \S\,\ref{sec:names}.

\ex
\ird{Janek} $\rightarrow$ \ird{Jan\v{c}ik}, \ird{Jan\v{c}ik\'om}\\
\ird{Marek} $\rightarrow$ \ird{Mar\v{c}ik}, \ird{Mar\v{c}ik\'om}\\
\ird{Tom\'a\v{s}} $\rightarrow$ \ird{Tom\'a\v{s}ik}, \ird{Tom\'a\v{s}k\'om}\\
\ird{Tereza} $\rightarrow$ \ird{Tere\v{z}ik}, \ird{Tere\v{z}k\'om}\\
\ird{Ag\'ata} $\rightarrow$ \ird{Ag\'a\v{c}ik}, \ird{Ag\'a\v{c}k\'om}
\xe

Double and triple diminutives are also common, formed using \ird{-(i)\v{s}ka} and \ird{-(i)si\v{c}ka}, respectively. Quadruple and quintuple diminutives are also possible (formed using \ird{-(i)nisi\v{c}ka} \ird{-(i)ni\v{z}esi\v{c}ka}, respectively), although their usage is not as neutral, and would often be used to mock or to exaggerate.\footnote{The suffixes \ird{-(i)\v{s}ka} and \ird{-(i)si\v{c}ka} are of Slavic origin while \ird{-(i)nisi\v{c}ka} \ird{-(i)ni\v{z}esi\v{c}ka} are Iridian innovations.}

Augmentatives\index{augmentative} are also used, although their usage is not as common as diminutives and their usage is often limited as pejoratives\index{pejorative}. Augmentatives are formed with the suffixes \ird{-(\v{z})ul\'am} or \ird{-(\v{z})urn\'am} or \ird{-(\v{z})ul\'ahma\v{s}}. These forms are not interchangeable and in general the longer the augmentative suffix is, the more pejorative is its connotation.

\subsection{Nouns From Nouns}

The suffix \ird{-(e)vnice} is used in deriving nouns from proper nouns. When used with names of places it generally has the meaning \trsl{resident of} or \trsl{native of}. Countries whose name end in the suffix \ird{-\'oma} drop the suffix first before adding \ird{-(e)vnice}.

\begin{multicols}{2}
  \ex
  \irdp{ircevnice}{Iridian}\\
  \irdp{ma\v{z}arevnice}{Hungarian}\\
  \irdp{\v{c}i\v{z}evnice}{Czech}\\
  \irdp{pol\v{s}\v{c}evnice}{Polish}\\
  \irdp{mu\v{s}houvnice}{Muscovite}\\
  \irdp{n\'evior\v{c}evnice}{New Yorker}\\
  \irdp{tur\v{c}evnice}{Turk}\\
  \irdp{ru\v{z}evnice}{Russian}\\
  \irdp{ameri\v{c}evnice}{American}\\
  \irdp{anglevnice}{English}
  \xe
\end{multicols}


The suffix \ird{-(h)\'ar} from the Czech \emph{-\'ar/-\'a\v{r}} indicates agency. It is often used to form nouns relating to professions, although it may appear with Latinate loanwords as the assimilated form of the French \emph{-aire}.

\ex
\irdp{revoluceh\'ar}{revolutionary} fr. \irdp{revoluce}{revolution}\\
\irdp{milion\'ar}{millionaire} fr. \irdp{milion}{million}\\
\irdp{trav\'ar}{baker} fr. \irdp{trava}{bread}\\
\irdp{kostl\'ar}{fisherman} fr. \irdp{kostel}{fish}\\
\irdp{zn\'ameh\'ar}{smith} fr. \irdp{zn\'ame}{metal}\\
\irdp{zak\'ar}{sailor} fr. \irdp{zak}{sea}\\
\irdp{ba\v{s}ketb\'ol\'ar}{basketball player} fr. \irdp{ba\v{s}ketb\'ol}{basketball}\\
\irdp{mi\v{s}t\'ar}{warrior} fr. \irdp{mie\v{s}t}{war}\\
\irdp{\'akceh\'ar}{shareholder} fr. \irdp{\'akce}{share of stock}\\
\irdp{nepod\'ar}{bureaucrat} fr. \irdp{nepod}{position, rank}
\xe

Variants of \ird{-(h)\'ar} include \ird{-(h)er} and \ird{-(h)or}, although their usage is much more limited.

\ex
\irdp{sen\'ator}{senator} fr. \irdp{sen\'at}{senate}\\
\irdp{avi\'ator}{aviator} fr. \irdp{aviace}{aviation}\\
\irdp{helder}{salaryman} fr. \irdp{held}{wage, salary}, itself from German \emph{Geld}
\xe

Another common suffix used to form agent nouns is \ird{-ist}. This suffix is often used on nouns ending in \ird{-i\v{z}mus}.

\ex
\irdp{komunist}{communist} fr. \irdp{komuni\v{z}mus}{communism}\\
\irdp{modernist}{modernist} fr. \irdp{moderni\v{z}mus}{modernism}\\
\irdp{avtist}{cabdriver} fr. \irdp{avt}{car}\\
\irdp{ma\v{s}inist}{engineer} fr. \irdp{ma\v{s}ina}{machine, engine}\\
\irdp{bankist}{banker} fr. \irdp{bank}{bank}\\
\irdp{\v{z}urn\'alist}{journalist} fr. \irdp{\v{z}urn\'al}{magazine}
\xe

The most common way of forming abstract nouns is through the suffix \ird{-(i)\v{z}n\'am}.

\ex
\irdp{vidli\v{z}n\'am}{slavery} fr. \irdp{videl}{slave}\\
\irdp{tieho\v{z}n\'am}{divinity} fr. \irdp{tieho}{god}\\
\irdp{te\v{s}ki\v{z}n\'am}{membership} fr. \irdp{te\v{s}ke}{member}\\
\irdp{stulti\v{z}n\'am}{puberty} fr. \irdp{st\'olet}{teenager}
\xe


\subsection{Nouns From Verbs and Adjectives}
\section{Verbal Derivation}

\section{Compounding}\index{compounding}\index{compound word}\label{sec:compounding}

\section{Linguistic Borrowing}\index{loanword}
A significant portion of the vocabulary of Iridian comes from loanwords from neighbouring languages, especially German\index{German}, Czech\index{Czech} and Polish\index{Polish}, and to a lesser extent Hungarian\index{Hungarian}. Like most languages from the area, Iridian also has a notable portion of its vocabulary derived from French\index{French} and Latin, mostly scientific and academic terms. In addition, after the advent of the internet, there has been an increasing amount derived from English and other world languages as well. Most loanwords are assimilated to conform with Iridian phonological rules, although most recent loanwords generally maintain the phonology of the language they were originally borrowed from.

In most cases, the loanwords or their assimilated forms coexist with their native Iridian counterparts. Often their usage is interchangeable

\subsection{German and Other Germanic Languages}

Like its neighboring Czech Republic and Slovakia, Iridia has had significant contact with the German-speaking peoples of Central Europe throughout the centuries, leading to a significant German influence on the language's vocabulary. Most of the words of German origin now in Iridian entered the language in the 16th century when the Duchy of Iridia (then a part of the Crown of Bohemia) was absorbed into the Habsburg Monarchy, with the influence continuing into the late 19th century. Starting the 1880s\footnote{Some sources point to the defeat of Austria and the Peace of Prague in 1866 as the beginning of the `de-Germanization' of Iridia. Nevertheless it was not until the Edict of Julmonc (then Olm\"utz) was issued in March 1882 that the de-Germanization of the Iridian language was formalized by Iridian state authorities.} however (in large part due to the spread of Romanticism and nationalism in the region), and until the collapse of the Austro-Hungarian Empire, attempts have been made to `de-Germanize' Iridian vocabulary by replacing German vocabulary with words from the native stock or more often with calques\index{calque}. This `de-Germanization' continued well into the first half of the 20th century, as a result of which, German loanwords in Iridian in constant use have significantly decreased from what they have been in the 16th to the 18th centuries, with most words of Germanic origin now considered archaic and are used primarily as an affectation (cf. English \emph{thou}, \emph{shew} and \emph{methinks}, for example).

Assimilation\index{assimilation of loanwords} of German phonemes that do not exist in Iridian is generally consistent, and is subject to the rules discussed in this section.

German has three falling diphthongs (\cite{wiese1996}): /aɪ̯/, /aʊ̯/ and /ɔʏ̯/, none of which have exact equivalents in Iridian. Nonetheless \bt{a\tsa{U}} assimilates to Iridian \bt{a\tsa{u}} (both spelled $\langle$au$\rangle$). \bt{a\tsa{I}} does occur marginally in Iridian, but most instances of \bt{a\tsa{I}} in German become either \bt{a:} or \bt{e\tsa{I}}.\footnote{Or just \nt{\"a:} and \nt{e:} given the monophthongization of \bt{e\tsa{I}} in most dialects.} Finally \bt{O\tsa{Y}} is never assimilated to the marginal \bt{O\tsa{I}} but becomes either \bt{e\tsa{I}} or \bt{a\tsa{u}}.


\ex
Assimilation of German diphthongs:\\
\irdp{Karl\v{s}t\'an}{Charles Castle} fr. \emph{Karlstein}\\
\irdp{B\'erna}{Bayern} fr. \emph{Bayern}\\
\irdp{bedautum}{significance, importance} fr. \emph{Bedeutung}\\
\irdp{Freid}{Freud} fr. \emph{Freud}
\xe

The raised vowels $\langle$\"a$\rangle$ and $\langle$\"o$\rangle$ become \bt{e:} $\langle$\'e$\rangle$ (or sometimes \bt{I} $\langle$i$\rangle$) in Iridian while $\langle$\"u$\rangle$ become \bt{y} $\langle$y$\rangle$.
