
\chapter*{Abbreviations}
\addcontentsline{toc}{chapter}{Abbreviations}
\printglossary[style=myglosses,type=\leipzigtype, title={Glossing abbreviations}]

\smallskip

{\small
\begin{itemize}
\item[*] These are glosses for grammatical terms seen in examples from other languages that are not in use in Iridian.
\item[**] These do not represent grammatical categories used in Iridian but were chosen \emph{hanc ob causam} to approximate the grammatical function of various particles.
\end{itemize}}

\bigskip
\printglossary[style=langlist,type=\acronymtype, title={Abbreviation of language names}]

\clearpage

\section*{Abbreviation of language names}
\iffalse
\begin{longtabu} {YY[2.5]}
	Cz.		& Czech\\
	Eng.	& English\\
	Fr.		& French\\
	Ger.	& German\\
	Gk.		& Greek\\
	Hu.		& Hungarian\\
	It.		& Italian\\
	Lat.	& Latin\\
	OCS		& Old Church Slavonic\\
	Pol.	& Polish\\
	Rus.	& Russian\\
	Sl.		& Common Slavic/Slavonic\\
	Slk.	& Slovakian\\
	Uk.		& Ukrainian\\
\end{longtabu}


\section*{Other Symbols}
\begin{longtabu} {YY[2.5]}
	C 		& consonant\\
	C\sx{j}	& palatalized consonant\\
	D 		& voiced stop\\
	N 		& nasal consonant\\
	P 		& stop\\
	T 		& unvoiced stop\\
	V		& vowel\\
	Ṽ	 	& nasalized vowel\\
	V\tss{u}& unstable vowel\\
	\orth{~}& orthographic representation\\
	/~/		& phonemic transcription\\
	\nt{ }	& phonetic transcription\\
	\sim	& alternates with\\
\end{longtabu}
\fi