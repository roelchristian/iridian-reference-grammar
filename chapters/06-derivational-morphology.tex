\chapter{Derivational morphology}\index{word}\index{word formation}\index{derivational morphology}

\section{Introduction}

In \S\,\ref{sec:wordclasses} we discussed how Iridian words can be classified
into two broad groups: content words\index{content word} and function
words\index{function word}. Due to their very nature, function words are largely
invariable in form; content words, on the other hand, vary constantly and their
form reflect the grammatical information they carry. We call this system of
variation {\scshape inflection},\index{inflection} and it is one of the ways
languages like Iridian form new words from pre-existing ones.\footnote{By ``new"
here we mean a form different from the original word; but since inflection is
primarily a grammatical operation, the difference in meaning occasioned by
inflection is often not significant.}

In this chapter we will discuss two more ways to form new words in Iridian: {\sc
derivation}\index{derivation} and {\sc compounding}\index{compound word} (cf.
\cite{booij2005}; \cite{velupillai2012}: 115). Compounding involves the
amalgamation of multiple words to form a new word; this is discussed in detail
in section \S\,\ref{sec:compounding}. Derivation, on the other hand, involves
modifying a word with affixes (in a similar way to inflection) to change its
meaning. Unlike inflectional affixes, however, derivational affixes do not carry
any grammatical information

\section{Nominal derivation}
\subsection{Diminutives and augmentatives}\label{sec:diminutive}
\index{diminutive}\index{augmentative}

Unlike English,\index{English} but similar to most Slavic\index{Slavic
languages} and Romance languages\index{Romance languages}, Iridian frequently
employs {\sc diminutives} (and to a lesser degree {\sc augmentatives}). The most
basic form of the diminutive is formed with the suffix \ird{-ka} (or \ird{-cka}
after vowels), which most linguists agree is a non-native
morpheme\index{borrowing}, and is most likely borrowed from Slavic.

\ex
\irdp{jec}{dog} → \irdp{jecka}{doggy, little dog}\\
\irdp{papír}{paper} →  \irdp{papírka}{piece of paper}\\
\irdp{dum}{house} →  \irdp{dumka}{little house}\\
\irdp{kávé}{coffee} →  \irdp{kávécka}{espresso}
\xe

Diminutives are used to express that something is small or insignificant. In the
spoken language\index{spoken Iridian}, however, it is more common to use the
diminutive to express endearment or affection\index{affect}. This same usage
makes it possible to use the diminutive patronizingly, to belittle or to be
dismissive.\index{pejorative} With mass nouns, the diminutive is also often used
to refer to a small quantity of something.

\pex
\a To express affection:\\
\begingl
\gla Jecka do vezdalnik.//
\glb dog-\mk{dim} \mk{1s.pat} to:gift-\Pv{}-\Pf{}//
\glft \trsl{This dog was given to me as a gift.}//
\endgl

\a To dismiss or belittle:\\
\begingl
\gla To na provízorká niho zábor.//
\glb this \Loc{} professor-\mk{dim-pat} \mk{nexst} knowledge//
\glft \trsl{This so-called ``professor'' doesn't know a thing.}//
\endgl

\a To express a small quantity of something:
\xe

When referring to members of one's own family\index{kinship terms}, that of a friend's, or of the person being addressed, the diminutive form is also used. Most kinship terms have irregular forms and are listed in \S\,\ref{sec:nuclear family}. In colloquial Iridian\index{colloquial Iridian} proper names are also often marked as diminutives, with the variant suffix \ird{-ik/-k} being more common. The first-person plural clitic \irdp{-óm}{our} is often used in conjunction with the diminutive. In addition to this, most names also have irregular diminutive forms and variants which are discussed in detail in \S\,\ref{sec:names}.

\ex
\ird{Janek} → \ird{Jančik}, \ird{Jančikóm}\\
\ird{Marek} → \ird{Marčik}, \ird{Marčikóm}\\
\ird{Tomáš} → \ird{Tomášik}, \ird{Tomáškóm}\\
\ird{Tereza} → \ird{Terežik}, \ird{Terežkóm}\\
\ird{Agáta} → \ird{Agáčik}, \ird{Agáčkóm}
\xe

Double and triple diminutives are also common, formed using \ird{-(i)ška} and \ird{-(i)sička}, respectively. Quadruple and quintuple diminutives are also possible (formed using \ird{-(i)nisička} \ird{-(i)nižesička}, respectively), although their usage is not as neutral, and would often be used to mock or to exaggerate.\footnote{The suffixes \ird{-(i)ška} and \ird{-(i)sička} are of Slavic\index{Slavic languages} origin while \ird{-(i)nisička} \ird{-(i)nižesička} are Iridian innovations.}

Augmentatives\index{augmentative} are also used, although their usage is not as common as diminutives and their usage is often limited as pejoratives\index{pejorative}. Augmentatives are formed with the suffixes \ird{-(ž)ulám} or \ird{-(ž)urnám} or \ird{-(ž)uláhmaš}. These forms are not interchangeable and in general the longer the augmentative suffix is, the more pejorative is its connotation.

\subsection{Nouns from nouns}

The suffix \ird{-(e)vnice} is used in deriving nouns from proper nouns. When used with names of places it generally has the meaning \trsl{resident of} or \trsl{native of}. Countries whose name end in the suffix \ird{-óma} drop the suffix first before adding \ird{-(e)vnice}. The variant \ird{-evnik} has the same meaning as \ird{-evnice} but can only be used derogatorily.

\begin{multicols}{2}
  \ex
  \irdp{ircevnice}{Iridian}\\
  \irdp{mažarevnice}{Hungarian}\\
  \irdp{čiževnice}{Czech}\\
  \irdp{polščevnice}{Polish}\\
  \irdp{mušhouvnice}{Muscovite}\\
  \irdp{néviorčevnice}{New Yorker}\\
  \irdp{turčevnice}{Turk}\\
  \irdp{ruževnice}{Russian}\\
  \irdp{američevnice}{American}\\
  \irdp{anglevnice}{English}
  \xe
\end{multicols}


The suffix \ird{-(h)ár} from the Czech \emph{-ár/-ář} indicates agency. It is often used to form nouns relating to professions, although it may appear with Latinate loanwords as the assimilated form of the French \emph{-aire}.

\ex
\irdp{revolucehár}{revolutionary} fr. \irdp{revoluce}{revolution}\\
\irdp{milionár}{millionaire} fr. \irdp{milion}{million}\\
\irdp{travár}{baker} fr. \irdp{trava}{bread}\\
\irdp{kostlár}{fisherman} fr. \irdp{kostel}{fish}\\
\irdp{známehár}{smith} fr. \irdp{známe}{metal}\\
\irdp{zakár}{sailor} fr. \irdp{zak}{sea}\\
\irdp{bašketbólár}{basketball player} fr. \irdp{bašketból}{basketball}\\
\irdp{míštár}{warrior} fr. \irdp{miešt}{war}\\
\irdp{ákcehár}{shareholder} fr. \irdp{ákce}{share of stock}\\
\irdp{nepodár}{bureaucrat} fr. \irdp{nepod}{position, rank}
\xe

Variants of \ird{-(h)ár} include \ird{-(h)er} and \ird{-(h)or}, although their usage is much more limited.

\ex
\irdp{senátor}{senator} fr. \irdp{senát}{senate}\\
\irdp{aviátor}{aviator} fr. \irdp{aviace}{aviation}\\
\irdp{helder}{salaryman} fr. \irdp{held}{wage, salary}, itself from German \emph{Geld}
\xe

Another common suffix used to form agent nouns is \ird{-ist}. This suffix is often used on nouns ending in \ird{-ižmus}.

\ex
\irdp{komunist}{communist} fr. \irdp{komunižmus}{communism}\\
\irdp{modernist}{modernist} fr. \irdp{modernižmus}{modernism}\\
\irdp{avtist}{cabdriver} fr. \irdp{avt}{car}\\
\irdp{mašinist}{engineer} fr. \irdp{mašina}{machine, engine}\\
\irdp{bankist}{banker} fr. \irdp{bank}{bank}\\
\irdp{žurnálist}{journalist} fr. \irdp{žurnál}{magazine}
\xe

The most common way of forming abstract nouns is through the suffix \ird{-(i)žnást}.

\ex
\irdp{vidližnást}{slavery} fr. \irdp{videl}{slave}\\
\irdp{tiehožnást}{divinity, holiness} fr. \irdp{tieho}{god}\\
\irdp{čeližnást}{membership} fr. \irdp{čelina}{member}\\
\irdp{stultižnást}{puberty} fr. \irdp{stólet}{teenager}
\xe

The suffix \ird{-(i)mašt} forms a place or location associated to a noun.

\ex
\irdp{piaštoumašt}{dining room, pantry} fr. \irdp{piaštou}{food}\\
\irdp{traumašt}{bakery} fr. \irdp{trava}{bread}\\
\irdp{jakomašt}{woods} fr. \irdp{jako}{tree}\\
\irdp{jelcimašt}{jungle} fr. \irdp{jelec}{forest}\\
\irdp{dílmašt}{nursery} fr. \irdp{diel}{infant}\\
\irdp{dohzámašt}{paradise} fr. \irdp{doház}{bliss}\\
\xe

\subsection{Nouns from verbs and adjectives}\label{sec:nomder-verb}

The suffix \ird{-ošc} (pronounced as if written \ird{ošt}) is used to form a noun that represents an actor or agent (usually a person) that performs the action denoted by the verb, often habitually, or less commonly, someone or something usually associated with the action described by the verb. The habitual nature of the action may be emphasized by using \ird{-ivošc} (which incorporates the aspectual marker \ird{-iv-}) instead of \ird{-ošc}. 

\ex
\irdp{umilošc}{heavy drinker, drunkard} fr. \irdp{umělá}{to get drunk}\\
\irdp{umilivošc}{drunkard} fr. \irdp{umělá}{to get drunk}\\
\irdp{jelzošc}{traveller} fr. \irdp{jelzá}{to travel, wander}\\
\irdp{hnervošc}{doctor, physician} fr. \irdp{hnervá}{to heal}\\
\xe

Agent/actor nouns can also be formed using \ird{-amite} or \ird{-amnite}. These pair contrasts with \ird{-ošc} and \ird{-ivošc} since they describe actions that are seen to affect someone or something else directly.

\ex
\irdp{nahradamite}{cook, chef} fr. \irdp{nahradá}{to cook}\\
\irdp{pardamite}{guard} fr. \irdp{pardá}{to guard, to watch}\\
\irdp{zěkamite}{spokesperson} fr. \irdp{zěká}{to say}
\xe

\section{Verbal Derivation}

\subsection{Verbs from nouns}

The suffix \ird{-istiná} is used to form stative verbs from nouns with the meaning `of or pertaining to.' The attributive is formed using \ird{ezní} instead of the regular \ird{*istní.} Nouns ending in the Greek-derived suffix \foreign{-logias} replace this latter suffix with \ird{-lóž-} first before adding \ird{istiná} or \ird{-ezní}. Latin loanwords ending in \ird{-us} or \ird{-um} and German loanwords ending in \ird{-um} (an assimilation of the original \foreign{-ung}) on the other hand drop these suffixes entirely before adding \ird{istiná} or \ird{-ezní}.

\ex
\irdp{hrumistiná, hrumezní}{ecclesiastical} fr. \irdp{hruma}{church}\\
\irdp{biolóžistiná, biolóžezní}{biological} fr. \irdp{biologias}{biology}\\
\xe

\section{Compounding}\index{compounding}\index{compound word}\label{sec:compounding}

\section{Linguistic Borrowing}\index{loanword}
A significant portion of the vocabulary of Iridian comes from loanwords from neighbouring languages, especially German\index{German}, Czech\index{Czech} and Polish\index{Polish}, and to a lesser extent Hungarian\index{Hungarian}. Like most languages from the area, Iridian also has a notable portion of its vocabulary derived from French\index{French} and Latin, mostly scientific and academic terms. In addition, after the advent of the internet, there has been an increasing amount derived from English and other world languages as well. Most loanwords are assimilated to conform with Iridian phonological rules, although most recent loanwords generally maintain the phonology of the language they were originally borrowed from.

In most cases, the loanwords or their assimilated forms coexist with their native Iridian counterparts. Often their usage is interchangeable

\subsection{From German and other Germanic languages}\index{German}\index{German loanwords}

Like its neighboring Czech Republic and Slovakia, Iridia has had significant contact with the German-speaking peoples of Central Europe throughout the centuries, leading to a significant German influence on the language's vocabulary. Most of the words of German origin now in Iridian entered the language in the 16th century when the Duchy of Iridia (then a part of the Crown of Bohemia) was absorbed into the Habsburg Monarchy, with the influence continuing into the late 19th century. Starting the 1880s\footnote{Some sources point to the defeat of Austria and the Peace of Prague in 1866 as the beginning of the `de-Germanization' of Iridia. Nevertheless it was not until the Edict of Julmonc (then Olmütz) was issued in March 1882 that the de-Germanization of the Iridian language was formalized by Iridian state authorities.} however (in large part due to the spread of Romanticism and nationalism in the region), and until the collapse of the Austro-Hungarian Empire, attempts have been made to `de-Germanize' Iridian vocabulary by replacing German vocabulary with words from the native stock or more often with calques\index{calque}. This `de-Germanization' continued well into the first half of the 20th century, as a result of which, German loanwords in Iridian in constant use have significantly decreased from what they have been in the 16th to the 18th centuries, with most words of Germanic origin now considered archaic and are used primarily as an affectation (cf. English \emph{thou}, \emph{shew} and \emph{methinks}, for example).

Assimilation\index{assimilation of loanwords} of German phonemes that do not exist in Iridian is generally consistent, and is subject to the rules discussed in this section.

German\index{German} has three falling diphthongs (\cite{wiese1996}): /aɪ̯/, /aʊ̯/ and /ɔʏ̯/, none of which have exact equivalents in Iridian. Nonetheless /aʊ̯/ assimilates to Iridian /au̯/ (both spelled $\langle$au$\rangle$). /aɪ̯/ does occur marginally in Iridian, but most instances of /aɪ̯/ in German become either /äː/ or /eɪ̯/.\footnote{Or just [äː] and [eː] given the monophthongization of /eɪ̯/ in most dialects.} Finally /ɔʏ̯/ is never assimilated to the marginal /ɔɪ̯/ but becomes either /eɪ̯/ or /au̯/.


\ex
Assimilation of German diphthongs:\\
\irdp{Karlštám}{Charles Castle} fr. \emph{Karlstein}\\
\irdp{Bérna}{Bayern} fr. \emph{Bayern}\\
\irdp{bedautum}{significance, importance} fr. \emph{Bedeutung}\\
\irdp{Freid}{Freud} fr. \emph{Freud}
\xe

The raised vowels \orth{ä} and \orth{ö} become /eː/ \orth{é} (or sometimes /ɪ/
\orth{i}) in Iridian while \orth{ü} become non-palatalising /ɪ/, spelled
\orth{y}.


\subsection{From Common Slavic and its descendants}

Common Slavic (or Proto-Slavic), and subsequently its successor Slavic
languages, has exerted a strong influence on the Iridian language, especially in
the form of loanwords. Similar to Romanian, Iridian has undergone a series of
`re-nativization' attempts during the 19th and 20th centuries, with the result
that many loanwords have been replaced by native words or calques. Nevertheless,
it is estimated that around 30 to as high as 45 percent of the language's
lexicon still have Slavic origins. Earlier slavic loanwords that have become
part of the basic vocabulary and which have more or less assimilated to Iridian
phonology have survived these attempts at `re-nativization'. For example, words
like \irdp{raja}{heaven} and \irdp{koleč}{key} are still in use, while words
like \irdp{pokoj}{peace} and \ird{hruzba}{threat} have been replaced by native
words \irdp{sivě} and \irdp{oblou} respectively, although both words can still
be seen occasionally but the usage is considered archaic or affected.



\subsection{From Latin and Greek}

\subsection{From French, English and other European languages}

\subsection{From other languages}