\chapter{Clause structure}\label{chap:clause-structure}

Thus far, we have focused our discussion on the behavior of individual words,
examining their functions, forms, and meanings. In this chapter, we will shift
our focus to how these words interact with each other to form sentences and
communicate meaning. Syntax is the set of rules that govern the structure and
arrangement of words in a language, and it plays a crucial role in ensuring the
effectivity of communication. In this chapter, we will examine the basic
syntactic structure of Iridian, then explore the different clause linking
strategies that makes it possible to form more complex sentences, and finally go
over special construction types such as questions, existential constructions,
copular constructions, etc.

\section{Basic clause structure}\label{sec:basic-clause-structure}

The two most salient features of Iridian syntax are the following: (1) SOV word
order and (2) strong head-finality. The constituent word order is
subject-object-verb (SOV); there is little variation in how the constituents of
a sentence are ordered, although the language might allow non-topic NP
constituents to be scrambled for stylistic or other reasons. Iridian is strongly
head-final, meaning that the head of a phrase always appears at the end of the
phrase. This includes all modifiers, including adverbials and determiners, and
non-main clauses, such as relative or subordinate clauses.

While we have been talking about an SOV word order, it would be more accurate to
analyze Iridian sentences to divided primarily into a topic part and a predicate
or comment part. The topic is what the sentence is about, while the predicate or
comment represents the information presented in the sentence about the topic.
While both the topic and the predicate are pragmatic constructs, the
topic-predicate construction is important as it determines how the rest of the
sentence is structured. In general, the topic of the sentence is the constituent
that is most prominent in the sentence and consequently appears first. The topic
of the sentence is also the constituent that determines the case marking of the
main verb and the other constituents of the sentence. The predicate, on the
other hand, provides supplementary information about the topic of the sentence.

\begin{figure}
  \begin{forest}
    [S,
      [{\sc top}] [{\sc pred}]]
  \end{forest}
  \caption{Nuclear structure of sentences}
  \label{fig:topic-pred}
\end{figure}

The topic, despite its prominence, can be considered non-essential, as most of
the information is encoded in the predicate. In fact, the primacy of the topic
and its influence on how the predicate is structured means that for the most
part, where the topic has already been established earlier, only the predicate
is needed to create a well-formed sentence. 

In most topic-prominent languages, the topic does not necessarily coincide with
the subject. For example, the following sentences are grammatical in Korean:

\pex
\a
\begingl
\gla seylinun paykhyellul coahanta.//
\glb Seri-\Top{} Baekhyun-\Obj{} like.//
\glft \trsl{As for Seri, she likes Baekhyun.}//
\endgl
\a
\begingl
\gla seylika paykhyennun coahanta.//
\glb Seri-\Subj{} Baekhyun-\Top{} like.//
\glft \trsl{As for Baekhyun, Seri likes him.}//
\endgl
\xe

In Iridian, however, the topic and the subject are the same. By promoting a noun
phrase as the topic, it automatically becomes the subject of the sentence, as
the voice marking in the main verb is determined by the thematic role of the topic in the sentence.

The topic function is not linked to a particular grammatical function.
Nevertheless, while it is theoretically possible to promote any constituent noun
phrase of a sentence to the topic position, in practice, the topic is usually a
noun phrase that is specific and referential. In most cases, this means that the
topic is a proper noun or definite noun phrase. As \textcite{kiss2004} observes,
\begin{quote}
  We tend to describe events from a human perspective, as statements about their
  human participants – and subjects are more often {\sc[+human]} than objects
  are. In the case of verbs with a {\sc[–human]} subject and a {\sc[+human]}
  accusative or oblique complement,  the  most  common  permutation  is  that
  in  which  the  accusative  or oblique complement occupies the topic
  position\,[.] When the possessor is the only human involved in an action or
  state, the possessor is usually topicalized[.]
\end{quote}

\section{Internal clause structure}\label{sec:internal-clause-structure}

In linguistics, head directionality describes the position of the head of a
phrase with relation to other constituents. The {\sc head} is the element that
determines the syntactic category of the whole phrase. For example, in the
English phrase \trsl{the big, red box}, the word \trsl{box} is the head, and
since it is a noun, the whole phrase is a noun phrase. Using the
head-directionality criteria, we can classify languages as either head-initial,
i.e., the head of the phrase appears before the other constituents, or
head-final, i.e., the head of the phrase appears after the other constituents.
English is a strongly head-initial language, as we see in the following
examples:

\pex
\a \([{}\lsub{VP}~[{}\lsub{V}~\text{ eat}] [{}\lsub{DP}~\text{an apple}]]\)
\a \([{}\lsub{NP}~[{}\lsub{N}~\text{girl}] [{}\lsub{CP}~\text{who gave me this book}]]\)
\a \([{}\lsub{DP}~[{}\lsub{D}~\text{this}] [{}\lsub{NP}~\text{ink}]]\)
\xe

Iridian, by contrast, is a strongly head-final language, as we see in the
following examples. The only exception are prepositional phrases like
(\ref{ex:head-initial-pp}), which are head-initial.

\pex
\a \([{}\lsub{VP}~[{}\lsub{DP}~\text{to lobek}] [{}\lsub{V}~\text{piaštnek}]]\)\\
  \irdp{to lobek piaštnek}{ate this apple}
\a \([{}\lsub{NP}~[{}\lsub{CP}~\text{to tóm dítnice ty}] [{}\lsub{N}~\text{hlem}] ]\)\\
  \irdp{to tóm dítnice ty hlem}{the girl who gave me this book}
\a\label{ex:head-initial-pp}\([{}\lsub{PP}~[{}\lsub{P}~\text{za}] [{}\lsub{NP}~\text{Marcí mlazke}] ]\)\\
  \irdp{za Marcí mlazke}{for Marek's brother}
\xe

One consequence of this behavior is that all modifiers must precede the noun
they modify. This is the case even with longer modifiers such as relative
clauses. In English, for example, an adjective like \trsl{big} will appear
before the noun it modifies, as in \trsl{the big box}; but a relative clause
like \trsl{which we found in the attic} will appear after the noun it modifies,
as in \trsl{the box which we found in the attic}. In Iridian, however, both
forms obligatorily appear before the noun.

\pex
\a
  \begingl
  \gla hlucé zbrím//
  \glb blue-\Att{} box//
  \glft \trsl{(the) blue box}//
  \endgl
\a
  \begingl
  \gla kramnu neleznice ty zbrím//
  \glb attic-\Ins{} find-\Pv{}-\Pf{}-\Cnj{} \Rel{} box//
  \glft \trsl{the box which we found in the attic}//
  \endgl
\xe



\section{Topicalization}\label{sec:topicalization}

\section{Clause linking strategies}\label{sec:clause-linking}

To communicate more complex ideas, it may sometimes be necessary to link two or
more sentences together in a single sentence or in a sequence of sentences. In
English this is commonly done by using conjunctions such as `and,' `but,' `or,'
etc. or relative pronouns like `that' or `which.' The choice of which connector
to use depends on the type of relationship between the two clauses. We will
consider three broad types of clause linking strategies: 

\pex
\a Relative clause:\\
  {The man [\emph{whom} we saw at the store yesterday] is my brother.}
\a Complementation:\\
  {He asked me [\emph{whether} I had seen the movie.]}
\a Conjunction or clause-linking:\\
  {[\emph{Although} we went to Dobroslau], we did not see him.}
\xe

We may observe that the sentence constituents enclosed in square brackets can be
freely removed from the above examples without making the sentences
ungrammatical. We will call this remnant part the {\sc main clause} and the
sentence constituent enclosed in square brackets the {\sc secondary clause}. In
general, the verb in the secondary clause will be marked in the conjunctive form
(see \ref{sec:conjunctive-form} in Iridian. For example, the above sentences
would be translated as follows:

\pex
\a
\begingl
  \gla Magazinu vednice ty maša mlazka.//
  \glb store-\Ins{} see-\Pv{}-\Pf{}-\Cnj{} \Rel{} man brother-\Dim{}//
  \glft \trsl{The man whom we saw at the store yesterday is my brother.}//
\endgl
\a
\begingl
  \gla To film vednice by hloupšek.//
  \glb this movie see-\Pv{}-\Pf{}-\Cnj{} \Quot{} ask-\Av{}-\Pf{}//
  \glft \trsl{He asked me whether I had seen the movie.}//
\endgl
\a
\begingl
  \gla Do Dobroslava stožicemá ša závednik.//
  \glb into Dobroslau-\Acc{} go-\Av{}-\Pf{}-\Cnj{}:although \Third{}\Sg{} \Neg{}-see-\Pv{}-\Pf{}//
  \glft \trsl{Although we went to Dobroslau, we did not see him.}//
\endgl
\xe

The secondary clause is terminated by a {\sc conjunctive ending}
\index{conjunctive ending} which indicates the clause's role in the sentence.
With the exception of \ird{ty} (used with relative clauses) and \ird{by} (used
with quotative constructions), the conjunctive ending are fused with the verb in
the secondary clause and do not appear as separate morphemes, as in English. In
general, the main clause follows the secondary clause(s) in a sentence, unless
the secondary clause is expressed as an afterthought.

\pex
\a canonical order: [secondary clause] + [conjunctive ending] + [main clause]
\a afterthought: [main clause] + [secondary clause]
\xe

Iridian has a large number of conjunctive 

\subsection{Temporal succession}
\label{sec:temporal-succession}

The conjunctive ending \irdp{-ní}{and then} is used to link two sequential
clauses; the secondary clause, which occurs first, indicates the first occuring
event while the main clause indicates the second event.

\pex
\begingl
  \gla Marku houčicení tětar zaby stojounek.//
  \glb Mark-\Ins{} meet-\Av{}-\Pf{}-\Cnj{}-and.then theater together
        go-\Lv{}-\Pf{}//
  \glft \trsl{I met Marek and then we went to the theater together.}//
\endgl
\xe

If the secondary clause(s) and the main clause share the same subject, the chain
is usually interpreted as describing a succession of events, as is the case in
the above examples. However, if the secondary clause(s) and the main clause have
different subjects, the chain may also take a `causal' interpretation, i.e., the
action in the secondary clause causes the action in the main clause.

\pex
\begingl
  \gla Mobil Jankam prodnicení mač všihnaševí.//
  \glb mobile.phone Mark-\Agt{} lose-\Pv{}-\Pf{}-\Cnj{}-and.then mother be.angry-\Av{}-\Cont{}//
  \glft \trsl{Mark lost his mobile phone and then/and so his mother got angry.}//
\endgl
\xe

The ending \irdp{-š}{and} has a similar meaning but offers a vaguer
description of the temporal relationship between the two clauses. Consider the
following sentences:

\pex
\a\label{ex:sz-perf}
\begingl
  \gla Marek do Praha stožiceš pislo Jankám prějenik.//
  \glb Marek into Prague-\Acc{} go-\Av{}-\Pf{}-\Cnj{}-and letter Janek-\Agt{}
  send-\Pv{}-\Pf{}//
  \glft \trsl{Marek went to Prague and Janek sent (him) a letter.}//
\endgl
\a\label{ex:sz-prog}
\begingl
  \gla Marek znousčiměš Janek uzdravžime.//
  \glb Marek study-\Av{}-\Prog{}-\Cnj{}-and Janek sleep-\Av{}-\Prog{}//
  \glft \trsl{Marek is studying and Janek is sleeping.}//
\endgl
\xe

Example (\ref{ex:sz-perf}) can be interpreted as either (1) \trsl{Marek went to
Prague \emph{and then} Janek sent him a letter} or \trsl{Marek went to Prague
\emph{at the same time that} Janek sent him a letter.} We can also consider a
third option, where the speaker is neutral as to the temporal relationship
between the two events because it is either unknown or irrelevant to the
conversation. In example (\ref{ex:sz-prog}), on the other hand, while the
clauses are still linked by \ird{-š}, most speakers would likely interpret
this as \trsl{Marek is studying \emph{and at the same time} Janek is sleeping}
as the progressive aspect would make it unlikely that the speaker meant for the
two events to be sequential. Nevertheless, the neutral interpretation
\trsl{Marek is studying \emph{and} Janek is sleeping} is still possible.

If a sentence contains multiple secondary clauses, the order of the clauses
correspond to the order of the events. Only \ird{-š} and not \ird{-ní} can be
used to chain these clauses.

\pex
\a\begingl
  \gla \ljudge{*}Marku houčicení tětar zaby stojounicení zaby prižek.//
  \glb Mark-\Ins{} meet-\Av{}-\Pf{}-\Cnj{}-and.then theater together
        go-\Lv{}-\Pf{}-\Cnj{}-and.then together have.dinner-\Av{}-\Pf{}//
  \glft \trsl{I met Marek and then we went to the theater together and had dinner.}//
\endgl
\a\begingl
\gla Marku houčiceš tětar zaby stojouniceš zaby prižek.//
\glb Mark-\Ins{} meet-\Av{}-\Pf{}-\Cnj{}-and theater together
      go-\Lv{}-\Pf{}-\Cnj{}-and together have.dinner-\Av{}-\Pf{}//
\glft \trsl{I met Marek and then we went to the theater together and had dinner.}//
\endgl
\xe

In brief, the ending \ird{-ní} presumes some level of relationship (whether
causal or not) between the linked clauses and thus cannot be used with patently
divergent or unrelated clauses. The ending \ird{-š} is more neutral and can be
used whether or not the clauses are related temporally or causally.

Other conjunctive endings are available to express more specific temporal
relationships such as \irdp{-mazy}{while}, \irdp{-zak}{until},
\irdp{-škady}{around the time when}, \irdp{-škany}{since}, \irdp{-šhoume}{as
soon as}, \irdp{-šbym}{after}, \irdp{-šdny}{before}, etc. Most of these endings
developed from postpositions, e.g., \ird{-šdny} from \irdp{dnou}{in front of}.

\pex
\a
\begingl
  \gla Zkuzy kadem anuncirnicešdny Janek kourneví.//
  \glb exam-\Gen{} result announce-\Pv{}-\Pf{}-\Cnj{}-before Janek worried-\Cont{}//
  \glft \trsl{Janek was worried before the exam results were announced.}//
\endgl
\a
\begingl
  \gla Zkuzy kadem anuncirnicešhoume Janek vysleví.//
  \glb exam-\Gen{} result announce-\Pv{}-\Pf{}-\Cnj{}-as.soon.as Janek delighted-\Cont{}//
  \glft \trsl{Janek was delighted as soon as the exam results were announced.}//
\endgl
\a
\begingl
  \gla Zkuzy kadem anuncirnicešbym Janek zuštaleví.//
  \glb exam-\Gen{} result announce-\Pv{}-\Pf{}-\Cnj{}-after Janek happy-\Cont{}//
  \glft \trsl{Janek was happy after the exam results were announced.}//
\endgl
\trailingcitation{(adapted from \cite{sohn2009korean})}
\xe

\subsection{Contrast}
\label{sec:contrast}

Contrast between two clauses is usually expressed by the ending \ird{-má}
usually translated in English as \trsl{although} or \trsl{but}.

\pex
\begingl
  \gla Marek do Praha stožicemá Janek závednaní.//
  \glb Marek into Prague-\Acc{} go-\Av{}-\Pf{}-\Cnj{}-but Janek \Neg{}-see-\Pv{}-\Ret{}//
  \glft \trsl{Although Marek went to Prague, he didn't meet Janek.}//
\endgl
\xe

\pex
\begingl
  \gla Tozevěmá prékveví.//
  \glb small-\Cont{}-\Cnj{}-but heavy-\Cont{}//
  \glft \trsl{It might be small, but it's heavy.}//
\endgl
\xe

The ending \ird{-má} may also be used even when the sentence does not
necessarily express contrast but the speaker wishes to `soften' the statement by
posing it as an afterthought or hinting uncertainty. It can make a statement
sound less argumentative or confrontational or give a hint as to what the
speaker wants to say without being explicit, creating a sort of lingering
effect. It can also be used to express humility or to acknowledge someone else's
opinion without necessarily agreeing with it. The secondary clause marked by
\ird{-má} may appear by itself without a main clause.

\pex
\begingl
  \glpreamble \lingcontext{The speaker is on the phone and is trying to make a reservation.}//
  \gla Mašé vtare. Vitěbounitemá.//
  \glb good-\Att{} morning make.a.reservation-\Lv{}-\SupP{}-\Cnj{}-but//
  \glft \trsl{Good morning. I'd like to make a reservation, but\ldots}//
\endgl
\xe


\subsection{Alternatives}
\label{sec:conj-alternatives}

Disjunction (X or Y) is expressed by the ending \irdp{-ký}{or}. Alternatively,
if the choices are limited to the two clauses present, the conditional ending
\irdp{-zmy}{if not} is used (never \ird{-byž}).

\pex
\a\begingl
  \gla Guláš piaštnažeký kolbaš piaštnách.//
  \glb goulash eat-\Pv{}-\Ctp{}-\Cnj{}-or sausage eat-\Pv{}-\Ctp{}//
  \glft \trsl{I will eat either goulash or sausage (or maybe something else).}//
\endgl
\a\begingl
  \gla Guláš piaštnažezmy kolbaš piaštnách.//
  \glb goulash eat-\Pv{}-\Ctp{}-\Cnj{}-if.not sausage eat-\Pv{}-\Ctp{}//
  \glft \trsl{I will eat goulash. If not, I will eat sausage.}//
\endgl
\xe

In addition to \ird{-ký}, the endings \ird{-na}, \ird{-nak}, and \ird{-nahy} can
also be used to express other forms of disjunction. The endings \ird{-na} and
\ird{-nak} can be translated as \trsl{instead (of)} or \trsl{rather than}.
Although both have the same meaning, the latter would often carry an implication
that the proposition in the main clause is preferable to or more desirable than
the proposition in the secondary clause. Finally, the ending \ird{-nahy} can be
translated as \trsl{neither} (which is marked in both the main and the secondary
clause).

\pex
\a\begingl
  \gla Guláš piaštnažena kolbaš piaštnách.//
  \glb goulash eat-\Pv{}-\Ctp{}-\Cnj{}-instead sausage eat-\Pv{}-\Ctp{}//
  \glft \trsl{Instead of eating goulash, I will eat sausage.}//
\endgl
\a\begingl
  \glpreamble \lingcontext{The speaker does not like goulash.}//
  \gla Guláš piaštnaženak kolbaš piaštnách.//
  \glb goulash eat-\Pv{}-\Ctp{}-\Cnj{}-instead sausage eat-\Pv{}-\Ctp{}//
  \glft \trsl{Instead of eating goulash, I will eat sausage.}//
\endgl
\a\begingl
  \gla Guláš piaštnaženahy kolbaš piaštnáženahy.//
  \glb goulash eat-\Pv{}-\Ctp{}-\Cnj{}-neither sausage eat-\Pv{}-\Ctp{}-\Cnj{}-neither//
  \glft \trsl{I will neither eat goulash nor sausage.}//
\endgl
\xe

\subsection{Causality}
\label{sec:causality}

As discussed in \S\,\ref{sec:temporal-succession}, the conjunctive ending
\ird{-ní} often takes a causal reading when the main clause and the secondary
clause have different subjects. Nevertheless, causation can be expressed more
explicitly by the use of \ird{-vlí}.

\pex
\a\begingl
  \gla Zabole zákupébicevlí byl kravnašime.//
  \glb ice:cream-\Acc{} \Neg{}-buy-\Ben{}-\Pf{}-\Cnj{}-because child cry-\Av{}-\Prog{}//
  \glft \trsl{The child is crying because (they) did not buy him ice cream.}//
\endgl
\a\begingl
  \gla Zabole zákupébicení byl kravnašime.//
  \glb ice:cream-\Acc{} \Neg{}-buy-\Ben{}-\Pf{}-\Cnj{}-and.then child cry-\Av{}-\Prog{}//
  \glft \trsl{(They) did not buy the child ice cream and so he is crying.}//
\endgl
\xe

Only \ird{-vlí} can be used to mark the cause or reason used as the basis of an
inference or judgment marked with the inferential particles \ird{izdy} or
\ird{hlavdy}. The verb in the main clause must be in the conditional mood if the
speaker is uncertain about the truth of the reason provided in the main clause,
as in (\ref{ex:causality-inference-uncertain}) below.

\pex
\a\label{ex:causality-inference-uncertain}
\begingl
  \gla Zabole zákupébilevlí(*-ní) byl izdy kravnašime.//
  \glb ice:cream-\Acc{} \Neg{}-buy-\Ben{}-\Cond.\Pf{}-\Cnj{}-because(*-and.then) child \Spec{} cry-\Av{}-\Prog{}//
  \glft \trsl{The child must be crying because (they) did not buy him ice cream.}//
\endgl
\a
\begingl
  \gla Marek do Budapešta stožicevlí magazin izdy zaromenik.//
  \glb Marek into Budapest-\Acc{} go-\Av{}-\Pf{}-\Cnj{}-because store \Spec{} close-\Pv{}-\Pf{}//
  \glft \trsl{Since Marek went to Budapest, his shop must be closed.}//
\endgl
\xe

\subsection{Conditional clauses}
\label{sec:conditional-clauses}
\index{conditional clause}

A conditional sentence is a statement of the form \trsl{If X, then Y.} Here X is
called the protasis or the condition and Y is called the apodosis or the result.
Both verb forms must be marked in the conditional. The protasis corresponds to
the secondary clause and the apodosis to the main clause and thus the former is
additionally marked in the conjunctive form and is commonly terminated by
\irdp{-my}{if} or its negative counterpart \irdp{-zmy}{if not}, or by
\irdp{-bymy}{if} or its negative counterpart \irdp{-byž}{if not}. The ending
\ird{-bymy} and \ird{byž} presuppose that the event described in the protasis
\emph{will} happen, but the exact timing of which is yet uncertain. \ird{-my}
and \ird{-zmy} on the other hand merely state a possibility, i.e., it is
uncertain whether or not the event described in the protasis will happen at all.

\pex\label{ex:conditional-clauses}
	\a
	\begingl
		\gla Piaščejímy, dá može piaščy.//
		\glb eat-\Av{}-\Cond{}.\Ipf{}-\Cnj{}-if I also eat-\Av{}-\Cond{}.\Ipf{}//
		\glft \trsl{If you eat, then I will also eat.}//
	\endgl
	\a
	\begingl
		\gla Nebo 100 centihradu nekraznejíbymy, ustrožy.//
		\glb water 100 celcius-\Ins{} \Caus{}-heat-\Pv{}-\Cond{}.\Ipf{}-\Cnj{}-if \Refl{}-boil-\Av{}-\Cond{}.\Ipf{}//
		\glft \trsl{If you heat the water to 100 degrees Celsius, then it will boil.}//
	\endgl
\xe

In example (\ref{ex:conditional-clauses}b), what is being described is merely
the logical consequence of the protasis happening, i.e., the water will boil if
it is heated to 100 degrees Celsius. In English, this usage of \ird{-bymy/byž}
would often be translated as \trsl{when}.

When using \ird{-bymy} or \ird{-byž} for habitual states or actions,
\irdp{ozle}{often}, \irdp{než}{sometimes}, \irdp{pouze}{rarely},
\irdp{žemě}{never} and \irdp{nimě}{always} may be used to indicate the frequency
of the event described in the main clause.

\pex
	\a
	\begingl
		\gla Dá na duma zmy, dá na gnaža.//
		\glb I \Loc{} house if.not I \Loc{} school//
		\glft \trsl{If I'm not at home, then I'm at school.}//
	\endgl
	\a
	\begingl
		\gla Dá na duma byž, dá ozle na gnaža.//
		\glb I \Loc{} house often I \Loc{} school//
		\glft \trsl{When I'm not at home, I'm often at school.}//
	\endgl
\xe

Counterfactuality is expressed by adding the adverbial particle \ird{mlada} to
the protasis or to both clauses. The use of \ird{mlada} is only compatible with
\ird{-my} or \ird{-zmy}. To further emphasize the counterfactual nature of the
sentence, the adverbial particle \irdp{sám}{only} may be used in addition to
\ird{mlada} in the secondary clause.
\pex
	\a
	\begingl
		\gla To bych mlada podatnilemy, prěnžil.//
		\glb this yesterday \Hyp{} submit-\Pv{}-\Cond{}.\Pf{}-\Cnj{}-if pass-\Av{}-\Cond{}.\Pf{}//
		\glft \trsl{If I had submitted this yesterday, I would have passed.}//
	\endgl
	\a
	\begingl
		\gla Dá nesté duhu do Vietnama mlada stožilezmy, Marek vednil.//
		\glb I last-\Att{} month-\Ins{} into Vietnam-\Acc{} \Hyp{} go-\Av{}-\Cond{}.\Pf{}-\Cnj{}-if.not Marek see-\Pv{}-\Cond{}.\Pf{}//
		\glft \trsl{If I hadn't gone to Vietnam last month, I would have seen Marek.}//
	\endgl
  \a
  \begingl
    \gla Dá mlada sám stožilemy, ježe děne po vedny?//
    \glb I \Hyp{} \Excl{} go-\Av{}-\Cond{}.\Pf{}-\Cnj{}-if what \Spec{} \Ipfv{} see-\Pv{}-\Cond{}.\Ipf{}//
    \glft \trsl{I wonder what else I would have seen if only I had gone?}//
  \endgl
  \xe

The protasis with \ird{mlada} (and optionally \ird{sám}) can be used by itself
to express wishes. This would often be written with an ellipsis in both English
and Iridian.

\pex
  \begingl
  \gla Dá do Roubžy mlada sám stožilemy\ldots//
  \glb I to Roubže-\Acc{} \Hyp{} \Excl{} go-\Av{}-\Cond{}.\Pf{}//
  \glft \trsl{If only I had gone to Roubže\ldots}//
  \endgl
\xe

Concessive clauses are considered a special case of conditional clauses. They
are usually translated in English as \trsl{Unless X, Y}. There are three main
conjunctive endings for consessive clauses in Iridian: \irdp{-kou}{unless},
\irdp{-kuzmy}{as long as} and \irdp{-kazy}{even if}. Some examples are given
below.

\pex
\a
\begingl
  \gla Marek sobotu mlada stožilekazy, opera zaby závednil.//
  \glb Marek saturday-\Ins{} \Hyp{} go-\Av{}-\Cond{}.\Pf{}-\Cnj{}-even.if opera together \Neg{}-see-\Pv{}-\Cond{}.\Pf{}//
  \glft \trsl{Even if Marek had come last Saturday, we wouldn't have been able to go to the opera together anyway.}//
\endgl
\a
\begingl
  \gla To prova Jankám vlastnejíkou, zákabežy.// 
  \glb this exam Janek-\Agt{} pass-\Pv{}-\Cond{}.\Ipf{}-\Cnj{}-unless \Neg{}-pass-\Av{}-\Cond{}.\Ipf{}//
  \glft \trsl{Unless Janek passes this exam, he won't be able to graduate.}//
\endgl
\xe

\subsection{Manner}
\label{sec:conjunctive-manner}

\subsection{Summary}
\label{sec:conjunctive-summary}

Table \ref{tab:conjunctive-endings} shows a summary of the most common
conjunctive endings in Iridian.

\begin{table}
  \sffamily\footnotesize
  \caption{List of conjunctive endings in Iridian}
  \label{tab:conjunctive-endings}
  \begin{tblr}{width=\textwidth, colspec={X[0.5]XX}}
      \toprule
      {\sc ending} & {\sc usage} & {\sc translation} \\
      \midrule
      -bymy & conditional & \trsl{if/when}\\
      -byž & conditional & \trsl{if not}\\
      -kazy & concessive & \trsl{even if}\\
      -kou & concessive & \trsl{unless}\\
      -kuzmy & concessive & \trsl{as long as}\\
      -ký & disjunction & \trsl{or}\\
      -má & contrast & \trsl{but, even though, although}\\
      -my & conditional & \trsl{if}\\
      -na & disjunction & \trsl{instead of}\\
      -nahy & disjunction & \trsl{neither}\\
      -nak & disjunction & \trsl{instead of}\\
      -ní & temporal succession & \trsl{and then, and so}\\
      -š & temporal succession & \trsl{and, and then }\\
      -vlí & reason & \trsl{because}\\
      -zmy & conditional & \trsl{if not}\\
      \bottomrule
  \end{tblr}
\end{table} 

\section{Relative clauses and apposition}
\label{sec:relative-clauses}\index{relative clause}\index{apposition}

Appositive constructions\index{apposition} involve the juxtaposition of two or
more noun phrases that have a single referent. The apposition and the noun or
noun phrase it modifies are linked by the particle \ird{ty}\footnote{The
particle \ird{ty} is more properly analyzed as a conjunctive ending (cf.
\S\,\ref{sec:clause-linking}). However, unlike other conjunctive endings,
\ird{ty} as well as the quotative \ird{by} are written as separate words instead
of being fused to the verb in the secondary clause.}. This particle however is
largely optional and is often dropped in case of shorter appositives. An
apposition can be nonrestrictive if the appositive can be removed freely without
changing the meaning of a sentence, or restrictive otherwise. This distinction
is only semantic in Iridian, as there are no separate forms for restrictive or
nonrestrictive appositive.

\pex\label{ex:appositive}
\begingl
  \gla Mlazka (ty) Karel po záščenžaní.//
  \glb brother-\Dim{} \Rel{} Karel still \Neg{}-arrive-\Av{}-\Ret{}//
  \glft \trsl{My brother Karel hasn't arrived yet.}//
\endgl
\xe

Like other modifiers, appositives must always appear before the noun or noun
phrase they modify. In appositive constructions, however, since both the
modifier and the modified element are noun phrases, it might be possible to
switch their positions, but with slight changes in the meaning. For example, the
example above can also be written as:

\pex\label{ex:appositive-switch}
\begingl
  \gla \ljudge{?}Karel ty mlazka po záščenžaní.//
  \glb Karel \Lnk{} brother-\Dim{} still \Neg{}-arrive-\Av{}-\Ret{}//
  \glft \trsl{My brother Karel hasn't arrived yet.}//
\endgl
\xe

In this second example, \ird{mlazka} is the main noun while \ird{Karel} is the
appositive. While (\ref{ex:appositive-switch}) is still grammatical, a
construction like this where a more specific noun phrase would be used to modify
a more general one would imply that the modified noun phrase refers to a group
with multiple elements, with the appositive referring to a specific member of
that group; i.e., (\ref{ex:appositive-switch}) would imply that the speaker has
more than one brother, and that Karel is one of them. There is no such
implication in (\ref{ex:appositive}), where \ird{mlazka} is merely descriptive
of \ird{Karel}; unless of course the discourse has previously established
multiple \ird{Karel}s, in which case \ird{mlazka} would be more specific than
\ird{Karel} and the same group restriction implication can be drawn from
(\ref{ex:appositive}).

It is possible, especially with longer appositives, for the appositive to appear
after the noun or noun phrase it modifies. In this case, however, the appositive
can more properly be analyzed as a parenthetical. The appositive is introduced
by \ird{to-ty} if the noun phrase being modified is inanimate or by \ird{ša-ty}
if the noun phrase being modified is animate. This parenthetical appositive is
then set off from the rest of the sentence by a pair of commas. While regular
appositives can be restrictive or nonrestrictive, parenthetical appositives are
always nonrestrictive. This usage is preferred only when the appositive is too
long as to cause ambiguity in the sentence or if the speaker is adding the
appositive phrase as an afterthought. Otherwise the pre-noun phrase appositive
is more common.

\pex
\begingl
  \gla Karel, ša-ty Marcí mlazka, po záščenžaní.//
  \glb Karel \Anim{}.\Lnk{} Marek-\Gen{} brother-\Dim{} still \Neg{}-arrive-\Av{}-\Ret{}//
  \glft \trsl{Karel, Marek's brother, hasn't arrived yet.}//
\endgl
\xe

A relative clause\index{relative clause} is a clause that modifies a noun or
noun phrase. In English\index{English}, a relative clause is introduced by a
relative pronoun (such as ``who,'' ``that,'' ``which,'' ``whose,'' or
``where''), such as in the examples below:
\pex
  \a The book, \emph{which was on the table,} was very interesting.
  \a The man \emph{whom I saw at the store} is my neighbor.
  \a The house \emph{where I grew up} is for sale.
\xe

Relative clauses are used to provide additional information about the noun or
noun phrase that they modify. Like appositive constructions, they can be
restrictive or nonrestrictive. Restrictive relative clauses are necessary to
identify the specific noun or noun phrase that is being referred to, while
nonrestrictive relative clauses provide additional, non-essential information
about the noun or noun phrase, which could be removed from a sentence altogether
without changing its meaning.

Unlike English, however, Iridian does not employ relative pronouns to link
relative clauses with their antecedents. Instead the main verb in the relative
clause is marked in the conjunctive form. Unlike simple nominal appositives
however, the use of \ird{ty} to link a relative clause with its antecedent is
not optional.

\pex
\a\begingl
  \gla Ša maša magazinu vednik.//
  \glb this man store-\Ins{} see-\Pv{}-\Pf{}//
  \glft \trsl{(I) saw this man at the store.}//
\endgl

\a\begingl
  \gla Magazinu vednice ty maša blež.//
  \glb store-\Ins{} see-\Pv{}-\Pf{}-\Cnj{} \Rel{} man neighbor//
  \glft \trsl{The man (I) saw at the store is (my) neighbor.}//
\endgl
\xe

Since the relative clause is a modifier, it must always appear before the noun
or noun phrase it modifies. As with appositives, the relative clause can however
come after the noun it modifies as a parenthetical, introduced by \ird{to-ty} if
the noun phrase being modified is inanimate or by \ird{ša-ty} if the noun phrase
being modified is animate. However, the verb in this resulting clause takes the
nominalizing suffix \ird{-ou} instead of the conjunctive form. In addition,
since parentheticals are always nonrestrictive, the relative clause can only be
shifted to this position if it is nonrestrictive as well. As with regular
appositives, postposing the relative clause enjoys less currency than the
pre-noun phrase relative clause and in addition is very rarely found in formal
writing and similar contexts.

\pex
\a\begingl
  \gla \ljudge{*}Maša ty magazinu vednice blež.//
  \glb man \Rel{} store-\Ins{} see-\Pv{}-\Pf{}-\Cnj{} neighbor//
  \glft \trsl{The man (I) saw at the store is (my) neighbor.}//
\endgl
\a\begingl
  \gla Marek, ša-ty magazinu vednikou, blež.//
  \glb Marek \Anim{}.\Rel{} store-\Ins{} see-\Pv{}-\Pf{}-\Nz{} neighbor//
  \glft \trsl{Marek, whom I saw at the store, is my neighbor.}//
\endgl
\xe

Iridian further restricts the formation of relative clauses by requiring that
the shared noun occupy the topic position in the embedded clause. Thus  a
sentence like (\ref{ex:incorrect-relativization}) is ungrammatical, since the
topic of the embedded clause is \irdp{dá}{I} rather than the shared noun
\ird{maša}.

\pex\label{ex:incorrect-relativization}
\begingl
  \gla \ljudge{*}Dá magazinu vižice ty maša blež.//
  \glb I store-\Ins{} see-\Av{}-\Pf{}-\Cnj{} \Rel{} man neighbor//
  \glft \ljudge{??/*}\trsl{The man that he saw something at the store is my neighbor}//
\endgl
\xe

A more permissive reading can be allowed and the sentence can be grammatical if
\ird{dá} is removed altogether, resulting in the phrase \irdp{magazinu vižice ty
maša}{the man who saw something at the store}.

\section{Complementation}\label{sec:complementation}

\subsection{Complement clauses: introduction}\label{sec:complement-clauses}

There are two main types of complement clauses in Iridian: quotative clauses 


\section{Quotative constructions and  evidentiality}\label{sec:reportedspeech}
\index{reported speech}\index{indirect speech|see{reported speech}}
\index{evidentiality}

\subsection{Quotative construction in general}

Superficially, the Iridian quotative is used to mark {\sc evidentiality}, a
grammatical category concerned with the explicit encoding of the source of
information or knowledge (i.e., evidence) which the speaker claims to have made
use of for producing the primary proposition of the utterance
(\cite[1-2]{diewald2010}). Iridian is unique among languages of Central Europe
(and of Europe in general) in possessing a grammaticalized evidentiality system.
Even non-Indo European languages in the region such as Hungarian (cf. author) or
Basque (cf. \cite{alcazar2010}) do not possess an overt evidential. Of course a
speaker’s source of information may be expressed through other methods 

The Iridian evidentiality system more or less falls under
\posscite{aikhenvald2004} A3 category, where the distinction is between the
marked quotative form for reported speech/hearsay and the unmarked ‘everything
else’ category which is evidentiality-neutral

In practice, however, the quotative is used in an array of other constructions
that is not necessarily predicated on evidentiality, but might be lexically or
semantically motivated as well, perhaps in the same way the subjunctive in
Romance languages have become grammaticalized into a subordination marker (cf.
\cite{poplacketal}).

\subsection{Quotative constructions and reported
speech}\label{sec:quotative-const}

The principal use of the quotative is to explicitly mark reported speech. The
reported clause is separated from the rest of the sentence by the particle
\ird{cy}. Direct quotations do not require the quotative, although they are
still separated from the main clause by \ird{cy}.

\pex
  \begingl
    \gla Koleč sní polšice cy Lukáš zíček.//
    \glb key \Refl{}.\Acc{} lose-\Av{}-\Pf{}-\Quot{} \Qp{} Lukáš say-\Av{}-\Pf{}//
    \glft ‘Lukáš said he lost his keys.’//
  \endgl
\xe

\pex
  \begingl
    \gla „Záščenžit” cy zíček.//
    \glb \First{}\Sg{} \Neg{}-come-\Av{}-\SupP{} \Qp{} say-\Av{}-\Pf{}//
    \glft \trsl{“I won’t be coming,” (he) said.}//
  \endgl
\xe

The use of pronouns in quoted clauses is similar to English, with the main
exception being the use of the reflexive \ird{se} if the subject of the quoted
clause is the same as the subject of the main clause. This is true even if the
subject of the main clause is a pronoun.

\pex
  \begingl
    \gla Se to obru na večera záščenžite cy Marek (dá) žiček. //
    \glb \Refl{} \Dem{} night-\Ins{} \Loc{} party-\Acc{} \Neg{}-come-\Av{}-\SupP{}-\Quot{} \Qp{} Marek \First\Sg{} say-\Av{}-\Pf{} //
    \glft \trsl{Marek/I said he/I won't be coming to the party tonight.}//
  \endgl
\xe

The verb \irdp{zěká}{to say} is called a \emph{verbum dicendi}\index{verbum
dicendi} from the Latin meaning ‘verb of speech/speaking.’ Other \emph{verba
dicendi} in Iridian include \irdp{vadá}{to think}; \irdp{kvuštá}{to hear};
\irdp{vidá}{to see}; \irdp{hloupá}{to ask}; \irdp{ohletá}{to remember};
\irdp{sehová}{to recount, to tell a story}. Note that although they are called
verbs ``of speaking'' they do not necessarily introduce speech as much as
function as grammaticalized tags marking the quotative,  which is more properly
analyzed to mark not just speech but inferentiality and evidentiality as well.

More complex \emph{verba dicendi} can be formed by using an imperfect converbial
construction (the converb form in \ird{-ěc}) with a canonical \emph{verbum
dicendi}. To illustrate this consider the following sentences in English:

\pex[*=?*,interpartskip=0pt]
\a\label{ex:vd1} She said no.
\a\label{ex:vd2} She whispered no.
\a\label{ex:vd3} She said no \emph{in a whisper}.
\a\label{ex:vd4} \ljudge{?} She said \emph{in a whisper} no.
\a\label{ex:vd5} \ljudge{??} She said \emph{whisperingly} no.
\xe

We see that both \emph{said} (\ref{ex:vd1}) and \emph{whispered} (\ref{ex:vd2})
are \emph{verba dicendi} in English. Nonetheless it's also obvious how
(\ref{ex:vd2}) is simply a function of (\ref{ex:vd1}), i.e., we can express
(\ref{ex:vd2}) in terms of (\ref{ex:vd1}), in this case using an adverbial
construction (\trsl{in a whisper}) as we see in (\ref{ex:vd3}) or the more
affected (\ref{ex:vd4}). Finally using a simple adverbial is theoretically allowed
in English (\ref{ex:vd5}), although as we see the resulting construction is
rather unwieldy or unnatural-sounding.

In Iridian, however, constructions like (\ref{ex:vd2}) are not permitted, with
preference given to adverbial (or more correctly, converbial)\index{converb}
constructions. Thus we translate (\ref{ex:vd2}) as:

\pex
\begingl
\gla Ne cy mišlec zíček.//
\glb no \mk{qp} whisper-\Cv{} say-\Av{}-\Pf{}//
\glft \trsl{(She) whispered no.}//
\endgl
\xe

When using \irdp{vadá}{to think} as the \emph{verbum dicendi} the verb in the
reported clause must be in the conditional. This is true whether or not the verb
in the main clause would otherwise have been in the conditional had it been in
an independent sentence.


The \emph{verbum dicendi} is often marked in the agentive voice, although
Iridian grammar also permits the verb to be marked in the patientive, but with
the resulting construction often having a more explanatory meaning.

\pex
\a
\begingl
  \gla Já mnou nehlí cy Martin spouvěc váževí.//
  \glb you correct \Cop{}.\Sbj{}.\Quot{} \Qp{} Martin agree-\Cv{}.\Ipf{} think-\Av{}-\Cont{}//
  \glft \trsl{Martin agrees that you are right.}//
\endgl
\a
\begingl
  \gla Já mnou nehlí cy Martin spouvěc vadneví.//
  \glb you correct \Cop{}.\Sbj{}.\Quot{} \Qp{} Martin agree-\Cv{}.\Ipf{} think-\Pv{}-\Cont{}//
  \glft \trsl{What Martin agrees to is that you are right.}//
\endgl
\xe

We see from  that when it comes to reported speech and similar constructions in
Iridian, the \ird{verbum dicendi}\index{verbum dicendi} is not necessary to
create a well-formed sentence. The same is true with the quotative particle
\ird{cy}. Both can be omitted without making the sentence grammatically
incorrect since the quotative particle is enough to identify the reported
clause.\index{reported speech}.

In most instances, however, removing either the main verb or the main verb and
the quotative particle can cause the resulting sentence to acquire a new
meaning. This is especially true when the quotative mood is used not to report
speech but to imply a certain unsureness on the part of the speaker about the
information being presented, or for the speaker to distance themself by implying
through the use of the quotative that the information is secondhand and not
theirs. Generally \ird{cy} is kept when the speaker is quoting themself, to
repeat or emphasize what they have said, or expletively, to express their
frustration or affirmation.

Interestingly, commands and requests are not treated as reported speech but as
regular subordinate clauses governed by \ird{to} and not by \ird{cy}.

When the quoted clause is a question, whether a direct one or not, the quoted
clause is preceded by the particle \irdp{a}{and} and the word
\irdp{ane}{whether} is used instead of \ird{cy}. The word \ird{ane} is also
used for verba dicendi that are interrogative in nature, such as
\irdp{préhoustá}{to ask},

\pex
\begingl
  \gla A Janek zdalšice ane préhousček.//
  \glb and Janek have:breakfast-\Av{}-\Pf{}-\Quot{} whether ask-\Av{}-\Pf{}//
  \glft \trsl{(He) asked (me) whether Janek has had breakfast yet.}//
\endgl
\xe

\pex
\begingl
  \gla A tóm to mládu hodinaže ane, ně svad postupeví.//
  \glb and book this year-\Ins{} finish-\mk{pv-ctpv-quot} whether \Pl{} fan be:excited-\Cont{}//
  \glft \trsl{His fans are excited to know if he'll finish his book this year.
}//
\endgl
\xe

The quotative is also triggered by phrases introduced with \irdp{ty}{according
to} or \irdp{záty}{contrary to,} with the latter requiring the subjunctive. 

\pex
\begingl
  \gla Messi a ty Marku debil neví.//
  \glb Messi and according:to Marek-\Ins{} spaz \Cop{}.\Sbj{}//
  \glft \trsl{Marek thinks Messi is a spaz.}//
\endgl
\xe

\pex
\begingl
  \gla Na Vrešlove a záty mamcě čestu papcě vednice stožišejí.//
  \glb \Loc{} Wrocław-\Acc{} and \Neg{}-according:to mother-\Dim{}-\Gen{} desire-\Ins{} father-\Gen{} see-\Pv{}-\SupP{} go-\Av{}-\Subj{}.\Pf{}-\Quot{}//
  \glft \trsl{Against my mother’s wishes, I went to Wrocław to see my father.}//
\endgl
\xe


\subsection{Bare quotatives and clause linking}

Quoted clauses in Iridian may also appear without an overt predicate, as well as
without being signalled by the quotative particle \ird{cy}. We will call this
construction a {\sc bare quotative} after the terminology in
\textcite{tomioka2019} in reference to embedded quotative constructions in
Japanese and Korean without overt predicates. The term as originally used by
these authors refer only to embedded quotatives in Japanese and Korean, but we
will be using it to refer to both an unselected (i.e., predicateless) quotative
in a subordinate clause (which we will call {\sc syntactic}) and in the main
clause (which we will call {\sc semantic}).

The choice to call the second type a semantic bare quotative is motivated by the
fact that an unselected quotative in the main clause is often used not to mark a
speech act but to indicate the epistemic value of (viz., to pass the speaker's
judgement on) a proposition. Nevertheless, we can still see it used as a true
quotative, as when the omission of the predicate or the quotative particle is
through mere ellipsis.

The first type, on the other hand, is mostly used as a clause-linking strategy.
The quotative construction is still considered as a speech act, but, like
converbial constructions or \ird{še} clauses, the relationship between the main
clause and the reported clause becomes interpreted as being one of causality, or
at least of dependency, although of course this causality or dependency is only
indirect, as we see in the examples below, where the embedded quotative and the
simple \ird{še} clause present to different interpretations.

\pex
  \a(adapted from \cite[3]{tomioka2019})\\
  \begingl
    \gla Pizba rážice še sad Markám nakdavtébik.//
    \glb rain stop-\Av{}-\Pf{}-\Quot{} \Com{} garden Marek-\Agt{} \Incp{}-clean-\Ben{}-\Pf{}//
    \glft \trsl{Marek began cleaning the garden, (saying/thinking) it finally stopped raining.}//
  \endgl
  \a\begingl
    \gla Pizba razek še sad Markám nakdavtébik.//
    \glb rain stop-\Av{}.\Pf{} \Com{} garden Marek-\Agt{} \Incp{}-clean-\Ben{}-\Pf{}//
    \glft \trsl{The rain having stopped, Marek began cleaning the garden.}//
  \endgl
\xe


\subsection{Epistemic extensions}

As in most other languages with an overt evidential system, the Iridian
quotative has secondary epistemic extensions. This may be realized either by
using the quotative by itself or through auxiliary epistemic markers. As we have
established in the previous sections, the quotative can be used by a speaker
both to distance themself from the statement on the one hand and to assert their
belief in its truthfulness on the other; the use of a secondary epistemic marker
eliminates this possible confusion in what would otherwise have been a
contradictory usage of the same grammatical category. These auxiliary particles,
nonetheless, may of course be left out in discourse if the speaker thinks the
epistemic usage of the quotative is clear enough from the context.

A speaker’s judgement of the truthfulness of a statement may be made clear by
the dubitative \ird{bude} or the affirmative \ird{toleto}. When using the
quotative to quote oneself, \ird{bude} expresses a disbelief predicated upon
surprise rather than on a judgement of a statement’s veracity; used the same
way, \ird{toleto} acquires a secondary meaning of insistence, even annoyance.

\pex
\begingl
  \gla Sól bude tahatnitejí.//
  \glb peace \Infer{} bring-\Pv{}-\SupP{}-\Quot{}//
  \glft \trsl{They say they come in peace but I don’t believe it.}//
\endgl
\xe

\pex
\begingl
  \gla Ma já bude ža konědnitejí to!//
  \glb but \Second{}\Sg{} \Infer{} already marry-\Pv{}-\SupP{}-\Quot{} \Rel{}//
  \glft \trsl{I still can’t believe you’re already getting married!}//
\endgl
\xe

\pex
\begingl
  \gla Marek toleto poslem všihnébice.//
  \glb Marek \Aff{} message-\Agt{} be:angry-\Ben{}-\Pf{}-\Quot{}//
  \glft \trsl{I’m telling you the message really made Marek angry.}//
\endgl
\xe

\pex
\begingl
  \gla Méva toleto sehovnáně!//
  \glb all \Aff{} recount-\Pv{}-\Ret{}-\Quot{}//
  \glft \trsl{But I’ve told you everything I know already!}//
\endgl
\xe

A speaker’s uncertainty may also be expressed using the quotative even when the
statement directly came from the speaker. The uncertainty may refer to both the
factuality of the statement or to its source. This strategy is used to signal
the speaker’s emotional or cognitive distance from the event. This may be
further complemented by the particle \ird{iz} which we will glossing here as
\Rep{} for reportative but only for the sake of convenience, in order to
distinguish the various auxiliary particles we have introduced here, as the
“reportative” does not exist as a true grammatical category in Iridian for our
purposes. \ird{Iz} implies a greater degree of disjunction between the speaker
and the statement than the plain quotative. Although it does not pass a
judgement on the truth value of the statement as do \ird{dube} or \ird{toleto},
\ird{iz} makes it clear that the statement did not come from the speaker and
that the responsibility for the statement does not lie on them. \ird{Iz} is
particularly common in newscasts or in other formal settings where the speaker
is communicating statements from another speaker or group and the identity of
the speaker or group has already been established earlier in the conversation
and is thus known to everyone.

\pex
\ird{Interiorministerium shléd o senátor Koupárám poto němstministar Novaka
dozakuzacunóvim arklaruma mnilounek. Na Ministerija še Ružómu ty
zěka\-mi\-te\-mu nežni posohredou, a viční němstministarí za Moshóva besuk
{\emph{iz}} Ministerija zázběro\-vnevíje. Akuzace \emph{iz} shlac
investěharnimejí a němstministar \emph{iz} udarklaržice za Ministara breví
paholžáše.}\smallskip\\
{\footnotesize\trsl{The Ministry of the Interior has released a statement today
regarding the accusations of misconduct levelled by Senator Koupár against
Deputy Minister Novak. According to its spokesperson, the ministry is currently
not in talks with Russia and has not sanctioned the reported Deputy Minister's
recent visit to Moscow. It is now investigating the allegations and has asked
Deputy Minister Novak to submit a brief to the Minister to explain his
actions.}}
\xe

Uncertainty on the truthfulness of the statement may also be expressed using the
inferential particles \ird{bylo} and \ird{atole}. Whereas \ird{iz} raises the
question of the character of the source and is neutral as to the speaker’s
commitment to it (although one can be understood simply by pointing out the fact
that the source is something other than oneself to be effectively passing
judgement) both \ird{bylo} and \ird{atole} reflect the speaker’s judgement.
\ird{Bylo} in general is used when the proposition is coming from the speaker
themself while \ird{atole} is used when the speaker thinks that the statement
can be inferred from the surrounding facts.

\pex
\begingl
  \gla Na Hospode bylo milestunitejí.//
  \glb \Loc{} Hospoda-\Acc{} perhaps have:dinner-\Lv{}-\SupP{}-\Quot{}//
  \glft \trsl{Maybe we can have dinner at the \emph{Hospoda} tonight?}//
\endgl
\xe

\pex
\begingl
  \gla Ně ruščevní šar atole na Roubžína ščenžáně.//
  \glb \Pl{} Russian-\Att{} tank \Infer{} \Loc{} Roubže-\Acc{} arrive-\Av{}-\Ret{}-\Quot{}//
  \glft \trsl{The Russian tanks must have reached Roubže by now.}//
\endgl
\xe

\section{Syntax of event and participant nominals}\index{nominalization!event
nominal}\label{sec:nomz-syntax}

\subsection{Gerunds and event nominals}

As we have established in \S~\ref{sec:nominalized}, Iridian has three forms of
nominalization\index{nominalization}: (1)~the mainly non-productive usage of the
nominalising \ird{-ou} with the verbal stem to form resultant nominals; (2)~the
use of \ird{-ou} together with the gerund-forming suffix\index{gerund} \ird{-c}
to form a verbal noun (which we call an event nominal or simply a gerund) and
which may either include the internal arguments of the parent verb or not; and
(3)~the formation of a participant nominal (cf.~\cite{okuna}) which nominalizes
not the event described by the verb but its participants.

Since gerunds\index{gerund} represent the nominalization of the
event\index{event nominal} described by the verb, they are therefore inherently
abstract and active in meaning. Since the nominalized forms are abstract, it
follows that they are also tenseless and aspectless. Iridian gerunds, however,
may be optionally marked for their lexical aspect or \foreign{aktionsart}
\index{aktionsart@\emph{aktionsart}}\index{lexical
aspect|see{\emph{aktionsart}}} using the continuous aspect suffix \ird{-eví}
(which subsequently becomes \ird{-év-} through sound change). It is important to
note that although a marker for grammatical aspect\index{aspect} is used,
what is being marked is lexical and not grammatical aspect; specifically, the
addition of \ird{-év-} only signifies that the action is iterative in nature and
thus the gerund itself remains tenseless\index{tense} and aspectless.\index{aspect}

\pex
    \a \ird{nidá} → \ird{nidouc}\\
        \trsl{the act of standing up}
    \a \ird{nidá} → \ird{nidévouc}\\
        \trsl{the act of standing up repeatedly}
\xe

In {\sc cen}s, both the agent and the patient are marked in the
genitive.\index{genitive}\footnote{\textcite{serekaite2020} argues that although
(in the case of Lithuanian, at least) the actor and the theme from the original
sentence both become marked in the genitive in the resulting complex event
nominal, the superficially indentical genitives are actually two distinct cases:
a higher genitive ({\sc gen.h}) assigned to agents and possessors and a lower
genitive ({\sc gen.l}) assigned to grammatical objects. Although this argument
is interesting and probably holds true as well in Iridian {\sc cen}s, we will
not make an effort to ascertain whether there is an actual difference in the two
genitive cases in Iridian as this is not needed for the purpose of this
grammar.} If both are present, the agent must always appear first. This
construction is quite common cross-linguistically, as we see in the examples
below.

\pex
\a\begingl
    \gla Mlazcí praví na Mnihe poznohouštou na zahrana nemniček.//
    \glb brother-\Dim{}-\Gen{} law-\Gen{} \Loc{} Munich-\Acc{} \Ger{}-study-\Nz{} \Loc{} beginning-\Acc{} surprise-\Av{}-\Pf{}//
    \glft \trsl{My brother's studying law (i.e., my brother's decision to study law) in Munich surprised us at first.}//
\endgl
\a Lithuanian\index{Lithuanian} (\cite[1]{serekaite2020})\\
\begingl
    \gla Jono augalų sunaikinimas.//
    \glb Jonas-\Gen{} plants-\Gen{} \Pfv{}-destroy-\Caus{}-\Nz-\Nom{}.\M{}.\Sg{}//
    \glft \trsl{Jonas' destruction of plants}//\deftagex{doubgen}\deftaglabel{lithuanian}
\endgl
\a Tagalog\index{Tagalog} (\cite[22]{hsieh2019})\\
\begingl
    \gla (Ang) Pagluluto ni Harvey (ng manok) ang nangyari.//
    \glb \Nom{} \Ger{}$\sim$cook \Gen{} Harvey \Gen{} chicken \Nom{} happen.\Pfv{}//
    \glft \trsl{What happened was Harvey's cooking (of chicken).}//\deftagex{doubgen}\deftaglabel{tagalog}
\endgl
\xe

The use of the genitive\index{genitive} to mark both the actor and the theme in
the original sentence is of course a recipe for ambiguity. When only one of
either the actor or the theme is present in the {\sc cen}, the ambiguity is on
whether the noun marked represents the one or the other, as, e.g., the phrase
\ird{Jancí podohletou} which can be interpreted to mean either \trsl{the act of
remembering Janek} or \trsl{Janek's act of remembering} without any further
information. A second ambiguity arises when both the actor and the theme are in
the sentence as it is unclear, without any context, the genitive is actually
being used to mark their thematic role in the originally or is in fact a
possessive. The same is true in, for example, Lithuanian\index{Lithuanian} where
as \textcite{serekaite2020} points out, sentence
(\getfullref{doubgen.lithuanian}) can also be alternatively translated as
\trsl{[the] destruction of Jonas's plants}.

The first type of ambiguity is resolved in English\index{English} by using word
order: in general, a prepositive genitive (i.e., using the clitic \foreign{'s}
or the possessive form of a pronoun) is used when the noun in the genitive case
in the {\sc cen} represents the actor (e.g., \trsl{John's remembering}) while a
postpositive genitive is used when the noun in the genitive represents the theme
(e.g., \trsl{the remembering of John}). This in turn, can be extended to the
second type, e.g., \trsl{John's remembering of Margaret}. However, the
obligatorily head-final nature of Iridian syntax means that such strategy is not
possible. Instead, the strategy used in Iridian is more similar to the one found
in Tagalog\index{Tagalog} where the theme may be marked using the oblique
\foreign{sa}\footnote{This becomes \foreign{kay} before proper nouns.} instead
of the genitive \foreign{ng}.\footnote{ To call \foreign{ng} (pronounced [nɐŋ])
as a genitive marker is simplistic (even erroneous) but should be enough for the
purpose of our discussion. } Thus we can restate (\getfullref{doubgen.tagalog})
as follows:

\pex{Tagalog\index{Tagalog} (modified from \cite[22]{hsieh2019})}\\
\begingl
    \gla (Ang) Pagluluto ni Harvey {sa} manok ang nangyari.//
    \glb \Nom{} \Ger{}$\sim$cook \Gen{} Harvey \Obl{} chicken \Nom{} happen.\Pfv{}//
    \glft \trsl{What happened was Harvey's cooking of \emph{the} chicken.}//
\endgl
\xe

An immediate consequence of replacing the genitive \foreign{ng} with the oblique
marker \foreign{sa/kay} is that the theme is now interpreted as definite
(cf.~\cite[3,\,40]{kaufman2009}). The use of the oblique to mark the theme can
be used even when only one element is present in the event nominal; in fact,
when the theme is known as definite for a fact (e.g., if it is a person), the
choice between the oblique and the genitive is what distinguishes the actor and
the theme. Thus we have

\pex[interpartskip=0pt]
    \a Choice between \Obl{} and \Gen{} distinguishing actor from theme
    \beginsubsub\index{Tagalog}
        \b{--}{\foreign{pagtawag kay {\nf{[\Obl{}]}} Harvey}\\ \trsl{the act of calling Harvey}}
        \b{--}{\foreign{pagtawag ni {\nf{[\Gen{}]}} Harvey}\\ \trsl{Harvey's act of calling}}
    \endsubsub
    \a Resolving ambiguity by obligatory replacement of \Gen{} by \Obl{} in the theme:
    \beginsubsub
        \b{--}{\foreign{pagtawag ni {\nf{[\Gen{}]}} Harvey sa {\nf{[\Obl{}]}} kasama}\\
        \trsl{Harvey's act of calling his \mbox{colleague}}}
        \b{--}{\foreign{pagtawag ni {\nf{[\Gen{}]}} Harvey ng {\nf{[\Gen{}]}} kasama}\\
        \trsl{Harvey's act of calling a colleague}}
    \endsubsub
    \a New ambiguity introduced by changing the word order:
    \beginsubsub
        \b{--} {\foreign{pagtawag ng {\nf{[\Gen{}]}} kasama ni {\nf{[\Gen{}]}} Harvey}\\
        \trsl{Harvey's act of calling a colleague} or \trsl{The act of calling Harvey's colleague}}
    \endsubsub
    \a Ungrammatical form, with both the theme and actor marked in the oblique:
    \beginsubsub
        \b{--}{\ljudge{*}\foreign{pagtawag kay {\nf{[\Obl{}]}} Harvey sa {\nf{[\Obl{}]}} kasama,}\\
        \trsl{Harvey's act of calling a colleague}}
    \endsubsub
    \a Double genitive, with both indefinite actor and theme:
    \beginsubsub
        \b{--}{\foreign{pagtawag ng {\nf{[\Gen{}]}} tao ng {\nf{[\Gen{}]}} kasama,}\\
        \trsl{a person's act of calling a colleague} or \trsl{a colleague's act of calling of a person}} 
    \endsubsub
\xe


In Iridian, the a \ird{na} clause corresponds to the Tagalog\index{Tagalog} use
of the oblique to indicate a definite theme in a {\sc cen}. 

% nemnetá from CS m{\yer}neti to think + ne not



\subsection{Participant nominals}

Participant nominals are formed by nominalizing a finite verb phrase with the
suffix \ird{-ou}. The resulting noun refers back to a participant in the event
rather than the event itself, with the role determined by the grammatical voice
in which the original verb phrase is marked. Consequently, participant nominals
are inherently definite in meaning.

\pex
\a\begingl
    \gla Jancí materška najevěc shradnaní.//
    \glb Janek-\Gen{} stepmother drive-\Cv{} die-\Pv{}-\Ret{}//
    \glft \trsl{Janek's stepmother was killed in a car crash.}//
\endgl
\a\begingl
    \gla Jancí materšcí najevěc shradněnou policám zánehévorneví.//
    \glb Janek-\Gen{} stepmother-\Gen{} drive-\Cv{} die-\Pv{}-\Ret{}-\Nz{} police-\Agt{} \Neg{}-\Caus{}-know-\Pv{}-\Cont{}//
    \glft \trsl{The police still hasn't identified the person Janek's stepmother has killed in the crash.}//
\endgl
\xe

The creation of participant nominals is a very common strategy in Iridian.
Participant nominalization is also used to shift the focus of the sentence from
the event to the participant. For example, transforming the sentence \irdp{Janek
shražek}{Janek died} into \irdp{Janek shražkou}{It is Janek who died} or more
emphatically, \irdp{Shražkou Janek}{It is Janek who died} changes the emphasis
in the sentence.

\section{Converbial constructions}\label{converbs-syntax}\index{converb}

\subsection{In general}

In \S~\ref{sec:converb}, we have defined a converb as a non-finite verb form
that is often used adverbially. In this section, we will discuss the syntax of
converbial constructions in Iridian.

The most common type of converbial constructions involves the main verb preceded
by the imperfective converbial form of a secondary verb. The secondary verb
normally specifies the manner or the means by which the action described by the
main verb is performed. Adverbial constructions such as these tend to be used
even where English, for example, would use a single verb. For example, in the
sentence \trsl{He cut the branch} would be analyzed in Iridian as \trsl{He
removed the branch by cutting} as Iridian would interpret the verb `cut' as used
in the first sentence as encoding both the action performed and the manner in
which it was performed. Although the second sentence below is not necessarily
incorrect, it would sound unnatural in Iridian.

\pex
\a\begingl
  \gla Platek odněc rutnik.//
  \glb leaf cut-\Cv{}.\Ipf{} remove-\Pv{}-\Pf{}//
  \glft \trsl{(He) removed the leaf by cutting it.}//
\endgl
\a\begingl
  \gla\ljudge{?}Platek odnenik.//
  \glb leaf cut-\Pv{}-\Pf{}//
  \glft \trsl{(He) cut the leaf.}//
\endgl
\xe

\subsection{Temporal constructions}

A converbial construction is often used in temporal clauses\index{temporal
clause}, with the imperfective converbial form used when the action is
unfinished or continuing and the perfective otherwise. When used in a temporal
clause, the converb may sometimes be separated from the main clause by the
particle \ird{si}.\footnote{\ird{Si} is virtually never used in the spoken
language.}

\pex
\begingl
\gla Otvěc (si) na Varšave možlašaní.//
\glb be:young-\Cv{}.\Ipf{} when \Loc{} Warsaw-\Acc{} understand-\Av{}-\Ret{}//
\glft \trsl{When I was young, we used to live in Warsaw.}//
\endgl
\xe

\subsection{Causal clauses}

Clauses expressing reason are usually expressed by a converbial construction.
The antecedent and the main clause may be connected with \irdp{am}{because,}
although this is often dropped in casual speech.

\pex
\begingl
\gla Za prove záznohouštu Martin meštnašek.//
\glb for exam-\Acc{} \Neg{}-study-\Cv{}.\Pf{} Martin fail-\Av{}-\Pf{}//
\glft \trsl{Martin failed the exam because he didn't study.}//
\endgl
\xe


\pex
\begingl
\gla Kinoteka stožílá to všihněc mámka zachovažek.//
\glb cinema-\Acc{} go-\Av{}-\Sbj{}.\Ipf{} \Rz{} be:angry-\Cv{}.\Ipf{} mother-\Dim{} allow-\Av{}-\Pf{}//
\glft \trsl{Since she was still mad at us, Mum did not let us go to the movies.}//
\endgl
\xe


\subsection{Similarities with the Czech and Slovak transgressive}

Converbs in Iridian have parallel usage as the
transgressive\index{transgressive} conjugations in \printlang{cs}\index{Czech}
and Slovak\index{Slovak}. It is the consensus among scholars of the languages,
though, that the converbial forms in Iridian and the transgressive forms in
Czech and Slovak, developed independently of each other; although to what extent
one influenced the other is still the subject of debate. The converbial forms in
Iridian have more varied uses than the transgressives in Czech (Slovak having
kept only the present transgressive form), and whereas the latter forms have
largely fallen in disuse (relegated to the literary register) in both Czech and
Slovak, converbial forms are still widely used in Iridian.

Although Czech grammarians use the terms `past' and `present' to distinguish
between the two forms used in the language, the distinction is actually one of
aspect\index{aspect}, as in Iridian. In general, the past transgressive form
corresponds with the perfect converbial form, and may be used to indicate a
foregoing action; the present transgressive, on the other hand, corresponds to
the imperfect converb and is used to indicate a coincident/contemporaneous
action.

This correspondence is not complete, however. For example, consider this
sentence in Czech\index{Czech}: \foreign{Děti, {vidouce} babičku, vyběhly
ven}{The children, seeing their grandmother, ran outside.} The verb in the
transgressive clause is in the present tense in this case, while in Iridian, the
same sentence will be translated with the perfective as follows:

\pex
\begingl
\gla Šášlika vedu byl naladěc mnilžek.//
\glb grandmother-\Dim{}-\Acc{} see-\Cv{}.\Pf{} children run-\Cv{}.\Ipf{} go:out-\Av{}-\Pf{}//
\glft \trsl{The children, having seen their grandmother, ran outside.}//
\endgl
\xe

The Czech\index{Czech} sentence above can alternatively be translated using the
imperfective converbial form, but this would put a stronger emphasis on the two
actions happening at the same time and so the original construction can be
considered as the more idiomatic one.

\subsection{In fixed expressions}

The past converbial form is used in expressing gratitude, approbation or
condolencess, or in asking for forgiveness. This usage is idiomatic and the
actions do not necessarily need to have been completed. The main clause is often
in the hortative mood\index{hortative mood} and separated from the converb
clause with \irdp{am}{because.} Moreover, this usage, unlike most converbial
constructions, allow the verb of the converb clause to have a different subject
as long as such subject is marked explicitly in the agentive case. However,
since the converbial form of verbs are invariable, if the subordinate clause
requires further complexity when it comes to the verb in the converb clause, a
dependent \ird{še} clause may be use instead of a converb.

\pex
\a Expressing gratitude:\\
\begingl
\gla Stranu am luhninká.//
\glb help-\Cv{}.\Pf{} because thank-\Pv{}-\Hort{}//
\glft `Thank you for helping.'//
\endgl
\a Asking for forgiveness:\\
\begingl
\gla Lěnu záščenu am rozvedniká.//
\glb on:time-\Ins{} \Neg{}-arrive-\Cv{}.\Pf{} because forgive-\Pv{}-\Hort{}//
\glft `Sorry for being late.'//
\endgl
\a Expressing condolences:\footnote{Compare this example to the following, where
a converb clause cannot be used:

\ex[lingstyle=fnex,belowexskip=-1em]
\begingl
\gla Pápka na puvode shradniš to množniká.//
\glb father \Loc{} war-\Acc{} die-\Pv{}-\Subj.\Pf{} \Rz{} with console-\Pv{}-\Hort{}//
\glft `I'm sorry to hear your father died (\emph{lit.,} was killed) in the war.'//
\endgl\xe}\\
\begingl
\gla Pápkám shradu am množniká.//
\glb father-\Dim{}-\Agt{} die-\Cv{}.\Pf{} because console-\Pv{}-\Hort{}//
\glft `I'm sorry for your father's death.'//
\endgl
\a Expressing approbation:\\
\begingl
\gla Prove vlastnu am prehodniká.//
\glb exam-\Acc{} pass-\Cv{}.\Pf{} because praise-\Pv{}-\Hort{}//
\glft \trsl{Congratulations for passing the exam!}//
\endgl
\xe

\section{Relative and comparative
constructions}\label{relativecomparative}\index{comparative construction}

The clitic\index{clitic} \ird{tám} is used to form simple comparative and
relative constructions. \ird{Tám} is often ommitted where the comparison can be
implied from context. In this construction, the standard of
comparison\index{standard of comparison} (the noun preceded by `than' in
English\index{English}) is unmarked and the noun being compared marked in the
agentive\index{agentive case} if it is a positive/negative comparison, or in the
instrumental\index{instrumental case} if it is a correlation.

\pex
\a\begingl
\gla Janek(-tám) Markám nestaževí.//
\glb Janek Marek-\Agt{} tall-\Cont{}//
\glft \trsl{Marek is taller than Janek.}//
\endgl
\a\begingl
\gla Janek(-tám) Marku nestaževí.//
\glb Janek Marek-\Ins{} tall-\Cont{}//
\glft \trsl{Marek is as tall as Janek.}//
\endgl
\xe

Note that \ird{tám} can only be used with the copulative form of the stative
verb\index{stative verb}, as the attributive and nominal forms have separate
conjugated comparative forms. When using these forms, however, the standard of
comparison is marked in the genitive\index{genitive case}. In relative
constructions, the instrumental\index{instrumental case} is also replaced with
the genitive\index{genitive case}, but the modifier \ird{zní}, \trsl{same} is
added before the stative verb\index{stative verb}.

\pex
\a
\begingl
\gla Jancí nestašení hloc mlazka.//
\glb Janek-\Gen{} tall-\Comp{}-\Att{} boy brother-\Dim{}//
\glft \trsl{The boy who is taller than Janek is my brother} (\emph{Lit.,} \trsl{The taller-than-Janek boy is my brother.})//
\endgl
\a
\begingl
\gla Jancí zní nestažení hloc mlazka.//
\glb Janek-\Gen{} same tall-\Comp{}-\Att{} boy brother-\Dim{}//
\glft \trsl{The boy who is as tall as Janek is my brother.}//
\endgl
\xe

\ird{Tám} can be relativized by appending the clitic\index{clitic} \ird{to}.
When used with \ird{tám-to} the standard of comparison is marked in the
patientive case\index{patientive case}. The use of tám-to in relative clauses is
discussed in further detail in the next chapter.

\ex
\begingl
\gla Viktor na shlopa tám-to nestážek.//
\glb Viktor \Loc{} siblings-\Acc{} \Comp{}=\Rz{} be:tall-\Av{}-\Pf{}//
\glft \trsl{Among the siblings, Viktor grew up to be the tallest.}//
\endgl
\xe

\ex
\begingl
\gla Jankám Marka tám-to zuštalébik ko Tereza//
\glb Janek-\Agt{} Marek-\Acc{} \Comp{}=\Rz{} be:happy-\Ben{}-\Pf{} \Lnk{} Tereza//
\glft \trsl{Tereza, whom Janek made happier than Marek}//
\endgl
\xe

\ex
\begingl
\gla Marka tám-tóví zuštalébik ko oblašc//
\glb Marek-\Acc{} \Comp{}=\Rz{}-\Gen{}= be:happy-\Ben{}-\Pf{} \Lnk{} pet//
\glft \trsl{the pet [of the person who was made happier than Marek]}//
\endgl
\xe

Iridian does not have a morphologically distinct superlative construction. For
example, \ird{pizdení} (from \ird{pizdá}, \trsl{to be big}) can either mean
\trsl{bigger} or \trsl{biggest} depending on context. Where the meaning cannot
be easily implied from context, the word \ird{ohnu} (derived from the word
\ird{ohna}, \trsl{first} in the instrumental case) is often used as quantifier.

\pex
\a
\begingl
\gla Univerzitet na razmeka pizdenou.//
\glb university \Loc{} city-\Acc{} be:big-\Comp{}-\Nz{}//
\glft \trsl{(This) university is the biggest in the city.}//
\endgl
\a
\begingl
\gla Univerzitet na razmeka ohnu pizdenou.//
\glb university \Loc{} city-\Acc{} first-\Ins{} be:big-\Comp{}-\Nz{}//
\glft \trsl{(This) university is the biggest in the city.}//
\endgl
\xe

When using an adverbial construction with the instrumental case to modify or
quantify the comparison, the adverbial phrase must immediately precede the
stative verb if in the attributive or nominal form, or the particle \ird{tám}
otherwise. The same is true with invariable modifiers like \ird{nahte},
\trsl{too much}, \ird{dnu}, \trsl{a bit}, etc.

\ex
\begingl
\gla To bagáž jánám u 10 kilográmu tám prékveví.//
\glb \Dem{}.\Prox{} baggage \Dem{}.\Med{}-\Agt{} around 10 kilogram-\Ins{} \Comp{}= heavy-\Cont{}//
\glft \trsl{This baggage is heavier by about 10 kilograms than that one.}//
\endgl
\xe

\ex
\begingl
\gla u 10 kilográmu prékvení bagáž//
\glb around 10 kilogram-\Ins{} heavy-\Comp{}-\Att{} baggage//
\glft \trsl{the baggage, which is heavier by about 10 kilograms}//
\endgl
\xe

\ex
\begingl
\gla Nahte pizdenou zmažnikóveš.//
\glb too:much big-\Comp{}-\Nz{} make-\Pv{}-\Pf{}-\Nz{}-\Second{}\Sg{}//
\glft \trsl{The much bigger one is the one you made.}//
\endgl
\xe

\section{Questions}\label{sec:questions-syntax}
\index{questions!syntax of}
\index{interrogative sentence|see{questions!syntax of}}

\subsection{Yes-no questions}\label{sec:questions-yesno}

A declarative sentence can be turned into a yes-no question by a simple rise in
intonation, as in English. Alternatively, the interrogative particle \ird{lí}
can be used. \ird{Lí} like all adverbial particles are proclitic and must
necessarily appear before the predicate. While \ird{lí} is optional, its
omission when forming a yes-no question would often be imply some level of
surprise or disbelief on the part of the speaker. Thus a sentence like
\irdp{Janek uzdravževí}{Janek is sleeping} can be transformed into a yes-no
question as either \irdp{Janek uzdravževí?}{Is Janek sleeping?} or \irdp{Janek
lí uzdravževí?}{Is Janek sleeping?} with the choice decided by the current
context. In the written language, especially in longer sentences or in formal
contexts, the interrogative particle is never omitted. The same is true with
copular sentences.

\pex
\begingl
  \gla Balžaróma Europevní Unijí lí čelina?//
  \glb Bulgaria European-\Att{} Union-\Gen{} \Q{} member//
  \glft \trsl{Is Bulgaria a member of the European Union?}//
\endgl
\xe

The particle \ird{děne} (v. \S\,\ref{sec:class3-particles}) is used to form
indirect questions, similar to the English \trsl{I wonder\ldots}. \ird{Děne} and
\ird{lí} may be used together in the same sentence, especially in the written
language, although in colloquial speech, \ird{lí} would often be dropped in such
cases. The use of \ird{děne} may sometimes be used to `soften' an otherwise
direct question by making it less direct.

\pex
\begingl
  \gla Marek na Praha děne lí že ščenžaní?//
  \glb Marek \Loc{} Prague-\Acc{} \Spec{} \Q{} \Pfv{} arrive-\Av{}-\Ret{}//
  \glft \trsl{I wonder if Marek has already arrived in Prague.}//
\endgl
\xe

The scope of a yes-no question is generally understood to be that of the whole
utterance, although the focus may be shifted to any constituent of the sentence,
subject to the usual rules for topicalization discussed in
\S\,\ref{sec:topicalization}.

\pex
\a\begingl
  \gla Stám Kovárž nevo séstu o leguánu děne lí hvaružnašách?//
  \glb mister Kovárž later convention-\Ins{} about iguana-\Ins{} \Spec{} \Q{} give:a:speech-\Av{}-\Ctp{}//
  \glft \trsl{Would Mr Kovárž be giving a speech about iguanas later at the convention?}//
  \endgl
\a\begingl
  \gla Stám Kovárž nevo séstu o leguánu děne lí hvaružnašáchou?//
  \glb mister Kovárž later convention-\Ins{} about iguana-\Ins{} \Spec{} \Q{} give:a:speech-\Av{}-\Ctp{}-\Nz{}//
  \glft \trsl{Would it be Mr Kovárž who would be giving a speech about iguanas later at the convention?}//
  \endgl
\xe

Yes-no questions are similarly formed from existential sentences by the use of
\ird{lí} and/or \ird{děne}. However, only \ird{ješ} accepts this transformation,
and a negative existential sentence would first need to be transformed into a
positive one before \ird{lí} and/or \ird{děne} can be added.

\pex
\a \begingl
  \gla Na to parka niho seh.//
  \glb \Loc{} \Dem{} park-\Acc{} \N{}\Exst{} flower//
  \glft \trsl{There are no flowers in this park.}//
  \endgl
\a \begingl
  \gla Na to parka lí ješ seh?//
  \glb \Loc{} \Dem{} park-\Acc{} \Q{} \Exst{} flower//
  \glft \trsl{Are there flowers in this park?}//
  \endgl
\xe

A sentence like \irdp{Na to parka lí niho seh?}{Aren't there any flowers in this
park?} is not grammatical.

Tag questions\index{tag question} may be formed by appending the phrase \irdp{lí
zám lět}{isn't it the truth?} or variants like \irdp{lí što zám lět}{is it
indeed the truth?} or \irdp{děne lí zám lět}{I wonder if it isn't the truth}
(cf. Russian\index{Russian} \textit{\cyrtext не правда ли}) to the end of the
sentence. In colloquial speech, these forms may be considered too formal or
old-fashioned and alternatives like \irdp{da}{yes,} \irdp{jó}{yeah,}
\irdp{let}{truth} or the Slavic borrowing \irdp{pravda}{truth} may be used
instead.

\pex
\begingl
\gla Traví kupšek, lí zám lět? /da?//
\glb bread-\Gen{} buy-\Av{}-\Pf{} \Q{} \Neg{} truth yes//
\glft \trsl{You bought some bread, didn't you? /right?}//
\endgl
\xe

Unlike in English, tag questions are invariable, i.e., sentences like
\trsl{Janek is studying, isn't he?} and \trsl{Janek studied yesterday, didn't
he?} can be translated as \ird{Janek znohouščime, da?} and \ird{Janek
znohoušček, da?} respectively, with both sentences having an identical tag part.

\subsection{Alternative questions}\label{sec:alternative-questions}
\index{alternative question}\index{choice question|see{alternative question}}

\subsection{Content questions}\label{sec:content-questions}
\index{wh-question@\emph{wh-}question}\index{information
question|see{\emph{wh}-question}}\index{content
question|see{\emph{wh}-question}}

Content questions, also known as \emph{wh}-questions, are formed using the
interrogative pronouns \irdp{jede}{who,} \irdp{ježe}{what,} \irdp{jena}{where,}
etc.\footnote{ A full list of interrogative pronouns can be found in
\S~\ref{sec:int-pron}. } Iridian requires the \emph{wh}-phrase to be moved to
the beginning of the sentence, thus causing it to occupy the topic position.
This \emph{wh}-fronting\index{wh-fronting@\emph{wh}-fronting} consequently
causes the voice of the main verb to be reframed to accomodate the new topic.
More commonly, especially colloquial Iridian\index{colloquial Iridian}, this
also means the nominalization\index{nominalization} of the main verb phrase,
essentially making the question an equational sentence.

\pex
\a\begingl
\gla Karel na Roubžení verštáta možlaševí.//
\glb Karel \Loc{} Roubže-\Gen{} suburbs-\Acc{} live-\Av{}-\Cont{}//
\glft \trsl{Karel lives in the suburbs of Roubže.}//
\endgl
\a\begingl
\gla Jena Karlám možlouneví? /možlounívou?//
\glb where Karel-\Agt{} live-\Lv{}-\Cont{} live-\Lv{}-\Cont{}-\Nz{} //
\glft \trsl{Where does Karel live?}//
\endgl
\xe

Alternatively, the element being questioned may be replaced with a question word
without changing the original word order, in which case the addition of the
clitic \ird{no} is required. Note that questions formed this way generally have
a more emphatic meaning.

\pex
\begingl
\gla Karel jena-no možlaševí?//
\glb Karel where=\Q{} live-\Av{}-\Cont{}//
\glft \trsl{Where did you say Karel lived?}//
\endgl
\xe


\emph{Wh}-fronting may sometimes cause peripheral elements of a phrase to be
moved together with the \emph{wh}-item to the beginning of the sentence, a
phenomenon linguists call `pied-piping' (\cite[263-4]{ross1967}). When this
occurs, Iridian is more conservative than English in that it usually keeps the
same question word instead of replacing it with a specialized one (in English,
normally, `which'); it may, however, use \irdp{jak}{which} if the expected
answer to the question is an element of a class, i.e., not unique. Consider, for
example, the two questions below:

%%%% zuscve, cf. Cz sodestvi, Ru. sodestvo
\pex
\a
\begingl
\gla Jena zuscve možlounívou?//
\glb where neighborhood live-\Lv{}-\Cont{}-\Nz{}//
\glft \trsl{Which (\emph{lit.,} where) neighborhood do you live in?}//
\endgl
\a
\begingl
\gla Jak kvartír možlounívou?//
\glb which apartment live-\Lv{}-\Cont{}-\Nz{}//
\glft \trsl{Which of these apartments is the you live in?}//
\endgl
\xe

In cases where there are multiple \emph{wh}-elements in the sentences, they are
normally all fronted, with the main question word first followed by the rest in
order of importance. Interestingly, too, any or all of the fronted
\emph{wh}-items may be pluralized with \ird{ně} if the speaker expects that the
answer is plural.

\pex
\a\begingl
\gla Jede ježe jena hloupškou?//
\glb who what where ask-\Av{}-\Pf{}-\Nz{}//
\glft \trsl{Who asked what where?}//
\endgl
\a\begingl
\gla Ně jede ježe jena hloupškou?//
\glb \Pl{}= who what where ask-\Av{}-\Pf{}-\Nz{}//
\glft \trsl{Which persons asked what where?}//
\endgl
\a\begingl
\gla Jede ně ježe jena hloupškou?//
\glb who \Pl{}= what where ask-\Av{}-\Pf{}-\Nz{}//
\glft \trsl{Who asked what things where?}//
\endgl
\xe

In the case of more complex \emph{wh}-questions involving the movement of a
  \emph{wh}-item from an embedded clause, Iridian is similar to
  Bulgarian\footnote{ \posscite{rudin1988} description on the nature of multiple
  \emph{wh}-fronting in Bulgarian\index{Bulgarian} involves the movement of the
  \emph{wh}-item to closest interrogative SpecCP, which does not necessarily
  need to occupy the topic position in the sentence. Compare, for example the
  following sentences in Bulgarian and Iridian.

  \ex[lingstyle=fnex,belowexskip=-1em,aboveglftskip=1pt]
  Bulgarian\index{Bulgarian} (\emph{ibid.,} 451)\smallskip\\
    \begingl 
    \gla Boris na kogo kakvo kaza [če šte {dade --- ---]}? //
    \glb Boris to whom what said that will give-\Third{}\Sg{}//
    \glft \trsl{What did Boris say that (he) would give to whom?}//
  \endgl
  \xe
\smallskip
  \ex[lingstyle=fnex,belowexskip=-1em,aboveglftskip=1pt]
    \begingl 
    \gla Ježe jehát Borisám ditnách to zíknou?//
    \glb what to:whom Boris-\Agt{} give-\Pv{}-\Ctp{} \Rz{} say-\Pv{}-\Pf{}-\Nz{}//
    \glft \trsl{What did Boris say that (he) would give to whom?}//
  \endgl
  \xe

} in requiring all the \emph{wh}-items to be fronted (cf.~\cite[450]{rudin1988}).

\pex
\begingl
\gla Ježe jehát dejatnách to zíknou?//
\glb what to:whom give-\Pv{}-\Ctp{} \Rz{} say-\Pv{}-\Pf{}-\Nz{}//
\glft \trsl{What did she say that she will give to whom?}//
\endgl
\xe


\subsection{Answering questions}\label{sec:ansyn}

Most yes-no questions may be answered by repeating the focal word or phrase in
the original question or echoing the syntax of the question itself.

\ex
\vtop{\halign{%
#\hfil& \qquad #\hfil\cr
\ird{---\,Kartuškí tak slouveževí?} & \trsl{\small Do they sell potatoes here?}\cr
\ird{---\,Slouveževí?} & \trsl{\small They do.}\cr
}}
\xe

Alternatively, the question may be answered by \irdp{da}{yes} or \irdp{ne}{no,}
both of which have been adapted from Common Slavic\index{Common Slavic}. In
colloquial speech it is also common to use \ird{já} or \ird{jó} for \trsl{yes}
(most likely borrowings from German\index{German}). These polarity words may be
used alone or in combination with the echo response. In general, the order does
not matter, although it is more common for the polarity word to appear after the
echo response. Unlike English \trsl{yes,} \ird{da} is used when confirming the
question posed by the speaker, whether or not it is in the affirmative or in the
negative. When denying or negating a question, Iridian uses \ird{ne} is used
when the original question was framed in the negative and \ird{ale} otherwise.

\ex
\vtop{\halign{%
#\hfil\hfil\cr
\ird{---\,Lošní Nolaní vilm ža oudnenik?}\cr
\ird{---\,Ža oudnenik, da. Má záčesčik.}\cr
\ird{---\,Ne, po zoudnenik.}\smallskip\cr
\trsl{Have you seen Nolan's new film?}\cr
\trsl{I've seen it, yes. But I didn't like it.}\cr
\trsl{No, I haven't seen it yet.}\cr
}}\xe

\ex\vtop{\halign{%
#\hfil\hfil\cr
\ird{---\,No daní trehlo za banka podarnílá cy Janek záléháček?}\cr
\ird{---\,Léháček, ale. Má avtem bych hebo.}\cr
\ird{---\,Záléháček, da.}\smallskip\cr
\trsl{Weren't you advised by Janek to submit your tax return to the bank?}\cr
\trsl{He did, yes. But my car broke down yesterday.}\cr
\trsl{No, he didn't advise me to.}\cr
}}
\xe

\ird{Da} (or sometimes \ird{a da}) may also preface answers to questions as a
form of intensifier, or to indicate that the speaker considers the answer to the
question as an obvious truth.

\ex
\vtop{\halign{%
#\hfil& \qquad #\hfil\cr
\ird{---\,Na muzla ješ vdenikou.} & \trsl{I saw someone at the mall today.}\cr
\ird{---\,Jede?} & \trsl{Who?}\cr
\ird{---\,Da Janek.} & \trsl{Well, Janek, of course.}\cr
}}\xe

The answer does not need to be positive for \ird{da} or \ird{a da} to be used.

\ex\vtop{\halign{%
#\hfil\hfil\cr
\ird{---\,Šabatu de koncerta stožit?}\cr
\ird{---\,A da ne. To kapela šem záčesčeví.}\smallskip\cr
\trsl{Are you coming to the concert on Saturday?}\cr
\trsl{Well no, I don't even like that band.}\cr
}}
\xe


As for questions involving existential constructions


\section{Negation}\label{sec:negation}
\index{negation}

In Iridian sentences, negation is performed by the particle \ird{zám}, which
attaches to the beginning of the word or phrase  it negates. The default
position of the negative particle is before the main verb where it surfaces as
\ird{z-} before vowels, \ird{ž-} before \emph{i}-glides, and \ird{zá-}
elswehere. This elision does not occur where \ird{zám} appears elswehere in
the sentence.

\pex
\a
\begingl
    \gla Janek Martina Markám {zá}hévoržébik.//
    \glb Janek Martin-\Acc{} Marek-\Agt{} \Neg{}know-\Ben{}-\Pf{}//
    \glft \trsl{Marek did not introduce Janek to Martin.}//
\endgl
\a
\begingl
    \gla {Zám} Janek Martina Markám hévoržébik.//
    \glb \Neg{} Janek Martin-\Acc{} Marek-\Agt{} know-\Ben{}-\Pf{}//
    \glft \trsl{It was not Janek whom Marek introduced to Martin.}//
\endgl
\a
\begingl
    \gla Janek {zám} Martina Markám hévoržébik.//
    \glb Janek \Neg{} Martin-\Acc{} Marek-\Agt{} know-\Ben{}-\Pf{}//
    \glft \trsl{It was not Martin whom Marek introduced Janek to.}//
\endgl
\a
\begingl
    \gla Janek Martina {zám} Markám hévoržébik.//
    \glb Janek Martin-\Acc{} \Neg{} Marek-\Agt{} know-\Ben{}-\Pf{}//
    \glft \trsl{It was not Marek who introduced Janek to Martin.}//
\endgl
\xe

\ird{Zám} attaches directly to the word or phrase it negates, although it is
also common, especially in spoken Iridian, to append the clitic \ird{-te} after
the word being negated by \ird{zám} to provide more emphasis on the negation.
This is a fairly recent development, and is not found in older texts or in the
written language.

\pex
\a
\begingl
    \gla {Zám} Janek{-te} Martina Markám hévoržébik.//
    \glb \Neg{} Janek=\Foc{} Martin-\Acc{} Marek-\Agt{} know-\Ben{}-\Pf{}//
    \glft \trsl{It was not Janek whom Marek introduced to Martin.}//
\endgl
\a
\begingl
    \gla Janek {zám} Martina{-te} Markám hévoržébik.//
    \glb Janek \Neg{} Martin-\Acc{}=\Foc{} Marek-\Agt{} know-\Ben{}-\Pf{}//
    \glft \trsl{It was not Martin whom Marek introduced Janek to.}//
\endgl
\a
\begingl
    \gla Janek Martina {zám} Markám{-te} hévoržébik.//
    \glb Janek Martin-\Acc{} \Neg{} Marek-\Agt{}=\Foc{} know-\Ben{}-\Pf{}//
    \glft \trsl{It was not Marek who introduced Janek to Martin.}//
\endgl
\xe

The different constituents of the sentence can be negated simultaneously; thus,
for example, the sentence below is grammatically permitted:

\pex
\begingl
    \gla {Zám} Janek {zám} Martina {zám} Markám {zá}hévoržébik.//
    \glb \Neg{} Janek \Neg{} Martin-\Acc{} \Neg{} Marek-\Agt{} \Neg{}-know-\Ben{}-\Pf{}//
    \glft \trsl{It was not Janek who was not introduced to someone who is not Martin by someone who is not Marek.}//
\endgl
\xe

Nonetheless, due to their general unwieldiness, forms like this are extremely
rare (both in the spoken and the written language), with preference given to
single and double negation instead. Since \ird{-te} can only appear in a
sentence once, where there are more than one negate constituent in a sentence,
\ird{-te} is appended to the element which has the most significance (usually
the topic); or, if there are two constituents negated and one of them is the
main verb, \ird{-te} is appended to that other element.

\pex
\begingl
    \gla {Zám} Janek{-te} Martina Markám {zá}hévoržébik.//
    \glb \Neg{} Janek=\Foc{} Martin-\Acc{} Marek-\Agt{} \Neg{}know-\Ben{}-\Pf{}//
    \glft \trsl{It was not Janek who was not introduced to Martin by Marek.}//
\endgl
\xe

Alternatively, if there is only one element/phrase negated in the sentence other
than the main verb (which itself may or may not be negated), it is common,
especially in colloquial Iridian\index{colloquial Iridian}, to
nominalize\index{nominalization} the whole verb phrase and transform the
sentence into a copular construction\index{copular construction}, with the
negated phrase as the new topic\index{topic} and the nominalized verb phrase as
the predicate\index{predicate}.

\pex
\begingl
    \gla Zám jájka na Praha zadačkou.//
    \glb \Neg{} daughter-\Dim{} \Loc{} Prague-\Acc{} move-\Av{}-\Pf{}-\Nz{}//
    \glft \trsl{It was not my daughter who moved to Prague.}//
\endgl
\xe




\section{Existential constructions}\index{existential construction}
\label{sec:exst}

\subsubsection{In general}
An existential sentence is a specialized construction used to express the
existence or presence of someone or something. The particle \ird{ješ} and its
inverse \ird{niho} are used to form existential sentences. 
\begin{multicols}{2}
\pex
\a\begingl
\gla Tak ješ zarno.//
\glb here \Exst{} people//
\glft \trsl{There are people here.}//
\endgl
\a\begingl
\gla Tak niho zarno.//
\glb here \N{}\Exst{} people//
\glft \trsl{There is no one here.}//
\endgl
\xe
\end{multicols}

The existential construction in Iridian was originally a
locative\index{locative} one, and this could still be seen in how the use of
\ird{ješ} and \ird{niho} requires both the noun or noun phrase whose existence
is posited and the location where such existence is said to be true to be
explicitly present in the sentence. In true existential sentences (e.g.,
\trsl{There is a God} or \trsl{There is still hope}) where the argument is the
existence of something and not just it's mere presence somewhere, the patientive
form of the reflexive verb \ird{se}, \ird{sní}, is used. In addition, where this
ostensible location is present in the sentence, it would occupy the
topic\index{topic} position\footnote{Although this location (often surfacing as
a \ird{na} clause) appears where the topic of the sentence normally would, it
would be more correct to analyze an existential construction as an inversion of
the regular topic-predicate word order in Iridian. Viewed this way, we can think
of \ird{ješ} or \ird{niho} as a pseudoverb, and the phrase consisting of the
first half of the sentence and ending with this pseudoverb is the predicate
while the unmarked second half is the topic. This approach has the benefit of
keeping the predicate with a verb-final internal word order and the topic as
unmarked, both in accordance with the basic rules of Iridian syntax; however,
this does not account for the use of the dummy \ird{sní} in true existential
clauses.} in the sentence, and unlike in regular sentences, must be explicitly
marked in the patientive.\index{patientive}


\begin{multicols}{2}
\pex
\a\begingl
\gla \ljudge{*}Ješ tieho.//
\glb \Exst{} god//
\glft \trsl{There is a God.}//
\endgl
\a\begingl
\gla Sní ješ tieho.//
\glb \Refl{}.\Acc{} \Exst{} god//
\glft \trsl{There is a God.}//
\endgl
\xe
\end{multicols}

The use of \ird{sní} as a placeholder is not required however if the noun or
noun phrase whose existence is the subject of the sentence is quantified, either
by a numeral or otherwise by an indefinite quantifier.

Statements expressing location use a copular construction, although an
existential construction may be used in the negative to convey an absence of
something, with the normal negative construction used where emphasis on one
element of the sentence is desired by the speaker.

\pex
\begingl
\gla Dá na duma.//
\glb \First{}\Sg{} \Loc{} house-\Acc{}//
\glft \trsl{I'm at home.}//
\endgl
\xe

\pex
\a\begingl
\gla Na duma niho dá.//
\glb \Loc{} house-\Acc{} \N{}\Exst{} \First{}\Sg{}//
\glft \trsl{I'm not at home.}//
\endgl
\a\begingl
\gla Zám dá na duma//
\glb \Neg{} \First{}\Sg{} \Loc{} house-\Acc{}//
\glft \trsl{It is not I who's at home.}//
\endgl
\a\begingl
\gla Dá zám na duma//
\glb \First{}\Sg{} \Neg{} \Loc{} house-\Acc{}//
\glft \trsl{I'm not at home (i.e., I'm somewhere else).}//
\endgl
\xe

The particles \ird{ješ} and \ird{niho} generally proceeds the noun or noun
phrase whose existence is being posited, but in the case of a modified noun or
noun phrase, the existential particle appears before all modifiers. On the other
hand, numerals or indefinite quantifiers appear before the existential particle.

\pex
\begingl
\gla Na duma men ješ mulaž.//
\glb \Loc{} house-\Acc{} two \Exst{} door//
\glft \trsl{There are two doors.}//
\endgl
\xe

\pex
\begingl
\gla Na ránema hroná ješ matematickí tóm.//
\glb \Loc{} desk-\Acc{} three \Exst{} mathematics book//
\glft \trsl{There are three mathematics books on my desk.}//
\endgl
\xe




\subsubsection{Possession}
Existential constructions are also used to indicate possession, with the
possessor marked in the patientive case.

\begin{multicols}{2}
\pex
  \begingl
    \gla Marka ješ oblašc.//
    \glb Marek-\Acc{} \Exst{} pet//
    \glft \trsl{Marek has a pet.}//
  \endgl
\xe
\pex
  \begingl
    \gla Tomáša niho mlaz.//
    \glb Tomáš-\Acc{} \N{}\Exst{} brother//
    \glft \trsl{Tomáš does not have a brother.}//
  \endgl
\xe
\end{multicols}

\subsubsection{Impersonal constructions}\index{impersonal construction}

Iridian prefers using existential constructions where English\index{English} and
other Indo-European languages would use indefinite pronouns. More formally,
sentences of this type are called impersonal constructions.\footnote{See, for
example, \textcite{lawtagalog} where the discussion in this section is largely
based.} In general an impersonal construction in Iridian is produced by
nominalizing\index{nominalization} a verb phrase which would otherwsise have
been the predicate of an indefinite pronoun. We can illustrate this in English
as follows:

\pex
\a  \deftagex{impeng}\deftaglabel{ind}Sentence with an indefinite pronoun as subject:\\
    \emph{Somebody} told me to come here to pick up the dress.
\a  Impersonal construction:\\
    \ljudge{?}\emph{There is somebody} who told me to come here to pick up the dress.
\xe

Sentences of the first type do not exist in Iridian. Instead sentences with an
indefinite element (not necessarily the subject of the sentence) are reframed as
existential constructions. To further illustrate the primacy of impersonal
constructions over indefinite pronouns in Iridian, we can replace the subject of
(\getfullref{impeng.ind}) with a definite noun:

\pex
\a\begingl
    \gla Tak muž nedvačernilá te Tereza ziček.//
    \glb here dress \Caus{}-get-\Pv{}-\Subj{}.\Ipf{} so:that Tereza say-\Av{}-\Pf{}//
    \glft \trsl{Tereza told me to come here to pick up the dress.}//
  \endgl
\a\begingl
    \gla Do ješ tak muž nedvačernilá te zičkou.//
    \glb \First{}\Sg{}.\Acc{} \Exst{} here dress \Caus{}-get-\Pv{}-\Subj{}.\Ipf{} so:that say-\Av{}-\Pf{}-\Nz{}//
    \glft \trsl{Somebody told me to come here to pick up the dress.} (\emph{Lit.,} I have someone who said (I) should come pick up the dress.)//
  \endgl
\xe


\pex
\begingl
\gla Martina ješ trešnikou na tropa.//
\glb Martin-\Acc{} \Exst{} write-\mk{pv-pf-nz} \Loc{} wall-\Acc{}//
\glft \trsl{Martin wrote something on the wall.}//
\endgl
\xe

\pex
\begingl
\gla Voštnikouva ža ješ piaščkou?//
\glb cook-\mk{pv-pf-nz-pat} already \Exst{} eat-\Av{}-\Pf{}-\Nz{}//
\glft \trsl{Did somebody eat what (I) cooked?}//
\endgl
\xe

\section{Copular constructions}
\subsubsection{Null copula}

Copular sentences are a minor sentence type where the predicate is not a verb.
For the purposes of this grammar, we narrow down our definition of copular
constructions to the following:
\pex
\a \textit{Equative:} Marek is the doctor (we are talking about).
\a \textit{Inclusive:} Marek is a doctor.
\a \textit{Attributive:} Marek is tall.
\a \textit{Locative:} Marek is in the hospital.
\xe

Iridian does not make a distinction between equative, inclusive and attributive
clauses. Locative clauses on the other hand, may be expressed using a copular or
an existential construction, as will be discussed in this section.

Iridian is a superficially a zero-copula language and the most common way to
form copular sentences is mere juxtaposition.

\pex<cop>
\begingl
\gla Marek doktor.//
\glb Marek doctor//
\glft \trsl{Marek (is a/the) doctor.}//
\endgl
\xe

The above example could either be taken to mean (1) Marek is a doctor
(inclusive), or (2) Marek is the doctor (equative). Generally, though, Iridian
uses word order to distinguish between equative and inclusive clauses.

\pex
\a \textit{Inclusive:} \{item in class\}\tss{N} $\varnothing$ \{class\}\tss{P}
\a \textit{Equative:} \{class\}\tss{N} $\varnothing$ \{item class\}\tss{P}
\xe

To avoid ambiguity, Example \getref{cop} can be reformulated to either of the
following sentences:

\pex<cop1>
\a
\begingl
\gla Marek doktor.//
\glb Marek doctor//
\glft \trsl{Marek is a doctor.}//
\endgl

\a
\begingl
\gla Doktor Marek.//
\glb doctor Marek//
\glft \trsl{Marek is the doctor.}//
\endgl

\xe

The inversion of word order is not strongly grammaticalized with NP-NP
sentences, i.e., both sentences in Example \getref{cop1} can still be used
interchangeably without a change in meaning and preference is given on the one
over the other when there is an ambiguity. This is not the case with attributive
clauses, i.e., sentences with adjective or adjective phrase predicates. Consider
for example the sentence below:

\pex
\begingl
\gla Marek rázym.//
\glb Marek tall//
\glft \trsl{Marek is tall.}//
\endgl
\xe

Inverting the word order of the sentence above would change the adjective to a
substantive since modifiers cannot occupy the topic position.

\pex
\begingl
\gla Rázym Marek.//
\glb tall Marek//
\glft \trsl{The tall one is Marek.}//
\endgl
\xe

Iridian also distinguishes between attributive clauses expressing permanent
conditions and clauses expressing temporary conditions, with the latter being
expressed using existential constructions in certain adjectives.

\pex
\begingl
\gla *Marek morec.//
\glb Marek hungry//
\glft \trsl{Marek is hungry}//
\endgl
\xe


\pex
\begingl
\gla Marka ješ morec.//
\glb Marek-\Acc{} \Exst{} hunger//
\glft \trsl{Marek is hungry}//
\endgl
\xe

A full list of adjectives/modifiers that use the existential construction can be
found in the section~\ref{sec:exst}.

The copula, however, cannot be ommitted in grammatical moods other than the
indicative.

\subsubsection{Negative copula}

Iridian has the negative copula \ird{česná}.

\pex
\begingl
\gla Marek doktor česná.//
\glb Marek doctor \Cop{}.\Neg{}//
\glft \trsl{Marek is not (a/the) doctor.}//
\endgl
\xe

The inversion of word order may also be used when one wants to avoid ambiguity:

\pex
\begingl
\gla Doktor Marek česná.//
\glb doctor Marek \Cop{}.\Neg{}//
\glft \trsl{Marek is not the doctor.}//
\endgl
\xe


\subsubsection{Conjugation paradigm}
