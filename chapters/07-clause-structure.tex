\chapter{Clause structure}

\section{Introduction}

The constituent word order of Iridian sentences is SOV, but the
agglutinativenature of the language and the presence of case-marking on nouns
makes word order typically flexible, with the only universal rule being that the
main verb should appear at the end of a sentence. This is not to say that word
order is completely free, however, as there are a number of restrictions on the
order of constituents in a sentence. In this chapter, we will discuss the basic
word order patterns of Iridian sentences, as well as the different types of
clauses that can be found in the language.

\section{Topic-predicate constructions}
\index{topic}\index{predicate}\label{sec:topic-pred}

The Iridian sentence can be divided primarily into a topic part and a predicate
or comment part. The topic is what the sentence is about, while the predicate or
comment represents the information presented in the sentence about the topic.
This is of course not very rigorous definitions of the terms `topic' and
`predicate', but they are sufficient for our purposes.

While both the topic and the predicate are pragmatic constructs, the
topic-predicate construction is important as it determines how the rest of the
sentence is structured. In general, the topic of the sentence is the constituent
that is most prominent in the sentence and consequently appears first. The topic
of the sentence is also the constituent that determines the case marking of the
main verb and the other constituents of the sentence. The predicate, on the
other hand, provides supplementary information about the topic of the sentence.

\begin{figure}[H]
  \begin{forest}
    [S,
      [{\sc top}] [{\sc pred}]]
  \end{forest}
  \caption{Nuclear structure of sentences}
  \label{fig:topic-pred}
\end{figure}

The topic, despite its prominence, can be considered non-essential, as most of
the information is encoded in the predicate. In fact, the primacy of the topic
and its influence on how the predicate is structured means that for the most
part, where the topic has already been established earlier, only the predicate
is needed to create a well-formed sentence. 

The topic of the sentence does not necessarily coincide with the subject of the
sentence. This is true as well in English, as we see in example
(\ref{ex:engtop}); although where English allows the topic to appear anywhere in
the sentence, as long as the subject is placed first, Iridian, typical of
topic-prominent languages, requires the topic to always be introduced first,
leaving the rest of the information afterwards. Where, however, other
topic-prominent languages like Japanese or Korean require the explicit marking
of the topic (e.g., by using a special topic marker like Japanese \trsl{wa} or
Korean \trsl{(n)eun}), Iridian does not; instead, Iridian relies solely on the
word order.

\pex\label{ex:engtop}
\a Martha saw John.
\a A dog bit \emph{Martha}.
\a It is raining \emph{today},
\xe

\pex
\a
\begingl
\gla \relax[Janek]\tss{\mk{Top}} [mlazka boulešik.]\tss{\mk{Pred}}//
\glft \trsl{As for Janek, he killed his brother}.//
\endgl

\a
\begingl
\gla \relax[Tereza]\tss{\mk{Top}} [jecám nalečnik.]\tss{\mk{Pred}}//
\glft \trsl{As for Tereza, she was bitten by a dog}//
\endgl

\a
\begingl
\gla \relax[Shléd]\tss{\mk{Top}} [zniepšime.]\tss{\mk{Pred}}//
\glft \trsl{As for today, it is raining.}//
\endgl
\xe

As has been demonstrated above, the topic function is not linked to a particular
grammatical function. Nevertheless, while it is theoretically possible to
promote any constituent noun phrase of a sentence to the topic position, in
practice, the topic is usually a noun phrase that is specific and referential.
In most cases, this means that the topic is a proper noun or definite noun
phrase. As \textcite{kiss2004} observes,
\begin{quote}
  We tend to describe events from a human perspective, as statements about their
  human participants – and subjects are more often {\sc[+human]} than objects
  are. In the case of verbs with a {\sc[–human]} subject and a {\sc[+human]}
  accusative or oblique complement,  the  most  common  permutation  is  that
  in  which  the  accusative  or oblique complement occupies the topic
  position\,[.] When the possessor is the only human involved in an action or
  state, the possessor is usually topicalized[.]
\end{quote}


\section{The noun phrase}

As mentioned in \S\,\ref{sec:topic-pred}, the topic of a sentence is usually a
noun phrase. In Iridian, the noun phrase is a syntactic category that is formed
by a head noun and its modifiers. Iridian is a strongly head-final language,
i.e., the head noun must always come last in the noun phrase.

\section{Topicless sentences}\label{sec:topicless}\index{topicless sentence}

As mentioned in \S\,\ref{sec:topic-pred}, the topic of a sentence is not always
necessary. In fact, in Iridian, it is possible to have a sentence without a
topic as long as the predicate is well-formed. 

We will distinguish between two types of topicless sentences: those that are
topicless by default (real), and those that are topicless by virtue of topic
ellipsis (apparent). In this section we will only discuss real topicless
sentences.



\section{Definiteness}\index{definiteness}\label{sec:definiteness}

Iridian lacks a specific class of articles\index{articles} such as English
\trsl{a} or \trsl{the} to mark the opposition between definite and indefinite
nouns. For example, the word \ird{jec} can mean both \trsl{a dog} or \trsl{the
dog} depending on the context (or in some environments the same word can be
interpreted as \trsl{dogs,} \trsl{some dogs} or \trsl{the dogs}).

A common way to specificy the definiteness of a noun is to promote it to the
topic position in the sentence. As discussed in \S\,\ref{sec:topic-pred}, the
topic of a sentence must be specific and referential, and therefore it is often,
but not always, definite. Consider for example the two sentences below.

\begin{multicols}{2}
  \pex
  \a
  \begingl
  \gla Pitár pižmo.//
  \glb Pitár farmer//
  \glft \trsl{Pitár is {a} farmer.}//
  \endgl
  \a
  \begingl
  \gla Pižmo Pitár.//
  \glb farmer Pitár//
  \glft \trsl{The farmer is Pitár.}//
  \endgl
  \xe
\end{multicols}

This can be extended to non-copular constructions.

\begin{multicols}{2}
  \pex
  \a
  \begingl
  \gla Vliče štanžek.//
  \glb milk-\Gen{} drink-\Av{}-\Pf{}//
  \glft \trsl{(I) drank some milk.}//
  \endgl
  \a
  \begingl
  \gla Vliko štaninek.//
  \glb milk drink-\Pv{}-\Pf{}//
  \glft \trsl{(I) drank the milk.}//
  \endgl
  \xe
\end{multicols}

If the topic is quantified\index{quantifier} by a numeral\index{numeral},
indefiniteness can be expressed by nominalizing\index{nominalization} the main
verb and promoting it to topic.

\pex
\a
\begingl
\gla Jaro okrád za propozica niebidček.// \glb five district for proposal-\Acc{}
vote:against-\Av{}-\Pf{}// \glft \trsl{The five districts voted against the
proposal.}//
\endgl
\a
\begingl
\gla Za propozica niebidečkou jaro okrád.// \glb for proposal-\Acc{}
vote:against-\Av{}-\Pf{}-\Nz{} five district// \glft \trsl{Five districts voted
against the proposal.}//
\endgl
\a
\begingl
\gla Za propozica niebidečkou ko okrád jaro.// \glb for proposal-\Acc{}
vote:against-\mk{av-pf-nz} \mk{rz} district five// \glft \trsl{Five is the
number of districts that voted against the proposal.}//
\endgl
\xe

The number one (\ird{oní})

\pex
\a
\begingl
\gla Tóm onaževí.//
\glb book be:lost-\mk{cont}//
\glft \trsl{The book is missing.}//
\endgl
\a
\begingl
\gla Oní tóm onaževí.//
\glb one book be:lost-\mk{cont}//
\glft \trsl{One of the books is missing.}//
\endgl
\a
\begingl
\gla Onaživou pní tóm.//
\glb one book be:lost-\mk{cont}//
\glft \trsl{One of the books is missing.}//
\endgl
\a
\begingl
\gla Onaživou pní tóm.//
\glb one book be:lost-\mk{cont}//
\glft \trsl{One of the books is missing.}//
\endgl
\xe


Note that this rule is not universal and the topic of a sentence does not necessarily have to be definite, especially where the sentence is merely expressing a fact or a general truth:

\pex
\begingl
\gla Jec hvárem.//
\glb dog animal//
\glft \trsl{Dogs are animals.}//
\endgl
\xe



\pex
\begingl
\gla To >>jec<< hvárem že: to robot//
\glb \Dem{} dog animal \mk{ncop} \Dem{} robot//
\glft \trsl{The ``dog'' is not a real animal but a robot.}//
\endgl
\xe


\section{Coordination} \index{coordination}

Iridian has three groups of coordinating conjunctions: the additive
\irdp{a}{and} and \irdp{še}{with}; the contrastive \ird{má} and \ird{ozná} (both
translated to \printlang{en}\index{English} as \trsl{but}); and the
disjunctive/correlative \ird{je}, \ird{le} and \ird{ni}.

\ird{A} corresponds to the English \trsl{and.} When coordinating simple noun
pairs, however, \irdp{še} is more often used though. The derived construction
\ird{a še} is also common and has a similar meaning to the English \trsl{and
also}.

\pex
\begingl
    \gla Mámka {še} pápku na Prahe spaníček.//
    \glb mother-\Dim{} \Com{} father-\Dim{}-\Ins{} \Loc{} Prague-\Acc{} vacation-\Av{}-\Pf{}//
    \glft \trsl{Mom and Dad went to Prague for vacation.}//
\endgl
\xe
\pex
\begingl
    \gla Janek {a} {še} Marku kurs hlupinžice.//
    \glb Janek and \Com{} Marek-\Ins{} class fail-\Av{}-\Pf{}-\Quot{}//
    \glft \trsl{Janek as well as Marek failed the class.}//
\endgl
\xe

In constructions with \ird{še} where one of the nouns coordinated is a pronoun
or a deictic\index{deictic}, the pronoun or deictic is presented first followed
by the other noun in the instrumental case\index{isnstrumental case}.

\pex
\begingl
    \gla Dá {še} Ivanu sohladoušce.// \glb \mk{1s.str} \Com{} Ivan-\Ins{}
    classmate// \glft \trsl{Ivan and I are classmates.}//
\endgl
\xe

In a few cases, \ird{a} is used instead of \ird{še} where the latter can be
interpreted as having an attributive meaning. Where the noun is marked, however,
only \ird{a} can be used.

\begin{multicols}{2}
\pex\a
\begingl
    \gla trava {še} lépu//
    \glb bread \Com{} cheese-\Ins{}//
    \glft \trsl{bread with cheese} i.e., \trsl{cheese sandwich}//
\endgl
\a
\begingl
    \gla trava {a} lép//
    \glb bread and cheese//
    \glft \trsl{bread and cheese}//
\endgl
\xe\end{multicols}

\pex
\begingl
    \gla To kurs-te Jankám {a} Markám hlupienince.//
    \glb this class-\Foc{} Janek-\Agt{} and Marek-\Agt{} class fail-\Pv{}-\Pf{}-\Quot{}//
    \glft \trsl{It was this class that Marek and Janek failed.}//
\endgl
\xe


The bisyndetic coordination (\cite{velupillai2012}) \ird{a} Y \ird{a} Y is also
with similar emphatic meaning as \ird{a še}.

\pex
\begingl
    \gla {a} plocem {a} ploceš.//
    \glb and family-\First{}\Sg{} and family-\mk{2s}//
    \glft \trsl{both my family and yours}//
\endgl
\xe

\pex
\begingl
    \gla {a} hastu {a} še zmenu zověc hloubižách.//
    \glb and suffering-\mk{} and \Com{} happiness-\Ins{} remain-\Cv{} love-\mk{av-ctpv}//
    \glft \trsl{Til death do us part.} \emph{Lit.,} \trsl{I will love you through both suffering and joy.}//
\endgl
\xe

With multiple nouns or noun phrases, especially in serial lists, the
coordinating conjunction is often simply dropped.

\pex
\begingl
    \gla Ivan, Jarek, Elena na meza.//
    \glb Ivan Jarek Elena \Loc{} room-\Acc{}//
    \glft \trsl{Ivan, Jarek, and Elena are in the room.}//
\endgl
\xe

\pex
\begingl
    \gla Morkve, hlepost, ruk, molec hladniževí.//
    \glb carrot asparagus broccoli cabbage to:displease-\Av{}-\Cont{}//
    \glft \trsl{I don't like carrots, asparagus, broccoli or cabbage.}//
\endgl
\xe

\ird{A} or \ird{še} however is required when two adjectives are used to modify a
noun, with \ird{še} used when the two adjectives describe the same noun and
\ird{a} (or often \ird{a še}) when describing two distinct
objects.\footnote{When used this way, the noun preceding \ird{še} or \ird{a še}
is not declined in the instrumental case.}

\pex
\a
\begingl
    \gla Sodoví {še} ludí kobera tahatnik.//
    \glb black with white shirt bring-\Pv{}-\Pf{}//
    \glft \trsl{I brought the black-and-white shirt.}//
\endgl
\a
\begingl
    \gla Sodoví {a} {(še)} ludí kobera tahatnik.//
    \glb black and with white shirt bring-\Pv{}-\Pf{}//
    \glft \trsl{I brought the black shirt as well as the white one.}//
\endgl
\xe

Other common uses of \ird{a} and \ird{še} are described in detail in section
\S\,\ref{sec:conn-conj}

The particle \irdp{nebí}{also} may take a conjunctive meaning when attached to
multiple elements in a sentence, similar to \irdp{a\ldots{}
a\ldots}{both\ldots{} and\ldots} but more emphatic.

\pex
\begingl
    \gla Lukáš nebí Marek nebí naž//
    \glb Lukáš also Marek also friend//
    \glft \trsl{Lukaš and Marek are also my friends.}//
\endgl
\xe

\ird{Má} and \ird{ozná} are used to express contrast, like the English
\trsl{but}. \ird{Ozná} however is more restrictive, and can only be used if the
first clause is in the negative and the second clause directly contradicts (or
provides an alternative to) the first. The clause introduced by \ird{ozná} must
directly correspond to the element in the first clause being negated. Where the
initial element is inflected, such inflection must also be reflected on the
alternative presented in the \ird{ozná} clause.\footnote{The syntax of the main
clause does not necessarily correspond to how the sentence would have otherwise
been constructed in isolation. For instance, the neutral syntax for example
(\getref{ozna}) without the \ird{ozná} would be: \ird{Bi\k{e}c záčesčeví.}}

\pex
\begingl
\gla Stožek má na duma niho čast.//
\glb go-\Av{}-\Pf{} but \Loc{} house-\Acc{} \mk{nexst} person//
\glft \trsl{I went but no one was home.}//
\endgl
\xe


\pex[tag=ozna]
\begingl
\gla Zám bi\k{e}c česčeví ozná jec.//
\glb \Neg{} cat to:please-\Av{}-\Cont{} but dog//
\glft \trsl{(I) don't like cats but I do like dogs.}//
\endgl
\xe

\ird{Ozná} does not allow a negative\index{negation} argument. If the main
clause is positive and the secondary clause is negative, \ird{má} is used
instead.

\pex
\begingl
\gla To jako odpizounilá to hrebe cešceví, má zám jáne.//
\glb \Dem{}.\Prox{} tree to:grow-\Loc{}-\Subj{}.\Ipf{} \Rz{} mushroom-\Acc{} to:please-\Av{}-\Cont{} but \Neg{} \Dem{}.\Med{}//
\glft \trsl{Mushrooms love to grow under this tree, but not under that one.}//
\endgl
\xe

\ird{Má} or its variant \ird{a má} (literally \trsl{and but}) is also used to
introduce exclamatory sentences. This usage is purely idiomatic and does not
require for there to be an actual contrastive meaning in the sentences.

\pex
\begingl
\gla A má duma nahte ašteví!// \glb and but house too:much be:pretty-\Cont{}//
\glft \trsl{Your house is very beautiful!}//
\endgl
\xe

Finally, the disjunctive conjunctions\index{disjunctive conjunction} \ird{je},
\ird{li}, and \ird{ni} are used to join phrases or sentences that are seen as
alternatives to each other. \irdp{Je}{or} may be used to separate the
alternatives proposed, or reduplicated, preceding each of the components of the
sentence (i.e., \irdp{je X je Y}{either X or Y}); this latter use often means
that the options being presented are the only ones available.
\ird{Ni}\footnote{\ird{Ni} is an Indo-European, possibly Slavic,
borrowing.\index{linguistic borrowing}} is the inverse of \ird{je} and must
always be used in pairs (\irdp{ni X ni Y}{neither X nor Y}) as when used alone
it functions as an adverb (similar to English \trsl{not even} or \trsl{at all}).
An obvious exception, however, would be in a conversation, when a speaker would
provide a negative alternative response to an already negative statement (see
example (\getfullref{ni.resp}) below).

\pex
\begingl
\gla Ni ircevní ni ruščevní malnovím zahviržéteví.//
\glb nor Iridian-\Att{} nor Russian-\Att{} tongue-\Ins{} speak-\Av{}-\Pot{}-\Cont{}//
\glft \trsl{I can't speak neither Iridian nor Russian.}//
\endgl
\xe

\pex\a\begingl
\gla Dá ircevní malnovím ni zazahviržéteví.//
\glb \First{}\Sg{}\Str{} Iridian-\Att{} tongue-\Ins{} not:even speak-\Av{}-\Pot{}-\Cont{}//
\glft \trsl{I can't speak any Iridian at all.}//
\endgl
\a\begingl
\gla Ni ircevní ni ruščevní malnovím zahviržéteví.//
\glb nor Iridian-\Att{} nor Russian-\Att{} tongue-\Ins{} speak-\Av{}-\Pot{}-\Cont{}//
\glft \trsl{I can't speak neither Iridian nor Russian.}//
\endgl
\a\vtop{\halign{%
#\hfil& \qquad #\hfil\cr
\ird{---\,Dá ruščevní malnovím zahviržéteví.} & \trsl{I don't speak Russian.}\cr
\ird{---\,Ni dá.} & \trsl{Neither do I.}\cr
}}\deftagex{ni}\deftaglabel{resp}
\xe

\ird{Le} (another possible Slavic\index{Slavic} borrowing\index{linguistic
borrowing}, adopted from Common Slavic \emph{li} or \emph{ili}) has a more
emphatic and contrastive meaning than \ird{je}. It is used when the speaker
thinks that the option being presented is counterfactual or doubtful. Unlike
\ird{je} or \ird{ni}, \ird{le} is added to the end of the word or phrase.
\ird{Le} is most often used in parenthetical statements or in responses; it
cannot be used by itself when both alternatives are present and must be
introduced instead by either \ird{je} or \ird{a}.

\pex\a\begingl
\gla Marek-le ruščevní malnovím zahviržéteví.//
\glb Marek=or Russian-\Att{} tongue-\Ins{} speak-\Av{}-\Pot{}-\Cont{}//
\glft \trsl{Or maybe Marek can speak Iridian.}//
\endgl
\a
\begingl
\gla Já Karlu je Terezu-le de Rume sostožit.//
\glb \Second{}\Sg{}.\Str{} Karel-\Ins{} or Tereza-\Ins{}=or \Lat{} Rome-\Acc{} \Rec{}-go-\Av{}-\Sup{}  //
\glft \trsl{Karel\,---\,or maybe even Tereza\,---\,can come with you to Rome.}//
\endgl\xe

\section{Apposition}\index{apposition}\label{sec:apposition}

Appositive constructions in Iridian involve the juxtaposition of two or more
noun phrases that have a single referent. An apposition can be non-restrictive
if the appositive can be removed freely without changing the meaning of a
sentence, or restrictive otherwise.

Formally both non-restrictive and restrictive appositives are treated as
modifier phrases but only the latter is grammaticalized. The restrictive
appositive must always precede the noun phrase it modifies, linked together by
the particle \ird{ko}. Non-restrictive appositives on the other hand are simply
juxtaposed together, although a comma is often inserted around the appositive if
it consists of more than one word.

\pex\a
\begingl\deftagex{appos}\deftaglabel{1}
    \gla \'Oto mlazka na Mnihe znohouščeví.//
    \glb  \'Oto brother-\Dim{} \Loc{} Munich-\Acc{} study-\Av{}-\Cont{}//
    \glft \trsl{My brother Otto is studying in Munich.}//
\endgl
\a\begingl\deftagex{appos}\deftaglabel{res}
    \gla \'Oto {ko} mlazka na Mnihe znohouščeví.//
    \glb  \'Oto \Lnk{} brother-\Dim{} \Loc{} Munich-\Acc{} study-\Av{}-\Cont{}//
    \glft \trsl{My brother Otto is studying in Munich.}//
\endgl
\xe

Examples(\getfullref{appos.1}) and (\getfullref{appos.res}) shows two different
translations of the English phrase \trsl{My brother \'Oto is studying in
Munich.} Example (\getfullref{appos.1}) is non-restrictive and can be
interpreted as \trsl{I have a brother namsed \'Oto who is studying in Munich}
while (\getfullref{appos.res}) being restrictive can be translated more on the
lines of \trsl{Among my brothers, it is \'Oto who is studying in Munich.} The
restrictive appositive implies specificity and by extension the existence of a
group where this specificity holds true; in (\getfullref{appos.res}) this is
taken to mean that a set of brothers exists and \'Oto is a member of this set.


\section{Syntax of event and participant nominals}\index{nominalization!event
nominal}\label{sec:nomz-syntax}

\subsection{Gerunds and event nominals}

As we have established in \S~\ref{nom-morph}, Iridian has three forms of
nominalization\index{nominalization}: (1)~the mainly non-productive usage of the
nominalising \ird{-ou} with the verbal stem to form resultant nominals; (2)~the
use of \ird{-ou} in conjunction with the gerund-forming prefix\index{gerund}
\ird{po(d)-} to form a verbal noun (which we call an event nominal or gerund)
and which may either include the internal arguments of the parent verb or not;
and (3)~the formation of a participant nominal (cf.~\cite{okuna}) which
nominalises not the event described by the verb but its participants.

Since gerunds\index{gerund} represent the nominalization of the
event\index{event nominal} described by the verb, they are therefore inherently
abstract and active in meaning. Since the nominalised forms are abstract, it
follows that they are also tenseless and aspectless. Iridian gerunds, however,
may be optionally marked for their lexical aspect or
\foreign{aktionsart}\index{aktionsart@\emph{aktionsart}}\index{lexical
aspect|see{\emph{aktionsart}}} using the continuous aspect suffix \ird{-eví}
(which subsequently becomes \ird{-ěv-} through sound change). It is important to
note though that although a marker for grammatical aspect\index{aspect} is used,
what is being marked is lexical and not grammatical aspect; specifically, the
addition of \ird{-ěv-} only signifies that the action is iterative in nature and
thus the gerund itself remains tenseless\index{tense} and aspectless.

\pex
    \a \ird{nidá}\,$\rightarrow$\,\ird{ponidou}\\
        \trsl{(my) standing up}
    \a \ird{nidá}\,$\rightarrow$\,\ird{poniděvou}\\
        \trsl{(my) standing up repeatedly}
\xe

In {\sc cen}s, both the agent and the patient are marked in the
genitive.\index{genitive}\footnote{\textcite{serekaite2020} argues that although
(in the case of Lithuanian, at least) the actor and the theme from the original
sentence both become marked in the genitive in the resulting complex event
nominal, the superficially indentical genitives are actually two distinct cases:
a higher genitive ({\sc gen.h}) assigned to agents and possessors and a lower
genitive ({\sc gen.l}) assigned to grammatical objects. Although this argument
is interesting and probably holds true as well in Iridian {\sc cen}s, we will
not make an effort to ascertain whether there is an actual difference in the two
genitive cases in Iridian as this is not needed for the purpose of this
grammar.} If both are present, the agent must always appear first. This
construction is quite common cross-linguistically, as we see in the examples
below.

\pex
\a\begingl
    \gla Mlazcě praví na Mnihe poznohouštou na zahrana nemniček.//
    \glb brother-\Dim{}-\Gen{} law-\Gen{} \Loc{} Munich-\Acc{} \Ger{}-study-\Nz{} \Loc{} beginning-\Acc{} surprise-\Av{}-\Pf{}//
    \glft \trsl{My brother's studying law (i.e., my brother's decision to study law) in Munich surprised us at first.}//
\endgl
\a Lithuanian\index{Lithuanian} (\cite[1]{serekaite2020})\\
\begingl
    \gla Jono augal\k{u} sunaikinimas.//
    \glb Jonas-\Gen{} plants-\Gen{} \Pfv{}-destroy-\Caus{}-\Nz-\Nom{}.\M{}.\Sg{}//
    \glft \trsl{Jonas' destruction of plants}//\deftagex{doubgen}\deftaglabel{lithuanian}
\endgl
\a Tagalog\index{Tagalog} (\cite[22]{hsieh2019})\\
\begingl
    \gla (Ang) Pagluluto ni Harvey (ng manok) ang nangyari.//
    \glb \Nom{} \Ger{}$\sim$cook \Gen{} Harvey \Gen{} chicken \Nom{} happen.\Pfv{}//
    \glft \trsl{What happened was Harvey's cooking (of chicken).}//\deftagex{doubgen}\deftaglabel{tagalog}
\endgl
\xe

The use of the genitive\index{genitive} to mark both the actor and the theme in
the original sentence is of course a recipe for ambiguity. When only one of
either the actor or the theme is present in the {\sc cen}, the ambiguity is on
whether the noun marked represents the one or the other, as, e.g., the phrase
\ird{Jancí podohletou} which can be interpreted to mean either \trsl{the act of
remembering Janek} or \trsl{Janek's act of remembering} without any further
information. A second ambiguity arises when both the actor and the theme are in
the sentence as it is unclear, without any context, the genitive is actually
being used to mark their thematic role in the originally or is in fact a
possessive. The same is true in, for example, Lithuanian\index{Lithuanian} where
as \textcite{serekaite2020} points out, sentence
(\getfullref{doubgen.lithuanian}) can also be alternatively translated as
\trsl{[the] destruction of Jonas's plants}.

The first type of ambiguity is resolved in English\index{English} by using word
order: in general, a prepositive genitive (i.e., using the clitic \foreign{'s}
or the possessive form of a pronoun) is used when the noun in the genitive case
in the {\sc cen} represents the actor (e.g., \trsl{John's remembering}) while a
postpositive genitive is used when the noun in the genitive represents the theme
(e.g., \trsl{the remembering of John}). This in turn, can be extended to the
second type, e.g., \trsl{John's remembering of Margaret}. However, the
obligatorily head-final nature of Iridian syntax means that such strategy is not
possible. Instead, the strategy used in Iridian is more similar to the one found
in Tagalog\index{Tagalog} where the theme may be marked using the oblique
\foreign{sa}\footnote{This becomes \foreign{kay} before proper nouns.} instead
of the genitive \foreign{ng}.\footnote{ To call \foreign{ng} (pronounced [nɐŋ])
as a genitive marker is simplistic (even erroneous) but should be enough for the
purpose of our discussion. } Thus we can restate (\getfullref{doubgen.tagalog})
as follows:

\pex{Tagalog\index{Tagalog} (modified from \cite[22]{hsieh2019})}\\
\begingl
    \gla (Ang) Pagluluto ni Harvey {sa} manok ang nangyari.//
    \glb \Nom{} \Ger{}$\sim$cook \Gen{} Harvey \Obl{} chicken \Nom{} happen.\Pfv{}//
    \glft \trsl{What happened was Harvey's cooking of \emph{the} chicken.}//
\endgl
\xe

An immediate consequence of replacing the genitive \foreign{ng} with the oblique
marker \foreign{sa/kay} is that the theme is now interpreted as definite
(cf.~\cite[3,\,40]{kaufman2009}). The use of the oblique to mark the theme can
be used even when only one element is present in the event nominal; in fact,
when the theme is known as definite for a fact (e.g., if it is a person), the
choice between the oblique and the genitive is what distinguishes the actor and
the theme. Thus we have

\pex[interpartskip=0pt]
    \a Choice between \Obl{} and \Gen{} distinguishing actor from theme
    \beginsubsub\index{Tagalog}
        \b{--}{\foreign{pagtawag kay {\nf{[\Obl{}]}} Harvey}\\ \trsl{the act of calling Harvey}}
        \b{--}{\foreign{pagtawag ni {\nf{[\Gen{}]}} Harvey}\\ \trsl{Harvey's act of calling}}
    \endsubsub
    \a Resolving ambiguity by obligatory replacement of \Gen{} by \Obl{} in the theme:
    \beginsubsub
        \b{--}{\foreign{pagtawag ni {\nf{[\Gen{}]}} Harvey sa {\nf{[\Obl{}]}} kasama}\\
        \trsl{Harvey's act of calling his \mbox{colleague}}}
        \b{--}{\foreign{pagtawag ni {\nf{[\Gen{}]}} Harvey ng {\nf{[\Gen{}]}} kasama}\\
        \trsl{Harvey's act of calling a colleague}}
    \endsubsub
    \a New ambiguity introduced by changing the word order:
    \beginsubsub
        \b{--} {\foreign{pagtawag ng {\nf{[\Gen{}]}} kasama ni {\nf{[\Gen{}]}} Harvey}\\
        \trsl{Harvey's act of calling a colleague} or \trsl{The act of calling Harvey's colleague}}
    \endsubsub
    \a Ungrammatical form, with both the theme and actor marked in the oblique:
    \beginsubsub
        \b{--}{\ljudge{*}\foreign{pagtawag kay {\nf{[\Obl{}]}} Harvey sa {\nf{[\Obl{}]}} kasama,}\\
        \trsl{Harvey's act of calling a colleague}}
    \endsubsub
    \a Double genitive, with both indefinite actor and theme:
    \beginsubsub
        \b{--}{\foreign{pagtawag ng {\nf{[\Gen{}]}} tao ng {\nf{[\Gen{}]}} kasama,}\\
        \trsl{a person's act of calling a colleague} or \trsl{a colleague's act of calling of a person}} 
    \endsubsub
\xe


In Iridian, the a \ird{na} clause corresponds to the Tagalog\index{Tagalog} use
of the oblique to indicate a definite theme in a {\sc cen}. 

% nemnetá from CS m{\yer}neti to think + ne not



\subsection{Participant nominals}

Participant nominals are formed by nominalising a finite verb phrase with the
suffix \ird{-ou}. The resulting noun refers back to a participant in the event
rather than the event itself, with the role determined by the grammatical voice
in which the original verb phrase is marked. Consequently, participant nominals
are inherently definite in meaning.

\pex
\a\begingl
    \gla Jancí materška najevěc shradnaní.//
    \glb Janek-\Gen{} stepmother drive-\Cv{} die-\Pv{}-\Ret{}//
    \glft \trsl{Janek's stepmother was killed in a car crash.}//
\endgl
\a\begingl
    \gla Jancí materšcí najevěc shradněnou policám zánehévorneví.//
    \glb Janek-\Gen{} stepmother-\Gen{} drive-\Cv{} die-\Pv{}-\Ret{}-\Nz{} police-\Agt{} \Neg{}-\Caus{}-know-\Pv{}-\Cont{}//
    \glft \trsl{The police still hasn't identified the person Janek's stepmother has killed in the crash.}//
\endgl
\xe

The creation of participant nominals is a very common strategy in Iridian.
Participant nominalization is also used to shift the focus of the sentence from
the event to the participant. For example, transforming the sentence \irdp{Janek
shražek}{Janek died} into \irdp{Janek shražkou}{It is Janek who died} or more
emphatically, \irdp{Shražkou Janek}{It is Janek who died} changes the emphasis
in the sentence.

\section{Subordinate clauses in general}

A subordinate clause is a clause that is dependent on another clause. In Iridian, 

\section{Clause-linking strategies}
\subsection{Clause-linking with \ird{še}}

\subsection{Temporal succession and causality}

Iridian has three main conjunctions used in linking clauses to indicate
causality and temporal sequency: \ird{vele}, \ird{dito}, and
\ird{děla}.\footnote{A fourth one exists, \ird{vělne}, in fact exists, but it is
essentially the same as \ird{vele}, but there is no real difference between it
and vele, and one can use the one or the other without changing the meaning of
the sentence. Nevertheless, one would find vele as the more common variant.} One
might think that the three conjunctions would correspond neatly with the three
levels in which we could interpret the causation, as we have discussed above,
but that is not the case. Indeed, as in any other language, there exists a
significant overlap in their usage.

\ird{Vele/vělne} and \ird{dito} are used in a propositional level causation.
They are often interchangeable but, in case of ellipsis (i.e., the omission of
either parts of the causational pair), both \ird{vele/vělne} and \ird{dito} may
only appear with the protatic clause (the `cause' in the cause-effect pair),
although \ird{dito} must be fronted first, appearing immediately before the
verb, which movement is only optional for \ird{vele/vělne.} Although both
clause-initial and clause-final vele/velne have the same meaning, the latter
would often be characterized as informal.


\pex
\a\begingl
  \gla Zabola ce zákupinenik vele byl kravnašime.//
  \glb ice:cream \Dem{}-\Acc{} \Neg{}-buy-\Pv{}-\Pf{} because child cry-\Av{}-\Prog{}//
  \glft \trsl{The child is crying because (they) did not buy him ice cream.}//
\endgl
\a\begingl
  \gla Zabola ce zákupinenik dito byl kravnašime.//
  \glb ice:cream \Dem{}-\Acc{} \Neg{}-buy-\Pv{}-\Pf{} because child cry-\Av{}-\Prog{}//
  \glft \trsl{The child is crying because (they) did not buy him ice cream.}//
\endgl
\a \ird{Zabola ce zákupinenik vele.}
\a \ljudge{*}\ird{Zabola ce zákupinenik dito.}
\a \ird{Zabola ce vele zákupinenik.} 
\a \ird{Zabola ce dito zákupinenik.}
\xe 

Another example illustrating the non-interchangeability of vele/vělne and dito is the unembeddability of vele/vělne unde semantic operators such as negation.

\pex
\a\begingl
  \gla Mamka těhto zám dito záščenžek.//
  \glb mother-\Dim{} sick \Neg{} because \Neg{}-arrive-\Av{}-\Pf{}//
\endgl
\a \ljudge{*}\ird{Mamka těhto zam vele záščenžek.}
\xe

\ird{Děla}, on the other hand, is often used in framing epistemic level
causality. In addition, \ird{děla}, unlike \ird{vele/vělne} and \ird{dito},
govern the apodotic clause (the `effect' in the cause-effect pair) and should
thus be translated more correctly as \trsl{thus/therefore}. Consequently,
\ird{děla} can appear in the same sentence with \ird{vele/vělne}, but not with
\ird{dito}; this latter usage has a greater explanatory force than the simple
use of either \ird{děla} or \ird{vele/vělne}.

\pex
\begingl
  \gla Pozbéšílá děla doja suměneví.//
  \glb rain-\Av{}-\SubjI{} therefore street be:wet-\Cont{}//
  \glft \trsl{It is must be raining since the streets are wet.}//
\endgl
\xe

\pex
\begingl
  \gla Ame ža hrupkašek děla na mlane spro.//
  \glb sun already set-\Av{}-\Pf{} therefore \Loc{} exterior-\Acc{} darkness//
  \glft \trsl{The sun has set so it must be dark outside.}//
\endgl
\xe

\pex
\begingl
  \gla Já Marka naž has vele děla na večera záprezitnik.//
  \glb \Second{}\Sg{}.\Str{} Marek-\Acc{} friend \Cop.\Neg{} because therefore \Loc{} party-\Acc{} \Neg{}-invite-\Pv{}-\Pf{}//
  \glft \trsl{It is because you are not Marek's friend that you were not invited.}//
\endgl
\xe




\section{Converbial constructions}\label{converbs-syntax}\index{converb}

\subsection{In general}

In \S~\ref{sec:converb}, we have defined a converb as a non-finite verb form
that is often used adverbially. In this section, we will discuss the syntax of
converbial constructions in Iridian.

The most common type of converbial constructions involves the main verb preceded
by the imperfective converbial form of a secondary verb. The secondary verb
normally specifies the manner or the means by which the action described by the
main verb is performed. Adverbial constructions such as these tend to be used
even where English, for example, would use a single verb. For example, in the
sentence \trsl{He cut the branch} would be analyzed in Iridian as \trsl{He
removed the branch by cutting} as Iridian would interpret the verb `cut' as used
in the first sentence as encoding both the action performed and the manner in
which it was performed. Although the second sentence below is not necessarily
incorrect, it would sound unnatural in Iridian.

\pex
\a\begingl
  \gla Platek odněc rutnik.//
  \glb leaf cut-\Cv{}.\Ipf{} remove-\Pv{}-\Pf{}//
  \glft \trsl{(He) removed the leaf by cutting it.}//
\endgl
\a\begingl
  \gla\ljudge{?}Platek odnenik.//
  \glb leaf cut-\Pv{}-\Pf{}//
  \glft \trsl{(He) cut the leaf.}//
\endgl
\xe

\subsection{Temporal constructions}

A converbial construction is often used in temporal clauses\index{temporal
clause}, with the imperfective converbial form used when the action is
unfinished or continuing and the perfective otherwise. When used in a temporal
clause, the converb may sometimes be separated from the main clause by the
particle \ird{si}.\footnote{\ird{Si} is virtually never used in the spoken
language.}

\pex
\begingl
\gla Otvěc (si) na Varšave možlašaní.//
\glb be:young-\Cv{}.\Ipf{} when \Loc{} Warsaw-\Acc{} understand-\Av{}-\Ret{}//
\glft \trsl{When I was young, we used to live in Warsaw.}//
\endgl
\xe

\subsection{Causal clauses}

Clauses expressing reason are usually expressed by a converbial construction.
The antecedent and the main clause may be connected with \irdp{am}{because,}
although this is often dropped in casual speech.

\pex
\begingl
\gla Za prove záznohouštu Martin meštnašek.//
\glb for exam-\Acc{} \Neg{}-study-\Cv{}.\Pf{} Martin fail-\Av{}-\Pf{}//
\glft \trsl{Martin failed the exam because he didn't study.}//
\endgl
\xe


\pex
\begingl
\gla Kinoteka stožílá to všihněc mámka zachovažek.//
\glb cinema-\Acc{} go-\Av{}-\Sbj{}.\Ipf{} \Rz{} be:angry-\Cv{}.\Ipf{} mother-\Dim{} allow-\Av{}-\Pf{}//
\glft \trsl{Since she was still mad at us, Mum did not let us go to the movies.}//
\endgl
\xe


\subsection{Similarities with the Czech and Slovak transgressive}

Converbs in Iridian have parallel usage as the
transgressive\index{transgressive} conjugations in \printlang{cs}\index{Czech}
and Slovak\index{Slovak}. It is the consensus among scholars of the languages,
though, that the converbial forms in Iridian and the transgressive forms in
Czech and Slovak, developed independently of each other; although to what extent
one influenced the other is still the subject of debate. The converbial forms in
Iridian have more varied uses than the transgressives in Czech (Slovak having
kept only the present transgressive form), and whereas the latter forms have
largely fallen in disuse (relegated to the literary register) in both Czech and
Slovak, converbial forms are still widely used in Iridian.

Although Czech grammarians use the terms `past' and `present' to distinguish
between the two forms used in the language, the distinction is actually one of
aspect\index{aspect}, as in Iridian. In general, the past transgressive form
corresponds with the perfect converbial form, and may be used to indicate a
foregoing action; the present transgressive, on the other hand, corresponds to
the imperfect converb and is used to indicate a coincident/contemporaneous
action.

This correspondence is not complete, however. For example, consider this
sentence in Czech\index{Czech}: \foreign{Děti, {vidouce} babičku, vyběhly
ven}{The children, seeing their grandmother, ran outside.} The verb in the
transgressive clause is in the present tense in this case, while in Iridian, the
same sentence will be translated with the perfective as follows:

\pex
\begingl
\gla \v{S}ášlika vedu byl naladěc mnilžek.//
\glb grandmother-\Dim{}-\Acc{} see-\Cv{}.\Pf{} children run-\Cv{}.\Ipf{} go:out-\Av{}-\Pf{}//
\glft \trsl{The children, having seen their grandmother, ran outside.}//
\endgl
\xe

The Czech\index{Czech} sentence above can alternatively be translated using the
imperfective converbial form, but this would put a stronger emphasis on the two
actions happening at the same time and so the original construction can be
considered as the more idiomatic one.

\subsection{In fixed expressions}

The past converbial form is used in expressing gratitude, approbation or
condolencess, or in asking for forgiveness. This usage is idiomatic and the
actions do not necessarily need to have been completed. The main clause is often
in the hortative mood\index{hortative mood} and separated from the converb
clause with \irdp{am}{because.} Moreover, this usage, unlike most converbial
constructions, allow the verb of the converb clause to have a different subject
as long as such subject is marked explicitly in the agentive case. However,
since the converbial form of verbs are invariable, if the subordinate clause
requires further complexity when it comes to the verb in the converb clause, a
dependent \ird{še} clause may be use instead of a converb.

\pex
\a Expressing gratitude:\\
\begingl
\gla Stranu am luhninká.//
\glb help-\Cv{}.\Pf{} because thank-\mk{pv-hort}//
\glft `Thank you for helping.'//
\endgl
\a Asking for forgiveness:\\
\begingl
\gla Lěnu záščenu am rozvedniká.//
\glb on:time-\Ins{} \Neg{}-arrive-\Cv{}.\Pf{} because forgive-\mk{pv-hort}//
\glft `Sorry for being late.'//
\endgl
\a Expressing condolences:\footnote{Compare this example to the following, where
a converb clause cannot be used:

\ex[lingstyle=fnex,belowexskip=-1em]
\begingl
\gla Pápka na puvode shradniš to množniká.//
\glb father \Loc{} war-\Acc{} die-\Pv{}-\Subj.\Pf{} \Rz{} with console-\mk{pv-hort}//
\glft `I'm sorry to hear your father died (\emph{lit.,} was killed) in the war.'//
\endgl\xe}\\
\begingl
\gla Pápkám shradu am množniká.//
\glb father-\Dim{}-\Agt{} die-\Cv{}.\Pf{} because console-\mk{pv-hort}//
\glft `I'm sorry for your father's death.'//
\endgl
\a Expressing approbation:\\
\begingl
\gla Prove vlastnu am prehodniká.//
\glb exam-\Acc{} pass-\Cv{}.\Pf{} because praise-\mk{pv-hort}//
\glft \trsl{Congratulations for passing the exam!}//
\endgl
\xe

\section{Conditional clauses}

A conditional sentence consists of two parts: the protasis (or the conditional
clause) upon which the truth of the whole statement depends, and the apodosis
(or the main clause) which expresses the consequence. The conditional mood is
used in Iridian to mark the verb in the conditional clause with a distinction
made between a \emph{realis} form representing logical implication and
sequentiality and an \emph{irrealis} form representing hypothetical, typically
counterfactual conditions, in addition to distinct negative and non-negative
forms for both.

The conditional mood is not used in statements expressing purely factual
implication, i.e., when the statement in the main clause is always known to hold
true when the conditional clause itself holds true. Instead a \ird{še}-clause is
used, with the verb in both clauses marked in the supine of purpose.

\pex
\begingl
\gla Nebo 100 céntihrádu nekrasnit še ustrožit.//
\glb water 100 Celcius-\Ins{} \Caus{}-heat-\Pv{}-\SupP{} \Com{} \Refl{}-boil-\Av{}-\SupP{}//
\glft \trsl{If you heat water to 100\,\degree{}C, it will boil.}//
\endgl
\xe

\pex
\begingl
\gla Nahte štanžit še udumělit.//
\glb too:much drink-\Av-\SupP{} \Com{} \Refl{}-be:drunk-\SupP{}//
\glft \trsl{If you drink too much, you will get drunk.}//
\endgl
\xe

\section{Quotative constructions and  evidentiality}\label{sec:reportedspeech}
\index{reported speech}\index{indirect speech|see{reported speech}}
\index{evidentiality}

\subsection{Quotative construction in general}

Superficially, the Iridian quotative is used to mark {\sc evidentiality}, a
grammatical category concerned with the explicit encoding of the source of
information or knowledge (i.e., evidence) which the speaker claims to have made
use of for producing the primary proposition of the utterance
(\cite[1-2]{diewald2010}). Iridian is unique among languages of Central Europe
(and of Europe in general) in possessing a grammaticalised evidentiality system.
Even non-Indo European languages in the region such as Hungarian (cf. author) or
Basque (cf. \cite{alcazar2010}) do not possess an overt evidential. Of course a
speaker’s source of information may be expressed through other methods 

The Iridian evidentiality system more or less falls under
\posscite{aikhenvald2004} A3 category, where the distinction is between the
marked quotative form for reported speech/hearsay and the unmarked ‘everything
else’ category which is evidentiality-neutral

In practice, however, the quotative is used in an array of other constructions
that is not necessarily predicated on evidentiality, but might be lexically or
semantically motivated as well, perhaps in the same way the subjunctive in
Romance languages have become grammaticalized into a subordination marker (cf.
\cite{poplacketal}).

\subsection{Quotative constructions and reported
speech}\label{sec:quotative-const}

The principal use of the quotative is to explicitly mark reported speech. The
reported clause is separated from the rest of the sentence by the particle
\ird{to-že}.\footnote{In colloquial speech, \ird{to-že} is often reduced to
\ird{če}, or less commonly \ird{dže}.}

\pex
  \begingl
    \gla Koleč sní polšice to-že Lukáš zíček.//
    \glb key \Refl{}.\Acc{} lose-\Av{}-\Pf{}-\Quot{} {\Rel{}=\Quot{} (:=\Qp{})} Lukáš say-\Av{}-\Pf{}//
    \glft ‘Lukáš said he lost his keys.’//
  \endgl
\xe

The particle \ird{to-že} is in fact made up of two separate clitics: the particle \ird{to} which is used to mark relative clauses, and \ird{-že} which is the primary quotative particle. This is made more evident in nested quotations, where \ird{-že} can only be attached to the rightmost reported clause:

\pex
  \begingl
    \gla Dá dněm vednice to Marek žičice to-že Lukáš zíček.//
    \glb \First{}\Sg{} \Dem{}.\Agt{} see-\Pv{}-\Pf{}-\Quot{} \Rel{} Marek say-\Av{}-\Pf{}-\Quot{} \Qp{} Lukáš say-\Av{}-\Pf{}//
    \glft \trsl{Lukáš said Marek said he (Marek) saw me.}//
  \endgl
\xe


Direct quotations do not require the quotative, although they are still
separated from the main clause by to-že.

\pex
  \begingl
    \gla „Dá záščenžit” to-že zíček.//
    \glb \First{}\Sg{} \Neg{}-come-\Av{}-\SupP{} \Qp{} say-\Av{}-\Pf{}//
    \glft \trsl{“I won’t be coming,” (he) said.}//
  \endgl
\xe

The use of pronouns in quoted clauses is similar to English, with the main
exception being the use of the reflexive se if the subject of the quoted clause
is the same as the subject of the main clause. This is true even if the subject
of the main clause is a pronoun.

\pex
  \begingl
    \gla Se to obru na večera záščenžitejí to-že Marek (dá) žiček. //
    \glb \Refl{} \Dem{} night-\Ins{} \Loc{} party-\Acc{} \Neg{}-come-\Av{}-\SupP{}-\Quot{} \Qp{} Marek \First\Sg{} say-\Av{}-\Pf{} //
    \glft \trsl{Marek/I said he/I won't be coming to the party tonight.}//
  \endgl
\xe

The verb \irdp{zěká}{to say} is called a \emph{verbum dicendi}\index{verbum
dicendi} from the Latin meaning ‘verb of speech/speaking.’ Other \emph{verba
dicendi} in Iridian include \irdp{vadá}{to think}; \irdp{kvuštá}{to hear};
\irdp{vidá}{to see}; \irdp{hloupá}{to ask}; \irdp{ohletá}{to remember};
\irdp{sehová}{to recount, to tell a story}. Note that although they are called
verbs ``of speaking'' they do not necessarily introduce speech as much as
function as grammaticalized tags marking the quotative,  which is more properly
analyzed to mark not just speech but inferentiality and evidentiality as well.

More complex \emph{verba dicendi} can be formed by using an imperfect converbial
construction (the converb form in \ird{-ěc}) with a canonical \emph{verbum
dicendi}. To illustrate this consider the following sentences in English:

\pex[*=?*,interpartskip=0pt]
\a She said no.\deftagex{vd}\deftaglabel{1}
\a She whispered no.\deftaglabel{2}
\a She said no \emph{in a whisper}.\deftaglabel{3}
\a \ljudge{?} She said \emph{in a whisper} no.\deftaglabel{4}
\a \ljudge{??} She said \emph{whisperingly} no.\deftaglabel{5}
\xe

We see that both \emph{said} (\getfullref{vd.1}) and \emph{whispered}
(\getfullref{vd.2}) are \emph{verba dicendi} in English. Nonetheless it's also
obvious how \getfullref{vd.2} is simply a function of (\getfullref{vd.1}), i.e.,
we can express (\getfullref{vd.2}) in terms of (\getfullref{vd.1}), in this case
using an adverbial construction (\trsl{in a whisper}) as we see in
\getfullref{vd.3} or the more affected \getfullref{vd.4}. Finally using a simple
adverbial is theoretically allowed in English (\getfullref{vd.5}), although as
we see the resulting construction is rather unwieldy or unnatural-sounding.

In Iridian, however, constructions like (\getfullref{vd.2}) are not permitted,
with preference given to adverbial (or more correctly,
converbial)\index{converb} constructions. Thus we translate (\getfullref{vd.2})
as:

\pex
\begingl
\gla Ne to-že mišlec zíček.//
\glb no \mk{qp} whisper-\Cv{} say-\Av{}-\Pf{}//
\glft \trsl{(She) whispered no.}//
\endgl
\xe

It should be noted as well how the verb \irdp{vadá}{to think} and its derived
forms, due to their inherent meanings, require the subjunctive to be used in the
reported clause. This is true whether or not the subjunctive would have been
used had the reported clause been a regular dependent clause.

\pex
\begingl
  \gla Já mnou nehlí to-že Martin spouvěc váževí.//
  \glb you correct \Cop{}.\Sbj{}.\Quot{} \Qp{} Martin agree-\Cv{}.\Ipf{} think-\Av{}-\Cont{}//
  \glft \trsl{Martin agrees that you are right.}//
\endgl
\xe

The \emph{verbum dicendi} is often marked in the agentive voice, although
Iridian grammar also permits the verb to be marked in the patientive, but with
the resulting construction often having a more explanatory meaning.

\pex
\a
\begingl
  \gla Já mnou nehlí to-že Martin spouvěc váževí.//
  \glb you correct \Cop{}.\Sbj{}.\Quot{} \Qp{} Martin agree-\Cv{}.\Ipf{} think-\Av{}-\Cont{}//
  \glft \trsl{Martin agrees that you are right.}//
\endgl
\a
\begingl
  \gla Já mnou nehlí to-že Martin spouvěc vadneví.//
  \glb you correct \Cop{}.\Sbj{}.\Quot{} \Qp{} Martin agree-\Cv{}.\Ipf{} think-\Pv{}-\Cont{}//
  \glft \trsl{What Martin agrees to is that you are right.}//
\endgl
\xe

We see from  that when it comes to reported speech and similar constructions in
Iridian, the \ird{verbum dicendi}\index{verbum dicendi} is not necessary to
create a well-formed sentence. The same is true with the quotative particle
\ird{to-že}. Both can be omitted without making the sentence grammatically
incorrect since the quotative particle is enough to identify the reported
clause.\index{reported speech}.

In most instances, however, removing either the main verb or the main verb and
the quotative particle can cause the resulting sentence to acquire a new
meaning. This is especially true when the quotative mood is used not to report
speech but to imply a certain unsureness on the part of the speaker about the
information being presented, or for the speaker to distance themself by implying
through the use of the quotative that the information is secondhand and not
theirs. Generally \ird{to-že} is kept when the speaker is quoting themself, to
repeat or emphasize what they have said, or expletively, to express their
frustration or affirmation.

Interestingly, commands and requests are not treated as reported speech but as
regular subordinate clauses governed by \ird{to} and not by \ird{to-že}.

When the quoted clause is a question, whether a direct one or not, the quoted
clause is preceded by the particle \irdp{a}{and} and the word
\irdp{ane}{whether} is used instead of \ird{to-že}. The word \ird{ane} is also
used for verba dicendi that are interrogative in nature, such as
\irdp{préhoustá}{to ask},

\pex
\begingl
  \gla A Janek zdalšice ane préhousček.//
  \glb and Janek have:breakfast-\Av{}-\Pf{}-\Quot{} whether ask-\Av{}-\Pf{}//
  \glft \trsl{(He) asked (me) whether Janek has had breakfast yet.}//
\endgl
\xe

\pex
\begingl
  \gla A tóm to mládu hodinaže ane, ně svad postupeví.//
  \glb and book this year-\Ins{} finish-\mk{pv-ctpv-quot} whether \Pl{} fan be:excited-\mk{cont}//
  \glft \trsl{His fans are excited to know if he'll finish his book this year.
}//
\endgl
\xe

The quotative is also triggered by phrases introduced with \irdp{ty}{according
to} or \irdp{záty}{contrary to,} with the latter requiring the subjunctive. 

\pex
\begingl
  \gla Messi a ty Marku debil neví.//
  \glb Messi and according:to Marek-\Ins{} spaz \Cop{}.\Sbj{}//
  \glft \trsl{Marek thinks Messi is a spaz.}//
\endgl
\xe

\pex
\begingl
  \gla Na Vrešlove a záty mamcě čestu papcě vednice stožišejí.//
  \glb \Loc{} Wrocław-\Acc{} and \Neg{}-according:to mother-\Dim{}-\Gen{} desire-\Ins{} father-\Gen{} see-\Pv{}-\SupP{} go-\Av{}-\Subj{}.\Pf{}-\Quot{}//
  \glft \trsl{Against my mother’s wishes, I went to Wrocław to see my father.}//
\endgl
\xe


\subsection{Bare quotatives and clause linking}

Quoted clauses in Iridian may also appear without an overt predicate, as well as
without being signalled by the quotative particle \ird{to-že}. We will call this
construction a {\sc bare quotative} after the terminology in
\textcite{tomioka2019} in reference to embedded quotative constructions in
Japanese and Korean without overt predicates. The term as originally used by
these authors refer only to embedded quotatives in Japanese and Korean, but we
will be using it to refer to both an unselected (i.e., predicateless) quotative
in a subordinate clause (which we will call {\sc syntactic}) and in the main
clause (which we will call {\sc semantic}).

The choice to call the second type a semantic bare quotative is motivated by the
fact that an unselected quotative in the main clause is often used not to mark a
speech act but to indicate the epistemic value of (viz., to pass the speaker's
judgement on) a proposition. Nevertheless, we can still see it used as a true
quotative, as when the omission of the predicate or the quotative particle is
through mere ellipsis.

The first type, on the other hand, is mostly used as a clause-linking strategy.
The quotative construction is still considered as a speech act, but, like
converbial constructions or \ird{še} clauses, the relationship between the main
clause and the reported clause becomes interpreted as being one of causality, or
at least of dependency, although of course this causality or dependency is only
indirect, as we see in the examples below, where the embedded quotative and the
simple \ird{še} clause present to different interpretations.

\pex
  \a(adapted from \cite[3]{tomioka2019})\\
  \begingl
    \gla Pizba rážice še sad Markám nakdavtébik.//
    \glb rain stop-\Av{}-\Pf{}-\Quot{} \Com{} garden Marek-\Agt{} \Incp{}-clean-\Ben{}-\Pf{}//
    \glft \trsl{Marek began cleaning the garden, (saying/thinking) it finally stopped raining.}//
  \endgl
  \a\begingl
    \gla Pizba razek še sad Markám nakdavtébik.//
    \glb rain stop-\Av{}.\Pf{} \Com{} garden Marek-\Agt{} \Incp{}-clean-\Ben{}-\Pf{}//
    \glft \trsl{The rain having stopped, Marek began cleaning the garden.}//
  \endgl
\xe


\subsection{Epistemic extensions}

As in most other languages with an overt evidential system, the Iridian
quotative has secondary epistemic extensions. This may be realised either by
using the quotative by itself or through auxiliary epistemic markers. As we have
established in the previous sections, the quotative can be used by a speaker
both to distance themself from the statement on the one hand and to assert their
belief in its truthfulness on the other; the use of a secondary epistemic marker
eliminates this possible confusion in what would otherwise have been a
contradictory usage of the same grammatical category. These auxiliary particles,
nonetheless, may of course be left out in discourse if the speaker thinks the
epistemic usage of the quotative is clear enough from the context.

A speaker’s judgement of the truthfulness of a statement may be made clear by
the dubitative \ird{bude} or the affirmative \ird{toleto}. When using the
quotative to quote oneself, \ird{bude} expresses a disbelief predicated upon
surprise rather than on a judgement of a statement’s veracity; used the same
way, \ird{toleto} acquires a secondary meaning of insistence, even annoyance.

\pex
\begingl
  \gla Sól bude tahatnitejí.//
  \glb peace \Dub{} bring-\Pv{}-\SupP{}-\Quot{}//
  \glft \trsl{They say they come in peace but I don’t believe it.}//
\endgl
\xe

\pex
\begingl
  \gla Ma já bude ža konědnitejí to!//
  \glb but \Second{}\Sg{} \Dub{} already marry-\Pv{}-\SupP{}-\Quot{} \Rel{}//
  \glft \trsl{I still can’t believe you’re already getting married!}//
\endgl
\xe

\pex
\begingl
  \gla Marek toleto poslem všihnébice.//
  \glb Marek \Aff{} message-\Agt{} be:angry-\Ben{}-\Pf{}-\Quot{}//
  \glft \trsl{I’m telling you the message really made Marek angry.}//
\endgl
\xe

\pex
\begingl
  \gla Méva toleto sehovnáně!//
  \glb all \Aff{} recount-\Pv{}-\Ret{}-\Quot{}//
  \glft \trsl{But I’ve told you everything I know already!}//
\endgl
\xe

A speaker’s uncertainty may also be expressed using the quotative even when the
statement directly came from the speaker. The uncertainty may refer to both the
factuality of the statement or to its source. This strategy is used to signal
the speaker’s emotional or cognitive distance from the event. This may be
further complemented by the particle \ird{iz} which we will glossing here as
\Rep{} for reportative but only for the sake of convenience, in order to
distinguish the various auxiliary particles we have introduced here, as the
“reportative” does not exist as a true grammatical category in Iridian for our
purposes. \ird{Iz} implies a greater degree of disjunction between the speaker
and the statement than the plain quotative. Although it does not pass a
judgement on the truth value of the statement as do \ird{dube} or \ird{toleto},
\ird{iz} makes it clear that the statement did not come from the speaker and
that the responsibility for the statement does not lie on them. \ird{Iz} is
particularly common in newscasts or in other formal settings where the speaker
is communicating statements from another speaker or group and the identity of
the speaker or group has already been established earlier in the conversation
and is thus known to everyone.

\pex
\ird{Interiorministerium shléd o senátor Koupárám poto němstministar Novaka
dozakuzacunóvim arklaruma mnilounek. Na Ministerija še Ružómu ty
zěka\-mi\-te\-mu nežni posohredou, a viční němstministarí za Moshóva besuk
{\emph{iz}} Ministerija zázběro\-vnevíje. Akuzace \emph{iz} shlac
investěharnimejí a němstministar \emph{iz} udarklaržice za Ministara breví
paholžáše.}\smallskip\\
{\footnotesize\trsl{The Ministry of the Interior has released a statement today
regarding the accusations of misconduct levelled by Senator Koupár against
Deputy Minister Novak. According to its spokesperson, the ministry is currently
not in talks with Russia and has not sanctioned the reported Deputy Minister's
recent visit to Moscow. It is now investigating the allegations and has asked
Deputy Minister Novak to submit a brief to the Minister to explain his
actions.}}
\xe

Uncertainty on the truthfulness of the statement may also be expressed using the
inferential particles \ird{bylo} and \ird{atole}. Whereas \ird{iz} raises the
question of the character of the source and is neutral as to the speaker’s
commitment to it (although one can be understood simply by pointing out the fact
that the source is something other than oneself to be effectively passing
judgement) both \ird{bylo} and \ird{atole} reflect the speaker’s judgement.
\ird{Bylo} in general is used when the proposition is coming from the speaker
themself while \ird{atole} is used when the speaker thinks that the statement
can be inferred from the surrounding facts.

\pex
\begingl
  \gla Na Hospode bylo milestunitejí.//
  \glb \Loc{} Hospoda-\Acc{} perhaps have:dinner-\Lv{}-\SupP{}-\Quot{}//
  \glft \trsl{Maybe we can have dinner at the \emph{Hospoda} tonight?}//
\endgl
\xe

\pex
\begingl
  \gla Ně ruščevní šar atole na Roubžína ščenžáně.//
  \glb \Pl{} Russian-\Att{} tank \Infer{} \Loc{} Roubže-\Acc{} arrive-\Av{}-\Ret{}-\Quot{}//
  \glft \trsl{The Russian tanks must have reached Roubže by now.}//
\endgl
\xe

\section{Relative and comparative
constructions}\label{relativecomparative}\index{comparative construction}

The clitic\index{clitic} \ird{tám} is used to form simple comparative and
relative constructions. \ird{Tám} is often ommitted where the comparison can be
implied from context. In this construction, the standard of
comparison\index{standard of comparison} (the noun preceded by `than' in
English\index{English}) is unmarked and the noun being compared marked in the
agentive\index{agentive case} if it is a positive/negative comparison, or in the
instrumental\index{instrumental case} if it is a correlation.

\pex
\a\begingl
\gla Janek(-tám) Markám nestaževí.//
\glb Janek Marek-\Agt{} tall-\mk{cont}//
\glft \trsl{Marek is taller than Janek.}//
\endgl
\a\begingl
\gla Janek(-tám) Marku nestaževí.//
\glb Janek Marek-\Ins{} tall-\mk{cont}//
\glft \trsl{Marek is as tall as Janek.}//
\endgl
\xe

Note that \ird{tám} can only be used with the copulative form of the stative
verb\index{stative verb}, as the attributive and nominal forms have separate
conjugated comparative forms. When using these forms, however, the standard of
comparison is marked in the genitive\index{genitive case}. In relative
constructions, the instrumental\index{instrumental case} is also replaced with
the genitive\index{genitive case}, but the modifier \ird{zní}, \trsl{same} is
added before the stative verb\index{stative verb}.

\pex
\a
\begingl
\gla Jancí nestašení hloc mlazka.//
\glb Janek-\Gen{} tall-\Comp{}-\Att{} boy brother-\Dim{}//
\glft \trsl{The boy who is taller than Janek is my brother} (\emph{Lit.,} \trsl{The taller-than-Janek boy is my brother.})//
\endgl
\a
\begingl
\gla Jancí zní nestažení hloc mlazka.//
\glb Janek-\Gen{} same tall-\Comp{}-\Att{} boy brother-\Dim{}//
\glft \trsl{The boy who is as tall as Janek is my brother.}//
\endgl
\xe

\ird{Tám} can be relativized by appending the clitic\index{clitic} \ird{to}.
When used with \ird{tám-to} the standard of comparison is marked in the
patientive case\index{patientive case}. The use of tám-to in relative clauses is
discussed in further detail in the next chapter.

\ex
\begingl
\gla Viktor na shlopa tám-to nestážek.//
\glb Viktor \Loc{} siblings-\Acc{} \Comp{}=\Rz{} be:tall-\Av{}-\Pf{}//
\glft \trsl{Among the siblings, Viktor grew up to be the tallest.}//
\endgl
\xe

\ex
\begingl
\gla Jankám Marka tám-to zuštalébik ko Tereza//
\glb Janek-\Agt{} Marek-\Acc{} \Comp{}=\Rz{} be:happy-\Ben{}-\Pf{} \Lnk{} Tereza//
\glft \trsl{Tereza, whom Janek made happier than Marek}//
\endgl
\xe

\ex
\begingl
\gla Marka tám-tóví zuštalébik ko oblašc//
\glb Marek-\Acc{} \Comp{}=\Rz{}-\Gen{}= be:happy-\Ben{}-\Pf{} \Lnk{} pet//
\glft \trsl{the pet [of the person who was made happier than Marek]}//
\endgl
\xe

Iridian does not have a morphologically distinct superlative construction. For
example, \ird{pizdení} (from \ird{pizdá}, \trsl{to be big}) can either mean
\trsl{bigger} or \trsl{biggest} depending on context. Where the meaning cannot
be easily implied from context, the word \ird{ohnu} (derived from the word
\ird{ohna}, \trsl{first} in the instrumental case) is often used as quantifier.

\pex
\a
\begingl
\gla Univerzitet na razmeka pizdenou.//
\glb university \Loc{} city-\Acc{} be:big-\Comp{}-\Nz{}//
\glft \trsl{(This) university is the biggest in the city.}//
\endgl
\a
\begingl
\gla Univerzitet na razmeka ohnu pizdenou.//
\glb university \Loc{} city-\Acc{} first-\Ins{} be:big-\Comp{}-\Nz{}//
\glft \trsl{(This) university is the biggest in the city.}//
\endgl
\xe

When using an adverbial construction with the instrumental case to modify or
quantify the comparison, the adverbial phrase must immediately precede the
stative verb if in the attributive or nominal form, or the particle \ird{tám}
otherwise. The same is true with invariable modifiers like \ird{nahte},
\trsl{too much}, \ird{dnu}, \trsl{a bit}, etc.

\ex
\begingl
\gla To bagáž jánám u 10 kilográmu tám prékveví.//
\glb \Dem{}.\Prox{} baggage \Dem{}.\Med{}-\Agt{} around 10 kilogram-\Ins{} \Comp{}= heavy-\Cont{}//
\glft \trsl{This baggage is heavier by about 10 kilograms than that one.}//
\endgl
\xe

\ex
\begingl
\gla u 10 kilográmu prékvení bagáž//
\glb around 10 kilogram-\Ins{} heavy-\Comp{}-\Att{} baggage//
\glft \trsl{the baggage, which is heavier by about 10 kilograms}//
\endgl
\xe

\ex
\begingl
\gla Nahte pizdenou zmažnikóveš.//
\glb too:much big-\Comp{}-\Nz{} make-\Pv{}-\Pf{}-\Nz{}-\Second{}\Sg{}//
\glft \trsl{The much bigger one is the one you made.}//
\endgl
\xe

\section{Specific construction types}
\subsection{Questions}\index{questions!syntax of}\index{interrogative
sentence|see{questions!syntax of}}

There are two  main  categories  of  interrogative  sentences in Iridian: yes-no
and  question-word questions (or \emph{wh-}questions).

\subsubsection{Yes-no questions}\index{questions!yes-no}\index{questions!syntax
of}

A declarative sentence can be made into a question by a simple rise in
intonation at the end of the phrase:

\pex
\a
\begingl
\gla Janek ža uzdravšek.//
\glb Janek already \Refl{}-sleep-\Av{}-\Pf{}//
\glft \trsl{Janek has fallen asleep.}//
\endgl
\a
\begingl
\gla Janek ža uzdravšek?//
\glb Janek already \Refl{}-sleep-\Av{}-\Pf{}//
\glft \trsl{Has Janek fallen asleep yet?}//
\endgl
\xe

Yes-no questions, especially longer ones, may also be formed using the clitic
\ird{no}, which immediately follows the element of the sentence being
questioned. To question the sentence as a whole, \ird{no} sentence-initially.
\ird{No} may also appear after other elements of the sentence, but the resulting
word order is generally more emphatic and often includes promoting the element
where \ird{no} to the topic position and the nominalization of the resulting
verb phrase if possible.


\pex
\a Sentence-initial \ird{no}:\\
  \begingl
  \gla No Balžaróma Europevní Unijí čelina?//
  \glb \Q{} Bulgaria European-\Att{} Union-\Gen{} member//
  \glft \trsl{Is Bulgaria a member of the European Union?}//
  \endgl
\a Cliticised \ird{no}:\\
  \begingl
  \gla Janek zmáčime-no?//
  \glb Janek run-\Av{}-\Prog{}=\Q{}//
  \glft \trsl{Is it running that Janek is doing now?}//
  \endgl
\a Cliticised \ird{no} triggering topicalization of questioned element:\\
  \begingl
  \gla Janek-no zmáčimou?//
  \glb Janek=\Q{} run-\Av{}-\Prog{}-\Nz{}//
  \glft \trsl{Is it Janek who is running now?}//
  \endgl
\xe

\irdp{Ane}{whether} may also be used instead of \ird{no} to indicate uncertainty
on the part of the speaker, or in polite or formal speech, to avoid asking a
direct question. \ird{Ane} functions the same way as \ird{no} and may be used
sentence initially or as a clitic.

\pex
\a\begingl
  \gla Ane Stám Kovárž niehu na sésta o leguánu hvaružnašách?//
  \glb whether mister Kovárž later-\Ins{} \Loc{} convention-\Acc{} about iguana-\Ins{} give:a:speech-\Av{}-\Ctp{}//
  \glft \trsl{Would Mr Kovárž be giving a speech about iguanas later at the convention?}//
  \endgl
\a\begingl
  \gla Stám Kovárž-ane niehu na sésta o leguánu hvaružnašit?//
  \glb mister Kovárž=whether later-\Ins{} \Loc{} convention-\Acc{} about iguana-\Ins{} give:a:speech-\Av{}-\Sup{}//
  \glft \trsl{Would it be Mr Kovárž who will be giving a speech about iguanas later at the convention?}//
  \endgl
\xe

To make an existential sentence\index{existential construction} a yes-no
question, it is first transformed to the negative and the particle \ird{no/ane}
is then attached to the word \ird{niho}. If however, the theme of the sentence
is quantified, the word \ird{ješ}\index{ješ@\emph{ješ}} is kept (but shifted to
the front of the quantifier), and \ird{no} is attached to the quantifier. The
form \ird{ješ-no} is ungrammatical. A sentence-initial \ird{no/ane} cannot be
used in transforming an existential construction.

\pex
\begingl
\gla Marka niho-no oblašc?//
\glb Marek-\Acc{} \N{}\Exst{}=\Q{} pet//
\glft \trsl{Does Marek have a pet?}//
\endgl
\xe

\pex
\a
\begingl
\gla Co bibliotécě Marka hroná ješ kupénenik ko tóm.//
\glb \Abl{} library-\Gen{} Marek-\Acc{} three \Exst{} borrow-\Pv{}-\Pf{} \Lnk{} book//
\glft \trsl{Marek borrowed three books from the library.}//
\endgl
\a
\begingl
\gla Co bibliotécě Marka ješ hroná-no kupénenik ko tóm?//
\glb \Abl{} library-\Gen{} Marek-\Acc{} \Exst{} three=\Q{} borrow-\Pv{}-\Pf{} \Lnk{} book//
\glft \trsl{Did Marek borrow three books from the library.}//
\endgl
\xe

Tag questions may be formed by appending the phrase \irdp{no/ane zám
le\v{t}}{isn't it the truth} (cf. Russian\index{Russian} \textit{\cyrtext не
правда ли}) to the end of the sentence. In colloquial speech, it is also common
to simply use \irdp{da}{yes} instead.

\pex
\begingl
\gla Traví kupénžek, no zám le\v{t}? /da?//
\glb bread-\Gen{} buy-\Av{}-\Pf{} \Q{}= \Neg{}= truth yes//
\glft \trsl{You bought some bread, didn't you? /right?}//
\endgl
\xe

\subsubsection{Contentquestions}\label{sec:content-questions}
\index{wh-question@\emph{wh-}question}\index{information
question|see{\emph{wh}-question}}\index{content
question|see{\emph{wh}-question}}

Content questions, also known as \emph{wh}-questions, are formed using the
interrogative pronouns \irdp{jede}{who,} \irdp{ježe}{what,} \irdp{jena}{where,}
etc.\footnote{ A full list of interrogative pronouns can be found in
\S~\ref{sec:int-pron}. } Iridian requires the \emph{wh}-phrase to be moved to
the beginning of the sentence, thus causing it to occupy the topic position.
This \emph{wh}-fronting\index{wh-fronting@\emph{wh}-fronting} consequently
causes the voice of the main verb to be reframed to accomodate the new topic.
More commonly, especially colloquial Iridian\index{colloquial Iridian}, this
also means the nominalization\index{nominalization} of the main verb phrase,
essentially making the question an equational sentence.

\pex
\a\begingl
\gla Karel na Roubžení verštáta možlaševí.//
\glb Karel \Loc{} Roubže-\Gen{} suburbs-\Acc{} live-\Av{}-\Cont{}//
\glft \trsl{Karel lives in the suburbs of Roubže.}//
\endgl
\a\begingl
\gla Jena Karlám možlouneví? /možlounívou?//
\glb where Karel-\Agt{} live-\Lv{}-\Cont{} live-\Lv{}-\Cont{}-\Nz{} //
\glft \trsl{Where does Karel live?}//
\endgl
\xe

Alternatively, the element being questioned may be replaced with a question word
without changing the original word order, in which case the addition of the
clitic \ird{no} is required. Note that questions formed this way generally have
a more emphatic meaning.

\pex
\begingl
\gla Karel jena-no možlaševí?//
\glb Karel where=\Q{} live-\Av{}-\Cont{}//
\glft \trsl{Where did you say Karel lived?}//
\endgl
\xe


\emph{Wh}-fronting may sometimes cause peripheral elements of a phrase to be
moved together with the \emph{wh}-item to the beginning of the sentence, a
phenomenon linguists call `pied-piping' (\cite[263-4]{ross1967}). When this
occurs, Iridian is more conservative than English in that it usually keeps the
same question word instead of replacing it with a specialized one (in English,
normally, `which'); it may, however, use \irdp{jak}{which} if the expected
answer to the question is an element of a class, i.e., not unique. Consider, for
example, the two questions below:

%%%% zuscve, cf. Cz sodestvi, Ru. sodestvo
\pex
\a
\begingl
\gla Jena zuscve možlounívou?//
\glb where neighborhood live-\Lv{}-\Cont{}-\Nz{}//
\glft \trsl{Which (\emph{lit.,} where) neighborhood do you live in?}//
\endgl
\a
\begingl
\gla Jak kvartír možlounívou?//
\glb which apartment live-\Lv{}-\Cont{}-\Nz{}//
\glft \trsl{Which of these apartments is the you live in?}//
\endgl
\xe

In cases where there are multiple \emph{wh}-elements in the sentences, they are
normally all fronted, with the main question word first followed by the rest in
order of importance. Interestingly, too, any or all of the fronted
\emph{wh}-items may be pluralised with \ird{ně} if the speaker expects that the
answer is plural.

\pex
\a\begingl
\gla Jede ježe jena hloupškou?//
\glb who what where ask-\Av{}-\Pf{}-\Nz{}//
\glft \trsl{Who asked what where?}//
\endgl
\a\begingl
\gla Ně jede ježe jena hloupškou?//
\glb \Pl{}= who what where ask-\Av{}-\Pf{}-\Nz{}//
\glft \trsl{Which persons asked what where?}//
\endgl
\a\begingl
\gla Jede ně ježe jena hloupškou?//
\glb who \Pl{}= what where ask-\Av{}-\Pf{}-\Nz{}//
\glft \trsl{Who asked what things where?}//
\endgl
\xe

In the case of more complex \emph{wh}-questions involving the movement of a
  \emph{wh}-item from an embedded clause, Iridian is similar to
  Bulgarian\footnote{ \posscite{rudin1988} description on the nature of multiple
  \emph{wh}-fronting in Bulgarian\index{Bulgarian} involves the movement of the
  \emph{wh}-item to closest interrogative SpecCP, which does not necessarily
  need to occupy the topic position in the sentence. Compare, for example the
  following sentences in Bulgarian and Iridian.

  \ex[lingstyle=fnex,belowexskip=-1em,aboveglftskip=1pt]
  Bulgarian\index{Bulgarian} (\emph{ibid.,} 451)\smallskip\\
    \begingl 
    \gla Boris na kogo kakvo kaza [če šte {dade --- ---]}? //
    \glb Boris to whom what said that will give-\Third{}\Sg{}//
    \glft \trsl{What did Boris say that (he) would give to whom?}//
  \endgl
  \xe
\smallskip
  \ex[lingstyle=fnex,belowexskip=-1em,aboveglftskip=1pt]
    \begingl 
    \gla Ježe jehát Borisám dejatnách to zíknou?//
    \glb what to:whom Boris-\Agt{} give-\Pv{}-\Ctp{} \Rz{} say-\Pv{}-\Pf{}-\Nz{}//
    \glft \trsl{What did Boris say that (he) would give to whom?}//
  \endgl
  \xe

} in requiring all the \emph{wh}-items to be fronted (cf.~\cite[450]{rudin1988}).

\pex
\begingl
\gla Ježe jehát dejatnách to zíknou?//
\glb what to:whom give-\Pv{}-\Ctp{} \Rz{} say-\Pv{}-\Pf{}-\Nz{}//
\glft \trsl{What did she say that she will give to whom?}//
\endgl
\xe


\subsubsection{Indirect questions}

Indirect questions are constructed in the subjunctive, with the addition of the
particle \ird{aš}.

\pex
\begingl
\gla Nú aš hošezíla.//
\glb tomorrow \mk{q.ind} rain\mk{av-sbj.ipf}.//
\glft \trsl{I wonder if it's gonna rain tomorrow.}//
\endgl
\xe

\subsubsection{Answering questions}\label{sec:ansyn}

Most yes-no questions may be answered by repeating the focal word or phrase in
the original question or echoing the syntax of the question itself.

\ex
\vtop{\halign{%
#\hfil& \qquad #\hfil\cr
\ird{---\,Kartuškí tak slouveževí?} & \trsl{\small Do they sell potatoes here?}\cr
\ird{---\,Slouveževí?} & \trsl{\small They do.}\cr
}}
\xe

Alternatively, the question may be answered by \irdp{da}{yes} or \irdp{ne}{no,}
both of which have been adapted from Common Slavic\index{Common Slavic}. In
colloquial speech it is also common to use \ird{já} or \ird{jó} for \trsl{yes}
(most likely borrowings from German\index{German}). These polarity words may be
used alone or in combination with the echo response. In general, the order does
not matter, although it is more common for the polarity word to appear after the
echo response. Unlike English \trsl{yes,} \ird{da} is used when confirming the
question posed by the speaker, whether or not it is in the affirmative or in the
negative. When denying or negating a question, Iridian uses \ird{ne} is used
when the original question was framed in the negative and \ird{ale} otherwise.

\ex
\vtop{\halign{%
#\hfil\hfil\cr
\ird{---\,Lošní Nolaní vilm ža oudnenik?}\cr
\ird{---\,\v{Z}a oudnenik, da. Má záčesčik.}\cr
\ird{---\,Ne, po zoudnenik.}\smallskip\cr
\trsl{Have you seen Nolan's new film?}\cr
\trsl{I've seen it, yes. But I didn't like it.}\cr
\trsl{No, I haven't seen it yet.}\cr
}}\xe

\ex\vtop{\halign{%
#\hfil\hfil\cr
\ird{---\,No daní trehlo za banka podarnílá to-že Janek záléháček?}\cr
\ird{---\,Léháček, ale. Má avtem bych hebo.}\cr
\ird{---\,Záléháček, da.}\smallskip\cr
\trsl{Weren't you advised by Janek to submit your tax return to the bank?}\cr
\trsl{He did, yes. But my car broke down yesterday.}\cr
\trsl{No, he didn't advise me to.}\cr
}}
\xe

\ird{Da} (or sometimes \ird{a da}) may also preface answers to questions as a
form of intensifier, or to indicate that the speaker considers the answer to the
question as an obvious truth.

\ex
\vtop{\halign{%
#\hfil& \qquad #\hfil\cr
\ird{---\,Na muzla ješ vdenikou.} & \trsl{I saw someone at the mall today.}\cr
\ird{---\,Jede?} & \trsl{Who?}\cr
\ird{---\,Da Janek.} & \trsl{Well, Janek, of course.}\cr
}}\xe

The answer does not need to be positive for \ird{da} or \ird{a da} to be used.

\ex\vtop{\halign{%
#\hfil\hfil\cr
\ird{---\,\v{S}abatu de koncerta stožit?}\cr
\ird{---\,A da ne. To kapela šem záčesčeví.}\smallskip\cr
\trsl{Are you coming to the concert on Saturday?}\cr
\trsl{Well no, I don't even like that band.}\cr
}}
\xe


As for questions involving existential constructions


\subsection{Negation}\index{negation}

In Iridian sentences, negation is performed by the particle \ird{zám}, which
attaches to the beginning of the word or phrase  it negates. The default
position of the negative particle is before the main verb where it surfaces as
\ird{z-} before vowels, \ird{ž-} before \emph{i}-glides, and \ird{zá-}
elswehere. This elision does not occur where \ird{zám} appears elswehere in
the sentence.

\pex
\a
\begingl
    \gla Janek Martina Markám {zá}hévoržébik.//
    \glb Janek Martin-\Acc{} Marek-\Agt{} \Neg{}know-\Ben{}-\Pf{}//
    \glft \trsl{Marek did not introduce Janek to Martin.}//
\endgl
\a
\begingl
    \gla {Zám} Janek Martina Markám hévoržébik.//
    \glb \Neg{} Janek Martin-\Acc{} Marek-\Agt{} know-\Ben{}-\Pf{}//
    \glft \trsl{It was not Janek whom Marek introduced to Martin.}//
\endgl
\a
\begingl
    \gla Janek {zám} Martina Markám hévoržébik.//
    \glb Janek \Neg{} Martin-\Acc{} Marek-\Agt{} know-\Ben{}-\Pf{}//
    \glft \trsl{It was not Martin whom Marek introduced Janek to.}//
\endgl
\a
\begingl
    \gla Janek Martina {zám} Markám hévoržébik.//
    \glb Janek Martin-\Acc{} \Neg{} Marek-\Agt{} know-\Ben{}-\Pf{}//
    \glft \trsl{It was not Marek who introduced Janek to Martin.}//
\endgl
\xe

\ird{Zám} attaches directly to the word or phrase it negates, although it is
also common, especially in spoken Iridian, to append the clitic \ird{-te} after
the word being negated by \ird{zám} to provide more emphasis on the negation.
This is a fairly recent development, and is not found in older texts or in the
written language.

\pex
\a
\begingl
    \gla {Zám} Janek{-te} Martina Markám hévoržébik.//
    \glb \Neg{} Janek=\Foc{} Martin-\Acc{} Marek-\Agt{} know-\Ben{}-\Pf{}//
    \glft \trsl{It was not Janek whom Marek introduced to Martin.}//
\endgl
\a
\begingl
    \gla Janek {zám} Martina{-te} Markám hévoržébik.//
    \glb Janek \Neg{} Martin-\Acc{}=\Foc{} Marek-\Agt{} know-\Ben{}-\Pf{}//
    \glft \trsl{It was not Martin whom Marek introduced Janek to.}//
\endgl
\a
\begingl
    \gla Janek Martina {zám} Markám{-te} hévoržébik.//
    \glb Janek Martin-\Acc{} \Neg{} Marek-\Agt{}=\Foc{} know-\Ben{}-\Pf{}//
    \glft \trsl{It was not Marek who introduced Janek to Martin.}//
\endgl
\xe

The different constituents of the sentence can be negated simultaneously; thus,
for example, the sentence below is grammatically permitted:

\pex
\begingl
    \gla {Zám} Janek {zám} Martina {zám} Markám {zá}hévoržébik.//
    \glb \Neg{} Janek \Neg{} Martin-\Acc{} \Neg{} Marek-\Agt{} \Neg{}-know-\Ben{}-\Pf{}//
    \glft \trsl{It was not Janek who was not introduced to someone who is not Martin by someone who is not Marek.}//
\endgl
\xe

Nonetheless, due to their general unwieldiness, forms like this are extremely
rare (both in the spoken and the written language), with preference given to
single and double negation instead. Since \ird{-te} can only appear in a
sentence once, where there are more than one negate constituent in a sentence,
\ird{-te} is appended to the element which has the most significance (usually
the topic); or, if there are two constituents negated and one of them is the
main verb, \ird{-te} is appended to that other element.

\pex
\begingl
    \gla {Zám} Janek{-te} Martina Markám {zá}hévoržébik.//
    \glb \Neg{} Janek=\Foc{} Martin-\Acc{} Marek-\Agt{} \Neg{}know-\Ben{}-\Pf{}//
    \glft \trsl{It was not Janek who was not introduced to Martin by Marek.}//
\endgl
\xe

Alternatively, if there is only one element/phrase negated in the sentence other
than the main verb (which itself may or may not be negated), it is common,
especially in colloquial Iridian\index{colloquial Iridian}, to
nominalize\index{nominalization} the whole verb phrase and transform the
sentence into a copular construction\index{copular construction}, with the
negated phrase as the new topic\index{topic} and the nominalized verb phrase as
the predicate\index{predicate}.

\pex
\begingl
    \gla Zám jájka na Praha zadačkou.//
    \glb \Neg{} daughter-\Dim{} \Loc{} Prague-\Acc{} move-\Av{}-\Pf{}-\Nz{}//
    \glft \trsl{It was not my daughter who moved to Prague.}//
\endgl
\xe




\subsection{Existential construction}\index{existential construction}
\label{sec:exst}

\subsubsection{In general}
An existential sentence is a specialized construction used to express the
existence or presence of someone or something. The particle \ird{ješ} and its
inverse \ird{niho} are used to form existential sentences. 
\begin{multicols}{2}
\pex
\a\begingl
\gla Tak ješ zarno.//
\glb here \Exst{} people//
\glft \trsl{There are people here.}//
\endgl
\a\begingl
\gla Tak niho zarno.//
\glb here \N{}\Exst{} people//
\glft \trsl{There is no one here.}//
\endgl
\xe
\end{multicols}

The existential construction in Iridian was originally a
locative\index{locative} one, and this could still be seen in how the use of
\ird{ješ} and \ird{niho} requires both the noun or noun phrase whose existence
is posited and the location where such existence is said to be true to be
explicitly present in the sentence. In true existential sentences (e.g.,
\trsl{There is a God} or \trsl{There is still hope}) where the argument is the
existence of something and not just it's mere presence somewhere, the patientive
form of the reflexive verb \ird{se}, \ird{sní}, is used. In addition, where this
ostensible location is present in the sentence, it would occupy the
topic\index{topic} position\footnote{Although this location (often surfacing as
a \ird{na} clause) appears where the topic of the sentence normally would, it
would be more correct to analyze an existential construction as an inversion of
the regular topic-predicate word order in Iridian. Viewed this way, we can think
of \ird{ješ} or \ird{niho} as a pseudoverb, and the phrase consisting of the
first half of the sentence and ending with this pseudoverb is the predicate
while the unmarked second half is the topic. This approach has the benefit of
keeping the predicate with a verb-final internal word order and the topic as
unmarked, both in accordance with the basic rules of Iridian syntax; however,
this does not account for the use of the dummy \ird{sní} in true existential
clauses.} in the sentence, and unlike in regular sentences, must be explicitly
marked in the patientive.\index{patientive}


\begin{multicols}{2}
\pex
\a\begingl
\gla \ljudge{*}Ješ tieho.//
\glb \Exst{} god//
\glft \trsl{There is a God.}//
\endgl
\a\begingl
\gla Sní ješ tieho.//
\glb \Refl{}.\Acc{} \Exst{} god//
\glft \trsl{There is a God.}//
\endgl
\xe
\end{multicols}

The use of \ird{sní} as a placeholder is not required however if the noun or
noun phrase whose existence is the subject of the sentence is quantified, either
by a numeral or otherwise by an indefinite quantifier.

Statements expressing location use a copular construction, although an
existential construction may be used in the negative to convey an absence of
something, with the normal negative construction used where emphasis on one
element of the sentence is desired by the speaker.

\pex
\begingl
\gla Dá na duma.//
\glb \mk{1s.str} \Loc{} house-\Acc{}//
\glft \trsl{I'm at home.}//
\endgl
\xe

\pex
\a\begingl
\gla Na duma niho dá.//
\glb \Loc{} house-\Acc{} \N{}\Exst{} \mk{1s.str}//
\glft \trsl{I'm not at home.}//
\endgl
\a\begingl
\gla Zám dá na duma//
\glb \Neg{} \mk{1s} \Loc{} house-\Acc{}//
\glft \trsl{It is not I who's at home.}//
\endgl
\a\begingl
\gla Dá zám na duma//
\glb \mk{1s} \Neg{} \Loc{} house-\Acc{}//
\glft \trsl{I'm not at home (i.e., I'm somewhere else).}//
\endgl
\xe

The particles \ird{ješ} and \ird{niho} generally proceeds the noun or noun
phrase whose existence is being posited, but in the case of a modified noun or
noun phrase, the existential particle appears before all modifiers. On the other
hand, numerals or indefinite quantifiers appear before the existential particle.

\pex
\begingl
\gla Na duma men ješ mulaž.//
\glb \Loc{} house-\Acc{} two \Exst{} door//
\glft \trsl{There are two doors.}//
\endgl
\xe

\pex
\begingl
\gla Na ránema hroná ješ matematickí tóm.//
\glb \Loc{} desk-\Acc{} three \Exst{} mathematics book//
\glft \trsl{There are three mathematics books on my desk.}//
\endgl
\xe




\subsubsection{Possession}
Existential constructions are also used to indicate possession, with the
possessor marked in the patientive case.

\begin{multicols}{2}
\pex
  \begingl
    \gla Marka ješ oblašc.//
    \glb Marek-\Acc{} \Exst{} pet//
    \glft \trsl{Marek has a pet.}//
  \endgl
\xe
\pex
  \begingl
    \gla Tomáša niho mlaz.//
    \glb Tomáš-\Acc{} \N{}\Exst{} brother//
    \glft \trsl{Tomáš does not have a brother.}//
  \endgl
\xe
\end{multicols}

\subsubsection{Impersonal constructions}\index{impersonal construction}

Iridian prefers using existential constructions where English\index{English} and
other Indo-European languages would use indefinite pronouns. More formally,
sentences of this type are called impersonal constructions.\footnote{See, for
example, \textcite{lawtagalog} where the discussion in this section is largely
based.} In general an impersonal construction in Iridian is produced by
nominalizing\index{nominalization} a verb phrase which would otherwsise have
been the predicate of an indefinite pronoun. We can illustrate this in English
as follows:

\pex
\a  \deftagex{impeng}\deftaglabel{ind}Sentence with an indefinite pronoun as subject:\\
    \emph{Somebody} told me to come here to pick up the dress.
\a  Impersonal construction:\\
    \ljudge{?}\emph{There is somebody} who told me to come here to pick up the dress.
\xe

Sentences of the first type do not exist in Iridian. Instead sentences with an
indefinite element (not necessarily the subject of the sentence) are reframed as
existential constructions. To further illustrate the primacy of impersonal
constructions over indefinite pronouns in Iridian, we can replace the subject of
(\getfullref{impeng.ind}) with a definite noun:

\pex
\a\begingl
    \gla Tak muž nedvačernilá te Tereza ziček.//
    \glb here dress \Caus{}-get-\Pv{}-\Subj{}.\Ipf{} so:that Tereza say-\Av{}-\Pf{}//
    \glft \trsl{Tereza told me to come here to pick up the dress.}//
  \endgl
\a\begingl
    \gla Do ješ tak muž nedvačernilá te zičkou.//
    \glb \First{}\Sg{}.\Acc{} \Exst{} here dress \Caus{}-get-\Pv{}-\Subj{}.\Ipf{} so:that say-\Av{}-\Pf{}-\Nz{}//
    \glft \trsl{Somebody told me to come here to pick up the dress.} (\emph{Lit.,} I have someone who said (I) should come pick up the dress.)//
  \endgl
\xe


\pex
\begingl
\gla Martina ješ trešnikou na tropa.//
\glb Martin-\Acc{} \Exst{} write-\mk{pv-pf-nz} \Loc{} wall-\Acc{}//
\glft \trsl{Martin wrote something on the wall.}//
\endgl
\xe

\pex
\begingl
\gla Voštnikouva ža ješ piaščkou?//
\glb cook-\mk{pv-pf-nz-pat} already \Exst{} eat-\mk{av-pf-nz}//
\glft \trsl{Did somebody eat what (I) cooked?}//
\endgl
\xe

\subsection{Copular constructions}
\subsubsection{Null copula}

Copular sentences are a minor sentence type where the predicate is not a verb.
For the purposes of this grammar, we narrow down our definition of copular
constructions to the following:
\pex
\a \textit{Equative:} Marek is the doctor (we are talking about).
\a \textit{Inclusive:} Marek is a doctor.
\a \textit{Attributive:} Marek is tall.
\a \textit{Locative:} Marek is in the hospital.
\xe

Iridian does not make a distinction between equative, inclusive and attributive
clauses. Locative clauses on the other hand, may be expressed using a copular or
an existential construction, as will be discussed in this section.

Iridian is a superficially a zero-copula language and the most common way to
form copular sentences is mere juxtaposition.

\pex<cop>
\begingl
\gla Marek doktor.//
\glb Marek doctor//
\glft \trsl{Marek (is a/the) doctor.}//
\endgl
\xe

The above example could either be taken to mean (1) Marek is a doctor
(inclusive), or (2) Marek is the doctor (equative). Generally, though, Iridian
uses word order to distinguish between equative and inclusive clauses.

\pex
\a \textit{Inclusive:} \{item in class\}\tss{N} $\varnothing$ \{class\}\tss{P}
\a \textit{Equative:} \{class\}\tss{N} $\varnothing$ \{item class\}\tss{P}
\xe

To avoid ambiguity, Example \getref{cop} can be reformulated to either of the
following sentences:

\pex<cop1>
\a
\begingl
\gla Marek doktor.//
\glb Marek doctor//
\glft \trsl{Marek is a doctor.}//
\endgl

\a
\begingl
\gla Doktor Marek.//
\glb doctor Marek//
\glft \trsl{Marek is the doctor.}//
\endgl

\xe

The inversion of word order is not strongly grammaticalized with NP-NP
sentences, i.e., both sentences in Example \getref{cop1} can still be used
interchangeably without a change in meaning and preference is given on the one
over the other when there is an ambiguity. This is not the case with attributive
clauses, i.e., sentences with adjective or adjective phrase predicates. Consider
for example the sentence below:

\pex
\begingl
\gla Marek rázym.//
\glb Marek tall//
\glft \trsl{Marek is tall.}//
\endgl
\xe

Inverting the word order of the sentence above would change the adjective to a
substantive since modifiers cannot occupy the topic position.

\pex
\begingl
\gla Rázym Marek.//
\glb tall Marek//
\glft \trsl{The tall one is Marek.}//
\endgl
\xe

Iridian also distinguishes between attributive clauses expressing permanent
conditions and clauses expressing temporary conditions, with the latter being
expressed using existential constructions in certain adjectives.

\pex
\begingl
\gla *Marek morec.//
\glb Marek hungry//
\glft \trsl{Marek is hungry}//
\endgl
\xe


\pex
\begingl
\gla Marka ješ morec.//
\glb Marek-\Acc{} \Exst{} hunger//
\glft \trsl{Marek is hungry}//
\endgl
\xe

A full list of adjectives/modifiers that use the existential construction can be
found in the section~\ref{sec:exst}.

The copula, however, cannot be ommitted in grammatical moods other than the
indicative.

\subsubsection{Negative copula}

Iridian has the negative copula \ird{česná}.

\pex
\begingl
\gla Marek doktor česná.//
\glb Marek doctor \mk{cop.neg}//
\glft \trsl{Marek is not (a/the) doctor.}//
\endgl
\xe

\par The inversion of word order may also be used when one wants to avoid
ambiguity:

\pex
\begingl
\gla Doktor Marek česná.//
\glb doctor Marek \mk{cop.neg}//
\glft \trsl{Marek is not the doctor.}//
\endgl
\xe


\subsubsection{Conjugation paradigm}
