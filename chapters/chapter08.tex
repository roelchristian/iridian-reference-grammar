\chapter{Semantics and Usage}

\section{Register}
\section{Forms of Address and Treatment}\index{forms of address}

\subsection{Terms of courtesy and honorifics}\index{honorifics}\index{terms of courtesy|see{honorifics}}\index{courtesy|see{honorifics}}

A {\sc honorific} is a form of address used to indicate respect or courtesy. The most common honorifics in Iridian are the masculine \ird{Staj} equivalent to the English \trsl{Sir} and the feminine \ird{Nau} equivalent to the English \trsl{Madame/Ma'am.} When addressing a person of an unknown gender\index{gender}, the term \irdp{Obečne}{mercy/grace} is used.

When adderessing or referring to multiple individuals the term \ird{maše} (originally meaning \trsl{crowd}) is exclusively used as a honorific

\subsection{Salutations}\index{salutation}\index{written correspondence}

The general salutation in most formal correspondence uses the honorific \irdp{Staj}{Sir} or \irdp{Nau}{Madame}. The last name of the addressee may also follow, although more often than not, the simple honorific should suffice. When addressing a collegiate entity or a collection of people, the term \irdp{Maše}{crowd} or \irdp{Prehodašce maše}{Esteemed/praiseworthy crowd} is used instead.

If the addressee holds a specific title, the title is included in the salutation. In some cases, the wife of the title-holder may be addressed using \ird{Nau} followed by the title, although this practice is slowly falling out of use, except in most diplomatic correspondence, where it is still considered standard. Below are some examples:


\begin{itemize}[nosep]
	\item \irdp{Staj/Nau Prezident}{Mister/Madame President}
	\item \irdp{Staj/Nau Brac}{Mister/Madame Member of the Parliament}
	\item \irdp{Staj/Nau Kanclár}{Mister/Madame Chancellor}
	\item \irdp{Staj/Nau Holva}{Mister/Madame Chairman/Chairwoman}
	\item \irdp{Staj/Nau Provízor}{Mister/Madame Professor}
\end{itemize}

Where the addressees are multiple individuals who hold specific titles, the honorific \ird{Staj} or \ird{Nau} is replaced with \irdp{prehodašce}{esteemed, praiseworthy}. When used this way, the title is normally not capitalized. Note also that \ird{prehodašce} will only be used in a salutation when there are multiple addressees.

\begin{itemize}[nosep]
	\item \irdp{Prehodašce brac}{Esteemed members of the Parliament}
	\item \irdp{Prehodašce provízor}{Esteemed members of the faculty}
\end{itemize}

When the addressee is a medical doctor, the salutation \irdp{Doktor}{doctor} is used. When writing to members of the clergy, it is customary to use \irdp{Pápka}{My father} or \irdp{Mlazka}{My brother.}

It is considered rude to use a person's first name by itself in the salutation. A more common way is to add the suffix \irdp{-óm}{our} or \irdp{-(e)m}{my} to the name or the diminutive form of the name. Alternatively the terms \irdp{kamarád}{colleague, comrade} or \irdp{naž}{friend} or their diminutives may also be used. This approach is particularly common in e-mail correspondence between work colleagues.

Specific examples of written correspondence in Iridian can be found in \S\,\ref{sec:writcorr}.

\subsection{Valedictions}\index{valediction}\index{written correspondence}

Standard valedictions used in formal written correspondence\index{correspondence|see{written correspondence}} in Iridian tend to be more complex than the ones used in English. Below is 

\begin{itemize}[nosep]
	\item \irdp{(Staj/Nau) oblostnení mavac/respekt akceptirniká}{Sir/Madame, please accept my sincerest regards (\emph{lit.}, wishes)/respect.}
	\item \irdp{Dá zespodení/spietnení pokár biležit}{I will remain your most humble/loyal servant.}
	\item \irdp{Dá zespodení/spietnení byl biležit}{I will remain your most humble child.}\footnote{This is often used among religious people when writing to members of the clergy.}
	\item \irdp{Oblostnení mavacu/respektu še hroznik.}{With the sincerest regards/respect has this letter been sent.}

\end{itemize}

Increasingly, especially in e-mail\index{e-mails} correspondence, it has become more common to use the following valedictions instead:

\begin{itemize}[nosep]
	\item \irdp{Mavac/\v{S}e mavacu}{Regards/with wishes/regards.}
	\item \irdp{Oblostnení}{Most sincere}
\end{itemize}

In more informal situations, such as between close friends and family, the following are used

\begin{itemize}[nosep]
	\item \irdp{Dá}{I/me}
	\item \irdp{Bes/Mach bes/Nic bes}{Hug/Two hundred hugs/A thousand hugs}
	\item \irdp{Beska/Mach beska/Nic beska}{Little hug/Two hundred little hugs/A thousand little hugs}
	\item \irdp{\v{S}e hloubu/Hloubževí}{With love/Loving}
	\item \irdp{\v{Z}už/Mach žuž/Nic žuž}{Kiss/Two hundred kisses/A thousand kisses}
\end{itemize}

As mentioned earlier, specific examples of written correspondence in Iridian can be found in \S\,\ref{sec:writcorr}.


\section{Idiomatic Expressions}\index{idiomatic expressions}

\section{Punctuation}