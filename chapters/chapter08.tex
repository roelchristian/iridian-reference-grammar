\chapter{Semantics and Usage}

\section{Register}
\section{Forms of Address and Treatment}\index{forms of address}

\subsection{Terms of courtesy}

The term \irdp{ma\v{s}e} is exclusively used as a honorific

\subsection{Salutations}\index{salutation}

The general salutation in most formal correspondence uses the honorific \irdp{Staj}{Sir} or \irdp{Nau}{Madame}. The last name of the addressee may also follow, although more often than not, the simple honorific should suffice. When addressing a collegiate entity or a collection of people, the term \irdp{Ma\v{s}e}{crowd} or \irdp{Prehoda\v{s}ce ma\v{s}e}{Esteemed/praiseworthy crowd} is used instead.

If the addressee holds a specific title, the title is included in the salutation. In some cases, the wife of the title-holder may be addressed using \ird{Nau} followed by the title, although this practice is slowly falling out of use, except in most diplomatic correspondence, where it is still considered standard. Below are some examples:


\begin{itemize}[nosep]
	\item \irdp{Staj/Nau Prezident}{Mister/Madame President}
	\item \irdp{Staj/Nau Brac}{Mister/Madame Member of the Parliament}
	\item \irdp{Staj/Nau Kancl\'ar}{Mister/Madame Chancellor}
	\item \irdp{Staj/Nau Holva}{Mister/Madame Chairman/Chairwoman}
	\item \irdp{Staj/Nau Prov\'izor}{Mister/Madame Professor}
\end{itemize}

Where the addressees are multiple individuals who hold specific titles, the honorific \ird{Staj} or \ird{Nau} is replaced with \irdp{prehoda\v{s}ce}{esteemed, praiseworthy}. When used this way, the title is normally not capitalized. Note also that \ird{prehoda\v{s}ce} will only be used in a salutation when there are multiple addressees.

\begin{itemize}[nosep]
	\item \irdp{Staj/Nau brac}{Esteemed members of the Parliament}
	\item \irdp{Prehoda\v{s}ce prov\'izor}{Esteemed members of the faculty}
\end{itemize}

\subsection{Valedictions}\index{valediction}

Standard valedictions used in formal written correspondence\index{correspondence|see{written correspondence}} in Iridian tend to be more complex than the ones used in English. Below is 

\begin{itemize}[nosep]
	\item \irdp{(Staj/Nau) oblostnen\'i mavac/respekt akceptirnik\'a}{Sir/Madame, please accept my sincerest regards (\emph{lit.}, wishes)/respect.}
	\item \irdp{D\'a zespoden\'i/spietnen\'i pok\'ar bile\v{z}it}{I will remain your most humble/loyal servant.}
	\item \irdp{D\'a zespoden\'i/spietnen\'i byl bile\v{z}it}{I will remain your most humble child.}\footnote{This is often used among religious people when writing to members of the clergy.}
	\item \irdp{Oblostnen\'i mavacu/respektu \v{s}e hroznik.}{With the sincerest regards/respect has this letter been sent.}

\end{itemize}

Increasingly, especially in e-mail\index{e-mails} correspondence, it has become more common to use the following valedictions instead:

\begin{itemize}[nosep]
	\item \irdp{Mavac/\v{S}e mavacu}{Regards/with wishes/regards.}
	\item \irdp{Oblostnen\'i}{Most sincere}
\end{itemize}

In more informal situations, such as between close friends and family, the following are used

\begin{itemize}[nosep]
	\item \irdp{D\'a}{I/me}
	\item \irdp{Bes/Mach bes/Nic bes}{Hug/Two hundred hugs/A thousand hugs}
	\item \irdp{Beska/Mach beska/Nic beska}{Little hug/Two hundred little hugs/A thousand little hugs}
	\item \irdp{\v{S}e hloubu/Hloub\v{z}ev\'i}{With love/Loving}
	\item \irdp{\v{Z}u\v{z}/Mach \v{z}u\v{z}/Nic \v{z}u\v{z}}{Kiss/Two hundred kisses/A thousand kisses}
\end{itemize}

\section{Idiomatic Expressions}\index{idiomatic expressions}

\section{Punctuation}