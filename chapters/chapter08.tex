\chapter{Semantics and Usage}

\section{Register}
\section{Forms of Address and Treatment}\index{forms of address}

\subsection{Terms of courtesy and honorifics}\index{honorific}\index{terms of courtesy|see{honorific}}\index{courtesy|see{honorific}}

A {\sc honorific} is a form of address used to indicate respect or courtesy. The most common honorifics in
Iridian are the masculine \ird{St\'am} equivalent to the English\index{English} \trsl{Sir} and the feminine
\ird{Nau} equivalent to the English \trsl{Madame/Ma'am.} When addressing a person of an unknown
gender\index{gender}, the term \irdp{Obečne}{mercy/grace} is used.

Both \ird{St\'am} and \ird{Nau} may be followed by the addressee's last name. They should never be used with
the first name as it would be considered sarcastic or rude. If the name of the person being addressed is not
known, the placeholders \irdp{vieda}{man} and \irdp{huzak}{woman} are used, thereby producing \ird{St\'am Vieda}
and \ird{Nau Huzak}. When writing, these are often abbreviated to {\sc s.v.} and {\sc n.h.}, respectively.
The usage of \ird{St\'am Vieda} and \ird{Nau Huzak} is similar to how the third person may sometimes be used
in English\index{English} to politely address someone (e.g., saying, \trsl{Will the gentleman yield?}) but while
it may sometimes appear dated or overly formal in English\index{English}, this practice is still commonly observed
in Iridian, especially when addressing strangers.

Other common titles include \ird{Doktor} used when addressing physicians, \ird{Majestet} or \ird{Kopi\v{z}n\'ast}
when addressing members of the royalty (with the latter reserved for reigning monarchs), \ird{Eselenc} when
addressing certain high-ranking officials such as senators, governors, and ambassadors, \ird{Eminenc} when addressing
cardinals of the Catholic Church, \ird{Obe\v{c}ne} or \ird{Prac} when addressing judges and magistrates, and
\ird{Tieho\v{z}n\'ast} or \ird{Hilda\v{z}n\'ast} or \ird{Hilden\'i T\'a\v{t}}\footnote{This form of address, meaning
\trsl{Holy Father} or more commonly its abbreviation {\sc h.t.}, is used in writing when referring to the Pope in the
third person.} when addressing the Pope or the religious leaders from other traditions.

When addressing or referring to multiple individuals the term \ird{maše} (originally meaning \trsl{crowd}
but now exclusively employed as a honorific) is used. This is often preceded, both in the written and spoken forms,
by the non-nominal supine\index{supine} \irdp{prehoda\v{s}ce}{esteemed/praiseworthy.}

\subsection{Salutations}\index{salutation}\index{written correspondence}

The general salutation in most formal correspondence uses the honorific \irdp{St\'am}{Sir} or \irdp{Nau}{Madame}.
The last name of the addressee may also follow, although more often than not, the simple honorific\index{honorific} 
should suffice. When addressing a collegiate entity or a collection of people, the term \irdp{Maše}{crowd} or
\irdp{Prehodašce maše}{Esteemed/praiseworthy crowd} is used instead.

If the addressee holds a specific title, the title is included in the salutation. In some cases, the wife of the
title-holder may be addressed using \ird{Nau} followed by the title, although this practice is slowly falling out
of use, except in most diplomatic correspondence, where it is still considered standard. Below are some examples:


\begin{itemize}[nosep]
	\item \irdp{St\'am/Nau Prezident}{Mister/Madame President}
	\item \irdp{St\'am/Nau Brac}{Mister/Madame Member of the Parliament}
	\item \irdp{St\'am/Nau Kanclár}{Mister/Madame Chancellor}
	\item \irdp{St\'am/Nau Holva}{Mister/Madame Chairman/Chairwoman}
	\item \irdp{St\'am/Nau Provízor}{Mister/Madame Professor}
\end{itemize}

Where the addressees are multiple individuals who hold specific titles, the honorific \ird{St\'am} or \ird{Nau} is replaced with \irdp{prehodašce}{esteemed, praiseworthy}. When used this way, the title is normally not capitalized. Note also that \ird{prehodašce} will only be used in a salutation when there are multiple addressees.

\begin{itemize}[nosep]
	\item \irdp{Prehodašce brac}{Esteemed members of the Parliament}
	\item \irdp{Prehodašce provízor}{Esteemed members of the faculty}
\end{itemize}

When the addressee is a medical doctor, the salutation \irdp{Doktor}{doctor} is used. When writing to members of the clergy, it is customary to use \irdp{Pápka}{My father} or \irdp{Mlazka}{My brother.}

It is considered rude to use a person's first name by itself in the salutation. A more common way is to add the suffix \irdp{-óm}{our} or \irdp{-(e)m}{my} to the name or the diminutive form of the name. Alternatively the terms \irdp{kamarád}{colleague, comrade} or \irdp{naž}{friend} or their diminutives may also be used. This approach is particularly common in e-mail correspondence between work colleagues.

Specific examples of written correspondence in Iridian can be found in \S\,\ref{sec:writcorr}.

\subsection{Valedictions}\index{valediction}\index{written correspondence}

Standard valedictions used in formal written correspondence\index{correspondence|see{written correspondence}} in Iridian tend to be more complex than the ones used in English. Below is 

\begin{itemize}[nosep]
	\item \irdp{(St\'am/Nau) oblostnení mavac/respekt akceptirniká}{Sir/Madame, please accept my sincerest regards (\emph{lit.}, wishes)/respect.}
	\item \irdp{Dá zespodení/spietnení pokár\'i biležit}{I will remain your most humble/loyal servant.}
	\item \irdp{Dá zespodení/spietnení byl\'i biležit}{I will remain your most humble child.}\footnote{This is often used among religious people when writing to members of the clergy.}
	\item \irdp{Oblostnení mavacu/respektu še hroznik.}{With the sincerest regards/respect has this letter been sent.}

\end{itemize}

Increasingly, especially in e-mail\index{e-mails} correspondence, it has become more common to use the following valedictions instead:

\begin{itemize}[nosep]
	\item \irdp{Mavac/\v{S}e mavacu}{Regards/with wishes/regards.}
	\item \irdp{Oblostnení}{Most sincere}
\end{itemize}

In more informal situations, such as between close friends and family, the following are used:

\begin{itemize}[nosep]
	\item \irdp{Dá}{I/me}
	\item \irdp{Bes/Mach bes/Nic bes}{Hug/Two hundred hugs/A thousand hugs}
	\item \irdp{Beska/Mach beska/Nic beska}{Little hug/Two hundred little hugs/A thousand little hugs}
	\item \irdp{\v{S}e hloubu/Hloubževí}{With love/Loving}
	\item \irdp{\v{Z}už/Mach žuž/Nic žuž}{Kiss/Two hundred kisses/A thousand kisses}
\end{itemize}

As mentioned earlier, specific examples of written correspondence in Iridian can be found in \S\,\ref{sec:writcorr}.


\section{Phatic Expressions and Social Formulas}

\section{Idiomatic Expressions}\index{idiomatic expressions}

\section{Punctuation}