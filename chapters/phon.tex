\chapter{Phonology}\label{ch:phon}

\section{Introduction}

This chapter provides an overview of the phonology of Iridian. The phonetic descriptions provided here are in IPA based on the standard dialect of Iridian (as spoken in Roubže and surrounding areas),

\section{Vowels}\index{vowel}\label{sec:vowels}
\subsection{Oral vowels}\index{vowel!oral}

Iridian has five pairs of corresponding long and short vowels. With the exception of /a\,aː/, long vowels are tenser than their short counterparts. In addition standard Iridian also features the high central vowel [ɨ] as an allophone of /ɛ/ and /ɪ/ and the low central [ɐ] as an allophone of /a/, in unstressed positions. As in Czech, 

\begin{table}[ht!]\index{vowel!inventory}
	\small
	\caption{Vowel inventory of standard Iridian.}
	\medskip
	\begin{tabularx}{0.7\textwidth}{YMMM}
		\toprule
					& {\sc front}		& {\sc central}			& {\sc back}	\\\midrule
		Close 		& ɪ\,i 				& (ɨ)					& ʊ\,uː			\\
		Mid 		& ɛ\,eː 			& 						& ɔ\,oː			\\
		Open 		& 					&(ɐ)					& a\,aː 		\\
		\bottomrule
		\label{table:vowels}
	\end{tabularx}
\end{table}

Phonetic realisation is generally consistent with orthography as seen in Table \ref{table:vowels-orth} 
below. There are a few observations worth nothing, nevertheless.

\begin{table}[ht!]
	\small
	\caption{Orthographic representation of vowels.}
	\medskip
	\begin{tabularx}{0.7\textwidth}{YYYYYY}
		\toprule
		& {\sc short} & {\sc long} & & {\sc short} & {\sc long}\\
		\midrule
		/a/ & a 	&á 			& /o/ 	& o &ó \\
		/e/ & e 	&é 			& /u/ 	& u &ú\\
		/i/ & i,\,y &í,\,\'y 	& 		& 	&\\
		\bottomrule
		\label{table:vowels-orth}
	\end{tabularx}
\end{table}

Both ⟨i⟩ and ⟨y⟩ and their long counterparts ⟨í⟩ and ⟨ý⟩ represent the high front vowel /i/. ⟨y/ý⟩ originally represented the high front rounded vowel /y/ (with the short /y/ realised as the tenser near-close near-front rounded vowel [ʏ]) but the pronunciation gradually shifted to the central front vowel [ɨ] before finally settling to /i/ in the 14th or 15th century. As in Czech\index{Czech} orthography, ⟨i,/\'i⟩ causes the palatalisation of the preceding consonant. The same distinction is found between the palatalising ⟨ě⟩ (another Czech loan originally written in Old Iridian as ⟨je⟩) and the normal ⟨e⟩. This is discussed further in the orthography section (\S\,\ref{sec:ortho}).

The short vowels /ɛ/ and /ɪ/ are reduced to [ɨ] in unstressed positions. In less careful speech, this could cause the elision of the vowel and the formation of consonant clusters or the realisation of the preceding consonant as syllabic (especially if it is a liquid). Final /ɛ/ is not reduced in a word-final position if preceding a pause.

\ex
	\irdp{a mert}{and the dead one} [ˈʔämɨɾt̚ ] or [ˈʔämɾ̩t̚] but\\
	\irdp{akuzace}{accusation} [ˈʔäxʊzɐt͡sɛ]
\xe

The low vowel /a/ is realised as the open central unrounded vowel /\"a/. Stressed /a/ is realised as [\ae] between palatal consonants, further reduced to [ɨ] when unstressed, e.g., \ird{piaštá} ['pʲæɕtäː] vs. \ird{nepiaštá} [ˈnɛpʲɨɕtäː]. Elsewhere /a/ is pronounced [ɐ] when in an unstressed position, although some dialects may further reduce it to a [ə].

\subsection{Diphthongs}\index{diphthong}
Iridian has three phonemic oral diphthongs: \ird{au}\,/au̯/, \ird{ei}\,/eɪ̯/ and \ird{ou}\,/ou̯/. In addition, the diphthongs \ird{oi}\,/ɔɪ̯/ and \ird{ui}\,/uɪ̯/  also occur phonetically, but their occurence is marginal, normally appearing only in fixed expressions (mostly interjections and expletives), such as \irdp{Avui}{Damn it!} [ʔɐˈʋuɪ̯ʔ], \irdp{pšehui}{annoying} [ˈpɕɛxuɪ̯ʔ] and \irdp{Oi}{Hey!} [ʔɔɪ̯ʔ].

In most dialects the diphthong /eɪ̯/ has almost completely merged with \ird{é} /eː/, although some divergent dialects in the south may realise the diphthong as [iː] (e.g., \irdp{neite}{word} /ˈneɪ̯tɛ/ but realised as [ˈneːtɛ] or ['ɲiːtɛ]).

\subsection{Vowel Length}\index{vowel length}\index{long vowel|see{vowel length}}

Vowel length is phonemic in Iridian. Length is represented by an acute accent\index{acute accent} over the long vowel. The short-long vowel pairs differ in quality as well as length, with the short vowels being more lax and the long vowels being tenser in addition to being longer.

\begin{table}[h!]
	\small
	\caption{Vowel length and quality.}
	\medskip
	\begin{tabu} to 0.7\textwidth{MMM}
		\toprule
		{\sc archiphoneme} & {\sc lax/short} &{\sc tense/long}\\ \midrule
		/a/	& [\"a]	& [\"aː]		\\
		/e/	& [ɛ]	& [eː]		\\
		/i/	& [ɪ]	& [iː]		\\
		/o/	& [ɔ]	& [oː]		\\
		/u/	& [ʊ] & [uː]		\\
		/y/	& [ʏ]	& [yː]		\\
		\bottomrule
	\end{tabu}
\end{table}

Sample minimal pairs featuring long and short vowels are listed below:

\pex
\a /\"a/ and /\"aː/\\
\vtop{\halign{%
#\hfil& \qquad  #\hfil\cr
\irdp{sam}{barn} [säm]			& \irdp{sám}{frog}	[säːm] \cr
\irdp{mate}{spoon} [ˈmätɛ]			& \irdp{máte}{check mate}	[ˈmäːtɛ] \cr
}}

\a /ɛ/ and /eː/	
\xe


\section{Consonants}\index{consonants}\label{sec:consonants}

Table \ref{table:fullconsonant} shows a complete list of consonant phonemes in Standard Iridian, with the allophones appearing in parentheses. In total, Iridian has 19 consonant phonemes but with 21 additional allophonic variants.
\begin{table}[h!]
	\small
	\caption{Full consonant inventory of standard Iridian.}\label{table:fullconsonant}
	\medskip
	\begin{tabu} to \textwidth{Y[2]YYYY}
		\toprule\addlinespace
											& {\sc labial}	& {\sc alveolar}		& {\sc palatal}	& {\sc velar}	\\
		\addlinespace\midrule\addlinespace
		Plosive					 	& p~b						& t~d								& c~ɟ 					& k~ɡ 		\\
		\addlinespace
		Nasal							& m~(ɱ)					& n									& ɲ							& (ŋ)			\\
		\addlinespace
		Liquid						&								& ɾ~(ʁ)~l						&	ʎ							&					\\
		\addlinespace
		Sib. Fric.				& 							& s~z	  						& ɕ~ʑ						&					\\
		\addlinespace
		Non-Sib. Fric.		& ʋ							&										& (ç) 					& x~ɣ   	\\
		\addlinespace
		Sib. Affricate    &								& t͡s~(d͡z)					& t͡ɕ~(d͡ʑ)			&				  \\
		\addlinespace
		Non-Sib. Aff. 		&								& 									&			  				& (k͡x~g͡ɣ)\\
		\addlinespace
		Approximant 			& (β̞)  				 & (ð̞)								& j				 			& (ʍ~w)		\\
		\addlinespace
		\bottomrule
	\end{tabu}
\end{table}


\subsection{Plosives}

\par Initial velar stops are affricated when following a pause, so that the pair /k~ɡ/ is often realised as [k͡x~ɡ͡ɣ]. Some Southeastern dialects, however, normally realise initial velar stops as aspirated [kʰ~ɡʰ] instead. This sound change can be traced to the initial aspirated stops \rec{\asp{k}}, \rec{\asp{g}}, \rec{\asp{t}} and \rec{\asp{d}} in Old Iridian weakening to affricates.\footnote{Old Iridian \rec{\asp{t}} and \rec{\asp{d}} became the Middle Iridian [t̪͡θ̞ ~d̪͡ð̞] but both have since simplified to /t~d/ in modern Iridian.} The labial stops /{p~b}/ are unaffected by this process as most instances of \rec{\asp{p}} and \rec{\asp{b}} have merged to /b/ or /ʋ/ in modern Iridian.

The velar stops /k~ɡ/ are lenited to the velar fricatives [x~ɣ] intervocalically, before a voiceless stop, after a vocalised l if followed by another vowel or a voiceless stop, or before the nasal consonants /n/ or /m/ if following a vowel immediately. This lenition also occurs word-finally unless followed by a voiced obstruent, in which case, subject to word-final devoicing, they merge to [x]. The voiced /ɡ/ itself has a limited distribution, mostly appearing in consonant clusters with liquids or nasals.

This lenition can also be observed with the voiced stops /b/ and /d/ which become the approximants [β̞	] and [ð̞] (written without the diacritic hereafter) intervocalically or between a vocalised /l/ and another vowel.

The glottal stop [ʔ] is often not regarded as a separate phoneme.
It can occur in three cases: (1) before an onset vowel when following a pause, e.g., \irdp{avt}{car} [ʔäft]; (2) between two vowels that do not form a diphthong, e.g., \irdp{naomá}{laundry} ['näʔɔmäː]; or (3) emphatically, especially in interjections, e.g., \irdp{Oi}{Hey!} [ʔɔɪ̯ʔ], \irdp{Káp!}{Look out!} \emph{lit.}, \trsl{danger} [k͡xäpʔ].


\subsection{Nasals}
Iridian has three nasal consonants /m~n~ɲ/. /n/ cannot appear before bilabials and similarly /m/ cannot appear before velars. Both /m/ and /n/ are realised as [m] before either /ʋ/ or /f/. Before velars /n/ is consistently realised as [ŋ], although [n] is also possible in emphatic pronunciation or in word boundaries.

The velar [ŋ] is not phonemic in Iridian but can sometimes be observed, especially in loanwords, where it can be realised as nasalisation of the preceding vowel when in the syllable coda or as [ŋ] intervocalically, although [ŋɡ] or [ŋk] is also common. Thus, for example, \irdp{anglevní}{English} can be realised as either [ˈɐ̃w̃lɛʋɲiː] or [ˈäŋlɛʋɲiː] or [ˈäŋɡlɛʋɲiː] in order of currency.


\subsection{Liquids}

Iridian has three liquids: the rhotic /r/ and the lateral /l/ and /l/.

The rhotic /r/ is realised in one of three ways. Word-initially it is pronounced as the uvular fricative [ʁ] (or as the uvular trill fricative [ʀ̝], depending on the speaker, but both transcribed here simply as [ʁ]). The realisation as [ʁ] is also often used when pronouncing words emphatically. When in the coda position and before a pause /r/ is realised as [ɾʑ] or simply as [ʑ]. This pronunciation was originally that of a voiceless alveolar trill [r̥] but this has simplified to [r̝] and finally to [ɾʑ] or [ʑ] in Standard Iridian. The  pronunciation as [r̥] or [r̝] may nevertheless persist in some southern dialects, primarily due to Czech\index{Czech} influence. Note that [ɾʑ] or [ʑ] is not affected by word-final devoicing. Elsewhere /r/ is realised as the flap [ɾ]. Palatal /rʲ/ is in general more stable, realised simply as [ɾʲ], although when in the coda position and if not followed by a vowel, it may be realised as [ɾʑ] or [ʑ].

The lateral /l/ is actually the velarised alveolar lateral approximant [ɫ]. Nonetheless the sound has been transcribed throughout as /l/. In the coda position /l/ is completely vocalised and is transcribed here as [w] in standard Iridian; most southern dialects nevertheless retain the pronunciation as [ɫ]. The palatalised /lʲ/ is the palatal lateral approximant [ʎ] and is transcribed as such.

\subsection{Fricatives and Affricates}

The palatal sibilants /ɕ~ʑ/ can be realised as either the palatal [ɕ~ʑ] or
the post-alveolar [ʃ~ʒ] with the former being more common. The same is true
with the palatal affricates /t͡ɕ~d͡ʑ/, realised as either [t͡ɕ~d͡ʑ] or [t͡ʃ~d͡ʒ],
with the former also being more prevalent.

The sequence /t͡sɪ/ and /t͡si:/ are realised as [t͡ɕɪ] and [t͡ɕiː] respectively
(viz., \irdp{cigra}{tiger} is realised as [ˈt͡ɕɪɣɾɐ] and not [ˈt͡sɪɣɾɐ]).
The stop fricative sequence [tɕ] can occur in syllable boundaries,
although as form of hypercorrection most speaker may lengthen the initial
stop to [tːɕ] or aspirate it (becoming [tʰ.ɕ]) to further distinguish it
from /t͡ɕ/ (cf. e.g., \irdp{otša}{cart} [ˈʔɔtːɕɐ] vs \irdp{oča}{bear}
[ˈʔɔt͡ɕɐ]).

The voiced affricates /d͡z/ and /d͡ʑ/, written \orth{dz} and \orth{dž}, respectively, are both marginal phonemes. They normally occur as voiced allophones of  /t͡s/ and /t͡ɕ/ before voiced obstruents. They do occur phonemically in a few words, though, mostly in loanwords. Nonetheless, in spoken Iridian loanwords containing [d͡ʑ] or [d͡ʒ] (mostly from English) are realised by speakers as [ʑ] (e.g., \irdp{džíns}{jeans} [dʑiːns] or more commonly just [ʑiːns]).


The voiceless labial fricative /f/ is another marginal phoneme, appearing usually as an allophobe of /ʋ/. Loanwords containing /f/ generally assimilate to /ʋ/, although most recent borrowings tend to keep the marginal /f/ (cf. \irdp{Vranca}{France} [vɾɐ̃w̃t͡sɐ] vs. \irdp{Feizbuk}{Facebook} [feːzbʊx]).

The approximant /ʋ/ is realised as [v] in onsets before vowels and voiced obstruents (e.g., \irdp{vdinice}{I thought I saw.} [ˈvɟɪnɨt͡sɛ]), as [f] in onsets before voiceless obstruents (e.g., \irdp{vternou}{bicycle} [ˈftɛɾnou̯]), and as [ʋ] or [u̯] in coda and elsewhere (e.g., \irdp{pilav}{pilaf} [ˈpʲɪɫäʋ] or [ˈpʲɪɫäu̯]). The sequence /kʋ/ and /ɡʋ/ is further lenited to the labialised velar fricatives [xʷ~ɣʷ]. The voiceless [xʷ] (from both \orth{kv} and \orth{hv}) is in free variation with [ʍ], with the latter being the more common pronunciation, especially among younger speakers. For simplicity both [xʷ] and [ʍ] will be transcribed as [ʍ].

Modern Iridian has lost the distinction between /h/ and /x/, with both $\langle$ch$\rangle$ and $\langle$h$\rangle$,\footnote{Most instances of $\langle$ch$\rangle$ have been replaced with $\langle$h$\rangle$ following various spelling reforms.} historically representing /x/ and /h/, respectively, merging to the velar fricative /x/. This becomes /ç/ before voiceless stops word-initially or when following a front vowel, or before the front vowels /i/ and /ɪ/. The sequence $\langle$hl$\rangle$ and $\langle$kl$\rangle$ are realised as /t͡ɬ/.


\section{Phonotactics}\index{phonotactics}\label{sec:phonotactics}

\subsection{Syllable structure}\index{syllable structure}\label{sec:syllable-structure}

Ignoring the possible complexity of the onset, nucleus or coda, the basic structure of an Iridian syllable is CV(C), with C representing a consonant and V a vowel.\footnote{An alternative view, founded upon the status of the glottal stop as a non-phoneme in Iridian, would be to consider the basic structure as (C)V(C) instead of CV(C), thus allowing for a null onset. This treats the addition of a glottal stop in word-initial syllables starting with a vowel as mere prothesis.} Iridian has relatively few phonotactic constraints, allowing, at a maximum, syllables of the form CCCCVCCC. Nevertheless, most syllables fall in either of the five groups CV, CVC, CCV, CCVC and CVCC

\begin{table}[h!]
	\small
	\caption{Blevin's criteria as they apply to Iridian.}
	\medskip
	\begin{tabularx}{0.6\textwidth}{YM}
		\toprule
		& {\sc parameter}\\
		\midrule
		Obligatory onset & Yes\\
		Coda & No\\
		Complex onset & Yes\\
		Complex nucleus & Yes*\\
		Complex coda & Yes\\
		Edge effect & \\
		\bottomrule
	\end{tabularx}
\end{table}


\subsection{Onset}

\par All consonant and vowel phonemes can appear in a syllable's onset. Iridian does not allow a null onset (vowel in the syllable onset), i.e., the most basic Iridian syllable should be of the form CV. Words that superficially appear as having a null onset syllable in the initial position are actually preceded by a glottal stop. An epenthetic glottal stop is also added between vowels in a sequence that do not otherwise form dipthongs, or before a vowel in a word-initial position in loanwords. Despite this, vowel-words are significantly rarer in comparison to consonant-initial ones.

\ex
Prothetic [ʔ] in native Iridian words:\\
\irdp{a}{and} [ˈʔä]\\
\irdp{umielá}{to get drunk} [ˈʔʊmʲɨläː]\\
\irdp{eg}{eyes} [ʔɛx]
\xe

\ex
Prothetic [ʔ] in loanwords:\\
\irdp{Americe}{Amerika} [ˈʔämɨɾʲɪt͡sɛ]\\
\irdp{autobus}{bus} [ˈʔau̯tɔβʊs] \\
\irdp{elefant}{elephant} [ˈɛlɨˌfänt]
\xe

In some eastern dialects, a prothetic [m] is added instead of [ʔ] on words that begin with vowels after a pause. This never occurs on loanwords or before the front vowels /e/ and /i/ and has been largely in decline, especially among younger speakers. With some speakers, the prothetic [m] may be realised as [mw].

\ex
\irdp{umielá}{to get drunk} [ˈmʊmʲɨläː] or [ˈmwʊmʲɨläː]\\
\irdp{očat}{bug} [ˈmɔt͡ɕɐt] or [ˈmwɔt͡ɕɐt]
\xe

A more widespread pattern in colloquial Iridian is the addition of a prothetic /j/ before the front vowels /e/ and /i/. This phenomenon could be observed in both native words and loans.

\ex
\irdp{Evrope}{Europe} [ʔɛʋɾɔpɛ], colloq. [jɛʋɾɔpɛ] \\
\irdp{eg}{eyes} [ʔɛx], colloq. [jɛx]\\
\irdp{éšte}{of course} [ˈʔeːɕtɛ], colloq. [ˈjeːɕtɛ]
\xe


\begin{table}[h!]
	\small \centering
	\caption{Allowed word-initial CC clusters}
	\begin{tabularx}{\textwidth}{YMMMMMMMMMMMMMMMMMMMM}
		\toprule
		&p&b&t&d&k&g&m&n&r&l&s&z&š&ž&v&č&dc&c&dz&h\\
		\midrule
		p&&&+&&&&&+&+&+&+&&+&&&&&&&\\
		b&&&&&&&&&+&+&&&&&&&&&&\\
		t&&&&&&&+&&+&+&&&&&+&&&&&\\
		d&&&&&&&+&+&+&+&&&&&+&&&&&\\
		k&&&+&+&&&&+&+&+&+&&+&&+&&&&&\\
		g&&&&&&&&+&+&+&&&&&+&&&&&\\
		m&&&&&&&&+&&&&&&&&&&&&\\
		n&&&&&&&&&&+&&&&&&&&&&\\
		r&&&&&&&&&&&&&&&&&&&&\\
		l&&&&&&&&&&&&&&&&&&&&\\
		s&&&&&&&&&&&&&&&&&&&&+\\
		z&&+&&+&&&+&+&+&+&&&&&+&&&&&\\
		š&+&&+&&+&&+&+&+&+&&&&&+&+&&+&&+\\
		ž&&&&&&&&&&&&&&&&&&&&\\
		v&&&+&+&+&&&+&+&+&+&&+&&&&&+&&\\
		č&&&+&&+&&&&&+&&&&&&&&&&\\
		c&&&+&&+&&&+&+&+&&&&&&&&&&+\\
		h&&&+&&&&&&+&+&&&&&+&&&&&\\
		\bottomrule

		\multicolumn{21}{l}{\footnotesize + allowed cluster}
	\end{tabularx}
\end{table}

\par The following CC clusters are allowed to be in onset position:

\pex
\a Stop followed by a liquid:\\
/pr/: \irdp{pragy}{sand} [pr\"ac]; \irdp{pramou}{petal} [ˈpɾämou̯]\\
/tr/: \irdp{trava}{bread} [ˈtɾävɐ]; \irdp{truk}{ball} [tɾʊx]\\
/kr/: \irdp{krova}{egg} [ˈkɾɔvɐ]; \irdp{kramy}{toe} [kɾämʲ]\\
/pl/: \irdp{plán}{plan} [pläːn]; \irdp{plúka}{knot} [ˈpluːxɐ]\\
/kl/: \irdp{kluk}{foot} [t͡ɬʊx]; \irdp{klúbe}{club} [ˈt͡ɬuːβɛ]\\
/br/: \irdp{bírok}{female teenager} [bʲiːɾɔx]; \irdp{bremy}{prise} [bɾɛmʲ]\\
/dr/: \\
/gr/: \irdp{grec}{flag} [ɣɾɛt͡s]; \irdp{greny}{peace} [ɣɾɛɲ]\\
/bl/: \irdp{bloht}{mud} [blɔxt̚]; \irdp{blau}{neck} [blau̯]\\
/dl/:
\xe

\section{Suprasegmentals}\index{suprasegmentals}

\subsection{Stress}\index{stress}
Iridian words generally have a single primary stress, falling on the first syllable, no matter if the word is simple (e.g., \irdp{študent}{student}), derived (e.g., \irdp{študenta}{student, pat.}) or compound (e.g., \irdp{študentrád}{dormitories}). Most loanwords follow this general pattern, although more recent borrowings, especially those referring to proper names, show a greater tendency to keep the phonology of the source language and not fully assimilate to Iridian's initial stress rule.

\pex
\a Loanwords showing assimilation to word-initial stress:\\
\phon{aristókrat}{ˈäɾɨstoːxɾɐt}{aristocrat}\\
\phon{koruna}{ˈk͡xɔɾʊnä}{crown}

\a Loanwords
\xe

Clitics\index{clitic} are not considered phonologically distinct and are treated as belonging to the same phonological word as the one after them. These include:

\begin{enumerate}[noitemsep,label=(\alph*)]
	\item Most monosyllabic and some disyllabic prepositions
	\item Most conjunctions:
	\item The pluralizing particle \ird{nie} and the negative particle \ird{zám}: 
	\item Demonstratives and the weak form of personal pronouns
\end{enumerate}

\subsection{Intonation}\index{intonation}

\section{Phonological Processes Involving Vowels}

\subsection{Vowel\,\sim\,Zero Alternations}

A vowel\,\sim\,zero alternation occurs when a vowel alternates with zero (i.e., gets deleted) in certain morphological contexts. We call this deleted vowel `unstable' (cf. \cite{siptar2000}, \cite{gussmann2007}). The most common type of vowel\,\sim\,zero alternation can be observed in stems of the type (C)VCVC containing a final short /e/ (and to a lesser extent /i/ and /o/).

\ex
Janek --- Janka
\xe

\subsection{Vowel\,\sim\,Vowel Alternations}
Vowel\,\sim\,vowel alternations (also called `ablaut') occurs when one vowel is substituted for another in some morphophonological contexts. Vowel\,\sim\,vowel alternations in Iridian can be broadly classified into two types: [ɛ] substitution and vowel raising.

Roots of the type --C\sx{j}aC(C) and --C\sx{j}oC(C) become --C\sx{j}eC(C) in the presence of palatalizing suffixes:

\ex
\vtop{\halign{%
#\hfil& &\qquad  #\hfil\cr
\irdp{bial}{money}			& \irdp{bielí}{gen.}
							& \irdp{biala}{pat.}\cr
\irdp{šviak}{soldier}	& \irdp{šviecí}{gen.}
							& \irdp{šviaka}{pat.}\cr
\irdp{pion}{nest}			& \irdp{piení}{gen.}
							& \irdp{piona}{pat.}\cr
\irdp{kážol}{threat}  & \irdp{káželí}{gen.}
							& \irdp{kážola}{pat.}\cr
}}
\xe

Vowel-raising alternations can be further grouped into two: (1) those triggered by the deletion of an unstable vowel in the final syllable of the root and (2) those caused by an open coda being closed off by the addition of a suffix. The front vowels [eː], [eɪ̯] and [ʲɛ] are subject to both types of alternations, merging with the high front vowel [iː]. As for back vowels, [ɔ\,\sim\,ʊ] is an example of the first type while [ou̯\,\sim\,oː] of the latter.

\pex
\a Vowel-raising triggered by deletion of an unstable vowel in the root:\\
\vtop{\halign{%
#\hfil& &\qquad  #\hfil\cr
\irdp{lobek}{apple}		& \irdp{lubka}{pat.} 			& not \ird{*lobka}\cr
\irdp{kostel}{fish}		& \irdp{kustlár}{fisherman}	& not \ird{*kostlár}\cr
\irdp{pieštel}{falcon}	& \irdp{píštlár}{falconer}	& not \ird{*pieštlár}

\cr
}}

\xe





\subsection{Compensatory vowel lengthening}



\section{Phonological Processes Involving Consonants}


Iridian consonants are generally affected by two systems of phonological opposition: a primary distinction between voice and unvoiced consonants, and a secondary distinction between hard and soft consonants (i.e., normal and palatalised consonants).


\subsection{Voicing}
Consonant voicing is phonemic. Voiced consonants are called muddy or dark (\ird{mierkní}) while unvoiced consonants are called clear (\ird{hezkí}). Most of the obstruents in Iridian come in pairs distinguished only by voicing: /k/ \phon{kapa}{k͡xäpɐ}{cape} vs /g/ \phon{gapa}{g͡ɣäpɐ}{liquor}; /p/ \phon{pac}{pät͡s}{stick} vs /b/ \phon{bac}{bät͡s}{underside}; /t/ \phon{tám}{t\"aːm}{more} vs /d/ \phon{dám}{d\"aːm}{by me} 



Another basic rule of consonant voicing is that in a cluster the last consonant usually determines whether the preceding ones are voiced or not. Note however that although the liquids /r/ and /l/ and the nasals /m/ and /n/ are intrinsically voiced, they do not cause the preceding consonant to assimilate.

\ex
\vtop{\halign{%
#\hfil& \qquad  #\hfil\cr
\phon{nazek}{ˈnäzɛx}{powder}			& \phon{nazka}{ˈnäskɐ}{powder, pat.}\cr
}}
\xe



\subsection{Intervocalic Lenition}




\subsection{Palatalisation}
\par Iridian consonants can either be hard or soft. Consonants are hard by default but become soft when followed by the vowels \orth{i} or \orth{í}. The vowel \textbf{y} is normally used to indicate non-palatalizing /i/, although it is used to indicate palatalisation word-finally or before \textbf{i}.

\par The use of \ird{-y} is a remnant of word final short \rec{i} from Old Iridian that has since disappeared. The same process has caused the shortening of long \rec{i} to /ɪ/. This sound change did not distinguish between palatalizing and non-palatalizing \rec{i} so that \rec{seni} `tooth' and \rec{seny} `blanket' both merged to modern Iridian \ird{seny} /sɛɲ/.

\par Softening involves palatal articulation of labial consonants (e.g., \textbf{be} \textipa{[bE]} vs \textbf{bie} \textipa{[b\sx{j}E]}) or the change to a palatal consonant for non-labials (e.g., \textbf{te} \textipa{[tE]} vs \textbf{tie} \textipa{[cE]}). Table \ref{table:softhard} shows how non-labials are affected by palatalisation in Iridian.

\section{Orthographic representation}\label{sec:ortho}
\subsection{Alphabet}

\par The Iridian language uses the Latin script with the following 29 letters: \textbf{a, b, c, č, d, e, f, g, h, i, j, k, l, m, n, o, p, q, r, s, š, t, u, v, w, x, y, z, ž}.

The language was originally written in its own script but after the Latin alphabet has been adapted and has been in use since the First Bohemian Union in the 14th century. Due to the historical ties with the Kingdom of Bohemia and its historical successors, Czech orthography has had a great influence on the orthography of Iridian.

The Cyrillic script coexisted with the Iridian Latin alphabet from the 12th until the early 16th century. Today Cyrillic is still used to write the Ukrainian dialects of Iridian.

\begin{table}
	\small
 	\caption{The Iridian alphabet.}\index{alphabet}
	\medskip
	\begin{tabu}to 0.9 \textwidth {YY[1.3]YYY[1.3]Y}
		\toprule
		{{\sc  symbol}} & {\sc name} & {\sc ipa} & {{\sc  symbol}} & {\sc name} & {\sc ipa}\\
		\midrule
		A a	  	& á 		& /a/				& O o 	& ó 				& /o/\\
		B b			& bé	& /b/				& P p		& pé				& /p/\\
		C c			& cét & /t͡s /		& Q q		& kvé				& --\\
		Č č			& ča		& /t͡ɕ/			& R r		& er					& /r/\\
		D d			& dé	& /d/				& S s		& es					& /s/\\
		E e			& é		& /e/				& Š š		& éš 				& /ɕ/\\
		F f			& fí	& /f/				& T t		& té				& /t/\\
		G g			& gé 	& /g/				& U u 	& ú					& /u/\\
		H h			& há 		& /x/				& V v 	& vé 				& /ʋ/\\
		I i			& í 		& /i/				& W w		& vének				& --\\
		J j			& j\'yt	& /j/				& X x		& iks 				& --\\
		K k 		& ká 		& /k/				& Y y		& ýpsý\'lon		& /y/\\
		L l 		& el 		& /l/				& Z z		& zet 				& /z/\\
		M m			& em 		& /m/				& 		& žes 				& /ʑ/ \\
		N n			& en		& /n/				&				&							&	\\
		\bottomrule
	\end{tabu}
\end{table}

\begin{table}[h]
	\small
 	\caption{Correspondence between the Iridian Latin and Cyrillic scripts.}\index{alphabet}
	\medskip
	\begin{tabu}to 0.8 \textwidth {YYYY}
		\toprule
		{{\sc  latin}} & {\sc cyrillic} & {{\sc  latin}} & {\sc cyrillic} \\
		\midrule\addlinespace
		A a 		& А а	& O o   & О о \\ 
		B b			& Б Б 	& P p 	& П п \\
		C c 		& Ц ц 	& Q q 	& -- \\
		Č č 		& Ч ч 	& R r 	& Р р \\
		D d 		& Д д	& S s 	& С с \\
		E e 		& Е е 	& Š š 	& Ш ш \\
		F f			& Ф ф	& Tt 	& Т т \\
		G g 		& Г г	& Uu	& У у \\
		H h			& Х х	& V v   & В в\\
		I i			& И и	& W w   & --\\
		J j			& --	& X x   & --\\
		K k			& К к	& Y y   & Ы ы\\
		L l			& Л л   & Z z   & З з\\
		M m 		& М м   & Ž ž   & Ж ж\\
		N n 		& Н н   &&\\\addlinespace
		\multicolumn{4}{l}{\emph{Letters unique to the Cyrillic script}}\\\addlinespace
		Dz dz 		& Ѕ ѕ 	& Dž dž & Џ џ\\
		/ja/ 		& Я я	&/je/ & Є є\\
		/jo/		& Ю ю   & &\\
		\k{A} \k{a} & Ѫ ѫ&\k{E} \k{e}&Ѧ ѧ\\ \addlinespace
		\bottomrule
	\end{tabu}
\end{table}
 
In addition to the caron (ˇ) found in č, š and ž used to indicate palatalisation, Iridian also uses two additional diacritics over vowels: the acute accent (´), which is used to mark long vowels, and the ogonek (˛) used to mark nasal vowels. Accented vowels are not considered as separate letters but as alternative forms of the same vowel.

\subsection{Orthographic Conventions}
Iridian spelling is fairly regular.


\subsection{Punctuation}
The use of the full stop (.), the colon (:), the semicolon (;), the question mark (?) and the exclamation mark (!) is similar to their use in other central European languages.

The full stop is also used to separate dates written numerically (e.g., 21.09.2019) or to denote ordinal numbers, often followed by an em-dash (e.g., 3.--- \irdp{ór}{third hour, i.e., 3 {\sc a.m.}})

Iridian uses reverse guillemets (\ird{citácrám}) to set off quotations:

\ex
\ird{Dálek: >>To <<}
\xe