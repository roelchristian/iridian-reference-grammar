\chapter{Complex Sentences}

\section{Coordination} \index{coordination}

Iridian has three groups of coordinating conjunctions: the additive \irdp{a}{and} and \irdp{\v{s}e}{with};
the contrastive \ird{m\'a} and \ird{ozn\'a} (both translated to \printlang{en}\index{English} as \trsl{but}); and
the disjunctive/correlative \ird{je}, \ird{le} and \ird{ni}.

\ird{A} corresponds to the English \trsl{and.} When coordinating simple noun pairs, however, \irdp{še} is more often used though. The derived construction \ird{a še} is also common and has a similar meaning to the English \trsl{and also}.

\pex
\begingl
    \gla Mámka \textbf{še} pápku na Prahe spaníček.//
    \glb mother-\Dim{} \Com{} father-\Dim{}-\Ins{} \Loc{} Prague-\Pat{} vacation-\Av{}-\Pf{}//
    \glft \trsl{Mom and Dad went to Prague for vacation.}//
\endgl
\xe
\pex
\begingl
    \gla Janek \textbf{a} \textbf{še} Marku kurs hlupinžice.//
    \glb Janek and \Com{} Marek-\Ins{} class fail-\Av{}-\Pf{}-\Quot{}//
    \glft \trsl{Janek as well as Marek failed the class.}//
\endgl
\xe

In constructions with \ird{še} where one of the nouns coordinated is a pronoun or a deictic\index{deictic}, the pronoun or deictic is presented first followed by the other noun in the instrumental case\index{instrumental case}.

\pex
\begingl
    \gla Dá \textbf{še} Ivanu sohladoušce.//
    \glb \mk{1s.str} \Com{} Ivan-\Ins{} classmate//
    \glft \trsl{Ivan and I are classmates.}//
\endgl
\xe

In a few cases, \ird{a} is used instead of \ird{še} where the latter can be interpreted as having an attributive meaning. Where the noun is marked, however, only \ird{a} can be used.

\begin{multicols}{2}
\pex\a
\begingl
    \gla trava \textbf{še} lépu//
    \glb bread \Com{} cheese-\Ins{}//
    \glft \trsl{bread with cheese} i.e., \trsl{cheese sandwich}//
\endgl
\a
\begingl
    \gla trava \textbf{a} lép//
    \glb bread and cheese//
    \glft \trsl{bread and cheese}//
\endgl
\xe\end{multicols}

\pex
\begingl
    \gla To kurs-te Jankám \textbf{a} Markám hlupienince.//
    \glb this class-\Foc{} Janek-\Agt{} and Marek-\Agt{} class fail-\Pv{}-\Pf{}-\Quot{}//
    \glft \trsl{It was this class that Marek and Janek failed.}//
\endgl
\xe


The bisyndetic coordination (\cite{velupillai2012}) \ird{a} Y \ird{a} Y is also with similar emphatic meaning as \ird{a še}.

\pex
\begingl
    \gla \textbf{a} plocem \textbf{a} ploceš.//
    \glb and family-\First{}\Sg{} and family-\mk{2s}//
    \glft \trsl{both my family and yours}//
\endgl
\xe

\pex
\begingl
    \gla \textbf{a} hastu \textbf{a} še zmenu zoviec hloubižách.//
    \glb and suffering-\mk{} and \Com{} happiness-\Ins{} remain-\Cv{} love-\mk{av-ctpv}//
    \glft \trsl{Til death do us part.} \emph{Lit.,} \trsl{I will love you through both suffering and joy.}//
\endgl
\xe

With multiple nouns or noun phrases, especially in serial lists, the coordinating conjunction is often simply dropped.

\pex
\begingl
    \gla Ivan, Jarek, Elena na meza.//
    \glb Ivan Jarek Elena \Loc{} room-\Pat{}//
    \glft \trsl{Ivan, Jarek, and Elena are in the room.}//
\endgl
\xe

\pex
\begingl
    \gla Morkve, hlepost, ruk, molec hladniževí.//
    \glb carrot asparagus broccoli cabbage to:displease-\Av{}-\Cont{}//
    \glft \trsl{I don't like carrots, asparagus, broccoli or cabbage.}//
\endgl
\xe

\ird{A} or \ird{še} however is required when two adjectives are used to modify a noun, with \ird{še} used when the two adjectives describe the same noun and \ird{a} (or often \ird{a še}) when describing two distinct objects.\footnote{When used this way, the noun preceding \ird{še} or \ird{a še} is not declined in the instrumental case.}

\pex
\a
\begingl
    \gla Sodoví \textbf{še} ludí kobera tahatnik.//
    \glb black with white shirt bring-\Pv{}-\Pf{}//
    \glft \trsl{I brought the black-and-white shirt.}//
\endgl
\a
\begingl
    \gla Sodoví \textbf{a} \textbf{(še)} ludí kobera tahatnik.//
    \glb black and with white shirt bring-\Pv{}-\Pf{}//
    \glft \trsl{I brought the black shirt as well as the white one.}//
\endgl
\xe

Sentences of the type

\ex
It is [\mk{adjective}] that[ \mk{subordinate clause}].
\xe

are normally translated in Iridian using an expletive-\ird{a} construction, with the adjective in the attributive form at the start of the phrase, followed by \ird{a}, and then by the rest of the main clause. Normally this construction is used for sentences that pass judgment to the action or state described in the main clause, although in some cases the adjective is simply used for description.

\pex
\begingl
    \gla Interezní a téknik znohouštnilá te prádelnik.//
    \glb interesting-\Att{} and engineering study\mk{-pv-sbj.ipf} \mk{rz} choose-\Pv{}-\Pf{}//
    \glft \trsl{It is interesting that you chose to study engineering.}//
\endgl
\xe
\pex
\begingl
    \gla Komí a já ščenžek.//
    \glb good-\Att{} and \mk{2s.str} arrive-\Av{}-\Pf{}//
    \glft \trsl{Good you're here now!}//
\endgl
\xe

Another common use of the expletive \ird{a} is with the word \irdp{shlac}{now} (pronounced [sxlat] instead of the more intuitive [sxlat͡s]) to form the phrase \ird{shlac a}\footnote{This is therefore pronounced [ˈsxlatɐ].}, which is used to introduce a subordinate clause, similar to \trsl{now that} in English.

\pex
\begingl
    \gla Shlac a provísor ščenžek, kurs šelčinách.//
    \glb now and professor arrive-\Av{}-\Pf{} class begin-\mk{pv-ctpv}//
    \glft \trsl{Now that the professor is here, we will begin our class.}//
\endgl
\xe


\ird{Má} and \ird{ozná} are used to express contrast, like the English \trsl{but}. \ird{Ozná} however is more restrictive, and can only be used if the first clause is in the negative and the second clause directly contradicts (or provides an alternative to) the first. The clause introduced by \ird{ozná} must directly correspond to the element in the first clause being negated. Where the initial element is inflected, such inflection must also be reflected on the alternative presented in the \ird{ozná} clause.\footnote{The syntax of the main clause does not necessarily correspond to how the sentence would have otherwise been constructed in isolation. For instance, the neutral syntax for example (\getref{ozna}) without the \ird{ozná} would be: \ird{Bi\k{e}c záčesčeví.}}

\pex
\begingl
\gla Stožek má na duma niho čast.//
\glb go-\Av{}-\Pf{} but \Loc{} house-\Pat{} \mk{nexst} person//
\glft \trsl{I went but no one was home.}//
\endgl
\xe


\pex[tag=ozna]
\begingl
\gla Zám bi\k{e}c česčeví ozná jec.//
\glb \Neg{} cat to:please-\Av{}-\Cont{} but dog//
\glft \trsl{(I) don't like cats but I do like dogs.}//
\endgl
\xe

\ird{Ozn\'a} does not allow a negative\index{negation} argument. If the main clause is positive and the secondary clause is negative, \ird{m\'a} is used instead.

\pex
\begingl
\gla To jako odpizdnounil\'a to hrebe ce\v{s}cev\'i, m\'a z\'am j\'an.//
\glb \Dem{}.\Prox{} tree to:grow-\Loc{}-\Subj{}.\Ipf{} \Rz{} mushroom-\Pat{} to:please-\Av{}-\Cont{} but \Neg{} \Dem{}.\Med{}//
\glft \trsl{Mushrooms love to grow under this tree, but not under that one.}//
\endgl
\xe

\ird{M\'a} or its variant \ird{a m\'a} (literally \trsl{and but}) is also used to introduce exclamatory sentences. This usage is purely idiomatic and does not require for there to be an actual contrastive meaning in the sentences.

\pex
\begingl
\gla A m\'a duma nahte a\v{s}tev\'i!//
\glb and but house too:much be:pretty-\Cont{}//
\glft \trsl{Your house is very beautiful!}//
\endgl
\xe

Finally, the disjunctive conjunctions\index{disjunctive conjunction} \ird{je}, \ird{li}, and \ird{ni} are used to join phrases or sentences that are seen as alternatives to each other. \irdp{Je}{or} may be used to separate the alternatives proposed, or reduplicated, preceding each of the components of the sentence (i.e., \irdp{je X je Y}{either X or Y}); this latter use often means that the options being presented are the only ones available. \ird{Ni}\footnote{\ird{Ni} is an Indo-European, possibly Slavic, borrowing.\index{linguistic borrowing}} is the inverse of \ird{je} and must always be used in pairs (\irdp{ni X ni Y}{neither X nor Y}) as when used alone it functions as an adverb (similar to English \trsl{not even} or \trsl{at all}). An obvious exception, however, would be in a conversation, when a speaker would provide a negative alternative response to an already negative statement (see example (\getfullref{ni.resp}) below).

\pex
\begingl
\gla Ni ircevn\'i ni ru\v{s}\v{c}evn\'i malnov\'im zahviržétev\'i.//
\glb nor Iridian-\Att{} nor Russian-\Att{} tongue-\Ins{} speak-\Av{}-\Pot{}-\Cont{}//
\glft \trsl{I can't speak neither Iridian nor Russian.}//
\endgl
\xe

\pex\a\begingl
\gla D\'a ircevn\'i malnov\'im ni zazahviržétev\'i.//
\glb \First{}\Sg{}\Str{} Iridian-\Att{} tongue-\Ins{} not:even speak-\Av{}-\Pot{}-\Cont{}//
\glft \trsl{I can't speak any Iridian at all.}//
\endgl
\a\begingl
\gla Ni ircevn\'i ni ru\v{s}\v{c}evn\'i malnov\'im zahviržétev\'i.//
\glb nor Iridian-\Att{} nor Russian-\Att{} tongue-\Ins{} speak-\Av{}-\Pot{}-\Cont{}//
\glft \trsl{I can't speak neither Iridian nor Russian.}//
\endgl
\a\vtop{\halign{%
#\hfil& \qquad #\hfil\cr
\ird{---\,D\'a ru\v{s}\v{c}evn\'i malnov\'im zahviržétev\'i.} & \trsl{I don't speak Russian.}\cr
\ird{---\,Ni d\'a.} & \trsl{Neither do I.}\cr
}}\deftagex{ni}\deftaglabel{resp}
\xe

\ird{Le} (another possible Slavic\index{Slavic} borrowing\index{linguistic borrowing}, adopted from Common Slavic \emph{li} or \emph{ili}) has a more emphatic and contrastive meaning than \ird{je}. It is used when the speaker thinks that the option being presented is counterfactual or doubtful. Unlike \ird{je} or \ird{ni}, \ird{le} is added to the end of the word or phrase. \ird{Le} is most often used in parenthetical statements or in responses.

\pex\a\begingl
\gla Marek-le ru\v{s}\v{c}evn\'i malnov\'im zahviržétev\'i.//
\glb Marek=or Russian-\Att{} tongue-\Ins{} speak-\Av{}-\Pot{}-\Cont{}//
\glft \trsl{Or maybe Marek can speak Iridian.}//
\endgl


\xe

\section{Apposition}\index{apposition}\label{sec:apposition}

Appositive constructions in Iridian involve the juxtaposition of two or more noun phrases that have a single referent. An apposition can be non-restrictive if the appositive can be removed freely without changing the meaning of a sentence, or restrictive otherwise.

Formally both non-restrictive and restrictive appositives are treated as modifier phrases but only the latter is grammaticalized. The restrictive appositive must always precede the noun phrase it modifies, linked together by the particle \ird{ko}. Non-restrictive appositives on the other hand are simply juxtaposed together, although a comma is often inserted around the appositive if it consists of more than one word.

\pex\a
\begingl\deftagex{appos}\deftaglabel{1}
    \gla \'Oto mlažka na Mnihe znohouščeví.//
    \glb  \'Oto brother-\Dim{} \Loc{} Munich-\Pat{} study-\Av{}-\Cont{}//
    \glft \trsl{My brother Otto is studying in Munich.}//
\endgl
\a\begingl\deftagex{appos}\deftaglabel{res}
    \gla \'Oto \textbf{ko} mlažka na Mnihe znohouščeví.//
    \glb  \'Oto \Lnk{} brother-\Dim{} \Loc{} Munich-\Pat{} study-\Av{}-\Cont{}//
    \glft \trsl{My brother Otto is studying in Munich.}//
\endgl
\xe

Examples(\getfullref{appos.1}) and (\getfullref{appos.res}) shows two different translations of the English phrase \trsl{My brother \'Oto is studying in Munich.} Example (\getfullref{appos.1}) is non-restrictive and can be interpreted as \trsl{I have a brother namsed \'Oto who is studying in Munich} while (\getfullref{appos.res}) being restrictive can be translated more on the lines of \trsl{Among my brothers, it is \'Oto who is studying in Munich.} The restrictive appositive implies specificity and by extension the existence of a group where this specificity holds true; in (\getfullref{appos.res}) this is taken to mean that a set of brothers exists and \'Oto is a member of this set.

\section{Subordinate clauses in general}

\section{Clause-linking with \ird{še}}



\section{Converbial constructions}\label{converbs-syntax}\index{converb}

\subsection{In general}

\subsection{Adverbial converbs}\index{converb}

A common type of compound verb construction involves the main verb preceded by the imperfective converbial form of a secondary verb. The secondary verb normally specifies the manner or the means by which the action descrubed by the main verb is performed.


\subsection{Temporal constructions}

A converbial construction is often used in temporal clauses\index{temporal clause}, with the imperfective converbial form used when the action is unfinished or continuing and the perfective otherwise. When used in a temporal clause, the converb may sometimes be separated from the main clause by the particle \ird{si}.\footnote{\ird{Si} is virtually never used in the spoken language.}

\pex
\begingl
\gla Otviec (si) na Varšave možlašaní.//
\glb be:young-\mk{cv.ipf} when \Loc{} Warsaw-\Pat{} understand-\mk{av-ret}//
\glft \trsl{When I was young, we used to live in Warsaw.}//
\endgl
\xe

\subsection{Causal clauses}

Clauses expressing reason are usually expressed by a converbial construction. The antecedent and the main clause may be connected with \irdp{am}{because,} although this is often dropped in casual speech.

\pex
\begingl
\gla Za prove záznohouštu Martin meštnašek.//
\glb for exam-\Pat{} \Neg{}-study-\mk{cv.pf} Martin fail-\Av{}-\Pf{}//
\glft \trsl{Martin failed the exam because he didn't study.}//
\endgl
\xe


\pex
\begingl
\gla Kinoteka stožílá to všihniec mámka zachovažek.//
\glb cinema-\Pat{} go-\Av{}-\Sbj{}.\Ipf{} \Rz{} be:angry-\Cv{}.\Ipf{} mother-\Dim{} allow-\Av{}-\Pf{}//
\glft \trsl{Since she was still mad at us, Mum did not let us go to the movies.}//
\endgl
\xe


\subsection{Transgressive clauses}\index{converb}

Converbs in Iridian have parallel usage as the transgressive\index{transgressive} conjugations in \printlang{cs}\index{Czech} and Slovak\index{Slovak}. It is the consensus among scholars of the languages, though, that the converbial forms in Iridian and the transgressive forms in Czech and Slovak, developed independently of each other; although to what extent one influenced the other is still the subject of debate. The converbial forms in Iridian have more varied uses than the transgressives in Czech (Slovak having kept only the present transgressive form), and whereas the latter forms have largely fallen in disuse (relegated to the literary register) in both Czech and Slovak, converbial forms are still widely used in Iridian.

Although Czech grammarians use the terms `past' and `present' to distinguish between the two forms used in the language, the distinction is actually one of aspect\index{aspect}, as in Iridian. In general, the past transgressive form corresponds with the perfect converbial form, and may be used to indicate a foregoing action; the present transgressive, on the other hand, corresponds to the imperfect converb and is used to indicate a coincident/contemporaneous action.

This correspondence is not complete, however. For example, consider this sentence in Czech\index{Czech}: \irdp{Děti, \textbf{vidouce} babičku, vyběhly ven}{The children, seeing their grandmother, ran outside.} The verb in the transgressive clause is in the present tense in this case, while in Iridian, the same sentence will be translated with the perfective as follows:

\pex
\begingl
\gla \v{S}ášlika vedu byl naladiec mnilžek.//
\glb grandmother-\mk{dim-pat} see-\mk{cv.pf} children run-\mk{cv.ipf} go:out-\Av{}-\Pf{}//
\glft \trsl{The children, having seen their grandmother, ran outside.}//
\endgl
\xe

The Czech\index{Czech} sentence above can alternatively be translated using the imperfective converbial form, but this would put a stronger emphasis on the two actions happening at the same time and so the original construction can be considered as the more idiomatic one.

\subsection{In fixed expressions}

The past converbial form is used in expressing gratitude, approbation or condolencess, or in asking for forgiveness. This usage is idiomatic and the actions do not necessarily need to have been completed. The main clause is often in the hortative mood\index{hortative mood} and separated from the converb clause with \irdp{am}{because.} Moreover, this usage, unlike most converbial constructions, allow the verb of the converb clause to have a different subject as long as such subject is marked explicitly in the agentive case. However, since the converbial form of verbs are invariable, if the subordinate clause requires further complexity when it comes to the verb in the converb clause, a dependent \ird{še} clause may be use instead of a converb.

\pex
\a Expressing gratitude:\\
\begingl
\gla Stranu am luhninká.//
\glb help-\mk{cv.pf} because thank-\mk{pv-hort}//
\glft `Thank you for helping.'//
\endgl
\a Asking for forgiveness:\\
\begingl
\gla Lienu záščenu am rozvedniká.//
\glb on:time-\Ins{} \Neg{}-arrive-\mk{cv.pf} because forgive-\mk{pv-hort}//
\glft `Sorry for being late.'//
\endgl
\a Expressing condolences:\footnote{Compare this example to the following, where a converb clause cannot be used:

\ex[lingstyle=fnex,belowexskip=-1em]
\begingl
\gla Pápka na puvode shradnice še množniká.//
\glb father \Loc{} war-\Pat{} die-\Pv{}-\Pf{}-\Quot{} with console-\mk{pv-hort}//
\glft `I'm sorry to hear your father died (\emph{lit.,} was killed) in the war.'//
\endgl\xe}\\
\begingl
\gla Pápkám shradu am množniká.//
\glb father-\mk{dim-agt} die-\mk{cv-pf} because console-\mk{pv-hort}//
\glft `I'm sorry for your father's death.'//
\endgl
\a Expressing approbation:\\
\begingl
\gla Prove vlastnu am prehodniká.//
\glb exam-\Pat{} pass-\mk{cv.pf} because praise-\mk{pv-hort}//
\glft \trsl{Congratulations for passing the exam!}//
\endgl
\xe


\section{Reported Speech}\label{sec:reportedspeech}\index{reported speech}\index{indirect speech|see{reported speech}}

The reported statement and the main clause are separated by the quotative particle \ird{to-že}\footnote{This particle will simply be glossed as {\scshape qp} even though it actually consist of two parts: the relativizing particle \ird{to} and the cliticized quotative particle \ird{že}.}.

\pex
\begingl
\gla Maša advokát neví to-že zíček.//
\glb Maša lawyer \mk{cop.quot} \mk{qp} say-\Av{}-\Pf{}//
\glft \trsl{(He) said that Maša is a lawyer}//
\endgl
\xe

The reported part is treated as a subordinate clause and must appear before the main clause. In general reported speech takes the form

\ex\deftagex{ex:repstruct}{\small
\bigg[ \Big[ \big[ [TOP*] [PRED in quotative mood] \big] + \big[\ird{to-že}*\big] \Big] \bigg] + \bigg[verbum dicendi*\bigg],
}\xe

\ex
\begin{forest}
  [S,
    [{TOP}, [TOP] [VP] ]
    [{PRED}, [QP] [VP,  [NP] [VP]]
    ]]
\end{forest}
\xe
where the elements followed by an asterisk (*) are optional.

The \emph{verbum dicendi}\index{verbum dicendi} (Latin for verb of speech/speaking) is the verb in the main clause that signals that the subordinate clause is a quoted clause and that its main clause should therefore appear in the quotative mood. Examples of \emph{verba dicendi} in Iridian include \irdp{zieká}{to say}; \irdp{vadá}{to think}; \irdp{kvuštá}{to hear}; \irdp{vidá}{to see}; \irdp{hloupá}{to ask}; \irdp{ohletá}{to remember}; \irdp{shová}{to recount, to tell a story}. Note that although they are called verbs ``of speaking'' they do not necessarily introduce speech as much as function as grammaticalized tags marking the quotative,  which is more properly analyzed to mark not just speech but inferentiality and evidentiality as well.

More complex \emph{verba dicendi} can be formed by using an imperfect converbial construction (the converb form in \ird{-iec}) with a canonical \emph{verbum dicendi}. To understand this consider the following sentences in English:

\pex[*=?*]
\a She said no.\deftagex{vd}\deftaglabel{1}
\a She whispered no.\deftaglabel{2}
\a She said no in a whisper.\deftaglabel{3}
\a \ljudge{?} She said \textbf{in a whisper} no.\deftaglabel{4}
\a \ljudge{??} She said \textbf{whisperingly} no.\deftaglabel{5}
\xe

\smallskip

We see that both \emph{said} (\getfullref{vd.1}) and \emph{whispered} (\getfullref{vd.2}) are \emph{verba dicendi} in English. Nonetheless it's also obvious how \getfullref{vd.2} is simply a function of (\getfullref{vd.1}), i.e., we can express (\getfullref{vd.2}) in terms of (\getfullref{vd.1}), in this case using an adverbial construction (\trsl{in a whisper}) as we see in \getfullref{vd.3} or the more affected \getfullref{vd.4}. Finally using a simple adverbial is theoretically allowed in English (\getfullref{vd.5}), although as we see the resulting construction is rather unwieldy or unnatural-sounding.

In Iridian, however, constructions like (\getfullref{vd.2}) are not permitted, with preference given to adverbial (or more correctly, converbial)\index{converb} constructions. Thus we translate (\getfullref{vd.2}) as:

\pex
\begingl
\gla Ne to-že mišlec zíček.//
\glb no \mk{qp} whisper-\Cv{} say-\Av{}-\Pf{}//
\glft \trsl{(She) whispered no.}//
\endgl
\xe


This converbial construction is not limited to what is essentially describing how the verbum dicendi was  Other more idiomatic treatments include

%% EXAMPLES HERE

It should be noted as well how the verb \irdp{vadá}{to think} and its derived forms, due to their inherent meanings, require the subjunctive to be used in the reported clause. This is true whether or not the subjunctive would have been used had the reported clause been a regular dependent clause.


\pex
\a
\begingl
  \gla Já mnou.//
  \glb you correct//
  \glft \trsl{You're right.}//
\endgl
\a
\begingl
  \gla Já mnou neví.//
  \glb you correct \mk{cop.quot}//
  \glft \trsl{(I heard) you're right}//
\endgl
\a
\begingl
  \gla Já mnou nehlí to-že Martin spouviec vážál.//
  \glb you correct \mk{cop.quot.sbj} \mk{qp} Martin agree-\Cv{} think-\Av{}-\Cont{}//
  \glft \trsl{Martin agrees that you are right.}//
\endgl
\xe



We see from (\getfullref{ex:repstruct}) that when it comes to reported speech and similar constructions in Iridian, the \ird{verbum dicendi}\index{verbum dicendi} is not necessary to create a well-formed sentence. The same is true with the quotative particle \ird{to-že}. Both can be omitted without making the sentence grammatically incorrect since the quotative particle is enough to identify the reported clause.\index{reported speech}.

In most instances, however, removing either the main verb or the main verb and the quotative particle can cause the resulting sentence to acquire a new meaning. This is especially true when the quotative mood is used not to report speech but to imply a certain unsureness on the part of the speaker about the information being presented, or for the speaker to distance themself by implying through the use of the quotative that the information is secondhand and not theirs.

Generally \ird{to-že} is kept when the speaker is quoting themself, to repeat or emphasize what they have said, or expletively, to express their frustration or affirmation.\footnote{When used this way the pronunciation of \ird{to-že} is closer to an emphatic \nt{"to\dpu\textctz{}E} or even \nt{"to\dpu:\textctz{}EP}}

%% TODO remove affrication of initial dental stops in Phonology section!!!

\pex
\begingl
\gla Mnou neví to-že!//
\glb correct \mk{cop.quot} \mk{qp}//
\glft \trsl{I've been telling you) it is right.}//
\endgl
\xe

\pex
\begingl
\gla Dá roctymút to!//
\glb \First{}\Sg{} dance-\mk{abl-quot.ipf} \mk{rz}//
\glft `(But) I can dance.'//
\endgl
\xe


Interestingly, commands and requests are not treated as reported speech but as regular subordinate clauses governed by \ird{to} and not by \ird{to-že}.

When the quoted clause is a question, whether a direct one or not, the quoted clause is preceded by the particle \irdp{a}{and} and the word \irdp{ane}{whether} is used instead of \ird{to-že}. The word \ird{ane} is also used for verba dicendi that are interrogative in nature, such as \irdp{préhoustá}{to ask},

\pex
\begingl
  \gla A Janek zdalšice ane préhousček.//
  \glb and Janek have:breakfast-\Av{}-\Pf{}-\Quot{} whether ask-\Av{}-\Pf{}//
  \glft \trsl{(He) asked (me) whether Janek has had breakfast yet.}//
\endgl
\xe

\pex
\begingl
  \gla A tóm to mládu hodinaže ane, nie svad postupál.//
  \glb and book this year-\Ins{} finish-\mk{pv-ctpv-quot} whether \Pl{} fan be:excited-\mk{cont}//
  \glft \trsl{His fans are excited to know if he'll finish his book this year.
}//
\endgl
\xe
