\chapter{Complex Sentences}

\section{Clause-linking with \ird{\v{s}e}}

\section{Converbial Constructions}\label{converbs-syntax}\index{converb}

\subsection{The imperfective in \ird{-iec}}



\subsection{The perfective in \ird{-e}}

The perfective \textit{-iêce} is often used in clause linking.

\pex
\begingl
\gla O\v{s}tiêce krazkem.//
\glb read-\mk{cv.pf} understand-\mk{pf-1s}//
\glft `I read and understood.'//
\endgl
\xe

Clauses expressing reason is usually expressed by a converbial construction.

\pex
\begingl
\gla Za eksama názhaziêce, Martin órek.//
\glb for exam-\mk{pat} \mk{neg}-study-\mk{cv.pf} Martin fail-\mk{pf}//
\glft `Martin failed the exam because he didn't study.'//
\endgl
\xe

\section{Reported Speech}\label{sec:reportedspeech}\index{reported speech}\index{indirect speech|see{reported speech}}

The reported statement and the main clause are separated by the quotative particle \ird{to-\v{z}e}\footnote{This particle will simply be glossed as {\scshape qp} even though it actually consists of two parts: the relativizing particle \ird{to} and the cliticized quotative particle \ird{\v{z}e}.}.

\pex
\begingl
\gla Ma\v{s}a advok\'at nev\'i to-\v{z}e z\'i\v{c}ek.//
\glb Ma\v{s}a lawyer \mk{cop.quot} \mk{qp} say-\mk{av-pf}//
\glft \trsl{(He) said that Ma\v{s}a is a lawyer}//
\endgl
\xe

The reported part is treated as a subordinate clause and must appear before the main clause. In general reported speech takes the form

\ex\label{ex:repstruct}{\small
\bigg[ \Big[ \big[ [TOP*] [PRED in quotative mood] \big] + [\ird{to-\v{z}e}*] \Big] \bigg] + [verbum dicendi*],
}\xe

where the elements followed by an asterisk (*) are optional.

The \emph{verbum dicendi}\index{verbum dicendi} (Latin for verb of speech/speaking) is the verb in the main clause that signals that the subordinate clause is a quoted clause and that its main clause should therefore appear in the quotative mood. Examples of \emph{verba dicendi} in Iridian include \irdp{ziek\'a}{to say}; \irdp{vad\'a}{to think}; \irdp{kvu\v{s}t\'a}{to hear}; \irdp{vid\'a}{to see}; \irdp{hloup\'a}{to ask}; \irdp{ohlet\'a}{to remember}; \irdp{shov\'a}{to recount, to tell a story}. Note that although they are called verbs ``of speaking'' they do not necessarily introduce speech as much as function as grammaticalized tags marking the quotative,  which is more properly analyzed to mark not just speech but inferentiality and evidentiality as well.

More complex \emph{verba dicendi} can be formed by using an imperfect converbial construction (the converb form in \ird{-iec}) with a canonical \emph{verbum dicendi}. To understand this consider the following sentences in English:

\pex[*=?*]
\a She said no.\deftagex{vd}\deftaglabel{1}
\a She whispered no.\deftaglabel{2}
\a She said no in a whisper.\deftaglabel{3}
\a \ljudge{??} She said \textbf{in a whisper} no.\deftaglabel{4}
\a \ljudge{??} She said \textbf{whisperingly} no.\deftaglabel{5}
\xe

\smallskip

We see that both \emph{said} (\getfullref{vd.1}) and \emph{whispered} (\getfullref{vd.2}) are \emph{verba dicendi} in English. Nonetheless it's also obvious how \getfullref{vd.2} is simply a function of (\getfullref{vd.1}), i.e., we can express (\getfullref{vd.2}) in terms of (\getfullref{vd.1}), in this case using an adverbial construction (\trsl{in a whisper}) as we see in \getfullref{vd.3} or the more affected \getfullref{vd.4}. Finally using a simple adverbial is theoretically allowed in English (\getfullref{vd.5}), although as we see the resulting construction is rather unwieldy or unnatural-sounding.

In Iridian, however, constructions like (\getfullref{vd.2}) are not permitted, with preference given to adverbial (or more correctly, converbial)\index{converb} constructions. Thus we translate (\getfullref{vd.2}) as:

\pex
\begingl
\gla Ne to-\v{z}e mi\v{s}lec z\'i\v{c}ek.//
\glb no \mk{qp} whisper-\mk{cv} say-\mk{av-pf}//
\glft \trsl{(She) whispered no.}//
\endgl
\xe

This is especially common w
Some examples are listed below:

\ex with \irdp{vad\'a}{to think}
\irdp{spouviec vad\'a}{to agree}
\xe

\par Secondary verba dicendi are formed with an adverbial construction using the imperfective converb in \textbf{-iec}.

\pex
\begingl
\gla Já mnou necim to Martin priviec vadê.//
\glb you correct \mk{cop.quot.sbj.npst} \mk{rz} Martin agree-\mk{cv} think-\mk{ipf}//
\glft `Martin agrees that you are right.'//
\endgl
\xe

\par The quotative is also used emphatically to repeat a quote (often made by the speaker himself or herself), or to express the speaker's frustration or affirmation. When used this way, the verbum dicendi is omitted, and the expletive \textbf{nó} is often added.

\pex
\begingl
\gla Mnou necim to nó!//
\glb correct \mk{cop.quot.sbj.npst} \mk{rz} \mk{expl}//
\glft `(I've been telling you) it is right.'//
\endgl
\xe

\pex
\begingl
\gla Dá roctymút to!//
\glb \mk{1s} dance-\mk{abl-quot.ipf} \mk{rz}//
\glft `(But) I can dance.'//
\endgl
\xe

\par The tense/aspect of the quotative mood follows that of the quoted clause, independent of the tense/aspect of the verbum dicendi.
