\chapter{Complex Sentences}

\section{Coordination} \index{coordination}

Iridian has [number here] coordinating conjunctions: \irdp{a}{and},

When coordinating simple noun pairs, however, the particle \irdp{\v{s}e}{with}\index{\v{s}e} is mostly used where English would have used \trsl{and}. The derived construction \ird{a \v{s}e} is also common and has a similar meaning to the English \trsl{and also}.

\pex
\begingl
    \gla M\'amka \textbf{\v{s}e} p\'apku na Prah\'a span\'i\v{c}ek.//
    \glb mother-\mk{dim} \mk{com} father-\mk{dim-inst} \mk{loc} Prague-\mk{pat} vacation-\mk{av-pf}//
    \glft \trsl{Mom and Dad went to Prague for vacation.}//
\endgl
\xe

\pex
\begingl
    \gla Janek \textbf{a \v{s}e} Marku kurs hlupin\v{z}ice.//
    \glb Janek and \mk{com} Marek-\mk{inst} class fail-\mk{av-pf-quot}//
    \glft \trsl{Janek as well as Marek failed the class.}//
\endgl
\xe

In constructions with \ird{\v{s}e} where one of the nouns coordinated is a pronoun or a deictic\index{deictic}, the pronoun or deictic is presented first followed by the other noun in the instrumental case.

\pex
\begingl
    \gla D\'a \textbf{\v{s}e} Ivanu sohladou\v{s}ce.//
    \glb \mk{1s.str} \mk{com} Ivan-\mk{inst} classmate//
    \glft \trsl{Ivan and I are classmates.}//
\endgl
\xe

In a few cases, \ird{a} is used instead of \ird{\v{s}e} where the latter can be interpreted as having an attributive meaning. Where the noun is marked, however, only \ird{a} can be used.

\pex
\a
\begingl
    \gla trava \textbf{\v{s}e} l\'epu//
    \glb bread \mk{com} cheese-\mk{inst}//
    \glft \trsl{bread with cheese} i.e., \trsl{cheese sandwich}//
\endgl
\a
\begingl
    \gla trava \textbf{a} l\'ep//
    \glb bread and cheese//
    \glft \trsl{bread and cheese}//
\endgl
\xe

\pex
\begingl
    \gla To kurs-te Jank\'am \textbf{a} Mark\'am hlupienince.//
    \glb this class-\mk{foc} Janek-\mk{agt} and Marek-\mk{agt} class fail-\mk{pv-pf-quot}//
    \glft \trsl{It was this class that Marek and Janek failed.}//
\endgl
\xe


The bisyndetic coordination (\cite{velupillai2012}) \ird{a} Y \ird{a} Y is also with similar emphatic meaning as \ird{a \v{s}e}.

\pex
\begingl
    \gla \textbf{a} plocem \textbf{a} ploce\v{s}.//
    \glb and family-\mk{1s} and family-\mk{2s}//
    \glft \trsl{both my family and yours}//
\endgl
\xe

\pex
\begingl
    \gla \textbf{a} \v{c}astu \textbf{a} \v{s}e zmenu zoviec hloubi\v{z}al\'i.//
    \glb and suffering-\mk{} and \mk{com} happiness-\mk{inst} remain-\mk{cv} love-\mk{av-prog}//
    \glft \trsl{til death do us part}//
\endgl
\xe

With multiple nouns or noun phrases, especially in serial lists, the coordinating conjunction is often simply dropped.

\pex
\begingl
    \gla Ivan, Jarek, Elena na meza.//
    \glb Ivan Jarek Elena \mk{loc} room-\mk{pat}//
    \glft \trsl{Ivan, Jarek, and Elena are in the room.}//
\endgl
\xe

\pex
\begingl
    \gla Morkve, hlepost, ruk, molec \v{z}a hladni\v{z}\'al.//
    \glb carrot asparagus broccoli cabbage \mk{1s.pat} to:notplease-\mk{av-cont}//
    \glft \trsl{I don't like carrots, asparagus, broccoli or cabbage.}//
\endgl
\xe



\section{Clause-linking with \ird{\v{s}e}}

\section{Converbial Constructions}\label{converbs-syntax}\index{converb}

\subsection{The imperfective in \ird{-iec}}



\subsection{The perfective in \ird{-e}}

The perfective \textit{-iêce} is often used in clause linking.

\pex
\begingl
\gla O\v{s}tiêce krazkem.//
\glb read-\mk{cv.pf} understand-\mk{pf-1s}//
\glft `I read and understood.'//
\endgl
\xe

Clauses expressing reason is usually expressed by a converbial construction.

\pex
\begingl
\gla Za eksama názhaziêce, Martin órek.//
\glb for exam-\mk{pat} \mk{neg}-study-\mk{cv.pf} Martin fail-\mk{pf}//
\glft `Martin failed the exam because he didn't study.'//
\endgl
\xe

\section{Reported Speech}\label{sec:reportedspeech}\index{reported speech}\index{indirect speech|see{reported speech}}

The reported statement and the main clause are separated by the quotative particle \ird{to-\v{z}e}\footnote{This particle will simply be glossed as {\scshape qp} even though it actually consist of two parts: the relativizing particle \ird{to} and the cliticized quotative particle \ird{\v{z}e}.}.

\pex
\begingl
\gla Ma\v{s}a advok\'at nev\'i to-\v{z}e z\'i\v{c}ek.//
\glb Ma\v{s}a lawyer \mk{cop.quot} \mk{qp} say-\mk{av-pf}//
\glft \trsl{(He) said that Ma\v{s}a is a lawyer}//
\endgl
\xe

The reported part is treated as a subordinate clause and must appear before the main clause. In general reported speech takes the form

\ex\deftagex{ex:repstruct}{\small
\bigg[ \Big[ \big[ [TOP*] [PRED in quotative mood] \big] + \big[\ird{to-\v{z}e}*\big] \Big] \bigg] + \bigg[verbum dicendi*\bigg],
}\xe

\ex
\begin{forest}
  [S,
    [{TOP}, [TOP] [VP] ]
    [{PRED}, [QP] [VP,  [NP] [VP]]
    ]]
\end{forest}
\xe
where the elements followed by an asterisk (*) are optional.

The \emph{verbum dicendi}\index{verbum dicendi} (Latin for verb of speech/speaking) is the verb in the main clause that signals that the subordinate clause is a quoted clause and that its main clause should therefore appear in the quotative mood. Examples of \emph{verba dicendi} in Iridian include \irdp{ziek\'a}{to say}; \irdp{vad\'a}{to think}; \irdp{kvu\v{s}t\'a}{to hear}; \irdp{vid\'a}{to see}; \irdp{hloup\'a}{to ask}; \irdp{ohlet\'a}{to remember}; \irdp{shov\'a}{to recount, to tell a story}. Note that although they are called verbs ``of speaking'' they do not necessarily introduce speech as much as function as grammaticalized tags marking the quotative,  which is more properly analyzed to mark not just speech but inferentiality and evidentiality as well.

More complex \emph{verba dicendi} can be formed by using an imperfect converbial construction (the converb form in \ird{-iec}) with a canonical \emph{verbum dicendi}. To understand this consider the following sentences in English:

\pex[*=?*]
\a She said no.\deftagex{vd}\deftaglabel{1}
\a She whispered no.\deftaglabel{2}
\a She said no in a whisper.\deftaglabel{3}
\a \ljudge{??} She said \textbf{in a whisper} no.\deftaglabel{4}
\a \ljudge{??} She said \textbf{whisperingly} no.\deftaglabel{5}
\xe

\smallskip

We see that both \emph{said} (\getfullref{vd.1}) and \emph{whispered} (\getfullref{vd.2}) are \emph{verba dicendi} in English. Nonetheless it's also obvious how \getfullref{vd.2} is simply a function of (\getfullref{vd.1}), i.e., we can express (\getfullref{vd.2}) in terms of (\getfullref{vd.1}), in this case using an adverbial construction (\trsl{in a whisper}) as we see in \getfullref{vd.3} or the more affected \getfullref{vd.4}. Finally using a simple adverbial is theoretically allowed in English (\getfullref{vd.5}), although as we see the resulting construction is rather unwieldy or unnatural-sounding.

In Iridian, however, constructions like (\getfullref{vd.2}) are not permitted, with preference given to adverbial (or more correctly, converbial)\index{converb} constructions. Thus we translate (\getfullref{vd.2}) as:

\pex
\begingl
\gla Ne to-\v{z}e mi\v{s}lec z\'i\v{c}ek.//
\glb no \mk{qp} whisper-\mk{cv} say-\mk{av-pf}//
\glft \trsl{(She) whispered no.}//
\endgl
\xe


This converbial construction is not limited to what is essentially describing how the verbum dicendi was  Other more idiomatic treatments include

%% EXAMPLES HERE

It should be noted as well how the verb \irdp{vad\'a}{to think} and its derived forms, due to their inherent meanings, require the subjunctive to be used in the reported clause. This is true whether or not the subjunctive would have been used had the reported clause been a regular dependent clause.


\pex
\a
\begingl
  \gla Já mnou.//
  \glb you correct//
  \glft \trsl{You're right.}//
\endgl
\a
\begingl
  \gla Já mnou nev\'i.//
  \glb you correct \mk{cop.quot}//
  \glft \trsl{(I heard) you're right}//
\endgl
\a
\begingl
  \gla Já mnou nehl\'i to-\v{z}e Martin spouviec v\'a\v{z}\'al.//
  \glb you correct \mk{cop.quot.sbj} \mk{qp} Martin agree-\mk{cv} think-\mk{av-cont}//
  \glft \trsl{Martin agrees that you are right.}//
\endgl
\xe



We see from (\getfullref{ex:repstruct}) that when it comes to reported speech and similar constructions in Iridian, the \ird{verbum dicendi}\index{verbum dicendi} is not necessary to create a well-formed sentence. The same is true with the quotative particle \ird{to-\v{z}e}. Both can be omitted without making the sentence grammatically incorrect since the quotative particle is enough to identify the reported clause.\index{reported speech}.

In most instances, however, removing either the main verb or the main verb and the quotative particle can cause the resulting sentence to acquire a new meaning. This is especially true when the quotative mood is used not to report speech but to imply a certain unsureness on the part of the speaker about the information being presented, or for the speaker to distance themself by implying through the use of the quotative that the information is secondhand and not theirs.

Generally \ird{to-\v{z}e} is kept when the speaker is quoting themself, to repeat or emphasize what they have said, or expletively, to express their frustration or affirmation.\footnote{When used this way the pronunciation of \ird{to-\v{z}e} is closer to an emphatic \nt{"to\dpu\textctz{}E} or even \nt{"to\dpu:\textctz{}EP}}

%% TODO remove affrication of initial dental stops in Phonology section!!!

\pex
\begingl
\gla Mnou nev\'i to-\v{z}e!//
\glb correct \mk{cop.quot} \mk{qp}//
\glft \trsl{I've been telling you) it is right.}//
\endgl
\xe

\pex
\begingl
\gla Dá roctymút to!//
\glb \mk{1s} dance-\mk{abl-quot.ipf} \mk{rz}//
\glft `(But) I can dance.'//
\endgl
\xe


Interestingly, commands and requests are not treated as reported speech but as regular subordinate clauses governed by \ird{to} and not by \ird{to-\v{z}e}.

When the quoted clause is a question, whether a direct one or not, the quoted clause is preceded by the particle \irdp{a}{and} and the word \irdp{ane}{whether} is used instead of \ird{to-\v{z}e}. The word \ird{ane} is also used for verba dicendi that are interrogative in nature, such as \irdp{pr\'ehoust\'a}{to ask},

\pex
\begingl
  \gla A Janek zdal\v{s}ice ane pr\'ehous\v{c}ek.//
  \glb and Janek have:breakfast-\mk{av-pf-quot} whether ask-\mk{av-pf}//
  \glft \trsl{(He) asked (me) whether Janek has had breakfast yet.}//
\endgl
\xe

\pex
\begingl
  \gla A t\'om to ml\'adu hodina\v{z}e ane, nie svad postup\'al.//
  \glb and book this year-\mk{inst} finish-\mk{pv-ctpv-quot} whether \mk{pl} fan be:excited-\mk{cont}//
  \glft \trsl{His fans are excited to know if he'll finish his book this year.
}//
\endgl
\xe
