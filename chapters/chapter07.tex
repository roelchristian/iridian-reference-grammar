\chapter{Complex Sentences}

\section{Clause-linking with \ird{\v{s}e}}

\section{Converbial Constructions}\label{converbs-syntax}\index{converb}

\subsection{The imperfective in \ird{-iec}}



\subsection{The perfective in \ird{-e}}

The perfective \textit{-iêce} is often used in clause linking.

\pex
\begingl
\gla O\v{s}tiêce krazkem.//
\glb read-\mk{cv.pf} understand-\mk{pf-1s}//
\glft `I read and understood.'//
\endgl
\xe

Clauses expressing reason is usually expressed by a converbial construction.

\pex
\begingl
\gla Za eksama názhaziêce, Martin órek.//
\glb for exam-\mk{pat} \mk{neg}-study-\mk{cv.pf} Martin fail-\mk{pf}//
\glft `Martin failed the exam because he didn't study.'//
\endgl
\xe

\section{Reported Speech}\label{sec:reportedspeech}\index{reported speech}\index{indirect speech|see{reported speech}}

The reported statement and the main clause are separated by the quotative particle \ird{to-\v{z}e}\footnote{This particle will simply be glossed as {\scshape qp} even though it actually consists of two parts: the relativizing particle \ird{to} and the cliticized quotative particle \ird{ze}.}. The reported part is treated as a subordinate clause and must appear before the main clause.

Ana advok\'at nev\'i to-\v{z}e z\'i\v{c}ek.\\
Ana lawyer \mk{cop-quot} \mk{qp} say-\mk{av-pf}


\pex
\begingl
\gla Já na duma ne\v{s}kec to maty dálmek.//
\glb you-\mk{str} \mk{loc} house-\mk{pat} \mk{cop.quot.ipf} \mk{rz} mother say-\mk{1s.pf}//
\glft `(My) mother told me you are at home.'//
\endgl
\xe

\pex
\begingl
\gla Já na duma necim to maty dálmek.//
\glb you-\mk{str} \mk{loc} house-\mk{pat} \mk{cop.quot.sbj.npst} \mk{rz} mother say-\mk{1s.pf}//
\glft `(My) mother told me you might be at home.'//
\endgl
\xe

\pex
\begingl
\gla Mnúcs tiezninát.//
\glb husband kill-\mk{pv-quot.pf}//
\glft `(She) killed (her) husband (or so I heard).'//
\endgl
\xe

\par Direct speech, however, does not use the subjunctive.
\pex
\begingl
\gla ---Tak dá, dálek Tomá\v{s}.//
\glb here \mk{1s.str} say-\mk{pf} Tomá\v{s}//
\glft ```I'm here,'' Tomá\v{s} said.'//
\endgl
\xe


\par The following verbs are considered verba dicendi in Iridian and would trigger the quotative: \textbf{dálá} `to say', \textbf{vadá} `to think', \textbf{kvu\v{s}tá} `to hear', \textbf{vydá} `to see', \textbf{ége\v{s}á} `to ask', \textbf{ohletá} `to remember', \textbf{hová} `to recount, tell a story' . The verb \textbf{vadá} is exclusively used with the subjunctive quotative.

\pex
\begingl
\gla Z \v{s}to óké necim to Luká\v{s} vadê.//
\glb already this OK \mk{cop.quot.sbj.npst} \mk{rz} Luká\v{s} think-\mk{ipf}//
\glft `Luká\v{s} thinks it should be OK by now.'//
\endgl
\xe

\pex
\begingl
\gla Marek bych jsenát to kvu\v{s}tkem.//
\glb Marek yesterday arrive-\mk{quot.pf} \mk{rz} hear-\mk{pf-1s}//
\glft `I heard Marek has arrived.'//
\endgl
\xe


\pex
\begingl
\gla Po\v{s}nelý tajomstác to kvu\v{s}tek.//
\glb father-\mk{2pl} die-\mk{quot.ret} \mk{rz} hear-\mk{pf}//
\glft `(We) heard that your father died.'//
\endgl
\xe

\pex
\begingl
\gla Dá tak bych vacim to náohletê.//
\glb \mk{1s.str} here yesterday \mk{cop.quot.sbj.pst} \mk{rz} \mk{neg}-remember-\mk{ipf}//
\glft `(I) don't remember if I was here yesterday.'//
\endgl
\xe

\par Secondary verba dicendi are formed with an adverbial construction using the imperfective converb in \textbf{-iec}.

\pex
\begingl
\gla Já mnou necim to Martin priviec vadê.//
\glb you correct \mk{cop.quot.sbj.npst} \mk{rz} Martin agree-\mk{cv} think-\mk{ipf}//
\glft `Martin agrees that you are right.'//
\endgl
\xe

\par The quotative is also used emphatically to repeat a quote (often made by the speaker himself or herself), or to express the speaker's frustration or affirmation. When used this way, the verbum dicendi is omitted, and the expletive \textbf{nó} is often added.

\pex
\begingl
\gla Mnou necim to nó!//
\glb correct \mk{cop.quot.sbj.npst} \mk{rz} \mk{expl}//
\glft `(I've been telling you) it is right.'//
\endgl
\xe

\pex
\begingl
\gla Dá roctymút to!//
\glb \mk{1s} dance-\mk{abl-quot.ipf} \mk{rz}//
\glft `(But) I can dance.'//
\endgl
\xe

\par The tense/aspect of the quotative mood follows that of the quoted clause, independent of the tense/aspect of the verbum dicendi.
